\documentclass[11pt]{amsart}

\usepackage{lipsum}

\usepackage[margin=1.2in]{geometry}
\usepackage[parfill]{parskip}
\usepackage{memo}
\usepackage{mathpazo}
\usepackage{natbib}
\usepackage{graphicx}
\usepackage{tikz}
\usetikzlibrary{external}
\tikzexternalize[prefix=figures/]
\usepackage{url}

%\usepackage{hyperref}

\begin{document}

\thispagestyle{empty}

\begin{memoheader}{Memorandum}
    \memotopline{To:}{
    }
    \memotopline{From:}{
        Cameron Bracken, Edith Zagona and Balaji Rajagopalan
        
        Center for Advanced Decision Support for Water and Environmental Systems
    }
    \memotopline{Date:}{\today}
    \memotopline{Subject:}{Stochastic Streamflow Simulations for the Otowi gage}
\end{memoheader}

%\onehalfspacing

In this memorandum we present the streamflow simulations for the Otowi Gage.  This work is similar to previous work dome by Subhrendu Gangopadhay and Ben Harding from AMEC.  We conduct simulations using a three state homogeneous Markov chain. Data is resampled from the period 1975-2007.


%%%%%%%%%%%%%%%%%%%%%%%%%%%%%%%%%%%%%%%%
%%%%%%%%%%%%%%%%%%%%%%%%%%%%%%%%%%%%%%%%
\subsection*{Homogeneous Markov Chain Simulations}
We use a homogeneous three state Markov chain to generate 1200 streamflow sequences of 50 years in length.  The number 1200 was chosen based on a recent study by \citep{Guimaraes:2011tc} that looked at the correct number of hydrologic sequences for input to planning models. 

The transition probability matrix (t.p.m) was generated from a paleo reconstructed record of the Palmer Drought Severity Index (PDSI) in the region near the Otowi Gage (\url{http://www.ncdc.noaa.gov/paleo/newpdsi.html}, grid point 103). The paleo PDSI record extends from year -1 to year 2003. Note that the previous work only used data from 1400-2003.  

We slightly modified the resampling algorithm because of the length of the period 1975 - 2007. Instead of sampling from the conditional pools of values (i.e. all the magnitudes that transition from state 2 to 3), we sample from only the pools defined by the states. For example if we simulate a transition from state 1 to 3, then we generate a magnitude by sampling randomly (without weight) from all of the values in state 3.

The basic simulation statistics and the drought statistics are shown in Figure 1 and Figure 2. The Basic statistics computed were mean, standard deviation (SD), coefficient of skewness, minimum, maximum, and the lag 1 autocorrelation value.  The drought statistics computed were the average drought run intensity, the maximum run length, the maximum drought deficit, and the average drought deficit.  All of the basic statistics are captured remarkably well. The drought statistics are also captured well besides the average drought run intensity which is just outside the IQR.  The distribution of maximum run length in particular shows that the simulations are capable of capturing longer drought spells, providing greater hydrologic variability. 

%We investigated using both a 4 state model and a nonhomogeneous model and found the results to be nearly identical (in terms of statistics) to the 3 state homogeneous model. 

\begin{figure}[htbp] %  figure placement: here, top, bottom, or page
   \centering
   \input{../plots/basic-stats-3.tikz} 
   \caption{Basic Simulation Statistics}
   \label{fig:basic-stats}
\end{figure}

\begin{figure}[htbp] %  figure placement: here, top, bottom, or page
   \centering
   \input{../plots/drought-stats-3.tikz} 
   \caption{Drought Statistics}
   \label{fig:drought-stats}
\end{figure}

\bibliographystyle{agufull04}
\bibliography{refs}

\end{document}