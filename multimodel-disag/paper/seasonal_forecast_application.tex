%% This is file `elsarticle-template-1-num.tex',
%%
%% Copyright 2009 Elsevier Ltd
%%
%% This file is part of the 'Elsarticle Bundle'.
%% ---------------------------------------------
%%
%% It may be distributed under the conditions of the LaTeX Project Public
%% License, either version 1.2 of this license or (at your option) any
%% later version.  The latest version of this license is in
%%    http://www.latex-project.org/lppl.txt
%% and version 1.2 or later is part of all distributions of LaTeX
%% version 1999/12/01 or later.
%%
%% The list of all files belonging to the 'Elsarticle Bundle' is
%% given in the file `manifest.txt'.
%%
%% Template article for Elsevier's document class `elsarticle'
%% with numbered style bibliographic references
%%
%% $Id: elsarticle-template-1-num.tex 149 2009-10-08 05:01:15Z rishi $
%% $URL: http://lenova.river-valley.com/svn/elsbst/trunk/elsarticle-template-1-num.tex $
%  

%% Use the option review to obtain double line spacing
% \documentclass[preprint,review,12pt,times,authoryear]{elsarticle}
 \documentclass[final,5p,times,twocolumn,authoryear]{elsarticle}

%% Use the options 1p,twocolumn; 3p; 3p,twocolumn; 5p; or 5p,twocolumn
%% for a journal layout:
%% \documentclass[final,1p,times]{elsarticle}
%% \documentclass[final,1p,times,twocolumn]{elsarticle}
%% \documentclass[final,3p,times]{elsarticle}
%% \documentclass[final,3p,times,twocolumn]{elsarticle}
%% \documentclass[final,5p,times]{elsarticle}
%% \documentclass[final,5p,times,twocolumn,authoryear]{elsarticle}

%% if you use PostScript figures in your article
%% use the graphics package for simple commands
%% \usepackage{graphics}
%% or use the graphicx package for more complicated commands
%% \usepackage{graphicx}
%% or use the epsfig package if you prefer to use the old commands
%% \usepackage{epsfig} 

%%------------tables
\usepackage{booktabs} %nice tables i.e. pleasing
\usepackage{rotating}
%\usepackage[nomarkers]{endfloat}
\usepackage{tikz}
\usetikzlibrary{external}
\tikzexternalize[prefix=figures/]
\usetikzlibrary{positioning,shapes,arrows,decorations.shapes}
\usepackage{url}
\usepackage{subfigure}

%% The amssymb package provides various useful mathematical symbols
\usepackage{amssymb,amsmath}
%% The amsthm package provides extended theorem environments
%% \usepackage{amsthm}

%% The lineno packages adds line numbers. Start line numbering with
%% \begin{linenumbers}, end it with \end{linenumbers}. Or switch it on
%% for the whole article with \linenumbers after \end{frontmatter}.
\usepackage{lineno}
\usepackage[colorlinks]{hyperref}
\usepackage[all]{hypcap}


%% natbib.sty is loaded by default. However, natbib options can be
%% provided with \biboptions{...} command. Following options are
%% valid:

%%   round  -  round parentheses are used (default)
%%   square -  square brackets are used   [option]
%%   curly  -  curly braces are used      {option}
%%   angle  -  angle brackets are used    <option>
%%   semicolon  -  multiple citations separated by semi-colon
%%   colon  - same as semicolon, an earlier confusion
%%   comma  -  separated by comma
%%   numbers-  selects numerical citations
%%   super  -  numerical citations as superscripts
%%   sort   -  sorts multiple citations according to order in ref. list
%%   sort&compress   -  like sort, but also compresses numerical citations
%%   compress - compresses without sorting
%%
%% \biboptions{comma,round}

% \biboptions{}


\journal{Journal of Hydrology}

\begin{document}

\begin{frontmatter}

%% Title, authors and addresses

%% use the tnoteref command within \title for footnotes;
%% use the tnotetext command for the associated footnote;
%% use the fnref command within \author or \address for footnotes;
%% use the fntext command for the associated footnote;
%% use the corref command within \author for corresponding author footnotes;
%% use the cortext command for the associated footnote;
%% use the ead command for the email address,
%% and the form \ead[url] for the home page:
%%
%% \title{Title\tnoteref{label1}}
%% \tnotetext[label1]{}
%% \author{Name\corref{cor1}\fnref{label2}}
%% \ead{email address}
%% \ead[url]{home page}
%% \fntext[label2]{}
%% \cortext[cor1]{}
%% \address{Address\fnref{label3}}
%% \fntext[label3]{}

\title{Seasonal Ensemble Forecast for Twenty sites in the Upper Colorado River Basin}

%% use optional labels to link authors explicitly to addresses:
 \author[civil]{Cameron Bracken}
 \author[civil,cires]{Balaji Rajagopalan}
 \author[civil]{Edith Zagona}
 
 \address[civil]{Department of Civil, Environmental, and Architectural Engineering, University of Colorado at Boulder}
 \address[cires]{Cooperative Institute for Research in Environmental Sciences (CIRES)}


\begin{abstract}
The generation of seasonal streamflow forecasts at several locations  is crucial for operations of the Upper Colorado River Basin water resources system. We show that the skillful framework of \cite{Bracken:2010cw} can be extended to generate monthly peak season (April-July) forecasts at twenty sites in the Upper Colorado River Basin at long lead times on the first of each month starting from preceding November. We also improve the framework by incorporating the flexible spatial and temporal disaggregation method of \cite{Nowak:2010ha} to obtain ensemble forecasts for each location and month in the peak season. With few exceptions, skills are positive for all sites, months and lead times. The November forecasts show an average of 13\% increase in skill over climatology and as high as 20\% increase in some months.  
\end{abstract}

\begin{keyword}
%% keywords here, in the form: keyword \sep keyword

%% MSC codes here, in the form: \MSC code \sep code
%% or \MSC[2008] code \sep code (2000 is the default)

\end{keyword}

\end{frontmatter}


%\linenumbers
%\pagestyle{empty}
%\raggedright


%%%%%%%%%%%%%%%%
%%%%%%%%%%%%%%%%
\section{Introduction}
In the last ten years water supply management in the Upper Colorado River Basin (UCRB) has come under increasing scrutiny and regulation. Drought \citep{Fulp:2005wy}, climate change \citep{Balaji2009}, environmental flows and increasing demands \citep{USDepartmentoftheInterior:2008vl} have all put the reliability of the system into question.  In 2007 the ``Interim Guidelines'' were passed as a response to some of these factors \citep{USDepartmentoftheInterior:2007ua}.  These guidelines lay a framework for coordinated operations of Lake Powell and Lake Mead which depend heavily on water supply forecasts.   These factors all underscore the need for skilful forecasts at long lead times at many locations throughout the UCRB.  

Current seasonal forecasts in the UCRB are issued by the Colorado Basin River Forecast Center (CBRFC). These forecasts are first issued in January when early snowpack measurements are available. Many recent studies have used climate data to produce streamflow forecasts 

The generation of forecasts at a number of sites on the river network is crucial for operations of the UCRB water resources system.  The ``24-Month Study'' is the Bureau of Reclamation's primary water supply forecast model in the Upper (and since January 2010) and Lower CRB.  This model is run monthly and requires inflow forecasts at 12 locations throughout the Upper Basin.  Operators use these inflow forecasts to generate reservoir outflows which are then run through the ``24-Month Study'' model to obtain projections of a suite of water resources system variables (e.g., reservoir levels, releases, power generation etc.) for the next 24 to 36 months. These then guide planning and operational decisions over the 24 month period. Beyond initial reservoir states, the predictive capability of the model depends largely on the skill of the inflow forecasts -- thus, skillful long lead forecasts of the peak season (Apr-Jul) are very important.

The nonparametric multisite ensemble forecast framework developed by Bracken et al. (2010) showed skill at long lead times in forecasts of peak season streamflows. They demonstrated their approach on four key sites in the Upper Colorado River Basin (Colorado River near Cisco, Utah, Green River at Green River, Utah, San Juan River near Bluff, Utah, and Colorado River at Lees Ferry, Arizona). In this forecasts of an ``index gauge'' that is composed of the flow at all the sites in the network is first generated based on a number of large scale climate predictors using local polynomial functional estimation approach Large scale climate predictors are identified for the index gauge flow at several lead times and local polynomials is used to obtain the best model. The generated seasonal forecast of the index gauge is then disaggregated using the approach of Prairie et al. (2007) to obtain forecasts at the four locations and for each month of the peak season.  As mentioned, significant skills were obtained at the four locations.

While promising, these results leave open the question of whether the skills translate to all the locations on the UCRB that are needed for planning and management of the water resources system. In this research we modified the approach of \cite{Bracken:2010cw} and apply this to all the 20 locations on the UCRB. Study area and data sets used are first described, followed by a brief description of the approach for the benefit of the readers. The validation and results are then presented with a summary and discussion.


%%%%%%%%%%%%%%%%%%%%%%%%%%%%%%%%%%%%%%%%%%%%%%%%%%%%%%%%%%%%
%%%%%%%%%%%%%%%%%%%%%%%%%%%%%%%%%%%%%%%%%%%%%%%%%%%%%%%%%%%%
\section{Study Area and Data}
The UCRB has a drainage area of 279,300 square kilometers.  The terrain varies over 4000 m from east to west. Lees Ferry is located at the outlet of the Upper Basin.  The sites used in this study (Figure \ref{fig:map}a) are all the natural flow nodes in the upper basin.  The sites used by the USBR in the 24-Month study model are a subset of these sites.


\begin{figure}[!htbp] %  figure placement: here, top, bottom, or page
   \centering
   \input{figures/uc_map.tex}
   % Created by Eps2pgf 0.7.0 (build on 2008-08-24) on Thu May 19 14:44:46 MDT 2011
\begin{tikzpicture}
\pgfpathmoveto{\pgfqpoint{0cm}{-0.071cm}}
\pgfpathlineto{\pgfqpoint{5.009cm}{-0.071cm}}
\pgfpathlineto{\pgfqpoint{5.009cm}{8.361cm}}
\pgfpathlineto{\pgfqpoint{0cm}{8.361cm}}
\pgfpathclose
\pgfusepath{clip}
\definecolor{eps2pgf_color}{gray}{0}\pgfsetstrokecolor{eps2pgf_color}\pgfsetfillcolor{eps2pgf_color}
\pgftext[x=2.952cm,y=0.247cm,rotate=0]{20}
\pgftext[x=3.879cm,y=2.505cm,rotate=0]{18}
\pgftext[x=1.128cm,y=2.506cm,rotate=0]{17}
\pgftext[x=2.12cm,y=2.506cm,rotate=0]{16}
\pgftext[x=0.216cm,y=4.765cm,rotate=0]{14}
\pgftext[x=1.197cm,y=4.765cm,rotate=0]{15}
\pgftext[x=2.113cm,y=7.02cm,rotate=0]{13}
\pgftext[x=3.109cm,y=7.023cm,rotate=0]{12}
\pgftext[x=1.043cm,y=5.894cm,rotate=0]{1}
\pgftext[x=1.216cm,y=5.894cm,rotate=0]{1}
\pgftext[x=1.134cm,y=7.02cm,rotate=0]{10}
\pgftext[x=1.112cm,y=8.148cm,rotate=0]{9}
\pgftext[x=2.984cm,y=2.505cm,rotate=0]{8}
\pgftext[x=3.119cm,y=4.761cm,rotate=0]{7}
\pgftext[x=3.902cm,y=4.764cm,rotate=0]{6}
\pgftext[x=3.931cm,y=5.894cm,rotate=0]{5}
\pgftext[x=3.933cm,y=7.023cm,rotate=0]{4}
\pgftext[x=3.94cm,y=8.149cm,rotate=0]{3}
\pgftext[x=4.784cm,y=5.894cm,rotate=0]{2}
\pgftext[x=4.784cm,y=7.023cm,rotate=0]{1}
\pgfsetdash{}{0cm}
\pgfsetlinewidth{0.353mm}
\pgfsetroundcap
\pgfsetroundjoin
\definecolor{eps2pgf_color}{cmyk}{0.75021,0.679683,0.670222,0.90164}\pgfsetstrokecolor{eps2pgf_color}\pgfsetfillcolor{eps2pgf_color}
\pgfpathmoveto{\pgfqpoint{2.401cm}{1.571cm}}
\pgfpathcurveto{\pgfqpoint{2.47cm}{1.502cm}}{\pgfqpoint{2.47cm}{1.39cm}}{\pgfqpoint{2.401cm}{1.322cm}}
\pgfpathcurveto{\pgfqpoint{2.332cm}{1.253cm}}{\pgfqpoint{2.22cm}{1.253cm}}{\pgfqpoint{2.151cm}{1.322cm}}
\pgfpathcurveto{\pgfqpoint{2.082cm}{1.39cm}}{\pgfqpoint{2.082cm}{1.502cm}}{\pgfqpoint{2.151cm}{1.571cm}}
\pgfpathcurveto{\pgfqpoint{2.22cm}{1.64cm}}{\pgfqpoint{2.332cm}{1.64cm}}{\pgfqpoint{2.401cm}{1.571cm}}
\pgfusepath{stroke}
\pgfsetdash{}{0cm}
\pgfpathmoveto{\pgfqpoint{2.695cm}{0.718cm}}
\pgfpathlineto{\pgfqpoint{2.373cm}{1.278cm}}
\pgfusepath{stroke}
\pgfpathmoveto{\pgfqpoint{2.793cm}{0.547cm}}
\pgfpathlineto{\pgfqpoint{2.759cm}{0.755cm}}
\pgfpathlineto{\pgfqpoint{2.631cm}{0.682cm}}
\pgfpathlineto{\pgfqpoint{2.793cm}{0.547cm}}
\pgfpathclose
\pgfusepath{fill}
\pgfsetdash{}{0cm}
\pgfsetbuttcap
\pgfsetmiterjoin
\pgfpathmoveto{\pgfqpoint{2.793cm}{0.547cm}}
\pgfpathlineto{\pgfqpoint{2.759cm}{0.755cm}}
\pgfpathlineto{\pgfqpoint{2.631cm}{0.682cm}}
\pgfpathlineto{\pgfqpoint{2.793cm}{0.547cm}}
\pgfpathclose
\pgfusepath{stroke}
\definecolor{eps2pgf_color}{gray}{0}\pgfsetstrokecolor{eps2pgf_color}\pgfsetfillcolor{eps2pgf_color}
\pgftext[x=3.67cm,y=1.374cm,rotate=0]{19}
\pgfsetdash{}{0cm}
\pgfsetroundcap
\pgfsetroundjoin
\definecolor{eps2pgf_color}{cmyk}{0.75021,0.679683,0.670222,0.90164}\pgfsetstrokecolor{eps2pgf_color}\pgfsetfillcolor{eps2pgf_color}
\pgfpathmoveto{\pgfqpoint{3.261cm}{0.714cm}}
\pgfpathlineto{\pgfqpoint{3.515cm}{1.124cm}}
\pgfusepath{stroke}
\pgfpathmoveto{\pgfqpoint{3.157cm}{0.546cm}}
\pgfpathlineto{\pgfqpoint{3.324cm}{0.675cm}}
\pgfpathlineto{\pgfqpoint{3.198cm}{0.753cm}}
\pgfpathlineto{\pgfqpoint{3.157cm}{0.546cm}}
\pgfpathclose
\pgfusepath{fill}
\pgfsetdash{}{0cm}
\pgfsetbuttcap
\pgfsetmiterjoin
\pgfpathmoveto{\pgfqpoint{3.157cm}{0.546cm}}
\pgfpathlineto{\pgfqpoint{3.324cm}{0.675cm}}
\pgfpathlineto{\pgfqpoint{3.198cm}{0.753cm}}
\pgfpathlineto{\pgfqpoint{3.157cm}{0.546cm}}
\pgfpathclose
\pgfusepath{stroke}
\pgfsetdash{}{0cm}
\pgfsetroundcap
\pgfsetroundjoin
\pgfpathmoveto{\pgfqpoint{3.768cm}{1.878cm}}
\pgfpathlineto{\pgfqpoint{3.837cm}{2.253cm}}
\pgfusepath{stroke}
\pgfpathmoveto{\pgfqpoint{3.732cm}{1.684cm}}
\pgfpathlineto{\pgfqpoint{3.841cm}{1.865cm}}
\pgfpathlineto{\pgfqpoint{3.695cm}{1.892cm}}
\pgfpathlineto{\pgfqpoint{3.732cm}{1.684cm}}
\pgfpathclose
\pgfusepath{fill}
\pgfsetdash{}{0cm}
\pgfsetbuttcap
\pgfsetmiterjoin
\pgfpathmoveto{\pgfqpoint{3.732cm}{1.684cm}}
\pgfpathlineto{\pgfqpoint{3.841cm}{1.865cm}}
\pgfpathlineto{\pgfqpoint{3.695cm}{1.892cm}}
\pgfpathlineto{\pgfqpoint{3.732cm}{1.684cm}}
\pgfpathclose
\pgfusepath{stroke}
\pgfsetdash{}{0cm}
\pgfsetroundcap
\pgfsetroundjoin
\pgfpathmoveto{\pgfqpoint{1.939cm}{1.758cm}}
\pgfpathlineto{\pgfqpoint{1.328cm}{2.322cm}}
\pgfusepath{stroke}
\pgfpathmoveto{\pgfqpoint{2.084cm}{1.624cm}}
\pgfpathlineto{\pgfqpoint{1.989cm}{1.812cm}}
\pgfpathlineto{\pgfqpoint{1.889cm}{1.703cm}}
\pgfpathlineto{\pgfqpoint{2.084cm}{1.624cm}}
\pgfpathclose
\pgfusepath{fill}
\pgfsetdash{}{0cm}
\pgfsetbuttcap
\pgfsetmiterjoin
\pgfpathmoveto{\pgfqpoint{2.084cm}{1.624cm}}
\pgfpathlineto{\pgfqpoint{1.989cm}{1.812cm}}
\pgfpathlineto{\pgfqpoint{1.889cm}{1.703cm}}
\pgfpathlineto{\pgfqpoint{2.084cm}{1.624cm}}
\pgfpathclose
\pgfusepath{stroke}
\pgfsetdash{}{0cm}
\pgfsetroundcap
\pgfsetroundjoin
\pgfpathmoveto{\pgfqpoint{2.209cm}{1.9cm}}
\pgfpathlineto{\pgfqpoint{2.158cm}{2.253cm}}
\pgfusepath{stroke}
\pgfpathmoveto{\pgfqpoint{2.238cm}{1.705cm}}
\pgfpathlineto{\pgfqpoint{2.283cm}{1.911cm}}
\pgfpathlineto{\pgfqpoint{2.136cm}{1.89cm}}
\pgfpathlineto{\pgfqpoint{2.238cm}{1.705cm}}
\pgfpathclose
\pgfusepath{fill}
\pgfsetdash{}{0cm}
\pgfsetbuttcap
\pgfsetmiterjoin
\pgfpathmoveto{\pgfqpoint{2.238cm}{1.705cm}}
\pgfpathlineto{\pgfqpoint{2.283cm}{1.911cm}}
\pgfpathlineto{\pgfqpoint{2.136cm}{1.89cm}}
\pgfpathlineto{\pgfqpoint{2.238cm}{1.705cm}}
\pgfpathclose
\pgfusepath{stroke}
\pgfsetdash{}{0cm}
\pgfsetroundcap
\pgfsetroundjoin
\pgfpathmoveto{\pgfqpoint{1.554cm}{3.829cm}}
\pgfpathcurveto{\pgfqpoint{1.623cm}{3.76cm}}{\pgfqpoint{1.623cm}{3.648cm}}{\pgfqpoint{1.554cm}{3.579cm}}
\pgfpathcurveto{\pgfqpoint{1.485cm}{3.51cm}}{\pgfqpoint{1.373cm}{3.51cm}}{\pgfqpoint{1.304cm}{3.579cm}}
\pgfpathcurveto{\pgfqpoint{1.236cm}{3.648cm}}{\pgfqpoint{1.236cm}{3.76cm}}{\pgfqpoint{1.304cm}{3.829cm}}
\pgfpathcurveto{\pgfqpoint{1.373cm}{3.898cm}}{\pgfqpoint{1.485cm}{3.898cm}}{\pgfqpoint{1.554cm}{3.829cm}}
\pgfusepath{stroke}
\pgfsetdash{}{0cm}
\pgfpathmoveto{\pgfqpoint{1.848cm}{2.976cm}}
\pgfpathlineto{\pgfqpoint{1.526cm}{3.536cm}}
\pgfusepath{stroke}
\pgfpathmoveto{\pgfqpoint{1.947cm}{2.805cm}}
\pgfpathlineto{\pgfqpoint{1.912cm}{3.013cm}}
\pgfpathlineto{\pgfqpoint{1.784cm}{2.939cm}}
\pgfpathlineto{\pgfqpoint{1.947cm}{2.805cm}}
\pgfpathclose
\pgfusepath{fill}
\pgfsetdash{}{0cm}
\pgfsetbuttcap
\pgfsetmiterjoin
\pgfpathmoveto{\pgfqpoint{1.947cm}{2.805cm}}
\pgfpathlineto{\pgfqpoint{1.912cm}{3.013cm}}
\pgfpathlineto{\pgfqpoint{1.784cm}{2.939cm}}
\pgfpathlineto{\pgfqpoint{1.947cm}{2.805cm}}
\pgfpathclose
\pgfusepath{stroke}
\pgfsetdash{}{0cm}
\pgfsetroundcap
\pgfsetroundjoin
\pgfpathmoveto{\pgfqpoint{1.083cm}{4.005cm}}
\pgfpathlineto{\pgfqpoint{0.411cm}{4.59cm}}
\pgfusepath{stroke}
\pgfpathmoveto{\pgfqpoint{1.232cm}{3.876cm}}
\pgfpathlineto{\pgfqpoint{1.132cm}{4.061cm}}
\pgfpathlineto{\pgfqpoint{1.035cm}{3.949cm}}
\pgfpathlineto{\pgfqpoint{1.232cm}{3.876cm}}
\pgfpathclose
\pgfusepath{fill}
\pgfsetdash{}{0cm}
\pgfsetbuttcap
\pgfsetmiterjoin
\pgfpathmoveto{\pgfqpoint{1.232cm}{3.876cm}}
\pgfpathlineto{\pgfqpoint{1.132cm}{4.061cm}}
\pgfpathlineto{\pgfqpoint{1.035cm}{3.949cm}}
\pgfpathlineto{\pgfqpoint{1.232cm}{3.876cm}}
\pgfpathclose
\pgfusepath{stroke}
\pgfsetdash{}{0cm}
\pgfsetroundcap
\pgfsetroundjoin
\pgfpathmoveto{\pgfqpoint{2.295cm}{4.958cm}}
\pgfpathcurveto{\pgfqpoint{2.364cm}{4.889cm}}{\pgfqpoint{2.364cm}{4.777cm}}{\pgfqpoint{2.295cm}{4.708cm}}
\pgfpathcurveto{\pgfqpoint{2.226cm}{4.639cm}}{\pgfqpoint{2.114cm}{4.639cm}}{\pgfqpoint{2.045cm}{4.708cm}}
\pgfpathcurveto{\pgfqpoint{1.976cm}{4.777cm}}{\pgfqpoint{1.976cm}{4.889cm}}{\pgfqpoint{2.045cm}{4.958cm}}
\pgfpathcurveto{\pgfqpoint{2.114cm}{5.027cm}}{\pgfqpoint{2.226cm}{5.027cm}}{\pgfqpoint{2.295cm}{4.958cm}}
\pgfusepath{stroke}
\pgfsetdash{}{0cm}
\pgfpathmoveto{\pgfqpoint{1.683cm}{4.087cm}}
\pgfpathlineto{\pgfqpoint{2.068cm}{4.668cm}}
\pgfusepath{stroke}
\pgfpathmoveto{\pgfqpoint{1.574cm}{3.922cm}}
\pgfpathlineto{\pgfqpoint{1.745cm}{4.046cm}}
\pgfpathlineto{\pgfqpoint{1.621cm}{4.127cm}}
\pgfpathlineto{\pgfqpoint{1.574cm}{3.922cm}}
\pgfpathclose
\pgfusepath{fill}
\pgfsetdash{}{0cm}
\pgfsetbuttcap
\pgfsetmiterjoin
\pgfpathmoveto{\pgfqpoint{1.574cm}{3.922cm}}
\pgfpathlineto{\pgfqpoint{1.745cm}{4.046cm}}
\pgfpathlineto{\pgfqpoint{1.621cm}{4.127cm}}
\pgfpathlineto{\pgfqpoint{1.574cm}{3.922cm}}
\pgfpathclose
\pgfusepath{stroke}
\pgfsetdash{}{0cm}
\pgfsetroundcap
\pgfsetroundjoin
\pgfpathmoveto{\pgfqpoint{1.851cm}{5.162cm}}
\pgfpathlineto{\pgfqpoint{1.341cm}{5.688cm}}
\pgfusepath{stroke}
\pgfpathmoveto{\pgfqpoint{1.988cm}{5.021cm}}
\pgfpathlineto{\pgfqpoint{1.904cm}{5.214cm}}
\pgfpathlineto{\pgfqpoint{1.797cm}{5.111cm}}
\pgfpathlineto{\pgfqpoint{1.988cm}{5.021cm}}
\pgfpathclose
\pgfusepath{fill}
\pgfsetdash{}{0cm}
\pgfsetbuttcap
\pgfsetmiterjoin
\pgfpathmoveto{\pgfqpoint{1.988cm}{5.021cm}}
\pgfpathlineto{\pgfqpoint{1.904cm}{5.214cm}}
\pgfpathlineto{\pgfqpoint{1.797cm}{5.111cm}}
\pgfpathlineto{\pgfqpoint{1.988cm}{5.021cm}}
\pgfpathclose
\pgfusepath{stroke}
\pgfsetdash{}{0cm}
\pgfsetroundcap
\pgfsetroundjoin
\pgfpathmoveto{\pgfqpoint{1.14cm}{6.398cm}}
\pgfpathlineto{\pgfqpoint{1.135cm}{6.769cm}}
\pgfusepath{stroke}
\pgfpathmoveto{\pgfqpoint{1.143cm}{6.201cm}}
\pgfpathlineto{\pgfqpoint{1.214cm}{6.399cm}}
\pgfpathlineto{\pgfqpoint{1.066cm}{6.397cm}}
\pgfpathlineto{\pgfqpoint{1.143cm}{6.201cm}}
\pgfpathclose
\pgfusepath{fill}
\pgfsetdash{}{0cm}
\pgfsetbuttcap
\pgfsetmiterjoin
\pgfpathmoveto{\pgfqpoint{1.143cm}{6.201cm}}
\pgfpathlineto{\pgfqpoint{1.214cm}{6.399cm}}
\pgfpathlineto{\pgfqpoint{1.066cm}{6.397cm}}
\pgfpathlineto{\pgfqpoint{1.143cm}{6.201cm}}
\pgfpathclose
\pgfusepath{stroke}
\pgfsetdash{}{0cm}
\pgfsetroundcap
\pgfsetroundjoin
\pgfpathmoveto{\pgfqpoint{2.295cm}{6.087cm}}
\pgfpathcurveto{\pgfqpoint{2.364cm}{6.018cm}}{\pgfqpoint{2.364cm}{5.906cm}}{\pgfqpoint{2.295cm}{5.837cm}}
\pgfpathcurveto{\pgfqpoint{2.226cm}{5.768cm}}{\pgfqpoint{2.114cm}{5.768cm}}{\pgfqpoint{2.045cm}{5.837cm}}
\pgfpathcurveto{\pgfqpoint{1.976cm}{5.906cm}}{\pgfqpoint{1.976cm}{6.018cm}}{\pgfqpoint{2.045cm}{6.087cm}}
\pgfpathcurveto{\pgfqpoint{2.114cm}{6.156cm}}{\pgfqpoint{2.226cm}{6.156cm}}{\pgfqpoint{2.295cm}{6.087cm}}
\pgfusepath{stroke}
\pgfsetdash{}{0cm}
\pgfpathmoveto{\pgfqpoint{2.172cm}{5.292cm}}
\pgfpathlineto{\pgfqpoint{2.174cm}{5.768cm}}
\pgfusepath{stroke}
\pgfpathmoveto{\pgfqpoint{2.171cm}{5.094cm}}
\pgfpathlineto{\pgfqpoint{2.246cm}{5.292cm}}
\pgfpathlineto{\pgfqpoint{2.098cm}{5.292cm}}
\pgfpathlineto{\pgfqpoint{2.171cm}{5.094cm}}
\pgfpathclose
\pgfusepath{fill}
\pgfsetdash{}{0cm}
\pgfsetbuttcap
\pgfsetmiterjoin
\pgfpathmoveto{\pgfqpoint{2.171cm}{5.094cm}}
\pgfpathlineto{\pgfqpoint{2.246cm}{5.292cm}}
\pgfpathlineto{\pgfqpoint{2.098cm}{5.292cm}}
\pgfpathlineto{\pgfqpoint{2.171cm}{5.094cm}}
\pgfpathclose
\pgfusepath{stroke}
\pgfsetdash{}{0cm}
\pgfsetroundcap
\pgfsetroundjoin
\pgfpathmoveto{\pgfqpoint{2.149cm}{6.42cm}}
\pgfpathlineto{\pgfqpoint{2.132cm}{6.769cm}}
\pgfusepath{stroke}
\pgfpathmoveto{\pgfqpoint{2.158cm}{6.223cm}}
\pgfpathlineto{\pgfqpoint{2.223cm}{6.424cm}}
\pgfpathlineto{\pgfqpoint{2.075cm}{6.417cm}}
\pgfpathlineto{\pgfqpoint{2.158cm}{6.223cm}}
\pgfpathclose
\pgfusepath{fill}
\pgfsetdash{}{0cm}
\pgfsetbuttcap
\pgfsetmiterjoin
\pgfpathmoveto{\pgfqpoint{2.158cm}{6.223cm}}
\pgfpathlineto{\pgfqpoint{2.223cm}{6.424cm}}
\pgfpathlineto{\pgfqpoint{2.075cm}{6.417cm}}
\pgfpathlineto{\pgfqpoint{2.158cm}{6.223cm}}
\pgfpathclose
\pgfusepath{stroke}
\pgfsetdash{}{0cm}
\pgfsetroundcap
\pgfsetroundjoin
\pgfpathmoveto{\pgfqpoint{2.475cm}{6.305cm}}
\pgfpathlineto{\pgfqpoint{2.916cm}{6.801cm}}
\pgfusepath{stroke}
\pgfpathmoveto{\pgfqpoint{2.344cm}{6.157cm}}
\pgfpathlineto{\pgfqpoint{2.53cm}{6.256cm}}
\pgfpathlineto{\pgfqpoint{2.42cm}{6.354cm}}
\pgfpathlineto{\pgfqpoint{2.344cm}{6.157cm}}
\pgfpathclose
\pgfusepath{fill}
\pgfsetdash{}{0cm}
\pgfsetbuttcap
\pgfsetmiterjoin
\pgfpathmoveto{\pgfqpoint{2.344cm}{6.157cm}}
\pgfpathlineto{\pgfqpoint{2.53cm}{6.256cm}}
\pgfpathlineto{\pgfqpoint{2.42cm}{6.354cm}}
\pgfpathlineto{\pgfqpoint{2.344cm}{6.157cm}}
\pgfpathclose
\pgfusepath{stroke}
\pgfsetdash{}{0cm}
\pgfsetroundcap
\pgfsetroundjoin
\pgfpathmoveto{\pgfqpoint{1.334cm}{4.153cm}}
\pgfpathlineto{\pgfqpoint{1.257cm}{4.511cm}}
\pgfusepath{stroke}
\pgfpathmoveto{\pgfqpoint{1.375cm}{3.96cm}}
\pgfpathlineto{\pgfqpoint{1.406cm}{4.168cm}}
\pgfpathlineto{\pgfqpoint{1.261cm}{4.137cm}}
\pgfpathlineto{\pgfqpoint{1.375cm}{3.96cm}}
\pgfpathclose
\pgfusepath{fill}
\pgfsetdash{}{0cm}
\pgfsetbuttcap
\pgfsetmiterjoin
\pgfpathmoveto{\pgfqpoint{1.375cm}{3.96cm}}
\pgfpathlineto{\pgfqpoint{1.406cm}{4.168cm}}
\pgfpathlineto{\pgfqpoint{1.261cm}{4.137cm}}
\pgfpathlineto{\pgfqpoint{1.375cm}{3.96cm}}
\pgfpathclose
\pgfusepath{stroke}
\pgfsetdash{}{0cm}
\pgfsetroundcap
\pgfsetroundjoin
\pgfpathmoveto{\pgfqpoint{2.536cm}{1.825cm}}
\pgfpathlineto{\pgfqpoint{2.896cm}{2.349cm}}
\pgfusepath{stroke}
\pgfpathmoveto{\pgfqpoint{2.424cm}{1.662cm}}
\pgfpathlineto{\pgfqpoint{2.597cm}{1.783cm}}
\pgfpathlineto{\pgfqpoint{2.475cm}{1.867cm}}
\pgfpathlineto{\pgfqpoint{2.424cm}{1.662cm}}
\pgfpathclose
\pgfusepath{fill}
\pgfsetdash{}{0cm}
\pgfsetbuttcap
\pgfsetmiterjoin
\pgfpathmoveto{\pgfqpoint{2.424cm}{1.662cm}}
\pgfpathlineto{\pgfqpoint{2.597cm}{1.783cm}}
\pgfpathlineto{\pgfqpoint{2.475cm}{1.867cm}}
\pgfpathlineto{\pgfqpoint{2.424cm}{1.662cm}}
\pgfpathclose
\pgfusepath{stroke}
\pgfsetdash{}{0cm}
\pgfsetroundcap
\pgfsetroundjoin
\pgfpathmoveto{\pgfqpoint{4.906cm}{4.958cm}}
\pgfpathcurveto{\pgfqpoint{4.975cm}{4.889cm}}{\pgfqpoint{4.975cm}{4.777cm}}{\pgfqpoint{4.906cm}{4.708cm}}
\pgfpathcurveto{\pgfqpoint{4.837cm}{4.639cm}}{\pgfqpoint{4.725cm}{4.639cm}}{\pgfqpoint{4.656cm}{4.708cm}}
\pgfpathcurveto{\pgfqpoint{4.587cm}{4.777cm}}{\pgfqpoint{4.587cm}{4.889cm}}{\pgfqpoint{4.656cm}{4.958cm}}
\pgfpathcurveto{\pgfqpoint{4.725cm}{5.027cm}}{\pgfqpoint{4.837cm}{5.027cm}}{\pgfqpoint{4.906cm}{4.958cm}}
\pgfusepath{stroke}
\pgfsetdash{}{0cm}
\pgfpathmoveto{\pgfqpoint{3.636cm}{3.829cm}}
\pgfpathcurveto{\pgfqpoint{3.705cm}{3.76cm}}{\pgfqpoint{3.705cm}{3.648cm}}{\pgfqpoint{3.636cm}{3.579cm}}
\pgfpathcurveto{\pgfqpoint{3.567cm}{3.51cm}}{\pgfqpoint{3.455cm}{3.51cm}}{\pgfqpoint{3.386cm}{3.579cm}}
\pgfpathcurveto{\pgfqpoint{3.317cm}{3.648cm}}{\pgfqpoint{3.317cm}{3.76cm}}{\pgfqpoint{3.386cm}{3.829cm}}
\pgfpathcurveto{\pgfqpoint{3.455cm}{3.898cm}}{\pgfqpoint{3.567cm}{3.898cm}}{\pgfqpoint{3.636cm}{3.829cm}}
\pgfusepath{stroke}
\pgfsetdash{}{0cm}
\pgfpathmoveto{\pgfqpoint{3.209cm}{2.991cm}}
\pgfpathlineto{\pgfqpoint{3.435cm}{3.525cm}}
\pgfusepath{stroke}
\pgfpathmoveto{\pgfqpoint{3.132cm}{2.809cm}}
\pgfpathlineto{\pgfqpoint{3.277cm}{2.962cm}}
\pgfpathlineto{\pgfqpoint{3.141cm}{3.019cm}}
\pgfpathlineto{\pgfqpoint{3.132cm}{2.809cm}}
\pgfpathclose
\pgfusepath{fill}
\pgfsetdash{}{0cm}
\pgfsetbuttcap
\pgfsetmiterjoin
\pgfpathmoveto{\pgfqpoint{3.132cm}{2.809cm}}
\pgfpathlineto{\pgfqpoint{3.277cm}{2.962cm}}
\pgfpathlineto{\pgfqpoint{3.141cm}{3.019cm}}
\pgfpathlineto{\pgfqpoint{3.132cm}{2.809cm}}
\pgfpathclose
\pgfusepath{stroke}
\pgfsetdash{}{0cm}
\pgfsetroundcap
\pgfsetroundjoin
\pgfpathmoveto{\pgfqpoint{3.854cm}{4.009cm}}
\pgfpathlineto{\pgfqpoint{4.636cm}{4.704cm}}
\pgfusepath{stroke}
\pgfpathmoveto{\pgfqpoint{3.706cm}{3.878cm}}
\pgfpathlineto{\pgfqpoint{3.903cm}{3.953cm}}
\pgfpathlineto{\pgfqpoint{3.805cm}{4.064cm}}
\pgfpathlineto{\pgfqpoint{3.706cm}{3.878cm}}
\pgfpathclose
\pgfusepath{fill}
\pgfsetdash{}{0cm}
\pgfsetbuttcap
\pgfsetmiterjoin
\pgfpathmoveto{\pgfqpoint{3.706cm}{3.878cm}}
\pgfpathlineto{\pgfqpoint{3.903cm}{3.953cm}}
\pgfpathlineto{\pgfqpoint{3.805cm}{4.064cm}}
\pgfpathlineto{\pgfqpoint{3.706cm}{3.878cm}}
\pgfpathclose
\pgfusepath{stroke}
\pgfsetdash{}{0cm}
\pgfsetroundcap
\pgfsetroundjoin
\pgfpathmoveto{\pgfqpoint{3.675cm}{4.132cm}}
\pgfpathlineto{\pgfqpoint{3.821cm}{4.511cm}}
\pgfusepath{stroke}
\pgfpathmoveto{\pgfqpoint{3.604cm}{3.948cm}}
\pgfpathlineto{\pgfqpoint{3.745cm}{4.106cm}}
\pgfpathlineto{\pgfqpoint{3.606cm}{4.159cm}}
\pgfpathlineto{\pgfqpoint{3.604cm}{3.948cm}}
\pgfpathclose
\pgfusepath{fill}
\pgfsetdash{}{0cm}
\pgfsetbuttcap
\pgfsetmiterjoin
\pgfpathmoveto{\pgfqpoint{3.604cm}{3.948cm}}
\pgfpathlineto{\pgfqpoint{3.745cm}{4.106cm}}
\pgfpathlineto{\pgfqpoint{3.606cm}{4.159cm}}
\pgfpathlineto{\pgfqpoint{3.604cm}{3.948cm}}
\pgfpathclose
\pgfusepath{stroke}
\pgfsetdash{}{0cm}
\pgfsetroundcap
\pgfsetroundjoin
\pgfpathmoveto{\pgfqpoint{4.789cm}{5.292cm}}
\pgfpathlineto{\pgfqpoint{4.795cm}{5.64cm}}
\pgfusepath{stroke}
\pgfpathmoveto{\pgfqpoint{4.785cm}{5.094cm}}
\pgfpathlineto{\pgfqpoint{4.863cm}{5.29cm}}
\pgfpathlineto{\pgfqpoint{4.715cm}{5.293cm}}
\pgfpathlineto{\pgfqpoint{4.785cm}{5.094cm}}
\pgfpathclose
\pgfusepath{fill}
\pgfsetdash{}{0cm}
\pgfsetbuttcap
\pgfsetmiterjoin
\pgfpathmoveto{\pgfqpoint{4.785cm}{5.094cm}}
\pgfpathlineto{\pgfqpoint{4.863cm}{5.29cm}}
\pgfpathlineto{\pgfqpoint{4.715cm}{5.293cm}}
\pgfpathlineto{\pgfqpoint{4.785cm}{5.094cm}}
\pgfpathclose
\pgfusepath{stroke}
\pgfsetdash{}{0cm}
\pgfsetroundcap
\pgfsetroundjoin
\pgfpathmoveto{\pgfqpoint{4.8cm}{6.398cm}}
\pgfpathlineto{\pgfqpoint{4.799cm}{6.769cm}}
\pgfusepath{stroke}
\pgfpathmoveto{\pgfqpoint{4.8cm}{6.201cm}}
\pgfpathlineto{\pgfqpoint{4.874cm}{6.398cm}}
\pgfpathlineto{\pgfqpoint{4.726cm}{6.398cm}}
\pgfpathlineto{\pgfqpoint{4.8cm}{6.201cm}}
\pgfpathclose
\pgfusepath{fill}
\pgfsetdash{}{0cm}
\pgfsetbuttcap
\pgfsetmiterjoin
\pgfpathmoveto{\pgfqpoint{4.8cm}{6.201cm}}
\pgfpathlineto{\pgfqpoint{4.874cm}{6.398cm}}
\pgfpathlineto{\pgfqpoint{4.726cm}{6.398cm}}
\pgfpathlineto{\pgfqpoint{4.8cm}{6.201cm}}
\pgfpathclose
\pgfusepath{stroke}
\pgfsetdash{}{0cm}
\pgfsetroundcap
\pgfsetroundjoin
\pgfpathmoveto{\pgfqpoint{3.934cm}{5.269cm}}
\pgfpathlineto{\pgfqpoint{3.945cm}{5.64cm}}
\pgfusepath{stroke}
\pgfpathmoveto{\pgfqpoint{3.928cm}{5.072cm}}
\pgfpathlineto{\pgfqpoint{4.008cm}{5.267cm}}
\pgfpathlineto{\pgfqpoint{3.86cm}{5.271cm}}
\pgfpathlineto{\pgfqpoint{3.928cm}{5.072cm}}
\pgfpathclose
\pgfusepath{fill}
\pgfsetdash{}{0cm}
\pgfsetbuttcap
\pgfsetmiterjoin
\pgfpathmoveto{\pgfqpoint{3.928cm}{5.072cm}}
\pgfpathlineto{\pgfqpoint{4.008cm}{5.267cm}}
\pgfpathlineto{\pgfqpoint{3.86cm}{5.271cm}}
\pgfpathlineto{\pgfqpoint{3.928cm}{5.072cm}}
\pgfpathclose
\pgfusepath{stroke}
\pgfsetdash{}{0cm}
\pgfsetroundcap
\pgfsetroundjoin
\pgfpathmoveto{\pgfqpoint{3.955cm}{6.398cm}}
\pgfpathlineto{\pgfqpoint{3.956cm}{6.769cm}}
\pgfusepath{stroke}
\pgfpathmoveto{\pgfqpoint{3.955cm}{6.201cm}}
\pgfpathlineto{\pgfqpoint{4.029cm}{6.398cm}}
\pgfpathlineto{\pgfqpoint{3.881cm}{6.398cm}}
\pgfpathlineto{\pgfqpoint{3.955cm}{6.201cm}}
\pgfpathclose
\pgfusepath{fill}
\pgfsetdash{}{0cm}
\pgfsetbuttcap
\pgfsetmiterjoin
\pgfpathmoveto{\pgfqpoint{3.955cm}{6.201cm}}
\pgfpathlineto{\pgfqpoint{4.029cm}{6.398cm}}
\pgfpathlineto{\pgfqpoint{3.881cm}{6.398cm}}
\pgfpathlineto{\pgfqpoint{3.955cm}{6.201cm}}
\pgfpathclose
\pgfusepath{stroke}
\pgfsetdash{}{0cm}
\pgfsetroundcap
\pgfsetroundjoin
\pgfpathmoveto{\pgfqpoint{1.132cm}{7.527cm}}
\pgfpathlineto{\pgfqpoint{1.131cm}{7.897cm}}
\pgfusepath{stroke}
\pgfpathmoveto{\pgfqpoint{1.133cm}{7.329cm}}
\pgfpathlineto{\pgfqpoint{1.206cm}{7.527cm}}
\pgfpathlineto{\pgfqpoint{1.058cm}{7.527cm}}
\pgfpathlineto{\pgfqpoint{1.133cm}{7.329cm}}
\pgfpathclose
\pgfusepath{fill}
\pgfsetdash{}{0cm}
\pgfsetbuttcap
\pgfsetmiterjoin
\pgfpathmoveto{\pgfqpoint{1.133cm}{7.329cm}}
\pgfpathlineto{\pgfqpoint{1.206cm}{7.527cm}}
\pgfpathlineto{\pgfqpoint{1.058cm}{7.527cm}}
\pgfpathlineto{\pgfqpoint{1.133cm}{7.329cm}}
\pgfpathclose
\pgfusepath{stroke}
\pgfsetdash{}{0cm}
\pgfsetroundcap
\pgfsetroundjoin
\pgfpathmoveto{\pgfqpoint{3.957cm}{7.527cm}}
\pgfpathlineto{\pgfqpoint{3.959cm}{7.897cm}}
\pgfusepath{stroke}
\pgfpathmoveto{\pgfqpoint{3.956cm}{7.329cm}}
\pgfpathlineto{\pgfqpoint{4.031cm}{7.527cm}}
\pgfpathlineto{\pgfqpoint{3.883cm}{7.527cm}}
\pgfpathlineto{\pgfqpoint{3.956cm}{7.329cm}}
\pgfpathclose
\pgfusepath{fill}
\pgfsetdash{}{0cm}
\pgfsetbuttcap
\pgfsetmiterjoin
\pgfpathmoveto{\pgfqpoint{3.956cm}{7.329cm}}
\pgfpathlineto{\pgfqpoint{4.031cm}{7.527cm}}
\pgfpathlineto{\pgfqpoint{3.883cm}{7.527cm}}
\pgfpathlineto{\pgfqpoint{3.956cm}{7.329cm}}
\pgfpathclose
\pgfusepath{stroke}
\pgfsetdash{}{0cm}
\pgfsetroundcap
\pgfsetroundjoin
\pgfpathmoveto{\pgfqpoint{3.36cm}{4.137cm}}
\pgfpathlineto{\pgfqpoint{3.229cm}{4.511cm}}
\pgfusepath{stroke}
\pgfpathmoveto{\pgfqpoint{3.425cm}{3.951cm}}
\pgfpathlineto{\pgfqpoint{3.429cm}{4.162cm}}
\pgfpathlineto{\pgfqpoint{3.29cm}{4.113cm}}
\pgfpathlineto{\pgfqpoint{3.425cm}{3.951cm}}
\pgfpathclose
\pgfusepath{fill}
\pgfsetdash{}{0cm}
\pgfsetbuttcap
\pgfsetmiterjoin
\pgfpathmoveto{\pgfqpoint{3.425cm}{3.951cm}}
\pgfpathlineto{\pgfqpoint{3.429cm}{4.162cm}}
\pgfpathlineto{\pgfqpoint{3.29cm}{4.113cm}}
\pgfpathlineto{\pgfqpoint{3.425cm}{3.951cm}}
\pgfpathclose
\pgfusepath{stroke}
\begin{pgfscope}
\end{pgfscope}
\end{tikzpicture}

   %\includegraphics[width=.8\textwidth]{map_both.pdf} 
   \caption{Map of UCRB with site numbers (above) and a schematic showing the network connectivity of all the sites (below).}
   \label{fig:map}
\end{figure}

\subsection{Flow Data}
We obtained the most recent monthly natural flow data developed by the Bureau of Reclamation in the UCRB that currently extends to 2007. This data set is developed and updated regularly by the United States Bureau of Reclamation.  Naturalized streamflow are computed by removing anthropogenic impacts (i.e., reservoir regulation, consumptive water use, etc.) from the recorded historic flows.  \cite{Prairie:2005tl} present a detailed description of methods and data used for the computation of natural flows in the CRB. The data is available from (\url{http://www.usbr.gov/lc/region/g4000/NaturalFlow/index. html}). The data is provided as both intervening (gains since the last upstream gauge) or as total flow (sum of all upstream intervening flow at a gauge). We use the intervening data but convert to total flow internally for reasons described in the methodology section (Figure \ref{fig:map}b). All the 20 locations are described In Table \ref{tab:sites}. 

% latex table generated in R 2.12.0 by xtable 1.5-6 package
% Fri Jan  7 16:43:47 2011
\begin{table}[ht]
\begin{center}
\caption{Site Information}\label{tab:sites}
\begin{tabular}{rrp{5cm}}
  \toprule
 Node & USGS Gauge & Site Name \\ 
  \midrule
  1  & 09072500 &  Colorado River At Glenwood Springs, CO        \\ 
  2  & 09095500 &  Colorado River Near Cameo, CO                 \\ 
  3  & 09109000 &  Taylor River Below Taylor Park Reservoir, CO  \\ 
  4  & 09124700 &  Gunnison River Above Blue Mesa Reservoir, CO \\ 
  5  & 09127800 &  Gunnison River At Crystal Reservoir, CO       \\ 
  6  & 09152500 &  Gunnison River Near Grand Junction, CO        \\ 
  7  & 09180000 &  Dolores River Near Cisco, UT                  \\ 
  8  & 09180500 &  Colorado River Near Cisco, UT                 \\ 
  9  & 09211200 &  Green R Bel Fontenelle Res, WY                \\ 
  10 & 09217000 &  Green R. Near Green River, WY                   \\ 
  11 & 09234500 &  Green River Near Greendale, UT                \\ 
  12 & 09251000 &  Yampa River Near Maybell, CO                  \\ 
  13 & 09260000 &  Little Snake River Near Lily, CO              \\ 
  14 & 09302000 &  Duchesne River Near Randlett, UT              \\ 
  15 & 09306500 &  White River Near Watson, UT                   \\ 
  16 & 09315000 &  Green River At Green River, UT                \\ 
  17 & 09328500 &  San Rafael River Near Green River, UT         \\ 
  18 & 09355500 &  San Juan River Near Archuleta, NM             \\ 
  19 & 09379500 &  San Juan River Near Bluff, UT                 \\ 
  20 & 09380000 &  Colorado R At Lees Ferry, AZ                  \\ 
   \bottomrule
\end{tabular}
\end{center}
\end{table}


\subsection{Climate data}
The climate variables are the same as \citep{Bracken:2010cw} though we obtained the most recent observations for all of the predictors (through 2010 where available). The variables used were zonal (ZNW) and meridional (MDW) wind, Sea Surface Temperatures (SST), geopotential height (GPH) and Palmer Drought Severity Index (PDSI, a surrogate for soil moisture) -- all from the NOAA Earth Science Research Laboratory as predictors of large scale climate. Regions of high correlation with Lees Ferry Flow were determined using the ESRL linear correlation tool (\url{http://www.esrl.noaa.gov/psd/data/correlation/}). The variables are averaged over these regions and the resulting time series is used as a predictor. This analysis is repeated for each lead time to obtain a suite of climate predictors. This technique is described in detail in \cite{Grantz:2005ve}and \cite{Regonda2006}.

\subsection{Snow Data}
The amount of snow water equivalent (SWE) data was greatly increased over that used in \cite{Bracken:2010cw}. \cite{Bracken:2010cw} used 10 representative sites  obtained from Natural Resources Conservation Service (NRCS) (\url{http://www.wcc.nrcs.usda.gov/snow}). We use the 86 sites (Figure \ref{fig:map-snow}) that go into the UCRB snowpack report (\url{http://www.usbr.gov/uc/water/notice/snowpack.html}).  As a result of using more data we were able to generate a snowpack predictor for January 1st as well as improve the snow predictors for all other lead times. 


\begin{figure}[htbp] %  figure placement: here, top, bottom, or page
   \centering
   % Created by tikzDevice version 0.6.1 on 2011-06-03 16:39:45
% !TEX encoding = UTF-8 Unicode
\begin{tikzpicture}[x=1pt,y=1pt]
\definecolor[named]{drawColor}{rgb}{0.00,0.00,0.00}
\definecolor[named]{fillColor}{rgb}{1.00,1.00,1.00}
\fill[color=fillColor,] (0,0) rectangle (252.94,361.35);
\begin{scope}
\path[clip] ( 49.20, 61.20) rectangle (227.75,312.15);

\draw[fill opacity=0.00,draw opacity=0.00,] ( 49.20, 61.20) rectangle (227.74,312.15);
\definecolor[named]{drawColor}{rgb}{0.00,0.00,0.00}

\draw[color=drawColor,line cap=round,line join=round,fill opacity=0.00,] (112.12,284.08) --
	(112.16,284.07) --
	(112.17,284.07) --
	(112.19,284.07) --
	(112.25,284.07) --
	(112.29,284.05) --
	(112.33,284.05) --
	(112.39,284.04) --
	(112.43,284.03) --
	(112.51,283.97) --
	(112.52,283.95) --
	(112.53,283.91) --
	(112.53,283.90) --
	(112.51,283.87) --
	(112.50,283.82) --
	(112.51,283.80) --
	(112.52,283.78) --
	(112.52,283.77) --
	(112.56,283.67) --
	(112.57,283.64) --
	(112.60,283.62) --
	(112.60,283.60) --
	(112.62,283.56) --
	(112.64,283.55) --
	(112.68,283.53) --
	(112.70,283.52) --
	(112.75,283.51) --
	(112.79,283.51) --
	(112.82,283.52) --
	(112.86,283.53) --
	(112.92,283.55) --
	(112.96,283.57) --
	(112.98,283.58) --
	(113.00,283.60) --
	(113.02,283.61) --
	(113.05,283.62) --
	(113.11,283.65) --
	(113.16,283.67) --
	(113.23,283.68) --
	(113.23,283.69) --
	(113.23,283.69) --
	(113.23,283.70) --
	(113.23,283.70) --
	(113.23,283.71) --
	(113.23,283.71) --
	(113.23,283.72) --
	(113.23,283.72) --
	(113.23,283.73) --
	(113.23,283.73) --
	(113.23,283.74) --
	(113.27,283.74) --
	(113.33,283.73) --
	(113.37,283.71) --
	(113.39,283.67) --
	(113.39,283.64) --
	(113.38,283.56) --
	(113.40,283.53) --
	(113.42,283.52) --
	(113.44,283.50) --
	(113.45,283.47) --
	(113.47,283.44) --
	(113.50,283.41) --
	(113.55,283.38) --
	(113.60,283.36) --
	(113.64,283.34) --
	(113.72,283.32) --
	(113.78,283.29) --
	(113.84,283.26) --
	(113.91,283.21) --
	(114.01,283.17) --
	(114.08,283.15) --
	(114.15,283.13) --
	(114.21,283.13) --
	(114.29,283.14) --
	(114.37,283.15) --
	(114.53,283.21) --
	(114.56,283.22) --
	(114.61,283.22) --
	(114.67,283.18) --
	(114.72,283.14) --
	(114.75,283.10) --
	(114.78,283.07) --
	(114.82,283.05) --
	(114.86,283.01) --
	(114.87,282.98) --
	(114.88,282.94) --
	(114.88,282.88) --
	(114.91,282.83) --
	(114.97,282.73) --
	(115.08,282.64) --
	(115.18,282.56) --
	(115.35,282.46) --
	(115.43,282.42) --
	(115.54,282.36) --
	(115.60,282.35) --
	(115.67,282.33) --
	(115.74,282.30) --
	(115.80,282.27) --
	(115.91,282.23) --
	(115.96,282.22) --
	(116.04,282.21) --
	(116.11,282.21) --
	(116.21,282.23) --
	(116.27,282.22) --
	(116.31,282.21) --
	(116.34,282.19) --
	(116.38,282.18) --
	(116.43,282.17) --
	(116.49,282.16) --
	(116.55,282.15) --
	(116.59,282.15) --
	(116.66,282.15) --
	(116.73,282.17) --
	(116.84,282.21) --
	(116.86,282.22) --
	(116.90,282.23) --
	(116.96,282.23) --
	(117.03,282.22) --
	(117.19,282.19) --
	(117.27,282.19) --
	(117.30,282.18) --
	(117.36,282.16) --
	(117.39,282.17) --
	(117.53,282.16) --
	(117.60,282.16) --
	(117.65,282.17) --
	(117.76,282.21) --
	(117.80,282.23) --
	(117.82,282.25) --
	(117.85,282.26) --
	(117.88,282.27) --
	(117.93,282.27) --
	(117.99,282.28) --
	(118.05,282.30) --
	(118.19,282.35) --
	(118.22,282.35) --
	(118.28,282.35) --
	(118.35,282.35) --
	(118.42,282.33) --
	(118.47,282.31) --
	(118.52,282.30) --
	(118.57,282.30) --
	(118.60,282.30) --
	(118.70,282.20) --
	(118.75,282.16) --
	(118.79,282.13) --
	(118.84,282.11) --
	(118.93,282.08) --
	(118.97,282.05) --
	(119.01,282.02) --
	(119.03,281.99) --
	(119.05,281.94) --
	(119.09,281.89) --
	(119.16,281.82) --
	(119.21,281.79) --
	(119.24,281.78) --
	(119.28,281.77) --
	(119.34,281.77) --
	(119.40,281.79) --
	(119.50,281.79) --
	(119.63,281.76) --
	(119.69,281.77) --
	(119.71,281.77) --
	(119.79,281.77) --
	(119.87,281.79) --
	(120.04,281.86) --
	(120.18,281.92) --
	(120.22,281.94) --
	(120.26,281.95) --
	(120.33,281.96) --
	(120.38,281.95) --
	(120.45,281.93) --
	(120.54,281.90) --
	(120.60,281.88) --
	(120.64,281.85) --
	(120.71,281.81) --
	(120.74,281.77) --
	(120.77,281.71) --
	(120.77,281.64) --
	(120.77,281.56) --
	(120.75,281.47) --
	(120.73,281.41) --
	(120.65,281.26) --
	(120.61,281.17) --
	(120.61,281.15) --
	(120.63,281.11) --
	(120.66,281.05) --
	(120.72,280.91) --
	(120.76,280.83) --
	(120.81,280.76) --
	(120.89,280.69) --
	(120.95,280.67) --
	(121.08,280.66) --
	(121.10,280.66) --
	(121.12,280.64) --
	(121.14,280.59) --
	(121.16,280.54) --
	(121.18,280.50) --
	(121.28,280.42) --
	(121.31,280.39) --
	(121.33,280.33) --
	(121.33,280.28) --
	(121.31,280.17) --
	(121.27,280.14) --
	(121.15,280.08) --
	(121.10,280.06) --
	(121.09,280.05) --
	(121.08,280.03) --
	(121.10,279.96) --
	(121.09,279.89) --
	(121.08,279.82) --
	(121.04,279.70) --
	(121.04,279.67) --
	(121.05,279.64) --
	(121.07,279.61) --
	(121.09,279.59) --
	(121.11,279.56) --
	(121.14,279.50) --
	(121.14,279.46) --
	(121.10,279.42) --
	(121.00,279.34) --
	(120.91,279.28) --
	(120.90,279.26) --
	(120.91,279.23) --
	(120.94,279.20) --
	(120.98,279.18) --
	(121.03,279.17) --
	(121.08,279.14) --
	(121.11,279.12) --
	(121.16,279.06) --
	(121.19,279.02) --
	(121.20,278.97) --
	(121.20,278.91) --
	(121.20,278.82) --
	(121.21,278.76) --
	(121.21,278.71) --
	(121.20,278.65) --
	(121.18,278.59) --
	(121.18,278.57) --
	(121.22,278.54) --
	(121.24,278.51) --
	(121.25,278.48) --
	(121.24,278.43) --
	(121.24,278.37) --
	(121.24,278.31) --
	(121.23,278.25) --
	(121.22,278.13) --
	(121.22,278.09) --
	(121.23,278.06) --
	(121.25,278.03) --
	(121.31,278.01) --
	(121.37,277.97) --
	(121.47,277.85) --
	(121.49,277.79) --
	(121.52,277.76) --
	(121.58,277.73) --
	(121.60,277.71) --
	(121.63,277.66) --
	(121.67,277.59) --
	(121.71,277.49) --
	(121.74,277.42) --
	(121.75,277.37) --
	(121.76,277.30) --
	(121.76,277.25) --
	(121.75,277.20) --
	(121.74,277.10) --
	(121.74,277.04) --
	(121.76,277.00) --
	(121.80,276.95) --
	(121.84,276.91) --
	(121.86,276.86) --
	(121.89,276.83) --
	(121.91,276.82) --
	(121.94,276.81) --
	(121.96,276.80) --
	(122.05,276.80) --
	(122.09,276.80) --
	(122.16,276.82) --
	(122.22,276.81) --
	(122.35,276.78) --
	(122.37,276.76) --
	(122.37,276.74) --
	(122.36,276.70) --
	(122.35,276.64) --
	(122.33,276.54) --
	(122.32,276.49) --
	(122.32,276.44) --
	(122.35,276.33) --
	(122.40,276.23) --
	(122.42,276.20) --
	(122.44,276.18) --
	(122.46,276.15) --
	(122.48,276.10) --
	(122.51,276.03) --
	(122.54,275.96) --
	(122.57,275.88) --
	(122.58,275.86) --
	(122.60,275.84) --
	(122.64,275.83) --
	(122.68,275.83) --
	(122.72,275.82) --
	(122.79,275.82) --
	(122.83,275.81) --
	(122.90,275.77) --
	(122.93,275.74) --
	(122.95,275.72) --
	(123.00,275.70) --
	(123.06,275.69) --
	(123.11,275.69) --
	(123.20,275.65) --
	(123.30,275.57) --
	(123.45,275.43) --
	(123.52,275.32) --
	(123.56,275.24) --
	(123.58,275.23) --
	(123.61,275.23) --
	(123.68,275.22) --
	(123.73,275.23) --
	(123.78,275.23) --
	(123.80,275.20) --
	(123.82,275.16) --
	(123.83,275.13) --
	(123.84,275.09) --
	(123.84,275.06) --
	(123.84,275.02) --
	(123.85,275.00) --
	(123.87,274.96) --
	(123.91,274.92) --
	(123.97,274.89) --
	(124.01,274.85) --
	(124.04,274.79) --
	(124.05,274.76) --
	(124.05,274.72) --
	(124.05,274.69) --
	(124.06,274.65) --
	(124.08,274.61) --
	(124.12,274.56) --
	(124.20,274.53) --
	(124.22,274.50) --
	(124.22,274.45) --
	(124.23,274.38) --
	(124.23,274.32) --
	(124.24,274.28) --
	(124.24,274.24) --
	(124.19,274.19) --
	(124.12,274.12) --
	(124.06,274.06) --
	(124.04,274.05) --
	(124.03,274.04) --
	(124.03,274.03) --
	(124.03,274.01) --
	(124.05,273.97) --
	(124.07,273.91) --
	(124.08,273.84) --
	(124.12,273.73) --
	(124.14,273.69) --
	(124.17,273.66) --
	(124.22,273.62) --
	(124.24,273.58) --
	(124.25,273.55) --
	(124.24,273.51) --
	(124.23,273.48) --
	(124.21,273.42) --
	(124.19,273.35) --
	(124.18,273.31) --
	(124.19,273.27) --
	(124.21,273.22) --
	(124.25,273.13) --
	(124.25,273.08) --
	(124.26,273.07) --
	(124.29,273.06) --
	(124.33,273.05) --
	(124.41,273.04) --
	(124.47,273.02) --
	(124.50,273.02) --
	(124.60,273.06) --
	(124.70,273.09) --
	(124.74,273.09) --
	(124.81,273.08) --
	(124.89,273.05) --
	(124.92,273.03) --
	(124.95,273.00) --
	(124.97,272.99) --
	(125.00,272.99) --
	(125.03,273.00) --
	(125.07,273.01) --
	(125.10,273.01) --
	(125.20,273.00) --
	(125.32,272.99) --
	(125.43,272.99) --
	(125.49,272.98) --
	(125.55,272.95) --
	(125.57,272.93) --
	(125.59,272.90) --
	(125.60,272.86) --
	(125.63,272.84) --
	(125.66,272.81) --
	(125.69,272.79) --
	(125.72,272.76) --
	(125.73,272.75) --
	(125.73,272.75) --
	(125.72,272.75) --
	(125.72,272.75) --
	(125.71,272.75) --
	(125.71,272.75) --
	(125.71,272.71) --
	(125.73,272.69) --
	(125.75,272.65) --
	(125.85,272.61) --
	(125.93,272.60) --
	(125.97,272.60) --
	(126.11,272.62) --
	(126.15,272.62) --
	(126.17,272.62) --
	(126.20,272.60) --
	(126.27,272.52) --
	(126.28,272.51) --
	(126.30,272.50) --
	(126.33,272.51) --
	(126.37,272.53) --
	(126.40,272.54) --
	(126.42,272.53) --
	(126.46,272.52) --
	(126.49,272.50) --
	(126.51,272.46) --
	(126.51,272.44) --
	(126.51,272.41) --
	(126.49,272.38) --
	(126.48,272.33) --
	(126.48,272.30) --
	(126.58,272.16) --
	(126.64,272.11) --
	(126.66,272.06) --
	(126.68,272.05) --
	(126.70,272.04) --
	(126.71,272.04) --
	(126.77,272.05) --
	(126.85,272.03) --
	(126.88,272.01) --
	(126.94,271.98) --
	(126.99,271.96) --
	(127.04,271.96) --
	(127.16,271.95) --
	(127.19,271.94) --
	(127.21,271.93) --
	(127.36,271.76) --
	(127.40,271.73) --
	(127.46,271.71) --
	(127.66,271.66) --
	(127.69,271.66) --
	(127.72,271.65) --
	(127.91,271.51) --
	(128.00,271.44) --
	(128.03,271.41) --
	(128.06,271.34) --
	(128.07,271.31) --
	(128.08,271.29) --
	(128.10,271.27) --
	(128.12,271.26) --
	(128.16,271.26) --
	(128.31,271.26) --
	(128.38,271.27) --
	(128.47,271.28) --
	(128.49,271.29) --
	(128.57,271.34) --
	(128.70,271.42) --
	(128.72,271.42) --
	(128.73,271.42) --
	(128.74,271.41) --
	(128.75,271.40) --
	(128.76,271.37) --
	(128.85,271.29) --
	(128.96,271.21) --
	(128.99,271.17) --
	(129.03,271.13) --
	(129.10,270.96) --
	(129.14,270.91) --
	(129.24,270.82) --
	(129.33,270.76) --
	(129.37,270.74) --
	(129.43,270.60) --
	(129.46,270.50) --
	(129.47,270.43) --
	(129.50,270.37) --
	(129.51,270.34) --
	(129.55,270.26) --
	(129.58,270.23) --
	(129.63,270.20) --
	(129.70,270.17) --
	(129.72,270.15) --
	(129.76,270.04) --
	(129.78,269.92) --
	(129.78,269.84) --
	(129.77,269.71) --
	(129.78,269.68) --
	(129.86,269.54) --
	(129.94,269.40) --
	(129.99,269.35) --
	(130.03,269.27) --
	(130.06,269.23) --
	(130.08,269.22) --
	(130.12,269.20) --
	(130.35,269.14) --
	(130.40,269.12) --
	(130.44,269.10) --
	(130.49,269.07) --
	(130.54,269.02) --
	(130.59,269.00) --
	(130.63,268.97) --
	(130.69,268.96) --
	(130.78,268.94) --
	(130.84,268.94) --
	(130.87,268.93) --
	(130.90,268.91) --
	(130.95,268.89) --
	(131.00,268.82) --
	(131.02,268.76) --
	(131.05,268.73) --
	(131.12,268.68) --
	(131.18,268.63) --
	(131.20,268.59) --
	(131.21,268.56) --
	(131.21,268.49) --
	(131.22,268.47) --
	(131.24,268.44) --
	(131.28,268.42) --
	(131.33,268.41) --
	(131.39,268.40) --
	(131.41,268.39) --
	(131.43,268.38) --
	(131.46,268.30) --
	(131.51,268.20) --
	(131.55,268.15) --
	(131.58,268.15) --
	(131.61,268.13) --
	(131.74,268.02) --
	(131.77,268.00) --
	(131.81,267.96) --
	(131.87,267.94) --
	(131.89,267.94) --
	(131.91,267.95) --
	(131.93,267.99) --
	(131.97,268.03) --
	(132.02,268.05) --
	(132.04,268.05) --
	(132.07,268.05) --
	(132.10,268.02) --
	(132.17,267.96) --
	(132.20,267.94) --
	(132.25,267.92) --
	(132.38,267.89) --
	(132.51,267.87) --
	(132.54,267.87) --
	(132.58,267.84) --
	(132.62,267.79) --
	(132.64,267.76) --
	(132.65,267.73) --
	(132.68,267.69) --
	(132.68,267.66) --
	(132.65,267.64) --
	(132.60,267.63) --
	(132.55,267.62) --
	(132.51,267.60) --
	(132.45,267.61) --
	(132.41,267.61) --
	(132.39,267.60) --
	(132.36,267.58) --
	(132.34,267.56) --
	(132.32,267.54) --
	(132.29,267.49) --
	(132.25,267.45) --
	(132.20,267.43) --
	(132.17,267.40) --
	(132.12,267.35) --
	(132.07,267.27) --
	(132.04,267.25) --
	(132.04,267.21) --
	(132.05,267.19) --
	(132.07,267.16) --
	(132.08,267.13) --
	(132.08,267.11) --
	(132.06,267.08) --
	(132.06,267.06) --
	(132.06,267.04) --
	(132.07,267.02) --
	(132.16,266.96) --
	(132.25,266.92) --
	(132.31,266.89) --
	(132.40,266.89) --
	(132.44,266.89) --
	(132.48,266.90) --
	(132.51,266.91) --
	(132.55,266.94) --
	(132.63,267.01) --
	(132.72,267.07) --
	(132.76,267.09) --
	(132.80,267.09) --
	(132.84,267.09) --
	(132.92,267.06) --
	(132.97,267.05) --
	(133.06,267.03) --
	(133.08,267.01) --
	(133.12,266.98) --
	(133.18,266.94) --
	(133.28,266.89) --
	(133.31,266.86) --
	(133.35,266.79) --
	(133.38,266.74) --
	(133.41,266.70) --
	(133.44,266.67) --
	(133.49,266.64) --
	(133.75,266.58) --
	(133.87,266.54) --
	(133.91,266.53) --
	(133.95,266.52) --
	(133.98,266.50) --
	(134.01,266.46) --
	(134.03,266.44) --
	(134.15,266.39) --
	(134.19,266.36) --
	(134.24,266.31) --
	(134.28,266.26) --
	(134.29,266.23) --
	(134.29,266.10) --
	(134.31,266.05) --
	(134.32,266.03) --
	(134.35,266.00) --
	(134.41,265.97) --
	(134.52,265.93) --
	(134.55,265.91) --
	(134.57,265.88) --
	(134.59,265.84) --
	(134.63,265.77) --
	(134.66,265.67) --
	(134.68,265.64) --
	(134.76,265.54) --
	(134.75,265.52) --
	(134.70,265.48) --
	(134.64,265.40) --
	(134.61,265.32) --
	(134.60,265.27) --
	(134.60,265.25) --
	(134.61,265.21) --
	(134.62,265.18) --
	(134.73,265.01) --
	(134.76,264.95) --
	(134.77,264.89) --
	(134.77,264.80) --
	(134.76,264.76) --
	(134.77,264.73) --
	(134.79,264.70) --
	(134.83,264.61) --
	(134.88,264.55) --
	(135.02,264.43) --
	(135.15,264.33) --
	(135.18,264.30) --
	(135.21,264.26) --
	(135.25,264.14) --
	(135.29,264.09) --
	(135.32,264.07) --
	(135.37,264.05) --
	(135.41,264.04) --
	(135.46,264.02) --
	(135.51,264.00) --
	(135.57,263.95) --
	(135.59,263.92) --
	(135.60,263.90) --
	(135.61,263.85) --
	(135.60,263.80) --
	(135.58,263.76) --
	(135.58,263.72) --
	(135.59,263.69) --
	(135.60,263.62) --
	(135.59,263.59) --
	(135.57,263.57) --
	(135.55,263.55) --
	(135.46,263.49) --
	(135.29,263.43) --
	(135.24,263.42) --
	(135.19,263.39) --
	(135.14,263.35) --
	(135.10,263.30) --
	(135.09,263.26) --
	(135.10,263.23) --
	(135.13,263.17) --
	(135.13,263.09) --
	(135.13,263.07) --
	(135.13,263.03) --
	(135.11,263.01) --
	(134.99,262.94) --
	(134.81,262.80) --
	(134.76,262.76) --
	(134.70,262.72) --
	(134.66,262.68) --
	(134.63,262.63) --
	(134.61,262.56) --
	(134.59,262.51) --
	(134.59,262.49) --
	(134.56,262.46) --
	(134.53,262.42) --
	(134.45,262.30) --
	(134.38,262.26) --
	(134.34,262.23) --
	(134.28,262.22) --
	(134.16,262.23) --
	(134.08,262.23) --
	(134.04,262.22) --
	(133.97,262.19) --
	(133.89,262.18) --
	(133.62,262.13) --
	(133.56,262.10) --
	(133.46,261.97) --
	(133.43,261.95) --
	(133.38,261.92) --
	(133.36,261.91) --
	(133.36,261.89) --
	(133.36,261.85) --
	(133.36,261.81) --
	(133.35,261.78) --
	(133.32,261.72) --
	(133.31,261.62) --
	(133.30,261.53) --
	(133.29,261.50) --
	(133.24,261.41) --
	(133.18,261.32) --
	(133.16,261.27) --
	(133.07,261.02) --
	(132.98,260.90) --
	(132.97,260.85) --
	(132.97,260.82) --
	(132.97,260.80) --
	(132.98,260.75) --
	(133.02,260.72) --
	(133.10,260.68) --
	(133.17,260.64) --
	(133.24,260.59) --
	(133.26,260.55) --
	(133.26,260.49) --
	(133.25,260.39) --
	(133.24,260.34) --
	(133.24,260.31) --
	(133.25,260.28) --
	(133.27,260.24) --
	(133.32,260.20) --
	(133.38,260.17) --
	(133.59,260.14) --
	(133.64,260.13) --
	(133.72,260.07) --
	(133.77,260.05) --
	(133.84,260.05) --
	(133.98,260.04) --
	(134.11,260.01) --
	(134.21,259.97) --
	(134.34,259.91) --
	(134.40,259.88) --
	(134.49,259.86) --
	(134.64,259.86) --
	(134.68,259.86) --
	(134.70,259.85) --
	(134.72,259.83) --
	(134.74,259.81) --
	(134.80,259.64) --
	(134.81,259.59) --
	(134.82,259.55) --
	(134.81,259.52) --
	(134.80,259.49) --
	(134.80,259.46) --
	(134.85,259.40) --
	(134.89,259.32) --
	(134.91,259.29) --
	(134.95,259.27) --
	(135.01,259.25) --
	(135.07,259.25) --
	(135.14,259.26) --
	(135.25,259.30) --
	(135.29,259.30) --
	(135.33,259.30) --
	(135.47,259.27) --
	(135.54,259.27) --
	(135.57,259.27) --
	(135.65,259.30) --
	(135.69,259.30) --
	(135.72,259.30) --
	(135.76,259.29) --
	(135.79,259.28) --
	(135.86,259.20) --
	(135.91,259.17) --
	(135.96,259.15) --
	(136.03,259.14) --
	(136.20,259.12) --
	(136.35,259.09) --
	(136.39,259.07) --
	(136.46,259.03) --
	(136.46,259.00) --
	(136.47,258.94) --
	(136.45,258.87) --
	(136.43,258.82) --
	(136.44,258.80) --
	(136.50,258.78) --
	(136.57,258.77) --
	(136.65,258.79) --
	(136.73,258.81) --
	(136.78,258.80) --
	(136.82,258.79) --
	(136.89,258.80) --
	(136.94,258.81) --
	(137.05,258.82) --
	(137.10,258.82) --
	(137.15,258.80) --
	(137.18,258.78) --
	(137.26,258.71) --
	(137.31,258.64) --
	(137.34,258.49) --
	(137.36,258.47) --
	(137.39,258.47) --
	(137.43,258.48) --
	(137.47,258.47) --
	(137.54,258.48) --
	(137.57,258.49) --
	(137.67,258.53) --
	(137.80,258.60) --
	(137.83,258.62) --
	(137.85,258.65) --
	(137.86,258.66) --
	(137.88,258.67) --
	(138.06,258.65) --
	(138.10,258.63) --
	(138.12,258.62) --
	(138.14,258.50) --
	(138.16,258.45) --
	(138.16,258.41) --
	(138.18,258.38) --
	(138.21,258.36) --
	(138.26,258.35) --
	(138.34,258.34) --
	(138.39,258.34) --
	(138.47,258.33) --
	(138.55,258.31) --
	(138.63,258.28) --
	(138.65,258.27) --
	(138.74,258.16) --
	(138.85,258.07) --
	(139.14,257.86) --
	(139.38,257.74) --
	(139.48,257.72) --
	(139.52,257.70) --
	(139.60,257.65) --
	(139.66,257.62) --
	(139.74,257.57) --
	(139.80,257.54) --
	(140.07,257.35) --
	(140.35,257.16) --
	(140.43,257.10) --
	(140.48,257.05) --
	(140.50,257.03) --
	(140.50,257.01) --
	(140.49,257.00) --
	(140.46,256.98) --
	(140.41,256.97) --
	(140.35,256.97) --
	(140.31,256.97) --
	(140.30,256.96) --
	(140.29,256.93) --
	(140.30,256.91) --
	(140.33,256.89) --
	(140.38,256.87) --
	(140.62,256.78) --
	(140.83,256.71) --
	(140.91,256.68) --
	(140.97,256.66) --
	(141.02,256.63) --
	(141.10,256.54) --
	(141.13,256.45) --
	(141.13,256.35) --
	(141.15,256.29) --
	(141.19,256.21) --
	(141.23,256.14) --
	(141.29,256.06) --
	(141.39,255.94) --
	(141.43,255.88) --
	(141.46,255.84) --
	(141.49,255.81) --
	(141.53,255.78) --
	(141.56,255.76) --
	(141.62,255.75) --
	(141.72,255.75) --
	(141.90,255.75) --
	(141.97,255.73) --
	(142.06,255.68) --
	(142.12,255.63) --
	(142.13,255.62) --
	(142.13,255.59) --
	(142.10,255.57) --
	(142.03,255.56) --
	(141.91,255.53) --
	(141.74,255.46) --
	(141.65,255.41) --
	(141.59,255.38) --
	(141.51,255.36) --
	(141.34,255.30) --
	(141.22,255.23) --
	(141.15,255.15) --
	(141.13,255.10) --
	(141.12,255.06) --
	(141.12,255.03) --
	(141.15,255.01) --
	(141.21,254.98) --
	(141.28,254.95) --
	(141.33,254.91) --
	(141.37,254.85) --
	(141.42,254.81) --
	(141.47,254.76) --
	(141.55,254.70) --
	(141.57,254.68) --
	(141.59,254.65) --
	(141.60,254.62) --
	(141.62,254.57) --
	(141.62,254.53) --
	(141.63,254.51) --
	(141.58,254.48) --
	(141.52,254.43) --
	(141.48,254.39) --
	(141.30,254.28) --
	(141.23,254.22) --
	(141.17,254.14) --
	(141.02,254.03) --
	(140.98,253.98) --
	(140.96,253.94) --
	(140.96,253.92) --
	(140.96,253.88) --
	(140.99,253.83) --
	(140.99,253.82) --
	(140.98,253.79) --
	(140.95,253.76) --
	(140.95,253.69) --
	(140.95,253.66) --
	(140.97,253.63) --
	(141.00,253.60) --
	(141.11,253.52) --
	(141.23,253.42) --
	(141.29,253.37) --
	(141.31,253.34) --
	(141.32,253.31) --
	(141.32,253.25) --
	(141.31,253.17) --
	(141.29,253.12) --
	(141.26,253.07) --
	(141.24,253.03) --
	(141.21,252.99) --
	(141.14,252.91) --
	(141.10,252.89) --
	(141.08,252.87) --
	(141.05,252.83) --
	(141.04,252.81) --
	(141.04,252.78) --
	(141.07,252.66) --
	(141.11,252.53) --
	(141.13,252.48) --
	(141.15,252.32) --
	(141.18,252.22) --
	(141.23,252.05) --
	(141.28,251.92) --
	(141.33,251.85) --
	(141.42,251.76) --
	(141.50,251.70) --
	(141.55,251.65) --
	(141.65,251.59) --
	(141.67,251.57) --
	(141.69,251.56) --
	(141.69,251.49) --
	(141.70,251.45) --
	(141.68,251.40) --
	(141.66,251.35) --
	(141.65,251.29) --
	(141.63,251.21) --
	(141.63,250.92) --
	(141.62,250.84) --
	(141.62,250.71) --
	(141.61,250.63) --
	(141.59,250.60) --
	(141.43,250.43) --
	(141.39,250.38) --
	(141.37,250.34) --
	(141.35,250.31) --
	(141.29,250.26) --
	(141.22,250.21) --
	(141.16,250.18) --
	(140.91,250.07) --
	(140.81,250.03) --
	(140.74,249.98) --
	(140.60,249.89) --
	(140.56,249.84) --
	(140.53,249.80) --
	(140.47,249.73) --
	(140.40,249.60) --
	(140.34,249.48) --
	(140.30,249.34) --
	(140.30,249.29) --
	(140.30,249.26) --
	(140.32,249.24) --
	(140.37,249.18) --
	(140.41,249.13) --
	(140.42,249.10) --
	(140.42,249.05) --
	(140.39,248.99) --
	(140.37,248.94) --
	(140.36,248.90) --
	(140.32,248.69) --
	(140.29,248.62) --
	(140.26,248.58) --
	(140.21,248.48) --
	(140.17,248.39) --
	(139.91,248.08) --
	(139.90,248.08) --
	(139.90,248.08) --
	(139.89,248.08) --
	(139.89,248.08) --
	(139.88,248.08) --
	(139.88,248.08) --
	(139.87,248.08) --
	(139.87,248.08) --
	(139.86,248.08) --
	(139.86,248.08) --
	(139.85,248.08) --
	(139.85,248.08) --
	(139.84,248.08) --
	(139.84,248.08) --
	(139.83,248.08) --
	(139.82,248.07) --
	(139.64,247.97) --
	(139.62,247.88) --
	(139.61,247.86) --
	(139.59,247.76) --
	(139.56,247.68) --
	(139.53,247.65) --
	(139.50,247.61) --
	(139.50,247.59) --
	(139.48,247.59) --
	(139.44,247.55) --
	(139.39,247.50) --
	(139.37,247.49) --
	(139.35,247.47) --
	(139.33,247.45) --
	(139.32,247.43) --
	(139.30,247.37) --
	(139.24,247.33) --
	(139.17,247.29) --
	(139.08,247.25) --
	(139.00,247.23) --
	(138.97,247.23) --
	(138.92,247.24) --
	(138.89,247.24) --
	(138.87,247.26) --
	(138.76,247.26) --
	(138.70,247.26) --
	(138.67,247.26) --
	(138.60,247.28) --
	(138.55,247.31) --
	(138.44,247.32) --
	(138.41,247.31) --
	(138.39,247.30) --
	(138.34,247.28) --
	(138.31,247.24) --
	(138.27,247.21) --
	(138.21,247.18) --
	(138.18,247.17) --
	(138.11,247.16) --
	(138.09,247.13) --
	(138.08,247.11) --
	(138.08,246.91) --
	(138.06,246.78) --
	(138.02,246.67) --
	(137.97,246.58) --
	(137.91,246.50) --
	(137.85,246.44) --
	(137.72,246.34) --
	(137.63,246.26) --
	(137.61,246.25) --
	(137.60,246.23) --
	(137.48,246.09) --
	(137.46,246.08) --
	(137.44,246.06) --
	(137.38,246.05) --
	(137.33,246.05) --
	(137.30,246.05) --
	(137.22,246.08) --
	(137.18,246.10) --
	(137.15,246.11) --
	(137.06,246.15) --
	(136.97,246.20) --
	(136.71,246.30) --
	(136.56,246.34) --
	(136.37,246.38) --
	(136.31,246.36) --
	(136.30,246.35) --
	(136.29,246.33) --
	(136.30,246.27) --
	(136.33,246.19) --
	(136.33,246.17) --
	(136.33,246.12) --
	(136.33,246.06) --
	(136.33,246.02) --
	(136.33,245.98) --
	(136.34,245.94) --
	(136.36,245.84) --
	(136.38,245.80) --
	(136.42,245.72) --
	(136.45,245.67) --
	(136.57,245.55) --
	(136.59,245.52) --
	(136.61,245.51) --
	(136.61,245.49) --
	(136.63,245.47) --
	(136.69,245.42) --
	(136.72,245.39) --
	(136.77,245.37) --
	(136.88,245.31) --
	(136.92,245.28) --
	(136.98,245.24) --
	(137.04,245.21) --
	(137.07,245.19) --
	(137.14,245.12) --
	(137.25,245.05) --
	(137.36,244.96) --
	(137.43,244.90) --
	(137.46,244.84) --
	(137.47,244.78) --
	(137.48,244.72) --
	(137.50,244.70) --
	(137.57,244.64) --
	(137.59,244.61) --
	(137.59,244.60) --
	(137.59,244.59) --
	(137.59,244.57) --
	(137.52,244.41) --
	(137.50,244.40) --
	(137.50,244.37) --
	(137.49,244.36) --
	(137.47,244.34) --
	(137.45,244.30) --
	(137.43,244.29) --
	(137.39,244.23) --
	(137.37,244.22) --
	(137.34,244.18) --
	(137.34,244.13) --
	(137.34,244.13) --
	(137.36,244.09) --
	(137.39,244.03) --
	(137.43,244.00) --
	(137.47,243.95) --
	(137.53,243.88) --
	(137.64,243.77) --
	(137.69,243.72) --
	(137.76,243.65) --
	(137.79,243.60) --
	(137.81,243.53) --
	(137.82,243.52) --
	(137.84,243.50) --
	(137.84,243.48) --
	(137.89,243.39) --
	(137.94,243.29) --
	(137.94,243.27) --
	(137.95,243.23) --
	(137.97,243.19) --
	(138.02,243.14) --
	(138.07,243.12) --
	(138.10,243.11) --
	(138.13,243.11) --
	(138.29,243.13) --
	(138.42,243.15) --
	(138.44,243.16) --
	(138.46,243.18) --
	(138.49,243.22) --
	(138.51,243.26) --
	(138.54,243.29) --
	(138.56,243.31) --
	(138.58,243.33) --
	(138.62,243.38) --
	(138.63,243.40) --
	(138.65,243.42) --
	(138.67,243.46) --
	(138.70,243.47) --
	(138.74,243.52) --
	(138.77,243.54) --
	(138.85,243.55) --
	(138.90,243.55) --
	(138.96,243.55) --
	(138.99,243.56) --
	(139.01,243.57) --
	(139.04,243.60) --
	(139.16,243.65) --
	(139.22,243.67) --
	(139.29,243.68) --
	(139.46,243.69) --
	(139.81,243.72) --
	(139.92,243.74) --
	(139.94,243.74) --
	(140.00,243.75) --
	(140.11,243.78) --
	(140.16,243.78) --
	(140.19,243.79) --
	(140.22,243.79) --
	(140.24,243.81) --
	(140.29,243.82) --
	(140.39,243.85) --
	(140.53,243.92) --
	(140.57,243.94) --
	(140.65,243.97) --
	(140.72,244.00) --
	(140.80,244.01) --
	(140.93,244.03) --
	(141.34,244.05) --
	(141.50,244.05) --
	(141.61,244.05) --
	(141.97,244.07) --
	(142.05,244.08) --
	(142.22,244.10) --
	(142.34,244.13) --
	(142.37,244.13) --
	(142.39,244.14) --
	(142.53,244.15) --
	(142.62,244.15) --
	(142.70,244.14) --
	(142.78,244.12) --
	(142.90,244.08) --
	(142.99,244.05) --
	(143.14,243.97) --
	(143.21,243.93) --
	(143.25,243.91) --
	(143.48,243.79) --
	(143.63,243.70) --
	(143.71,243.64) --
	(143.77,243.60) --
	(143.80,243.59) --
	(143.91,243.52) --
	(143.95,243.50) --
	(144.10,243.42) --
	(144.15,243.40) --
	(144.20,243.38) --
	(144.28,243.36) --
	(144.50,243.36) --
	(144.56,243.36) --
	(144.67,243.36) --
	(144.86,243.36) --
	(144.91,243.36) --
	(144.99,243.38) --
	(145.16,243.39) --
	(145.21,243.39) --
	(145.32,243.37) --
	(145.36,243.35) --
	(145.40,243.32) --
	(145.46,243.28) --
	(145.47,243.26) --
	(145.51,243.23) --
	(145.60,243.14) --
	(145.62,243.12) --
	(145.63,243.11) --
	(145.73,243.03) --
	(145.78,243.00) --
	(145.78,242.99) --
	(145.78,242.99) --
	(145.80,242.97) --
	(145.84,242.95) --
	(145.87,242.94) --
	(145.89,242.93) --
	(145.97,242.90) --
	(146.02,242.88) --
	(146.04,242.88) --
	(146.06,242.86) --
	(146.16,242.83) --
	(146.24,242.81) --
	(146.27,242.80) --
	(146.37,242.79) --
	(146.46,242.79) --
	(146.48,242.78) --
	(146.50,242.77) --
	(146.56,242.73) --
	(146.61,242.68) --
	(146.64,242.64) --
	(146.68,242.57) --
	(146.69,242.55) --
	(146.72,242.51) --
	(146.80,242.46) --
	(146.87,242.42) --
	(147.06,242.35) --
	(147.30,242.29) --
	(147.32,242.28) --
	(147.40,242.25) --
	(147.43,242.24) --
	(147.53,242.20) --
	(147.56,242.19) --
	(147.60,242.17) --
	(147.64,242.14) --
	(147.71,242.11) --
	(147.78,242.07) --
	(147.84,242.03) --
	(147.98,241.91) --
	(147.99,241.89) --
	(148.01,241.87) --
	(148.06,241.80) --
	(148.11,241.70) --
	(148.11,241.68) --
	(148.13,241.64) --
	(148.14,241.61) --
	(148.14,241.57) --
	(148.14,241.50) --
	(148.11,241.36) --
	(148.07,241.12) --
	(148.07,241.10) --
	(148.07,240.93) --
	(148.07,240.85) --
	(148.07,240.83) --
	(148.08,240.81) --
	(148.09,240.77) --
	(148.10,240.73) --
	(148.11,240.68) --
	(148.12,240.47) --
	(148.12,240.43) --
	(148.10,240.41) --
	(147.99,240.32) --
	(147.95,240.29) --
	(147.91,240.26) --
	(147.83,240.21) --
	(147.74,240.15) --
	(147.68,240.12) --
	(147.66,240.10) --
	(147.64,240.08) --
	(147.64,240.04) --
	(147.65,240.02) --
	(147.66,240.00) --
	(147.67,239.99) --
	(147.72,239.97) --
	(147.78,239.95) --
	(147.80,239.94) --
	(147.90,239.91) --
	(148.11,239.88) --
	(148.17,239.88) --
	(148.22,239.87) --
	(148.33,239.88) --
	(148.42,239.89) --
	(148.55,239.91) --
	(148.82,239.94) --
	(148.98,239.97) --
	(149.37,240.05) --
	(149.51,240.07) --
	(149.61,240.09) --
	(149.69,240.10) --
	(149.81,240.10) --
	(149.86,240.09) --
	(150.09,240.05) --
	(150.12,240.04) --
	(150.21,240.01) --
	(150.24,240.00) --
	(150.29,239.98) --
	(150.31,239.94) --
	(150.35,239.91) --
	(150.41,239.87) --
	(150.43,239.86) --
	(150.51,239.84) --
	(150.57,239.83) --
	(150.78,239.86) --
	(150.84,239.86) --
	(150.95,239.85) --
	(151.00,239.84) --
	(151.08,239.82) --
	(151.20,239.78) --
	(151.34,239.72) --
	(151.39,239.69) --
	(151.41,239.69) --
	(151.43,239.68) --
	(151.56,239.64) --
	(151.69,239.61) --
	(151.79,239.59) --
	(151.96,239.57) --
	(152.10,239.57) --
	(152.12,239.57) --
	(152.15,239.57) --
	(152.25,239.59) --
	(152.31,239.61) --
	(152.59,239.67) --
	(152.70,239.68) --
	(152.72,239.68) --
	(152.73,239.68) --
	(152.78,239.69) --
	(152.84,239.69) --
	(152.85,239.68) --
	(152.88,239.68) --
	(152.92,239.69) --
	(153.01,239.67) --
	(153.09,239.66) --
	(153.14,239.65) --
	(153.16,239.65) --
	(153.19,239.64) --
	(153.24,239.62) --
	(153.29,239.61) --
	(153.32,239.60) --
	(153.39,239.58) --
	(153.47,239.55) --
	(153.59,239.51) --
	(153.69,239.48) --
	(153.79,239.44) --
	(153.86,239.41) --
	(153.93,239.38) --
	(153.94,239.36) --
	(153.98,239.33) --
	(154.02,239.30) --
	(154.03,239.26) --
	(154.09,239.19) --
	(154.12,239.11) --
	(154.14,239.03) --
	(154.14,238.99) --
	(154.12,238.91) --
	(154.09,238.86) --
	(154.04,238.76) --
	(154.04,238.72) --
	(154.02,238.66) --
	(154.03,238.62) --
	(154.03,238.59) --
	(154.03,238.46) --
	(154.04,238.43) --
	(154.03,238.39) --
	(154.03,238.36) --
	(153.99,238.24) --
	(153.96,238.15) --
	(153.95,237.99) --
	(153.95,237.97) --
	(153.96,237.93) --
	(153.98,237.83) --
	(154.01,237.80) --
	(154.07,237.75) --
	(154.24,237.69) --
	(154.39,237.63) --
	(154.54,237.56) --
	(154.59,237.54) --
	(154.65,237.50) --
	(154.83,237.40) --
	(154.96,237.33) --
	(155.03,237.30) --
	(155.10,237.27) --
	(155.15,237.25) --
	(155.18,237.23) --
	(155.21,237.22) --
	(155.23,237.22) --
	(155.25,237.20) --
	(155.33,237.16) --
	(155.34,237.16) --
	(155.39,237.17) --
	(155.47,237.18) --
	(155.69,237.23) --
	(155.71,237.24) --
	(155.74,237.24) --
	(155.79,237.25) --
	(155.84,237.25) --
	(155.87,237.25) --
	(155.94,237.27) --
	(156.02,237.29) --
	(156.04,237.30) --
	(156.15,237.33) --
	(156.33,237.35) --
	(156.41,237.37) --
	(156.50,237.38) --
	(156.63,237.39) --
	(156.68,237.40) --
	(156.73,237.42) --
	(156.83,237.44) --
	(156.88,237.46) --
	(156.94,237.48) --
	(157.11,237.53) --
	(157.15,237.56) --
	(157.18,237.56) --
	(157.23,237.58) --
	(157.35,237.62) --
	(157.59,237.67) --
	(157.62,237.68) --
	(157.69,237.69) --
	(157.78,237.70) --
	(157.81,237.70) --
	(157.94,237.72) --
	(158.15,237.77) --
	(158.26,237.79) --
	(158.31,237.80) --
	(158.34,237.80) --
	(158.39,237.82) --
	(158.62,237.86) --
	(158.81,237.90) --
	(158.89,237.91) --
	(158.95,237.90) --
	(159.03,237.90) --
	(159.08,237.89) --
	(159.16,237.89) --
	(159.22,237.89) --
	(159.33,237.90) --
	(159.35,237.90) --
	(159.43,237.93) --
	(159.52,238.00) --
	(159.54,238.02) --
	(159.58,238.04) --
	(159.62,238.08) --
	(159.64,238.09) --
	(159.69,238.14) --
	(159.72,238.17) --
	(159.74,238.21) --
	(159.74,238.27) --
	(159.71,238.31) --
	(159.69,238.35) --
	(159.68,238.37) --
	(159.57,238.46) --
	(159.56,238.48) --
	(159.55,238.49) --
	(159.54,238.50) --
	(159.48,238.57) --
	(159.47,238.59) --
	(159.44,238.62) --
	(159.37,238.71) --
	(159.35,238.76) --
	(159.36,238.78) --
	(159.37,238.85) --
	(159.40,238.88) --
	(159.51,238.94) --
	(159.58,238.97) --
	(159.71,239.01) --
	(159.85,239.03) --
	(160.03,239.04) --
	(160.20,239.05) --
	(160.45,239.05) --
	(160.53,239.05) --
	(160.55,239.05) --
	(160.66,239.05) --
	(160.80,239.06) --
	(160.99,239.05) --
	(161.12,239.03) --
	(161.20,239.02) --
	(161.26,239.02) --
	(161.34,239.03) --
	(161.45,239.02) --
	(161.66,238.99) --
	(161.77,238.97) --
	(161.82,238.96) --
	(162.05,238.89) --
	(162.10,238.87) --
	(162.18,238.82) --
	(162.29,238.73) --
	(162.51,238.55) --
	(162.54,238.54) --
	(162.56,238.53) --
	(162.56,238.53) --
	(162.56,238.54) --
	(162.56,238.54) --
	(162.56,238.55) --
	(162.58,238.53) --
	(162.62,238.48) --
	(162.75,238.38) --
	(162.78,238.34) --
	(162.87,238.26) --
	(163.09,238.08) --
	(163.14,238.01) --
	(163.16,237.97) --
	(163.18,237.96) --
	(163.31,237.76) --
	(163.40,237.66) --
	(163.48,237.60) --
	(163.72,237.35) --
	(163.75,237.32) --
	(163.87,237.18) --
	(163.99,237.07) --
	(164.03,237.02) --
	(164.10,236.94) --
	(164.12,236.92) --
	(164.15,236.89) --
	(164.25,236.79) --
	(164.38,236.69) --
	(164.47,236.64) --
	(164.57,236.57) --
	(164.59,236.56) --
	(164.64,236.48) --
	(164.67,236.45) --
	(164.78,236.36) --
	(164.85,236.32) --
	(164.92,236.29) --
	(165.16,236.21) --
	(165.19,236.20) --
	(165.27,236.15) --
	(165.33,236.11) --
	(165.41,236.09) --
	(165.46,236.07) --
	(165.51,236.07) --
	(165.60,236.07) --
	(165.68,236.07) --
	(165.84,236.07) --
	(165.95,236.06) --
	(166.16,236.06) --
	(166.22,236.05) --
	(166.35,236.05) --
	(166.41,236.06) --
	(166.44,236.06) --
	(166.46,236.06) --
	(166.62,236.08) --
	(166.73,236.10) --
	(166.83,236.12) --
	(166.93,236.16) --
	(166.98,236.17) --
	(167.17,236.25) --
	(167.25,236.30) --
	(167.29,236.33) --
	(167.37,236.42) --
	(167.42,236.47) --
	(167.44,236.51) --
	(167.45,236.52) --
	(167.47,236.54) --
	(167.51,236.59) --
	(167.59,236.67) --
	(167.63,236.70) --
	(167.64,236.72) --
	(167.67,236.73) --
	(167.78,236.82) --
	(167.81,236.83) --
	(167.82,236.85) --
	(167.85,236.86) --
	(167.90,236.88) --
	(167.92,236.89) --
	(168.01,236.94) --
	(168.05,236.97) --
	(168.12,237.00) --
	(168.16,237.02) --
	(168.19,237.03) --
	(168.24,237.05) --
	(168.40,237.16) --
	(168.58,237.25) --
	(168.71,237.29) --
	(168.78,237.31) --
	(168.96,237.35) --
	(168.99,237.36) --
	(169.02,237.36) --
	(169.04,237.37) --
	(169.07,237.38) --
	(169.10,237.39) --
	(169.23,237.42) --
	(169.25,237.42) --
	(169.28,237.43) --
	(169.33,237.44) --
	(169.35,237.43) --
	(169.40,237.34) --
	(169.44,237.26) --
	(169.44,237.24) --
	(169.48,237.14) --
	(169.51,237.11) --
	(169.52,237.08) --
	(169.54,237.07) --
	(169.57,237.04) --
	(169.61,237.01) --
	(169.66,236.99) --
	(169.70,236.97) --
	(169.76,236.96) --
	(169.98,236.94) --
	(170.03,236.93) --
	(170.11,236.91) --
	(170.15,236.90) --
	(170.37,236.81) --
	(170.52,236.77) --
	(170.57,236.75) --
	(170.66,236.70) --
	(170.69,236.67) --
	(170.71,236.63) --
	(170.74,236.53) --
	(170.75,236.51) --
	(170.76,236.47) --
	(170.77,236.46) --
	(170.79,236.45) --
	(170.80,236.43) --
	(170.85,236.41) --
	(170.87,236.40) --
	(170.95,236.39) --
	(171.04,236.40) --
	(171.19,236.45) --
	(171.24,236.47) --
	(171.32,236.52) --
	(171.37,236.54) --
	(171.47,236.58) --
	(171.52,236.59) --
	(171.57,236.59) --
	(171.61,236.58) --
	(171.63,236.57) --
	(171.66,236.54) --
	(171.69,236.42) --
	(171.69,236.35) --
	(171.70,236.34) --
	(171.74,236.33) --
	(171.82,236.32) --
	(171.85,236.33) --
	(171.90,236.35) --
	(172.14,236.50) --
	(172.19,236.53) --
	(172.25,236.57) --
	(172.29,236.59) --
	(172.40,236.65) --
	(172.45,236.67) --
	(172.53,236.70) --
	(172.58,236.70) --
	(172.64,236.70) --
	(172.66,236.70) --
	(172.71,236.68) --
	(172.74,236.67) --
	(172.77,236.64) --
	(172.79,236.50) --
	(172.84,236.38) --
	(172.85,236.36) --
	(172.88,236.33) --
	(172.88,236.30) --
	(172.93,236.23) --
	(172.94,236.19) --
	(173.06,236.08) --
	(173.08,236.07) --
	(173.11,236.08) --
	(173.13,236.10) --
	(173.16,236.13) --
	(173.22,236.17) --
	(173.25,236.19) --
	(173.30,236.24) --
	(173.34,236.26) --
	(173.39,236.28) --
	(173.44,236.30) --
	(173.56,236.33) --
	(173.69,236.40) --
	(173.73,236.42) --
	(173.77,236.45) --
	(173.84,236.52) --
	(173.87,236.57) --
	(173.88,236.59) --
	(173.93,236.70) --
	(173.97,236.78) --
	(174.02,236.86) --
	(174.18,237.00) --
	(174.24,237.04) --
	(174.27,237.05) --
	(174.29,237.06) --
	(174.31,237.07) --
	(174.34,237.08) --
	(174.41,237.10) --
	(174.46,237.11) --
	(174.55,237.13) --
	(174.60,237.13) --
	(174.62,237.13) --
	(174.65,237.11) --
	(174.66,237.09) --
	(174.71,236.98) --
	(174.72,236.96) --
	(174.73,236.94) --
	(174.80,236.85) --
	(174.82,236.81) --
	(174.85,236.78) --
	(174.90,236.73) --
	(174.94,236.70) --
	(175.03,236.63) --
	(175.07,236.55) --
	(175.09,236.54) --
	(175.11,236.52) --
	(175.16,236.51) --
	(175.19,236.51) --
	(175.24,236.50) --
	(175.35,236.45) --
	(175.36,236.45) --
	(175.40,236.45) --
	(175.44,236.47) --
	(175.49,236.50) --
	(175.63,236.59) --
	(175.73,236.66) --
	(175.76,236.69) --
	(175.79,236.75) --
	(175.82,236.79) --
	(175.90,236.85) --
	(175.94,236.87) --
	(175.97,236.91) --
	(176.00,236.92) --
	(176.02,236.93) --
	(176.09,236.96) --
	(176.20,236.99) --
	(176.31,237.00) --
	(176.39,237.01) --
	(176.43,237.03) --
	(176.49,237.03) --
	(176.57,237.04) --
	(176.57,237.04) --
	(176.60,237.04) --
	(176.66,237.04) --
	(176.68,237.04) --
	(176.76,237.05) --
	(176.84,237.07) --
	(176.93,237.07) --
	(176.98,237.07) --
	(177.01,237.06) --
	(177.04,237.06) --
	(177.06,237.05) --
	(177.13,237.03) --
	(177.15,237.01) --
	(177.16,236.99) --
	(177.17,236.97) --
	(177.19,236.95) --
	(177.21,236.94) --
	(177.28,236.91) --
	(177.31,236.91) --
	(177.47,236.87) --
	(177.59,236.83) --
	(177.75,236.81) --
	(177.78,236.81) --
	(177.83,236.80) --
	(178.00,236.81) --
	(178.13,236.79) --
	(178.13,236.79) --
	(178.23,236.77) --
	(178.32,236.76) --
	(178.37,236.76) --
	(178.48,236.75) --
	(178.50,236.75) --
	(178.61,236.74) --
	(178.67,236.73) --
	(178.71,236.72) --
	(178.78,236.68) --
	(178.80,236.67) --
	(178.95,236.61) --
	(179.00,236.60) --
	(179.24,236.50) --
	(179.27,236.47) --
	(179.32,236.44) --
	(179.42,236.35) --
	(179.43,236.33) --
	(179.51,236.20) --
	(179.53,236.17) --
	(179.53,236.16) --
	(179.53,236.11) --
	(179.52,236.07) --
	(179.46,235.98) --
	(179.44,235.94) --
	(179.38,235.85) --
	(179.31,235.79) --
	(179.24,235.73) --
	(179.20,235.67) --
	(179.19,235.65) --
	(179.18,235.61) --
	(179.17,235.55) --
	(179.15,235.49) --
	(179.15,235.47) --
	(179.15,235.45) --
	(179.14,235.43) --
	(179.14,235.39) --
	(179.16,235.35) --
	(179.17,235.31) --
	(179.18,235.29) --
	(179.22,235.23) --
	(179.24,235.20) --
	(179.28,235.17) --
	(179.42,235.07) --
	(179.44,235.06) --
	(179.57,235.03) --
	(179.62,235.02) --
	(179.65,235.01) --
	(179.67,235.00) --
	(179.78,234.94) --
	(179.85,234.91) --
	(179.89,234.88) --
	(179.90,234.86) --
	(179.92,234.82) --
	(179.93,234.78) --
	(179.94,234.76) --
	(179.94,234.74) --
	(179.93,234.70) --
	(179.91,234.66) --
	(179.89,234.65) --
	(179.87,234.63) --
	(179.80,234.61) --
	(179.70,234.58) --
	(179.62,234.56) --
	(179.59,234.55) --
	(179.50,234.51) --
	(179.45,234.50) --
	(179.40,234.48) --
	(179.37,234.47) --
	(179.22,234.42) --
	(179.14,234.39) --
	(179.08,234.35) --
	(179.04,234.32) --
	(179.02,234.29) --
	(178.98,234.26) --
	(178.97,234.24) --
	(178.94,234.20) --
	(178.91,234.19) --
	(178.81,234.18) --
	(178.73,234.17) --
	(178.71,234.16) --
	(178.68,234.13) --
	(178.67,234.09) --
	(178.66,234.08) --
	(178.67,234.06) --
	(178.67,234.01) --
	(178.68,234.00) --
	(178.70,233.98) --
	(178.75,233.96) --
	(178.80,233.91) --
	(178.81,233.89) --
	(178.81,233.83) --
	(178.82,233.77) --
	(178.85,233.58) --
	(178.88,233.42) --
	(178.87,233.40) --
	(178.87,233.38) --
	(178.88,233.34) --
	(178.89,233.32) --
	(178.91,233.30) --
	(179.08,233.20) --
	(179.22,233.14) --
	(179.33,233.08) --
	(179.38,233.06) --
	(179.43,233.06) --
	(179.54,233.06) --
	(179.60,233.05) --
	(179.62,233.04) --
	(179.67,233.02) --
	(179.71,232.96) --
	(179.80,232.89) --
	(179.86,232.85) --
	(179.89,232.84) --
	(179.91,232.82) --
	(179.95,232.80) --
	(179.98,232.79) --
	(180.08,232.70) --
	(180.13,232.67) --
	(180.18,232.66) --
	(180.24,232.66) --
	(180.39,232.70) --
	(180.52,232.74) --
	(180.59,232.77) --
	(180.61,232.78) --
	(180.63,232.82) --
	(180.63,232.84) --
	(180.64,232.88) --
	(180.65,232.94) --
	(180.66,232.96) --
	(180.68,232.98) --
	(180.70,232.99) --
	(180.93,233.06) --
	(180.95,233.07) --
	(180.98,233.08) --
	(181.02,233.11) --
	(181.05,233.14) --
	(181.11,233.18) --
	(181.15,233.21) --
	(181.17,233.22) --
	(181.25,233.24) --
	(181.28,233.25) --
	(181.34,233.29) --
	(181.37,233.29) --
	(181.42,233.30) --
	(181.50,233.30) --
	(181.53,233.29) --
	(181.60,233.27) --
	(181.64,233.24) --
	(181.73,233.16) --
	(181.80,233.10) --
	(181.85,233.03) --
	(181.90,233.00) --
	(181.98,232.95) --
	(182.01,232.94) --
	(182.03,232.94) --
	(182.06,232.93) --
	(182.08,232.91) --
	(182.09,232.90) --
	(182.08,232.88) --
	(182.08,232.87) --
	(182.08,232.85) --
	(182.07,232.83) --
	(182.06,232.81) --
	(182.06,232.77) --
	(182.06,232.71) --
	(182.07,232.69) --
	(182.11,232.66) --
	(182.16,232.64) --
	(182.20,232.62) --
	(182.23,232.61) --
	(182.27,232.59) --
	(182.43,232.50) --
	(182.49,232.47) --
	(182.51,232.47) --
	(182.53,232.45) --
	(182.56,232.45) --
	(182.56,232.44) --
	(182.62,232.43) --
	(182.67,232.42) --
	(182.78,232.40) --
	(182.83,232.41) --
	(182.88,232.41) --
	(182.91,232.41) --
	(182.97,232.40) --
	(183.02,232.39) --
	(183.04,232.38) --
	(183.13,232.34) --
	(183.15,232.32) --
	(183.18,232.29) --
	(183.19,232.27) --
	(183.19,232.22) --
	(183.16,232.02) --
	(183.16,231.88) --
	(183.16,231.79) --
	(183.17,231.76) --
	(183.17,231.67) --
	(183.19,231.66) --
	(183.42,231.63) --
	(183.47,231.62) --
	(183.52,231.61) --
	(183.56,231.60) --
	(183.59,231.60) --
	(183.61,231.60) --
	(183.70,231.58) --
	(183.75,231.57) --
	(183.78,231.57) --
	(183.80,231.56) --
	(183.86,231.55) --
	(183.88,231.54) --
	(183.90,231.52) --
	(183.91,231.50) --
	(183.92,231.46) --
	(183.92,231.42) --
	(183.87,231.33) --
	(183.84,231.29) --
	(183.82,231.28) --
	(183.71,231.16) --
	(183.67,231.11) --
	(183.64,231.05) --
	(183.59,231.00) --
	(183.43,230.86) --
	(183.39,230.83) --
	(183.30,230.79) --
	(183.13,230.73) --
	(183.12,230.72) --
	(183.19,230.70) --
	(183.26,230.69) --
	(183.34,230.67) --
	(183.48,230.60) --
	(183.62,230.52) --
	(183.66,230.48) --
	(183.69,230.45) --
	(183.70,230.42) --
	(183.71,230.41) --
	(183.73,230.37) --
	(183.75,230.33) --
	(183.78,230.30) --
	(183.82,230.27) --
	(183.85,230.24) --
	(183.90,230.21) --
	(183.96,230.14) --
	(183.98,230.11) --
	(184.06,229.98) --
	(184.09,229.95) --
	(184.12,229.91) --
	(184.16,229.88) --
	(184.21,229.86) --
	(184.23,229.85) --
	(184.34,229.79) --
	(184.36,229.77) --
	(184.39,229.74) --
	(184.42,229.68) --
	(184.44,229.64) --
	(184.43,229.62) --
	(184.45,229.58) --
	(184.45,229.56) --
	(184.43,229.46) --
	(184.42,229.42) --
	(184.37,229.37) --
	(184.29,229.29) --
	(184.19,229.20) --
	(184.13,229.10) --
	(184.10,229.07) --
	(184.08,229.05) --
	(184.08,229.04) --
	(184.09,229.02) --
	(184.12,229.01) --
	(184.14,229.00) --
	(184.19,228.98) --
	(184.21,228.97) --
	(184.27,228.98) --
	(184.38,228.99) --
	(184.45,228.99) --
	(184.51,228.98) --
	(184.53,228.98) --
	(184.60,228.94) --
	(184.64,228.91) --
	(184.65,228.89) --
	(184.67,228.85) --
	(184.76,228.70) --
	(184.78,228.67) --
	(184.79,228.65) --
	(184.80,228.63) --
	(184.84,228.58) --
	(184.91,228.49) --
	(184.98,228.43) --
	(185.02,228.40) --
	(185.06,228.35) --
	(185.09,228.34) --
	(185.12,228.34) --
	(185.15,228.34) --
	(185.20,228.36) --
	(185.25,228.37) --
	(185.35,228.40) --
	(185.43,228.41) --
	(185.51,228.40) --
	(185.56,228.39) --
	(185.63,228.35) --
	(185.68,228.31) --
	(185.78,228.24) --
	(185.87,228.19) --
	(185.91,228.16) --
	(186.06,228.08) --
	(186.13,228.05) --
	(186.18,228.03) --
	(186.22,228.00) --
	(186.24,227.99) --
	(186.32,227.98) --
	(186.40,228.00) --
	(186.45,228.02) --
	(186.50,228.04) --
	(186.52,228.05) --
	(186.68,227.97) --
	(186.76,227.92) --
	(186.82,227.90) --
	(186.84,227.89) --
	(186.94,227.86) --
	(186.96,227.85) --
	(187.05,227.77) --
	(187.07,227.76) --
	(187.09,227.75) --
	(187.12,227.74) --
	(187.15,227.74) --
	(187.18,227.74) --
	(187.29,227.73) --
	(187.31,227.73) --
	(187.33,227.71) --
	(187.38,227.67) --
	(187.44,227.63) --
	(187.54,227.59) --
	(187.56,227.58) --
	(187.61,227.57) --
	(187.66,227.54) --
	(187.68,227.53) --
	(187.70,227.47) --
	(187.70,227.45) --
	(187.69,227.35) --
	(187.70,227.33) --
	(187.73,227.27) --
	(187.75,227.23) --
	(187.77,227.19) --
	(187.79,227.15) --
	(187.85,227.09) --
	(187.88,227.07) --
	(187.88,227.06) --
	(187.87,227.04) --
	(187.81,226.99) --
	(187.77,226.96) --
	(187.73,226.93) --
	(187.71,226.92) --
	(187.70,226.90) --
	(187.68,226.89) --
	(187.67,226.87) --
	(187.65,226.85) --
	(187.63,226.84) --
	(187.60,226.83) --
	(187.54,226.82) --
	(187.47,226.80) --
	(187.43,226.77) --
	(187.42,226.75) --
	(187.42,226.73) --
	(187.40,226.67) --
	(187.39,226.65) --
	(187.36,226.64) --
	(187.33,226.60) --
	(187.28,226.58) --
	(187.27,226.57) --
	(187.21,226.52) --
	(187.19,226.51) --
	(187.14,226.49) --
	(187.12,226.47) --
	(187.08,226.42) --
	(187.01,226.28) --
	(187.00,226.27) --
	(186.99,226.25) --
	(186.99,226.23) --
	(186.99,226.21) --
	(187.03,226.21) --
	(187.14,226.20) --
	(187.17,226.20) --
	(187.20,226.20) --
	(187.25,226.20) --
	(187.42,226.21) --
	(187.49,226.21) --
	(187.55,226.22) --
	(187.60,226.23) --
	(187.76,226.27) --
	(187.78,226.28) --
	(187.80,226.29) --
	(187.89,226.33) --
	(187.96,226.36) --
	(188.00,226.40) --
	(188.04,226.42) --
	(188.06,226.42) --
	(188.07,226.42) --
	(188.09,226.38) --
	(188.10,226.36) --
	(188.10,226.32) --
	(188.10,226.28) --
	(188.10,226.26) --
	(188.11,226.20) --
	(188.13,226.14) --
	(188.16,226.08) --
	(188.20,226.05) --
	(188.22,226.03) --
	(188.30,225.90) --
	(188.33,225.87) --
	(188.36,225.84) --
	(188.39,225.83) --
	(188.47,225.78) --
	(188.58,225.75) --
	(188.71,225.73) --
	(188.99,225.67) --
	(189.11,225.63) --
	(189.14,225.62) --
	(189.16,225.61) --
	(189.30,225.61) --
	(189.33,225.60) --
	(189.37,225.58) --
	(189.47,225.51) --
	(189.50,225.51) --
	(189.65,225.48) --
	(189.66,225.49) --
	(189.76,225.45) --
	(189.81,225.43) --
	(189.88,225.37) --
	(189.88,225.36) --
	(189.94,225.31) --
	(190.01,225.27) --
	(190.03,225.25) --
	(190.03,225.23) --
	(190.03,225.21) --
	(190.00,225.17) --
	(189.94,225.13) --
	(189.73,224.97) --
	(189.71,224.95) --
	(189.69,224.94) --
	(189.63,224.93) --
	(189.55,224.93) --
	(189.52,224.92) --
	(189.50,224.92) --
	(189.47,224.92) --
	(189.45,224.91) --
	(189.42,224.92) --
	(189.39,224.91) --
	(189.29,224.88) --
	(189.27,224.86) --
	(189.25,224.85) --
	(189.17,224.65) --
	(189.16,224.63) --
	(189.07,224.43) --
	(189.05,224.42) --
	(189.05,224.40) --
	(189.04,224.38) --
	(188.95,224.25) --
	(188.92,224.22) --
	(188.80,224.14) --
	(188.76,224.11) --
	(188.75,224.09) --
	(188.74,224.05) --
	(188.76,223.99) --
	(188.81,223.90) --
	(188.86,223.85) --
	(188.89,223.84) --
	(188.99,223.82) --
	(189.06,223.79) --
	(189.12,223.75) --
	(189.15,223.74) --
	(189.18,223.73) --
	(189.23,223.73) --
	(189.31,223.75) --
	(189.36,223.75) --
	(189.50,223.75) --
	(189.55,223.76) --
	(189.58,223.77) --
	(189.63,223.78) --
	(189.73,223.81) --
	(189.78,223.82) --
	(189.84,223.82) --
	(189.91,223.79) --
	(189.94,223.78) --
	(190.01,223.69) --
	(190.02,223.68) --
	(190.18,223.60) --
	(190.22,223.57) --
	(190.26,223.55) --
	(190.29,223.54) --
	(190.34,223.53) --
	(190.37,223.54) --
	(190.42,223.54) --
	(190.45,223.53) --
	(190.47,223.52) --
	(190.50,223.51) --
	(190.51,223.51) --
	(190.52,223.50) --
	(190.51,223.47) --
	(190.48,223.42) --
	(190.48,223.42) --
	(190.48,223.42) --
	(190.49,223.42) --
	(190.49,223.42) --
	(190.50,223.42) --
	(190.50,223.42) --
	(190.51,223.42) --
	(190.51,223.42) --
	(190.52,223.42) --
	(190.52,223.42) --
	(190.53,223.42) --
	(190.53,223.42) --
	(190.54,223.42) --
	(190.54,223.42) --
	(190.55,223.42) --
	(190.55,223.42) --
	(190.56,223.42) --
	(190.56,223.42) --
	(190.57,223.42) --
	(190.57,223.42) --
	(190.58,223.42) --
	(190.58,223.42) --
	(190.59,223.42) --
	(190.59,223.42) --
	(190.60,223.42) --
	(190.60,223.42) --
	(190.61,223.42) --
	(190.61,223.42) --
	(190.62,223.42) --
	(190.62,223.42) --
	(190.63,223.42) --
	(190.63,223.42) --
	(190.64,223.42) --
	(190.64,223.42) --
	(190.65,223.42) --
	(190.65,223.42) --
	(190.65,223.41) --
	(190.69,223.38) --
	(190.71,223.37) --
	(190.83,223.29) --
	(190.85,223.27) --
	(190.85,223.24) --
	(190.73,223.11) --
	(190.71,223.07) --
	(190.70,223.04) --
	(190.69,223.02) --
	(190.67,222.97) --
	(190.66,222.96) --
	(190.63,222.96) --
	(190.61,222.96) --
	(190.57,222.98) --
	(190.54,222.99) --
	(190.51,222.99) --
	(190.45,222.97) --
	(190.43,222.95) --
	(190.40,222.90) --
	(190.40,222.88) --
	(190.39,222.80) --
	(190.39,222.75) --
	(190.43,222.61) --
	(190.43,222.51) --
	(190.45,222.39) --
	(190.47,222.35) --
	(190.48,222.33) --
	(190.52,222.29) --
	(190.55,222.27) --
	(190.59,222.23) --
	(190.60,222.21) --
	(190.61,222.18) --
	(190.61,222.08) --
	(190.61,221.99) --
	(190.63,221.94) --
	(190.65,221.91) --
	(190.69,221.91) --
	(190.72,221.91) --
	(190.75,221.90) --
	(190.77,221.88) --
	(190.80,221.87) --
	(190.85,221.84) --
	(190.88,221.81) --
	(190.91,221.77) --
	(190.91,221.76) --
	(190.92,221.69) --
	(190.93,221.66) --
	(190.94,221.66) --
	(190.95,221.66) --
	(190.97,221.65) --
	(191.03,221.63) --
	(191.07,221.63) --
	(191.14,221.65) --
	(191.18,221.66) --
	(191.21,221.66) --
	(191.25,221.65) --
	(191.28,221.63) --
	(191.29,221.63) --
	(191.33,221.62) --
	(191.36,221.63) --
	(191.42,221.65) --
	(191.44,221.67) --
	(191.47,221.69) --
	(191.51,221.76) --
	(191.53,221.77) --
	(191.54,221.78) --
	(191.55,221.75) --
	(191.57,221.73) --
	(191.60,221.69) --
	(191.61,221.67) --
	(191.62,221.65) --
	(191.63,221.62) --
	(191.67,221.53) --
	(191.68,221.50) --
	(191.68,221.48) --
	(191.70,221.41) --
	(191.74,221.37) --
	(191.77,221.33) --
	(191.80,221.31) --
	(191.87,221.26) --
	(191.89,221.24) --
	(191.92,221.17) --
	(191.93,221.12) --
	(191.93,221.07) --
	(191.92,221.05) --
	(191.89,221.00) --
	(191.83,220.97) --
	(191.81,220.94) --
	(191.81,220.94) --
	(191.81,220.92) --
	(191.83,220.91) --
	(191.86,220.90) --
	(191.92,220.88) --
	(191.95,220.88) --
	(191.98,220.87) --
	(192.03,220.87) --
	(192.10,220.87) --
	(192.17,220.88) --
	(192.20,220.89) --
	(192.26,220.90) --
	(192.29,220.92) --
	(192.39,220.93) --
	(192.41,220.94) --
	(192.44,220.95) --
	(192.46,220.98) --
	(192.48,220.99) --
	(192.50,221.02) --
	(192.56,221.10) --
	(192.59,221.12) --
	(192.66,221.12) --
	(192.69,221.12) --
	(192.75,221.09) --
	(192.92,221.02) --
	(192.93,221.00) --
	(192.95,220.98) --
	(193.01,220.92) --
	(193.03,220.89) --
	(193.05,220.88) --
	(193.10,220.85) --
	(193.16,220.83) --
	(193.23,220.82) --
	(193.36,220.81) --
	(193.42,220.81) --
	(193.48,220.78) --
	(193.51,220.78) --
	(193.56,220.76) --
	(193.56,220.74) --
	(193.56,220.71) --
	(193.55,220.66) --
	(193.53,220.64) --
	(193.51,220.57) --
	(193.52,220.54) --
	(193.54,220.54) --
	(193.55,220.50) --
	(193.58,220.40) --
	(193.58,220.39) --
	(193.58,220.36) --
	(193.60,220.35) --
	(193.61,220.32) --
	(193.64,220.31) --
	(193.68,220.30) --
	(193.71,220.30) --
	(193.74,220.31) --
	(193.78,220.31) --
	(193.80,220.31) --
	(193.91,220.25) --
	(193.94,220.24) --
	(193.98,220.23) --
	(194.03,220.25) --
	(194.06,220.25) --
	(194.05,220.25) --
	(194.10,220.26) --
	(194.16,220.29) --
	(194.19,220.30) --
	(194.29,220.30) --
	(194.32,220.31) --
	(194.35,220.31) --
	(194.39,220.32) --
	(194.41,220.34) --
	(194.43,220.36) --
	(194.44,220.38) --
	(194.55,220.47) --
	(194.57,220.49) --
	(194.58,220.54) --
	(194.60,220.56) --
	(194.73,220.67) --
	(194.76,220.68) --
	(194.80,220.69) --
	(194.83,220.70) --
	(194.89,220.68) --
	(194.92,220.67) --
	(194.94,220.65) --
	(194.95,220.63) --
	(194.95,220.58) --
	(194.95,220.55) --
	(194.95,220.50) --
	(195.02,220.41) --
	(195.06,220.28) --
	(195.09,220.21) --
	(195.11,220.19) --
	(195.19,220.15) --
	(195.38,220.10) --
	(195.41,220.06) --
	(195.43,220.00) --
	(195.49,219.92) --
	(195.68,219.77) --
	(195.72,219.74) --
	(195.73,219.71) --
	(195.74,219.69) --
	(195.72,219.67) --
	(195.69,219.62) --
	(195.59,219.55) --
	(195.58,219.53) --
	(195.55,219.46) --
	(195.54,219.41) --
	(195.52,219.36) --
	(195.51,219.34) --
	(195.50,219.32) --
	(195.48,219.27) --
	(195.48,219.24) --
	(195.49,219.23) --
	(195.52,219.20) --
	(195.68,219.10) --
	(195.77,219.09) --
	(195.80,219.08) --
	(195.85,219.05) --
	(195.88,219.03) --
	(195.94,219.02) --
	(196.01,218.99) --
	(196.02,218.97) --
	(196.05,218.94) --
	(196.05,218.92) --
	(196.10,218.82) --
	(196.13,218.81) --
	(196.16,218.77) --
	(196.16,218.75) --
	(196.20,218.66) --
	(196.21,218.65) --
	(196.21,218.63) --
	(196.20,218.58) --
	(196.18,218.55) --
	(196.18,218.51) --
	(196.18,218.49) --
	(196.18,218.47) --
	(196.20,218.34) --
	(196.20,218.32) --
	(196.19,218.30) --
	(196.15,218.23) --
	(196.09,218.20) --
	(196.07,218.18) --
	(196.05,218.16) --
	(196.03,218.14) --
	(196.00,218.13) --
	(195.97,218.12) --
	(195.93,218.08) --
	(195.88,218.05) --
	(195.87,218.02) --
	(195.85,217.95) --
	(195.84,217.92) --
	(195.83,217.90) --
	(195.80,217.89) --
	(195.71,217.86) --
	(195.68,217.85) --
	(195.67,217.84) --
	(195.65,217.82) --
	(195.62,217.80) --
	(195.60,217.79) --
	(195.56,217.75) --
	(195.55,217.72) --
	(195.51,217.66) --
	(195.49,217.64) --
	(195.42,217.60) --
	(195.42,217.59) --
	(195.42,217.59) --
	(195.43,217.57) --
	(195.43,217.56) --
	(195.44,217.53) --
	(195.47,217.48) --
	(195.48,217.46) --
	(195.51,217.41) --
	(195.52,217.39) --
	(195.51,217.35) --
	(195.48,217.32) --
	(195.46,217.30) --
	(195.44,217.25) --
	(195.43,217.18) --
	(195.41,217.08) --
	(195.40,217.05) --
	(195.33,217.00) --
	(195.30,216.95) --
	(195.27,216.88) --
	(195.26,216.81) --
	(195.27,216.79) --
	(195.30,216.74) --
	(195.30,216.73) --
	(195.29,216.70) --
	(195.27,216.68) --
	(195.25,216.66) --
	(195.25,216.64) --
	(195.25,216.61) --
	(195.27,216.54) --
	(195.27,216.49) --
	(195.26,216.44) --
	(195.25,216.42) --
	(195.23,216.40) --
	(195.20,216.38) --
	(195.18,216.37) --
	(195.15,216.36) --
	(195.14,216.33) --
	(195.13,216.30) --
	(195.15,216.28) --
	(195.17,216.26) --
	(195.19,216.25) --
	(195.22,216.23) --
	(195.30,216.16) --
	(195.31,216.14) --
	(195.32,216.12) --
	(195.33,216.10) --
	(195.34,216.08) --
	(195.35,216.05) --
	(195.36,216.03) --
	(195.39,215.93) --
	(195.39,215.93) --
	(195.39,215.90) --
	(195.39,215.85) --
	(195.37,215.79) --
	(195.38,215.79) --
	(195.37,215.77) --
	(195.36,215.75) --
	(195.35,215.74) --
	(195.32,215.72) --
	(195.25,215.69) --
	(195.24,215.69) --
	(195.20,215.68) --
	(195.17,215.67) --
	(195.09,215.62) --
	(195.07,215.61) --
	(195.00,215.61) --
	(194.98,215.60) --
	(194.96,215.58) --
	(194.95,215.57) --
	(194.95,215.56) --
	(194.95,215.55) --
	(194.96,215.52) --
	(194.99,215.50) --
	(195.03,215.49) --
	(195.05,215.48) --
	(195.06,215.48) --
	(195.06,215.47) --
	(195.05,215.44) --
	(195.01,215.40) --
	(194.97,215.37) --
	(194.96,215.36) --
	(194.95,215.31) --
	(194.96,215.29) --
	(194.97,215.21) --
	(194.97,215.19) --
	(194.97,215.14) --
	(194.96,215.12) --
	(194.93,215.10) --
	(194.91,215.08) --
	(194.89,215.06) --
	(194.87,215.04) --
	(194.85,215.02) --
	(194.85,214.97) --
	(194.86,214.94) --
	(194.88,214.92) --
	(194.90,214.90) --
	(194.97,214.86) --
	(195.03,214.80) --
	(195.06,214.75) --
	(195.06,214.73) --
	(195.06,214.68) --
	(195.05,214.65) --
	(195.01,214.61) --
	(194.98,214.60) --
	(194.97,214.59) --
	(194.96,214.58) --
	(194.96,214.53) --
	(194.96,214.50) --
	(194.95,214.48) --
	(194.93,214.43) --
	(194.91,214.41) --
	(194.84,214.37) --
	(194.74,214.30) --
	(194.72,214.28) --
	(194.66,214.22) --
	(194.60,214.17) --
	(194.58,214.14) --
	(194.57,214.11) --
	(194.55,214.09) --
	(194.55,214.07) --
	(194.54,214.04) --
	(194.54,214.02) --
	(194.54,214.01) --
	(194.52,213.95) --
	(194.52,213.92) --
	(194.53,213.83) --
	(194.54,213.76) --
	(194.56,213.72) --
	(194.57,213.69) --
	(194.59,213.65) --
	(194.63,213.61) --
	(194.72,213.48) --
	(194.73,213.46) --
	(194.75,213.41) --
	(194.75,213.38) --
	(194.71,213.24) --
	(194.69,213.17) --
	(194.70,213.12) --
	(194.69,213.10) --
	(194.68,213.07) --
	(194.65,213.05) --
	(194.63,213.03) --
	(194.63,213.02) --
	(194.63,213.02) --
	(194.64,213.00) --
	(194.70,212.94) --
	(194.71,212.91) --
	(194.77,212.75) --
	(194.77,212.70) --
	(194.76,212.66) --
	(194.75,212.65) --
	(194.75,212.63) --
	(194.73,212.60) --
	(194.72,212.58) --
	(194.73,212.57) --
	(194.72,212.57) --
	(194.73,212.54) --
	(194.79,212.48) --
	(194.81,212.46) --
	(194.83,212.45) --
	(194.97,212.35) --
	(195.01,212.31) --
	(195.03,212.26) --
	(195.03,212.23) --
	(195.03,212.21) --
	(195.04,212.19) --
	(195.06,212.17) --
	(195.09,212.15) --
	(195.19,212.09) --
	(195.25,212.07) --
	(195.28,212.06) --
	(195.33,212.03) --
	(195.39,211.94) --
	(195.40,211.89) --
	(195.39,211.84) --
	(195.33,211.76) --
	(195.30,211.72) --
	(195.29,211.69) --
	(195.31,211.64) --
	(195.35,211.60) --
	(195.37,211.58) --
	(195.46,211.56) --
	(195.52,211.54) --
	(195.54,211.52) --
	(195.56,211.50) --
	(195.58,211.48) --
	(195.60,211.40) --
	(195.61,211.38) --
	(195.68,211.29) --
	(195.69,211.27) --
	(195.71,211.22) --
	(195.72,211.20) --
	(195.73,211.07) --
	(195.76,210.96) --
	(195.77,210.88) --
	(195.77,210.86) --
	(195.76,210.71) --
	(195.74,210.69) --
	(195.72,210.67) --
	(195.67,210.64) --
	(195.61,210.62) --
	(195.59,210.60) --
	(195.56,210.55) --
	(195.56,210.53) --
	(195.55,210.50) --
	(195.54,210.48) --
	(195.51,210.42) --
	(195.51,210.39) --
	(195.54,210.36) --
	(195.57,210.35) --
	(195.64,210.37) --
	(195.70,210.38) --
	(195.73,210.37) --
	(195.76,210.35) --
	(195.77,210.32) --
	(195.78,210.30) --
	(195.77,210.26) --
	(195.76,210.23) --
	(195.74,210.20) --
	(195.67,210.13) --
	(195.66,210.10) --
	(195.69,210.06) --
	(195.73,210.03) --
	(195.80,210.01) --
	(195.83,210.00) --
	(195.85,209.98) --
	(195.85,209.95) --
	(195.86,209.93) --
	(195.86,209.83) --
	(195.83,209.76) --
	(195.82,209.69) --
	(195.82,209.66) --
	(195.83,209.61) --
	(195.89,209.53) --
	(195.93,209.49) --
	(195.96,209.48) --
	(196.07,209.46) --
	(196.08,209.45) --
	(196.14,209.44) --
	(196.27,209.42) --
	(196.43,209.41) --
	(196.48,209.38) --
	(196.50,209.35) --
	(196.50,209.33) --
	(196.50,209.32) --
	(196.48,209.29) --
	(196.42,209.23) --
	(196.41,209.21) --
	(196.41,209.19) --
	(196.40,209.13) --
	(196.41,209.12) --
	(196.42,209.11) --
	(196.54,208.97) --
	(196.56,208.95) --
	(196.56,208.92) --
	(196.57,208.90) --
	(196.58,208.88) --
	(196.60,208.78) --
	(196.64,208.66) --
	(196.64,208.64) --
	(196.72,208.42) --
	(196.75,208.38) --
	(196.77,208.36) --
	(196.79,208.34) --
	(196.81,208.29) --
	(196.82,208.24) --
	(196.82,208.17) --
	(196.83,208.15) --
	(196.84,208.12) --
	(196.89,208.09) --
	(196.95,208.06) --
	(197.04,208.04) --
	(197.07,208.04) --
	(197.14,208.06) --
	(197.17,208.07) --
	(197.19,208.09) --
	(197.22,208.10) --
	(197.26,208.14) --
	(197.28,208.16) --
	(197.31,208.18) --
	(197.38,208.18) --
	(197.41,208.17) --
	(197.45,208.18) --
	(197.46,208.17) --
	(197.50,208.15) --
	(197.53,208.14) --
	(197.56,208.12) --
	(197.65,208.11) --
	(197.81,208.10) --
	(197.98,208.09) --
	(198.06,208.10) --
	(198.10,208.10) --
	(198.14,208.09) --
	(198.16,208.07) --
	(198.17,208.05) --
	(198.19,208.02) --
	(198.21,207.95) --
	(198.23,207.93) --
	(198.26,207.91) --
	(198.28,207.90) --
	(198.31,207.90) --
	(198.34,207.90) --
	(198.37,207.90) --
	(198.41,207.89) --
	(198.44,207.88) --
	(198.54,207.82) --
	(198.61,207.77) --
	(198.64,207.76) --
	(198.70,207.76) --
	(198.74,207.76) --
	(198.77,207.77) --
	(198.79,207.79) --
	(198.83,207.83) --
	(198.89,207.91) --
	(198.90,207.94) --
	(198.93,208.01) --
	(198.93,208.03) --
	(198.94,208.06) --
	(198.91,208.08) --
	(198.91,208.10) --
	(198.90,208.15) --
	(198.91,208.18) --
	(198.92,208.21) --
	(198.93,208.22) --
	(198.95,208.23) --
	(199.07,208.32) --
	(199.10,208.33) --
	(199.11,208.33) --
	(199.16,208.32) --
	(199.19,208.32) --
	(199.23,208.32) --
	(199.25,208.33) --
	(199.29,208.37) --
	(199.30,208.43) --
	(199.30,208.46) --
	(199.31,208.47) --
	(199.32,208.49) --
	(199.33,208.50) --
	(199.36,208.53) --
	(199.40,208.56) --
	(199.43,208.56) --
	(199.46,208.57) --
	(199.48,208.60) --
	(199.48,208.60) --
	(199.49,208.61) --
	(199.49,208.62) --
	(199.49,208.63) --
	(199.49,208.64) --
	(199.50,208.67) --
	(199.52,208.69) --
	(199.55,208.69) --
	(199.65,208.71) --
	(199.68,208.72) --
	(199.71,208.76) --
	(199.71,208.79) --
	(199.73,208.81) --
	(199.73,208.84) --
	(199.74,208.84) --
	(199.74,208.85) --
	(199.75,208.85) --
	(199.76,208.86) --
	(199.76,208.86) --
	(199.77,208.86) --
	(199.78,208.88) --
	(199.81,208.89) --
	(199.85,208.90) --
	(199.88,208.90) --
	(199.91,208.89) --
	(199.97,208.86) --
	(200.04,208.81) --
	(200.07,208.77) --
	(200.11,208.74) --
	(200.13,208.71) --
	(200.15,208.67) --
	(200.12,208.56) --
	(200.10,208.52) --
	(200.06,208.43) --
	(200.06,208.40) --
	(200.08,208.38) --
	(200.10,208.38) --
	(200.13,208.37) --
	(200.15,208.35) --
	(200.19,208.31) --
	(200.26,208.26) --
	(200.27,208.22) --
	(200.27,208.22) --
	(200.24,208.19) --
	(200.23,208.16) --
	(200.23,208.13) --
	(200.33,208.11) --
	(200.35,208.09) --
	(200.38,208.08) --
	(200.43,208.05) --
	(200.44,208.04) --
	(200.43,208.01) --
	(200.42,207.96) --
	(200.44,207.91) --
	(200.46,207.89) --
	(200.48,207.87) --
	(200.60,207.79) --
	(200.61,207.77) --
	(200.62,207.72) --
	(200.62,207.69) --
	(200.60,207.67) --
	(200.58,207.65) --
	(200.55,207.63) --
	(200.55,207.62) --
	(200.55,207.61) --
	(200.56,207.60) --
	(200.59,207.59) --
	(200.66,207.59) --
	(200.68,207.59) --
	(200.71,207.57) --
	(200.75,207.53) --
	(200.79,207.46) --
	(200.83,207.42) --
	(200.85,207.40) --
	(200.88,207.39) --
	(200.91,207.38) --
	(200.97,207.37) --
	(201.10,207.38) --
	(201.13,207.39) --
	(201.31,207.45) --
	(201.34,207.46) --
	(201.37,207.48) --
	(201.40,207.48) --
	(201.43,207.48) --
	(201.47,207.47) --
	(201.51,207.44) --
	(201.55,207.43) --
	(201.58,207.43) --
	(201.61,207.43) --
	(201.67,207.41) --
	(201.76,207.37) --
	(201.77,207.35) --
	(201.79,207.33) --
	(201.82,207.32) --
	(201.88,207.30) --
	(201.91,207.30) --
	(201.94,207.28) --
	(201.98,207.25) --
	(201.99,207.22) --
	(202.02,207.15) --
	(202.03,207.13) --
	(202.05,207.11) --
	(202.07,207.09) --
	(202.11,207.08) --
	(202.13,207.06) --
	(202.15,207.05) --
	(202.18,207.04) --
	(202.22,207.04) --
	(202.24,207.05) --
	(202.25,207.08) --
	(202.23,207.12) --
	(202.23,207.15) --
	(202.22,207.17) --
	(202.21,207.20) --
	(202.21,207.27) --
	(202.22,207.31) --
	(202.25,207.36) --
	(202.28,207.38) --
	(202.30,207.44) --
	(202.31,207.44) --
	(202.32,207.45) --
	(202.39,207.44) --
	(202.43,207.45) --
	(202.45,207.46) --
	(202.46,207.49) --
	(202.48,207.52) --
	(202.50,207.52) --
	(202.53,207.51) --
	(202.57,207.49) --
	(202.58,207.49) --
	(202.59,207.50) --
	(202.64,207.53) --
	(202.68,207.57) --
	(202.71,207.58) --
	(202.74,207.59) --
	(202.79,207.58) --
	(202.80,207.57) --
	(202.82,207.56) --
	(202.83,207.54) --
	(202.85,207.53) --
	(202.86,207.53) --
	(202.90,207.54) --
	(202.93,207.56) --
	(202.95,207.58) --
	(203.04,207.61) --
	(203.07,207.61) --
	(203.10,207.61) --
	(203.13,207.59) --
	(203.19,207.53) --
	(203.29,207.43) --
	(203.31,207.41) --
	(203.34,207.39) --
	(203.37,207.35) --
	(203.40,207.32) --
	(203.44,207.29) --
	(203.51,207.24) --
	(203.56,207.18) --
	(203.60,207.14) --
	(203.63,207.12) --
	(203.73,207.11) --
	(203.76,207.11) --
	(203.82,207.09) --
	(203.85,207.08) --
	(204.01,206.99) --
	(204.04,206.98) --
	(204.16,206.95) --
	(204.19,206.94) --
	(204.23,206.92) --
	(204.27,206.90) --
	(204.31,206.90) --
	(204.39,206.90) --
	(204.44,206.91) --
	(204.51,206.92) --
	(204.56,206.95) --
	(204.59,206.98) --
	(204.59,206.99) --
	(204.58,207.01) --
	(204.57,207.04) --
	(204.58,207.05) --
	(204.58,207.06) --
	(204.59,207.06) --
	(204.60,207.07) --
	(204.62,207.08) --
	(204.62,207.11) --
	(204.61,207.14) --
	(204.56,207.22) --
	(204.54,207.30) --
	(204.54,207.37) --
	(204.55,207.42) --
	(204.57,207.44) --
	(204.59,207.44) --
	(204.61,207.44) --
	(204.63,207.45) --
	(204.61,207.48) --
	(204.61,207.50) --
	(204.61,207.52) --
	(204.62,207.54) --
	(204.64,207.56) --
	(204.70,207.59) --
	(204.72,207.60) --
	(204.78,207.62) --
	(204.85,207.63) --
	(204.86,207.63) --
	(204.93,207.59) --
	(204.96,207.58) --
	(204.98,207.56) --
	(205.01,207.55) --
	(205.08,207.54) --
	(205.11,207.54) --
	(205.14,207.55) --
	(205.17,207.55) --
	(205.23,207.58) --
	(205.28,207.61) --
	(205.33,207.67) --
	(205.35,207.70) --
	(205.37,207.69) --
	(205.38,207.68) --
	(205.43,207.64) --
	(205.48,207.59) --
	(205.50,207.54) --
	(205.63,207.37) --
	(205.65,207.32) --
	(205.65,207.30) --
	(205.67,207.22) --
	(205.75,207.11) --
	(205.78,207.07) --
	(205.98,206.91) --
	(206.03,206.85) --
	(206.08,206.82) --
	(206.13,206.79) --
	(206.19,206.77) --
	(206.25,206.76) --
	(206.29,206.75) --
	(206.38,206.78) --
	(206.46,206.82) --
	(206.54,206.87) --
	(206.55,206.87) --
	(206.57,206.87) --
	(206.67,206.84) --
	(206.70,206.84) --
	(206.73,206.83) --
	(206.86,206.86) --
	(206.92,206.85) --
	(207.05,206.85) --
	(207.16,206.85) --
	(207.26,206.89) --
	(207.29,206.91) --
	(207.32,206.92) --
	(207.36,206.93) --
	(207.39,206.95) --
	(207.41,206.96) --
	(207.47,207.02) --
	(207.49,207.06) --
	(207.50,207.09) --
	(207.51,207.11) --
	(207.57,207.20) --
	(207.63,207.25) --
	(207.67,207.27) --
	(207.73,207.27) --
	(207.85,207.25) --
	(207.94,207.22) --
	(207.97,207.21) --
	(208.04,207.15) --
	(208.07,207.11) --
	(208.11,207.07) --
	(208.16,207.04) --
	(208.20,207.00) --
	(208.21,206.98) --
	(208.24,206.97) --
	(208.27,206.95) --
	(208.52,206.91) --
	(208.57,206.90) --
	(208.58,206.91) --
	(208.58,206.92) --
	(208.56,206.94) --
	(208.58,206.99) --
	(208.58,207.02) --
	(208.59,207.04) --
	(208.60,207.07) --
	(208.62,207.08) --
	(208.66,207.09) --
	(208.68,207.10) --
	(208.78,207.12) --
	(208.81,207.14) --
	(208.83,207.16) --
	(208.87,207.18) --
	(208.87,207.20) --
	(208.89,207.22) --
	(208.91,207.24) --
	(208.97,207.29) --
	(208.99,207.31) --
	(209.02,207.33) --
	(209.12,207.34) --
	(209.15,207.35) --
	(209.23,207.38) --
	(209.25,207.40) --
	(209.28,207.42) --
	(209.31,207.43) --
	(209.34,207.44) --
	(209.36,207.46) --
	(209.40,207.47) --
	(209.45,207.50) --
	(209.46,207.50) --
	(209.49,207.49) --
	(209.63,207.45) --
	(209.69,207.43) --
	(209.72,207.42) --
	(209.74,207.39) --
	(209.74,207.37) --
	(209.77,207.34) --
	(209.80,207.34) --
	(209.82,207.35) --
	(209.97,207.40) --
	(210.24,207.54) --
	(210.31,207.58) --
	(210.34,207.60) --
	(210.37,207.60) --
	(210.47,207.60) --
	(210.53,207.62) --
	(210.59,207.63) --
	(210.62,207.65) --
	(210.65,207.66) --
	(210.77,207.70) --
	(210.82,207.72) --
	(210.87,207.75) --
	(210.98,207.81) --
	(211.02,207.84) --
	(211.05,207.86) --
	(211.21,207.88) --
	(211.39,207.95) --
	(211.42,207.95) --
	(211.45,207.95) --
	(211.49,207.94) --
	(211.52,207.92) --
	(211.52,207.91) --
	(211.57,207.88) --
	(211.67,207.75) --
	(211.68,207.74) --
	(211.70,207.71) --
	(211.73,207.70) --
	(211.75,207.71) --
	(211.81,207.72) --
	(211.85,207.74) --
	(211.87,207.74) --
	(211.89,207.74) --
	(211.89,207.74) --
	(211.89,207.73) --
	(211.93,207.72) --
	(212.04,207.73) --
	(212.11,207.73) --
	(212.19,207.72) --
	(212.22,207.71) --
	(212.29,207.68) --
	(212.39,207.62) --
	(212.44,207.60) --
	(212.49,207.58) --
	(212.53,207.56) --
	(212.55,207.54) --
	(212.61,207.48) --
	(212.66,207.41) --
	(212.68,207.37) --
	(212.70,207.36) --
	(212.76,207.35) --
	(212.81,207.34) --
	(212.89,207.34) --
	(212.91,207.34) --
	(212.94,207.35) --
	(212.96,207.37) --
	(212.98,207.38) --
	(213.04,207.45) --
	(213.04,207.47) --
	(213.08,207.53) --
	(213.10,207.59) --
	(213.13,207.62) --
	(213.17,207.65) --
	(213.21,207.68) --
	(213.30,207.72) --
	(213.38,207.74) --
	(213.40,207.76) --
	(213.42,207.77) --
	(213.43,207.79) --
	(213.45,207.80) --
	(213.47,207.86) --
	(213.47,207.88) --
	(213.47,207.90) --
	(213.46,207.92) --
	(213.30,208.04) --
	(213.26,208.09) --
	(213.25,208.11) --
	(213.25,208.13) --
	(213.27,208.17) --
	(213.30,208.21) --
	(213.34,208.23) --
	(213.38,208.25) --
	(213.56,208.31) --
	(213.60,208.33) --
	(213.69,208.38) --
	(213.71,208.39) --
	(213.77,208.46) --
	(213.78,208.48) --
	(213.85,208.57) --
	(213.88,208.60) --
	(213.92,208.63) --
	(214.00,208.65) --
	(214.02,208.65) --
	(214.07,208.65) --
	(214.12,208.64) --
	(214.15,208.63) --
	(214.25,208.59) --
	(214.26,208.59) --
	(214.27,208.62) --
	(214.29,208.64) --
	(214.30,208.66) --
	(214.31,208.68) --
	(214.32,208.70) --
	(214.33,208.71) --
	(214.33,208.74) --
	(214.34,208.76) --
	(214.35,208.78) --
	(214.36,208.80) --
	(214.36,208.82) --
	(214.40,208.96) --
	(214.41,209.02) --
	(214.42,209.08) --
	(214.42,209.10) --
	(214.43,209.12) --
	(214.42,209.14) --
	(214.43,209.17) --
	(214.42,209.18) --
	(214.43,209.21) --
	(214.42,209.23) --
	(214.42,209.25) --
	(214.41,209.27) --
	(214.41,209.29) --
	(214.37,209.39) --
	(214.36,209.43) --
	(214.37,209.51) --
	(214.40,209.59) --
	(214.42,209.61) --
	(214.42,209.63) --
	(214.44,209.65) --
	(214.44,209.67) --
	(214.45,209.69) --
	(214.45,209.71) --
	(214.46,209.73) --
	(214.46,209.81) --
	(214.45,209.89) --
	(214.44,209.93) --
	(214.45,210.03) --
	(214.46,210.05) --
	(214.47,210.07) --
	(214.47,210.09) --
	(214.48,210.14) --
	(214.47,210.24) --
	(214.48,210.28) --
	(214.49,210.30) --
	(214.50,210.31) --
	(214.52,210.33) --
	(214.59,210.39) --
	(214.63,210.47) --
	(214.65,210.48) --
	(214.67,210.50) --
	(214.73,210.51) --
	(214.75,210.50) --
	(214.81,210.49) --
	(214.83,210.49) --
	(214.91,210.49) --
	(214.94,210.49) --
	(215.22,210.57) --
	(215.34,210.62) --
	(215.50,210.70) --
	(215.59,210.76) --
	(215.65,210.71) --
	(215.67,210.70) --
	(215.73,210.69) --
	(215.83,210.67) --
	(215.85,210.66) --
	(215.91,210.66) --
	(215.96,210.65) --
	(216.01,210.64) --
	(216.06,210.63) --
	(216.11,210.60) --
	(216.23,210.53) --
	(216.29,210.49) --
	(216.34,210.44) --
	(216.39,210.39) --
	(216.44,210.37) --
	(216.52,210.35) --
	(216.56,210.34) --
	(216.62,210.33) --
	(216.69,210.30) --
	(216.72,210.27) --
	(216.75,210.24) --
	(216.76,210.22) --
	(216.78,210.20) --
	(216.80,210.16) --
	(216.80,210.12) --
	(216.74,209.94) --
	(216.73,209.88) --
	(216.72,209.86) --
	(216.72,209.84) --
	(216.71,209.82) --
	(216.71,209.80) --
	(216.70,209.78) --
	(216.69,209.76) --
	(216.59,209.61) --
	(216.54,209.56) --
	(216.51,209.48) --
	(216.51,209.44) --
	(216.52,209.42) --
	(216.52,209.40) --
	(216.52,209.38) --
	(216.51,209.36) --
	(216.51,209.34) --
	(216.49,209.30) --
	(216.50,209.27) --
	(216.51,209.26) --
	(216.52,209.24) --
	(216.58,209.19) --
	(216.66,209.12) --
	(216.66,209.12) --
	(216.67,209.09) --
	(216.69,209.08) --
	(216.70,209.06) --
	(216.71,209.04) --
	(216.72,209.02) --
	(216.74,209.01) --
	(216.75,208.99) --
	(216.78,208.95) --
	(216.84,208.86) --
	(216.89,208.81) --
	(216.92,208.76) --
	(216.95,208.72) --
	(216.98,208.66) --
	(216.99,208.64) --
	(217.00,208.63) --
	(217.01,208.60) --
	(217.03,208.59) --
	(217.03,208.56) --
	(217.04,208.55) --
	(217.11,208.35) --
	(217.11,208.33) --
	(217.11,208.31) --
	(217.12,208.29) --
	(217.13,208.27) --
	(217.14,208.24) --
	(217.20,208.14) --
	(217.42,207.90) --
	(217.48,207.83) --
	(217.56,207.75) --
	(217.90,207.46) --
	(217.97,207.40) --
	(218.06,207.35) --
	(218.08,207.34) --
	(218.16,207.33) --
	(218.29,207.33) --
	(218.32,207.33) --
	(218.37,207.31) --
	(218.42,207.29) --
	(218.46,207.24) --
	(218.47,207.21) --
	(218.49,207.16) --
	(218.50,207.14) --
	(218.52,207.12) --
	(218.54,207.11) --
	(218.59,207.06) --
	(218.62,207.03) --
	(218.65,207.02) --
	(218.67,207.01) --
	(218.77,206.99) --
	(218.85,206.97) --
	(218.94,206.93) --
	(218.98,206.90) --
	(219.02,206.87) --
	(219.09,206.83) --
	(219.10,206.83) --
	(219.12,206.83) --
	(219.13,206.81) --
	(219.20,206.78) --
	(219.28,206.72) --
	(219.31,206.69) --
	(219.33,206.65) --
	(219.33,206.61) --
	(219.31,206.57) --
	(219.27,206.52) --
	(219.27,206.51) --
	(219.26,206.49) --
	(219.27,206.47) --
	(219.31,206.43) --
	(219.33,206.42) --
	(219.40,206.40) --
	(219.49,206.40) --
	(219.51,206.39) --
	(219.52,206.37) --
	(219.52,206.36) --
	(219.50,206.31) --
	(219.51,206.30) --
	(219.52,206.28) --
	(219.54,206.27) --
	(219.62,206.25) --
	(219.68,206.25) --
	(219.68,206.24) --
	(219.69,206.24) --
	(219.69,206.22) --
	(219.63,206.17) --
	(219.61,206.17) --
	(219.61,206.14) --
	(219.60,206.13) --
	(219.58,206.08) --
	(219.59,206.06) --
	(219.60,206.04) --
	(219.62,206.03) --
	(219.64,206.02) --
	(219.69,206.00) --
	(219.71,205.98) --
	(219.72,205.98) --
	(219.72,205.96) --
	(219.71,205.96) --
	(219.71,205.95) --
	(219.69,205.94) --
	(219.64,205.90) --
	(219.62,205.85) --
	(219.63,205.83) --
	(219.70,205.77) --
	(219.72,205.75) --
	(219.75,205.72) --
	(219.77,205.71) --
	(219.79,205.65) --
	(219.79,205.60) --
	(219.76,205.50) --
	(219.76,205.46) --
	(219.77,205.44) --
	(219.79,205.43) --
	(219.81,205.41) --
	(219.89,205.40) --
	(219.94,205.39) --
	(219.97,205.38) --
	(220.02,205.38) --
	(220.13,205.36) --
	(220.17,205.34) --
	(220.27,205.30) --
	(220.34,205.24) --
	(220.41,205.18) --
	(220.50,205.06) --
	(220.51,205.05) --
	(220.52,205.03) --
	(220.51,205.02) --
	(220.51,205.01) --
	(220.49,204.99) --
	(220.44,204.97) --
	(220.39,204.95) --
	(220.34,204.95) --
	(220.28,204.96) --
	(220.25,204.97) --
	(220.24,204.97) --
	(220.22,204.96) --
	(220.22,204.95) --
	(220.21,204.94) --
	(220.21,204.93) --
	(220.21,204.91) --
	(220.23,204.88) --
	(220.26,204.84) --
	(220.27,204.80) --
	(220.28,204.78) --
	(220.27,204.76) --
	(220.23,204.71) --
	(220.21,204.69) --
	(220.19,204.67) --
	(220.14,204.63) --
	(220.10,204.60) --
	(220.09,204.56) --
	(220.09,204.52) --
	(220.10,204.46) --
	(220.09,204.42) --
	(220.04,204.34) --
	(220.03,204.30) --
	(220.04,204.24) --
	(220.05,204.22) --
	(220.05,204.16) --
	(220.04,204.14) --
	(220.03,204.12) --
	(220.00,204.06) --
	(220.00,204.04) --
	(220.03,203.96) --
	(220.03,203.94) --
	(220.01,203.88) --
	(220.00,203.86) --
	(219.98,203.84) --
	(219.96,203.83) --
	(219.89,203.76) --
	(219.87,203.73) --
	(219.85,203.65) --
	(219.85,203.63) --
	(219.85,203.61) --
	(219.85,203.58) --
	(219.85,203.56) --
	(219.85,203.54) --
	(219.85,203.52) --
	(219.85,203.50) --
	(219.85,203.48) --
	(219.87,203.42) --
	(219.91,203.34) --
	(219.92,203.32) --
	(219.93,203.31) --
	(219.94,203.29) --
	(219.98,203.24) --
	(220.00,203.22) --
	(220.02,203.20) --
	(220.05,203.17) --
	(220.07,203.13) --
	(220.08,203.05) --
	(220.08,203.01) --
	(220.05,202.95) --
	(220.03,202.91) --
	(220.03,202.83) --
	(220.07,202.77) --
	(220.10,202.72) --
	(220.13,202.69) --
	(220.21,202.63) --
	(220.25,202.60) --
	(220.44,202.49) --
	(220.46,202.48) --
	(220.51,202.47) --
	(220.57,202.46) --
	(220.65,202.46) --
	(220.68,202.45) --
	(220.73,202.41) --
	(220.74,202.40) --
	(220.77,202.39) --
	(220.80,202.38) --
	(220.88,202.37) --
	(220.90,202.36) --
	(220.91,202.34) --
	(220.92,202.30) --
	(220.92,202.24) --
	(220.93,202.21) --
	(220.93,202.21) --
	(220.96,202.19) --
	(221.02,202.19) --
	(221.06,202.17) --
	(221.06,202.16) --
	(221.06,202.12) --
	(221.06,202.11) --
	(221.06,202.10) --
	(221.07,202.08) --
	(221.06,202.06) --
	(221.07,202.06) --
	(221.07,202.05) --
	(221.08,202.04) --
	(221.09,202.02) --
	(221.11,201.96) --
	(221.12,201.94) --
	(221.13,201.90) --
	(221.13,201.82) --
	(221.13,201.78) --
	(221.13,201.76) --
	(221.12,201.74) --
	(221.13,201.72) --
	(221.12,201.70) --
	(221.10,201.66) --
	(221.01,201.59) --
	(220.99,201.57) --
	(220.98,201.56) --
	(220.96,201.54) --
	(220.94,201.52) --
	(220.91,201.49) --
	(220.89,201.45) --
	(220.88,201.37) --
	(220.89,201.33) --
	(220.90,201.29) --
	(220.93,201.25) --
	(220.97,201.18) --
	(220.98,201.16) --
	(220.97,201.13) --
	(220.95,201.11) --
	(220.88,201.08) --
	(220.86,201.06) --
	(220.85,201.04) --
	(220.85,200.95) --
	(220.83,200.91) --
	(220.83,200.89) --
	(220.86,200.85) --
	(220.91,200.80) --
	(220.95,200.75) --
	(220.95,200.74) --
	(220.93,200.72) --
	(220.91,200.71) --
	(220.86,200.63) --
	(220.85,200.61) --
	(220.84,200.59) --
	(220.82,200.57) --
	(220.77,200.56) --
	(220.74,200.54) --
	(220.73,200.51) --
	(220.75,200.47) --
	(220.75,200.44) --
	(220.73,200.40) --
	(220.69,200.37) --
	(220.59,200.33) --
	(220.58,200.32) --
	(220.55,200.28) --
	(220.54,200.24) --
	(220.54,200.22) --
	(220.54,200.20) --
	(220.56,200.18) --
	(220.62,200.14) --
	(220.63,200.10) --
	(220.63,200.05) --
	(220.63,200.03) --
	(220.65,200.00) --
	(220.66,199.98) --
	(220.68,199.97) --
	(220.72,199.94) --
	(220.80,199.91) --
	(220.81,199.90) --
	(220.82,199.83) --
	(220.78,199.70) --
	(220.74,199.62) --
	(220.68,199.55) --
	(220.63,199.51) --
	(220.60,199.48) --
	(220.54,199.43) --
	(220.51,199.38) --
	(220.50,199.34) --
	(220.49,199.32) --
	(220.50,199.25) --
	(220.50,199.24) --
	(220.48,199.22) --
	(220.46,199.20) --
	(220.44,199.19) --
	(220.41,199.18) --
	(220.39,199.18) --
	(220.28,199.19) --
	(220.20,199.18) --
	(220.10,199.14) --
	(220.05,199.14) --
	(219.99,199.14) --
	(219.89,199.13) --
	(219.71,199.10) --
	(219.65,199.09) --
	(219.61,199.07) --
	(219.60,199.05) --
	(219.59,199.03) --
	(219.59,199.01) --
	(219.59,198.98) --
	(219.59,198.96) --
	(219.60,198.93) --
	(219.58,198.89) --
	(219.59,198.85) --
	(219.63,198.75) --
	(219.64,198.74) --
	(219.68,198.71) --
	(219.69,198.69) --
	(219.71,198.61) --
	(219.71,198.57) --
	(219.69,198.55) --
	(219.68,198.49) --
	(219.69,198.47) --
	(219.68,198.45) --
	(219.68,198.43) --
	(219.70,198.39) --
	(219.70,198.35) --
	(219.70,198.31) --
	(219.70,198.29) --
	(219.69,198.21) --
	(219.67,198.15) --
	(219.60,198.03) --
	(219.58,197.97) --
	(219.58,197.95) --
	(219.57,197.91) --
	(219.58,197.87) --
	(219.61,197.84) --
	(219.63,197.80) --
	(219.64,197.78) --
	(219.65,197.76) --
	(219.63,197.72) --
	(219.61,197.68) --
	(219.58,197.65) --
	(219.50,197.61) --
	(219.47,197.57) --
	(219.45,197.51) --
	(219.45,197.49) --
	(219.45,197.47) --
	(219.46,197.42) --
	(219.45,197.34) --
	(219.45,197.32) --
	(219.49,197.27) --
	(219.51,197.26) --
	(219.54,197.24) --
	(219.61,197.22) --
	(219.66,197.20) --
	(219.68,197.19) --
	(219.70,197.17) --
	(219.72,197.16) --
	(219.75,197.13) --
	(219.78,197.12) --
	(219.83,197.12) --
	(219.86,197.11) --
	(219.88,197.10) --
	(219.90,197.07) --
	(219.89,197.06) --
	(219.83,197.01) --
	(219.77,196.94) --
	(219.76,196.90) --
	(219.74,196.84) --
	(219.74,196.82) --
	(219.75,196.75) --
	(219.75,196.73) --
	(219.74,196.71) --
	(219.68,196.65) --
	(219.62,196.58) --
	(219.60,196.54) --
	(219.59,196.50) --
	(219.59,196.49) --
	(219.59,196.48) --
	(219.60,196.47) --
	(219.60,196.40) --
	(219.60,196.38) --
	(219.58,196.37) --
	(219.49,196.26) --
	(219.49,196.24) --
	(219.49,196.22) --
	(219.48,196.20) --
	(219.46,196.18) --
	(219.42,196.16) --
	(219.36,196.11) --
	(219.35,196.10) --
	(219.33,196.06) --
	(219.31,196.02) --
	(219.31,196.00) --
	(219.35,195.90) --
	(219.35,195.88) --
	(219.34,195.84) --
	(219.34,195.82) --
	(219.33,195.78) --
	(219.33,195.74) --
	(219.33,195.68) --
	(219.34,195.64) --
	(219.40,195.40) --
	(219.40,195.38) --
	(219.42,195.28) --
	(219.45,195.20) --
	(219.48,195.16) --
	(219.54,195.11) --
	(219.53,195.11) --
	(219.52,195.09) --
	(219.49,195.05) --
	(219.48,195.04) --
	(219.47,195.02) --
	(219.47,195.00) --
	(219.45,194.96) --
	(219.44,194.85) --
	(219.42,194.80) --
	(219.39,194.76) --
	(219.34,194.74) --
	(219.27,194.71) --
	(219.19,194.69) --
	(219.17,194.70) --
	(219.02,194.74) --
	(218.99,194.74) --
	(218.96,194.74) --
	(218.91,194.72) --
	(218.88,194.69) --
	(218.86,194.65) --
	(218.85,194.63) --
	(218.85,194.61) --
	(218.85,194.59) --
	(218.85,194.55) --
	(218.81,194.45) --
	(218.80,194.41) --
	(218.76,194.35) --
	(218.70,194.29) --
	(218.68,194.27) --
	(218.55,194.17) --
	(218.53,194.16) --
	(218.49,194.13) --
	(218.48,194.11) --
	(218.44,194.06) --
	(218.44,194.03) --
	(218.43,193.97) --
	(218.42,193.95) --
	(218.41,193.94) --
	(218.34,193.90) --
	(218.25,193.86) --
	(218.18,193.84) --
	(218.02,193.81) --
	(217.97,193.79) --
	(217.86,193.74) --
	(217.78,193.72) --
	(217.73,193.71) --
	(217.68,193.72) --
	(217.62,193.73) --
	(217.53,193.76) --
	(217.50,193.77) --
	(217.45,193.78) --
	(217.38,193.79) --
	(217.31,193.82) --
	(217.20,193.88) --
	(217.18,193.89) --
	(217.15,193.89) --
	(217.09,193.88) --
	(217.05,193.87) --
	(217.02,193.86) --
	(216.99,193.87) --
	(216.97,193.87) --
	(216.92,193.89) --
	(216.89,193.89) --
	(216.86,193.89) --
	(216.84,193.88) --
	(216.82,193.87) --
	(216.77,193.82) --
	(216.75,193.80) --
	(216.70,193.79) --
	(216.66,193.77) --
	(216.65,193.74) --
	(216.64,193.72) --
	(216.60,193.65) --
	(216.56,193.62) --
	(216.49,193.59) --
	(216.44,193.58) --
	(216.39,193.56) --
	(216.35,193.53) --
	(216.28,193.50) --
	(216.26,193.49) --
	(216.23,193.49) --
	(215.89,193.53) --
	(215.79,193.56) --
	(215.74,193.55) --
	(215.68,193.54) --
	(215.66,193.54) --
	(215.63,193.54) --
	(215.61,193.55) --
	(215.57,193.58) --
	(215.55,193.59) --
	(215.53,193.63) --
	(215.48,193.70) --
	(215.46,193.72) --
	(215.41,193.73) --
	(215.33,193.74) --
	(215.17,193.74) --
	(215.04,193.75) --
	(215.01,193.76) --
	(214.99,193.77) --
	(214.96,193.77) --
	(214.91,193.77) --
	(214.86,193.77) --
	(214.81,193.76) --
	(214.73,193.75) --
	(214.70,193.74) --
	(214.68,193.73) --
	(214.67,193.71) --
	(214.67,193.69) --
	(214.66,193.60) --
	(214.66,193.54) --
	(214.65,193.50) --
	(214.62,193.47) --
	(214.58,193.42) --
	(214.57,193.39) --
	(214.57,193.38) --
	(214.58,193.34) --
	(214.63,193.26) --
	(214.64,193.24) --
	(214.65,193.20) --
	(214.67,193.17) --
	(214.69,193.11) --
	(214.70,193.01) --
	(214.73,192.97) --
	(214.76,192.93) --
	(214.79,192.90) --
	(214.82,192.87) --
	(214.83,192.85) --
	(214.84,192.83) --
	(214.85,192.81) --
	(214.84,192.77) --
	(214.82,192.73) --
	(214.75,192.65) --
	(214.74,192.63) --
	(214.74,192.61) --
	(214.74,192.59) --
	(214.76,192.55) --
	(214.79,192.49) --
	(214.80,192.45) --
	(214.79,192.41) --
	(214.79,192.35) --
	(214.77,192.26) --
	(214.72,192.19) --
	(214.69,192.16) --
	(214.65,192.13) --
	(214.60,192.11) --
	(214.55,192.10) --
	(214.50,192.09) --
	(214.37,192.08) --
	(214.29,192.06) --
	(214.25,192.03) --
	(214.23,192.02) --
	(214.21,191.98) --
	(214.17,191.88) --
	(214.13,191.81) --
	(214.11,191.79) --
	(213.99,191.68) --
	(213.91,191.63) --
	(213.88,191.59) --
	(213.86,191.55) --
	(213.84,191.52) --
	(213.81,191.48) --
	(213.80,191.44) --
	(213.78,191.36) --
	(213.80,191.31) --
	(213.81,191.28) --
	(213.84,191.23) --
	(213.87,191.15) --
	(213.88,191.10) --
	(213.89,191.08) --
	(213.89,191.06) --
	(213.87,190.98) --
	(213.87,190.96) --
	(213.84,190.93) --
	(213.80,190.90) --
	(213.76,190.82) --
	(213.75,190.80) --
	(213.75,190.78) --
	(213.76,190.74) --
	(213.80,190.67) --
	(213.82,190.66) --
	(213.89,190.58) --
	(213.97,190.50) --
	(214.03,190.43) --
	(214.06,190.42) --
	(214.08,190.42) --
	(214.11,190.42) --
	(214.16,190.44) --
	(214.18,190.45) --
	(214.20,190.46) --
	(214.22,190.47) --
	(214.30,190.49) --
	(214.32,190.49) --
	(214.35,190.49) --
	(214.40,190.47) --
	(214.48,190.42) --
	(214.55,190.36) --
	(214.57,190.34) --
	(214.59,190.34) --
	(214.62,190.33) --
	(214.68,190.33) --
	(214.72,190.34) --
	(214.75,190.35) --
	(214.94,190.46) --
	(214.99,190.48) --
	(215.09,190.50) --
	(215.17,190.52) --
	(215.25,190.52) --
	(215.27,190.52) --
	(215.30,190.51) --
	(215.32,190.49) --
	(215.34,190.46) --
	(215.46,190.28) --
	(215.49,190.25) --
	(215.51,190.23) --
	(215.55,190.17) --
	(215.56,190.11) --
	(215.60,190.06) --
	(215.65,190.01) --
	(215.68,190.00) --
	(215.75,189.99) --
	(215.78,189.98) --
	(215.93,189.90) --
	(215.98,189.88) --
	(216.06,189.87) --
	(216.08,189.87) --
	(216.14,189.87) --
	(216.22,189.88) --
	(216.27,189.88) --
	(216.30,189.87) --
	(216.32,189.86) --
	(216.33,189.84) --
	(216.37,189.75) --
	(216.38,189.73) --
	(216.40,189.71) --
	(216.43,189.71) --
	(216.56,189.75) --
	(216.64,189.76) --
	(216.66,189.75) --
	(216.77,189.75) --
	(216.82,189.75) --
	(216.87,189.76) --
	(216.98,189.76) --
	(217.01,189.76) --
	(217.05,189.74) --
	(217.09,189.71) --
	(217.20,189.66) --
	(217.22,189.64) --
	(217.23,189.62) --
	(217.24,189.60) --
	(217.25,189.58) --
	(217.26,189.57) --
	(217.26,189.54) --
	(217.28,189.53) --
	(217.28,189.50) --
	(217.28,189.46) --
	(217.29,189.36) --
	(217.32,189.26) --
	(217.33,189.24) --
	(217.36,189.19) --
	(217.39,189.14) --
	(217.39,189.13) --
	(217.42,189.05) --
	(217.42,189.03) --
	(217.41,189.01) --
	(217.40,188.99) --
	(217.33,188.94) --
	(217.32,188.93) --
	(217.31,188.93) --
	(217.24,188.90) --
	(217.22,188.88) --
	(217.09,188.82) --
	(216.98,188.75) --
	(216.84,188.66) --
	(216.73,188.60) --
	(216.71,188.59) --
	(216.69,188.56) --
	(216.70,188.55) --
	(216.68,188.51) --
	(216.66,188.49) --
	(216.62,188.47) --
	(216.48,188.42) --
	(216.43,188.41) --
	(216.35,188.42) --
	(216.32,188.42) --
	(216.27,188.43) --
	(216.24,188.43) --
	(216.19,188.42) --
	(216.13,188.37) --
	(216.09,188.35) --
	(215.99,188.28) --
	(215.94,188.26) --
	(215.92,188.25) --
	(215.91,188.23) --
	(215.90,188.20) --
	(215.91,188.17) --
	(215.92,188.15) --
	(215.96,188.12) --
	(216.02,188.08) --
	(216.05,188.05) --
	(216.09,188.02) --
	(216.17,187.91) --
	(216.18,187.87) --
	(216.20,187.83) --
	(216.23,187.78) --
	(216.23,187.76) --
	(216.26,187.72) --
	(216.28,187.71) --
	(216.38,187.68) --
	(216.41,187.66) --
	(216.41,187.65) --
	(216.41,187.63) --
	(216.40,187.59) --
	(216.37,187.47) --
	(216.37,187.45) --
	(216.39,187.41) --
	(216.40,187.37) --
	(216.38,187.32) --
	(216.36,187.31) --
	(216.33,187.31) --
	(216.32,187.29) --
	(216.30,187.27) --
	(216.29,187.25) --
	(216.25,187.19) --
	(216.24,187.18) --
	(216.21,187.18) --
	(216.16,187.17) --
	(216.06,187.17) --
	(216.03,187.17) --
	(215.80,187.06) --
	(215.75,187.04) --
	(215.73,187.04) --
	(215.65,187.04) --
	(215.60,187.05) --
	(215.56,187.08) --
	(215.48,187.13) --
	(215.43,187.15) --
	(215.41,187.16) --
	(215.38,187.16) --
	(215.36,187.16) --
	(215.31,187.13) --
	(215.27,187.10) --
	(215.26,187.08) --
	(215.25,187.05) --
	(215.15,186.90) --
	(215.13,186.83) --
	(215.14,186.82) --
	(215.16,186.81) --
	(215.20,186.79) --
	(215.28,186.77) --
	(215.30,186.76) --
	(215.32,186.75) --
	(215.38,186.66) --
	(215.43,186.58) --
	(215.44,186.56) --
	(215.43,186.54) --
	(215.40,186.52) --
	(215.37,186.52) --
	(215.33,186.49) --
	(215.27,186.45) --
	(215.25,186.44) --
	(215.23,186.38) --
	(215.22,186.29) --
	(215.21,186.28) --
	(215.20,186.28) --
	(215.11,186.27) --
	(215.06,186.26) --
	(215.01,186.24) --
	(214.97,186.21) --
	(214.94,186.18) --
	(214.90,186.13) --
	(214.88,186.07) --
	(214.86,186.05) --
	(214.84,186.04) --
	(214.81,186.03) --
	(214.74,186.02) --
	(214.66,186.03) --
	(214.61,186.02) --
	(214.59,186.02) --
	(214.57,186.00) --
	(214.58,185.94) --
	(214.58,185.92) --
	(214.58,185.88) --
	(214.55,185.84) --
	(214.54,185.83) --
	(214.49,185.78) --
	(214.46,185.77) --
	(214.41,185.72) --
	(214.39,185.68) --
	(214.36,185.63) --
	(214.35,185.57) --
	(214.33,185.53) --
	(214.30,185.45) --
	(214.27,185.41) --
	(214.21,185.37) --
	(214.16,185.36) --
	(214.14,185.35) --
	(214.11,185.35) --
	(213.93,185.38) --
	(213.80,185.39) --
	(213.64,185.39) --
	(213.43,185.36) --
	(213.41,185.35) --
	(213.38,185.34) --
	(213.37,185.32) --
	(213.36,185.30) --
	(213.33,185.18) --
	(213.33,185.12) --
	(213.33,185.10) --
	(213.31,185.04) --
	(213.30,185.00) --
	(213.28,184.96) --
	(213.27,184.95) --
	(213.24,184.93) --
	(213.19,184.92) --
	(213.14,184.91) --
	(213.01,184.91) --
	(212.85,184.93) --
	(212.80,184.94) --
	(212.74,184.94) --
	(212.74,184.93) --
	(212.76,184.89) --
	(212.78,184.81) --
	(212.82,184.76) --
	(212.94,184.65) --
	(212.98,184.60) --
	(213.01,184.54) --
	(213.03,184.48) --
	(213.06,184.39) --
	(213.07,184.34) --
	(213.06,184.32) --
	(213.05,184.30) --
	(213.04,184.30) --
	(213.03,184.29) --
	(212.99,184.29) --
	(212.91,184.29) --
	(212.81,184.26) --
	(212.77,184.24) --
	(212.72,184.23) --
	(212.59,184.20) --
	(212.57,184.19) --
	(212.52,184.17) --
	(212.45,184.11) --
	(212.40,184.09) --
	(212.34,184.05) --
	(212.30,184.03) --
	(212.26,184.00) --
	(212.23,184.00) --
	(212.18,183.99) --
	(212.13,183.98) --
	(212.11,183.97) --
	(212.08,183.95) --
	(212.06,183.91) --
	(212.04,183.86) --
	(212.00,183.82) --
	(211.96,183.79) --
	(211.93,183.78) --
	(211.89,183.76) --
	(211.89,183.77) --
	(211.89,183.77) --
	(211.82,183.75) --
	(211.72,183.69) --
	(211.69,183.67) --
	(211.66,183.66) --
	(211.63,183.65) --
	(211.55,183.64) --
	(211.52,183.63) --
	(211.50,183.61) --
	(211.49,183.59) --
	(211.46,183.56) --
	(211.44,183.52) --
	(211.44,183.51) --
	(211.45,183.47) --
	(211.48,183.43) --
	(211.49,183.40) --
	(211.49,183.38) --
	(211.44,183.12) --
	(211.43,183.10) --
	(211.40,183.07) --
	(211.36,183.03) --
	(211.33,183.02) --
	(211.30,183.00) --
	(211.23,183.01) --
	(211.20,183.00) --
	(211.15,182.98) --
	(211.12,182.96) --
	(211.09,182.95) --
	(210.92,182.94) --
	(210.79,182.94) --
	(210.76,182.94) --
	(210.70,182.94) --
	(210.64,182.95) --
	(210.42,182.97) --
	(210.36,182.98) --
	(210.33,183.00) --
	(210.16,183.08) --
	(210.02,183.17) --
	(209.98,183.21) --
	(209.94,183.22) --
	(209.91,183.24) --
	(209.89,183.24) --
	(209.85,183.23) --
	(209.78,183.24) --
	(209.74,183.25) --
	(209.71,183.27) --
	(209.68,183.29) --
	(209.63,183.33) --
	(209.62,183.35) --
	(209.60,183.37) --
	(209.57,183.40) --
	(209.52,183.42) --
	(209.32,183.42) --
	(209.23,183.42) --
	(209.21,183.42) --
	(209.19,183.41) --
	(209.15,183.40) --
	(209.11,183.39) --
	(209.04,183.39) --
	(208.90,183.39) --
	(208.87,183.41) --
	(208.83,183.43) --
	(208.80,183.44) --
	(208.75,183.45) --
	(208.62,183.44) --
	(208.59,183.45) --
	(208.53,183.46) --
	(208.51,183.45) --
	(208.49,183.44) --
	(208.48,183.41) --
	(208.47,183.34) --
	(208.45,183.30) --
	(208.40,183.22) --
	(208.38,183.21) --
	(208.34,183.19) --
	(208.31,183.18) --
	(208.27,183.18) --
	(208.22,183.18) --
	(208.20,183.17) --
	(208.17,183.14) --
	(208.15,183.11) --
	(208.11,183.09) --
	(208.06,183.05) --
	(208.03,183.03) --
	(208.00,183.02) --
	(207.98,183.02) --
	(207.90,183.04) --
	(207.76,183.12) --
	(207.73,183.13) --
	(207.62,183.17) --
	(207.51,183.20) --
	(207.36,183.23) --
	(207.29,183.24) --
	(207.24,183.25) --
	(207.13,183.23) --
	(207.07,183.22) --
	(207.02,183.21) --
	(207.00,183.21) --
	(206.97,183.22) --
	(206.94,183.23) --
	(206.91,183.25) --
	(206.89,183.27) --
	(206.82,183.36) --
	(206.79,183.41) --
	(206.78,183.45) --
	(206.76,183.47) --
	(206.71,183.52) --
	(206.71,183.54) --
	(206.64,183.57) --
	(206.60,183.58) --
	(206.57,183.58) --
	(206.50,183.57) --
	(206.44,183.56) --
	(206.33,183.52) --
	(206.31,183.51) --
	(206.29,183.50) --
	(206.29,183.45) --
	(206.27,183.39) --
	(206.27,183.36) --
	(206.25,183.36) --
	(206.20,183.33) --
	(206.16,183.31) --
	(206.11,183.27) --
	(206.08,183.25) --
	(206.03,183.20) --
	(205.95,183.14) --
	(205.88,183.11) --
	(205.83,183.09) --
	(205.79,183.07) --
	(205.74,183.03) --
	(205.73,183.02) --
	(205.71,183.00) --
	(205.67,182.96) --
	(205.64,182.95) --
	(205.61,182.96) --
	(205.56,182.99) --
	(205.49,183.00) --
	(205.46,183.00) --
	(205.44,182.99) --
	(205.38,182.97) --
	(205.36,182.96) --
	(205.30,182.93) --
	(205.28,182.93) --
	(205.24,182.92) --
	(205.20,182.91) --
	(205.18,182.90) --
	(205.12,182.88) --
	(204.97,182.78) --
	(204.94,182.77) --
	(204.91,182.76) --
	(204.84,182.76) --
	(204.77,182.78) --
	(204.63,182.81) --
	(204.59,182.83) --
	(204.54,182.86) --
	(204.47,182.91) --
	(204.41,182.94) --
	(204.32,182.99) --
	(204.28,183.01) --
	(204.21,183.03) --
	(204.16,183.03) --
	(204.11,183.02) --
	(204.05,183.01) --
	(203.90,182.99) --
	(203.83,182.98) --
	(203.67,182.95) --
	(203.62,182.96) --
	(203.58,182.96) --
	(203.46,182.94) --
	(203.43,182.93) --
	(203.35,182.88) --
	(203.30,182.86) --
	(203.22,182.84) --
	(203.18,182.84) --
	(203.14,182.84) --
	(203.05,182.87) --
	(203.01,182.88) --
	(202.99,182.90) --
	(202.86,182.96) --
	(202.72,183.03) --
	(202.68,183.05) --
	(202.58,183.13) --
	(202.56,183.15) --
	(202.50,183.24) --
	(202.48,183.29) --
	(202.43,183.35) --
	(202.41,183.37) --
	(202.35,183.39) --
	(202.30,183.41) --
	(202.21,183.44) --
	(202.02,183.49) --
	(201.99,183.50) --
	(201.97,183.49) --
	(201.95,183.48) --
	(201.91,183.46) --
	(201.89,183.43) --
	(201.87,183.41) --
	(201.86,183.39) --
	(201.84,183.36) --
	(201.81,183.32) --
	(201.73,183.24) --
	(201.70,183.22) --
	(201.62,183.14) --
	(201.54,183.11) --
	(201.53,183.10) --
	(201.50,183.07) --
	(201.47,183.06) --
	(201.44,183.06) --
	(201.39,183.04) --
	(201.33,183.00) --
	(201.32,182.99) --
	(201.30,182.96) --
	(201.28,182.93) --
	(201.27,182.91) --
	(201.23,182.88) --
	(201.20,182.83) --
	(201.15,182.74) --
	(201.14,182.70) --
	(201.13,182.67) --
	(201.13,182.62) --
	(201.14,182.55) --
	(201.15,182.50) --
	(201.21,182.41) --
	(201.28,182.38) --
	(201.29,182.37) --
	(201.29,182.34) --
	(201.29,182.31) --
	(201.28,182.28) --
	(201.29,182.22) --
	(201.29,182.19) --
	(201.30,182.15) --
	(201.30,182.11) --
	(201.29,182.08) --
	(201.28,182.01) --
	(201.21,181.92) --
	(201.19,181.90) --
	(201.17,181.89) --
	(201.15,181.89) --
	(201.13,181.89) --
	(201.10,181.89) --
	(201.07,181.88) --
	(201.02,181.87) --
	(200.96,181.85) --
	(200.92,181.86) --
	(200.81,181.88) --
	(200.78,181.88) --
	(200.74,181.88) --
	(200.72,181.87) --
	(200.68,181.84) --
	(200.64,181.83) --
	(200.59,181.83) --
	(200.56,181.83) --
	(200.50,181.82) --
	(200.43,181.79) --
	(200.40,181.77) --
	(200.35,181.74) --
	(200.34,181.71) --
	(200.33,181.71) --
	(200.31,181.70) --
	(200.28,181.70) --
	(200.24,181.70) --
	(200.22,181.71) --
	(200.19,181.73) --
	(200.16,181.75) --
	(200.14,181.78) --
	(200.12,181.82) --
	(200.11,181.84) --
	(200.09,181.85) --
	(200.07,181.86) --
	(200.06,181.87) --
	(200.05,181.87) --
	(200.00,181.86) --
	(199.98,181.85) --
	(199.96,181.83) --
	(199.93,181.81) --
	(199.80,181.74) --
	(199.78,181.72) --
	(199.76,181.71) --
	(199.74,181.69) --
	(199.73,181.67) --
	(199.72,181.63) --
	(199.70,181.61) --
	(199.69,181.60) --
	(199.68,181.59) --
	(199.61,181.56) --
	(199.56,181.54) --
	(199.53,181.52) --
	(199.49,181.49) --
	(199.48,181.46) --
	(199.47,181.43) --
	(199.47,181.41) --
	(199.48,181.34) --
	(199.49,181.28) --
	(199.46,181.18) --
	(199.46,181.16) --
	(199.46,181.14) --
	(199.47,181.12) --
	(199.50,181.08) --
	(199.54,181.06) --
	(199.56,181.03) --
	(199.58,180.99) --
	(199.60,180.97) --
	(199.63,180.97) --
	(199.68,180.95) --
	(199.70,180.94) --
	(199.77,180.89) --
	(199.83,180.85) --
	(199.89,180.81) --
	(199.91,180.79) --
	(199.91,180.78) --
	(199.92,180.75) --
	(199.92,180.72) --
	(199.89,180.62) --
	(199.87,180.58) --
	(199.87,180.55) --
	(199.89,180.53) --
	(199.91,180.49) --
	(199.90,180.46) --
	(199.90,180.42) --
	(199.87,180.37) --
	(199.85,180.35) --
	(199.82,180.28) --
	(199.83,180.24) --
	(199.81,180.22) --
	(199.79,180.21) --
	(199.78,180.20) --
	(199.73,180.19) --
	(199.69,180.18) --
	(199.63,180.13) --
	(199.60,180.08) --
	(199.58,180.03) --
	(199.57,180.00) --
	(199.57,179.97) --
	(199.56,179.88) --
	(199.57,179.87) --
	(199.58,179.86) --
	(199.62,179.84) --
	(199.67,179.80) --
	(199.70,179.77) --
	(199.75,179.72) --
	(199.75,179.70) --
	(199.75,179.69) --
	(199.73,179.65) --
	(199.73,179.63) --
	(199.73,179.60) --
	(199.73,179.57) --
	(199.70,179.53) --
	(199.69,179.50) --
	(199.69,179.48) --
	(199.71,179.42) --
	(199.71,179.38) --
	(199.71,179.35) --
	(199.69,179.28) --
	(199.68,179.25) --
	(199.68,179.24) --
	(199.72,179.16) --
	(199.73,179.13) --
	(199.72,179.07) --
	(199.70,179.00) --
	(199.69,178.97) --
	(199.64,178.91) --
	(199.63,178.89) --
	(199.62,178.87) --
	(199.62,178.84) --
	(199.58,178.74) --
	(199.55,178.66) --
	(199.54,178.62) --
	(199.52,178.60) --
	(199.50,178.57) --
	(199.39,178.50) --
	(199.37,178.48) --
	(199.35,178.46) --
	(199.32,178.42) --
	(199.31,178.40) --
	(199.30,178.38) --
	(199.33,178.25) --
	(199.33,178.21) --
	(199.32,178.19) --
	(199.31,178.17) --
	(199.29,178.16) --
	(199.23,178.12) --
	(199.19,178.10) --
	(199.16,178.08) --
	(199.08,178.01) --
	(199.05,177.98) --
	(199.03,177.98) --
	(199.01,177.98) --
	(198.99,177.98) --
	(198.94,178.00) --
	(198.90,178.02) --
	(198.87,178.05) --
	(198.84,178.07) --
	(198.79,178.09) --
	(198.72,178.10) --
	(198.71,178.10) --
	(198.69,178.10) --
	(198.68,178.09) --
	(198.66,178.07) --
	(198.64,178.05) --
	(198.62,178.03) --
	(198.59,178.01) --
	(198.56,178.01) --
	(198.53,178.01) --
	(198.49,178.01) --
	(198.45,178.02) --
	(198.39,178.06) --
	(198.34,178.07) --
	(198.31,178.08) --
	(198.26,178.06) --
	(198.24,178.05) --
	(198.22,178.03) --
	(198.15,177.98) --
	(198.10,177.96) --
	(198.05,177.91) --
	(198.03,177.88) --
	(198.04,177.82) --
	(198.04,177.79) --
	(198.07,177.71) --
	(198.10,177.66) --
	(198.11,177.63) --
	(198.09,177.58) --
	(198.04,177.50) --
	(197.96,177.40) --
	(197.96,177.38) --
	(197.97,177.36) --
	(197.99,177.32) --
	(198.02,177.28) --
	(198.10,177.19) --
	(198.12,177.16) --
	(198.11,177.10) --
	(198.05,177.00) --
	(198.02,176.98) --
	(198.00,176.95) --
	(197.97,176.93) --
	(197.89,176.88) --
	(197.82,176.80) --
	(197.80,176.76) --
	(197.73,176.69) --
	(197.70,176.68) --
	(197.60,176.64) --
	(197.55,176.61) --
	(197.45,176.52) --
	(197.43,176.50) --
	(197.42,176.48) --
	(197.42,176.45) --
	(197.44,176.39) --
	(197.44,176.37) --
	(197.43,176.34) --
	(197.42,176.31) --
	(197.42,176.26) --
	(197.43,176.22) --
	(197.45,176.18) --
	(197.47,176.16) --
	(197.50,176.14) --
	(197.53,176.12) --
	(197.56,176.11) --
	(197.58,176.10) --
	(197.63,176.07) --
	(197.64,176.06) --
	(197.63,176.04) --
	(197.63,176.03) --
	(197.59,175.97) --
	(197.60,175.95) --
	(197.60,175.91) --
	(197.61,175.87) --
	(197.64,175.83) --
	(197.65,175.82) --
	(197.66,175.79) --
	(197.65,175.72) --
	(197.65,175.60) --
	(197.64,175.54) --
	(197.64,175.39) --
	(197.61,175.35) --
	(197.55,175.29) --
	(197.54,175.28) --
	(197.49,175.26) --
	(197.31,175.21) --
	(197.23,175.21) --
	(197.21,175.20) --
	(197.16,175.17) --
	(197.14,175.13) --
	(197.15,175.10) --
	(197.15,175.05) --
	(197.14,175.03) --
	(197.14,175.02) --
	(197.12,175.00) --
	(197.11,174.98) --
	(197.12,174.94) --
	(197.12,174.93) --
	(197.15,174.90) --
	(197.18,174.87) --
	(197.23,174.85) --
	(197.28,174.81) --
	(197.29,174.80) --
	(197.30,174.75) --
	(197.29,174.73) --
	(197.23,174.65) --
	(197.22,174.64) --
	(197.22,174.61) --
	(197.22,174.52) --
	(197.20,174.43) --
	(197.19,174.34) --
	(197.20,174.32) --
	(197.22,174.30) --
	(197.23,174.27) --
	(197.23,174.25) --
	(197.22,174.23) --
	(197.22,174.21) --
	(197.22,174.17) --
	(197.20,174.13) --
	(197.20,174.09) --
	(197.19,174.09) --
	(197.18,174.09) --
	(197.18,174.09) --
	(197.17,174.09) --
	(197.17,174.09) --
	(197.16,174.09) --
	(197.16,174.09) --
	(197.15,174.09) --
	(197.15,174.09) --
	(197.14,174.09) --
	(197.14,174.09) --
	(197.13,174.09) --
	(197.13,174.09) --
	(197.12,174.09) --
	(197.10,174.04) --
	(197.13,174.02) --
	(197.15,173.99) --
	(197.16,173.96) --
	(197.18,173.92) --
	(197.22,173.88) --
	(197.24,173.87) --
	(197.27,173.87) --
	(197.29,173.89) --
	(197.33,173.90) --
	(197.39,173.92) --
	(197.43,173.93) --
	(197.47,173.92) --
	(197.60,173.89) --
	(197.66,173.86) --
	(197.68,173.86) --
	(197.69,173.87) --
	(197.73,173.91) --
	(197.78,173.92) --
	(197.89,173.93) --
	(197.93,173.94) --
	(197.96,173.95) --
	(197.99,173.98) --
	(198.06,174.02) --
	(198.10,174.05) --
	(198.14,174.06) --
	(198.18,174.06) --
	(198.21,174.06) --
	(198.23,174.05) --
	(198.24,174.03) --
	(198.26,173.95) --
	(198.28,173.93) --
	(198.30,173.91) --
	(198.34,173.88) --
	(198.38,173.87) --
	(198.44,173.85) --
	(198.47,173.84) --
	(198.49,173.82) --
	(198.53,173.79) --
	(198.59,173.75) --
	(198.64,173.70) --
	(198.69,173.66) --
	(198.72,173.65) --
	(198.81,173.64) --
	(198.84,173.63) --
	(198.86,173.62) --
	(198.90,173.53) --
	(198.92,173.52) --
	(198.93,173.49) --
	(198.95,173.44) --
	(199.01,173.31) --
	(199.02,173.28) --
	(198.99,173.24) --
	(198.97,173.21) --
	(198.89,173.14) --
	(198.87,173.10) --
	(198.87,173.08) --
	(198.88,173.06) --
	(198.90,173.02) --
	(198.92,172.99) --
	(198.97,172.96) --
	(199.03,172.94) --
	(199.06,172.93) --
	(199.09,172.92) --
	(199.13,172.87) --
	(199.16,172.82) --
	(199.18,172.80) --
	(199.22,172.77) --
	(199.26,172.75) --
	(199.37,172.74) --
	(199.39,172.73) --
	(199.41,172.72) --
	(199.42,172.68) --
	(199.41,172.64) --
	(199.41,172.59) --
	(199.42,172.56) --
	(199.43,172.53) --
	(199.44,172.51) --
	(199.48,172.50) --
	(199.52,172.49) --
	(199.57,172.48) --
	(199.69,172.42) --
	(199.76,172.38) --
	(199.78,172.36) --
	(199.79,172.34) --
	(199.80,172.31) --
	(199.89,172.18) --
	(199.99,172.14) --
	(200.01,172.11) --
	(200.04,172.09) --
	(200.08,172.07) --
	(200.14,172.04) --
	(200.21,172.02) --
	(200.25,172.00) --
	(200.29,171.96) --
	(200.36,171.89) --
	(200.40,171.87) --
	(200.43,171.87) --
	(200.51,171.88) --
	(200.54,171.88) --
	(200.60,171.86) --
	(200.62,171.86) --
	(200.64,171.87) --
	(200.66,171.89) --
	(200.68,171.91) --
	(200.69,171.92) --
	(200.70,171.93) --
	(200.74,171.93) --
	(200.78,171.93) --
	(200.80,171.94) --
	(200.91,171.97) --
	(200.94,171.99) --
	(200.98,172.00) --
	(201.03,171.99) --
	(201.10,171.98) --
	(201.14,171.96) --
	(201.16,171.96) --
	(201.17,171.96) --
	(201.17,171.97) --
	(201.17,172.00) --
	(201.18,172.03) --
	(201.19,172.08) --
	(201.24,172.13) --
	(201.28,172.20) --
	(201.32,172.24) --
	(201.36,172.25) --
	(201.40,172.27) --
	(201.43,172.28) --
	(201.48,172.30) --
	(201.52,172.31) --
	(201.53,172.31) --
	(201.55,172.31) --
	(201.57,172.30) --
	(201.59,172.28) --
	(201.62,172.21) --
	(201.64,172.13) --
	(201.65,172.11) --
	(201.68,172.08) --
	(201.70,172.06) --
	(201.71,172.04) --
	(201.73,172.00) --
	(201.76,171.98) --
	(201.79,171.97) --
	(201.87,171.94) --
	(201.89,171.93) --
	(201.91,171.91) --
	(201.93,171.89) --
	(201.97,171.87) --
	(201.99,171.86) --
	(202.02,171.87) --
	(202.04,171.88) --
	(202.08,171.88) --
	(202.15,171.88) --
	(202.18,171.90) --
	(202.21,171.93) --
	(202.21,171.98) --
	(202.21,172.03) --
	(202.23,172.06) --
	(202.26,172.11) --
	(202.30,172.14) --
	(202.35,172.14) --
	(202.39,172.13) --
	(202.53,172.13) --
	(202.57,172.11) --
	(202.59,172.10) --
	(202.60,172.07) --
	(202.61,172.01) --
	(202.61,171.97) --
	(202.62,171.89) --
	(202.63,171.83) --
	(202.64,171.80) --
	(202.66,171.76) --
	(202.67,171.75) --
	(202.70,171.74) --
	(202.74,171.74) --
	(202.78,171.75) --
	(202.83,171.78) --
	(202.86,171.80) --
	(202.90,171.84) --
	(202.95,171.88) --
	(203.00,171.90) --
	(203.06,171.90) --
	(203.10,171.90) --
	(203.14,171.89) --
	(203.17,171.87) --
	(203.23,171.83) --
	(203.29,171.80) --
	(203.32,171.80) --
	(203.36,171.80) --
	(203.40,171.80) --
	(203.45,171.80) --
	(203.52,171.81) --
	(203.64,171.88) --
	(203.69,171.89) --
	(203.86,171.88) --
	(203.88,171.87) --
	(203.89,171.86) --
	(203.90,171.85) --
	(203.91,171.79) --
	(203.92,171.71) --
	(203.93,171.66) --
	(203.99,171.52) --
	(204.00,171.51) --
	(204.01,171.48) --
	(204.00,171.46) --
	(203.99,171.44) --
	(203.98,171.43) --
	(203.90,171.37) --
	(203.89,171.35) --
	(203.88,171.24) --
	(203.85,171.19) --
	(203.83,171.18) --
	(203.78,171.17) --
	(203.73,171.15) --
	(203.70,171.15) --
	(203.65,171.14) --
	(203.64,171.13) --
	(203.58,171.02) --
	(203.55,170.91) --
	(203.54,170.90) --
	(203.52,170.89) --
	(203.49,170.89) --
	(203.44,170.88) --
	(203.38,170.85) --
	(203.33,170.81) --
	(203.30,170.76) --
	(203.26,170.67) --
	(203.25,170.64) --
	(203.25,170.62) --
	(203.26,170.55) --
	(203.26,170.52) --
	(203.26,170.50) --
	(203.25,170.50) --
	(203.22,170.49) --
	(203.20,170.49) --
	(203.12,170.49) --
	(203.06,170.49) --
	(202.92,170.46) --
	(202.89,170.45) --
	(202.89,170.43) --
	(202.88,170.40) --
	(202.89,170.32) --
	(202.89,170.28) --
	(202.87,170.25) --
	(202.84,170.22) --
	(202.77,170.13) --
	(202.75,170.11) --
	(202.66,170.09) --
	(202.56,170.03) --
	(202.53,170.01) --
	(202.51,169.99) --
	(202.49,169.95) --
	(202.47,169.89) --
	(202.46,169.85) --
	(202.44,169.79) --
	(202.44,169.77) --
	(202.42,169.76) --
	(202.39,169.75) --
	(202.21,169.73) --
	(202.16,169.73) --
	(202.11,169.74) --
	(202.07,169.76) --
	(202.01,169.80) --
	(201.91,169.86) --
	(201.86,169.87) --
	(201.83,169.87) --
	(201.80,169.86) --
	(201.75,169.83) --
	(201.66,169.78) --
	(201.64,169.76) --
	(201.63,169.73) --
	(201.63,169.70) --
	(201.66,169.63) --
	(201.64,169.55) --
	(201.63,169.53) --
	(201.57,169.48) --
	(201.49,169.42) --
	(201.48,169.37) --
	(201.48,169.35) --
	(201.49,169.32) --
	(201.51,169.30) --
	(201.55,169.24) --
	(201.63,169.17) --
	(201.65,169.14) --
	(201.66,169.11) --
	(201.67,169.05) --
	(201.68,169.02) --
	(201.68,168.98) --
	(201.67,168.94) --
	(201.66,168.91) --
	(201.62,168.87) --
	(201.60,168.82) --
	(201.61,168.79) --
	(201.62,168.77) --
	(201.67,168.70) --
	(201.69,168.68) --
	(201.69,168.66) --
	(201.69,168.64) --
	(201.67,168.59) --
	(201.66,168.52) --
	(201.64,168.46) --
	(201.61,168.40) --
	(201.59,168.35) --
	(201.58,168.32) --
	(201.59,168.31) --
	(201.62,168.26) --
	(201.62,168.25) --
	(201.63,168.17) --
	(201.64,168.14) --
	(201.64,168.11) --
	(201.64,168.09) --
	(201.61,168.05) --
	(201.57,168.01) --
	(201.55,167.97) --
	(201.55,167.94) --
	(201.57,167.87) --
	(201.58,167.83) --
	(201.57,167.79) --
	(201.58,167.74) --
	(201.55,167.67) --
	(201.55,167.64) --
	(201.58,167.58) --
	(201.61,167.54) --
	(201.62,167.51) --
	(201.62,167.46) --
	(201.62,167.41) --
	(201.61,167.37) --
	(201.57,167.33) --
	(201.54,167.30) --
	(201.50,167.26) --
	(201.48,167.21) --
	(201.47,167.19) --
	(201.47,167.17) --
	(201.47,167.14) --
	(201.47,167.12) --
	(201.45,167.09) --
	(201.45,167.05) --
	(201.41,166.97) --
	(201.39,166.94) --
	(201.37,166.94) --
	(201.33,166.94) --
	(201.27,166.94) --
	(201.15,166.93) --
	(201.09,166.92) --
	(201.03,166.92) --
	(200.97,166.92) --
	(200.94,166.90) --
	(200.92,166.89) --
	(200.85,166.82) --
	(200.80,166.78) --
	(200.77,166.72) --
	(200.75,166.68) --
	(200.75,166.65) --
	(200.76,166.63) --
	(200.79,166.60) --
	(200.80,166.57) --
	(200.83,166.54) --
	(200.91,166.45) --
	(200.96,166.42) --
	(201.00,166.39) --
	(201.05,166.35) --
	(201.09,166.30) --
	(201.12,166.20) --
	(201.12,166.15) --
	(201.12,166.12) --
	(201.12,166.03) --
	(201.14,166.01) --
	(201.16,165.98) --
	(201.20,165.94) --
	(201.23,165.90) --
	(201.26,165.85) --
	(201.29,165.77) --
	(201.31,165.61) --
	(201.30,165.56) --
	(201.30,165.53) --
	(201.31,165.51) --
	(201.33,165.49) --
	(201.39,165.46) --
	(201.49,165.39) --
	(201.56,165.33) --
	(201.58,165.32) --
	(201.59,165.32) --
	(201.62,165.34) --
	(201.64,165.36) --
	(201.67,165.38) --
	(201.71,165.40) --
	(201.74,165.40) --
	(201.80,165.41) --
	(201.84,165.41) --
	(201.89,165.40) --
	(201.96,165.37) --
	(201.99,165.34) --
	(202.00,165.32) --
	(201.99,165.26) --
	(202.00,165.24) --
	(202.01,165.20) --
	(202.02,165.20) --
	(202.05,165.20) --
	(202.07,165.20) --
	(202.11,165.22) --
	(202.14,165.23) --
	(202.19,165.23) --
	(202.24,165.23) --
	(202.30,165.23) --
	(202.34,165.21) --
	(202.37,165.20) --
	(202.47,165.20) --
	(202.49,165.19) --
	(202.50,165.18) --
	(202.50,165.17) --
	(202.47,165.13) --
	(202.45,165.11) --
	(202.42,165.08) --
	(202.39,165.03) --
	(202.38,165.00) --
	(202.37,164.92) --
	(202.38,164.89) --
	(202.39,164.85) --
	(202.42,164.81) --
	(202.46,164.77) --
	(202.51,164.74) --
	(202.56,164.73) --
	(202.65,164.72) --
	(202.74,164.71) --
	(202.83,164.67) --
	(202.90,164.62) --
	(202.92,164.59) --
	(202.92,164.53) --
	(202.90,164.41) --
	(202.90,164.37) --
	(202.91,164.34) --
	(202.93,164.31) --
	(202.99,164.27) --
	(203.02,164.23) --
	(203.03,164.21) --
	(203.03,164.20) --
	(203.03,164.18) --
	(203.01,164.15) --
	(203.01,164.13) --
	(203.01,164.10) --
	(203.02,164.08) --
	(203.04,164.07) --
	(203.06,164.04) --
	(203.11,163.98) --
	(203.16,163.92) --
	(203.18,163.87) --
	(203.20,163.85) --
	(203.22,163.81) --
	(203.22,163.72) --
	(203.23,163.67) --
	(203.24,163.64) --
	(203.26,163.60) --
	(203.31,163.56) --
	(203.36,163.50) --
	(203.39,163.46) --
	(203.40,163.43) --
	(203.39,163.41) --
	(203.38,163.40) --
	(203.36,163.39) --
	(203.34,163.36) --
	(203.33,163.32) --
	(203.34,163.28) --
	(203.34,163.24) --
	(203.34,163.21) --
	(203.33,163.19) --
	(203.32,163.18) --
	(203.27,163.13) --
	(203.25,163.10) --
	(203.23,163.04) --
	(203.21,163.00) --
	(203.17,162.93) --
	(203.12,162.84) --
	(203.12,162.80) --
	(203.13,162.76) --
	(203.14,162.71) --
	(203.15,162.66) --
	(203.15,162.60) --
	(203.17,162.58) --
	(203.18,162.57) --
	(203.20,162.55) --
	(203.32,162.49) --
	(203.36,162.46) --
	(203.42,162.37) --
	(203.44,162.33) --
	(203.44,162.28) --
	(203.43,162.26) --
	(203.42,162.23) --
	(203.40,162.19) --
	(203.37,162.14) --
	(203.35,162.10) --
	(203.33,162.04) --
	(203.34,162.01) --
	(203.35,161.98) --
	(203.43,161.88) --
	(203.49,161.83) --
	(203.52,161.82) --
	(203.53,161.80) --
	(203.58,161.72) --
	(203.60,161.70) --
	(203.64,161.67) --
	(203.67,161.65) --
	(203.71,161.64) --
	(203.76,161.63) --
	(203.79,161.64) --
	(203.87,161.71) --
	(203.91,161.72) --
	(203.95,161.73) --
	(203.97,161.73) --
	(203.99,161.72) --
	(204.00,161.71) --
	(204.03,161.68) --
	(204.06,161.66) --
	(204.10,161.65) --
	(204.15,161.65) --
	(204.17,161.64) --
	(204.19,161.62) --
	(204.21,161.59) --
	(204.26,161.51) --
	(204.29,161.47) --
	(204.34,161.36) --
	(204.36,161.29) --
	(204.40,161.22) --
	(204.41,161.21) --
	(204.43,161.20) --
	(204.47,161.18) --
	(204.57,161.17) --
	(204.66,161.14) --
	(204.70,161.11) --
	(204.71,161.08) --
	(204.73,161.01) --
	(204.74,160.98) --
	(204.75,160.93) --
	(204.76,160.85) --
	(204.77,160.82) --
	(204.81,160.76) --
	(204.87,160.69) --
	(204.92,160.65) --
	(204.95,160.64) --
	(204.99,160.64) --
	(205.03,160.64) --
	(205.07,160.64) --
	(205.13,160.66) --
	(205.16,160.66) --
	(205.17,160.65) --
	(205.18,160.63) --
	(205.18,160.61) --
	(205.18,160.58) --
	(205.18,160.56) --
	(205.19,160.52) --
	(205.25,160.39) --
	(205.31,160.35) --
	(205.35,160.34) --
	(205.41,160.33) --
	(205.58,160.34) --
	(205.64,160.33) --
	(205.71,160.32) --
	(205.77,160.30) --
	(205.80,160.29) --
	(205.81,160.28) --
	(205.82,160.26) --
	(205.82,160.24) --
	(205.82,160.23) --
	(205.80,160.21) --
	(205.78,160.19) --
	(205.75,160.15) --
	(205.73,160.11) --
	(205.72,160.09) --
	(205.75,160.00) --
	(205.77,159.90) --
	(205.79,159.81) --
	(205.81,159.74) --
	(205.81,159.71) --
	(205.81,159.66) --
	(205.81,159.61) --
	(205.81,159.57) --
	(205.80,159.55) --
	(205.79,159.53) --
	(205.75,159.48) --
	(205.71,159.41) --
	(205.67,159.35) --
	(205.67,159.31) --
	(205.67,159.29) --
	(205.67,159.27) --
	(205.68,159.25) --
	(205.70,159.22) --
	(205.73,159.18) --
	(205.79,159.11) --
	(205.81,159.09) --
	(205.88,159.03) --
	(205.90,159.01) --
	(205.91,158.98) --
	(205.92,158.94) --
	(205.93,158.91) --
	(205.94,158.88) --
	(205.97,158.87) --
	(206.01,158.84) --
	(206.03,158.82) --
	(206.03,158.80) --
	(206.03,158.78) --
	(206.03,158.75) --
	(206.00,158.69) --
	(206.00,158.65) --
	(206.00,158.62) --
	(206.02,158.55) --
	(206.03,158.51) --
	(206.06,158.46) --
	(206.08,158.42) --
	(206.08,158.41) --
	(206.07,158.39) --
	(206.06,158.38) --
	(205.99,158.35) --
	(205.95,158.33) --
	(205.91,158.28) --
	(205.87,158.21) --
	(205.88,158.15) --
	(205.86,158.13) --
	(205.84,158.10) --
	(205.79,158.06) --
	(205.73,158.04) --
	(205.69,158.03) --
	(205.66,158.01) --
	(205.62,158.03) --
	(205.57,158.05) --
	(205.52,158.07) --
	(205.46,158.08) --
	(205.43,158.08) --
	(205.41,158.09) --
	(205.34,158.13) --
	(205.32,158.15) --
	(205.26,158.16) --
	(205.23,158.16) --
	(205.19,158.15) --
	(205.17,158.13) --
	(205.11,158.10) --
	(205.04,158.08) --
	(204.98,158.06) --
	(204.92,158.04) --
	(204.75,157.96) --
	(204.72,157.95) --
	(204.69,157.94) --
	(204.63,157.93) --
	(204.60,157.93) --
	(204.54,157.93) --
	(204.50,157.93) --
	(204.45,157.92) --
	(204.42,157.91) --
	(204.38,157.91) --
	(204.33,157.92) --
	(204.30,157.93) --
	(204.24,157.92) --
	(204.16,157.90) --
	(204.14,157.88) --
	(204.11,157.86) --
	(204.10,157.84) --
	(204.08,157.82) --
	(204.07,157.80) --
	(204.06,157.77) --
	(204.06,157.75) --
	(204.07,157.71) --
	(204.06,157.66) --
	(204.04,157.62) --
	(203.96,157.51) --
	(203.92,157.48) --
	(203.90,157.45) --
	(203.89,157.42) --
	(203.88,157.35) --
	(203.85,157.24) --
	(203.82,157.18) --
	(203.79,157.09) --
	(203.76,156.94) --
	(203.71,156.78) --
	(203.67,156.66) --
	(203.65,156.62) --
	(203.63,156.60) --
	(203.60,156.59) --
	(203.58,156.58) --
	(203.50,156.58) --
	(203.48,156.58) --
	(203.44,156.59) --
	(203.37,156.61) --
	(203.28,156.62) --
	(203.22,156.62) --
	(203.16,156.61) --
	(203.10,156.59) --
	(203.07,156.58) --
	(203.02,156.58) --
	(202.86,156.61) --
	(202.80,156.63) --
	(202.77,156.64) --
	(202.75,156.64) --
	(202.71,156.64) --
	(202.67,156.65) --
	(202.60,156.66) --
	(202.56,156.66) --
	(202.54,156.65) --
	(202.49,156.61) --
	(202.44,156.51) --
	(202.40,156.48) --
	(202.37,156.46) --
	(202.22,156.40) --
	(202.20,156.40) --
	(202.16,156.41) --
	(202.05,156.45) --
	(201.95,156.47) --
	(201.91,156.48) --
	(201.87,156.48) --
	(201.84,156.48) --
	(201.82,156.48) --
	(201.81,156.47) --
	(201.80,156.45) --
	(201.78,156.41) --
	(201.76,156.39) --
	(201.74,156.37) --
	(201.71,156.36) --
	(201.68,156.36) --
	(201.66,156.36) --
	(201.56,156.37) --
	(201.50,156.39) --
	(201.44,156.41) --
	(201.42,156.41) --
	(201.38,156.40) --
	(201.32,156.38) --
	(201.29,156.38) --
	(201.24,156.39) --
	(201.20,156.41) --
	(201.15,156.44) --
	(201.12,156.47) --
	(201.07,156.52) --
	(201.01,156.57) --
	(200.98,156.59) --
	(200.94,156.60) --
	(200.90,156.62) --
	(200.82,156.68) --
	(200.80,156.72) --
	(200.77,156.76) --
	(200.73,156.78) --
	(200.64,156.83) --
	(200.61,156.85) --
	(200.57,156.91) --
	(200.55,156.95) --
	(200.47,157.15) --
	(200.44,157.18) --
	(200.42,157.19) --
	(200.38,157.19) --
	(200.31,157.20) --
	(200.28,157.21) --
	(200.24,157.23) --
	(200.21,157.27) --
	(200.20,157.30) --
	(200.18,157.32) --
	(200.16,157.32) --
	(200.12,157.31) --
	(200.07,157.28) --
	(200.01,157.25) --
	(199.96,157.24) --
	(199.92,157.23) --
	(199.87,157.22) --
	(199.83,157.22) --
	(199.78,157.22) --
	(199.74,157.21) --
	(199.70,157.21) --
	(199.66,157.20) --
	(199.59,157.20) --
	(199.56,157.20) --
	(199.53,157.21) --
	(199.48,157.22) --
	(199.45,157.21) --
	(199.42,157.13) --
	(199.39,157.08) --
	(199.36,157.04) --
	(199.35,157.01) --
	(199.35,156.98) --
	(199.34,156.95) --
	(199.31,156.92) --
	(199.27,156.90) --
	(199.25,156.88) --
	(199.24,156.87) --
	(199.24,156.85) --
	(199.23,156.82) --
	(199.22,156.79) --
	(199.20,156.76) --
	(199.16,156.73) --
	(199.13,156.71) --
	(199.12,156.70) --
	(199.10,156.66) --
	(199.09,156.61) --
	(199.07,156.57) --
	(199.07,156.53) --
	(199.04,156.45) --
	(199.00,156.39) --
	(198.96,156.35) --
	(198.93,156.31) --
	(198.89,156.27) --
	(198.80,156.06) --
	(198.77,156.02) --
	(198.75,155.99) --
	(198.69,155.90) --
	(198.68,155.89) --
	(198.66,155.88) --
	(198.62,155.88) --
	(198.58,155.86) --
	(198.55,155.85) --
	(198.49,155.76) --
	(198.45,155.73) --
	(198.42,155.72) --
	(198.36,155.71) --
	(198.34,155.70) --
	(198.32,155.69) --
	(198.21,155.69) --
	(198.17,155.68) --
	(198.14,155.66) --
	(198.13,155.64) --
	(198.12,155.61) --
	(198.12,155.59) --
	(198.13,155.57) --
	(198.16,155.54) --
	(198.19,155.52) --
	(198.20,155.51) --
	(198.21,155.49) --
	(198.21,155.46) --
	(198.20,155.44) --
	(198.18,155.40) --
	(198.13,155.34) --
	(198.11,155.30) --
	(198.09,155.27) --
	(198.10,155.24) --
	(198.11,155.20) --
	(198.12,155.16) --
	(198.17,155.12) --
	(198.20,155.10) --
	(198.22,155.08) --
	(198.22,155.06) --
	(198.22,155.04) --
	(198.19,155.01) --
	(198.16,154.98) --
	(198.12,154.95) --
	(198.09,154.92) --
	(198.05,154.90) --
	(197.98,154.86) --
	(197.93,154.84) --
	(197.92,154.82) --
	(197.88,154.80) --
	(197.86,154.80) --
	(197.84,154.78) --
	(197.81,154.75) --
	(197.78,154.72) --
	(197.73,154.69) --
	(197.71,154.65) --
	(197.69,154.59) --
	(197.67,154.53) --
	(197.65,154.49) --
	(197.57,154.42) --
	(197.50,154.28) --
	(197.47,154.24) --
	(197.42,154.15) --
	(197.42,154.11) --
	(197.41,154.10) --
	(197.40,154.09) --
	(197.39,154.10) --
	(197.37,154.10) --
	(197.31,154.11) --
	(197.26,154.11) --
	(197.21,154.10) --
	(197.17,154.07) --
	(197.16,154.05) --
	(197.17,154.01) --
	(197.20,153.97) --
	(197.21,153.94) --
	(197.22,153.93) --
	(197.22,153.90) --
	(197.22,153.88) --
	(197.19,153.83) --
	(197.19,153.80) --
	(197.17,153.78) --
	(197.15,153.75) --
	(197.14,153.72) --
	(197.12,153.69) --
	(197.10,153.65) --
	(197.10,153.62) --
	(197.10,153.61) --
	(197.11,153.56) --
	(197.10,153.53) --
	(197.09,153.51) --
	(197.08,153.47) --
	(197.09,153.43) --
	(197.11,153.39) --
	(197.14,153.37) --
	(197.16,153.34) --
	(197.18,153.32) --
	(197.17,153.29) --
	(197.16,153.27) --
	(197.12,153.24) --
	(197.09,153.19) --
	(197.07,153.17) --
	(197.05,153.16) --
	(197.04,153.13) --
	(197.04,153.10) --
	(197.04,153.06) --
	(197.02,153.03) --
	(196.93,152.95) --
	(196.89,152.88) --
	(196.86,152.77) --
	(196.80,152.68) --
	(196.78,152.66) --
	(196.76,152.65) --
	(196.74,152.64) --
	(196.71,152.64) --
	(196.68,152.65) --
	(196.66,152.65) --
	(196.60,152.65) --
	(196.56,152.63) --
	(196.47,152.61) --
	(196.45,152.61) --
	(196.42,152.61) --
	(196.40,152.61) --
	(196.37,152.61) --
	(196.34,152.60) --
	(196.33,152.58) --
	(196.28,152.50) --
	(196.25,152.46) --
	(196.18,152.43) --
	(196.09,152.37) --
	(196.06,152.34) --
	(196.04,152.32) --
	(196.04,152.29) --
	(196.03,152.27) --
	(196.02,152.25) --
	(196.01,152.24) --
	(195.97,152.19) --
	(195.95,152.16) --
	(195.93,152.15) --
	(195.91,152.14) --
	(195.88,152.14) --
	(195.85,152.14) --
	(195.79,152.13) --
	(195.76,152.11) --
	(195.75,152.08) --
	(195.72,151.98) --
	(195.70,151.96) --
	(195.68,151.94) --
	(195.64,151.93) --
	(195.62,151.92) --
	(195.60,151.90) --
	(195.56,151.84) --
	(195.52,151.81) --
	(195.48,151.79) --
	(195.42,151.76) --
	(195.38,151.74) --
	(195.36,151.72) --
	(195.35,151.69) --
	(195.34,151.67) --
	(195.35,151.61) --
	(195.36,151.59) --
	(195.36,151.57) --
	(195.36,151.55) --
	(195.35,151.54) --
	(195.33,151.53) --
	(195.30,151.53) --
	(195.25,151.53) --
	(195.17,151.51) --
	(195.15,151.51) --
	(195.06,151.50) --
	(195.03,151.49) --
	(195.01,151.47) --
	(195.00,151.45) --
	(194.98,151.43) --
	(194.93,151.41) --
	(194.87,151.40) --
	(194.83,151.40) --
	(194.79,151.40) --
	(194.73,151.40) --
	(194.68,151.40) --
	(194.65,151.40) --
	(194.52,151.40) --
	(194.48,151.41) --
	(194.44,151.43) --
	(194.41,151.45) --
	(194.38,151.48) --
	(194.36,151.50) --
	(194.30,151.54) --
	(194.24,151.56) --
	(194.23,151.57) --
	(194.19,151.61) --
	(194.16,151.64) --
	(194.08,151.67) --
	(194.04,151.67) --
	(193.95,151.66) --
	(193.85,151.67) --
	(193.83,151.68) --
	(193.79,151.68) --
	(193.72,151.68) --
	(193.67,151.68) --
	(193.59,151.70) --
	(193.55,151.72) --
	(193.51,151.73) --
	(193.49,151.73) --
	(193.47,151.73) --
	(193.44,151.69) --
	(193.43,151.65) --
	(193.41,151.62) --
	(193.37,151.55) --
	(193.35,151.54) --
	(193.32,151.53) --
	(193.30,151.52) --
	(193.27,151.53) --
	(193.25,151.54) --
	(193.20,151.54) --
	(193.14,151.53) --
	(193.11,151.51) --
	(193.10,151.48) --
	(193.08,151.46) --
	(193.03,151.43) --
	(192.98,151.38) --
	(192.97,151.36) --
	(192.96,151.33) --
	(192.96,151.29) --
	(192.95,151.24) --
	(192.95,151.22) --
	(192.93,151.20) --
	(192.91,151.19) --
	(192.90,151.19) --
	(192.84,151.18) --
	(192.79,151.17) --
	(192.74,151.13) --
	(192.71,151.08) --
	(192.68,150.95) --
	(192.65,150.89) --
	(192.61,150.84) --
	(192.55,150.78) --
	(192.48,150.74) --
	(192.44,150.70) --
	(192.38,150.61) --
	(192.35,150.58) --
	(192.29,150.55) --
	(192.24,150.51) --
	(192.20,150.47) --
	(192.13,150.34) --
	(192.11,150.30) --
	(192.09,150.26) --
	(192.06,150.22) --
	(192.03,150.19) --
	(192.00,150.01) --
	(191.99,149.94) --
	(191.94,149.87) --
	(191.93,149.84) --
	(191.94,149.81) --
	(191.97,149.77) --
	(192.02,149.72) --
	(192.07,149.68) --
	(192.12,149.66) --
	(192.14,149.65) --
	(192.18,149.64) --
	(192.28,149.58) --
	(192.36,149.54) --
	(192.43,149.52) --
	(192.47,149.50) --
	(192.48,149.49) --
	(192.55,149.45) --
	(192.59,149.42) --
	(192.60,149.42) --
	(192.60,149.42) --
	(192.61,149.42) --
	(192.61,149.42) --
	(192.62,149.42) --
	(192.62,149.42) --
	(192.63,149.42) --
	(192.63,149.42) --
	(192.64,149.42) --
	(192.64,149.42) --
	(192.65,149.42) --
	(192.65,149.42) --
	(192.65,149.42) --
	(192.66,149.42) --
	(192.66,149.42) --
	(192.67,149.42) --
	(192.67,149.42) --
	(192.68,149.42) --
	(192.68,149.42) --
	(192.69,149.42) --
	(192.69,149.42) --
	(192.70,149.42) --
	(192.70,149.42) --
	(192.71,149.42) --
	(192.71,149.42) --
	(192.72,149.42) --
	(192.72,149.42) --
	(192.73,149.42) --
	(192.73,149.42) --
	(192.74,149.42) --
	(192.74,149.42) --
	(192.75,149.42) --
	(192.75,149.42) --
	(192.76,149.42) --
	(192.76,149.42) --
	(192.77,149.42) --
	(192.77,149.42) --
	(192.78,149.42) --
	(192.78,149.42) --
	(192.79,149.42) --
	(192.79,149.42) --
	(192.80,149.42) --
	(192.80,149.42) --
	(192.81,149.42) --
	(192.81,149.42) --
	(192.82,149.42) --
	(192.82,149.42) --
	(192.83,149.42) --
	(192.83,149.42) --
	(192.84,149.42) --
	(192.84,149.42) --
	(192.85,149.42) --
	(192.85,149.42) --
	(192.86,149.42) --
	(192.86,149.42) --
	(192.87,149.42) --
	(192.87,149.42) --
	(192.88,149.42) --
	(192.88,149.42) --
	(192.89,149.42) --
	(192.89,149.42) --
	(192.90,149.42) --
	(192.90,149.42) --
	(192.91,149.42) --
	(192.91,149.42) --
	(192.92,149.42) --
	(192.92,149.42) --
	(192.93,149.42) --
	(192.93,149.42) --
	(192.94,149.42) --
	(192.94,149.42) --
	(192.95,149.42) --
	(192.95,149.42) --
	(192.96,149.42) --
	(192.96,149.42) --
	(192.97,149.42) --
	(192.97,149.42) --
	(192.98,149.42) --
	(192.98,149.42) --
	(192.99,149.42) --
	(192.99,149.42) --
	(193.00,149.42) --
	(193.00,149.42) --
	(193.01,149.42) --
	(192.98,149.34) --
	(192.88,149.23) --
	(192.82,149.16) --
	(192.71,149.09) --
	(192.64,149.05) --
	(192.54,149.00) --
	(192.39,148.92) --
	(192.25,148.84) --
	(192.15,148.78) --
	(192.06,148.71) --
	(192.03,148.68) --
	(191.97,148.61) --
	(191.81,148.42) --
	(191.76,148.36) --
	(191.72,148.32) --
	(191.67,148.31) --
	(191.60,148.31) --
	(191.45,148.35) --
	(191.27,148.42) --
	(191.21,148.44) --
	(191.18,148.45) --
	(191.15,148.45) --
	(191.10,148.44) --
	(191.08,148.43) --
	(191.05,148.42) --
	(190.98,148.35) --
	(190.85,148.22) --
	(190.81,148.16) --
	(190.75,148.11) --
	(190.67,147.99) --
	(190.64,147.95) --
	(190.57,147.88) --
	(190.55,147.92) --
	(190.53,147.95) --
	(190.47,148.01) --
	(190.45,148.02) --
	(190.33,148.09) --
	(190.25,148.12) --
	(190.22,148.14) --
	(190.21,148.16) --
	(190.21,148.18) --
	(190.21,148.19) --
	(190.23,148.27) --
	(190.23,148.30) --
	(190.24,148.34) --
	(190.23,148.39) --
	(190.22,148.42) --
	(190.18,148.44) --
	(190.05,148.46) --
	(190.01,148.47) --
	(189.99,148.49) --
	(189.96,148.52) --
	(189.92,148.55) --
	(189.85,148.61) --
	(189.82,148.62) --
	(189.78,148.63) --
	(189.75,148.63) --
	(189.73,148.63) --
	(189.69,148.62) --
	(189.65,148.61) --
	(189.62,148.60) --
	(189.60,148.60) --
	(189.49,148.60) --
	(189.33,148.60) --
	(189.26,148.59) --
	(189.18,148.57) --
	(189.09,148.55) --
	(189.03,148.53) --
	(188.96,148.49) --
	(188.90,148.46) --
	(188.86,148.43) --
	(188.77,148.34) --
	(188.74,148.32) --
	(188.73,148.31) --
	(188.71,148.31) --
	(188.66,148.31) --
	(188.63,148.31) --
	(188.61,148.32) --
	(188.56,148.34) --
	(188.48,148.40) --
	(188.41,148.46) --
	(188.31,148.55) --
	(188.21,148.64) --
	(188.17,148.66) --
	(188.12,148.69) --
	(188.06,148.71) --
	(188.02,148.72) --
	(187.94,148.72) --
	(187.84,148.70) --
	(187.74,148.65) --
	(187.45,148.46) --
	(187.42,148.45) --
	(187.40,148.44) --
	(187.34,148.43) --
	(187.23,148.40) --
	(187.19,148.39) --
	(187.16,148.37) --
	(187.14,148.35) --
	(187.12,148.33) --
	(187.09,148.27) --
	(187.07,148.17) --
	(187.05,148.13) --
	(187.05,148.12) --
	(187.01,148.09) --
	(187.00,148.08) --
	(186.98,148.09) --
	(186.91,148.12) --
	(186.74,148.19) --
	(186.65,148.24) --
	(186.59,148.26) --
	(186.55,148.27) --
	(186.52,148.27) --
	(186.50,148.27) --
	(186.45,148.25) --
	(186.41,148.23) --
	(186.36,148.19) --
	(186.29,148.17) --
	(186.22,148.16) --
	(186.18,148.16) --
	(186.13,148.17) --
	(186.08,148.18) --
	(186.03,148.20) --
	(185.98,148.22) --
	(185.92,148.25) --
	(185.88,148.27) --
	(185.79,148.34) --
	(185.76,148.37) --
	(185.70,148.44) --
	(185.69,148.45) --
	(185.66,148.47) --
	(185.60,148.46) --
	(185.57,148.44) --
	(185.55,148.40) --
	(185.53,148.38) --
	(185.49,148.35) --
	(185.46,148.33) --
	(185.40,148.30) --
	(185.27,148.26) --
	(185.12,148.23) --
	(185.01,148.20) --
	(184.97,148.20) --
	(184.94,148.20) --
	(184.92,148.20) --
	(184.89,148.21) --
	(184.84,148.24) --
	(184.79,148.28) --
	(184.76,148.29) --
	(184.72,148.29) --
	(184.67,148.23) --
	(184.63,148.17) --
	(184.61,148.13) --
	(184.59,148.10) --
	(184.58,148.09) --
	(184.54,148.06) --
	(184.48,148.02) --
	(184.46,148.01) --
	(184.35,147.98) --
	(184.26,147.95) --
	(184.17,147.93) --
	(183.94,147.89) --
	(183.89,147.89) --
	(183.76,147.89) --
	(183.56,147.90) --
	(183.45,147.91) --
	(183.39,147.91) --
	(183.24,147.91) --
	(183.13,147.90) --
	(182.83,147.86) --
	(182.64,147.85) --
	(182.59,147.86) --
	(182.49,147.84) --
	(182.40,147.82) --
	(182.32,147.79) --
	(182.15,147.70) --
	(182.10,147.68) --
	(182.05,147.66) --
	(181.93,147.63) --
	(181.82,147.61) --
	(181.70,147.61) --
	(181.62,147.60) --
	(181.41,147.61) --
	(181.33,147.62) --
	(181.24,147.64) --
	(181.09,147.67) --
	(181.03,147.68) --
	(180.92,147.73) --
	(180.82,147.79) --
	(180.77,147.82) --
	(180.75,147.82) --
	(180.71,147.83) --
	(180.67,147.82) --
	(180.64,147.81) --
	(180.62,147.80) --
	(180.59,147.77) --
	(180.44,147.61) --
	(180.28,147.45) --
	(180.15,147.25) --
	(180.05,147.14) --
	(180.01,147.08) --
	(179.95,147.00) --
	(179.89,146.92) --
	(179.84,146.88) --
	(179.81,146.85) --
	(179.77,146.83) --
	(179.70,146.79) --
	(179.67,146.78) --
	(179.63,146.76) --
	(179.56,146.70) --
	(179.54,146.67) --
	(179.50,146.63) --
	(179.47,146.57) --
	(179.44,146.52) --
	(179.41,146.49) --
	(179.32,146.43) --
	(179.28,146.41) --
	(179.11,146.32) --
	(179.04,146.28) --
	(178.99,146.23) --
	(178.93,146.18) --
	(178.89,146.14) --
	(178.85,146.08) --
	(178.82,146.03) --
	(178.77,145.95) --
	(178.75,145.91) --
	(178.73,145.90) --
	(178.66,145.89) --
	(178.59,145.89) --
	(178.26,145.93) --
	(178.18,145.94) --
	(178.10,145.96) --
	(178.05,145.97) --
	(178.02,145.97) --
	(177.99,145.98) --
	(177.96,145.97) --
	(177.89,145.96) --
	(177.76,145.93) --
	(177.61,145.90) --
	(177.51,145.87) --
	(177.41,145.86) --
	(177.32,145.85) --
	(177.24,145.84) --
	(177.14,145.83) --
	(177.09,145.84) --
	(176.99,145.85) --
	(176.91,145.86) --
	(176.80,145.87) --
	(176.72,145.85) --
	(176.70,145.83) --
	(176.64,145.78) --
	(176.57,145.69) --
	(176.51,145.64) --
	(176.45,145.61) --
	(176.41,145.58) --
	(176.37,145.57) --
	(176.32,145.57) --
	(176.31,145.58) --
	(176.27,145.58) --
	(176.24,145.60) --
	(176.21,145.62) --
	(176.13,145.68) --
	(176.04,145.75) --
	(175.86,145.88) --
	(175.83,145.88) --
	(175.78,145.87) --
	(175.75,145.85) --
	(175.72,145.83) --
	(175.63,145.70) --
	(175.61,145.67) --
	(175.57,145.66) --
	(175.51,145.65) --
	(175.45,145.65) --
	(175.37,145.64) --
	(175.23,145.66) --
	(175.19,145.66) --
	(175.15,145.65) --
	(175.01,145.63) --
	(174.96,145.60) --
	(174.90,145.57) --
	(174.78,145.50) --
	(174.74,145.48) --
	(174.66,145.41) --
	(174.53,145.29) --
	(174.52,145.27) --
	(174.51,145.21) --
	(174.53,145.19) --
	(174.53,145.17) --
	(174.52,145.16) --
	(174.48,145.12) --
	(174.45,145.10) --
	(174.43,145.08) --
	(174.42,145.06) --
	(174.40,145.02) --
	(174.40,144.97) --
	(174.40,144.93) --
	(174.40,144.91) --
	(174.39,144.89) --
	(174.39,144.87) --
	(174.38,144.86) --
	(174.36,144.84) --
	(174.31,144.81) --
	(174.29,144.79) --
	(174.19,144.69) --
	(174.16,144.66) --
	(174.12,144.62) --
	(174.09,144.60) --
	(174.06,144.59) --
	(173.97,144.56) --
	(173.93,144.56) --
	(173.89,144.55) --
	(173.88,144.55) --
	(173.82,144.51) --
	(173.79,144.49) --
	(173.76,144.48) --
	(173.75,144.45) --
	(173.70,144.37) --
	(173.69,144.34) --
	(173.65,144.28) --
	(173.60,144.20) --
	(173.59,144.14) --
	(173.59,144.02) --
	(173.58,143.93) --
	(173.57,143.91) --
	(173.53,143.84) --
	(173.50,143.72) --
	(173.49,143.62) --
	(173.50,143.57) --
	(173.51,143.54) --
	(173.53,143.49) --
	(173.56,143.44) --
	(173.59,143.41) --
	(173.60,143.39) --
	(173.60,143.37) --
	(173.61,143.35) --
	(173.60,143.27) --
	(173.61,143.13) --
	(173.64,143.07) --
	(173.66,143.02) --
	(173.83,142.85) --
	(173.87,142.81) --
	(173.88,142.78) --
	(173.88,142.76) --
	(173.90,142.69) --
	(173.90,142.68) --
	(173.93,142.65) --
	(173.98,142.62) --
	(174.04,142.58) --
	(174.06,142.54) --
	(174.07,142.47) --
	(174.08,142.42) --
	(174.08,142.39) --
	(174.06,142.30) --
	(174.06,142.27) --
	(174.08,142.24) --
	(174.09,142.24) --
	(174.12,142.21) --
	(174.14,142.20) --
	(174.28,142.19) --
	(174.32,142.19) --
	(174.37,142.18) --
	(174.40,142.17) --
	(174.42,142.17) --
	(174.44,142.16) --
	(174.46,142.14) --
	(174.52,142.07) --
	(174.55,142.05) --
	(174.59,142.02) --
	(174.68,141.97) --
	(174.72,141.93) --
	(174.75,141.90) --
	(174.76,141.87) --
	(174.77,141.84) --
	(174.77,141.79) --
	(174.76,141.76) --
	(174.74,141.72) --
	(174.67,141.60) --
	(174.67,141.58) --
	(174.67,141.56) --
	(174.70,141.52) --
	(174.71,141.52) --
	(174.77,141.50) --
	(174.84,141.49) --
	(174.86,141.49) --
	(174.88,141.49) --
	(174.97,141.41) --
	(175.01,141.36) --
	(175.02,141.33) --
	(175.02,141.27) --
	(175.03,141.18) --
	(175.05,141.15) --
	(175.09,141.10) --
	(175.16,141.04) --
	(175.16,141.02) --
	(175.17,141.00) --
	(175.17,140.97) --
	(175.15,140.94) --
	(175.11,140.87) --
	(175.10,140.85) --
	(175.12,140.81) --
	(175.15,140.77) --
	(175.20,140.73) --
	(175.25,140.69) --
	(175.45,140.55) --
	(175.55,140.47) --
	(175.68,140.38) --
	(175.77,140.34) --
	(175.82,140.33) --
	(175.85,140.33) --
	(175.88,140.34) --
	(175.92,140.35) --
	(176.03,140.38) --
	(176.07,140.38) --
	(176.10,140.39) --
	(176.15,140.39) --
	(176.20,140.38) --
	(176.31,140.37) --
	(176.34,140.38) --
	(176.38,140.43) --
	(176.40,140.45) --
	(176.41,140.46) --
	(176.44,140.47) --
	(176.48,140.47) --
	(176.54,140.49) --
	(176.56,140.49) --
	(176.64,140.53) --
	(176.67,140.55) --
	(176.69,140.57) --
	(176.71,140.61) --
	(176.73,140.65) --
	(176.78,140.73) --
	(176.80,140.75) --
	(176.84,140.77) --
	(176.91,140.78) --
	(176.99,140.79) --
	(177.15,140.80) --
	(177.18,140.80) --
	(177.22,140.79) --
	(177.35,140.79) --
	(177.38,140.80) --
	(177.42,140.81) --
	(177.42,140.82) --
	(177.43,140.85) --
	(177.44,140.97) --
	(177.46,141.04) --
	(177.47,141.06) --
	(177.52,141.14) --
	(177.54,141.20) --
	(177.55,141.24) --
	(177.55,141.27) --
	(177.54,141.31) --
	(177.54,141.38) --
	(177.54,141.42) --
	(177.54,141.46) --
	(177.55,141.53) --
	(177.56,141.55) --
	(177.58,141.57) --
	(177.60,141.60) --
	(177.70,141.66) --
	(177.73,141.68) --
	(177.81,141.70) --
	(177.90,141.70) --
	(178.01,141.69) --
	(178.08,141.68) --
	(178.21,141.65) --
	(178.36,141.61) --
	(178.46,141.58) --
	(178.59,141.55) --
	(178.69,141.52) --
	(178.79,141.51) --
	(178.81,141.51) --
	(178.85,141.54) --
	(178.97,141.60) --
	(179.00,141.61) --
	(179.06,141.62) --
	(179.13,141.63) --
	(179.22,141.63) --
	(179.29,141.63) --
	(179.38,141.62) --
	(179.45,141.62) --
	(179.65,141.61) --
	(179.73,141.62) --
	(179.78,141.63) --
	(179.83,141.65) --
	(179.88,141.67) --
	(180.00,141.71) --
	(180.05,141.73) --
	(180.18,141.76) --
	(180.22,141.75) --
	(180.29,141.74) --
	(180.33,141.73) --
	(180.36,141.71) --
	(180.39,141.70) --
	(180.42,141.66) --
	(180.44,141.63) --
	(180.46,141.59) --
	(180.46,141.57) --
	(180.46,141.53) --
	(180.46,141.52) --
	(180.45,141.48) --
	(180.42,141.45) --
	(180.40,141.42) --
	(180.35,141.38) --
	(180.24,141.32) --
	(180.21,141.30) --
	(180.14,141.25) --
	(180.06,141.21) --
	(180.00,141.19) --
	(179.97,141.18) --
	(179.92,141.16) --
	(179.90,141.15) --
	(179.87,141.12) --
	(179.86,141.10) --
	(179.85,141.06) --
	(179.85,141.04) --
	(179.86,141.02) --
	(179.89,141.03) --
	(179.99,141.06) --
	(180.08,141.09) --
	(180.16,141.08) --
	(180.21,141.07) --
	(180.24,141.06) --
	(180.26,141.04) --
	(180.29,141.01) --
	(180.36,140.94) --
	(180.38,140.90) --
	(180.41,140.85) --
	(180.41,140.80) --
	(180.38,140.59) --
	(180.38,140.54) --
	(180.38,140.52) --
	(180.39,140.49) --
	(180.41,140.47) --
	(180.46,140.44) --
	(180.49,140.43) --
	(180.52,140.40) --
	(180.55,140.36) --
	(180.60,140.31) --
	(180.61,140.31) --
	(180.67,140.30) --
	(180.75,140.31) --
	(180.82,140.31) --
	(180.85,140.30) --
	(180.87,140.29) --
	(180.90,140.26) --
	(180.90,140.26) --
	(180.91,140.24) --
	(180.90,140.22) --
	(180.88,140.16) --
	(180.86,140.13) --
	(180.85,140.11) --
	(180.86,140.07) --
	(180.87,140.04) --
	(180.88,140.00) --
	(180.91,139.97) --
	(180.93,139.95) --
	(181.02,139.90) --
	(181.04,139.89) --
	(181.07,139.84) --
	(181.09,139.78) --
	(181.12,139.76) --
	(181.14,139.75) --
	(181.17,139.74) --
	(181.21,139.73) --
	(181.25,139.71) --
	(181.33,139.66) --
	(181.35,139.65) --
	(181.36,139.62) --
	(181.39,139.58) --
	(181.41,139.56) --
	(181.48,139.53) --
	(181.50,139.53) --
	(181.52,139.52) --
	(181.53,139.51) --
	(181.55,139.48) --
	(181.57,139.47) --
	(181.59,139.46) --
	(181.65,139.45) --
	(181.77,139.53) --
	(181.89,139.59) --
	(182.02,139.65) --
	(182.08,139.69) --
	(182.12,139.72) --
	(182.16,139.76) --
	(182.26,139.82) --
	(182.29,139.84) --
	(182.30,139.87) --
	(182.34,139.91) --
	(182.35,139.92) --
	(182.34,139.97) --
	(182.34,139.98) --
	(182.28,140.09) --
	(182.27,140.11) --
	(182.27,140.13) --
	(182.28,140.15) --
	(182.31,140.17) --
	(182.33,140.17) --
	(182.35,140.18) --
	(182.39,140.18) --
	(182.52,140.20) --
	(182.61,140.22) --
	(182.81,140.25) --
	(182.86,140.26) --
	(182.89,140.27) --
	(182.91,140.28) --
	(182.99,140.36) --
	(183.05,140.38) --
	(183.17,140.43) --
	(183.27,140.46) --
	(183.32,140.47) --
	(183.36,140.47) --
	(183.39,140.47) --
	(183.43,140.46) --
	(183.49,140.44) --
	(183.54,140.41) --
	(183.57,140.40) --
	(183.63,140.35) --
	(183.65,140.32) --
	(183.67,140.29) --
	(183.69,140.23) --
	(183.71,140.17) --
	(183.72,140.08) --
	(183.71,140.00) --
	(183.70,139.92) --
	(183.69,139.89) --
	(183.69,139.86) --
	(183.70,139.82) --
	(183.71,139.78) --
	(183.75,139.70) --
	(183.80,139.64) --
	(183.84,139.62) --
	(183.87,139.61) --
	(183.90,139.60) --
	(183.94,139.60) --
	(184.18,139.69) --
	(184.23,139.70) --
	(184.44,139.74) --
	(184.48,139.74) --
	(184.51,139.73) --
	(184.54,139.71) --
	(184.57,139.69) --
	(184.59,139.64) --
	(184.61,139.58) --
	(184.63,139.56) --
	(184.67,139.55) --
	(184.72,139.53) --
	(184.83,139.53) --
	(184.90,139.50) --
	(184.93,139.48) --
	(184.97,139.43) --
	(184.99,139.40) --
	(185.00,139.39) --
	(185.00,139.37) --
	(185.00,139.35) --
	(184.99,139.32) --
	(184.96,139.27) --
	(184.94,139.24) --
	(184.91,139.19) --
	(184.89,139.14) --
	(184.87,139.10) --
	(184.87,139.08) --
	(184.87,139.04) --
	(184.87,139.02) --
	(184.89,138.99) --
	(184.96,138.91) --
	(185.00,138.89) --
	(185.05,138.86) --
	(185.11,138.84) --
	(185.20,138.81) --
	(185.25,138.81) --
	(185.37,138.80) --
	(185.43,138.80) --
	(185.56,138.82) --
	(185.66,138.84) --
	(185.84,138.85) --
	(185.87,138.84) --
	(185.96,138.83) --
	(186.02,138.80) --
	(186.03,138.79) --
	(186.06,138.77) --
	(186.09,138.73) --
	(186.12,138.63) --
	(186.13,138.62) --
	(186.16,138.60) --
	(186.20,138.60) --
	(186.27,138.61) --
	(186.30,138.62) --
	(186.35,138.62) --
	(186.47,138.61) --
	(186.53,138.60) --
	(186.58,138.60) --
	(186.66,138.61) --
	(186.72,138.63) --
	(186.75,138.64) --
	(186.76,138.65) --
	(186.82,138.70) --
	(186.77,138.88) --
	(186.78,138.90) --
	(186.82,138.91) --
	(186.89,138.92) --
	(187.04,138.92) --
	(187.08,138.91) --
	(187.33,138.85) --
	(187.42,138.85) --
	(187.49,138.86) --
	(187.62,138.87) --
	(187.66,138.87) --
	(187.71,138.87) --
	(187.73,138.86) --
	(187.75,138.86) --
	(187.78,138.85) --
	(187.83,138.81) --
	(187.87,138.78) --
	(187.97,138.72) --
	(187.99,138.72) --
	(188.03,138.71) --
	(188.07,138.70) --
	(188.10,138.70) --
	(188.14,138.71) --
	(188.17,138.72) --
	(188.22,138.73) --
	(188.28,138.76) --
	(188.33,138.78) --
	(188.41,138.81) --
	(188.51,138.83) --
	(188.53,138.82) --
	(188.63,138.80) --
	(188.76,138.75) --
	(188.79,138.76) --
	(188.98,138.80) --
	(189.06,138.81) --
	(189.17,138.81) --
	(189.26,138.80) --
	(189.39,138.76) --
	(189.46,138.75) --
	(189.65,138.73) --
	(189.75,138.71) --
	(189.81,138.70) --
	(189.87,138.69) --
	(189.93,138.67) --
	(190.02,138.65) --
	(190.05,138.64) --
	(190.08,138.62) --
	(190.11,138.60) --
	(190.15,138.57) --
	(190.16,138.56) --
	(190.16,138.53) --
	(190.16,138.51) --
	(190.15,138.48) --
	(190.14,138.44) --
	(190.14,138.41) --
	(190.14,138.37) --
	(190.15,138.35) --
	(190.21,138.25) --
	(190.22,138.24) --
	(190.22,138.22) --
	(190.22,138.21) --
	(190.21,138.14) --
	(190.20,138.10) --
	(190.13,137.89) --
	(190.13,137.88) --
	(190.14,137.82) --
	(190.18,137.74) --
	(190.18,137.71) --
	(190.18,137.69) --
	(190.17,137.67) --
	(190.10,137.60) --
	(190.05,137.54) --
	(190.03,137.51) --
	(190.03,137.49) --
	(190.03,137.47) --
	(190.03,137.42) --
	(190.04,137.41) --
	(190.09,137.35) --
	(190.28,137.15) --
	(190.32,137.12) --
	(190.39,137.07) --
	(190.43,137.05) --
	(190.48,137.04) --
	(190.61,137.03) --
	(190.66,137.01) --
	(190.71,137.00) --
	(190.75,136.99) --
	(190.77,136.98) --
	(190.80,136.96) --
	(190.81,136.95) --
	(190.87,136.88) --
	(190.89,136.87) --
	(190.91,136.86) --
	(190.92,136.87) --
	(190.92,136.88) --
	(190.92,136.89) --
	(190.91,136.92) --
	(190.92,136.96) --
	(190.93,136.98) --
	(190.95,137.02) --
	(190.97,137.03) --
	(191.03,137.04) --
	(191.11,137.07) --
	(191.17,137.08) --
	(191.22,137.09) --
	(191.25,137.09) --
	(191.29,137.08) --
	(191.31,137.07) --
	(191.34,137.06) --
	(191.42,137.01) --
	(191.48,136.98) --
	(191.55,136.98) --
	(191.59,136.99) --
	(191.65,137.02) --
	(191.73,137.05) --
	(191.81,137.03) --
	(191.88,137.00) --
	(191.90,136.98) --
	(191.92,136.96) --
	(191.95,136.94) --
	(192.08,136.89) --
	(192.10,136.87) --
	(192.11,136.85) --
	(192.11,136.84) --
	(192.12,136.78) --
	(192.10,136.73) --
	(192.08,136.53) --
	(192.07,136.48) --
	(192.05,136.44) --
	(192.04,136.39) --
	(192.00,136.29) --
	(192.00,136.26) --
	(192.01,136.23) --
	(192.01,136.22) --
	(192.02,136.20) --
	(192.05,136.17) --
	(192.09,136.14) --
	(192.11,136.12) --
	(192.15,136.08) --
	(192.16,136.06) --
	(192.28,136.01) --
	(192.34,136.00) --
	(192.47,135.99) --
	(192.51,135.99) --
	(192.56,135.98) --
	(192.61,135.97) --
	(192.65,135.96) --
	(192.70,135.93) --
	(192.75,135.89) --
	(192.79,135.88) --
	(192.82,135.87) --
	(192.84,135.87) --
	(192.86,135.87) --
	(192.88,135.88) --
	(192.89,135.90) --
	(192.91,135.92) --
	(192.93,135.99) --
	(192.96,136.03) --
	(192.99,136.06) --
	(193.02,136.09) --
	(193.10,136.13) --
	(193.14,136.14) --
	(193.20,136.14) --
	(193.25,136.13) --
	(193.31,136.10) --
	(193.37,136.08) --
	(193.40,136.06) --
	(193.44,136.06) --
	(193.47,136.06) --
	(193.62,136.07) --
	(193.65,136.07) --
	(193.68,136.07) --
	(193.71,136.06) --
	(193.75,136.04) --
	(193.77,136.02) --
	(193.79,135.99) --
	(193.84,135.87) --
	(193.86,135.83) --
	(193.91,135.78) --
	(193.94,135.77) --
	(194.03,135.76) --
	(194.05,135.76) --
	(194.08,135.75) --
	(194.09,135.75) --
	(194.13,135.69) --
	(194.16,135.67) --
	(194.21,135.65) --
	(194.23,135.65) --
	(194.32,135.68) --
	(194.35,135.69) --
	(194.41,135.74) --
	(194.47,135.78) --
	(194.52,135.82) --
	(194.55,135.85) --
	(194.57,135.85) --
	(194.60,135.85) --
	(194.62,135.85) --
	(194.65,135.83) --
	(194.70,135.80) --
	(194.74,135.77) --
	(194.77,135.73) --
	(194.80,135.67) --
	(194.86,135.60) --
	(194.90,135.56) --
	(195.00,135.50) --
	(195.03,135.48) --
	(195.09,135.47) --
	(195.16,135.45) --
	(195.30,135.43) --
	(195.40,135.41) --
	(195.45,135.40) --
	(195.47,135.40) --
	(195.49,135.38) --
	(195.56,135.33) --
	(195.69,135.22) --
	(195.74,135.16) --
	(195.81,135.08) --
	(195.85,135.03) --
	(195.88,134.99) --
	(195.91,134.93) --
	(195.91,134.92) --
	(195.91,134.89) --
	(195.90,134.86) --
	(195.89,134.84) --
	(195.88,134.83) --
	(195.85,134.82) --
	(195.74,134.79) --
	(195.67,134.78) --
	(195.61,134.76) --
	(195.56,134.73) --
	(195.53,134.67) --
	(195.52,134.62) --
	(195.50,134.60) --
	(195.47,134.56) --
	(195.39,134.49) --
	(195.34,134.46) --
	(195.32,134.44) --
	(195.26,134.41) --
	(195.23,134.39) --
	(195.20,134.36) --
	(195.17,134.34) --
	(195.15,134.32) --
	(195.08,134.14) --
	(195.07,134.08) --
	(195.05,134.05) --
	(195.02,134.01) --
	(194.99,133.97) --
	(194.96,133.94) --
	(194.94,133.92) --
	(194.82,133.87) --
	(194.80,133.86) --
	(194.76,133.82) --
	(194.73,133.78) --
	(194.72,133.72) --
	(194.71,133.63) --
	(194.70,133.53) --
	(194.70,133.42) --
	(194.70,133.36) --
	(194.70,133.34) --
	(194.68,133.32) --
	(194.61,133.26) --
	(194.56,133.22) --
	(194.53,133.18) --
	(194.51,133.16) --
	(194.49,133.09) --
	(194.49,133.03) --
	(194.49,133.00) --
	(194.50,132.98) --
	(194.52,132.94) --
	(194.55,132.90) --
	(194.58,132.88) --
	(194.64,132.83) --
	(194.70,132.81) --
	(194.82,132.77) --
	(194.86,132.76) --
	(194.90,132.75) --
	(194.98,132.75) --
	(195.05,132.76) --
	(195.08,132.76) --
	(195.11,132.76) --
	(195.15,132.76) --
	(195.18,132.75) --
	(195.22,132.73) --
	(195.26,132.71) --
	(195.39,132.66) --
	(195.45,132.62) --
	(195.47,132.60) --
	(195.48,132.57) --
	(195.49,132.55) --
	(195.50,132.52) --
	(195.49,132.50) --
	(195.47,132.41) --
	(195.44,132.36) --
	(195.42,132.34) --
	(195.15,132.08) --
	(195.08,131.98) --
	(195.07,131.95) --
	(195.06,131.92) --
	(195.04,131.82) --
	(194.99,131.75) --
	(194.96,131.70) --
	(194.92,131.67) --
	(194.86,131.61) --
	(194.78,131.52) --
	(194.77,131.51) --
	(194.78,131.47) --
	(194.79,131.46) --
	(194.85,131.38) --
	(194.88,131.34) --
	(194.92,131.27) --
	(194.94,131.22) --
	(194.96,131.20) --
	(195.01,131.14) --
	(195.04,131.12) --
	(195.07,131.10) --
	(195.20,131.07) --
	(195.26,131.06) --
	(195.36,131.06) --
	(195.90,131.06) --
	(196.03,131.09) --
	(196.05,131.08) --
	(196.08,131.08) --
	(196.11,131.07) --
	(196.15,131.06) --
	(196.17,131.04) --
	(196.20,131.00) --
	(196.23,130.97) --
	(196.24,130.94) --
	(196.25,130.90) --
	(196.29,130.73) --
	(196.31,130.69) --
	(196.34,130.64) --
	(196.36,130.57) --
	(196.37,130.40) --
	(196.39,130.32) --
	(196.42,130.28) --
	(196.44,130.26) --
	(196.47,130.24) --
	(196.51,130.22) --
	(196.60,130.19) --
	(196.74,130.17) --
	(196.93,130.15) --
	(197.00,130.12) --
	(197.03,130.10) --
	(197.07,130.07) --
	(197.13,130.03) --
	(197.21,129.95) --
	(197.24,129.90) --
	(197.26,129.85) --
	(197.30,129.65) --
	(197.33,129.48) --
	(197.33,129.38) --
	(197.33,129.24) --
	(197.33,129.18) --
	(197.30,129.02) --
	(197.31,128.95) --
	(197.38,128.64) --
	(197.39,128.49) --
	(197.39,128.45) --
	(197.39,128.41) --
	(197.37,128.37) --
	(197.34,128.33) --
	(197.31,128.30) --
	(197.27,128.26) --
	(197.21,128.22) --
	(197.03,128.15) --
	(196.88,128.11) --
	(196.82,128.09) --
	(196.75,128.08) --
	(196.71,127.99) --
	(196.69,127.93) --
	(196.68,127.91) --
	(196.55,127.77) --
	(196.52,127.71) --
	(196.51,127.62) --
	(196.51,127.45) --
	(196.51,127.40) --
	(196.51,127.38) --
	(196.56,127.28) --
	(196.58,127.25) --
	(196.60,127.19) --
	(196.61,127.15) --
	(196.60,127.11) --
	(196.59,127.07) --
	(196.56,127.00) --
	(196.50,126.86) --
	(196.35,126.54) --
	(196.29,126.40) --
	(196.18,126.18) --
	(196.18,126.15) --
	(196.18,126.13) --
	(196.18,126.11) --
	(196.32,125.88) --
	(196.36,125.82) --
	(196.39,125.77) --
	(196.39,125.76) --
	(196.38,125.74) --
	(196.35,125.69) --
	(196.25,125.67) --
	(196.23,125.66) --
	(196.13,125.66) --
	(195.97,125.62) --
	(195.92,125.60) --
	(195.83,125.57) --
	(195.80,125.57) --
	(195.60,125.57) --
	(195.53,125.57) --
	(195.36,125.54) --
	(195.30,125.53) --
	(195.24,125.52) --
	(195.18,125.50) --
	(195.03,125.46) --
	(194.94,125.40) --
	(194.90,125.37) --
	(194.87,125.30) --
	(194.87,125.22) --
	(194.86,125.15) --
	(194.87,125.11) --
	(194.87,125.08) --
	(194.89,125.04) --
	(194.90,125.00) --
	(194.90,124.94) --
	(194.90,124.91) --
	(194.88,124.87) --
	(194.73,124.76) --
	(194.74,124.76) --
	(194.74,124.76) --
	(194.75,124.76) --
	(194.72,124.71) --
	(194.70,124.69) --
	(194.70,124.67) --
	(194.70,124.64) --
	(194.69,124.62) --
	(194.68,124.59) --
	(194.67,124.57) --
	(194.64,124.55) --
	(194.57,124.52) --
	(194.54,124.49) --
	(194.47,124.47) --
	(194.41,124.44) --
	(194.39,124.43) --
	(194.36,124.43) --
	(194.33,124.43) --
	(194.25,124.42) --
	(194.21,124.43) --
	(194.19,124.42) --
	(194.16,124.43) --
	(194.14,124.43) --
	(194.13,124.42) --
	(194.13,124.37) --
	(194.14,124.34) --
	(194.13,124.30) --
	(194.12,124.25) --
	(194.08,124.20) --
	(194.06,124.18) --
	(194.03,124.16) --
	(194.03,124.16) --
	(194.05,124.14) --
	(194.09,124.12) --
	(194.11,124.10) --
	(194.11,124.08) --
	(194.11,124.06) --
	(194.07,124.04) --
	(194.05,124.01) --
	(194.04,123.99) --
	(194.01,123.95) --
	(193.98,123.93) --
	(193.94,123.88) --
	(193.92,123.87) --
	(193.89,123.86) --
	(193.85,123.84) --
	(193.82,123.82) --
	(193.82,123.81) --
	(193.82,123.80) --
	(193.87,123.77) --
	(193.89,123.77) --
	(193.92,123.77) --
	(193.98,123.76) --
	(194.03,123.75) --
	(194.07,123.73) --
	(194.09,123.71) --
	(194.12,123.69) --
	(194.16,123.66) --
	(194.22,123.65) --
	(194.25,123.63) --
	(194.26,123.60) --
	(194.28,123.56) --
	(194.29,123.49) --
	(194.27,123.42) --
	(194.24,123.36) --
	(194.22,123.31) --
	(194.27,123.30) --
	(194.30,123.31) --
	(194.34,123.31) --
	(194.39,123.30) --
	(194.43,123.30) --
	(194.47,123.27) --
	(194.52,123.23) --
	(194.55,123.21) --
	(194.56,123.17) --
	(194.55,123.17) --
	(194.53,123.15) --
	(194.52,123.14) --
	(194.52,123.12) --
	(194.53,123.05) --
	(194.52,123.02) --
	(194.50,122.98) --
	(194.50,122.95) --
	(194.50,122.92) --
	(194.51,122.91) --
	(194.54,122.88) --
	(194.56,122.87) --
	(194.61,122.79) --
	(194.63,122.74) --
	(194.63,122.67) --
	(194.61,122.58) --
	(194.60,122.55) --
	(194.61,122.46) --
	(194.62,122.43) --
	(194.63,122.40) --
	(194.61,122.36) --
	(194.60,122.28) --
	(194.60,122.26) --
	(194.54,122.23) --
	(194.42,122.21) --
	(194.33,122.20) --
	(194.24,122.19) --
	(194.20,122.20) --
	(194.17,122.21) --
	(194.16,122.21) --
	(194.15,122.21) --
	(194.13,122.21) --
	(194.11,122.20) --
	(194.10,122.19) --
	(194.09,122.17) --
	(194.08,122.12) --
	(194.06,122.09) --
	(194.03,122.06) --
	(193.99,122.04) --
	(193.95,122.01) --
	(193.92,121.98) --
	(193.90,121.96) --
	(193.88,121.95) --
	(193.83,121.93) --
	(193.72,121.92) --
	(193.64,121.91) --
	(193.61,121.90) --
	(193.49,121.82) --
	(193.44,121.80) --
	(193.41,121.80) --
	(193.39,121.79) --
	(193.35,121.81) --
	(193.33,121.82) --
	(193.30,121.82) --
	(193.26,121.81) --
	(193.24,121.79) --
	(193.24,121.78) --
	(193.24,121.75) --
	(193.22,121.69) --
	(193.22,121.68) --
	(193.21,121.68) --
	(193.18,121.69) --
	(193.14,121.69) --
	(193.12,121.70) --
	(193.08,121.69) --
	(193.05,121.69) --
	(193.01,121.68) --
	(192.97,121.67) --
	(192.93,121.66) --
	(192.87,121.60) --
	(192.86,121.58) --
	(192.85,121.58) --
	(192.82,121.60) --
	(192.79,121.60) --
	(192.76,121.61) --
	(192.74,121.60) --
	(192.73,121.60) --
	(192.71,121.58) --
	(192.69,121.57) --
	(192.67,121.57) --
	(192.66,121.56) --
	(192.55,121.57) --
	(192.53,121.57) --
	(192.51,121.56) --
	(192.45,121.51) --
	(192.41,121.48) --
	(192.36,121.44) --
	(192.34,121.43) --
	(192.32,121.41) --
	(192.29,121.34) --
	(192.28,121.33) --
	(192.27,121.33) --
	(192.23,121.32) --
	(192.15,121.28) --
	(192.13,121.27) --
	(192.12,121.24) --
	(192.11,121.13) --
	(192.09,121.08) --
	(192.07,121.04) --
	(192.04,120.99) --
	(191.95,120.89) --
	(191.90,120.85) --
	(191.86,120.83) --
	(191.73,120.78) --
	(191.65,120.75) --
	(191.62,120.73) --
	(191.53,120.67) --
	(191.48,120.63) --
	(191.46,120.60) --
	(191.43,120.55) --
	(191.41,120.51) --
	(191.37,120.49) --
	(191.27,120.44) --
	(191.15,120.36) --
	(191.03,120.26) --
	(191.00,120.24) --
	(190.96,120.21) --
	(190.95,120.21) --
	(190.93,120.20) --
	(190.87,120.20) --
	(190.85,120.20) --
	(190.82,120.19) --
	(190.79,120.17) --
	(190.77,120.16) --
	(190.71,120.14) --
	(190.68,120.14) --
	(190.56,120.09) --
	(190.37,120.02) --
	(190.32,120.00) --
	(190.29,119.99) --
	(190.16,119.97) --
	(190.14,119.96) --
	(190.11,119.98) --
	(190.05,120.04) --
	(190.02,120.06) --
	(189.97,120.07) --
	(189.91,120.09) --
	(189.88,120.10) --
	(189.85,120.11) --
	(189.81,120.10) --
	(189.73,120.07) --
	(189.65,120.03) --
	(189.63,120.02) --
	(189.59,120.02) --
	(189.50,120.05) --
	(189.39,120.08) --
	(189.34,120.09) --
	(189.30,120.10) --
	(189.25,120.10) --
	(189.20,120.11) --
	(189.17,120.12) --
	(189.15,120.13) --
	(189.11,120.10) --
	(189.08,120.09) --
	(189.05,120.07) --
	(189.04,120.04) --
	(188.93,119.95) --
	(188.90,119.94) --
	(188.87,119.94) --
	(188.84,119.93) --
	(188.82,119.94) --
	(188.81,119.93) --
	(188.80,119.92) --
	(188.79,119.91) --
	(188.80,119.84) --
	(188.81,119.80) --
	(188.81,119.76) --
	(188.82,119.72) --
	(188.83,119.67) --
	(188.82,119.65) --
	(188.81,119.63) --
	(188.80,119.63) --
	(188.76,119.64) --
	(188.74,119.64) --
	(188.72,119.63) --
	(188.70,119.61) --
	(188.67,119.57) --
	(188.67,119.57) --
	(188.67,119.54) --
	(188.70,119.52) --
	(188.73,119.51) --
	(188.75,119.48) --
	(188.76,119.45) --
	(188.79,119.43) --
	(188.82,119.41) --
	(188.83,119.40) --
	(188.86,119.30) --
	(188.89,119.21) --
	(188.92,119.16) --
	(188.95,119.14) --
	(188.96,119.11) --
	(188.99,119.02) --
	(188.99,119.00) --
	(188.98,118.95) --
	(188.98,118.94) --
	(188.99,118.89) --
	(189.02,118.83) --
	(189.04,118.78) --
	(189.05,118.76) --
	(189.03,118.72) --
	(188.98,118.68) --
	(188.95,118.63) --
	(188.92,118.61) --
	(188.84,118.57) --
	(188.66,118.48) --
	(188.64,118.46) --
	(188.62,118.43) --
	(188.63,118.39) --
	(188.64,118.36) --
	(188.64,118.33) --
	(188.61,118.28) --
	(188.57,118.25) --
	(188.53,118.19) --
	(188.50,118.11) --
	(188.49,118.07) --
	(188.47,118.02) --
	(188.47,117.98) --
	(188.47,117.95) --
	(188.43,117.92) --
	(188.35,117.88) --
	(188.33,117.86) --
	(188.31,117.84) --
	(188.31,117.82) --
	(188.35,117.76) --
	(188.37,117.73) --
	(188.37,117.67) --
	(188.42,117.60) --
	(188.48,117.56) --
	(188.50,117.54) --
	(188.51,117.51) --
	(188.51,117.45) --
	(188.46,117.38) --
	(188.45,117.34) --
	(188.43,117.33) --
	(188.39,117.30) --
	(188.28,117.24) --
	(188.27,117.23) --
	(188.26,117.22) --
	(188.23,117.21) --
	(188.22,117.20) --
	(188.22,117.19) --
	(188.23,117.15) --
	(188.22,117.11) --
	(188.19,117.06) --
	(188.17,117.00) --
	(188.21,116.93) --
	(188.20,116.91) --
	(188.16,116.84) --
	(188.13,116.79) --
	(188.10,116.74) --
	(188.10,116.72) --
	(188.11,116.68) --
	(188.11,116.66) --
	(188.09,116.62) --
	(188.06,116.59) --
	(188.02,116.57) --
	(187.93,116.56) --
	(187.85,116.54) --
	(187.75,116.52) --
	(187.71,116.51) --
	(187.62,116.49) --
	(187.57,116.47) --
	(187.53,116.44) --
	(187.50,116.42) --
	(187.47,116.41) --
	(187.41,116.37) --
	(187.39,116.36) --
	(187.35,116.36) --
	(187.33,116.35) --
	(187.35,116.33) --
	(187.39,116.27) --
	(187.41,116.25) --
	(187.42,116.24) --
	(187.43,116.22) --
	(187.42,116.20) --
	(187.42,116.17) --
	(187.37,116.13) --
	(187.35,116.11) --
	(187.34,116.08) --
	(187.34,116.04) --
	(187.34,115.99) --
	(187.35,115.93) --
	(187.34,115.88) --
	(187.33,115.84) --
	(187.32,115.81) --
	(187.32,115.78) --
	(187.33,115.72) --
	(187.32,115.70) --
	(187.29,115.65) --
	(187.24,115.62) --
	(187.17,115.58) --
	(187.16,115.56) --
	(187.14,115.55) --
	(187.14,115.51) --
	(187.17,115.47) --
	(187.20,115.44) --
	(187.22,115.39) --
	(187.23,115.35) --
	(187.24,115.30) --
	(187.30,115.22) --
	(187.39,115.15) --
	(187.47,115.11) --
	(187.53,115.09) --
	(187.56,115.08) --
	(187.57,115.08) --
	(187.58,115.07) --
	(187.60,115.02) --
	(187.61,114.98) --
	(187.61,114.92) --
	(187.60,114.90) --
	(187.62,114.85) --
	(187.64,114.83) --
	(187.65,114.81) --
	(187.71,114.76) --
	(187.77,114.71) --
	(187.86,114.63) --
	(187.92,114.61) --
	(187.96,114.58) --
	(187.96,114.53) --
	(187.95,114.51) --
	(187.94,114.47) --
	(187.92,114.46) --
	(187.92,114.35) --
	(187.90,114.29) --
	(187.90,114.26) --
	(187.90,114.24) --
	(187.84,114.19) --
	(187.80,114.16) --
	(187.77,114.12) --
	(187.74,114.08) --
	(187.69,114.03) --
	(187.61,113.94) --
	(187.58,113.90) --
	(187.58,113.87) --
	(187.57,113.84) --
	(187.59,113.81) --
	(187.92,113.51) --
	(187.94,113.48) --
	(187.94,113.38) --
	(187.94,113.35) --
	(187.92,113.33) --
	(187.92,113.31) --
	(187.94,113.26) --
	(187.95,113.23) --
	(187.99,113.18) --
	(188.02,113.15) --
	(188.04,113.11) --
	(188.09,112.99) --
	(188.10,112.93) --
	(188.12,112.92) --
	(188.15,112.88) --
	(188.21,112.85) --
	(188.26,112.81) --
	(188.27,112.77) --
	(188.24,112.75) --
	(188.22,112.75) --
	(188.17,112.70) --
	(188.14,112.68) --
	(188.12,112.67) --
	(188.10,112.66) --
	(188.08,112.65) --
	(188.03,112.61) --
	(188.00,112.54) --
	(187.98,112.51) --
	(187.94,112.48) --
	(187.94,112.46) --
	(187.94,112.35) --
	(187.93,112.31) --
	(187.92,112.28) --
	(187.90,112.25) --
	(187.89,112.23) --
	(187.89,112.19) --
	(187.89,112.15) --
	(187.87,112.12) --
	(187.82,112.08) --
	(187.76,112.04) --
	(187.72,112.00) --
	(187.70,111.96) --
	(187.70,111.90) --
	(187.72,111.86) --
	(187.74,111.81) --
	(187.75,111.79) --
	(187.75,111.76) --
	(187.74,111.73) --
	(187.64,111.66) --
	(187.58,111.61) --
	(187.57,111.59) --
	(187.58,111.57) --
	(187.62,111.54) --
	(187.68,111.50) --
	(187.69,111.48) --
	(187.68,111.45) --
	(187.67,111.42) --
	(187.59,111.37) --
	(187.55,111.34) --
	(187.50,111.32) --
	(187.45,111.31) --
	(187.36,111.27) --
	(187.30,111.23) --
	(187.27,111.22) --
	(187.23,111.20) --
	(187.19,111.19) --
	(187.08,111.17) --
	(187.00,111.13) --
	(186.95,111.08) --
	(186.82,111.04) --
	(186.80,111.03) --
	(186.76,110.98) --
	(186.74,110.95) --
	(186.73,110.92) --
	(186.73,110.89) --
	(186.76,110.86) --
	(186.81,110.84) --
	(186.86,110.78) --
	(186.90,110.74) --
	(186.97,110.70) --
	(187.04,110.67) --
	(187.08,110.64) --
	(187.09,110.60) --
	(187.13,110.57) --
	(187.20,110.52) --
	(187.26,110.53) --
	(187.31,110.51) --
	(187.35,110.48) --
	(187.37,110.44) --
	(187.37,110.40) --
	(187.39,110.34) --
	(187.39,110.30) --
	(187.36,110.19) --
	(187.37,110.14) --
	(187.39,110.08) --
	(187.39,110.04) --
	(187.37,109.98) --
	(187.37,109.96) --
	(187.40,109.94) --
	(187.45,109.93) --
	(187.50,109.92) --
	(187.54,109.93) --
	(187.55,109.93) --
	(187.57,109.93) --
	(187.57,109.93) --
	(187.57,109.89) --
	(187.57,109.83) --
	(187.59,109.78) --
	(187.60,109.75) --
	(187.61,109.73) --
	(187.60,109.70) --
	(187.57,109.64) --
	(187.56,109.62) --
	(187.57,109.61) --
	(187.68,109.59) --
	(187.73,109.59) --
	(187.75,109.59) --
	(187.76,109.58) --
	(187.77,109.57) --
	(187.78,109.51) --
	(187.77,109.46) --
	(187.75,109.42) --
	(187.73,109.38) --
	(187.71,109.34) --
	(187.64,109.28) --
	(187.64,109.27) --
	(187.69,109.21) --
	(187.73,109.14) --
	(187.77,109.10) --
	(187.83,109.05) --
	(187.88,109.03) --
	(187.97,109.01) --
	(188.04,109.00) --
	(188.09,108.99) --
	(188.16,108.94) --
	(188.17,108.93) --
	(188.20,108.89) --
	(188.23,108.78) --
	(188.24,108.75) --
	(188.27,108.69) --
	(188.34,108.57) --
	(188.37,108.53) --
	(188.43,108.48) --
	(188.43,108.47) --
	(188.44,108.45) --
	(188.43,108.39) --
	(188.43,108.37) --
	(188.49,108.30) --
	(188.48,108.27) --
	(188.47,108.16) --
	(188.47,108.10) --
	(188.46,108.00) --
	(188.45,107.98) --
	(188.41,107.94) --
	(188.38,107.89) --
	(188.35,107.84) --
	(188.29,107.69) --
	(188.22,107.59) --
	(188.20,107.51) --
	(188.18,107.49) --
	(188.15,107.44) --
	(188.12,107.39) --
	(188.09,107.31) --
	(188.08,107.25) --
	(188.08,107.21) --
	(188.07,107.18) --
	(188.05,107.15) --
	(188.05,107.12) --
	(188.03,107.06) --
	(188.03,107.01) --
	(188.05,106.96) --
	(188.05,106.92) --
	(188.06,106.86) --
	(188.05,106.78) --
	(188.03,106.71) --
	(187.98,106.56) --
	(187.95,106.51) --
	(187.93,106.44) --
	(187.92,106.38) --
	(187.91,106.34) --
	(187.88,106.31) --
	(187.86,106.28) --
	(187.85,106.26) --
	(187.87,106.23) --
	(187.89,106.21) --
	(187.93,106.19) --
	(187.95,106.15) --
	(187.94,106.10) --
	(187.95,106.05) --
	(187.99,106.01) --
	(188.03,105.99) --
	(188.08,105.99) --
	(188.18,106.01) --
	(188.27,106.02) --
	(188.33,106.04) --
	(188.35,106.05) --
	(188.37,106.07) --
	(188.38,106.09) --
	(188.40,106.10) --
	(188.42,106.09) --
	(188.43,106.08) --
	(188.44,106.05) --
	(188.45,106.00) --
	(188.47,105.96) --
	(188.47,105.92) --
	(188.45,105.87) --
	(188.45,105.85) --
	(188.47,105.82) --
	(188.48,105.79) --
	(188.51,105.72) --
	(188.52,105.68) --
	(188.51,105.65) --
	(188.48,105.60) --
	(188.47,105.56) --
	(188.48,105.50) --
	(188.47,105.47) --
	(188.45,105.43) --
	(188.42,105.40) --
	(188.40,105.37) --
	(188.39,105.34) --
	(188.39,105.32) --
	(188.36,105.27) --
	(188.35,105.24) --
	(188.35,105.19) --
	(188.35,105.14) --
	(188.34,105.11) --
	(188.33,105.07) --
	(188.30,105.03) --
	(188.29,105.02) --
	(188.25,105.01) --
	(188.22,105.01) --
	(188.20,105.00) --
	(188.17,104.98) --
	(188.12,104.96) --
	(188.04,104.93) --
	(187.97,104.90) --
	(187.92,104.84) --
	(187.89,104.81) --
	(187.88,104.81) --
	(187.85,104.81) --
	(187.82,104.81) --
	(187.71,104.81) --
	(187.69,104.82) --
	(187.68,104.82) --
	(187.64,104.81) --
	(187.62,104.80) --
	(187.57,104.80) --
	(187.50,104.81) --
	(187.38,104.87) --
	(187.35,104.89) --
	(187.32,104.90) --
	(187.28,104.91) --
	(187.25,104.90) --
	(187.21,104.89) --
	(187.15,104.85) --
	(186.99,104.76) --
	(186.96,104.74) --
	(186.92,104.73) --
	(186.83,104.71) --
	(186.80,104.69) --
	(186.78,104.67) --
	(186.76,104.66) --
	(186.72,104.69) --
	(186.70,104.72) --
	(186.69,104.74) --
	(186.67,104.75) --
	(186.63,104.74) --
	(186.58,104.74) --
	(186.53,104.72) --
	(186.50,104.71) --
	(186.47,104.70) --
	(186.46,104.69) --
	(186.45,104.67) --
	(186.47,104.63) --
	(186.47,104.60) --
	(186.49,104.54) --
	(186.51,104.51) --
	(186.53,104.48) --
	(186.54,104.46) --
	(186.53,104.42) --
	(186.51,104.41) --
	(186.45,104.40) --
	(186.41,104.39) --
	(186.39,104.37) --
	(186.37,104.34) --
	(186.36,104.31) --
	(186.36,104.27) --
	(186.37,104.20) --
	(186.38,104.16) --
	(186.38,104.12) --
	(186.40,104.07) --
	(186.42,104.02) --
	(186.40,104.00) --
	(186.31,103.97) --
	(186.22,103.88) --
	(186.21,103.84) --
	(186.20,103.80) --
	(186.20,103.77) --
	(186.19,103.74) --
	(186.16,103.67) --
	(186.13,103.61) --
	(186.04,103.49) --
	(186.02,103.47) --
	(186.00,103.44) --
	(185.97,103.41) --
	(185.96,103.39) --
	(185.96,103.33) --
	(185.97,103.29) --
	(185.96,103.27) --
	(186.00,103.20) --
	(186.02,103.13) --
	(186.02,103.10) --
	(186.01,103.08) --
	(185.92,103.01) --
	(185.89,102.97) --
	(185.85,102.93) --
	(185.80,102.90) --
	(185.77,102.87) --
	(185.71,102.79) --
	(185.67,102.74) --
	(185.63,102.71) --
	(185.59,102.67) --
	(185.57,102.67) --
	(185.50,102.69) --
	(185.42,102.70) --
	(185.37,102.70) --
	(185.35,102.71) --
	(185.32,102.73) --
	(185.27,102.79) --
	(185.19,102.81) --
	(185.10,102.83) --
	(185.03,102.83) --
	(184.98,102.82) --
	(184.87,102.78) --
	(184.85,102.78) --
	(184.83,102.78) --
	(184.79,102.80) --
	(184.77,102.81) --
	(184.75,102.82) --
	(184.74,102.82) --
	(184.73,102.81) --
	(184.72,102.81) --
	(184.71,102.76) --
	(184.70,102.72) --
	(184.68,102.68) --
	(184.64,102.65) --
	(184.59,102.62) --
	(184.55,102.60) --
	(184.50,102.57) --
	(184.47,102.54) --
	(184.39,102.45) --
	(184.34,102.42) --
	(184.29,102.40) --
	(184.25,102.35) --
	(184.22,102.32) --
	(184.20,102.29) --
	(184.18,102.26) --
	(184.16,102.22) --
	(184.05,102.15) --
	(183.99,102.10) --
	(183.96,102.07) --
	(183.95,102.05) --
	(183.93,102.00) --
	(183.92,101.97) --
	(183.90,101.94) --
	(183.88,101.92) --
	(183.87,101.89) --
	(183.86,101.87) --
	(183.85,101.85) --
	(183.83,101.83) --
	(183.79,101.82) --
	(183.74,101.82) --
	(183.71,101.80) --
	(183.66,101.78) --
	(183.61,101.77) --
	(183.54,101.75) --
	(183.49,101.73) --
	(183.46,101.71) --
	(183.42,101.67) --
	(183.38,101.64) --
	(183.34,101.59) --
	(183.30,101.55) --
	(183.28,101.54) --
	(183.26,101.53) --
	(183.22,101.55) --
	(183.19,101.57) --
	(183.16,101.57) --
	(183.10,101.54) --
	(183.08,101.51) --
	(183.03,101.50) --
	(182.99,101.51) --
	(182.97,101.52) --
	(182.95,101.52) --
	(182.90,101.48) --
	(182.87,101.49) --
	(182.86,101.49) --
	(182.85,101.50) --
	(182.85,101.51) --
	(182.88,101.56) --
	(182.88,101.58) --
	(182.87,101.59) --
	(182.86,101.61) --
	(182.84,101.63) --
	(182.83,101.63) --
	(182.80,101.64) --
	(182.76,101.64) --
	(182.69,101.65) --
	(182.63,101.67) --
	(182.56,101.70) --
	(182.50,101.73) --
	(182.46,101.75) --
	(182.41,101.76) --
	(182.40,101.76) --
	(182.36,101.74) --
	(182.34,101.70) --
	(182.35,101.67) --
	(182.37,101.63) --
	(182.37,101.59) --
	(182.38,101.56) --
	(182.38,101.54) --
	(182.37,101.52) --
	(182.36,101.52) --
	(182.34,101.51) --
	(182.31,101.52) --
	(182.29,101.53) --
	(182.27,101.54) --
	(182.26,101.55) --
	(182.24,101.57) --
	(182.21,101.61) --
	(182.20,101.63) --
	(182.15,101.67) --
	(182.10,101.70) --
	(182.07,101.70) --
	(182.04,101.69) --
	(182.02,101.69) --
	(181.99,101.69) --
	(181.96,101.70) --
	(181.93,101.70) --
	(181.88,101.68) --
	(181.87,101.67) --
	(181.85,101.67) --
	(181.78,101.67) --
	(181.67,101.63) --
	(181.62,101.62) --
	(181.54,101.61) --
	(181.49,101.62) --
	(181.47,101.62) --
	(181.43,101.62) --
	(181.40,101.62) --
	(181.37,101.63) --
	(181.32,101.63) --
	(181.21,101.63) --
	(181.16,101.62) --
	(181.09,101.59) --
	(181.06,101.58) --
	(181.01,101.57) --
	(180.96,101.57) --
	(180.91,101.56) --
	(180.89,101.56) --
	(180.86,101.56) --
	(180.76,101.56) --
	(180.72,101.57) --
	(180.69,101.57) --
	(180.64,101.56) --
	(180.54,101.54) --
	(180.50,101.53) --
	(180.44,101.54) --
	(180.36,101.55) --
	(180.32,101.56) --
	(180.29,101.60) --
	(180.25,101.66) --
	(180.22,101.67) --
	(180.17,101.68) --
	(180.14,101.68) --
	(180.11,101.68) --
	(180.03,101.66) --
	(179.99,101.66) --
	(179.93,101.67) --
	(179.78,101.66) --
	(179.72,101.67) --
	(179.64,101.71) --
	(179.61,101.74) --
	(179.58,101.77) --
	(179.55,101.77) --
	(179.52,101.77) --
	(179.51,101.76) --
	(179.51,101.74) --
	(179.48,101.71) --
	(179.44,101.71) --
	(179.40,101.73) --
	(179.36,101.73) --
	(179.32,101.72) --
	(179.29,101.71) --
	(179.27,101.70) --
	(179.25,101.69) --
	(179.23,101.71) --
	(179.18,101.72) --
	(179.13,101.72) --
	(179.04,101.71) --
	(178.99,101.69) --
	(178.91,101.65) --
	(178.79,101.56) --
	(178.73,101.52) --
	(178.68,101.49) --
	(178.60,101.45) --
	(178.57,101.44) --
	(178.55,101.43) --
	(178.53,101.42) --
	(178.51,101.41) --
	(178.49,101.39) --
	(178.48,101.37) --
	(178.45,101.35) --
	(178.40,101.32) --
	(178.38,101.31) --
	(178.35,101.30) --
	(178.32,101.23) --
	(178.30,101.21) --
	(178.26,101.18) --
	(178.24,101.16) --
	(178.22,101.14) --
	(178.22,101.12) --
	(178.25,101.09) --
	(178.27,101.07) --
	(178.29,101.04) --
	(178.29,101.02) --
	(178.29,101.01) --
	(178.27,100.99) --
	(178.16,100.94) --
	(178.11,100.92) --
	(178.07,100.91) --
	(178.03,100.88) --
	(177.94,100.81) --
	(177.86,100.73) --
	(177.81,100.68) --
	(177.74,100.63) --
	(177.68,100.58) --
	(177.64,100.54) --
	(177.60,100.52) --
	(177.57,100.49) --
	(177.54,100.47) --
	(177.50,100.45) --
	(177.44,100.45) --
	(177.43,100.44) --
	(177.42,100.42) --
	(177.42,100.41) --
	(177.44,100.39) --
	(177.48,100.35) --
	(177.58,100.29) --
	(177.63,100.27) --
	(177.68,100.25) --
	(177.71,100.23) --
	(177.74,100.22) --
	(177.74,100.22) --
	(177.76,100.20) --
	(177.79,100.09) --
	(177.78,100.09) --
	(177.78,100.09) --
	(177.77,100.09) --
	(177.78,100.06) --
	(177.79,100.01) --
	(177.75, 99.72) --
	(177.75, 99.68) --
	(177.70, 99.62) --
	(177.64, 99.54) --
	(177.62, 99.52) --
	(177.61, 99.50) --
	(177.61, 99.48) --
	(177.61, 99.44) --
	(177.61, 99.37) --
	(177.62, 99.31) --
	(177.62, 99.27) --
	(177.59, 99.08) --
	(177.55, 98.92) --
	(177.54, 98.87) --
	(177.53, 98.84) --
	(177.52, 98.82) --
	(177.51, 98.80) --
	(177.48, 98.76) --
	(177.43, 98.70) --
	(177.35, 98.62) --
	(177.22, 98.52) --
	(177.12, 98.45) --
	(177.07, 98.41) --
	(177.05, 98.40) --
	(177.01, 98.36) --
	(176.98, 98.29) --
	(176.94, 98.24) --
	(176.92, 98.21) --
	(176.81, 98.17) --
	(176.79, 98.16) --
	(176.76, 98.13) --
	(176.73, 98.08) --
	(176.71, 98.00) --
	(176.70, 97.97) --
	(176.69, 97.94) --
	(176.67, 97.91) --
	(176.64, 97.89) --
	(176.59, 97.87) --
	(176.50, 97.83) --
	(176.42, 97.81) --
	(176.29, 97.78) --
	(176.16, 97.76) --
	(176.03, 97.73) --
	(175.88, 97.69) --
	(175.85, 97.68) --
	(175.81, 97.66) --
	(175.77, 97.64) --
	(175.75, 97.62) --
	(175.72, 97.59) --
	(175.70, 97.56) --
	(175.69, 97.51) --
	(175.66, 97.21) --
	(175.66, 97.19) --
	(175.64, 97.14) --
	(175.62, 97.11) --
	(175.54, 97.02) --
	(175.48, 96.94) --
	(175.41, 96.84) --
	(175.34, 96.77) --
	(175.01, 96.45) --
	(174.87, 96.30) --
	(174.60, 96.08) --
	(174.49, 96.01) --
	(174.40, 95.95) --
	(174.26, 95.83) --
	(174.18, 95.78) --
	(174.13, 95.75) --
	(174.11, 95.75) --
	(174.08, 95.74) --
	(174.05, 95.75) --
	(174.02, 95.76) --
	(174.01, 95.77) --
	(173.99, 95.78) --
	(173.95, 95.84) --
	(173.90, 95.90) --
	(173.85, 95.95) --
	(173.80, 96.00) --
	(173.77, 96.02) --
	(173.61, 96.07) --
	(173.53, 96.09) --
	(173.49, 96.10) --
	(173.47, 96.10) --
	(173.42, 96.09) --
	(173.39, 96.08) --
	(173.36, 96.07) --
	(173.33, 96.05) --
	(173.29, 96.01) --
	(173.25, 95.97) --
	(173.22, 95.95) --
	(173.17, 95.90) --
	(173.05, 95.82) --
	(172.93, 95.76) --
	(172.89, 95.75) --
	(172.82, 95.72) --
	(172.79, 95.72) --
	(172.75, 95.72) --
	(172.70, 95.72) --
	(172.65, 95.74) --
	(172.57, 95.76) --
	(172.55, 95.78) --
	(172.51, 95.79) --
	(172.46, 95.80) --
	(172.42, 95.81) --
	(172.40, 95.80) --
	(172.37, 95.79) --
	(172.34, 95.76) --
	(172.33, 95.72) --
	(172.33, 95.70) --
	(172.32, 95.66) --
	(172.29, 95.63) --
	(172.18, 95.57) --
	(172.13, 95.54) --
	(172.05, 95.50) --
	(171.99, 95.47) --
	(171.92, 95.44) --
	(171.85, 95.41) --
	(171.78, 95.38) --
	(171.64, 95.32) --
	(171.56, 95.30) --
	(171.50, 95.29) --
	(171.44, 95.28) --
	(171.37, 95.27) --
	(171.29, 95.27) --
	(171.25, 95.27) --
	(171.22, 95.29) --
	(171.18, 95.30) --
	(171.16, 95.30) --
	(171.12, 95.28) --
	(171.10, 95.27) --
	(171.08, 95.24) --
	(171.08, 95.21) --
	(171.08, 95.18) --
	(171.10, 95.07) --
	(171.10, 95.06) --
	(171.09, 95.05) --
	(171.08, 95.04) --
	(171.03, 95.02) --
	(171.00, 95.01) --
	(170.95, 95.00) --
	(170.88, 94.99) --
	(170.76, 94.98) --
	(170.68, 94.97) --
	(170.62, 94.97) --
	(170.59, 94.97) --
	(170.58, 94.95) --
	(170.57, 94.93) --
	(170.56, 94.92) --
	(170.55, 94.89) --
	(170.57, 94.83) --
	(170.57, 94.79) --
	(170.56, 94.77) --
	(170.55, 94.75) --
	(170.52, 94.73) --
	(170.48, 94.71) --
	(170.43, 94.69) --
	(170.39, 94.68) --
	(170.29, 94.66) --
	(170.25, 94.66) --
	(170.17, 94.65) --
	(170.10, 94.63) --
	(169.96, 94.63) --
	(169.91, 94.63) --
	(169.34, 94.58) --
	(169.30, 94.58) --
	(169.26, 94.57) --
	(169.19, 94.52) --
	(169.14, 94.48) --
	(169.04, 94.41) --
	(168.97, 94.36) --
	(168.86, 94.30) --
	(168.80, 94.28) --
	(168.67, 94.23) --
	(168.60, 94.20) --
	(168.52, 94.18) --
	(168.42, 94.17) --
	(168.27, 94.13) --
	(168.24, 94.13) --
	(168.19, 94.10) --
	(168.15, 94.07) --
	(168.12, 94.03) --
	(168.09, 93.99) --
	(168.09, 93.98) --
	(168.08, 93.95) --
	(168.07, 93.84) --
	(168.07, 93.65) --
	(168.06, 93.57) --
	(168.06, 93.28) --
	(168.06, 93.24) --
	(168.05, 93.14) --
	(168.05, 93.10) --
	(168.04, 93.03) --
	(168.02, 92.96) --
	(168.01, 92.92) --
	(167.99, 92.81) --
	(167.97, 92.77) --
	(167.95, 92.70) --
	(167.89, 92.63) --
	(167.74, 92.49) --
	(167.62, 92.40) --
	(167.61, 92.38) --
	(167.58, 92.35) --
	(167.54, 92.33) --
	(167.50, 92.30) --
	(167.47, 92.27) --
	(167.45, 92.24) --
	(167.43, 92.22) --
	(167.43, 92.22) --
	(167.35, 92.18) --
	(167.31, 92.15) --
	(167.27, 92.13) --
	(167.24, 92.10) --
	(167.23, 92.08) --
	(167.20, 92.07) --
	(167.19, 92.06) --
	(167.11, 92.03) --
	(167.06, 92.00) --
	(167.01, 91.93) --
	(166.97, 91.88) --
	(166.93, 91.82) --
	(166.88, 91.76) --
	(166.77, 91.67) --
	(166.72, 91.63) --
	(166.69, 91.58) --
	(166.68, 91.56) --
	(166.67, 91.40) --
	(166.66, 91.36) --
	(166.61, 91.27) --
	(166.50, 91.12) --
	(166.45, 91.06) --
	(166.40, 91.01) --
	(166.35, 90.93) --
	(166.29, 90.88) --
	(166.22, 90.81) --
	(166.03, 90.66) --
	(165.99, 90.63) --
	(165.94, 90.60) --
	(165.87, 90.57) --
	(165.79, 90.54) --
	(165.61, 90.47) --
	(165.51, 90.45) --
	(165.44, 90.42) --
	(165.33, 90.38) --
	(165.26, 90.36) --
	(165.09, 90.33) --
	(164.97, 90.31) --
	(164.86, 90.29) --
	(164.83, 90.28) --
	(164.79, 90.25) --
	(164.76, 90.22) --
	(164.75, 90.19) --
	(164.74, 90.17) --
	(164.74, 90.13) --
	(164.72, 90.02) --
	(164.71, 89.97) --
	(164.69, 89.90) --
	(164.68, 89.87) --
	(164.60, 89.78) --
	(164.56, 89.75) --
	(164.51, 89.71) --
	(164.31, 89.59) --
	(164.25, 89.56) --
	(164.21, 89.54) --
	(164.11, 89.50) --
	(164.06, 89.49) --
	(164.00, 89.48) --
	(163.94, 89.46) --
	(163.87, 89.43) --
	(163.81, 89.41) --
	(163.76, 89.40) --
	(163.62, 89.37) --
	(163.54, 89.36) --
	(163.45, 89.34) --
	(163.41, 89.34) --
	(163.38, 89.33) --
	(163.26, 89.29) --
	(163.23, 89.27) --
	(163.23, 89.29) --
	(163.23, 89.32) --
	(163.18, 89.39) --
	(163.15, 89.42) --
	(163.12, 89.45) --
	(163.10, 89.47) --
	(163.09, 89.49) --
	(163.07, 89.53) --
	(163.06, 89.56) --
	(163.06, 89.66) --
	(163.04, 89.74) --
	(163.04, 89.76) --
	(163.02, 89.78) --
	(163.00, 89.79) --
	(162.97, 89.81) --
	(162.93, 89.84) --
	(162.91, 89.84) --
	(162.87, 89.84) --
	(162.84, 89.84) --
	(162.82, 89.83) --
	(162.79, 89.81) --
	(162.77, 89.79) --
	(162.74, 89.76) --
	(162.72, 89.74) --
	(162.69, 89.72) --
	(162.66, 89.71) --
	(162.56, 89.72) --
	(162.56, 89.71) --
	(162.56, 89.70) --
	(162.56, 89.70) --
	(162.56, 89.69) --
	(162.48, 89.71) --
	(162.44, 89.72) --
	(162.43, 89.73) --
	(162.41, 89.76) --
	(162.40, 89.77) --
	(162.40, 89.80) --
	(162.41, 89.83) --
	(162.40, 89.86) --
	(162.39, 89.88) --
	(162.38, 89.90) --
	(162.39, 89.96) --
	(162.40, 90.00) --
	(162.40, 90.02) --
	(162.40, 90.03) --
	(162.38, 90.06) --
	(162.32, 90.12) --
	(162.31, 90.14) --
	(162.30, 90.15) --
	(162.28, 90.16) --
	(162.27, 90.17) --
	(162.23, 90.18) --
	(162.15, 90.18) --
	(162.03, 90.20) --
	(161.94, 90.23) --
	(161.93, 90.23) --
	(161.89, 90.23) --
	(161.82, 90.21) --
	(161.77, 90.19) --
	(161.60, 90.10) --
	(161.52, 90.08) --
	(161.48, 90.07) --
	(161.45, 90.07) --
	(161.39, 90.08) --
	(161.34, 90.10) --
	(161.27, 90.15) --
	(161.20, 90.21) --
	(161.19, 90.22) --
	(161.15, 90.23) --
	(161.13, 90.23) --
	(161.04, 90.21) --
	(161.01, 90.21) --
	(160.95, 90.21) --
	(160.90, 90.20) --
	(160.83, 90.18) --
	(160.68, 90.16) --
	(160.62, 90.13) --
	(160.57, 90.11) --
	(160.48, 90.03) --
	(160.42, 90.00) --
	(160.31, 90.00) --
	(160.27, 89.99) --
	(160.26, 89.98) --
	(160.23, 89.96) --
	(160.22, 89.93) --
	(160.22, 89.88) --
	(160.21, 89.84) --
	(160.21, 89.82) --
	(160.18, 89.80) --
	(160.15, 89.78) --
	(160.13, 89.78) --
	(160.10, 89.77) --
	(160.07, 89.75) --
	(160.04, 89.74) --
	(159.97, 89.73) --
	(159.89, 89.71) --
	(159.85, 89.70) --
	(159.78, 89.66) --
	(159.74, 89.64) --
	(159.67, 89.61) --
	(159.64, 89.60) --
	(159.60, 89.60) --
	(159.55, 89.60) --
	(159.46, 89.62) --
	(159.43, 89.63) --
	(159.38, 89.64) --
	(159.28, 89.65) --
	(159.25, 89.64) --
	(159.21, 89.63) --
	(159.18, 89.62) --
	(159.15, 89.63) --
	(159.13, 89.63) --
	(159.10, 89.66) --
	(159.08, 89.70) --
	(159.06, 89.72) --
	(158.96, 89.79) --
	(158.91, 89.81) --
	(158.88, 89.82) --
	(158.86, 89.82) --
	(158.77, 89.81) --
	(158.57, 89.80) --
	(158.53, 89.80) --
	(158.56, 89.81) --
	(158.72, 90.01) --
	(158.77, 90.10) --
	(158.79, 90.12) --
	(158.84, 90.19) --
	(158.87, 90.21) --
	(158.94, 90.25) --
	(158.99, 90.27) --
	(159.02, 90.27) --
	(159.03, 90.26) --
	(159.08, 90.24) --
	(159.10, 90.23) --
	(159.12, 90.23) --
	(159.15, 90.25) --
	(159.17, 90.27) --
	(159.19, 90.29) --
	(159.20, 90.32) --
	(159.21, 90.40) --
	(159.24, 90.47) --
	(159.25, 90.57) --
	(159.24, 90.60) --
	(159.23, 90.61) --
	(159.20, 90.63) --
	(159.16, 90.66) --
	(159.13, 90.68) --
	(159.10, 90.71) --
	(159.08, 90.72) --
	(159.04, 90.73) --
	(159.02, 90.75) --
	(159.00, 90.76) --
	(158.94, 90.76) --
	(158.87, 90.77) --
	(158.84, 90.79) --
	(158.79, 90.85) --
	(158.78, 90.86) --
	(158.78, 90.88) --
	(158.78, 90.91) --
	(158.81, 90.96) --
	(158.85, 91.05) --
	(158.86, 91.08) --
	(158.87, 91.10) --
	(158.86, 91.14) --
	(158.84, 91.16) --
	(158.77, 91.21) --
	(158.72, 91.25) --
	(158.64, 91.29) --
	(158.58, 91.32) --
	(158.47, 91.35) --
	(158.39, 91.34) --
	(158.21, 91.34) --
	(158.17, 91.34) --
	(158.14, 91.35) --
	(158.05, 91.40) --
	(157.95, 91.43) --
	(157.90, 91.45) --
	(157.84, 91.46) --
	(157.78, 91.45) --
	(157.73, 91.44) --
	(157.64, 91.41) --
	(157.60, 91.39) --
	(157.57, 91.37) --
	(157.53, 91.34) --
	(157.47, 91.29) --
	(157.45, 91.26) --
	(157.39, 91.20) --
	(157.38, 91.20) --
	(157.34, 91.18) --
	(157.33, 91.18) --
	(157.29, 91.18) --
	(157.15, 91.22) --
	(157.13, 91.22) --
	(157.06, 91.22) --
	(157.02, 91.22) --
	(156.97, 91.23) --
	(156.92, 91.25) --
	(156.84, 91.29) --
	(156.78, 91.32) --
	(156.66, 91.34) --
	(156.63, 91.36) --
	(156.61, 91.39) --
	(156.60, 91.43) --
	(156.59, 91.46) --
	(156.57, 91.49) --
	(156.55, 91.51) --
	(156.54, 91.52) --
	(156.51, 91.53) --
	(156.46, 91.55) --
	(156.43, 91.57) --
	(156.40, 91.60) --
	(156.34, 91.66) --
	(156.31, 91.68) --
	(156.27, 91.71) --
	(156.22, 91.73) --
	(156.18, 91.76) --
	(156.05, 91.87) --
	(156.02, 91.90) --
	(156.00, 91.91) --
	(155.95, 91.92) --
	(155.90, 91.91) --
	(155.86, 91.90) --
	(155.74, 91.82) --
	(155.58, 91.73) --
	(155.47, 91.68) --
	(155.40, 91.64) --
	(155.33, 91.55) --
	(155.29, 91.52) --
	(155.26, 91.50) --
	(155.18, 91.49) --
	(155.07, 91.44) --
	(155.05, 91.44) --
	(155.02, 91.43) --
	(154.98, 91.43) --
	(154.93, 91.45) --
	(154.86, 91.44) --
	(154.84, 91.44) --
	(154.80, 91.45) --
	(154.77, 91.46) --
	(154.74, 91.48) --
	(154.73, 91.49) --
	(154.72, 91.50) --
	(154.70, 91.51) --
	(154.68, 91.51) --
	(154.59, 91.50) --
	(154.51, 91.51) --
	(154.45, 91.54) --
	(154.42, 91.56) --
	(154.41, 91.58) --
	(154.39, 91.64) --
	(154.39, 91.66) --
	(154.34, 91.68) --
	(154.33, 91.69) --
	(154.22, 91.71) --
	(154.16, 91.71) --
	(154.11, 91.71) --
	(153.93, 91.67) --
	(153.87, 91.66) --
	(153.83, 91.66) --
	(153.81, 91.66) --
	(153.71, 91.69) --
	(153.65, 91.73) --
	(153.62, 91.75) --
	(153.61, 91.78) --
	(153.61, 91.80) --
	(153.62, 91.83) --
	(153.64, 91.87) --
	(153.70, 91.96) --
	(153.74, 92.04) --
	(153.74, 92.06) --
	(153.73, 92.09) --
	(153.71, 92.13) --
	(153.68, 92.18) --
	(153.63, 92.24) --
	(153.59, 92.27) --
	(153.56, 92.29) --
	(153.52, 92.30) --
	(153.47, 92.30) --
	(153.43, 92.30) --
	(153.40, 92.29) --
	(153.37, 92.30) --
	(153.32, 92.32) --
	(153.29, 92.32) --
	(153.24, 92.31) --
	(153.04, 92.24) --
	(152.96, 92.21) --
	(152.89, 92.19) --
	(152.75, 92.15) --
	(152.68, 92.14) --
	(152.42, 92.15) --
	(152.19, 92.14) --
	(152.04, 92.14) --
	(152.01, 92.15) --
	(151.94, 92.17) --
	(151.93, 92.19) --
	(151.92, 92.20) --
	(151.93, 92.26) --
	(151.93, 92.28) --
	(151.92, 92.29) --
	(151.89, 92.31) --
	(151.86, 92.33) --
	(151.81, 92.34) --
	(151.75, 92.36) --
	(151.71, 92.38) --
	(151.70, 92.39) --
	(151.67, 92.45) --
	(151.62, 92.49) --
	(151.59, 92.51) --
	(151.58, 92.51) --
	(151.56, 92.53) --
	(151.42, 92.58) --
	(151.36, 92.60) --
	(151.30, 92.62) --
	(151.21, 92.65) --
	(151.13, 92.68) --
	(151.09, 92.68) --
	(151.06, 92.67) --
	(151.02, 92.64) --
	(150.98, 92.63) --
	(150.95, 92.62) --
	(150.91, 92.62) --
	(150.89, 92.62) --
	(150.88, 92.62) --
	(150.85, 92.64) --
	(150.82, 92.66) --
	(150.79, 92.67) --
	(150.75, 92.67) --
	(150.72, 92.66) --
	(150.63, 92.61) --
	(150.57, 92.58) --
	(150.46, 92.56) --
	(150.41, 92.56) --
	(150.28, 92.56) --
	(150.13, 92.59) --
	(149.97, 92.60) --
	(149.94, 92.59) --
	(149.85, 92.51) --
	(149.83, 92.48) --
	(149.79, 92.45) --
	(149.69, 92.38) --
	(149.65, 92.36) --
	(149.59, 92.35) --
	(149.57, 92.35) --
	(149.54, 92.36) --
	(149.52, 92.36) --
	(149.45, 92.38) --
	(149.42, 92.40) --
	(149.40, 92.42) --
	(149.38, 92.45) --
	(149.39, 92.51) --
	(149.40, 92.56) --
	(149.40, 92.61) --
	(149.41, 92.73) --
	(149.40, 92.76) --
	(149.39, 92.79) --
	(149.37, 92.82) --
	(149.34, 92.82) --
	(149.33, 92.82) --
	(149.24, 92.79) --
	(149.13, 92.78) --
	(149.03, 92.77) --
	(148.97, 92.77) --
	(148.82, 92.76) --
	(148.77, 92.75) --
	(148.72, 92.73) --
	(148.65, 92.70) --
	(148.62, 92.70) --
	(148.58, 92.70) --
	(148.51, 92.70) --
	(148.47, 92.71) --
	(148.41, 92.73) --
	(148.35, 92.73) --
	(148.32, 92.74) --
	(148.28, 92.76) --
	(148.25, 92.76) --
	(148.21, 92.76) --
	(148.18, 92.76) --
	(148.13, 92.73) --
	(148.09, 92.72) --
	(148.02, 92.73) --
	(147.89, 92.75) --
	(147.86, 92.74) --
	(147.84, 92.74) --
	(147.82, 92.73) --
	(147.82, 92.71) --
	(147.83, 92.69) --
	(147.86, 92.66) --
	(147.87, 92.63) --
	(147.87, 92.61) --
	(147.84, 92.58) --
	(147.81, 92.57) --
	(147.75, 92.56) --
	(147.67, 92.55) --
	(147.63, 92.53) --
	(147.61, 92.51) --
	(147.59, 92.49) --
	(147.58, 92.47) --
	(147.56, 92.45) --
	(147.43, 92.39) --
	(147.41, 92.37) --
	(147.39, 92.34) --
	(147.38, 92.32) --
	(147.38, 92.23) --
	(147.37, 92.20) --
	(147.35, 92.16) --
	(147.33, 92.07) --
	(147.31, 92.04) --
	(147.29, 92.03) --
	(147.26, 92.02) --
	(147.19, 92.01) --
	(147.16, 92.00) --
	(147.05, 92.05) --
	(147.03, 92.06) --
	(146.99, 92.06) --
	(146.89, 92.02) --
	(146.76, 92.00) --
	(146.70, 92.00) --
	(146.67, 92.00) --
	(146.61, 91.99) --
	(146.56, 91.98) --
	(146.52, 91.96) --
	(146.38, 91.88) --
	(146.33, 91.84) --
	(146.22, 91.72) --
	(146.17, 91.66) --
	(146.14, 91.62) --
	(146.10, 91.57) --
	(146.07, 91.51) --
	(146.01, 91.40) --
	(145.96, 91.28) --
	(145.92, 91.23) --
	(145.86, 91.15) --
	(145.80, 91.05) --
	(145.77, 90.98) --
	(145.75, 90.92) --
	(145.74, 90.87) --
	(145.74, 90.84) --
	(145.72, 90.82) --
	(145.67, 90.77) --
	(145.64, 90.72) --
	(145.62, 90.68) --
	(145.60, 90.64) --
	(145.59, 90.59) --
	(145.60, 90.50) --
	(145.61, 90.43) --
	(145.61, 90.41) --
	(145.58, 90.39) --
	(145.57, 90.39) --
	(145.56, 90.39) --
	(145.53, 90.40) --
	(145.51, 90.41) --
	(145.47, 90.40) --
	(145.46, 90.39) --
	(145.44, 90.38) --
	(145.42, 90.36) --
	(145.41, 90.33) --
	(145.39, 90.31) --
	(145.32, 90.26) --
	(145.29, 90.24) --
	(145.26, 90.20) --
	(145.22, 90.17) --
	(145.19, 90.16) --
	(145.17, 90.15) --
	(145.15, 90.15) --
	(145.13, 90.16) --
	(145.12, 90.16) --
	(145.09, 90.20) --
	(145.06, 90.22) --
	(144.98, 90.23) --
	(144.94, 90.23) --
	(144.93, 90.23) --
	(144.86, 90.26) --
	(144.84, 90.26) --
	(144.82, 90.29) --
	(144.81, 90.31) --
	(144.80, 90.33) --
	(144.80, 90.45) --
	(144.77, 90.55) --
	(144.76, 90.59) --
	(144.77, 90.66) --
	(144.76, 90.70) --
	(144.75, 90.72) --
	(144.71, 90.74) --
	(144.67, 90.76) --
	(144.56, 90.79) --
	(144.47, 90.79) --
	(144.46, 90.79) --
	(144.42, 90.81) --
	(144.40, 90.83) --
	(144.39, 90.86) --
	(144.38, 90.89) --
	(144.39, 90.95) --
	(144.39, 91.01) --
	(144.39, 91.03) --
	(144.39, 91.07) --
	(144.40, 91.11) --
	(144.39, 91.14) --
	(144.39, 91.15) --
	(144.38, 91.17) --
	(144.36, 91.17) --
	(144.13, 91.21) --
	(144.04, 91.23) --
	(143.97, 91.24) --
	(143.89, 91.24) --
	(143.84, 91.24) --
	(143.75, 91.24) --
	(143.64, 91.24) --
	(143.59, 91.26) --
	(143.55, 91.27) --
	(143.39, 91.28) --
	(143.32, 91.30) --
	(143.24, 91.34) --
	(143.16, 91.40) --
	(143.06, 91.48) --
	(142.98, 91.58) --
	(142.92, 91.67) --
	(142.80, 91.82) --
	(142.74, 91.86) --
	(142.72, 91.87) --
	(142.70, 91.87) --
	(142.65, 91.86) --
	(142.61, 91.86) --
	(142.55, 91.89) --
	(142.48, 91.92) --
	(142.34, 91.94) --
	(142.18, 91.97) --
	(142.14, 91.98) --
	(142.13, 91.99) --
	(142.11, 92.00) --
	(142.11, 92.01) --
	(142.12, 92.02) --
	(142.11, 92.02) --
	(142.09, 92.04) --
	(142.03, 92.10) --
	(142.00, 92.13) --
	(141.96, 92.15) --
	(141.94, 92.16) --
	(141.91, 92.16) --
	(141.69, 92.16) --
	(141.67, 92.16) --
	(141.65, 92.15) --
	(141.62, 92.12) --
	(141.58, 92.09) --
	(141.55, 92.09) --
	(141.53, 92.09) --
	(141.50, 92.10) --
	(141.48, 92.11) --
	(141.38, 92.22) --
	(141.36, 92.23) --
	(141.33, 92.24) --
	(141.31, 92.24) --
	(141.25, 92.19) --
	(141.23, 92.18) --
	(141.21, 92.18) --
	(141.17, 92.19) --
	(141.16, 92.21) --
	(141.10, 92.36) --
	(141.06, 92.43) --
	(141.04, 92.45) --
	(141.02, 92.45) --
	(140.95, 92.47) --
	(140.82, 92.46) --
	(140.76, 92.48) --
	(140.68, 92.48) --
	(140.66, 92.49) --
	(140.61, 92.50) --
	(140.59, 92.52) --
	(140.58, 92.54) --
	(140.57, 92.53) --
	(140.52, 92.57) --
	(140.31, 92.61) --
	(140.26, 92.62) --
	(140.22, 92.65) --
	(140.17, 92.72) --
	(140.12, 92.76) --
	(140.10, 92.77) --
	(140.07, 92.80) --
	(140.06, 92.82) --
	(140.04, 92.86) --
	(140.03, 92.93) --
	(140.03, 92.99) --
	(139.98, 93.13) --
	(139.96, 93.19) --
	(139.96, 93.22) --
	(139.96, 93.26) --
	(139.97, 93.34) --
	(139.97, 93.36) --
	(139.96, 93.40) --
	(139.90, 93.49) --
	(139.87, 93.53) --
	(139.87, 93.57) --
	(139.88, 93.61) --
	(139.91, 93.66) --
	(139.92, 93.68) --
	(139.93, 93.75) --
	(139.90, 93.75) --
	(139.86, 93.78) --
	(139.85, 93.79) --
	(139.82, 93.82) --
	(139.79, 93.88) --
	(139.77, 93.92) --
	(139.74, 93.94) --
	(139.72, 93.95) --
	(139.71, 94.04) --
	(139.70, 94.06) --
	(139.69, 94.08) --
	(139.68, 94.09) --
	(139.64, 94.12) --
	(139.62, 94.14) --
	(139.56, 94.20) --
	(139.55, 94.21) --
	(139.50, 94.23) --
	(139.42, 94.25) --
	(139.39, 94.28) --
	(139.37, 94.32) --
	(139.38, 94.41) --
	(139.40, 94.49) --
	(139.40, 94.53) --
	(139.40, 94.57) --
	(139.39, 94.61) --
	(139.38, 94.63) --
	(139.34, 94.66) --
	(139.32, 94.67) --
	(139.29, 94.70) --
	(139.28, 94.71) --
	(139.19, 94.86) --
	(139.16, 94.89) --
	(139.14, 94.90) --
	(139.13, 94.92) --
	(139.06, 94.97) --
	(138.99, 95.03) --
	(138.97, 95.06) --
	(138.95, 95.10) --
	(138.94, 95.12) --
	(138.94, 95.16) --
	(138.95, 95.20) --
	(138.96, 95.24) --
	(138.97, 95.28) --
	(139.00, 95.33) --
	(139.05, 95.42) --
	(139.07, 95.46) --
	(139.07, 95.66) --
	(139.08, 95.68) --
	(139.07, 95.70) --
	(139.08, 95.72) --
	(139.08, 95.74) --
	(139.08, 95.76) --
	(139.09, 95.78) --
	(139.10, 95.80) --
	(139.12, 95.88) --
	(139.12, 95.94) --
	(139.04, 96.01) --
	(139.02, 96.05) --
	(139.00, 96.09) --
	(139.01, 96.13) --
	(139.02, 96.16) --
	(139.05, 96.20) --
	(139.10, 96.24) --
	(139.13, 96.28) --
	(139.18, 96.32) --
	(139.21, 96.34) --
	(139.32, 96.40) --
	(139.45, 96.49) --
	(139.57, 96.55) --
	(139.64, 96.58) --
	(139.78, 96.69) --
	(139.81, 96.72) --
	(139.84, 96.73) --
	(139.86, 96.72) --
	(139.89, 96.72) --
	(139.91, 96.71) --
	(139.93, 96.71) --
	(139.96, 96.71) --
	(139.98, 96.72) --
	(140.01, 96.73) --
	(140.03, 96.76) --
	(140.08, 96.83) --
	(140.12, 96.91) --
	(140.13, 96.96) --
	(140.16, 97.04) --
	(140.20, 97.20) --
	(140.21, 97.22) --
	(140.28, 97.34) --
	(140.29, 97.36) --
	(140.28, 97.40) --
	(140.27, 97.54) --
	(140.28, 97.58) --
	(140.31, 97.62) --
	(140.32, 97.63) --
	(140.44, 97.67) --
	(140.56, 97.76) --
	(140.61, 97.79) --
	(140.65, 97.80) --
	(140.68, 97.79) --
	(140.76, 97.78) --
	(140.81, 97.77) --
	(140.84, 97.77) --
	(140.86, 97.78) --
	(140.91, 97.79) --
	(140.95, 97.80) --
	(141.03, 97.80) --
	(141.10, 97.80) --
	(141.17, 97.78) --
	(141.37, 97.71) --
	(141.47, 97.66) --
	(141.55, 97.61) --
	(141.59, 97.59) --
	(141.63, 97.57) --
	(141.66, 97.56) --
	(141.73, 97.56) --
	(141.81, 97.57) --
	(141.84, 97.60) --
	(141.87, 97.63) --
	(141.88, 97.65) --
	(141.88, 97.67) --
	(141.83, 97.74) --
	(141.78, 97.83) --
	(141.76, 97.89) --
	(141.77, 97.91) --
	(141.79, 97.97) --
	(141.82, 98.00) --
	(141.84, 98.01) --
	(141.86, 98.03) --
	(141.86, 98.05) --
	(141.85, 98.11) --
	(141.78, 98.22) --
	(141.76, 98.25) --
	(141.75, 98.31) --
	(141.74, 98.35) --
	(141.73, 98.39) --
	(141.71, 98.43) --
	(141.68, 98.46) --
	(141.68, 98.48) --
	(141.68, 98.50) --
	(141.74, 98.61) --
	(141.81, 98.76) --
	(141.84, 98.79) --
	(141.90, 98.82) --
	(141.93, 98.83) --
	(141.98, 98.83) --
	(142.00, 98.84) --
	(142.03, 98.84) --
	(142.06, 98.87) --
	(142.07, 98.89) --
	(142.11, 98.94) --
	(142.11, 98.97) --
	(142.11, 99.00) --
	(142.12, 99.01) --
	(142.12, 99.08) --
	(142.11, 99.18) --
	(142.10, 99.20) --
	(142.10, 99.23) --
	(142.09, 99.26) --
	(142.10, 99.28) --
	(142.09, 99.30) --
	(142.08, 99.36) --
	(142.07, 99.40) --
	(142.08, 99.42) --
	(142.08, 99.44) --
	(142.08, 99.46) --
	(142.08, 99.47) --
	(142.09, 99.50) --
	(142.08, 99.53) --
	(142.07, 99.54) --
	(142.07, 99.55) --
	(142.07, 99.57) --
	(142.07, 99.62) --
	(142.08, 99.62) --
	(142.11, 99.64) --
	(142.12, 99.65) --
	(142.16, 99.67) --
	(142.19, 99.70) --
	(142.22, 99.73) --
	(142.24, 99.77) --
	(142.24, 99.78) --
	(142.23, 99.81) --
	(142.22, 99.82) --
	(142.19, 99.83) --
	(142.17, 99.85) --
	(142.12, 99.90) --
	(142.09, 99.92) --
	(142.07, 99.92) --
	(142.04, 99.92) --
	(142.02, 99.91) --
	(141.97, 99.91) --
	(141.90, 99.92) --
	(141.85, 99.93) --
	(141.74, 99.97) --
	(141.70, 99.99) --
	(141.66,100.01) --
	(141.64,100.03) --
	(141.62,100.06) --
	(141.61,100.09) --
	(141.62,100.09) --
	(141.61,100.10) --
	(141.50,100.12) --
	(141.45,100.14) --
	(141.43,100.15) --
	(141.39,100.17) --
	(141.36,100.21) --
	(141.36,100.23) --
	(141.35,100.24) --
	(141.35,100.27) --
	(141.36,100.29) --
	(141.37,100.32) --
	(141.37,100.33) --
	(141.42,100.37) --
	(141.48,100.41) --
	(141.59,100.46) --
	(141.61,100.47) --
	(141.64,100.51) --
	(141.65,100.55) --
	(141.66,100.62) --
	(141.68,100.65) --
	(141.69,100.68) --
	(141.70,100.71) --
	(141.70,100.73) --
	(141.80,100.88) --
	(141.83,100.91) --
	(141.87,100.97) --
	(141.88,101.00) --
	(141.88,101.05) --
	(141.85,101.12) --
	(141.82,101.15) --
	(141.80,101.16) --
	(141.76,101.19) --
	(141.67,101.23) --
	(141.63,101.25) --
	(141.58,101.27) --
	(141.54,101.30) --
	(141.49,101.34) --
	(141.46,101.37) --
	(141.44,101.41) --
	(141.43,101.45) --
	(141.44,101.65) --
	(141.44,101.70) --
	(141.41,101.70) --
	(141.31,101.71) --
	(141.24,101.74) --
	(141.19,101.74) --
	(141.12,101.75) --
	(141.04,101.76) --
	(140.99,101.76) --
	(140.94,101.75) --
	(140.87,101.73) --
	(140.82,101.72) --
	(140.78,101.69) --
	(140.75,101.66) --
	(140.75,101.63) --
	(140.73,101.60) --
	(140.70,101.57) --
	(140.50,101.32) --
	(140.50,101.29) --
	(140.49,101.28) --
	(140.41,101.16) --
	(140.34,101.00) --
	(140.33,100.98) --
	(140.32,100.94) --
	(140.32,100.90) --
	(140.33,100.89) --
	(140.32,100.87) --
	(140.34,100.81) --
	(140.34,100.79) --
	(140.33,100.78) --
	(140.31,100.75) --
	(140.28,100.74) --
	(140.28,100.72) --
	(140.25,100.71) --
	(140.15,100.62) --
	(140.13,100.58) --
	(140.10,100.56) --
	(140.07,100.53) --
	(140.05,100.52) --
	(140.05,100.51) --
	(139.98,100.45) --
	(139.97,100.43) --
	(139.95,100.41) --
	(139.93,100.40) --
	(139.88,100.39) --
	(139.83,100.38) --
	(139.81,100.38) --
	(139.75,100.34) --
	(139.71,100.30) --
	(139.70,100.27) --
	(139.68,100.17) --
	(139.64,100.09) --
	(139.65,100.09) --
	(139.65,100.09) --
	(139.58,100.03) --
	(139.56,100.01) --
	(139.54,100.00) --
	(139.49, 99.99) --
	(139.44, 99.99) --
	(139.34,100.01) --
	(139.27,100.00) --
	(139.23, 99.97) --
	(139.15, 99.92) --
	(139.09, 99.89) --
	(139.05, 99.87) --
	(139.02, 99.81) --
	(139.00, 99.78) --
	(138.98, 99.76) --
	(138.96, 99.75) --
	(138.94, 99.74) --
	(138.91, 99.74) --
	(138.79, 99.73) --
	(138.77, 99.73) --
	(138.62, 99.66) --
	(138.57, 99.65) --
	(138.43, 99.63) --
	(138.40, 99.64) --
	(138.38, 99.64) --
	(138.34, 99.67) --
	(138.29, 99.71) --
	(138.24, 99.78) --
	(138.18, 99.82) --
	(138.14, 99.83) --
	(138.06, 99.84) --
	(138.01, 99.84) --
	(137.93, 99.83) --
	(137.87, 99.83) --
	(137.72, 99.81) --
	(137.67, 99.82) --
	(137.60, 99.81) --
	(137.52, 99.82) --
	(137.50, 99.83) --
	(137.46, 99.86) --
	(137.42, 99.88) --
	(137.40, 99.92) --
	(137.35, 99.98) --
	(137.31,100.07) --
	(137.30,100.09) --
	(137.30,100.09) --
	(137.31,100.09) --
	(137.31,100.09) --
	(137.31,100.10) --
	(137.22,100.27) --
	(137.14,100.40) --
	(137.09,100.49) --
	(137.01,100.64) --
	(136.98,100.68) --
	(136.98,100.69) --
	(136.87,100.86) --
	(136.82,100.95) --
	(136.79,100.97) --
	(136.78,100.97) --
	(136.74,100.95) --
	(136.72,100.94) --
	(136.70,100.94) --
	(136.57,100.87) --
	(136.53,100.84) --
	(136.49,100.82) --
	(136.30,100.72) --
	(136.12,100.60) --
	(136.06,100.57) --
	(136.01,100.55) --
	(135.89,100.48) --
	(135.82,100.42) --
	(135.76,100.36) --
	(135.74,100.32) --
	(135.72,100.30) --
	(135.68,100.21) --
	(135.66,100.12) --
	(135.66,100.09) --
	(135.65,100.09) --
	(135.65,100.00) --
	(135.65, 99.96) --
	(135.69, 99.82) --
	(135.69, 99.78) --
	(135.68, 99.76) --
	(135.65, 99.73) --
	(135.57, 99.68) --
	(135.54, 99.65) --
	(135.52, 99.61) --
	(135.45, 99.49) --
	(135.42, 99.46) --
	(135.37, 99.41) --
	(135.36, 99.39) --
	(135.34, 99.38) --
	(135.26, 99.26) --
	(135.16, 99.12) --
	(135.11, 99.05) --
	(135.02, 98.89) --
	(134.97, 98.78) --
	(134.95, 98.74) --
	(134.92, 98.71) --
	(134.90, 98.69) --
	(134.88, 98.69) --
	(134.85, 98.68) --
	(134.73, 98.69) --
	(134.66, 98.68) --
	(134.48, 98.62) --
	(134.46, 98.61) --
	(134.39, 98.59) --
	(134.34, 98.58) --
	(134.30, 98.57) --
	(134.25, 98.57) --
	(134.12, 98.59) --
	(134.07, 98.59) --
	(134.03, 98.58) --
	(133.98, 98.57) --
	(133.96, 98.56) --
	(133.93, 98.53) --
	(133.91, 98.49) --
	(133.82, 98.35) --
	(133.76, 98.29) --
	(133.71, 98.24) --
	(133.65, 98.21) --
	(133.51, 98.13) --
	(133.46, 98.08) --
	(133.38, 97.98) --
	(133.37, 97.97) --
	(133.34, 97.94) --
	(133.10, 97.80) --
	(133.05, 97.75) --
	(133.03, 97.72) --
	(133.01, 97.66) --
	(133.00, 97.60) --
	(132.98, 97.46) --
	(132.97, 97.45) --
	(132.92, 97.34) --
	(132.88, 97.35) --
	(132.86, 97.39) --
	(132.85, 97.43) --
	(132.85, 97.50) --
	(132.83, 97.60) --
	(132.82, 97.62) --
	(132.81, 97.64) --
	(132.80, 97.65) --
	(132.77, 97.67) --
	(132.75, 97.67) --
	(132.70, 97.65) --
	(132.45, 97.53) --
	(132.40, 97.51) --
	(132.26, 97.48) --
	(132.09, 97.48) --
	(132.04, 97.47) --
	(132.00, 97.45) --
	(131.98, 97.44) --
	(131.95, 97.41) --
	(131.89, 97.35) --
	(131.85, 97.32) --
	(131.79, 97.29) --
	(131.74, 97.28) --
	(131.67, 97.26) --
	(131.63, 97.24) --
	(131.56, 97.22) --
	(131.54, 97.22) --
	(131.51, 97.23) --
	(131.50, 97.24) --
	(131.49, 97.25) --
	(131.43, 97.31) --
	(131.41, 97.37) --
	(131.39, 97.43) --
	(131.38, 97.43) --
	(131.36, 97.44) --
	(131.24, 97.46) --
	(131.03, 97.47) --
	(130.83, 97.48) --
	(130.73, 97.48) --
	(130.51, 97.49) --
	(130.27, 97.50) --
	(130.15, 97.51) --
	(130.08, 97.52) --
	(129.59, 97.55) --
	(129.49, 97.55) --
	(129.47, 97.54) --
	(129.43, 97.52) --
	(129.39, 97.49) --
	(129.28, 97.44) --
	(129.25, 97.42) --
	(129.18, 97.36) --
	(129.11, 97.30) --
	(128.81, 97.08) --
	(128.72, 97.00) --
	(128.66, 96.97) --
	(128.60, 96.93) --
	(128.36, 96.83) --
	(128.33, 96.81) --
	(128.29, 96.78) --
	(128.26, 96.75) --
	(128.19, 96.62) --
	(128.14, 96.55) --
	(128.12, 96.54) --
	(128.08, 96.49) --
	(128.01, 96.44) --
	(127.93, 96.38) --
	(127.83, 96.30) --
	(127.62, 96.13) --
	(127.42, 95.90) --
	(127.34, 95.81) --
	(127.25, 95.69) --
	(127.18, 95.63) --
	(127.15, 95.60) --
	(127.05, 95.48) --
	(127.01, 95.43) --
	(126.97, 95.39) --
	(126.87, 95.32) --
	(126.85, 95.31) --
	(126.81, 95.31) --
	(126.79, 95.32) --
	(126.74, 95.36) --
	(126.70, 95.41) --
	(126.69, 95.45) --
	(126.67, 95.51) --
	(126.66, 95.57) --
	(126.63, 95.60) --
	(126.60, 95.63) --
	(126.55, 95.64) --
	(126.53, 95.64) --
	(126.50, 95.64) --
	(126.43, 95.61) --
	(126.35, 95.57) --
	(126.15, 95.50) --
	(126.07, 95.45) --
	(125.99, 95.40) --
	(125.89, 95.31) --
	(125.82, 95.23) --
	(125.73, 95.14) --
	(125.71, 95.12) --
	(125.72, 95.12) --
	(125.72, 95.09) --
	(125.70, 95.08) --
	(125.64, 95.01) --
	(125.61, 94.97) --
	(125.53, 94.90) --
	(125.47, 94.87) --
	(125.42, 94.86) --
	(125.39, 94.86) --
	(125.37, 94.86) --
	(125.35, 94.87) --
	(125.28, 94.89) --
	(125.26, 94.91) --
	(125.21, 94.95) --
	(125.17, 95.00) --
	(125.14, 95.02) --
	(125.10, 95.05) --
	(125.05, 95.07) --
	(124.93, 95.07) --
	(124.91, 95.08) --
	(124.86, 95.10) --
	(124.83, 95.12) --
	(124.75, 95.20) --
	(124.73, 95.21) --
	(124.66, 95.29) --
	(124.64, 95.33) --
	(124.59, 95.40) --
	(124.56, 95.42) --
	(124.55, 95.42) --
	(124.52, 95.42) --
	(124.45, 95.41) --
	(124.43, 95.40) --
	(124.38, 95.39) --
	(124.34, 95.37) --
	(124.30, 95.34) --
	(124.27, 95.31) --
	(124.21, 95.23) --
	(124.19, 95.17) --
	(124.13, 95.04) --
	(124.11, 94.98) --
	(124.01, 94.87) --
	(123.85, 94.58) --
	(123.81, 94.53) --
	(123.79, 94.49) --
	(123.72, 94.43) --
	(123.67, 94.39) --
	(123.43, 94.22) --
	(123.30, 94.13) --
	(123.24, 94.10) --
	(123.17, 94.10) --
	(123.09, 94.10) --
	(122.97, 94.08) --
	(122.92, 94.07) --
	(122.88, 94.05) --
	(122.86, 94.04) --
	(122.82, 94.01) --
	(122.77, 93.97) --
	(122.71, 93.93) --
	(122.69, 93.93) --
	(122.67, 93.91) --
	(122.65, 93.91) --
	(122.59, 93.86) --
	(122.52, 93.76) --
	(122.50, 93.74) --
	(122.48, 93.73) --
	(122.46, 93.72) --
	(122.44, 93.71) --
	(122.39, 93.71) --
	(122.37, 93.71) --
	(122.15, 93.73) --
	(122.07, 93.74) --
	(122.05, 93.74) --
	(122.03, 93.76) --
	(122.00, 93.79) --
	(121.93, 93.89) --
	(121.91, 93.91) --
	(121.90, 93.92) --
	(121.85, 93.94) --
	(121.76, 94.00) --
	(121.73, 94.03) --
	(121.70, 94.09) --
	(121.66, 94.16) --
	(121.66, 94.20) --
	(121.66, 94.22) --
	(121.66, 94.26) --
	(121.67, 94.32) --
	(121.68, 94.38) --
	(121.69, 94.46) --
	(121.68, 94.48) --
	(121.64, 94.50) --
	(121.62, 94.52) --
	(121.58, 94.55) --
	(121.54, 94.59) --
	(121.49, 94.64) --
	(121.47, 94.65) --
	(121.43, 94.66) --
	(121.38, 94.67) --
	(121.33, 94.66) --
	(121.29, 94.64) --
	(120.98, 94.39) --
	(120.96, 94.35) --
	(120.93, 94.32) --
	(120.88, 94.25) --
	(120.85, 94.09) --
	(120.83, 94.05) --
	(120.82, 94.04) --
	(120.77, 93.99) --
	(120.74, 93.96) --
	(120.73, 93.94) --
	(120.72, 93.92) --
	(120.67, 93.83) --
	(120.65, 93.81) --
	(120.63, 93.80) --
	(120.57, 93.78) --
	(120.54, 93.77) --
	(120.52, 93.77) --
	(120.45, 93.79) --
	(120.43, 93.79) --
	(120.30, 93.78) --
	(120.28, 93.79) --
	(120.25, 93.79) --
	(120.21, 93.81) --
	(120.15, 93.85) --
	(120.13, 93.86) --
	(120.08, 93.97) --
	(120.06, 94.01) --
	(120.02, 94.03) --
	(120.00, 94.04) --
	(119.95, 94.06) --
	(119.88, 94.06) --
	(119.35, 93.99) --
	(119.28, 93.98) --
	(119.23, 93.98) --
	(119.20, 93.99) --
	(119.16, 94.01) --
	(119.09, 94.03) --
	(119.03, 94.06) --
	(118.98, 94.10) --
	(118.82, 94.20) --
	(118.75, 94.25) --
	(118.68, 94.31) --
	(118.64, 94.33) --
	(118.61, 94.36) --
	(118.56, 94.38) --
	(118.55, 94.39) --
	(118.50, 94.42) --
	(118.49, 94.43) --
	(118.45, 94.44) --
	(118.43, 94.46) --
	(118.36, 94.52) --
	(118.35, 94.54) --
	(118.34, 94.55) --
	(118.31, 94.57) --
	(118.30, 94.58) --
	(118.28, 94.61) --
	(118.28, 94.64) --
	(118.27, 94.65) --
	(118.27, 94.78) --
	(118.26, 94.81) --
	(118.26, 94.83) --
	(118.25, 94.85) --
	(118.25, 94.87) --
	(118.26, 94.89) --
	(118.27, 94.93) --
	(118.27, 94.94) --
	(118.27, 94.95) --
	(118.24, 94.98) --
	(118.22, 95.00) --
	(118.17, 95.02) --
	(118.13, 95.03) --
	(118.10, 95.03) --
	(118.05, 95.03) --
	(117.96, 95.05) --
	(117.93, 95.05) --
	(117.79, 95.04) --
	(117.74, 95.04) --
	(117.60, 95.08) --
	(117.58, 95.09) --
	(117.55, 95.10) --
	(117.42, 95.18) --
	(117.33, 95.25) --
	(117.29, 95.28) --
	(117.21, 95.31) --
	(117.03, 95.37) --
	(116.99, 95.40) --
	(116.95, 95.43) --
	(116.94, 95.44) --
	(116.92, 95.45) --
	(116.90, 95.49) --
	(116.84, 95.63) --
	(116.80, 95.75) --
	(116.77, 95.83) --
	(116.73, 96.07) --
	(116.73, 96.15) --
	(116.72, 96.17) --
	(116.70, 96.24) --
	(116.69, 96.28) --
	(116.68, 96.32) --
	(116.69, 96.34) --
	(116.70, 96.40) --
	(116.76, 96.51) --
	(116.78, 96.57) --
	(116.79, 96.61) --
	(116.81, 96.62) --
	(116.83, 96.64) --
	(116.85, 96.65) --
	(116.90, 96.65) --
	(117.04, 96.63) --
	(117.09, 96.63) --
	(117.16, 96.65) --
	(117.18, 96.66) --
	(117.20, 96.68) --
	(117.24, 96.71) --
	(117.26, 96.74) --
	(117.26, 96.76) --
	(117.26, 96.80) --
	(117.26, 96.84) --
	(117.26, 96.88) --
	(117.21, 97.01) --
	(117.12, 97.13) --
	(117.04, 97.21) --
	(117.01, 97.24) --
	(116.97, 97.26) --
	(116.95, 97.27) --
	(116.91, 97.30) --
	(116.90, 97.31) --
	(116.91, 97.34) --
	(116.89, 97.42) --
	(116.69, 97.69) --
	(116.67, 97.74) --
	(116.65, 97.80) --
	(116.63, 97.90) --
	(116.61, 98.00) --
	(116.55, 98.13) --
	(116.51, 98.22) --
	(116.48, 98.34) --
	(116.41, 98.63) --
	(116.40, 98.67) --
	(116.39, 98.70) --
	(116.38, 98.74) --
	(116.36, 98.78) --
	(116.36, 98.84) --
	(116.35, 98.91) --
	(116.35, 98.95) --
	(116.34, 99.00) --
	(116.31, 99.05) --
	(116.27, 99.10) --
	(116.14, 99.21) --
	(116.10, 99.26) --
	(116.07, 99.32) --
	(116.01, 99.51) --
	(115.99, 99.53) --
	(115.95, 99.58) --
	(115.90, 99.64) --
	(115.86, 99.72) --
	(115.78, 99.79) --
	(115.76, 99.83) --
	(115.73, 99.86) --
	(115.72, 99.88) --
	(115.66, 99.91) --
	(115.63, 99.94) --
	(115.61,100.00) --
	(115.54,100.09) --
	(115.56,100.09) --
	(115.55,100.09) --
	(115.55,100.09) --
	(115.54,100.09) --
	(115.54,100.09) --
	(115.48,100.18) --
	(115.42,100.27) --
	(115.40,100.31) --
	(115.38,100.35) --
	(115.38,100.41) --
	(115.37,100.45) --
	(115.37,100.57) --
	(115.40,100.65) --
	(115.40,100.67) --
	(115.39,100.71) --
	(115.38,100.73) --
	(115.36,100.77) --
	(115.32,100.80) --
	(115.28,100.82) --
	(115.22,100.85) --
	(115.11,100.90) --
	(115.07,100.90) --
	(115.02,100.95) --
	(114.98,100.97) --
	(114.91,100.99) --
	(114.85,101.00) --
	(114.75,101.00) --
	(114.73,101.01) --
	(114.69,101.02) --
	(114.67,101.04) --
	(114.61,101.08) --
	(114.58,101.11) --
	(114.56,101.13) --
	(114.55,101.14) --
	(114.51,101.22) --
	(114.44,101.41) --
	(114.42,101.48) --
	(114.39,101.53) --
	(114.39,101.56) --
	(114.38,101.59) --
	(114.36,101.70) --
	(114.33,101.84) --
	(114.28,101.99) --
	(114.27,102.06) --
	(114.26,102.26) --
	(114.24,102.31) --
	(114.24,102.33) --
	(114.23,102.37) --
	(114.22,102.41) --
	(114.21,102.43) --
	(114.22,102.55) --
	(114.26,102.69) --
	(114.29,102.75) --
	(114.35,102.84) --
	(114.38,102.87) --
	(114.44,102.91) --
	(114.50,102.95) --
	(114.59,102.99) --
	(114.68,103.02) --
	(114.75,103.03) --
	(114.82,103.05) --
	(114.87,103.07) --
	(114.90,103.09) --
	(114.92,103.11) --
	(114.93,103.13) --
	(114.95,103.14) --
	(114.95,103.16) --
	(114.96,103.17) --
	(114.97,103.27) --
	(114.96,103.35) --
	(114.96,103.41) --
	(114.97,103.47) --
	(114.97,103.51) --
	(115.00,103.59) --
	(115.02,103.65) --
	(115.05,103.71) --
	(115.09,103.78) --
	(115.11,103.82) --
	(115.12,103.84) --
	(115.21,103.94) --
	(115.40,104.08) --
	(115.51,104.19) --
	(115.54,104.22) --
	(115.56,104.23) --
	(115.61,104.28) --
	(115.73,104.38) --
	(115.76,104.41) --
	(115.78,104.43) --
	(115.79,104.45) --
	(115.85,104.54) --
	(115.86,104.56) --
	(115.91,104.58) --
	(116.00,104.65) --
	(116.12,104.72) --
	(116.18,104.74) --
	(116.20,104.74) --
	(116.28,104.74) --
	(116.30,104.74) --
	(116.32,104.74) --
	(116.37,104.75) --
	(116.40,104.76) --
	(116.43,104.78) --
	(116.49,104.85) --
	(116.53,104.91) --
	(116.65,105.01) --
	(116.67,105.02) --
	(116.75,105.07) --
	(116.77,105.09) --
	(116.79,105.10) --
	(116.86,105.13) --
	(116.88,105.14) --
	(116.89,105.15) --
	(116.90,105.16) --
	(116.93,105.18) --
	(116.97,105.21) --
	(116.97,105.23) --
	(116.98,105.27) --
	(116.97,105.29) --
	(116.98,105.33) --
	(116.99,105.35) --
	(117.01,105.39) --
	(117.04,105.47) --
	(117.06,105.50) --
	(117.06,105.52) --
	(117.06,105.54) --
	(117.03,105.57) --
	(117.01,105.58) --
	(117.00,105.60) --
	(116.99,105.62) --
	(116.99,105.66) --
	(117.06,105.75) --
	(117.15,105.82) --
	(117.26,105.93) --
	(117.38,106.12) --
	(117.45,106.20) --
	(117.49,106.25) --
	(117.52,106.30) --
	(117.54,106.32) --
	(117.56,106.35) --
	(117.61,106.42) --
	(117.64,106.45) --
	(117.68,106.50) --
	(117.72,106.55) --
	(117.73,106.57) --
	(117.77,106.65) --
	(117.80,106.75) --
	(117.83,106.80) --
	(117.85,106.84) --
	(117.88,106.94) --
	(117.88,106.96) --
	(117.89,106.98) --
	(117.89,107.00) --
	(117.90,107.04) --
	(117.90,107.27) --
	(117.88,107.46) --
	(117.86,107.56) --
	(117.85,107.60) --
	(117.83,107.63) --
	(117.82,107.65) --
	(117.79,107.68) --
	(117.74,107.73) --
	(117.72,107.74) --
	(117.69,107.80) --
	(117.67,107.86) --
	(117.63,107.95) --
	(117.62,107.97) --
	(117.60,107.98) --
	(117.60,107.99) --
	(117.56,108.04) --
	(117.54,108.08) --
	(117.49,108.19) --
	(117.47,108.29) --
	(117.47,108.31) --
	(117.46,108.35) --
	(117.44,108.41) --
	(117.43,108.50) --
	(117.44,108.57) --
	(117.45,108.59) --
	(117.46,108.68) --
	(117.47,108.82) --
	(117.46,108.86) --
	(117.44,108.92) --
	(117.35,109.02) --
	(117.32,109.13) --
	(117.32,109.17) --
	(117.31,109.19) --
	(117.31,109.21) --
	(117.32,109.23) --
	(117.32,109.25) --
	(117.28,109.31) --
	(117.23,109.35) --
	(117.19,109.38) --
	(117.14,109.39) --
	(117.12,109.40) --
	(117.01,109.40) --
	(116.86,109.41) --
	(116.84,109.40) --
	(116.81,109.40) --
	(116.76,109.41) --
	(116.74,109.42) --
	(116.66,109.41) --
	(116.64,109.40) --
	(116.60,109.38) --
	(116.57,109.37) --
	(116.55,109.36) --
	(116.50,109.34) --
	(116.45,109.33) --
	(116.44,109.33) --
	(116.40,109.33) --
	(116.38,109.34) --
	(116.31,109.36) --
	(116.18,109.43) --
	(116.14,109.48) --
	(116.12,109.49) --
	(116.10,109.53) --
	(116.09,109.55) --
	(116.06,109.58) --
	(116.03,109.59) --
	(115.99,109.61) --
	(115.96,109.61) --
	(115.84,109.60) --
	(115.76,109.59) --
	(115.58,109.62) --
	(115.56,109.61) --
	(115.53,109.60) --
	(115.46,109.59) --
	(115.43,109.58) --
	(115.41,109.58) --
	(115.39,109.58) --
	(115.31,109.58) --
	(115.26,109.60) --
	(115.22,109.62) --
	(115.20,109.63) --
	(115.17,109.67) --
	(115.14,109.70) --
	(115.11,109.73) --
	(115.07,109.76) --
	(115.00,109.84) --
	(114.98,109.88) --
	(114.96,109.92) --
	(114.96,109.96) --
	(114.95,110.02) --
	(114.95,110.06) --
	(114.94,110.08) --
	(114.93,110.10) --
	(114.95,110.18) --
	(114.97,110.22) --
	(114.99,110.30) --
	(114.99,110.35) --
	(115.01,110.38) --
	(115.01,110.40) --
	(115.00,110.42) --
	(114.98,110.44) --
	(114.96,110.45) --
	(114.90,110.46) --
	(114.83,110.44) --
	(114.82,110.42) --
	(114.79,110.41) --
	(114.72,110.39) --
	(114.70,110.39) --
	(114.60,110.43) --
	(114.56,110.45) --
	(114.51,110.47) --
	(114.47,110.49) --
	(114.44,110.50) --
	(114.43,110.51) --
	(114.36,110.60) --
	(114.33,110.63) --
	(114.31,110.67) --
	(114.29,110.71) --
	(114.28,110.73) --
	(114.21,110.85) --
	(114.19,110.87) --
	(114.17,110.91) --
	(114.14,110.97) --
	(114.12,110.99) --
	(114.11,111.00) --
	(114.08,111.00) --
	(114.03,111.01) --
	(113.98,111.02) --
	(113.96,111.03) --
	(113.91,111.05) --
	(113.89,111.06) --
	(113.85,111.11) --
	(113.82,111.15) --
	(113.78,111.18) --
	(113.74,111.20) --
	(113.70,111.22) --
	(113.63,111.24) --
	(113.55,111.25) --
	(113.40,111.25) --
	(113.32,111.25) --
	(113.30,111.26) --
	(113.27,111.26) --
	(113.26,111.28) --
	(113.23,111.29) --
	(113.23,111.30) --
	(113.20,111.30) --
	(113.18,111.29) --
	(113.11,111.24) --
	(113.08,111.24) --
	(113.06,111.24) --
	(113.05,111.25) --
	(113.03,111.26) --
	(113.00,111.31) --
	(112.99,111.34) --
	(112.99,111.36) --
	(113.00,111.38) --
	(113.02,111.42) --
	(113.04,111.46) --
	(113.06,111.52) --
	(113.07,111.55) --
	(113.07,111.57) --
	(113.04,111.59) --
	(113.02,111.60) --
	(112.91,111.63) --
	(112.88,111.65) --
	(112.82,111.67) --
	(112.75,111.71) --
	(112.73,111.72) --
	(112.69,111.73) --
	(112.63,111.73) --
	(112.60,111.73) --
	(112.59,111.76) --
	(112.57,111.79) --
	(112.53,111.84) --
	(112.48,111.89) --
	(112.43,111.92) --
	(112.36,111.93) --
	(112.33,111.93) --
	(112.31,111.92) --
	(112.31,111.91) --
	(112.33,111.89) --
	(112.34,111.83) --
	(112.36,111.73) --
	(112.36,111.69) --
	(112.39,111.57) --
	(112.40,111.54) --
	(112.38,111.49) --
	(112.38,111.48) --
	(112.35,111.45) --
	(112.30,111.43) --
	(112.25,111.40) --
	(112.22,111.40) --
	(112.20,111.40) --
	(112.14,111.41) --
	(112.11,111.43) --
	(112.07,111.48) --
	(112.04,111.50) --
	(112.01,111.50) --
	(111.95,111.50) --
	(111.92,111.50) --
	(111.87,111.51) --
	(111.85,111.51) --
	(111.83,111.51) --
	(111.82,111.50) --
	(111.85,111.41) --
	(111.85,111.38) --
	(111.85,111.37) --
	(111.84,111.36) --
	(111.78,111.30) --
	(111.71,111.27) --
	(111.65,111.27) --
	(111.63,111.26) --
	(111.59,111.25) --
	(111.58,111.24) --
	(111.55,111.24) --
	(111.54,111.24) --
	(111.52,111.26) --
	(111.50,111.28) --
	(111.43,111.31) --
	(111.39,111.34) --
	(111.35,111.39) --
	(111.34,111.41) --
	(111.31,111.42) --
	(111.30,111.42) --
	(111.21,111.49) --
	(111.18,111.55) --
	(111.17,111.57) --
	(111.15,111.57) --
	(111.15,111.60) --
	(111.17,111.69) --
	(111.17,111.73) --
	(111.16,111.77) --
	(111.15,111.79) --
	(111.13,111.83) --
	(111.12,111.88) --
	(111.12,111.89) --
	(111.10,111.89) --
	(111.09,111.88) --
	(111.08,111.87) --
	(111.01,111.84) --
	(110.99,111.84) --
	(110.89,111.86) --
	(110.87,111.87) --
	(110.86,111.87) --
	(110.84,111.86) --
	(110.84,111.85) --
	(110.84,111.84) --
	(110.85,111.82) --
	(110.86,111.81) --
	(110.88,111.75) --
	(110.88,111.73) --
	(110.88,111.72) --
	(110.86,111.71) --
	(110.80,111.70) --
	(110.77,111.69) --
	(110.75,111.70) --
	(110.73,111.71) --
	(110.71,111.71) --
	(110.69,111.71) --
	(110.67,111.71) --
	(110.60,111.66) --
	(110.56,111.65) --
	(110.51,111.66) --
	(110.46,111.67) --
	(110.43,111.67) --
	(110.35,111.67) --
	(110.32,111.64) --
	(110.30,111.63) --
	(110.28,111.63) --
	(110.24,111.64) --
	(110.18,111.66) --
	(110.15,111.66) --
	(110.13,111.66) --
	(110.08,111.67) --
	(110.05,111.68) --
	(110.01,111.70) --
	(109.99,111.72) --
	(109.97,111.75) --
	(109.96,111.76) --
	(109.92,111.77) --
	(109.91,111.79) --
	(109.91,111.80) --
	(109.93,111.86) --
	(109.94,111.90) --
	(109.94,111.95) --
	(109.94,111.98) --
	(109.93,112.00) --
	(109.85,112.13) --
	(109.80,112.19) --
	(109.76,112.23) --
	(109.74,112.25) --
	(109.72,112.25) --
	(109.68,112.23) --
	(109.67,112.22) --
	(109.64,112.20) --
	(109.62,112.18) --
	(109.59,112.18) --
	(109.55,112.17) --
	(109.51,112.17) --
	(109.48,112.15) --
	(109.43,112.14) --
	(109.38,112.13) --
	(109.34,112.13) --
	(109.31,112.16) --
	(109.29,112.18) --
	(109.27,112.24) --
	(109.24,112.25) --
	(109.16,112.30) --
	(109.10,112.36) --
	(109.01,112.39) --
	(108.96,112.40) --
	(108.90,112.42) --
	(108.84,112.45) --
	(108.81,112.47) --
	(108.78,112.52) --
	(108.75,112.54) --
	(108.74,112.58) --
	(108.72,112.61) --
	(108.69,112.66) --
	(108.69,112.69) --
	(108.69,112.71) --
	(108.72,112.77) --
	(108.72,112.79) --
	(108.72,112.80) --
	(108.71,112.81) --
	(108.69,112.82) --
	(108.65,112.83) --
	(108.56,112.83) --
	(108.49,112.82) --
	(108.47,112.82) --
	(108.46,112.82) --
	(108.45,112.84) --
	(108.40,112.89) --
	(108.38,112.92) --
	(108.38,112.93) --
	(108.38,112.95) --
	(108.39,112.97) --
	(108.40,113.01) --
	(108.42,113.06) --
	(108.42,113.09) --
	(108.42,113.10) --
	(108.41,113.11) --
	(108.40,113.12) --
	(108.38,113.13) --
	(108.29,113.13) --
	(108.24,113.13) --
	(108.12,113.14) --
	(108.08,113.15) --
	(108.06,113.15) --
	(108.04,113.17) --
	(108.03,113.19) --
	(107.96,113.32) --
	(107.96,113.36) --
	(107.98,113.40) --
	(107.97,113.42) --
	(107.97,113.45) --
	(107.98,113.47) --
	(108.00,113.48) --
	(108.03,113.50) --
	(108.06,113.53) --
	(108.07,113.55) --
	(108.07,113.57) --
	(108.07,113.58) --
	(108.02,113.66) --
	(108.02,113.69) --
	(108.02,113.72) --
	(108.04,113.74) --
	(108.06,113.75) --
	(108.07,113.77) --
	(108.09,113.80) --
	(108.09,113.83) --
	(108.10,113.86) --
	(108.10,113.88) --
	(108.23,113.95) --
	(108.24,113.96) --
	(108.25,113.98) --
	(108.24,113.99) --
	(108.23,114.02) --
	(108.25,114.16) --
	(108.24,114.19) --
	(108.23,114.21) --
	(108.19,114.26) --
	(108.17,114.28) --
	(108.17,114.38) --
	(108.16,114.42) --
	(108.14,114.46) --
	(108.13,114.48) --
	(108.12,114.50) --
	(108.10,114.50) --
	(108.02,114.52) --
	(107.94,114.54) --
	(107.90,114.57) --
	(107.83,114.58) --
	(107.79,114.59) --
	(107.75,114.61) --
	(107.73,114.62) --
	(107.69,114.62) --
	(107.66,114.62) --
	(107.64,114.62) --
	(107.63,114.64) --
	(107.58,114.67) --
	(107.48,114.74) --
	(107.46,114.76) --
	(107.43,114.81) --
	(107.39,114.89) --
	(107.37,114.91) --
	(107.34,114.93) --
	(107.32,114.95) --
	(107.30,115.01) --
	(107.30,115.04) --
	(107.30,115.07) --
	(107.30,115.10) --
	(107.30,115.14) --
	(107.31,115.30) --
	(107.31,115.33) --
	(107.30,115.35) --
	(107.31,115.40) --
	(107.30,115.44) --
	(107.24,115.46) --
	(107.21,115.49) --
	(107.19,115.52) --
	(107.17,115.55) --
	(107.14,115.58) --
	(107.12,115.62) --
	(107.09,115.65) --
	(107.03,115.70) --
	(106.99,115.73) --
	(106.94,115.75) --
	(106.92,115.77) --
	(106.91,115.79) --
	(106.91,115.81) --
	(106.92,115.84) --
	(106.90,115.89) --
	(106.89,115.98) --
	(106.88,116.00) --
	(106.87,116.02) --
	(106.83,116.05) --
	(106.81,116.07) --
	(106.80,116.12) --
	(106.79,116.13) --
	(106.77,116.15) --
	(106.75,116.15) --
	(106.61,116.12) --
	(106.57,116.11) --
	(106.53,116.11) --
	(106.51,116.12) --
	(106.43,116.14) --
	(106.41,116.15) --
	(106.40,116.14) --
	(106.36,116.12) --
	(106.36,116.10) --
	(106.36,116.09) --
	(106.39,116.03) --
	(106.40,116.00) --
	(106.41,115.98) --
	(106.40,115.96) --
	(106.40,115.95) --
	(106.38,115.94) --
	(106.37,115.93) --
	(106.34,115.93) --
	(106.24,115.94) --
	(106.17,115.96) --
	(106.10,116.00) --
	(106.08,116.01) --
	(106.02,116.01) --
	(106.01,116.01) --
	(105.99,116.02) --
	(105.96,116.03) --
	(105.92,116.02) --
	(105.90,116.02) --
	(105.88,116.00) --
	(105.86,115.98) --
	(105.84,115.97) --
	(105.77,115.94) --
	(105.75,115.94) --
	(105.72,115.94) --
	(105.70,115.96) --
	(105.62,115.98) --
	(105.60,115.99) --
	(105.59,116.01) --
	(105.59,116.03) --
	(105.57,116.05) --
	(105.52,116.09) --
	(105.45,116.12) --
	(105.42,116.14) --
	(105.39,116.19) --
	(105.35,116.24) --
	(105.32,116.26) --
	(105.30,116.27) --
	(105.26,116.28) --
	(105.22,116.30) --
	(105.16,116.35) --
	(105.13,116.36) --
	(105.10,116.36) --
	(105.02,116.29) --
	(104.99,116.28) --
	(104.92,116.23) --
	(104.88,116.21) --
	(104.86,116.19) --
	(104.83,116.19) --
	(104.80,116.17) --
	(104.75,116.15) --
	(104.72,116.14) --
	(104.69,116.13) --
	(104.61,116.12) --
	(104.59,116.12) --
	(104.49,116.11) --
	(104.45,116.10) --
	(104.38,116.08) --
	(104.32,116.07) --
	(104.29,116.07) --
	(104.15,116.13) --
	(104.08,116.15) --
	(103.99,116.15) --
	(103.90,116.18) --
	(103.86,116.19) --
	(103.82,116.20) --
	(103.80,116.21) --
	(103.70,116.20) --
	(103.65,116.22) --
	(103.60,116.22) --
	(103.57,116.23) --
	(103.53,116.23) --
	(103.47,116.21) --
	(103.40,116.16) --
	(103.37,116.14) --
	(103.34,116.11) --
	(103.31,116.08) --
	(103.29,116.06) --
	(103.25,116.01) --
	(103.22,115.98) --
	(103.14,115.88) --
	(103.11,115.86) --
	(103.09,115.83) --
	(103.07,115.82) --
	(103.00,115.78) --
	(102.91,115.67) --
	(102.90,115.65) --
	(102.88,115.64) --
	(102.85,115.62) --
	(102.79,115.60) --
	(102.77,115.59) --
	(102.76,115.57) --
	(102.72,115.54) --
	(102.69,115.50) --
	(102.67,115.46) --
	(102.63,115.42) --
	(102.61,115.40) --
	(102.59,115.40) --
	(102.54,115.43) --
	(102.51,115.43) --
	(102.49,115.44) --
	(102.46,115.43) --
	(102.44,115.42) --
	(102.41,115.39) --
	(102.34,115.35) --
	(102.32,115.32) --
	(102.26,115.29) --
	(102.21,115.25) --
	(102.16,115.20) --
	(102.15,115.18) --
	(102.13,115.13) --
	(102.09,115.02) --
	(102.06,114.97) --
	(102.05,114.95) --
	(102.03,114.81) --
	(102.02,114.78) --
	(101.99,114.71) --
	(101.97,114.68) --
	(101.96,114.65) --
	(101.96,114.60) --
	(101.96,114.58) --
	(101.93,114.52) --
	(101.91,114.49) --
	(101.89,114.46) --
	(101.89,114.43) --
	(101.89,114.40) --
	(101.89,114.38) --
	(101.88,114.35) --
	(101.86,114.33) --
	(101.82,114.25) --
	(101.78,114.16) --
	(101.76,114.11) --
	(101.75,114.06) --
	(101.74,114.03) --
	(101.72,113.93) --
	(101.72,113.91) --
	(101.71,113.88) --
	(101.70,113.85) --
	(101.65,113.80) --
	(101.63,113.77) --
	(101.61,113.74) --
	(101.57,113.71) --
	(101.54,113.69) --
	(101.48,113.66) --
	(101.45,113.64) --
	(101.42,113.61) --
	(101.39,113.60) --
	(101.36,113.58) --
	(101.26,113.56) --
	(101.17,113.53) --
	(101.13,113.51) --
	(101.10,113.49) --
	(101.08,113.49) --
	(101.05,113.48) --
	(101.04,113.49) --
	(101.01,113.52) --
	(101.00,113.54) --
	(100.98,113.57) --
	(100.95,113.63) --
	(100.92,113.67) --
	(100.87,113.75) --
	(100.86,113.84) --
	(100.88,113.87) --
	(100.88,113.91) --
	(100.89,114.00) --
	(100.91,114.07) --
	(100.92,114.09) --
	(100.91,114.12) --
	(100.89,114.14) --
	(100.86,114.20) --
	(100.82,114.26) --
	(100.82,114.29) --
	(100.83,114.30) --
	(100.84,114.32) --
	(100.89,114.37) --
	(100.93,114.40) --
	(100.95,114.42) --
	(100.98,114.46) --
	(101.02,114.50) --
	(101.14,114.66) --
	(101.15,114.68) --
	(101.14,114.69) --
	(101.13,114.71) --
	(101.11,114.75) --
	(101.09,114.77) --
	(101.07,114.87) --
	(101.06,114.89) --
	(101.05,114.91) --
	(100.98,115.00) --
	(100.92,115.09) --
	(100.90,115.12) --
	(100.87,115.15) --
	(100.86,115.18) --
	(100.85,115.22) --
	(100.84,115.25) --
	(100.84,115.30) --
	(100.84,115.33) --
	(100.80,115.37) --
	(100.80,115.38) --
	(100.78,115.44) --
	(100.78,115.48) --
	(100.77,115.50) --
	(100.74,115.51) --
	(100.73,115.53) --
	(100.72,115.56) --
	(100.70,115.64) --
	(100.70,115.71) --
	(100.68,115.76) --
	(100.67,115.80) --
	(100.66,115.83) --
	(100.65,115.87) --
	(100.61,115.93) --
	(100.59,115.97) --
	(100.58,115.99) --
	(100.57,116.02) --
	(100.57,116.11) --
	(100.57,116.16) --
	(100.58,116.19) --
	(100.58,116.21) --
	(100.57,116.30) --
	(100.54,116.35) --
	(100.53,116.36) --
	(100.51,116.37) --
	(100.48,116.39) --
	(100.45,116.41) --
	(100.40,116.43) --
	(100.37,116.45) --
	(100.33,116.48) --
	(100.31,116.50) --
	(100.30,116.52) --
	(100.26,116.60) --
	(100.24,116.63) --
	(100.22,116.65) --
	(100.18,116.66) --
	(100.16,116.66) --
	(100.11,116.65) --
	(100.08,116.64) --
	(100.05,116.65) --
	(100.03,116.65) --
	(100.01,116.66) --
	( 99.95,116.70) --
	( 99.92,116.73) --
	( 99.90,116.76) --
	( 99.89,116.81) --
	( 99.87,116.85) --
	( 99.86,116.88) --
	( 99.84,116.92) --
	( 99.83,116.94) --
	( 99.81,116.96) --
	( 99.78,116.98) --
	( 99.64,117.03) --
	( 99.59,117.04) --
	( 99.56,117.07) --
	( 99.54,117.09) --
	( 99.53,117.12) --
	( 99.53,117.14) --
	( 99.56,117.21) --
	( 99.62,117.26) --
	( 99.65,117.30) --
	( 99.72,117.35) --
	( 99.75,117.40) --
	( 99.76,117.43) --
	( 99.75,117.44) --
	( 99.73,117.46) --
	( 99.71,117.48) --
	( 99.64,117.48) --
	( 99.61,117.49) --
	( 99.58,117.49) --
	( 99.55,117.48) --
	( 99.52,117.48) --
	( 99.50,117.47) --
	( 99.48,117.48) --
	( 99.43,117.50) --
	( 99.41,117.52) --
	( 99.39,117.54) --
	( 99.38,117.57) --
	( 99.37,117.61) --
	( 99.35,117.66) --
	( 99.33,117.70) --
	( 99.30,117.73) --
	( 99.28,117.75) --
	( 99.24,117.78) --
	( 99.21,117.78) --
	( 99.17,117.80) --
	( 99.08,117.82) --
	( 99.05,117.83) --
	( 99.01,117.85) --
	( 98.99,117.87) --
	( 98.97,117.89) --
	( 98.94,117.96) --
	( 98.91,118.02) --
	( 98.90,118.06) --
	( 98.89,118.08) --
	( 98.84,118.10) --
	( 98.83,118.12) --
	( 98.87,118.15) --
	( 98.88,118.17) --
	( 98.88,118.19) --
	( 98.88,118.22) --
	( 98.86,118.25) --
	( 98.85,118.29) --
	( 98.84,118.31) --
	( 98.83,118.34) --
	( 98.84,118.36) --
	( 98.85,118.37) --
	( 98.87,118.38) --
	( 98.91,118.38) --
	( 99.00,118.39) --
	( 99.01,118.40) --
	( 99.02,118.41) --
	( 99.02,118.43) --
	( 98.99,118.47) --
	( 98.98,118.49) --
	( 98.98,118.51) --
	( 99.01,118.59) --
	( 99.04,118.60) --
	( 99.12,118.61) --
	( 99.13,118.61) --
	( 99.14,118.62) --
	( 99.15,118.65) --
	( 99.15,118.68) --
	( 99.17,118.79) --
	( 99.18,118.82) --
	( 99.19,118.85) --
	( 99.24,118.88) --
	( 99.28,118.94) --
	( 99.29,118.96) --
	( 99.29,118.99) --
	( 99.28,119.02) --
	( 99.26,119.04) --
	( 99.24,119.08) --
	( 99.22,119.10) --
	( 99.18,119.14) --
	( 99.16,119.15) --
	( 99.14,119.18) --
	( 99.13,119.20) --
	( 99.13,119.23) --
	( 99.13,119.31) --
	( 99.13,119.35) --
	( 99.15,119.44) --
	( 99.18,119.58) --
	( 99.20,119.62) --
	( 99.19,119.67) --
	( 99.16,119.69) --
	( 99.13,119.70) --
	( 99.11,119.70) --
	( 99.08,119.71) --
	( 99.06,119.72) --
	( 99.04,119.73) --
	( 99.01,119.73) --
	( 98.98,119.73) --
	( 98.95,119.73) --
	( 98.93,119.72) --
	( 98.90,119.71) --
	( 98.86,119.68) --
	( 98.83,119.67) --
	( 98.80,119.67) --
	( 98.78,119.67) --
	( 98.73,119.67) --
	( 98.71,119.67) --
	( 98.69,119.66) --
	( 98.66,119.63) --
	( 98.60,119.55) --
	( 98.58,119.53) --
	( 98.54,119.51) --
	( 98.49,119.45) --
	( 98.44,119.42) --
	( 98.39,119.38) --
	( 98.38,119.35) --
	( 98.37,119.33) --
	( 98.37,119.31) --
	( 98.36,119.29) --
	( 98.28,119.22) --
	( 98.24,119.19) --
	( 98.22,119.18) --
	( 98.20,119.18) --
	( 98.13,119.19) --
	( 98.10,119.19) --
	( 98.07,119.19) --
	( 98.01,119.17) --
	( 97.99,119.18) --
	( 97.97,119.18) --
	( 97.95,119.20) --
	( 97.92,119.23) --
	( 97.89,119.26) --
	( 97.84,119.29) --
	( 97.81,119.30) --
	( 97.77,119.31) --
	( 97.73,119.32) --
	( 97.70,119.31) --
	( 97.66,119.31) --
	( 97.62,119.31) --
	( 97.57,119.32) --
	( 97.54,119.33) --
	( 97.50,119.36) --
	( 97.48,119.38) --
	( 97.48,119.40) --
	( 97.48,119.50) --
	( 97.48,119.56) --
	( 97.48,119.58) --
	( 97.47,119.60) --
	( 97.45,119.60) --
	( 97.43,119.60) --
	( 97.42,119.60) --
	( 97.35,119.56) --
	( 97.27,119.53) --
	( 97.13,119.49) --
	( 97.07,119.48) --
	( 97.01,119.46) --
	( 96.97,119.44) --
	( 96.94,119.43) --
	( 96.92,119.42) --
	( 96.90,119.41) --
	( 96.87,119.37) --
	( 96.82,119.32) --
	( 96.80,119.29) --
	( 96.75,119.24) --
	( 96.73,119.22) --
	( 96.68,119.20) --
	( 96.66,119.19) --
	( 96.64,119.17) --
	( 96.62,119.16) --
	( 96.63,119.15) --
	( 96.64,119.11) --
	( 96.64,119.06) --
	( 96.61,118.96) --
	( 96.61,118.91) --
	( 96.60,118.87) --
	( 96.61,118.84) --
	( 96.60,118.81) --
	( 96.60,118.74) --
	( 96.58,118.68) --
	( 96.57,118.65) --
	( 96.57,118.62) --
	( 96.57,118.55) --
	( 96.56,118.51) --
	( 96.56,118.50) --
	( 96.52,118.46) --
	( 96.43,118.41) --
	( 96.42,118.37) --
	( 96.41,118.34) --
	( 96.40,118.31) --
	( 96.38,118.28) --
	( 96.36,118.26) --
	( 96.30,118.16) --
	( 96.26,118.13) --
	( 96.21,118.07) --
	( 96.17,118.02) --
	( 96.17,117.98) --
	( 96.15,117.95) --
	( 96.11,117.91) --
	( 96.03,117.84) --
	( 96.01,117.82) --
	( 95.96,117.75) --
	( 95.95,117.73) --
	( 95.88,117.68) --
	( 95.85,117.65) --
	( 95.83,117.64) --
	( 95.81,117.62) --
	( 95.80,117.59) --
	( 95.79,117.51) --
	( 95.78,117.48) --
	( 95.70,117.40) --
	( 95.67,117.37) --
	( 95.64,117.35) --
	( 95.55,117.28) --
	( 95.51,117.24) --
	( 95.48,117.21) --
	( 95.44,117.13) --
	( 95.42,117.09) --
	( 95.41,117.05) --
	( 95.37,117.02) --
	( 95.32,116.97) --
	( 95.29,116.95) --
	( 95.25,116.92) --
	( 95.14,116.87) --
	( 95.11,116.85) --
	( 95.07,116.83) --
	( 95.02,116.80) --
	( 94.92,116.75) --
	( 94.87,116.71) --
	( 94.84,116.69) --
	( 94.82,116.66) --
	( 94.76,116.57) --
	( 94.73,116.54) --
	( 94.70,116.50) --
	( 94.65,116.40) --
	( 94.58,116.29) --
	( 94.57,116.26) --
	( 94.53,116.22) --
	( 94.50,116.18) --
	( 94.40,116.04) --
	( 94.37,115.95) --
	( 94.34,115.85) --
	( 94.29,115.73) --
	( 94.26,115.69) --
	( 94.23,115.66) --
	( 94.21,115.65) --
	( 94.10,115.53) --
	( 94.02,115.46) --
	( 93.99,115.42) --
	( 93.96,115.38) --
	( 93.84,115.25) --
	( 93.79,115.20) --
	( 93.75,115.18) --
	( 93.72,115.16) --
	( 93.70,115.14) --
	( 93.65,115.12) --
	( 93.56,115.12) --
	( 93.52,115.11) --
	( 93.48,115.08) --
	( 93.47,115.06) --
	( 93.45,115.04) --
	( 93.41,114.96) --
	( 93.40,114.94) --
	( 93.39,114.93) --
	( 93.38,114.93) --
	( 93.27,114.95) --
	( 93.20,114.96) --
	( 93.09,114.97) --
	( 92.98,114.98) --
	( 92.93,114.99) --
	( 92.90,115.00) --
	( 92.85,115.02) --
	( 92.83,115.02) --
	( 92.77,115.02) --
	( 92.71,115.02) --
	( 92.68,115.02) --
	( 92.66,115.02) --
	( 92.61,115.02) --
	( 92.53,114.99) --
	( 92.40,114.96) --
	( 92.34,114.95) --
	( 92.30,114.93) --
	( 92.28,114.92) --
	( 92.27,114.90) --
	( 92.25,114.86) --
	( 92.22,114.79) --
	( 92.21,114.76) --
	( 92.15,114.71) --
	( 92.13,114.70) --
	( 92.11,114.70) --
	( 91.99,114.69) --
	( 91.91,114.68) --
	( 91.85,114.64) --
	( 91.78,114.59) --
	( 91.75,114.57) --
	( 91.71,114.56) --
	( 91.69,114.56) --
	( 91.58,114.56) --
	( 91.48,114.57) --
	( 91.44,114.57) --
	( 91.41,114.56) --
	( 91.33,114.55) --
	( 91.31,114.55) --
	( 91.24,114.57) --
	( 91.15,114.61) --
	( 91.08,114.61) --
	( 90.96,114.61) --
	( 90.91,114.61) --
	( 90.80,114.61) --
	( 90.75,114.62) --
	( 90.66,114.66) --
	( 90.63,114.66) --
	( 90.59,114.66) --
	( 90.57,114.66) --
	( 90.53,114.64) --
	( 90.49,114.61) --
	( 90.45,114.59) --
	( 90.35,114.56) --
	( 90.32,114.53) --
	( 90.29,114.50) --
	( 90.25,114.42) --
	( 90.23,114.25) --
	( 90.22,114.20) --
	( 90.21,114.17) --
	( 90.21,114.14) --
	( 90.20,114.11) --
	( 90.19,114.09) --
	( 90.18,114.07) --
	( 90.17,114.04) --
	( 90.14,113.99) --
	( 90.08,113.89) --
	( 90.05,113.84) --
	( 90.03,113.80) --
	( 90.00,113.78) --
	( 89.99,113.77) --
	( 89.93,113.77) --
	( 89.90,113.76) --
	( 89.86,113.75) --
	( 89.81,113.73) --
	( 89.78,113.71) --
	( 89.74,113.67) --
	( 89.72,113.66) --
	( 89.70,113.64) --
	( 89.69,113.62) --
	( 89.69,113.60) --
	( 89.68,113.56) --
	( 89.68,113.54) --
	( 89.63,113.49) --
	( 89.58,113.42) --
	( 89.57,113.40) --
	( 89.44,113.36) --
	( 89.28,113.31) --
	( 89.23,113.29) --
	( 88.99,113.21) --
	( 88.90,113.16) --
	( 88.80,113.10) --
	( 88.71,113.08) --
	( 88.66,113.08) --
	( 88.53,113.11) --
	( 88.44,113.12) --
	( 88.41,113.12) --
	( 88.39,113.12) --
	( 88.36,113.12) --
	( 88.32,113.09) --
	( 88.24,113.06) --
	( 88.21,113.03) --
	( 88.15,112.99) --
	( 88.13,112.97) --
	( 88.05,112.90) --
	( 88.01,112.86) --
	( 87.96,112.84) --
	( 87.89,112.80) --
	( 87.85,112.77) --
	( 87.79,112.74) --
	( 87.77,112.73) --
	( 87.74,112.73) --
	( 87.72,112.73) --
	( 87.69,112.72) --
	( 87.65,112.69) --
	( 87.63,112.68) --
	( 87.61,112.65) --
	( 87.58,112.63) --
	( 87.54,112.60) --
	( 87.46,112.56) --
	( 87.39,112.53) --
	( 87.31,112.51) --
	( 87.28,112.50) --
	( 87.25,112.50) --
	( 87.14,112.52) --
	( 87.03,112.52) --
	( 87.00,112.52) --
	( 86.97,112.53) --
	( 86.94,112.54) --
	( 86.88,112.55) --
	( 86.87,112.56) --
	( 86.80,112.62) --
	( 86.74,112.67) --
	( 86.66,112.71) --
	( 86.64,112.71) --
	( 86.56,112.71) --
	( 86.46,112.70) --
	( 86.42,112.70) --
	( 86.28,112.69) --
	( 86.23,112.70) --
	( 86.17,112.69) --
	( 86.11,112.69) --
	( 86.08,112.69) --
	( 85.95,112.70) --
	( 85.77,112.73) --
	( 85.74,112.74) --
	( 85.63,112.76) --
	( 85.51,112.76) --
	( 85.35,112.77) --
	( 85.25,112.77) --
	( 85.22,112.77) --
	( 85.15,112.80) --
	( 85.11,112.81) --
	( 85.08,112.82) --
	( 85.05,112.83) --
	( 85.02,112.84) --
	( 84.92,112.90) --
	( 84.84,112.96) --
	( 84.76,113.01) --
	( 84.74,113.04) --
	( 84.73,113.05) --
	( 84.70,113.06) --
	( 84.67,113.06) --
	( 84.65,113.07) --
	( 84.64,113.09) --
	( 84.63,113.11) --
	( 84.59,113.16) --
	( 84.55,113.19) --
	( 84.54,113.20) --
	( 84.53,113.21) --
	( 84.53,113.24) --
	( 84.52,113.25) --
	( 84.51,113.26) --
	( 84.50,113.27) --
	( 84.50,113.31) --
	( 84.49,113.33) --
	( 84.48,113.33) --
	( 84.47,113.34) --
	( 84.45,113.34) --
	( 84.44,113.33) --
	( 84.38,113.32) --
	( 84.34,113.30) --
	( 84.30,113.28) --
	( 84.26,113.28) --
	( 84.23,113.28) --
	( 84.21,113.28) --
	( 84.18,113.29) --
	( 84.14,113.31) --
	( 84.04,113.36) --
	( 84.01,113.38) --
	( 83.98,113.38) --
	( 83.95,113.39) --
	( 83.88,113.38) --
	( 83.85,113.38) --
	( 83.81,113.36) --
	( 83.74,113.31) --
	( 83.72,113.28) --
	( 83.69,113.27) --
	( 83.66,113.26) --
	( 83.62,113.26) --
	( 83.59,113.27) --
	( 83.56,113.27) --
	( 83.50,113.26) --
	( 83.47,113.27) --
	( 83.40,113.27) --
	( 83.36,113.28) --
	( 83.24,113.30) --
	( 83.19,113.32) --
	( 83.16,113.33) --
	( 83.11,113.35) --
	( 83.09,113.36) --
	( 83.06,113.35) --
	( 83.01,113.33) --
	( 82.91,113.26) --
	( 82.88,113.25) --
	( 82.86,113.24) --
	( 82.84,113.24) --
	( 82.83,113.25) --
	( 82.78,113.26) --
	( 82.77,113.27) --
	( 82.74,113.32) --
	( 82.71,113.33) --
	( 82.68,113.33) --
	( 82.66,113.33) --
	( 82.64,113.33) --
	( 82.59,113.30) --
	( 82.56,113.30) --
	( 82.53,113.29) --
	( 82.49,113.28) --
	( 82.25,113.28) --
	( 82.21,113.28) --
	( 82.18,113.27) --
	( 82.15,113.24) --
	( 82.11,113.22) --
	( 82.06,113.21) --
	( 82.04,113.22) --
	( 82.02,113.23) --
	( 82.01,113.24) --
	( 82.00,113.28) --
	( 81.98,113.31) --
	( 81.98,113.33) --
	( 81.94,113.34) --
	( 81.92,113.34) --
	( 81.90,113.33) --
	( 81.81,113.24) --
	( 81.79,113.21) --
	( 81.77,113.19) --
	( 81.73,113.13) --
	( 81.73,113.10) --
	( 81.70,113.05) --
	( 81.69,113.03) --
	( 81.68,113.00) --
	( 81.65,112.97) --
	( 81.62,112.93) --
	( 81.59,112.90) --
	( 81.51,112.85) --
	( 81.47,112.84) --
	( 81.44,112.81) --
	( 81.38,112.80) --
	( 81.33,112.79) --
	( 81.27,112.79) --
	( 81.19,112.78) --
	( 81.16,112.78) --
	( 81.10,112.80) --
	( 81.04,112.82) --
	( 80.97,112.84) --
	( 80.95,112.84) --
	( 80.93,112.84) --
	( 80.91,112.83) --
	( 80.90,112.82) --
	( 80.89,112.81) --
	( 80.87,112.75) --
	( 80.86,112.73) --
	( 80.85,112.72) --
	( 80.84,112.71) --
	( 80.83,112.69) --
	( 80.81,112.66) --
	( 80.72,112.59) --
	( 80.65,112.55) --
	( 80.62,112.53) --
	( 80.60,112.52) --
	( 80.58,112.50) --
	( 80.52,112.46) --
	( 80.49,112.44) --
	( 80.46,112.41) --
	( 80.39,112.39) --
	( 80.35,112.38) --
	( 80.29,112.36) --
	( 80.25,112.32) --
	( 80.23,112.30) --
	( 80.21,112.27) --
	( 80.17,112.20) --
	( 80.13,112.14) --
	( 80.07,112.05) --
	( 80.04,112.03) --
	( 80.02,112.03) --
	( 79.99,112.03) --
	( 79.93,112.05) --
	( 79.87,112.08) --
	( 79.83,112.11) --
	( 79.73,112.17) --
	( 79.68,112.19) --
	( 79.66,112.22) --
	( 79.62,112.26) --
	( 79.56,112.30) --
	( 79.54,112.33) --
	( 79.53,112.35) --
	( 79.53,112.38) --
	( 79.51,112.41) --
	( 79.50,112.46) --
	( 79.50,112.54) --
	( 79.51,112.64) --
	( 79.50,112.73) --
	( 79.50,112.78) --
	( 79.49,112.86) --
	( 79.48,112.89) --
	( 79.44,113.03) --
	( 79.43,113.07) --
	( 79.41,113.12) --
	( 79.39,113.19) --
	( 79.39,113.23) --
	( 79.35,113.38) --
	( 79.34,113.41) --
	( 79.32,113.44) --
	( 79.26,113.59) --
	( 79.20,113.77) --
	( 79.18,113.80) --
	( 79.17,113.84) --
	( 79.12,113.96) --
	( 79.07,114.06) --
	( 79.04,114.12) --
	( 79.03,114.17) --
	( 79.01,114.21) --
	( 79.00,114.23) --
	( 79.00,114.27) --
	( 79.00,114.28) --
	( 78.99,114.31) --
	( 78.95,114.47) --
	( 78.94,114.51) --
	( 78.88,114.64) --
	( 78.82,114.76) --
	( 78.81,114.80) --
	( 78.78,114.82) --
	( 78.75,114.83) --
	( 78.72,114.82) --
	( 78.67,114.80) --
	( 78.61,114.79) --
	( 78.59,114.79) --
	( 78.57,114.80) --
	( 78.56,114.80) --
	( 78.52,114.82) --
	( 78.48,114.86) --
	( 78.46,114.88) --
	( 78.44,114.93) --
	( 78.44,114.97) --
	( 78.42,115.02) --
	( 78.41,115.07) --
	( 78.39,115.12) --
	( 78.37,115.15) --
	( 78.35,115.15) --
	( 78.30,115.16) --
	( 78.27,115.16) --
	( 78.21,115.15) --
	( 78.20,115.15) --
	( 78.18,115.16) --
	( 78.17,115.17) --
	( 78.16,115.19) --
	( 78.16,115.24) --
	( 78.10,115.33) --
	( 78.09,115.34) --
	( 78.04,115.36) --
	( 78.02,115.37) --
	( 77.99,115.36) --
	( 77.96,115.36) --
	( 77.95,115.36) --
	( 77.94,115.38) --
	( 77.94,115.40) --
	( 77.94,115.41) --
	( 77.94,115.43) --
	( 77.94,115.46) --
	( 77.94,115.53) --
	( 77.92,115.55) --
	( 77.91,115.56) --
	( 77.89,115.57) --
	( 77.83,115.57) --
	( 77.79,115.56) --
	( 77.69,115.56) --
	( 77.66,115.57) --
	( 77.64,115.57) --
	( 77.56,115.56) --
	( 77.53,115.57) --
	( 77.46,115.57) --
	( 77.38,115.57) --
	( 77.33,115.57) --
	( 77.29,115.56) --
	( 77.27,115.55) --
	( 77.23,115.54) --
	( 77.18,115.51) --
	( 77.12,115.47) --
	( 77.09,115.45) --
	( 77.07,115.42) --
	( 77.02,115.38) --
	( 76.97,115.37) --
	( 76.91,115.36) --
	( 76.83,115.36) --
	( 76.80,115.36) --
	( 76.71,115.30) --
	( 76.67,115.27) --
	( 76.67,115.25) --
	( 76.69,115.17) --
	( 76.69,115.13) --
	( 76.66,115.00) --
	( 76.65,114.97) --
	( 76.63,114.93) --
	( 76.62,114.90) --
	( 76.61,114.87) --
	( 76.60,114.84) --
	( 76.60,114.82) --
	( 76.61,114.77) --
	( 76.61,114.75) --
	( 76.63,114.72) --
	( 76.64,114.65) --
	( 76.63,114.62) --
	( 76.63,114.61) --
	( 76.51,114.42) --
	( 76.43,114.34) --
	( 76.39,114.30) --
	( 76.35,114.26) --
	( 76.29,114.19) --
	( 76.25,114.15) --
	( 76.21,114.11) --
	( 76.20,114.08) --
	( 76.18,114.05) --
	( 76.17,114.01) --
	( 76.17,113.99) --
	( 76.17,113.92) --
	( 76.15,113.84) --
	( 76.13,113.81) --
	( 76.09,113.72) --
	( 76.05,113.65) --
	( 76.03,113.56) --
	( 76.01,113.54) --
	( 76.00,113.52) --
	( 75.98,113.51) --
	( 75.97,113.50) --
	( 75.93,113.49) --
	( 75.89,113.45) --
	( 75.87,113.44) --
	( 75.85,113.43) --
	( 75.83,113.43) --
	( 75.76,113.40) --
	( 75.72,113.39) --
	( 75.70,113.38) --
	( 75.69,113.37) --
	( 75.65,113.35) --
	( 75.58,113.30) --
	( 75.54,113.28) --
	( 75.51,113.27) --
	( 75.44,113.23) --
	( 75.41,113.22) --
	( 75.36,113.21) --
	( 75.17,113.19) --
	( 75.13,113.18) --
	( 75.03,113.18) --
	( 74.91,113.15) --
	( 74.89,113.14) --
	( 74.87,113.13) --
	( 74.83,113.09) --
	( 74.76,112.99) --
	( 74.75,112.97) --
	( 74.73,112.94) --
	( 74.71,112.92) --
	( 74.69,112.94) --
	( 74.63,112.99) --
	( 74.59,113.01) --
	( 74.52,113.05) --
	( 74.42,113.07) --
	( 74.37,113.08) --
	( 74.32,113.11) --
	( 74.26,113.16) --
	( 74.25,113.18) --
	( 74.22,113.20) --
	( 74.17,113.22) --
	( 74.13,113.25) --
	( 74.10,113.27) --
	( 74.08,113.29) --
	( 74.08,113.31) --
	( 74.09,113.34) --
	( 74.09,113.37) --
	( 74.08,113.41) --
	( 74.06,113.43) --
	( 73.97,113.51) --
	( 73.92,113.55) --
	( 73.85,113.58) --
	( 73.83,113.59) --
	( 73.82,113.60) --
	( 73.80,113.65) --
	( 73.77,113.69) --
	( 73.72,113.75) --
	( 73.65,113.82) --
	( 73.56,113.90) --
	( 73.48,113.98) --
	( 73.46,114.01) --
	( 73.45,114.03) --
	( 73.45,114.05) --
	( 73.46,114.08) --
	( 73.48,114.10) --
	( 73.49,114.12) --
	( 73.53,114.18) --
	( 73.55,114.21) --
	( 73.58,114.24) --
	( 73.60,114.26) --
	( 73.61,114.26) --
	( 73.69,114.29) --
	( 73.74,114.29) --
	( 73.84,114.30) --
	( 73.91,114.31) --
	( 73.93,114.31) --
	( 73.94,114.32) --
	( 73.93,114.38) --
	( 73.94,114.39) --
	( 73.96,114.41) --
	( 73.96,114.42) --
	( 73.96,114.44) --
	( 73.92,114.51) --
	( 73.92,114.53) --
	( 73.92,114.57) --
	( 73.94,114.58) --
	( 73.96,114.60) --
	( 74.06,114.65) --
	( 74.11,114.67) --
	( 74.16,114.70) --
	( 74.21,114.72) --
	( 74.25,114.73) --
	( 74.27,114.75) --
	( 74.28,114.76) --
	( 74.28,114.80) --
	( 74.28,114.82) --
	( 74.33,114.86) --
	( 74.41,114.87) --
	( 74.50,114.89) --
	( 74.51,114.89) --
	( 74.56,114.92) --
	( 74.61,114.97) --
	( 74.62,114.99) --
	( 74.71,115.00) --
	( 74.72,115.01) --
	( 74.73,115.04) --
	( 74.75,115.07) --
	( 74.79,115.15) --
	( 74.84,115.25) --
	( 74.88,115.35) --
	( 74.89,115.37) --
	( 74.88,115.41) --
	( 74.87,115.45) --
	( 74.87,115.48) --
	( 74.86,115.50) --
	( 74.84,115.53) --
	( 74.77,115.64) --
	( 74.75,115.66) --
	( 74.73,115.67) --
	( 74.69,115.69) --
	( 74.61,115.73) --
	( 74.54,115.75) --
	( 74.50,115.76) --
	( 74.48,115.78) --
	( 74.46,115.79) --
	( 74.45,115.82) --
	( 74.44,115.87) --
	( 74.44,115.91) --
	( 74.43,115.93) --
	( 74.44,115.96) --
	( 74.44,115.98) --
	( 74.43,116.02) --
	( 74.42,116.10) --
	( 74.42,116.13) --
	( 74.43,116.16) --
	( 74.43,116.19) --
	( 74.44,116.22) --
	( 74.43,116.27) --
	( 74.41,116.29) --
	( 74.40,116.31) --
	( 74.39,116.32) --
	( 74.36,116.33) --
	( 74.05,116.41) --
	( 74.02,116.40) --
	( 74.00,116.40) --
	( 73.99,116.39) --
	( 73.97,116.37) --
	( 73.92,116.34) --
	( 73.91,116.33) --
	( 73.84,116.33) --
	( 73.80,116.33) --
	( 73.78,116.32) --
	( 73.76,116.31) --
	( 73.71,116.30) --
	( 73.68,116.30) --
	( 73.65,116.30) --
	( 73.63,116.30) --
	( 73.57,116.30) --
	( 73.54,116.31) --
	( 73.53,116.31) --
	( 73.52,116.33) --
	( 73.49,116.36) --
	( 73.47,116.38) --
	( 73.41,116.40) --
	( 73.38,116.40) --
	( 73.33,116.42) --
	( 73.33,116.43) --
	( 73.31,116.47) --
	( 73.30,116.51) --
	( 73.25,116.63) --
	( 73.22,116.68) --
	( 73.19,116.75) --
	( 73.18,116.80) --
	( 73.19,116.82) --
	( 73.20,116.84) --
	( 73.20,116.88) --
	( 73.20,116.90) --
	( 73.18,116.95) --
	( 73.16,116.96) --
	( 73.12,116.98) --
	( 73.11,116.98) --
	( 73.09,117.00) --
	( 73.09,117.05) --
	( 73.17,117.14) --
	( 73.18,117.16) --
	( 73.19,117.18) --
	( 73.19,117.21) --
	( 73.18,117.25) --
	( 73.13,117.30) --
	( 73.09,117.33) --
	( 73.08,117.35) --
	( 73.09,117.37) --
	( 73.11,117.39) --
	( 73.12,117.40) --
	( 73.14,117.40) --
	( 73.24,117.43) --
	( 73.34,117.47) --
	( 73.35,117.48) --
	( 73.35,117.49) --
	( 73.33,117.52) --
	( 73.33,117.54) --
	( 73.33,117.57) --
	( 73.39,117.69) --
	( 73.41,117.72) --
	( 73.43,117.74) --
	( 73.46,117.76) --
	( 73.48,117.78) --
	( 73.49,117.80) --
	( 73.49,117.82) --
	( 73.50,117.84) --
	( 73.51,117.87) --
	( 73.52,117.93) --
	( 73.54,117.97) --
	( 73.56,118.00) --
	( 73.57,118.03) --
	( 73.63,118.16) --
	( 73.65,118.28) --
	( 73.65,118.37) --
	( 73.66,118.45) --
	( 73.66,118.48) --
	( 73.67,118.50) --
	( 73.70,118.55) --
	( 73.75,118.63) --
	( 73.76,118.67) --
	( 73.78,118.67) --
	( 73.79,118.68) --
	( 73.86,118.69) --
	( 73.87,118.70) --
	( 73.88,118.71) --
	( 73.90,118.75) --
	( 73.92,118.78) --
	( 73.93,118.81) --
	( 73.96,118.83) --
	( 74.00,118.88) --
	( 74.01,118.91) --
	( 74.01,118.96) --
	( 74.02,119.00) --
	( 74.02,119.07) --
	( 74.02,119.10) --
	( 74.01,119.13) --
	( 74.01,119.16) --
	( 74.01,119.21) --
	( 74.04,119.29) --
	( 74.05,119.32) --
	( 74.05,119.34) --
	( 74.05,119.36) --
	( 74.04,119.38) --
	( 74.02,119.48) --
	( 74.01,119.57) --
	( 74.01,119.60) --
	( 74.05,119.66) --
	( 74.06,119.68) --
	( 74.07,119.72) --
	( 74.08,119.74) --
	( 74.09,119.81) --
	( 74.12,119.96) --
	( 74.14,120.06) --
	( 74.15,120.16) --
	( 74.16,120.26) --
	( 74.16,120.31) --
	( 74.15,120.33) --
	( 74.13,120.34) --
	( 74.10,120.36) --
	( 74.08,120.37) --
	( 74.07,120.38) --
	( 74.06,120.40) --
	( 74.06,120.43) --
	( 74.06,120.49) --
	( 74.07,120.52) --
	( 74.11,120.60) --
	( 74.12,120.63) --
	( 74.13,120.65) --
	( 74.13,120.67) --
	( 74.12,120.72) --
	( 74.11,120.74) --
	( 74.06,120.80) --
	( 74.06,120.81) --
	( 74.06,120.85) --
	( 74.05,120.89) --
	( 74.04,120.92) --
	( 74.02,120.95) --
	( 74.01,120.96) --
	( 73.99,120.97) --
	( 73.97,120.97) --
	( 73.89,120.95) --
	( 73.83,120.92) --
	( 73.81,120.92) --
	( 73.78,120.92) --
	( 73.73,120.92) --
	( 73.70,120.92) --
	( 73.57,120.94) --
	( 73.54,120.94) --
	( 73.48,120.97) --
	( 73.46,120.99) --
	( 73.36,121.07) --
	( 73.34,121.09) --
	( 73.30,121.14) --
	( 73.25,121.18) --
	( 73.17,121.27) --
	( 73.13,121.30) --
	( 73.09,121.31) --
	( 73.03,121.32) --
	( 72.87,121.38) --
	( 72.83,121.39) --
	( 72.72,121.40) --
	( 72.68,121.40) --
	( 72.63,121.41) --
	( 72.58,121.43) --
	( 72.50,121.44) --
	( 72.46,121.44) --
	( 72.43,121.43) --
	( 72.41,121.43) --
	( 72.39,121.42) --
	( 72.33,121.45) --
	( 72.31,121.46) --
	( 72.24,121.48) --
	( 72.21,121.48) --
	( 72.15,121.48) --
	( 72.13,121.48) --
	( 72.06,121.46) --
	( 72.05,121.46) --
	( 72.03,121.43) --
	( 72.01,121.37) --
	( 72.00,121.35) --
	( 72.00,121.33) --
	( 71.99,121.32) --
	( 71.96,121.30) --
	( 71.91,121.26) --
	( 71.85,121.23) --
	( 71.81,121.21) --
	( 71.80,121.20) --
	( 71.78,121.19) --
	( 71.72,121.19) --
	( 71.70,121.18) --
	( 71.63,121.16) --
	( 71.60,121.15) --
	( 71.56,121.15) --
	( 71.53,121.15) --
	( 71.49,121.13) --
	( 71.46,121.11) --
	( 71.37,121.05) --
	( 71.29,121.00) --
	( 71.24,120.98) --
	( 71.22,120.96) --
	( 71.20,120.93) --
	( 71.18,120.90) --
	( 71.15,120.87) --
	( 71.11,120.84) --
	( 71.01,120.77) --
	( 70.97,120.74) --
	( 70.97,120.62) --
	( 70.97,120.58) --
	( 70.97,120.54) --
	( 70.97,120.53) --
	( 70.95,120.53) --
	( 70.84,120.52) --
	( 70.73,120.47) --
	( 70.68,120.46) --
	( 70.63,120.44) --
	( 70.60,120.43) --
	( 70.58,120.41) --
	( 70.46,120.34) --
	( 70.40,120.33) --
	( 70.37,120.32) --
	( 70.36,120.31) --
	( 70.31,120.27) --
	( 70.27,120.22) --
	( 70.16,120.10) --
	( 70.07,120.01) --
	( 70.05,119.97) --
	( 70.02,119.93) --
	( 69.97,119.86) --
	( 69.94,119.84) --
	( 69.91,119.84) --
	( 69.88,119.83) --
	( 69.87,119.82) --
	( 69.85,119.80) --
	( 69.84,119.77) --
	( 69.84,119.74) --
	( 69.82,119.72) --
	( 69.79,119.69) --
	( 69.76,119.66) --
	( 69.68,119.59) --
	( 69.65,119.57) --
	( 69.62,119.54) --
	( 69.56,119.51) --
	( 69.51,119.49) --
	( 69.50,119.46) --
	( 69.48,119.43) --
	( 69.47,119.36) --
	( 69.45,119.35) --
	( 69.43,119.31) --
	( 69.43,119.30) --
	( 69.44,119.28) --
	( 69.44,119.27) --
	( 69.44,119.26) --
	( 69.42,119.24) --
	( 69.39,119.23) --
	( 69.37,119.22) --
	( 69.35,119.19) --
	( 69.33,119.15) --
	( 69.27,119.08) --
	( 69.27,119.06) --
	( 69.26,119.06) --
	( 69.25,119.06) --
	( 69.22,119.07) --
	( 69.19,119.09) --
	( 69.17,119.09) --
	( 69.14,119.09) --
	( 69.11,119.08) --
	( 69.08,119.05) --
	( 69.01,118.92) --
	( 69.00,118.90) --
	( 69.00,118.88) --
	( 69.06,118.77) --
	( 69.07,118.75) --
	( 69.07,118.74) --
	( 69.07,118.72) --
	( 69.02,118.69) --
	( 68.98,118.66) --
	( 68.93,118.65) --
	( 68.88,118.63) --
	( 68.84,118.59) --
	( 68.82,118.56) --
	( 68.80,118.51) --
	( 68.78,118.47) --
	( 68.76,118.43) --
	( 68.71,118.38) --
	( 68.70,118.35) --
	( 68.68,118.33) --
	( 68.66,118.31) --
	( 68.59,118.27) --
	( 68.56,118.26) --
	( 68.50,118.25) --
	( 68.45,118.26) --
	( 68.43,118.25) --
	( 68.40,118.24) --
	( 68.37,118.23) --
	( 68.33,118.22) --
	( 68.28,118.22) --
	( 68.22,118.23) --
	( 68.20,118.23) --
	( 68.18,118.22) --
	( 68.17,118.20) --
	( 68.17,118.08) --
	( 68.16,118.05) --
	( 68.15,118.02) --
	( 68.13,117.99) --
	( 68.08,117.94) --
	( 68.05,117.89) --
	( 68.02,117.86) --
	( 67.97,117.81) --
	( 67.94,117.80) --
	( 67.93,117.78) --
	( 67.88,117.79) --
	( 67.85,117.81) --
	( 67.83,117.83) --
	( 67.79,117.85) --
	( 67.76,117.87) --
	( 67.70,117.90) --
	( 67.67,117.91) --
	( 67.59,117.92) --
	( 67.52,117.92) --
	( 67.49,117.92) --
	( 67.43,117.95) --
	( 67.35,117.96) --
	( 67.32,117.96) --
	( 67.24,118.00) --
	( 67.16,118.06) --
	( 67.07,118.12) --
	( 67.04,118.15) --
	( 67.01,118.17) --
	( 66.98,118.21) --
	( 66.93,118.25) --
	( 66.90,118.26) --
	( 66.87,118.27) --
	( 66.85,118.27) --
	( 66.81,118.26) --
	( 66.81,118.26) --
	( 66.78,118.26) --
	( 66.72,118.28) --
	( 66.55,118.35) --
	( 66.51,118.35) --
	( 66.47,118.35) --
	( 66.44,118.36) --
	( 66.42,118.37) --
	( 66.39,118.40) --
	( 66.33,118.43) --
	( 66.24,118.50) --
	( 66.17,118.53) --
	( 66.05,118.59) --
	( 66.03,118.60) --
	( 65.99,118.60) --
	( 65.94,118.61) --
	( 65.91,118.62) --
	( 65.87,118.64) --
	( 65.84,118.67) --
	( 65.81,118.69) --
	( 65.76,118.71) --
	( 65.70,118.74) --
	( 65.58,118.80) --
	( 65.55,118.81) --
	( 65.51,118.82) --
	( 65.47,118.83) --
	( 65.42,118.83) --
	( 65.39,118.84) --
	( 65.35,118.85) --
	( 65.26,118.90) --
	( 65.20,118.90) --
	( 65.16,118.92) --
	( 65.08,118.93) --
	( 64.95,118.93) --
	( 64.84,118.94) --
	( 64.80,118.93) --
	( 64.71,118.92) --
	( 64.60,118.89) --
	( 64.47,118.91) --
	( 64.45,118.91) --
	( 64.42,118.90) --
	( 64.39,118.89) --
	( 64.38,118.88) --
	( 64.36,118.88) --
	( 64.34,118.88) --
	( 64.27,118.89) --
	( 64.24,118.89) --
	( 64.19,118.90) --
	( 64.15,118.91) --
	( 64.11,118.92) --
	( 64.06,118.96) --
	( 64.03,118.98) --
	( 63.97,119.02) --
	( 63.91,119.06) --
	( 63.90,119.06) --
	( 63.90,119.06) --
	( 63.90,119.05) --
	( 63.83,119.08) --
	( 63.78,119.10) --
	( 63.72,119.12) --
	( 63.66,119.15) --
	( 63.61,119.17) --
	( 63.55,119.21) --
	( 63.50,119.24) --
	( 63.44,119.27) --
	( 63.39,119.30) --
	( 63.34,119.34) --
	( 63.26,119.39) --
	( 63.23,119.42) --
	( 63.19,119.46) --
	( 63.14,119.51) --
	( 63.10,119.55) --
	( 63.07,119.59) --
	( 63.06,119.61) --
	( 63.03,119.65) --
	( 63.01,119.69) --
	( 62.98,119.73) --
	( 62.97,119.76) --
	( 62.95,119.79) --
	( 62.94,119.82) --
	( 62.92,119.86) --
	( 62.91,119.91) --
	( 62.90,119.95) --
	( 62.86,120.04) --
	( 62.84,120.09) --
	( 62.82,120.14) --
	( 62.81,120.18) --
	( 62.79,120.24) --
	( 62.76,120.30) --
	( 62.74,120.36) --
	( 62.72,120.41) --
	( 62.71,120.46) --
	( 62.69,120.50) --
	( 62.68,120.52) --
	( 62.65,120.58) --
	( 62.64,120.61) --
	( 62.63,120.64) --
	( 62.62,120.66) --
	( 62.61,120.69) --
	( 62.57,120.75) --
	( 62.56,120.78) --
	( 62.54,120.81) --
	( 62.53,120.84) --
	( 62.52,120.87) --
	( 62.50,120.89) --
	( 62.50,120.90) --
	( 62.48,120.92) --
	( 62.46,120.94) --
	( 62.46,120.96) --
	( 62.44,120.98) --
	( 62.41,121.00) --
	( 62.39,121.02) --
	( 62.37,121.03) --
	( 62.35,121.04) --
	( 62.33,121.05) --
	( 62.31,121.05) --
	( 62.25,121.06) --
	( 62.22,121.07) --
	( 62.20,121.06) --
	( 62.14,121.06) --
	( 62.10,121.06) --
	( 62.05,121.05) --
	( 62.02,121.05) --
	( 61.99,121.05) --
	( 61.96,121.05) --
	( 61.94,121.05) --
	( 61.90,121.05) --
	( 61.87,121.06) --
	( 61.85,121.06) --
	( 61.82,121.07) --
	( 61.81,121.07) --
	( 61.80,121.08) --
	( 61.78,121.10) --
	( 61.76,121.12) --
	( 61.75,121.15) --
	( 61.74,121.18) --
	( 61.74,121.20) --
	( 61.74,121.22) --
	( 61.73,121.26) --
	( 61.73,121.27) --
	( 61.72,121.29) --
	( 61.70,121.30) --
	( 61.69,121.32) --
	( 61.67,121.33) --
	( 61.65,121.34) --
	( 61.63,121.35) --
	( 61.61,121.35) --
	( 61.60,121.36) --
	( 61.57,121.35) --
	( 61.53,121.35) --
	( 61.50,121.35) --
	( 61.46,121.35) --
	( 61.44,121.34) --
	( 61.37,121.34) --
	( 61.35,121.34) --
	( 61.32,121.34) --
	( 61.31,121.34) --
	( 61.29,121.34) --
	( 61.22,121.35) --
	( 61.18,121.35) --
	( 61.15,121.37) --
	( 61.12,121.39) --
	( 61.10,121.40) --
	( 61.08,121.43) --
	( 61.06,121.46) --
	( 61.05,121.49) --
	( 61.03,121.52) --
	( 61.02,121.55) --
	( 61.02,121.59) --
	( 61.01,121.62) --
	( 61.01,121.66) --
	( 61.01,121.73) --
	( 61.02,121.78) --
	( 61.03,121.82) --
	( 61.03,121.85) --
	( 61.03,121.89) --
	( 61.03,121.91) --
	( 61.03,121.98) --
	( 61.03,122.02) --
	( 61.02,122.10) --
	( 61.01,122.14) --
	( 61.01,122.18) --
	( 61.00,122.22) --
	( 60.98,122.26) --
	( 60.97,122.30) --
	( 60.96,122.33) --
	( 60.95,122.36) --
	( 60.94,122.39) --
	( 60.92,122.44) --
	( 60.92,122.46) --
	( 60.91,122.48) --
	( 60.88,122.52) --
	( 60.85,122.56) --
	( 60.82,122.61) --
	( 60.80,122.66) --
	( 60.79,122.71) --
	( 60.77,122.75) --
	( 60.75,122.80) --
	( 60.74,122.86) --
	( 60.74,122.89) --
	( 60.73,122.92) --
	( 60.73,122.95) --
	( 60.72,123.03) --
	( 60.72,123.12) --
	( 60.72,123.15) --
	( 60.72,123.19) --
	( 60.73,123.23) --
	( 60.74,123.27) --
	( 60.75,123.33) --
	( 60.78,123.39) --
	( 60.80,123.45) --
	( 60.83,123.50) --
	( 60.87,123.56) --
	( 60.90,123.60) --
	( 60.93,123.63) --
	( 60.97,123.68) --
	( 61.00,123.71) --
	( 61.04,123.75) --
	( 61.06,123.81) --
	( 61.07,123.84) --
	( 61.08,123.86) --
	( 61.08,123.90) --
	( 61.09,123.93) --
	( 61.10,123.97) --
	( 61.12,124.02) --
	( 61.12,124.06) --
	( 61.13,124.09) --
	( 61.14,124.12) --
	( 61.15,124.14) --
	( 61.15,124.16) --
	( 61.15,124.19) --
	( 61.15,124.22) --
	( 61.15,124.25) --
	( 61.15,124.29) --
	( 61.14,124.35) --
	( 61.14,124.37) --
	( 61.13,124.40) --
	( 61.12,124.44) --
	( 61.10,124.48) --
	( 61.09,124.51) --
	( 61.08,124.53) --
	( 61.07,124.55) --
	( 61.05,124.59) --
	( 61.04,124.61) --
	( 61.02,124.67) --
	( 61.01,124.71) --
	( 61.00,124.74) --
	( 60.99,124.76) --
	( 60.99,124.76) --
	( 60.98,124.76) --
	( 60.99,124.78) --
	( 61.01,124.82) --
	( 61.03,124.84) --
	( 61.04,124.86) --
	( 61.06,124.89) --
	( 61.11,124.94) --
	( 61.15,124.98) --
	( 61.20,125.02) --
	( 61.23,125.05) --
	( 61.26,125.07) --
	( 61.28,125.08) --
	( 61.30,125.10) --
	( 61.31,125.12) --
	( 61.35,125.19) --
	( 61.38,125.24) --
	( 61.39,125.26) --
	( 61.41,125.31) --
	( 61.42,125.34) --
	( 61.42,125.35) --
	( 61.42,125.37) --
	( 61.40,125.43) --
	( 61.39,125.46) --
	( 61.38,125.49) --
	( 61.36,125.53) --
	( 61.33,125.57) --
	( 61.30,125.62) --
	( 61.29,125.64) --
	( 61.25,125.66) --
	( 61.23,125.68) --
	( 61.17,125.74) --
	( 61.09,125.82) --
	( 61.07,125.85) --
	( 61.06,125.87) --
	( 61.06,125.88) --
	( 61.05,125.92) --
	( 61.03,125.94) --
	( 61.01,125.97) --
	( 61.00,125.98) --
	( 60.99,126.00) --
	( 60.98,126.01) --
	( 60.95,126.05) --
	( 60.90,126.08) --
	( 60.85,126.12) --
	( 60.84,126.13) --
	( 60.80,126.17) --
	( 60.79,126.18) --
	( 60.78,126.19) --
	( 60.76,126.21) --
	( 60.73,126.23) --
	( 60.69,126.24) --
	( 60.65,126.27) --
	( 60.62,126.29) --
	( 60.60,126.31) --
	( 60.57,126.33) --
	( 60.54,126.36) --
	( 60.52,126.39) --
	( 60.51,126.41) --
	( 60.51,126.42) --
	( 60.48,126.44) --
	( 60.45,126.46) --
	( 60.41,126.48) --
	( 60.39,126.49) --
	( 60.32,126.52) --
	( 60.29,126.53) --
	( 60.24,126.54) --
	( 60.15,126.54) --
	( 59.92,126.55) --
	( 59.84,126.55) --
	( 59.68,126.55) --
	( 59.57,126.55) --
	( 59.55,126.55) --
	( 59.53,126.55) --
	( 59.46,126.53) --
	( 59.41,126.53) --
	( 59.40,126.53) --
	( 59.38,126.54) --
	( 59.35,126.56) --
	( 59.32,126.57) --
	( 59.31,126.59) --
	( 59.31,126.60) --
	( 59.31,126.61) --
	( 59.31,126.62) --
	( 59.31,126.63) --
	( 59.33,126.65) --
	( 59.33,126.67) --
	( 59.35,126.71) --
	( 59.36,126.74) --
	( 59.36,126.76) --
	( 59.36,126.79) --
	( 59.36,126.81) --
	( 59.37,126.92) --
	( 59.37,126.94) --
	( 59.37,126.96) --
	( 59.37,126.98) --
	( 59.36,127.00) --
	( 59.35,127.03) --
	( 59.34,127.05) --
	( 59.32,127.10) --
	( 59.29,127.13) --
	( 59.25,127.18) --
	( 59.23,127.20) --
	( 59.17,127.23) --
	( 59.14,127.26) --
	( 59.06,127.32) --
	( 59.03,127.34) --
	( 59.01,127.35) --
	( 58.97,127.37) --
	( 58.91,127.39) --
	( 58.90,127.40) --
	( 58.82,127.41) --
	( 58.80,127.42) --
	( 58.72,127.44) --
	( 58.70,127.45) --
	( 58.68,127.45) --
	( 58.62,127.47) --
	( 58.59,127.48) --
	( 58.54,127.49) --
	( 58.53,127.50) --
	( 58.52,127.51) --
	( 58.50,127.53) --
	( 58.49,127.55) --
	( 58.49,127.56) --
	( 58.49,127.58) --
	( 58.49,127.65) --
	( 58.49,127.66) --
	( 58.49,127.69) --
	( 58.46,127.77) --
	( 58.40,127.88) --
	( 58.37,127.92) --
	( 58.35,127.96) --
	( 58.34,128.02) --
	( 58.34,128.03) --
	( 58.34,128.05) --
	( 58.37,128.10) --
	( 58.39,128.13) --
	( 58.44,128.22) --
	( 58.45,128.24) --
	( 58.46,128.25) --
	( 58.48,128.27) --
	( 58.50,128.28) --
	( 58.52,128.29) --
	( 58.57,128.30) --
	( 58.59,128.30) --
	( 58.61,128.31) --
	( 58.63,128.31) --
	( 58.65,128.32) --
	( 58.67,128.33) --
	( 58.67,128.34) --
	( 58.68,128.35) --
	( 58.69,128.37) --
	( 58.69,128.38) --
	( 58.68,128.43) --
	( 58.68,128.46) --
	( 58.70,128.51) --
	( 58.73,128.55) --
	( 58.76,128.63) --
	( 58.77,128.65) --
	( 58.79,128.68) --
	( 58.81,128.75) --
	( 58.82,128.77) --
	( 58.82,128.78) --
	( 58.83,128.78) --
	( 58.87,128.80) --
	( 58.88,128.81) --
	( 58.90,128.81) --
	( 58.92,128.80) --
	( 59.02,128.79) --
	( 59.06,128.78) --
	( 59.09,128.78) --
	( 59.12,128.76) --
	( 59.15,128.76) --
	( 59.19,128.75) --
	( 59.22,128.75) --
	( 59.25,128.76) --
	( 59.27,128.76) --
	( 59.28,128.77) --
	( 59.30,128.78) --
	( 59.30,128.80) --
	( 59.29,128.81) --
	( 59.28,128.82) --
	( 59.26,128.84) --
	( 59.20,128.88) --
	( 59.17,128.90) --
	( 59.13,128.92) --
	( 59.08,128.94) --
	( 59.04,128.96) --
	( 58.98,129.00) --
	( 58.93,129.03) --
	( 58.91,129.04) --
	( 58.88,129.05) --
	( 58.86,129.06) --
	( 58.83,129.06) --
	( 58.79,129.08) --
	( 58.74,129.11) --
	( 58.71,129.14) --
	( 58.69,129.18) --
	( 58.67,129.22) --
	( 58.66,129.24) --
	( 58.66,129.26) --
	( 58.65,129.29) --
	( 58.65,129.30) --
	( 58.65,129.37) --
	( 58.65,129.40) --
	( 58.63,129.46) --
	( 58.62,129.51) --
	( 58.60,129.54) --
	( 58.58,129.59) --
	( 58.56,129.62) --
	( 58.52,129.69) --
	( 58.49,129.72) --
	( 58.47,129.75) --
	( 58.40,129.81) --
	( 58.38,129.83) --
	( 58.35,129.85) --
	( 58.30,129.89) --
	( 58.22,129.93) --
	( 58.20,129.94) --
	( 58.16,129.95) --
	( 58.12,129.96) --
	( 58.05,129.99) --
	( 57.99,130.00) --
	( 57.98,130.01) --
	( 57.97,130.02) --
	( 57.96,130.03) --
	( 57.96,130.03) --
	( 57.98,130.08) --
	( 57.99,130.09) --
	( 58.00,130.10) --
	( 58.06,130.15) --
	( 58.08,130.17) --
	( 58.11,130.19) --
	( 58.13,130.20) --
	( 58.16,130.25) --
	( 58.18,130.29) --
	( 58.19,130.31) --
	( 58.20,130.34) --
	( 58.20,130.35) --
	( 58.19,130.38) --
	( 58.19,130.40) --
	( 58.18,130.42) --
	( 58.17,130.45) --
	( 58.13,130.53) --
	( 58.10,130.58) --
	( 58.06,130.65) --
	( 58.01,130.70) --
	( 57.97,130.74) --
	( 57.96,130.75) --
	( 57.95,130.76) --
	( 57.90,130.78) --
	( 57.88,130.79) --
	( 57.78,130.85) --
	( 57.75,130.86) --
	( 57.71,130.87) --
	( 57.68,130.88) --
	( 57.60,130.89) --
	( 57.58,130.90) --
	( 57.51,130.92) --
	( 57.47,130.93) --
	( 57.33,130.96) --
	( 57.31,130.96) --
	( 57.29,130.97) --
	( 57.28,130.99) --
	( 57.24,131.03) --
	( 57.23,131.05) --
	( 57.20,131.08) --
	( 57.11,131.17) --
	( 57.10,131.18) --
	( 57.10,131.19) --
	( 57.09,131.25) --
	( 57.09,131.29) --
	( 57.09,131.30) --
	( 57.06,131.44) --
	( 57.05,131.46) --
	( 57.04,131.50) --
	( 57.04,131.51) --
	( 57.03,131.56) --
	( 57.02,131.63) --
	( 57.01,131.65) --
	( 57.00,131.67) --
	( 56.99,131.70) --
	( 56.96,131.76) --
	( 56.95,131.77) --
	( 56.93,131.80) --
	( 56.86,131.89) --
	( 56.83,131.94) --
	( 56.83,131.96) --
	( 56.82,131.98) --
	( 56.82,131.99) --
	( 56.82,132.00) --
	( 56.83,132.02) --
	( 56.83,132.03) --
	( 56.83,132.05) --
	( 56.83,132.07) --
	( 56.84,132.09) --
	( 56.86,132.13) --
	( 56.87,132.15) --
	( 56.90,132.19) --
	( 56.91,132.21) --
	( 56.99,132.32) --
	( 57.01,132.37) --
	( 57.01,132.41) --
	( 57.02,132.43) --
	( 57.02,132.44) --
	( 57.02,132.46) --
	( 57.01,132.47) --
	( 57.01,132.48) --
	( 57.00,132.50) --
	( 56.99,132.51) --
	( 56.98,132.52) --
	( 56.89,132.58) --
	( 56.88,132.59) --
	( 56.84,132.61) --
	( 56.82,132.62) --
	( 56.80,132.63) --
	( 56.74,132.70) --
	( 56.72,132.72) --
	( 56.69,132.74) --
	( 56.67,132.75) --
	( 56.64,132.77) --
	( 56.62,132.78) --
	( 56.60,132.79) --
	( 56.58,132.80) --
	( 56.56,132.80) --
	( 56.55,132.80) --
	( 56.54,132.80) --
	( 56.53,132.79) --
	( 56.47,132.75) --
	( 56.45,132.75) --
	( 56.43,132.75) --
	( 56.41,132.75) --
	( 56.39,132.75) --
	( 56.36,132.75) --
	( 56.33,132.76) --
	( 56.32,132.76) --
	( 56.30,132.77) --
	( 56.26,132.78) --
	( 56.23,132.79) --
	( 56.21,132.81) --
	( 56.19,132.82) --
	( 56.16,132.84) --
	( 56.12,132.87) --
	( 56.12,132.88) --
	( 56.12,132.89) --
	( 56.12,132.90) --
	( 56.12,132.92) --
	( 56.12,132.93) --
	( 56.13,132.94) --
	( 56.14,132.96) --
	( 56.15,132.97) --
	( 56.17,133.00) --
	( 56.21,133.04) --
	( 56.22,133.05) --
	( 56.23,133.06) --
	( 56.25,133.07) --
	( 56.27,133.08) --
	( 56.29,133.09) --
	( 56.35,133.11) --
	( 56.39,133.12) --
	( 56.42,133.13) --
	( 56.47,133.14) --
	( 56.49,133.15) --
	( 56.51,133.16) --
	( 56.52,133.17) --
	( 56.54,133.20) --
	( 56.55,133.21) --
	( 56.56,133.23) --
	( 56.56,133.24) --
	( 56.56,133.26) --
	( 56.56,133.29) --
	( 56.56,133.31) --
	( 56.53,133.35) --
	( 56.50,133.39) --
	( 56.49,133.41) --
	( 56.48,133.44) --
	( 56.48,133.46) --
	( 56.48,133.48) --
	( 56.49,133.50) --
	( 56.49,133.52) --
	( 56.50,133.53) --
	( 56.50,133.55) --
	( 56.54,133.61) --
	( 56.58,133.67) --
	( 56.60,133.72) --
	( 56.60,133.74) --
	( 56.62,133.78) --
	( 56.62,133.79) --
	( 56.61,133.82) --
	( 56.60,133.85) --
	( 56.60,133.86) --
	( 56.59,133.87) --
	( 56.55,133.90) --
	( 56.54,133.91) --
	( 56.51,133.96) --
	( 56.51,133.98) --
	( 56.50,133.99) --
	( 56.49,134.03) --
	( 56.47,134.06) --
	( 56.43,134.10) --
	( 56.42,134.11) --
	( 56.41,134.13) --
	( 56.41,134.13) --
	( 56.40,134.15) --
	( 56.38,134.17) --
	( 56.36,134.18) --
	( 56.35,134.19) --
	( 56.33,134.19) --
	( 56.30,134.20) --
	( 56.25,134.20) --
	( 56.20,134.20) --
	( 56.17,134.20) --
	( 56.15,134.20) --
	( 56.14,134.21) --
	( 56.13,134.21) --
	( 56.13,134.22) --
	( 56.13,134.24) --
	( 56.14,134.27) --
	( 56.16,134.31) --
	( 56.19,134.37) --
	( 56.19,134.37) --
	( 56.20,134.41) --
	( 56.21,134.45) --
	( 56.21,134.46) --
	( 56.21,134.48) --
	( 56.21,134.50) --
	( 56.20,134.51) --
	( 56.19,134.51) --
	( 56.14,134.53) --
	( 56.13,134.53) --
	( 56.12,134.54) --
	( 56.12,134.55) --
	( 56.11,134.57) --
	( 56.11,134.60) --
	( 56.11,134.61) --
	( 56.10,134.62) --
	( 56.10,134.63) --
	( 56.09,134.65) --
	( 56.07,134.66) --
	( 56.04,134.69) --
	( 56.01,134.70) --
	( 56.00,134.71) --
	( 55.94,134.76) --
	( 55.93,134.76) --
	( 55.91,134.79) --
	( 55.88,134.81) --
	( 55.87,134.82) --
	( 55.86,134.83) --
	( 55.83,134.86) --
	( 55.82,134.86) --
	( 55.82,134.87) --
	( 55.82,134.88) --
	( 55.81,134.89) --
	( 55.81,134.90) --
	( 55.83,134.94) --
	( 55.84,134.95) --
	( 55.86,134.96) --
	( 55.89,134.99) --
	( 55.89,135.00) --
	( 55.90,135.02) --
	( 55.91,135.04) --
	( 55.93,135.06) --
	( 55.94,135.09) --
	( 55.95,135.11) --
	( 55.95,135.11) --
	( 55.93,135.13) --
	( 55.94,135.14) --
	( 55.95,135.15) --
	( 55.96,135.16) --
	( 55.98,135.18) --
	( 56.00,135.21) --
	( 56.03,135.24) --
	( 56.04,135.25) --
	( 56.04,135.26) --
	( 56.05,135.27) --
	( 56.07,135.31) --
	( 56.08,135.32) --
	( 56.09,135.35) --
	( 56.09,135.37) --
	( 56.10,135.38) --
	( 56.13,135.44) --
	( 56.13,135.47) --
	( 56.14,135.53) --
	( 56.16,135.59) --
	( 56.18,135.64) --
	( 56.21,135.68) --
	( 56.22,135.70) --
	( 56.23,135.70) --
	( 56.24,135.70) --
	( 56.25,135.71) --
	( 56.28,135.71) --
	( 56.30,135.71) --
	( 56.32,135.71) --
	( 56.34,135.70) --
	( 56.37,135.69) --
	( 56.40,135.68) --
	( 56.45,135.67) --
	( 56.49,135.66) --
	( 56.52,135.66) --
	( 56.56,135.65) --
	( 56.58,135.65) --
	( 56.60,135.66) --
	( 56.66,135.67) --
	( 56.70,135.67) --
	( 56.78,135.70) --
	( 56.89,135.74) --
	( 56.92,135.75) --
	( 56.94,135.76) --
	( 56.96,135.77) --
	( 56.98,135.77) --
	( 57.01,135.78) --
	( 57.03,135.78) --
	( 57.07,135.76) --
	( 57.12,135.75) --
	( 57.17,135.72) --
	( 57.22,135.70) --
	( 57.24,135.69) --
	( 57.28,135.68) --
	( 57.30,135.68) --
	( 57.32,135.70) --
	( 57.33,135.74) --
	( 57.34,135.74) --
	( 57.34,135.75) --
	( 57.35,135.76) --
	( 57.40,135.82) --
	( 57.41,135.84) --
	( 57.42,135.86) --
	( 57.42,135.88) --
	( 57.42,135.90) --
	( 57.42,135.93) --
	( 57.43,135.94) --
	( 57.42,135.96) --
	( 57.42,135.98) --
	( 57.41,135.99) --
	( 57.41,136.02) --
	( 57.41,136.06) --
	( 57.41,136.07) --
	( 57.42,136.09) --
	( 57.42,136.10) --
	( 57.44,136.12) --
	( 57.47,136.15) --
	( 57.50,136.16) --
	( 57.52,136.17) --
	( 57.53,136.17) --
	( 57.56,136.18) --
	( 57.58,136.18) --
	( 57.68,136.17) --
	( 57.69,136.19) --
	( 57.69,136.26) --
	( 57.70,136.30) --
	( 57.71,136.31) --
	( 57.72,136.32) --
	( 57.74,136.34) --
	( 57.76,136.35) --
	( 57.77,136.36) --
	( 57.78,136.36) --
	( 57.79,136.36) --
	( 57.85,136.38) --
	( 57.90,136.39) --
	( 57.90,136.39) --
	( 57.92,136.41) --
	( 57.93,136.43) --
	( 57.95,136.45) --
	( 57.95,136.46) --
	( 57.95,136.46) --
	( 57.94,136.48) --
	( 57.92,136.49) --
	( 57.90,136.51) --
	( 57.89,136.52) --
	( 57.87,136.55) --
	( 57.86,136.56) --
	( 57.86,136.57) --
	( 57.84,136.58) --
	( 57.82,136.60) --
	( 57.81,136.60) --
	( 57.80,136.61) --
	( 57.77,136.61) --
	( 57.76,136.62) --
	( 57.75,136.62) --
	( 57.75,136.63) --
	( 57.74,136.64) --
	( 57.73,136.67) --
	( 57.72,136.69) --
	( 57.71,136.70) --
	( 57.71,136.71) --
	( 57.69,136.73) --
	( 57.67,136.77) --
	( 57.65,136.81) --
	( 57.63,136.85) --
	( 57.60,136.88) --
	( 57.59,136.90) --
	( 57.59,136.91) --
	( 57.58,136.93) --
	( 57.58,136.96) --
	( 57.57,136.97) --
	( 57.55,137.00) --
	( 57.54,137.02) --
	( 57.51,137.05) --
	( 57.48,137.06) --
	( 57.45,137.08) --
	( 57.44,137.09) --
	( 57.44,137.09) --
	( 57.45,137.09) --
	( 57.45,137.09) --
	( 57.46,137.09) --
	( 57.46,137.09) --
	( 57.47,137.09) --
	( 57.47,137.09) --
	( 57.48,137.09) --
	( 57.48,137.09) --
	( 57.49,137.09) --
	( 57.47,137.11) --
	( 57.45,137.13) --
	( 57.42,137.15) --
	( 57.40,137.18) --
	( 57.40,137.19) --
	( 57.37,137.23) --
	( 57.35,137.26) --
	( 57.33,137.29) --
	( 57.33,137.30) --
	( 57.33,137.36) --
	( 57.34,137.40) --
	( 57.35,137.49) --
	( 57.35,137.52) --
	( 57.35,137.53) --
	( 57.35,137.57) --
	( 57.41,137.72) --
	( 57.43,137.74) --
	( 57.43,137.76) --
	( 57.47,137.83) --
	( 57.47,137.86) --
	( 57.49,137.88) --
	( 57.50,137.89) --
	( 57.51,137.92) --
	( 57.57,137.99) --
	( 57.58,138.02) --
	( 57.63,138.06) --
	( 57.71,138.14) --
	( 57.80,138.23) --
	( 57.82,138.25) --
	( 57.82,138.28) --
	( 57.84,138.32) --
	( 57.86,138.35) --
	( 57.88,138.39) --
	( 57.90,138.47) --
	( 57.91,138.49) --
	( 57.94,138.55) --
	( 57.94,138.56) --
	( 57.95,138.58) --
	( 57.96,138.59) --
	( 57.97,138.61) --
	( 57.99,138.64) --
	( 58.01,138.69) --
	( 58.05,138.74) --
	( 58.07,138.77) --
	( 58.11,138.81) --
	( 58.12,138.85) --
	( 58.14,138.86) --
	( 58.22,138.95) --
	( 58.28,139.02) --
	( 58.29,139.04) --
	( 58.36,139.11) --
	( 58.37,139.12) --
	( 58.39,139.14) --
	( 58.50,139.20) --
	( 58.56,139.21) --
	( 58.66,139.24) --
	( 58.72,139.26) --
	( 58.85,139.29) --
	( 58.86,139.29) --
	( 58.96,139.33) --
	( 59.01,139.34) --
	( 59.04,139.36) --
	( 59.09,139.37) --
	( 59.23,139.43) --
	( 59.26,139.44) --
	( 59.29,139.46) --
	( 59.33,139.48) --
	( 59.35,139.48) --
	( 59.41,139.52) --
	( 59.43,139.53) --
	( 59.45,139.55) --
	( 59.49,139.60) --
	( 59.51,139.62) --
	( 59.53,139.67) --
	( 59.55,139.70) --
	( 59.58,139.76) --
	( 59.60,139.79) --
	( 59.65,139.90) --
	( 59.66,139.95) --
	( 59.70,140.12) --
	( 59.74,140.25) --
	( 59.78,140.32) --
	( 59.81,140.37) --
	( 59.83,140.40) --
	( 59.85,140.43) --
	( 59.88,140.49) --
	( 59.90,140.52) --
	( 59.94,140.60) --
	( 59.98,140.66) --
	( 60.01,140.69) --
	( 60.03,140.73) --
	( 60.04,140.75) --
	( 60.05,140.77) --
	( 60.06,140.80) --
	( 60.07,140.82) --
	( 60.09,140.83) --
	( 60.09,140.85) --
	( 60.10,140.87) --
	( 60.19,141.05) --
	( 60.21,141.12) --
	( 60.23,141.17) --
	( 60.23,141.19) --
	( 60.26,141.25) --
	( 60.26,141.29) --
	( 60.28,141.32) --
	( 60.30,141.43) --
	( 60.34,141.53) --
	( 60.44,141.69) --
	( 60.47,141.73) --
	( 60.49,141.75) --
	( 60.50,141.76) --
	( 60.55,141.79) --
	( 60.56,141.80) --
	( 60.58,141.81) --
	( 60.63,141.84) --
	( 60.69,141.86) --
	( 60.71,141.86) --
	( 60.78,141.88) --
	( 60.83,141.89) --
	( 60.86,141.90) --
	( 60.92,141.91) --
	( 60.98,141.93) --
	( 60.99,141.94) --
	( 61.02,141.95) --
	( 61.04,141.96) --
	( 61.07,141.96) --
	( 61.16,141.98) --
	( 61.35,142.04) --
	( 61.38,142.07) --
	( 61.40,142.08) --
	( 61.41,142.09) --
	( 61.50,142.16) --
	( 61.52,142.18) --
	( 61.54,142.19) --
	( 61.56,142.21) --
	( 61.65,142.26) --
	( 61.71,142.29) --
	( 61.74,142.31) --
	( 61.78,142.34) --
	( 61.90,142.39) --
	( 61.95,142.42) --
	( 62.00,142.46) --
	( 62.07,142.51) --
	( 62.15,142.56) --
	( 62.17,142.58) --
	( 62.19,142.60) --
	( 62.21,142.62) --
	( 62.25,142.66) --
	( 62.35,142.75) --
	( 62.39,142.79) --
	( 62.41,142.80) --
	( 62.44,142.82) --
	( 62.45,142.84) --
	( 62.48,142.85) --
	( 62.64,142.93) --
	( 62.65,142.94) --
	( 62.68,142.96) --
	( 62.72,142.98) --
	( 62.75,142.98) --
	( 62.76,142.99) --
	( 62.80,142.99) --
	( 62.82,143.00) --
	( 62.87,143.01) --
	( 62.92,143.00) --
	( 62.97,142.99) --
	( 63.01,142.98) --
	( 63.03,142.98) --
	( 63.06,142.95) --
	( 63.09,142.94) --
	( 63.15,142.90) --
	( 63.21,142.85) --
	( 63.25,142.82) --
	( 63.28,142.80) --
	( 63.30,142.78) --
	( 63.35,142.75) --
	( 63.37,142.72) --
	( 63.49,142.59) --
	( 63.57,142.53) --
	( 63.59,142.51) --
	( 63.67,142.45) --
	( 63.68,142.45) --
	( 63.72,142.42) --
	( 63.73,142.42) --
	( 63.78,142.38) --
	( 63.83,142.32) --
	( 63.87,142.28) --
	( 63.88,142.27) --
	( 63.90,142.25) --
	( 63.90,142.26) --
	( 63.90,142.26) --
	( 63.90,142.27) --
	( 63.90,142.27) --
	( 63.90,142.28) --
	( 63.90,142.28) --
	( 63.90,142.29) --
	( 63.90,142.29) --
	( 63.90,142.29) --
	( 63.90,142.30) --
	( 63.90,142.30) --
	( 63.90,142.31) --
	( 63.90,142.31) --
	( 63.90,142.32) --
	( 63.90,142.32) --
	( 63.90,142.33) --
	( 63.90,142.33) --
	( 63.90,142.34) --
	( 63.90,142.34) --
	( 63.90,142.35) --
	( 63.90,142.35) --
	( 63.90,142.36) --
	( 63.90,142.36) --
	( 63.90,142.37) --
	( 63.90,142.37) --
	( 63.90,142.38) --
	( 63.90,142.38) --
	( 63.90,142.39) --
	( 63.90,142.39) --
	( 63.90,142.40) --
	( 63.90,142.40) --
	( 63.90,142.41) --
	( 63.93,142.41) --
	( 63.96,142.42) --
	( 64.00,142.42) --
	( 64.12,142.43) --
	( 64.16,142.43) --
	( 64.39,142.40) --
	( 64.48,142.38) --
	( 64.58,142.37) --
	( 64.68,142.37) --
	( 64.72,142.38) --
	( 64.78,142.39) --
	( 64.81,142.40) --
	( 64.85,142.42) --
	( 64.89,142.44) --
	( 64.96,142.47) --
	( 65.03,142.49) --
	( 65.16,142.52) --
	( 65.21,142.53) --
	( 65.26,142.53) --
	( 65.30,142.53) --
	( 65.33,142.53) --
	( 65.36,142.53) --
	( 65.43,142.55) --
	( 65.71,142.65) --
	( 65.79,142.67) --
	( 65.86,142.68) --
	( 65.89,142.68) --
	( 65.92,142.67) --
	( 65.93,142.67) --
	( 65.97,142.67) --
	( 66.00,142.67) --
	( 66.10,142.71) --
	( 66.12,142.73) --
	( 66.22,142.77) --
	( 66.32,142.80) --
	( 66.44,142.83) --
	( 66.50,142.84) --
	( 66.55,142.84) --
	( 66.58,142.83) --
	( 66.62,142.82) --
	( 66.66,142.81) --
	( 66.73,142.78) --
	( 66.76,142.78) --
	( 66.81,142.78) --
	( 66.84,142.77) --
	( 66.86,142.78) --
	( 66.88,142.79) --
	( 66.93,142.82) --
	( 66.98,142.85) --
	( 67.09,142.91) --
	( 67.10,142.92) --
	( 67.12,142.94) --
	( 67.17,143.03) --
	( 67.20,143.07) --
	( 67.22,143.10) --
	( 67.23,143.15) --
	( 67.23,143.18) --
	( 67.23,143.19) --
	( 67.22,143.20) --
	( 67.19,143.22) --
	( 67.07,143.28) --
	( 67.05,143.30) --
	( 67.04,143.32) --
	( 67.03,143.34) --
	( 67.02,143.36) --
	( 67.02,143.50) --
	( 67.01,143.52) --
	( 67.00,143.53) --
	( 66.99,143.55) --
	( 66.96,143.57) --
	( 66.92,143.60) --
	( 66.88,143.62) --
	( 66.85,143.64) --
	( 66.84,143.66) --
	( 66.80,143.75) --
	( 66.78,143.80) --
	( 66.77,143.89) --
	( 66.76,143.95) --
	( 66.72,144.09) --
	( 66.70,144.21) --
	( 66.68,144.29) --
	( 66.66,144.37) --
	( 66.64,144.42) --
	( 66.63,144.44) --
	( 66.60,144.48) --
	( 66.58,144.51) --
	( 66.58,144.56) --
	( 66.57,144.59) --
	( 66.58,144.63) --
	( 66.59,144.65) --
	( 66.60,144.68) --
	( 66.69,144.78) --
	( 66.72,144.82) --
	( 66.75,144.87) --
	( 66.78,144.92) --
	( 66.79,144.98) --
	( 66.80,145.06) --
	( 66.80,145.14) --
	( 66.82,145.24) --
	( 66.84,145.29) --
	( 66.87,145.38) --
	( 66.88,145.41) --
	( 66.87,145.46) --
	( 66.83,145.68) --
	( 66.84,145.71) --
	( 66.84,145.72) --
	( 66.86,145.76) --
	( 66.88,145.79) --
	( 66.95,145.86) --
	( 66.97,145.89) --
	( 66.98,145.91) --
	( 67.00,145.96) --
	( 67.08,146.22) --
	( 67.10,146.27) --
	( 67.10,146.31) --
	( 67.09,146.42) --
	( 67.09,146.47) --
	( 67.09,146.58) --
	( 67.07,146.66) --
	( 67.06,146.70) --
	( 67.04,146.76) --
	( 67.01,146.81) --
	( 66.96,146.88) --
	( 66.89,146.96) --
	( 66.85,146.99) --
	( 66.80,147.02) --
	( 66.75,147.04) --
	( 66.68,147.07) --
	( 66.64,147.10) --
	( 66.61,147.12) --
	( 66.63,147.13) --
	( 66.73,147.21) --
	( 66.76,147.24) --
	( 66.78,147.29) --
	( 66.81,147.36) --
	( 66.84,147.51) --
	( 66.86,147.55) --
	( 66.88,147.58) --
	( 66.91,147.61) --
	( 66.93,147.62) --
	( 66.96,147.64) --
	( 67.00,147.65) --
	( 67.03,147.65) --
	( 67.10,147.65) --
	( 67.12,147.65) --
	( 67.18,147.64) --
	( 67.24,147.65) --
	( 67.27,147.66) --
	( 67.32,147.68) --
	( 67.39,147.72) --
	( 67.44,147.76) --
	( 67.53,147.84) --
	( 67.58,147.88) --
	( 67.68,147.93) --
	( 67.77,147.99) --
	( 67.81,148.03) --
	( 67.86,148.06) --
	( 67.94,148.10) --
	( 68.02,148.12) --
	( 68.14,148.15) --
	( 68.23,148.17) --
	( 68.34,148.19) --
	( 68.38,148.21) --
	( 68.46,148.24) --
	( 68.57,148.31) --
	( 68.65,148.38) --
	( 68.73,148.46) --
	( 68.75,148.48) --
	( 68.80,148.50) --
	( 68.85,148.52) --
	( 68.90,148.55) --
	( 68.94,148.57) --
	( 68.99,148.59) --
	( 69.06,148.60) --
	( 69.08,148.60) --
	( 69.11,148.60) --
	( 69.14,148.59) --
	( 69.16,148.58) --
	( 69.20,148.56) --
	( 69.21,148.54) --
	( 69.27,148.46) --
	( 69.32,148.39) --
	( 69.35,148.33) --
	( 69.42,148.27) --
	( 69.50,148.20) --
	( 69.55,148.17) --
	( 69.64,148.12) --
	( 69.69,148.10) --
	( 69.74,148.08) --
	( 69.82,148.06) --
	( 69.95,148.03) --
	( 69.98,148.01) --
	( 70.01,148.00) --
	( 70.04,147.98) --
	( 70.10,147.93) --
	( 70.14,147.90) --
	( 70.15,147.90) --
	( 70.17,147.90) --
	( 70.19,147.91) --
	( 70.22,147.92) --
	( 70.24,147.94) --
	( 70.32,147.99) --
	( 70.37,148.03) --
	( 70.39,148.05) --
	( 70.40,148.07) --
	( 70.41,148.10) --
	( 70.42,148.13) --
	( 70.43,148.21) --
	( 70.44,148.27) --
	( 70.45,148.32) --
	( 70.51,148.47) --
	( 70.53,148.50) --
	( 70.56,148.56) --
	( 70.60,148.69) --
	( 70.62,148.74) --
	( 70.65,148.77) --
	( 70.68,148.81) --
	( 70.69,148.84) --
	( 70.71,148.96) --
	( 70.73,149.00) --
	( 70.79,149.08) --
	( 70.88,149.16) --
	( 70.93,149.19) --
	( 70.98,149.22) --
	( 71.02,149.26) --
	( 71.06,149.31) --
	( 71.08,149.38) --
	( 71.10,149.42) --
	( 71.09,149.42) --
	( 71.09,149.42) --
	( 71.08,149.42) --
	( 71.08,149.42) --
	( 71.08,149.43) --
	( 71.08,149.45) --
	( 71.08,149.50) --
	( 71.07,149.53) --
	( 71.08,149.55) --
	( 71.08,149.58) --
	( 71.08,149.62) --
	( 71.08,149.64) --
	( 71.08,149.66) --
	( 71.09,149.68) --
	( 71.08,149.72) --
	( 71.08,149.76) --
	( 71.05,149.89) --
	( 71.04,149.92) --
	( 71.04,149.98) --
	( 71.04,150.03) --
	( 71.04,150.07) --
	( 71.02,150.12) --
	( 71.02,150.16) --
	( 70.99,150.21) --
	( 70.97,150.25) --
	( 70.95,150.29) --
	( 70.94,150.33) --
	( 70.91,150.37) --
	( 70.86,150.45) --
	( 70.82,150.57) --
	( 70.82,150.58) --
	( 70.83,150.65) --
	( 70.84,150.68) --
	( 70.86,150.72) --
	( 70.92,150.77) --
	( 70.95,150.80) --
	( 70.97,150.81) --
	( 71.00,150.85) --
	( 71.02,150.89) --
	( 71.02,150.91) --
	( 71.02,150.93) --
	( 71.02,150.95) --
	( 71.02,150.98) --
	( 71.01,151.00) --
	( 71.00,151.02) --
	( 70.98,151.04) --
	( 70.96,151.06) --
	( 70.93,151.08) --
	( 70.89,151.09) --
	( 70.83,151.11) --
	( 70.78,151.12) --
	( 70.71,151.12) --
	( 70.65,151.13) --
	( 70.58,151.15) --
	( 70.50,151.17) --
	( 70.44,151.18) --
	( 70.37,151.21) --
	( 70.31,151.23) --
	( 70.25,151.24) --
	( 70.18,151.26) --
	( 70.11,151.27) --
	( 70.04,151.28) --
	( 69.98,151.29) --
	( 69.90,151.29) --
	( 69.83,151.30) --
	( 69.78,151.30) --
	( 69.72,151.30) --
	( 69.67,151.29) --
	( 69.60,151.29) --
	( 69.52,151.27) --
	( 69.45,151.26) --
	( 69.37,151.24) --
	( 69.19,151.21) --
	( 69.12,151.20) --
	( 69.06,151.20) --
	( 69.00,151.20) --
	( 68.90,151.21) --
	( 68.85,151.22) --
	( 68.79,151.23) --
	( 68.74,151.25) --
	( 68.68,151.27) --
	( 68.66,151.29) --
	( 68.61,151.35) --
	( 68.55,151.42) --
	( 68.52,151.47) --
	( 68.47,151.53) --
	( 68.43,151.57) --
	( 68.38,151.61) --
	( 68.35,151.64) --
	( 68.29,151.70) --
	( 68.22,151.76) --
	( 68.16,151.81) --
	( 68.14,151.84) --
	( 68.12,151.87) --
	( 68.10,151.89) --
	( 68.10,151.93) --
	( 68.09,152.00) --
	( 68.10,152.05) --
	( 68.11,152.08) --
	( 68.13,152.11) --
	( 68.14,152.13) --
	( 68.16,152.15) --
	( 68.20,152.18) --
	( 68.29,152.24) --
	( 68.34,152.28) --
	( 68.39,152.33) --
	( 68.41,152.37) --
	( 68.41,152.40) --
	( 68.42,152.42) --
	( 68.42,152.45) --
	( 68.41,152.52) --
	( 68.41,152.55) --
	( 68.39,152.59) --
	( 68.38,152.60) --
	( 68.34,152.62) --
	( 68.30,152.65) --
	( 68.24,152.69) --
	( 68.19,152.71) --
	( 68.15,152.74) --
	( 68.11,152.76) --
	( 68.07,152.79) --
	( 68.04,152.82) --
	( 68.00,152.84) --
	( 67.98,152.84) --
	( 67.94,152.85) --
	( 67.89,152.87) --
	( 67.81,152.88) --
	( 67.78,152.88) --
	( 67.74,152.88) --
	( 67.69,152.87) --
	( 67.63,152.86) --
	( 67.45,152.82) --
	( 67.31,152.78) --
	( 67.27,152.77) --
	( 67.21,152.76) --
	( 67.14,152.76) --
	( 67.08,152.75) --
	( 67.02,152.74) --
	( 66.91,152.73) --
	( 66.90,152.73) --
	( 66.81,152.74) --
	( 66.79,152.74) --
	( 66.78,152.74) --
	( 66.76,152.75) --
	( 66.73,152.78) --
	( 66.70,152.81) --
	( 66.69,152.83) --
	( 66.68,152.88) --
	( 66.68,152.90) --
	( 66.69,152.94) --
	( 66.72,152.98) --
	( 66.75,153.01) --
	( 66.77,153.05) --
	( 66.80,153.08) --
	( 66.82,153.12) --
	( 66.84,153.17) --
	( 66.86,153.22) --
	( 66.86,153.25) --
	( 66.88,153.30) --
	( 66.87,153.33) --
	( 66.88,153.41) --
	( 66.88,153.47) --
	( 66.87,153.50) --
	( 66.87,153.54) --
	( 66.87,153.59) --
	( 66.87,153.65) --
	( 66.88,153.71) --
	( 66.89,153.76) --
	( 66.92,153.80) --
	( 66.94,153.85) --
	( 66.99,153.91) --
	( 67.04,153.98) --
	( 67.09,154.01) --
	( 67.15,154.06) --
	( 67.20,154.10) --
	( 67.24,154.12) --
	( 67.28,154.16) --
	( 67.37,154.25) --
	( 67.41,154.29) --
	( 67.45,154.34) --
	( 67.48,154.37) --
	( 67.51,154.41) --
	( 67.55,154.45) --
	( 67.59,154.48) --
	( 67.62,154.50) --
	( 67.64,154.51) --
	( 67.70,154.54) --
	( 67.83,154.58) --
	( 67.85,154.60) --
	( 67.88,154.62) --
	( 67.89,154.63) --
	( 67.90,154.65) --
	( 67.90,154.72) --
	( 67.89,154.75) --
	( 67.85,154.78) --
	( 67.78,154.81) --
	( 67.73,154.82) --
	( 67.69,154.83) --
	( 67.67,154.83) --
	( 67.65,154.85) --
	( 67.61,154.86) --
	( 67.57,154.87) --
	( 67.54,154.89) --
	( 67.51,154.92) --
	( 67.49,154.96) --
	( 67.47,155.01) --
	( 67.43,155.10) --
	( 67.42,155.15) --
	( 67.40,155.19) --
	( 67.37,155.25) --
	( 67.33,155.30) --
	( 67.30,155.34) --
	( 67.25,155.40) --
	( 67.21,155.45) --
	( 67.17,155.49) --
	( 67.07,155.65) --
	( 67.03,155.72) --
	( 66.99,155.80) --
	( 66.95,155.87) --
	( 66.93,155.94) --
	( 66.90,156.01) --
	( 66.89,156.05) --
	( 66.88,156.11) --
	( 66.87,156.16) --
	( 66.85,156.19) --
	( 66.82,156.22) --
	( 66.79,156.25) --
	( 66.78,156.26) --
	( 66.75,156.27) --
	( 66.70,156.30) --
	( 66.65,156.32) --
	( 66.60,156.34) --
	( 66.60,156.37) --
	( 66.58,156.43) --
	( 66.53,156.59) --
	( 66.47,156.67) --
	( 66.44,156.71) --
	( 66.40,156.77) --
	( 66.39,156.81) --
	( 66.38,156.85) --
	( 66.39,156.91) --
	( 66.40,156.94) --
	( 66.42,156.99) --
	( 66.44,157.05) --
	( 66.48,157.09) --
	( 66.51,157.15) --
	( 66.52,157.17) --
	( 66.53,157.21) --
	( 66.54,157.24) --
	( 66.55,157.27) --
	( 66.54,157.33) --
	( 66.53,157.36) --
	( 66.55,157.38) --
	( 66.56,157.38) --
	( 66.59,157.39) --
	( 66.62,157.40) --
	( 66.65,157.41) --
	( 66.67,157.44) --
	( 66.68,157.46) --
	( 66.71,157.59) --
	( 66.73,157.64) --
	( 66.75,157.70) --
	( 66.79,157.76) --
	( 66.82,157.81) --
	( 66.85,157.86) --
	( 66.86,157.91) --
	( 66.88,157.96) --
	( 66.89,158.02) --
	( 66.90,158.09) --
	( 66.90,158.17) --
	( 66.91,158.21) --
	( 66.92,158.26) --
	( 66.94,158.31) --
	( 66.98,158.39) --
	( 67.01,158.45) --
	( 67.05,158.50) --
	( 67.06,158.55) --
	( 67.07,158.61) --
	( 67.08,158.67) --
	( 67.10,158.73) --
	( 67.11,158.79) --
	( 67.13,158.87) --
	( 67.16,158.93) --
	( 67.18,159.00) --
	( 67.20,159.04) --
	( 67.23,159.11) --
	( 67.30,159.19) --
	( 67.38,159.32) --
	( 67.44,159.41) --
	( 67.50,159.51) --
	( 67.55,159.60) --
	( 67.62,159.71) --
	( 67.66,159.79) --
	( 67.69,159.84) --
	( 67.74,159.90) --
	( 67.84,160.00) --
	( 67.94,160.11) --
	( 68.01,160.21) --
	( 68.08,160.30) --
	( 68.12,160.37) --
	( 68.15,160.41) --
	( 68.17,160.46) --
	( 68.18,160.51) --
	( 68.21,160.54) --
	( 68.28,160.62) --
	( 68.35,160.69) --
	( 68.42,160.78) --
	( 68.48,160.87) --
	( 68.52,160.93) --
	( 68.55,160.98) --
	( 68.57,161.05) --
	( 68.58,161.13) --
	( 68.58,161.19) --
	( 68.59,161.25) --
	( 68.59,161.29) --
	( 68.59,161.33) --
	( 68.62,161.42) --
	( 68.66,161.49) --
	( 68.72,161.58) --
	( 68.78,161.65) --
	( 68.87,161.76) --
	( 68.97,161.84) --
	( 69.16,162.01) --
	( 69.26,162.08) --
	( 69.34,162.15) --
	( 69.44,162.22) --
	( 69.53,162.28) --
	( 69.58,162.32) --
	( 69.64,162.38) --
	( 69.66,162.42) --
	( 69.68,162.45) --
	( 69.69,162.47) --
	( 69.70,162.52) --
	( 69.70,162.56) --
	( 69.70,162.62) --
	( 69.68,162.68) --
	( 69.66,162.75) --
	( 69.64,162.81) --
	( 69.62,162.88) --
	( 69.60,162.92) --
	( 69.58,163.07) --
	( 69.58,163.09) --
	( 69.60,163.12) --
	( 69.65,163.15) --
	( 69.72,163.18) --
	( 69.78,163.20) --
	( 69.82,163.21) --
	( 69.91,163.22) --
	( 69.99,163.24) --
	( 70.07,163.25) --
	( 70.10,163.27) --
	( 70.14,163.30) --
	( 70.18,163.33) --
	( 70.20,163.37) --
	( 70.21,163.42) --
	( 70.23,163.47) --
	( 70.23,163.52) --
	( 70.23,163.54) --
	( 70.23,163.58) --
	( 70.23,163.62) --
	( 70.26,163.71) --
	( 70.29,163.76) --
	( 70.32,163.80) --
	( 70.34,163.83) --
	( 70.35,163.89) --
	( 70.36,163.94) --
	( 70.36,163.98) --
	( 70.35,164.02) --
	( 70.34,164.07) --
	( 70.31,164.15) --
	( 70.30,164.19) --
	( 70.30,164.24) --
	( 70.30,164.26) --
	( 70.31,164.31) --
	( 70.33,164.37) --
	( 70.34,164.40) --
	( 70.37,164.49) --
	( 70.40,164.54) --
	( 70.45,164.60) --
	( 70.51,164.66) --
	( 70.55,164.70) --
	( 70.62,164.74) --
	( 70.67,164.78) --
	( 70.70,164.80) --
	( 70.73,164.84) --
	( 70.76,164.88) --
	( 70.87,165.06) --
	( 70.93,165.15) --
	( 70.98,165.23) --
	( 71.03,165.31) --
	( 71.06,165.38) --
	( 71.08,165.44) --
	( 71.11,165.51) --
	( 71.16,165.60) --
	( 71.19,165.67) --
	( 71.21,165.73) --
	( 71.23,165.79) --
	( 71.24,165.84) --
	( 71.25,165.88) --
	( 71.26,165.94) --
	( 71.27,165.98) --
	( 71.27,166.02) --
	( 71.28,166.05) --
	( 71.27,166.08) --
	( 71.24,166.15) --
	( 71.20,166.20) --
	( 71.16,166.24) --
	( 71.11,166.29) --
	( 71.05,166.35) --
	( 71.00,166.40) --
	( 70.97,166.43) --
	( 70.95,166.45) --
	( 70.92,166.49) --
	( 70.90,166.53) --
	( 70.88,166.55) --
	( 70.87,166.57) --
	( 70.81,166.60) --
	( 70.79,166.62) --
	( 70.76,166.64) --
	( 70.75,166.65) --
	( 70.74,166.66) --
	( 70.74,166.67) --
	( 70.78,166.71) --
	( 70.82,166.71) --
	( 70.86,166.72) --
	( 70.90,166.73) --
	( 71.00,166.75) --
	( 71.05,166.76) --
	( 71.11,166.76) --
	( 71.16,166.76) --
	( 71.21,166.75) --
	( 71.28,166.74) --
	( 71.33,166.74) --
	( 71.37,166.74) --
	( 71.41,166.74) --
	( 71.46,166.76) --
	( 71.49,166.79) --
	( 71.53,166.82) --
	( 71.58,166.87) --
	( 71.61,166.90) --
	( 71.64,166.94) --
	( 71.67,166.99) --
	( 71.68,166.99) --
	( 71.71,167.00) --
	( 71.74,167.00) --
	( 71.78,167.00) --
	( 71.82,167.00) --
	( 71.88,166.99) --
	( 71.91,166.98) --
	( 71.99,166.97) --
	( 72.07,166.96) --
	( 72.16,166.94) --
	( 72.23,166.94) --
	( 72.32,166.94) --
	( 72.38,166.95) --
	( 72.43,166.96) --
	( 72.50,166.98) --
	( 72.54,167.01) --
	( 72.55,167.02) --
	( 72.55,167.04) --
	( 72.55,167.06) --
	( 72.55,167.10) --
	( 72.54,167.13) --
	( 72.54,167.16) --
	( 72.54,167.18) --
	( 72.55,167.19) --
	( 72.55,167.21) --
	( 72.57,167.23) --
	( 72.59,167.25) --
	( 72.62,167.27) --
	( 72.67,167.28) --
	( 72.75,167.30) --
	( 72.82,167.32) --
	( 72.87,167.35) --
	( 72.92,167.38) --
	( 72.96,167.41) --
	( 73.00,167.45) --
	( 73.02,167.49) --
	( 73.06,167.53) --
	( 73.09,167.58) --
	( 73.11,167.61) --
	( 73.15,167.66) --
	( 73.17,167.69) --
	( 73.21,167.72) --
	( 73.25,167.74) --
	( 73.30,167.74) --
	( 73.34,167.75) --
	( 73.38,167.75) --
	( 73.42,167.73) --
	( 73.45,167.72) --
	( 73.49,167.70) --
	( 73.53,167.68) --
	( 73.58,167.64) --
	( 73.69,167.58) --
	( 73.74,167.55) --
	( 73.88,167.50) --
	( 73.92,167.47) --
	( 73.97,167.47) --
	( 74.02,167.46) --
	( 74.07,167.46) --
	( 74.11,167.46) --
	( 74.14,167.46) --
	( 74.19,167.47) --
	( 74.26,167.50) --
	( 74.31,167.52) --
	( 74.34,167.54) --
	( 74.37,167.56) --
	( 74.41,167.58) --
	( 74.44,167.62) --
	( 74.49,167.65) --
	( 74.55,167.72) --
	( 74.60,167.76) --
	( 74.63,167.79) --
	( 74.66,167.83) --
	( 74.69,167.87) --
	( 74.71,167.90) --
	( 74.75,167.95) --
	( 74.80,168.00) --
	( 74.84,168.03) --
	( 74.88,168.05) --
	( 74.93,168.07) --
	( 74.97,168.10) --
	( 75.06,168.14) --
	( 75.11,168.16) --
	( 75.17,168.17) --
	( 75.20,168.18) --
	( 75.24,168.17) --
	( 75.27,168.16) --
	( 75.35,168.14) --
	( 75.36,168.11) --
	( 75.37,168.09) --
	( 75.38,168.04) --
	( 75.40,168.00) --
	( 75.40,167.97) --
	( 75.41,167.95) --
	( 75.42,167.94) --
	( 75.42,167.93) --
	( 75.43,167.92) --
	( 75.44,167.92) --
	( 75.45,167.91) --
	( 75.46,167.91) --
	( 75.48,167.91) --
	( 75.50,167.91) --
	( 75.51,167.91) --
	( 75.52,167.92) --
	( 75.54,167.94) --
	( 75.55,167.96) --
	( 75.56,167.98) --
	( 75.57,168.01) --
	( 75.57,168.03) --
	( 75.58,168.07) --
	( 75.59,168.12) --
	( 75.59,168.14) --
	( 75.63,168.20) --
	( 75.65,168.22) --
	( 75.71,168.26) --
	( 75.73,168.27) --
	( 75.78,168.28) --
	( 75.89,168.30) --
	( 75.94,168.31) --
	( 75.99,168.32) --
	( 76.03,168.31) --
	( 76.08,168.31) --
	( 76.14,168.30) --
	( 76.21,168.29) --
	( 76.25,168.29) --
	( 76.27,168.30) --
	( 76.29,168.30) --
	( 76.33,168.32) --
	( 76.46,168.38) --
	( 76.56,168.49) --
	( 76.58,168.55) --
	( 76.60,168.58) --
	( 76.62,168.62) --
	( 76.64,168.65) --
	( 76.66,168.70) --
	( 76.67,168.73) --
	( 76.67,168.76) --
	( 76.67,168.79) --
	( 76.67,168.85) --
	( 76.67,168.87) --
	( 76.68,168.88) --
	( 76.69,168.90) --
	( 76.71,168.92) --
	( 76.80,168.98) --
	( 76.84,169.00) --
	( 76.87,169.02) --
	( 76.90,169.04) --
	( 76.91,169.06) --
	( 76.91,169.07) --
	( 76.91,169.09) --
	( 76.91,169.10) --
	( 76.92,169.12) --
	( 76.91,169.14) --
	( 76.88,169.17) --
	( 76.86,169.18) --
	( 76.84,169.20) --
	( 76.81,169.21) --
	( 76.76,169.22) --
	( 76.72,169.23) --
	( 76.69,169.25) --
	( 76.66,169.27) --
	( 76.65,169.27) --
	( 76.63,169.32) --
	( 76.62,169.35) --
	( 76.61,169.38) --
	( 76.59,169.42) --
	( 76.58,169.47) --
	( 76.56,169.54) --
	( 76.54,169.58) --
	( 76.51,169.62) --
	( 76.50,169.65) --
	( 76.50,169.67) --
	( 76.50,169.69) --
	( 76.50,169.72) --
	( 76.50,169.75) --
	( 76.51,169.78) --
	( 76.52,169.81) --
	( 76.53,169.83) --
	( 76.54,169.85) --
	( 76.57,169.88) --
	( 76.60,169.90) --
	( 76.63,169.92) --
	( 76.65,169.94) --
	( 76.66,169.96) --
	( 76.67,169.98) --
	( 76.68,170.01) --
	( 76.68,170.04) --
	( 76.69,170.07) --
	( 76.69,170.11) --
	( 76.68,170.15) --
	( 76.67,170.18) --
	( 76.67,170.20) --
	( 76.66,170.22) --
	( 76.63,170.24) --
	( 76.61,170.27) --
	( 76.57,170.29) --
	( 76.57,170.30) --
	( 76.57,170.32) --
	( 76.57,170.33) --
	( 76.59,170.35) --
	( 76.62,170.38) --
	( 76.66,170.41) --
	( 76.69,170.43) --
	( 76.72,170.44) --
	( 76.80,170.48) --
	( 76.84,170.50) --
	( 76.89,170.52) --
	( 76.90,170.53) --
	( 76.91,170.55) --
	( 76.92,170.56) --
	( 76.92,170.57) --
	( 76.92,170.59) --
	( 76.92,170.61) --
	( 76.91,170.62) --
	( 76.90,170.64) --
	( 76.88,170.65) --
	( 76.86,170.67) --
	( 76.85,170.69) --
	( 76.84,170.71) --
	( 76.84,170.73) --
	( 76.85,170.75) --
	( 76.94,170.80) --
	( 76.99,170.84) --
	( 77.00,170.86) --
	( 77.00,170.88) --
	( 76.99,170.90) --
	( 76.98,170.93) --
	( 76.96,170.97) --
	( 76.93,170.99) --
	( 76.90,171.02) --
	( 76.87,171.05) --
	( 76.83,171.09) --
	( 76.76,171.15) --
	( 76.72,171.16) --
	( 76.68,171.17) --
	( 76.65,171.18) --
	( 76.58,171.18) --
	( 76.55,171.19) --
	( 76.52,171.19) --
	( 76.47,171.19) --
	( 76.41,171.21) --
	( 76.39,171.23) --
	( 76.37,171.26) --
	( 76.35,171.29) --
	( 76.34,171.33) --
	( 76.33,171.36) --
	( 76.33,171.39) --
	( 76.33,171.43) --
	( 76.34,171.45) --
	( 76.36,171.48) --
	( 76.38,171.53) --
	( 76.39,171.58) --
	( 76.40,171.66) --
	( 76.39,171.69) --
	( 76.38,171.72) --
	( 76.36,171.75) --
	( 76.34,171.78) --
	( 76.32,171.82) --
	( 76.32,171.85) --
	( 76.32,171.87) --
	( 76.33,171.90) --
	( 76.35,171.93) --
	( 76.36,171.95) --
	( 76.40,171.96) --
	( 76.42,171.97) --
	( 76.43,171.99) --
	( 76.46,172.01) --
	( 76.48,172.03) --
	( 76.51,172.06) --
	( 76.55,172.14) --
	( 76.56,172.16) --
	( 76.58,172.19) --
	( 76.61,172.23) --
	( 76.62,172.26) --
	( 76.64,172.29) --
	( 76.65,172.31) --
	( 76.70,172.33) --
	( 76.73,172.35) --
	( 76.76,172.36) --
	( 76.80,172.36) --
	( 76.84,172.37) --
	( 76.87,172.38) --
	( 76.88,172.39) --
	( 76.90,172.41) --
	( 76.91,172.43) --
	( 76.91,172.45) --
	( 76.91,172.48) --
	( 76.92,172.52) --
	( 76.92,172.57) --
	( 76.93,172.61) --
	( 76.95,172.64) --
	( 76.97,172.67) --
	( 77.00,172.70) --
	( 77.04,172.73) --
	( 77.06,172.75) --
	( 77.13,172.81) --
	( 77.16,172.84) --
	( 77.17,172.85) --
	( 77.21,172.87) --
	( 77.27,172.89) --
	( 77.31,172.90) --
	( 77.36,172.92) --
	( 77.40,172.93) --
	( 77.44,172.94) --
	( 77.46,172.95) --
	( 77.49,172.96) --
	( 77.51,172.97) --
	( 77.53,172.98) --
	( 77.55,173.00) --
	( 77.57,173.00) --
	( 77.58,173.00) --
	( 77.60,173.00) --
	( 77.62,173.00) --
	( 77.64,173.00) --
	( 77.67,173.00) --
	( 77.73,173.00) --
	( 77.74,173.00) --
	( 77.74,173.02) --
	( 77.74,173.03) --
	( 77.74,173.06) --
	( 77.73,173.09) --
	( 77.73,173.11) --
	( 77.71,173.14) --
	( 77.68,173.17) --
	( 77.66,173.20) --
	( 77.64,173.23) --
	( 77.63,173.26) --
	( 77.63,173.28) --
	( 77.64,173.33) --
	( 77.64,173.36) --
	( 77.64,173.38) --
	( 77.63,173.41) --
	( 77.62,173.45) --
	( 77.61,173.49) --
	( 77.61,173.52) --
	( 77.60,173.55) --
	( 77.60,173.57) --
	( 77.61,173.59) --
	( 77.61,173.64) --
	( 77.62,173.67) --
	( 77.63,173.70) --
	( 77.64,173.73) --
	( 77.64,173.77) --
	( 77.64,173.80) --
	( 77.63,173.83) --
	( 77.61,173.84) --
	( 77.58,173.87) --
	( 77.56,173.89) --
	( 77.53,173.91) --
	( 77.50,173.93) --
	( 77.47,173.94) --
	( 77.42,173.96) --
	( 77.37,173.97) --
	( 77.32,173.99) --
	( 77.28,173.99) --
	( 77.24,173.99) --
	( 77.18,174.00) --
	( 77.12,174.00) --
	( 77.05,174.01) --
	( 76.94,174.00) --
	( 76.77,173.99) --
	( 76.72,173.98) --
	( 76.65,173.98) --
	( 76.61,173.99) --
	( 76.56,173.99) --
	( 76.50,174.00) --
	( 76.46,174.01) --
	( 76.43,174.03) --
	( 76.40,174.05) --
	( 76.38,174.07) --
	( 76.35,174.09) --
	( 76.35,174.09) --
	( 76.34,174.09) --
	( 76.34,174.09) --
	( 76.33,174.09) --
	( 76.33,174.09) --
	( 76.32,174.09) --
	( 76.32,174.09) --
	( 76.31,174.09) --
	( 76.31,174.09) --
	( 76.30,174.09) --
	( 76.30,174.09) --
	( 76.29,174.09) --
	( 76.29,174.09) --
	( 76.28,174.09) --
	( 76.28,174.09) --
	( 76.27,174.12) --
	( 76.26,174.14) --
	( 76.24,174.17) --
	( 76.19,174.22) --
	( 76.17,174.24) --
	( 76.15,174.28) --
	( 76.14,174.29) --
	( 76.12,174.32) --
	( 76.11,174.36) --
	( 76.10,174.44) --
	( 76.09,174.46) --
	( 76.06,174.56) --
	( 76.05,174.58) --
	( 76.02,174.64) --
	( 75.98,174.71) --
	( 75.97,174.72) --
	( 75.97,174.74) --
	( 75.97,174.76) --
	( 75.96,174.78) --
	( 75.96,174.79) --
	( 75.97,174.83) --
	( 75.98,174.89) --
	( 75.98,174.92) --
	( 76.00,174.98) --
	( 76.01,175.05) --
	( 76.01,175.14) --
	( 76.02,175.20) --
	( 76.01,175.24) --
	( 76.00,175.30) --
	( 75.99,175.31) --
	( 75.98,175.34) --
	( 75.98,175.35) --
	( 75.98,175.40) --
	( 75.99,175.41) --
	( 75.99,175.42) --
	( 75.99,175.44) --
	( 76.00,175.47) --
	( 76.00,175.49) --
	( 76.01,175.50) --
	( 76.01,175.51) --
	( 76.04,175.55) --
	( 76.06,175.57) --
	( 76.07,175.58) --
	( 76.09,175.60) --
	( 76.16,175.66) --
	( 76.19,175.69) --
	( 76.20,175.71) --
	( 76.21,175.73) --
	( 76.22,175.76) --
	( 76.22,175.78) --
	( 76.22,175.80) --
	( 76.22,175.82) --
	( 76.21,175.84) --
	( 76.21,175.85) --
	( 76.17,175.92) --
	( 76.16,175.95) --
	( 76.16,175.96) --
	( 76.16,175.98) --
	( 76.16,176.00) --
	( 76.16,176.02) --
	( 76.20,176.11) --
	( 76.21,176.14) --
	( 76.22,176.19) --
	( 76.22,176.21) --
	( 76.23,176.25) --
	( 76.25,176.35) --
	( 76.27,176.41) --
	( 76.28,176.44) --
	( 76.28,176.47) --
	( 76.29,176.49) --
	( 76.29,176.52) --
	( 76.30,176.62) --
	( 76.31,176.66) --
	( 76.31,176.69) --
	( 76.31,176.75) --
	( 76.30,176.76) --
	( 76.29,176.84) --
	( 76.29,176.88) --
	( 76.28,176.90) --
	( 76.28,176.92) --
	( 76.26,176.95) --
	( 76.22,177.01) --
	( 76.21,177.02) --
	( 76.18,177.05) --
	( 76.16,177.07) --
	( 76.15,177.08) --
	( 76.15,177.09) --
	( 76.15,177.10) --
	( 76.15,177.10) --
	( 76.16,177.11) --
	( 76.16,177.12) --
	( 76.17,177.13) --
	( 76.17,177.13) --
	( 76.20,177.13) --
	( 76.27,177.14) --
	( 76.35,177.16) --
	( 76.37,177.16) --
	( 76.39,177.17) --
	( 76.48,177.22) --
	( 76.52,177.24) --
	( 76.53,177.26) --
	( 76.55,177.28) --
	( 76.57,177.31) --
	( 76.59,177.33) --
	( 76.59,177.35) --
	( 76.60,177.38) --
	( 76.63,177.41) --
	( 76.64,177.42) --
	( 76.64,177.44) --
	( 76.65,177.45) --
	( 76.65,177.49) --
	( 76.65,177.50) --
	( 76.61,177.58) --
	( 76.59,177.63) --
	( 76.59,177.66) --
	( 76.60,177.68) --
	( 76.61,177.72) --
	( 76.61,177.75) --
	( 76.61,177.76) --
	( 76.60,177.79) --
	( 76.59,177.82) --
	( 76.57,177.86) --
	( 76.56,177.88) --
	( 76.53,178.00) --
	( 76.53,178.01) --
	( 76.53,178.07) --
	( 76.55,178.15) --
	( 76.56,178.20) --
	( 76.56,178.22) --
	( 76.56,178.25) --
	( 76.55,178.28) --
	( 76.50,178.39) --
	( 76.48,178.42) --
	( 76.47,178.44) --
	( 76.47,178.46) --
	( 76.47,178.53) --
	( 76.47,178.58) --
	( 76.47,178.59) --
	( 76.47,178.61) --
	( 76.49,178.64) --
	( 76.51,178.66) --
	( 76.52,178.67) --
	( 76.54,178.68) --
	( 76.59,178.72) --
	( 76.60,178.73) --
	( 76.62,178.75) --
	( 76.64,178.80) --
	( 76.65,178.82) --
	( 76.65,178.85) --
	( 76.66,178.87) --
	( 76.65,178.92) --
	( 76.64,179.01) --
	( 76.63,179.12) --
	( 76.62,179.22) --
	( 76.62,179.24) --
	( 76.62,179.28) --
	( 76.62,179.30) --
	( 76.63,179.32) --
	( 76.64,179.37) --
	( 76.66,179.40) --
	( 76.66,179.41) --
	( 76.67,179.42) --
	( 76.69,179.47) --
	( 76.72,179.49) --
	( 76.75,179.53) --
	( 76.76,179.54) --
	( 76.76,179.56) --
	( 76.78,179.58) --
	( 76.78,179.60) --
	( 76.78,179.63) --
	( 76.79,179.66) --
	( 76.79,179.68) --
	( 76.79,179.69) --
	( 76.79,179.71) --
	( 76.80,179.73) --
	( 76.81,179.75) --
	( 76.83,179.79) --
	( 76.85,179.83) --
	( 76.87,179.85) --
	( 76.88,179.87) --
	( 76.91,179.90) --
	( 76.93,179.92) --
	( 76.94,179.94) --
	( 76.99,179.96) --
	( 77.00,179.98) --
	( 77.01,179.99) --
	( 77.02,180.00) --
	( 77.02,180.02) --
	( 77.03,180.04) --
	( 77.03,180.06) --
	( 77.03,180.08) --
	( 77.04,180.10) --
	( 77.04,180.11) --
	( 77.06,180.14) --
	( 77.08,180.15) --
	( 77.09,180.16) --
	( 77.10,180.16) --
	( 77.12,180.17) --
	( 77.15,180.17) --
	( 77.17,180.18) --
	( 77.25,180.18) --
	( 77.31,180.19) --
	( 77.32,180.19) --
	( 77.33,180.19) --
	( 77.39,180.21) --
	( 77.40,180.21) --
	( 77.41,180.22) --
	( 77.42,180.23) --
	( 77.44,180.24) --
	( 77.49,180.31) --
	( 77.51,180.33) --
	( 77.53,180.35) --
	( 77.55,180.38) --
	( 77.56,180.39) --
	( 77.56,180.42) --
	( 77.56,180.46) --
	( 77.55,180.48) --
	( 77.55,180.50) --
	( 77.54,180.53) --
	( 77.54,180.57) --
	( 77.53,180.59) --
	( 77.53,180.61) --
	( 77.53,180.63) --
	( 77.54,180.64) --
	( 77.54,180.66) --
	( 77.56,180.69) --
	( 77.57,180.71) --
	( 77.58,180.72) --
	( 77.61,180.76) --
	( 77.62,180.78) --
	( 77.64,180.81) --
	( 77.64,180.84) --
	( 77.64,180.88) --
	( 77.64,180.90) --
	( 77.63,180.92) --
	( 77.62,180.96) --
	( 77.60,180.99) --
	( 77.59,181.00) --
	( 77.59,181.02) --
	( 77.57,181.04) --
	( 77.55,181.08) --
	( 77.54,181.12) --
	( 77.53,181.13) --
	( 77.53,181.15) --
	( 77.53,181.18) --
	( 77.54,181.22) --
	( 77.55,181.23) --
	( 77.56,181.26) --
	( 77.58,181.29) --
	( 77.62,181.32) --
	( 77.63,181.33) --
	( 77.64,181.38) --
	( 77.66,181.44) --
	( 77.66,181.46) --
	( 77.65,181.49) --
	( 77.64,181.51) --
	( 77.63,181.52) --
	( 77.60,181.56) --
	( 77.59,181.58) --
	( 77.58,181.59) --
	( 77.57,181.60) --
	( 77.55,181.62) --
	( 77.55,181.63) --
	( 77.55,181.64) --
	( 77.54,181.66) --
	( 77.54,181.69) --
	( 77.54,181.70) --
	( 77.54,181.71) --
	( 77.56,181.74) --
	( 77.57,181.76) --
	( 77.58,181.77) --
	( 77.60,181.78) --
	( 77.64,181.82) --
	( 77.66,181.87) --
	( 77.68,181.91) --
	( 77.69,181.93) --
	( 77.69,181.94) --
	( 77.69,181.96) --
	( 77.69,181.97) --
	( 77.69,181.98) --
	( 77.68,182.00) --
	( 77.66,182.02) --
	( 77.65,182.04) --
	( 77.63,182.06) --
	( 77.60,182.10) --
	( 77.60,182.12) --
	( 77.59,182.14) --
	( 77.58,182.16) --
	( 77.56,182.18) --
	( 77.55,182.19) --
	( 77.54,182.20) --
	( 77.52,182.22) --
	( 77.48,182.24) --
	( 77.46,182.25) --
	( 77.46,182.26) --
	( 77.45,182.27) --
	( 77.44,182.29) --
	( 77.44,182.31) --
	( 77.44,182.32) --
	( 77.45,182.33) --
	( 77.46,182.37) --
	( 77.48,182.38) --
	( 77.50,182.41) --
	( 77.51,182.42) --
	( 77.53,182.44) --
	( 77.57,182.51) --
	( 77.59,182.54) --
	( 77.61,182.61) --
	( 77.62,182.63) --
	( 77.62,182.64) --
	( 77.65,182.72) --
	( 77.68,182.77) --
	( 77.70,182.81) --
	( 77.73,182.83) --
	( 77.74,182.84) --
	( 77.75,182.85) --
	( 77.78,182.89) --
	( 77.79,182.90) --
	( 77.81,182.94) --
	( 77.82,182.98) --
	( 77.82,183.01) --
	( 77.82,183.03) --
	( 77.83,183.05) --
	( 77.84,183.07) --
	( 77.85,183.09) --
	( 77.86,183.11) --
	( 77.88,183.13) --
	( 77.91,183.15) --
	( 77.93,183.17) --
	( 77.95,183.18) --
	( 77.97,183.19) --
	( 78.01,183.20) --
	( 78.04,183.21) --
	( 78.07,183.21) --
	( 78.10,183.22) --
	( 78.12,183.22) --
	( 78.14,183.21) --
	( 78.22,183.20) --
	( 78.25,183.20) --
	( 78.29,183.19) --
	( 78.32,183.19) --
	( 78.36,183.18) --
	( 78.40,183.18) --
	( 78.42,183.18) --
	( 78.44,183.19) --
	( 78.51,183.21) --
	( 78.53,183.22) --
	( 78.54,183.23) --
	( 78.56,183.25) --
	( 78.61,183.29) --
	( 78.63,183.31) --
	( 78.64,183.32) --
	( 78.65,183.33) --
	( 78.66,183.34) --
	( 78.69,183.35) --
	( 78.70,183.35) --
	( 78.74,183.37) --
	( 78.75,183.38) --
	( 78.75,183.38) --
	( 78.76,183.40) --
	( 78.78,183.42) --
	( 78.78,183.43) --
	( 78.80,183.46) --
	( 78.80,183.48) --
	( 78.80,183.50) --
	( 78.81,183.54) --
	( 78.81,183.55) --
	( 78.83,183.58) --
	( 78.84,183.59) --
	( 78.87,183.60) --
	( 78.88,183.60) --
	( 79.05,183.63) --
	( 79.06,183.64) --
	( 79.07,183.64) --
	( 79.08,183.65) --
	( 79.08,183.65) --
	( 79.09,183.66) --
	( 79.10,183.67) --
	( 79.11,183.71) --
	( 79.12,183.73) --
	( 79.12,183.76) --
	( 79.13,183.77) --
	( 79.14,183.82) --
	( 79.15,183.83) --
	( 79.15,183.85) --
	( 79.17,183.86) --
	( 79.18,183.88) --
	( 79.19,183.89) --
	( 79.21,183.90) --
	( 79.23,183.92) --
	( 79.27,183.94) --
	( 79.29,183.95) --
	( 79.31,183.95) --
	( 79.31,183.96) --
	( 79.33,183.97) --
	( 79.40,184.00) --
	( 79.40,184.01) --
	( 79.40,184.02) --
	( 79.41,184.03) --
	( 79.41,184.05) --
	( 79.40,184.07) --
	( 79.39,184.09) --
	( 79.38,184.13) --
	( 79.36,184.17) --
	( 79.35,184.19) --
	( 79.35,184.23) --
	( 79.35,184.24) --
	( 79.36,184.26) --
	( 79.37,184.27) --
	( 79.38,184.29) --
	( 79.40,184.32) --
	( 79.41,184.33) --
	( 79.43,184.38) --
	( 79.44,184.39) --
	( 79.45,184.40) --
	( 79.51,184.42) --
	( 79.56,184.44) --
	( 79.58,184.45) --
	( 79.64,184.48) --
	( 79.66,184.49) --
	( 79.67,184.51) --
	( 79.68,184.52) --
	( 79.70,184.53) --
	( 79.73,184.57) --
	( 79.74,184.59) --
	( 79.75,184.62) --
	( 79.75,184.63) --
	( 79.76,184.69) --
	( 79.76,184.72) --
	( 79.77,184.75) --
	( 79.77,184.78) --
	( 79.79,184.82) --
	( 79.79,184.83) --
	( 79.81,184.85) --
	( 79.83,184.87) --
	( 79.85,184.87) --
	( 79.89,184.90) --
	( 79.94,184.92) --
	( 79.99,184.94) --
	( 80.01,184.95) --
	( 80.04,184.96) --
	( 80.06,184.97) --
	( 80.09,184.98) --
	( 80.11,184.99) --
	( 80.15,185.01) --
	( 80.19,185.03) --
	( 80.22,185.04) --
	( 80.24,185.05) --
	( 80.25,185.06) --
	( 80.32,185.08) --
	( 80.35,185.09) --
	( 80.37,185.10) --
	( 80.39,185.11) --
	( 80.40,185.12) --
	( 80.41,185.14) --
	( 80.43,185.18) --
	( 80.46,185.24) --
	( 80.48,185.27) --
	( 80.49,185.28) --
	( 80.50,185.28) --
	( 80.51,185.30) --
	( 80.52,185.30) --
	( 80.57,185.32) --
	( 80.59,185.33) --
	( 80.62,185.36) --
	( 80.65,185.39) --
	( 80.67,185.41) --
	( 80.68,185.44) --
	( 80.69,185.47) --
	( 80.69,185.49) --
	( 80.69,185.53) --
	( 80.68,185.56) --
	( 80.66,185.61) --
	( 80.66,185.63) --
	( 80.62,185.66) --
	( 80.60,185.67) --
	( 80.58,185.68) --
	( 80.57,185.69) --
	( 80.55,185.72) --
	( 80.54,185.73) --
	( 80.53,185.75) --
	( 80.53,185.77) --
	( 80.53,185.78) --
	( 80.53,185.80) --
	( 80.53,185.81) --
	( 80.55,185.84) --
	( 80.56,185.85) --
	( 80.56,185.86) --
	( 80.56,185.88) --
	( 80.56,185.90) --
	( 80.57,185.94) --
	( 80.57,185.96) --
	( 80.58,185.98) --
	( 80.59,186.00) --
	( 80.61,186.05) --
	( 80.64,186.09) --
	( 80.64,186.10) --
	( 80.65,186.13) --
	( 80.66,186.19) --
	( 80.67,186.21) --
	( 80.68,186.23) --
	( 80.70,186.26) --
	( 80.71,186.27) --
	( 80.74,186.31) --
	( 80.77,186.34) --
	( 80.79,186.41) --
	( 80.80,186.42) --
	( 80.80,186.42) --
	( 80.81,186.42) --
	( 80.81,186.42) --
	( 80.82,186.42) --
	( 80.82,186.42) --
	( 80.83,186.42) --
	( 80.83,186.42) --
	( 80.84,186.42) --
	( 80.84,186.42) --
	( 80.85,186.42) --
	( 80.85,186.42) --
	( 80.86,186.42) --
	( 80.86,186.42) --
	( 80.88,186.45) --
	( 80.90,186.47) --
	( 80.96,186.51) --
	( 81.02,186.54) --
	( 81.07,186.56) --
	( 81.08,186.58) --
	( 81.09,186.60) --
	( 81.11,186.63) --
	( 81.11,186.66) --
	( 81.12,186.67) --
	( 81.13,186.69) --
	( 81.13,186.70) --
	( 81.12,186.71) --
	( 81.11,186.72) --
	( 81.10,186.75) --
	( 81.09,186.77) --
	( 81.09,186.78) --
	( 81.09,186.79) --
	( 81.10,186.82) --
	( 81.11,186.84) --
	( 81.11,186.85) --
	( 81.14,186.89) --
	( 81.15,186.92) --
	( 81.16,186.95) --
	( 81.15,186.99) --
	( 81.14,187.03) --
	( 81.15,187.06) --
	( 81.15,187.10) --
	( 81.17,187.13) --
	( 81.17,187.15) --
	( 81.18,187.18) --
	( 81.20,187.24) --
	( 81.21,187.30) --
	( 81.21,187.32) --
	( 81.21,187.33) --
	( 81.21,187.34) --
	( 81.21,187.35) --
	( 81.19,187.37) --
	( 81.15,187.43) --
	( 81.15,187.45) --
	( 81.14,187.47) --
	( 81.14,187.50) --
	( 81.15,187.56) --
	( 81.16,187.59) --
	( 81.16,187.60) --
	( 81.15,187.61) --
	( 81.13,187.65) --
	( 81.10,187.69) --
	( 81.08,187.72) --
	( 81.03,187.79) --
	( 81.01,187.81) --
	( 81.00,187.83) --
	( 80.99,187.86) --
	( 80.96,187.90) --
	( 80.95,187.92) --
	( 80.95,187.95) --
	( 80.95,187.96) --
	( 80.96,187.97) --
	( 80.97,187.99) --
	( 80.99,188.02) --
	( 81.00,188.03) --
	( 81.01,188.06) --
	( 81.01,188.07) --
	( 81.01,188.09) --
	( 81.00,188.13) --
	( 80.99,188.14) --
	( 80.98,188.15) --
	( 80.96,188.16) --
	( 80.94,188.17) --
	( 80.86,188.20) --
	( 80.84,188.21) --
	( 80.82,188.22) --
	( 80.81,188.22) --
	( 80.81,188.23) --
	( 80.81,188.23) --
	( 80.80,188.25) --
	( 80.80,188.27) --
	( 80.79,188.30) --
	( 80.78,188.34) --
	( 80.77,188.36) --
	( 80.77,188.38) --
	( 80.78,188.42) --
	( 80.79,188.44) --
	( 80.81,188.46) --
	( 80.83,188.48) --
	( 80.85,188.52) --
	( 80.86,188.54) --
	( 80.88,188.57) --
	( 80.90,188.60) --
	( 80.93,188.66) --
	( 80.94,188.68) --
	( 80.95,188.74) --
	( 80.96,188.75) --
	( 80.97,188.77) --
	( 81.00,188.83) --
	( 81.00,188.84) --
	( 80.99,188.87) --
	( 80.96,188.92) --
	( 80.94,188.97) --
	( 80.92,189.00) --
	( 80.91,189.02) --
	( 80.89,189.03) --
	( 80.87,189.06) --
	( 80.86,189.08) --
	( 80.86,189.10) --
	( 80.85,189.11) --
	( 80.85,189.14) --
	( 80.85,189.16) --
	( 80.87,189.22) --
	( 80.89,189.25) --
	( 80.89,189.27) --
	( 80.90,189.29) --
	( 80.90,189.30) --
	( 80.90,189.32) --
	( 80.89,189.33) --
	( 80.88,189.36) --
	( 80.86,189.38) --
	( 80.81,189.44) --
	( 80.79,189.46) --
	( 80.75,189.49) --
	( 80.71,189.51) --
	( 80.68,189.52) --
	( 80.64,189.54) --
	( 80.60,189.56) --
	( 80.57,189.57) --
	( 80.52,189.60) --
	( 80.50,189.62) --
	( 80.50,189.63) --
	( 80.49,189.64) --
	( 80.48,189.65) --
	( 80.47,189.68) --
	( 80.47,189.70) --
	( 80.46,189.74) --
	( 80.45,189.77) --
	( 80.44,189.81) --
	( 80.43,189.89) --
	( 80.42,189.91) --
	( 80.42,189.93) --
	( 80.42,189.95) --
	( 80.42,189.97) --
	( 80.43,189.99) --
	( 80.43,190.01) --
	( 80.44,190.05) --
	( 80.44,190.09) --
	( 80.44,190.11) --
	( 80.44,190.15) --
	( 80.44,190.17) --
	( 80.44,190.19) --
	( 80.44,190.23) --
	( 80.45,190.25) --
	( 80.46,190.31) --
	( 80.47,190.37) --
	( 80.47,190.40) --
	( 80.47,190.43) --
	( 80.47,190.46) --
	( 80.48,190.51) --
	( 80.50,190.55) --
	( 80.51,190.57) --
	( 80.55,190.62) --
	( 80.57,190.64) --
	( 80.61,190.65) --
	( 80.62,190.66) --
	( 80.70,190.67) --
	( 80.73,190.68) --
	( 80.76,190.69) --
	( 80.78,190.70) --
	( 80.79,190.72) --
	( 80.81,190.74) --
	( 80.83,190.78) --
	( 80.85,190.81) --
	( 80.84,190.90) --
	( 80.84,190.95) --
	( 80.83,191.02) --
	( 80.83,191.03) --
	( 80.83,191.06) --
	( 80.81,191.09) --
	( 80.76,191.13) --
	( 80.72,191.16) --
	( 80.69,191.18) --
	( 80.63,191.20) --
	( 80.57,191.23) --
	( 80.54,191.24) --
	( 80.50,191.27) --
	( 80.47,191.31) --
	( 80.46,191.36) --
	( 80.44,191.42) --
	( 80.43,191.45) --
	( 80.43,191.47) --
	( 80.44,191.49) --
	( 80.45,191.51) --
	( 80.45,191.53) --
	( 80.49,191.57) --
	( 80.52,191.62) --
	( 80.53,191.64) --
	( 80.55,191.66) --
	( 80.56,191.68) --
	( 80.58,191.71) --
	( 80.58,191.73) --
	( 80.58,191.74) --
	( 80.59,191.75) --
	( 80.58,191.77) --
	( 80.54,191.80) --
	( 80.51,191.82) --
	( 80.48,191.84) --
	( 80.44,191.87) --
	( 80.41,191.89) --
	( 80.39,191.91) --
	( 80.35,191.95) --
	( 80.33,191.98) --
	( 80.32,192.00) --
	( 80.31,192.02) --
	( 80.30,192.09) --
	( 80.30,192.17) --
	( 80.30,192.28) --
	( 80.31,192.31) --
	( 80.31,192.37) --
	( 80.31,192.40) --
	( 80.32,192.44) --
	( 80.33,192.46) --
	( 80.33,192.47) --
	( 80.38,192.50) --
	( 80.42,192.51) --
	( 80.52,192.54) --
	( 80.54,192.55) --
	( 80.56,192.56) --
	( 80.57,192.57) --
	( 80.59,192.58) --
	( 80.60,192.60) --
	( 80.60,192.64) --
	( 80.58,192.68) --
	( 80.55,192.70) --
	( 80.53,192.75) --
	( 80.52,192.76) --
	( 80.53,192.85) --
	( 80.53,192.88) --
	( 80.52,192.90) --
	( 80.50,192.94) --
	( 80.47,192.98) --
	( 80.46,192.99) --
	( 80.43,193.01) --
	( 80.40,193.03) --
	( 80.31,193.08) --
	( 80.29,193.10) --
	( 80.25,193.11) --
	( 80.22,193.13) --
	( 80.20,193.15) --
	( 80.16,193.16) --
	( 80.07,193.20) --
	( 80.03,193.22) --
	( 80.01,193.23) --
	( 79.93,193.26) --
	( 79.88,193.29) --
	( 79.86,193.30) --
	( 79.82,193.33) --
	( 79.77,193.37) --
	( 79.75,193.39) --
	( 79.74,193.41) --
	( 79.72,193.43) --
	( 79.71,193.46) --
	( 79.71,193.49) --
	( 79.70,193.50) --
	( 79.70,193.51) --
	( 79.70,193.54) --
	( 79.70,193.55) --
	( 79.71,193.56) --
	( 79.72,193.57) --
	( 79.73,193.59) --
	( 79.74,193.61) --
	( 79.74,193.63) --
	( 79.74,193.64) --
	( 79.73,193.66) --
	( 79.71,193.70) --
	( 79.70,193.71) --
	( 79.69,193.76) --
	( 79.69,193.76) --
	( 79.69,193.80) --
	( 79.73,193.87) --
	( 79.73,193.88) --
	( 79.73,193.89) --
	( 79.72,193.91) --
	( 79.70,193.96) --
	( 79.70,193.98) --
	( 79.70,193.99) --
	( 79.70,194.01) --
	( 79.70,194.03) --
	( 79.70,194.04) --
	( 79.72,194.07) --
	( 79.74,194.09) --
	( 79.76,194.13) --
	( 79.77,194.14) --
	( 79.77,194.15) --
	( 79.77,194.22) --
	( 79.78,194.24) --
	( 79.78,194.26) --
	( 79.79,194.28) --
	( 79.80,194.30) --
	( 79.82,194.34) --
	( 79.81,194.35) --
	( 79.81,194.37) --
	( 79.80,194.42) --
	( 79.79,194.44) --
	( 79.78,194.49) --
	( 79.77,194.53) --
	( 79.77,194.56) --
	( 79.78,194.59) --
	( 79.78,194.60) --
	( 79.80,194.62) --
	( 79.81,194.62) --
	( 79.82,194.63) --
	( 79.85,194.63) --
	( 79.89,194.63) --
	( 79.92,194.63) --
	( 79.94,194.62) --
	( 79.97,194.62) --
	( 80.00,194.63) --
	( 80.02,194.63) --
	( 80.04,194.64) --
	( 80.07,194.65) --
	( 80.11,194.67) --
	( 80.13,194.69) --
	( 80.16,194.70) --
	( 80.20,194.71) --
	( 80.25,194.73) --
	( 80.27,194.74) --
	( 80.31,194.76) --
	( 80.32,194.77) --
	( 80.32,194.78) --
	( 80.33,194.80) --
	( 80.33,194.82) --
	( 80.33,194.83) --
	( 80.32,194.86) --
	( 80.31,194.88) --
	( 80.31,194.90) --
	( 80.32,194.94) --
	( 80.32,194.95) --
	( 80.33,194.97) --
	( 80.34,194.98) --
	( 80.36,194.99) --
	( 80.37,195.00) --
	( 80.41,195.01) --
	( 80.44,195.01) --
	( 80.47,195.01) --
	( 80.49,195.02) --
	( 80.51,195.03) --
	( 80.53,195.05) --
	( 80.54,195.07) --
	( 80.54,195.08) --
	( 80.55,195.09) --
	( 80.55,195.12) --
	( 80.54,195.15) --
	( 80.54,195.17) --
	( 80.54,195.18) --
	( 80.55,195.20) --
	( 80.56,195.22) --
	( 80.57,195.24) --
	( 80.58,195.25) --
	( 80.59,195.25) --
	( 80.62,195.26) --
	( 80.65,195.27) --
	( 80.68,195.27) --
	( 80.70,195.27) --
	( 80.73,195.27) --
	( 80.75,195.27) --
	( 80.77,195.26) --
	( 80.80,195.25) --
	( 80.81,195.25) --
	( 80.89,195.21) --
	( 80.91,195.19) --
	( 80.94,195.18) --
	( 80.96,195.17) --
	( 80.99,195.17) --
	( 81.07,195.16) --
	( 81.09,195.15) --
	( 81.11,195.14) --
	( 81.14,195.12) --
	( 81.16,195.10) --
	( 81.17,195.08) --
	( 81.20,195.06) --
	( 81.21,195.05) --
	( 81.24,195.03) --
	( 81.29,194.98) --
	( 81.31,194.96) --
	( 81.31,194.91) --
	( 81.33,194.86) --
	( 81.34,194.82) --
	( 81.35,194.80) --
	( 81.38,194.75) --
	( 81.40,194.74) --
	( 81.45,194.71) --
	( 81.47,194.70) --
	( 81.50,194.69) --
	( 81.54,194.68) --
	( 81.58,194.67) --
	( 81.71,194.61) --
	( 81.74,194.60) --
	( 81.80,194.58) --
	( 81.85,194.56) --
	( 81.88,194.55) --
	( 81.93,194.54) --
	( 81.97,194.53) --
	( 82.11,194.48) --
	( 82.14,194.47) --
	( 82.16,194.47) --
	( 82.19,194.47) --
	( 82.25,194.46) --
	( 82.33,194.47) --
	( 82.38,194.47) --
	( 82.42,194.47) --
	( 82.44,194.48) --
	( 82.47,194.49) --
	( 82.51,194.50) --
	( 82.54,194.50) --
	( 82.57,194.51) --
	( 82.61,194.51) --
	( 82.64,194.50) --
	( 82.66,194.49) --
	( 82.70,194.48) --
	( 82.73,194.46) --
	( 82.75,194.45) --
	( 82.75,194.44) --
	( 82.76,194.43) --
	( 82.77,194.39) --
	( 82.78,194.38) --
	( 82.79,194.37) --
	( 82.84,194.35) --
	( 82.88,194.34) --
	( 82.89,194.34) --
	( 82.94,194.34) --
	( 82.98,194.34) --
	( 82.99,194.33) --
	( 83.00,194.31) --
	( 83.02,194.27) --
	( 83.02,194.25) --
	( 83.05,194.15) --
	( 83.07,194.13) --
	( 83.10,194.10) --
	( 83.15,194.07) --
	( 83.18,194.06) --
	( 83.22,194.04) --
	( 83.24,194.03) --
	( 83.28,194.02) --
	( 83.31,194.02) --
	( 83.34,194.02) --
	( 83.38,194.01) --
	( 83.41,194.00) --
	( 83.45,193.98) --
	( 83.48,193.97) --
	( 83.56,193.94) --
	( 83.61,193.92) --
	( 83.67,193.88) --
	( 83.70,193.86) --
	( 83.73,193.83) --
	( 83.76,193.81) --
	( 83.88,193.76) --
	( 83.93,193.75) --
	( 83.95,193.75) --
	( 83.97,193.74) --
	( 83.99,193.72) --
	( 84.00,193.71) --
	( 84.02,193.69) --
	( 84.02,193.68) --
	( 84.03,193.68) --
	( 84.05,193.68) --
	( 84.07,193.69) --
	( 84.09,193.71) --
	( 84.11,193.72) --
	( 84.12,193.73) --
	( 84.13,193.75) --
	( 84.14,193.77) --
	( 84.17,193.83) --
	( 84.18,193.89) --
	( 84.18,193.91) --
	( 84.19,193.94) --
	( 84.19,193.95) --
	( 84.20,193.97) --
	( 84.21,193.99) --
	( 84.23,194.01) --
	( 84.24,194.02) --
	( 84.26,194.03) --
	( 84.27,194.03) --
	( 84.31,194.03) --
	( 84.34,194.03) --
	( 84.37,194.03) --
	( 84.39,194.03) --
	( 84.41,194.03) --
	( 84.45,194.03) --
	( 84.46,194.03) --
	( 84.57,194.04) --
	( 84.61,194.04) --
	( 84.62,194.04) --
	( 84.66,194.06) --
	( 84.67,194.07) --
	( 84.69,194.09) --
	( 84.71,194.12) --
	( 84.71,194.14) --
	( 84.72,194.16) --
	( 84.72,194.18) --
	( 84.72,194.26) --
	( 84.72,194.29) --
	( 84.70,194.34) --
	( 84.68,194.36) --
	( 84.66,194.38) --
	( 84.65,194.40) --
	( 84.63,194.45) --
	( 84.61,194.50) --
	( 84.60,194.55) --
	( 84.60,194.59) --
	( 84.60,194.60) --
	( 84.60,194.64) --
	( 84.62,194.68) --
	( 84.62,194.69) --
	( 84.62,194.73) --
	( 84.62,194.78) --
	( 84.64,194.84) --
	( 84.67,194.88) --
	( 84.68,194.91) --
	( 84.70,194.94) --
	( 84.70,194.96) --
	( 84.70,194.99) --
	( 84.70,195.02) --
	( 84.69,195.05) --
	( 84.69,195.08) --
	( 84.69,195.10) --
	( 84.70,195.13) --
	( 84.71,195.15) --
	( 84.74,195.20) --
	( 84.81,195.27) --
	( 84.82,195.30) --
	( 84.85,195.36) --
	( 84.87,195.40) --
	( 84.88,195.42) --
	( 84.88,195.44) --
	( 84.88,195.47) --
	( 84.88,195.53) --
	( 84.86,195.62) --
	( 84.85,195.65) --
	( 84.85,195.68) --
	( 84.84,195.75) --
	( 84.83,195.78) --
	( 84.82,195.86) --
	( 84.82,195.90) --
	( 84.82,195.94) --
	( 84.83,195.95) --
	( 84.84,195.97) --
	( 84.89,196.03) --
	( 84.91,196.05) --
	( 84.95,196.08) --
	( 84.97,196.10) --
	( 84.99,196.11) --
	( 85.00,196.11) --
	( 85.02,196.11) --
	( 85.05,196.11) --
	( 85.15,196.10) --
	( 85.20,196.09) --
	( 85.24,196.09) --
	( 85.28,196.09) --
	( 85.32,196.09) --
	( 85.39,196.09) --
	( 85.42,196.09) --
	( 85.51,196.10) --
	( 85.55,196.11) --
	( 85.60,196.12) --
	( 85.62,196.13) --
	( 85.64,196.14) --
	( 85.64,196.15) --
	( 85.65,196.16) --
	( 85.70,196.20) --
	( 85.73,196.23) --
	( 85.75,196.25) --
	( 85.76,196.27) --
	( 85.77,196.30) --
	( 85.78,196.32) --
	( 85.79,196.34) --
	( 85.81,196.37) --
	( 85.83,196.38) --
	( 85.93,196.44) --
	( 86.00,196.48) --
	( 86.05,196.51) --
	( 86.16,196.57) --
	( 86.21,196.61) --
	( 86.26,196.66) --
	( 86.37,196.76) --
	( 86.43,196.81) --
	( 86.53,196.88) --
	( 86.57,196.91) --
	( 86.62,196.95) --
	( 86.64,196.98) --
	( 86.67,197.00) --
	( 86.70,197.05) --
	( 86.73,197.10) --
	( 86.74,197.12) --
	( 86.74,197.14) --
	( 86.74,197.16) --
	( 86.74,197.17) --
	( 86.73,197.18) --
	( 86.71,197.19) --
	( 86.69,197.20) --
	( 86.61,197.22) --
	( 86.55,197.24) --
	( 86.47,197.27) --
	( 86.37,197.30) --
	( 86.28,197.32) --
	( 86.22,197.34) --
	( 86.18,197.35) --
	( 86.13,197.37) --
	( 86.09,197.37) --
	( 86.04,197.38) --
	( 85.99,197.39) --
	( 85.97,197.39) --
	( 85.94,197.38) --
	( 85.88,197.36) --
	( 85.84,197.35) --
	( 85.80,197.35) --
	( 85.76,197.35) --
	( 85.73,197.35) --
	( 85.71,197.36) --
	( 85.68,197.36) --
	( 85.66,197.37) --
	( 85.63,197.39) --
	( 85.62,197.40) --
	( 85.59,197.41) --
	( 85.57,197.41) --
	( 85.54,197.41) --
	( 85.50,197.41) --
	( 85.47,197.42) --
	( 85.45,197.42) --
	( 85.44,197.43) --
	( 85.43,197.43) --
	( 85.39,197.44) --
	( 85.36,197.45) --
	( 85.35,197.46) --
	( 85.35,197.47) --
	( 85.34,197.48) --
	( 85.34,197.49) --
	( 85.35,197.50) --
	( 85.38,197.52) --
	( 85.44,197.57) --
	( 85.46,197.59) --
	( 85.51,197.62) --
	( 85.53,197.64) --
	( 85.55,197.67) --
	( 85.56,197.69) --
	( 85.58,197.71) --
	( 85.59,197.76) --
	( 85.61,197.78) --
	( 85.62,197.80) --
	( 85.62,197.81) --
	( 85.63,197.84) --
	( 85.64,197.90) --
	( 85.64,197.93) --
	( 85.64,197.95) --
	( 85.63,197.97) --
	( 85.62,197.98) --
	( 85.60,197.99) --
	( 85.58,198.00) --
	( 85.56,198.01) --
	( 85.51,198.02) --
	( 85.47,198.02) --
	( 85.45,198.02) --
	( 85.44,198.02) --
	( 85.43,198.02) --
	( 85.42,198.02) --
	( 85.40,198.03) --
	( 85.39,198.04) --
	( 85.37,198.05) --
	( 85.36,198.06) --
	( 85.34,198.08) --
	( 85.31,198.09) --
	( 85.29,198.10) --
	( 85.28,198.11) --
	( 85.26,198.14) --
	( 85.26,198.17) --
	( 85.27,198.20) --
	( 85.27,198.23) --
	( 85.26,198.25) --
	( 85.23,198.26) --
	( 85.18,198.30) --
	( 85.16,198.33) --
	( 85.10,198.36) --
	( 85.06,198.41) --
	( 85.04,198.42) --
	( 85.03,198.43) --
	( 85.02,198.43) --
	( 84.98,198.42) --
	( 84.96,198.42) --
	( 84.94,198.41) --
	( 84.91,198.41) --
	( 84.88,198.41) --
	( 84.86,198.42) --
	( 84.81,198.42) --
	( 84.78,198.44) --
	( 84.74,198.47) --
	( 84.69,198.51) --
	( 84.66,198.52) --
	( 84.65,198.53) --
	( 84.60,198.54) --
	( 84.56,198.56) --
	( 84.54,198.57) --
	( 84.50,198.60) --
	( 84.47,198.64) --
	( 84.45,198.67) --
	( 84.45,198.68) --
	( 84.45,198.70) --
	( 84.43,198.75) --
	( 84.42,198.75) --
	( 84.42,198.75) --
	( 84.41,198.75) --
	( 84.41,198.75) --
	( 84.40,198.75) --
	( 84.39,198.75) --
	( 84.40,198.79) --
	( 84.39,198.88) --
	( 84.37,198.93) --
	( 84.33,199.00) --
	( 84.29,199.04) --
	( 84.27,199.07) --
	( 84.24,199.10) --
	( 84.22,199.12) --
	( 84.19,199.14) --
	( 84.15,199.17) --
	( 84.02,199.27) --
	( 83.95,199.33) --
	( 83.92,199.36) --
	( 83.90,199.38) --
	( 83.89,199.41) --
	( 83.88,199.44) --
	( 83.86,199.46) --
	( 83.70,199.58) --
	( 83.66,199.62) --
	( 83.65,199.65) --
	( 83.64,199.75) --
	( 83.61,199.79) --
	( 83.58,199.90) --
	( 83.58,199.94) --
	( 83.55,199.99) --
	( 83.49,199.97) --
	( 83.45,199.95) --
	( 83.39,199.92) --
	( 83.26,199.81) --
	( 83.22,199.79) --
	( 83.19,199.77) --
	( 83.16,199.75) --
	( 83.11,199.72) --
	( 83.07,199.68) --
	( 83.05,199.67) --
	( 82.98,199.63) --
	( 82.95,199.62) --
	( 82.89,199.61) --
	( 82.84,199.62) --
	( 82.80,199.62) --
	( 82.73,199.66) --
	( 82.71,199.67) --
	( 82.69,199.69) --
	( 82.69,199.71) --
	( 82.68,199.76) --
	( 82.68,199.77) --
	( 82.66,199.80) --
	( 82.65,199.82) --
	( 82.62,199.83) --
	( 82.58,199.85) --
	( 82.55,199.86) --
	( 82.53,199.88) --
	( 82.52,199.90) --
	( 82.51,199.93) --
	( 82.50,199.96) --
	( 82.51,200.02) --
	( 82.50,200.04) --
	( 82.47,200.07) --
	( 82.46,200.15) --
	( 82.46,200.17) --
	( 82.48,200.18) --
	( 82.55,200.27) --
	( 82.55,200.28) --
	( 82.56,200.30) --
	( 82.62,200.37) --
	( 82.66,200.40) --
	( 82.68,200.44) --
	( 82.68,200.46) --
	( 82.68,200.49) --
	( 82.67,200.51) --
	( 82.65,200.53) --
	( 82.64,200.58) --
	( 82.64,200.61) --
	( 82.67,200.66) --
	( 82.69,200.72) --
	( 82.68,200.78) --
	( 82.67,200.80) --
	( 82.65,200.82) --
	( 82.63,200.85) --
	( 82.62,200.88) --
	( 82.59,200.93) --
	( 82.59,200.96) --
	( 82.60,200.99) --
	( 82.62,201.03) --
	( 82.65,201.06) --
	( 82.70,201.12) --
	( 82.75,201.16) --
	( 82.76,201.18) --
	( 82.77,201.20) --
	( 82.77,201.23) --
	( 82.76,201.26) --
	( 82.72,201.32) --
	( 82.71,201.33) --
	( 82.71,201.35) --
	( 82.73,201.37) --
	( 82.74,201.38) --
	( 82.87,201.41) --
	( 82.90,201.42) --
	( 82.92,201.43) --
	( 82.93,201.45) --
	( 82.98,201.64) --
	( 83.01,201.73) --
	( 83.01,201.76) --
	( 83.08,201.83) --
	( 83.08,201.85) --
	( 83.08,201.88) --
	( 83.07,201.91) --
	( 83.07,201.93) --
	( 83.08,202.02) --
	( 83.07,202.06) --
	( 83.06,202.08) --
	( 83.05,202.11) --
	( 83.03,202.22) --
	( 83.01,202.29) --
	( 83.01,202.32) --
	( 83.00,202.34) --
	( 82.97,202.37) --
	( 82.93,202.44) --
	( 82.91,202.47) --
	( 82.90,202.50) --
	( 82.90,202.53) --
	( 82.94,202.62) --
	( 82.95,202.66) --
	( 82.96,202.68) --
	( 82.97,202.83) --
	( 82.97,202.89) --
	( 82.97,202.93) --
	( 82.96,202.98) --
	( 82.91,203.03) --
	( 82.88,203.08) --
	( 82.77,203.18) --
	( 82.76,203.19) --
	( 82.75,203.21) --
	( 82.68,203.30) --
	( 82.65,203.33) --
	( 82.56,203.40) --
	( 82.48,203.49) --
	( 82.46,203.51) --
	( 82.40,203.55) --
	( 82.36,203.57) --
	( 82.32,203.58) --
	( 82.29,203.61) --
	( 82.25,203.66) --
	( 82.22,203.72) --
	( 82.19,203.76) --
	( 82.18,203.79) --
	( 82.17,203.80) --
	( 82.15,203.82) --
	( 82.10,203.85) --
	( 82.09,203.86) --
	( 82.07,203.91) --
	( 82.03,203.96) --
	( 81.98,204.02) --
	( 81.96,204.05) --
	( 81.93,204.06) --
	( 81.91,204.09) --
	( 81.84,204.17) --
	( 81.83,204.21) --
	( 81.84,204.33) --
	( 81.86,204.38) --
	( 81.87,204.40) --
	( 81.87,204.43) --
	( 81.86,204.55) --
	( 81.86,204.57) --
	( 81.90,204.60) --
	( 81.92,204.64) --
	( 81.94,204.65) --
	( 81.97,204.67) --
	( 82.00,204.67) --
	( 82.02,204.69) --
	( 82.03,204.72) --
	( 82.03,204.75) --
	( 82.04,204.76) --
	( 82.03,204.78) --
	( 82.02,204.81) --
	( 82.00,204.84) --
	( 81.96,204.88) --
	( 81.94,204.91) --
	( 81.90,204.94) --
	( 81.80,205.03) --
	( 81.76,205.07) --
	( 81.74,205.11) --
	( 81.71,205.18) --
	( 81.69,205.21) --
	( 81.61,205.28) --
	( 81.57,205.32) --
	( 81.51,205.36) --
	( 81.44,205.39) --
	( 81.31,205.45) --
	( 81.26,205.46) --
	( 81.05,205.52) --
	( 81.08,205.56) --
	( 81.10,205.59) --
	( 81.21,205.76) --
	( 81.25,205.79) --
	( 81.27,205.81) --
	( 81.29,205.82) --
	( 81.43,205.89) --
	( 81.47,205.92) --
	( 81.63,206.03) --
	( 81.66,206.05) --
	( 81.70,206.07) --
	( 81.91,206.14) --
	( 81.97,206.16) --
	( 82.00,206.17) --
	( 82.04,206.18) --
	( 82.09,206.20) --
	( 82.14,206.22) --
	( 82.19,206.24) --
	( 82.23,206.26) --
	( 82.27,206.29) --
	( 82.28,206.30) --
	( 82.31,206.33) --
	( 82.33,206.36) --
	( 82.35,206.40) --
	( 82.34,206.42) --
	( 82.34,206.43) --
	( 82.32,206.47) --
	( 82.32,206.50) --
	( 82.32,206.51) --
	( 82.32,206.54) --
	( 82.32,206.58) --
	( 82.30,206.68) --
	( 82.30,206.71) --
	( 82.30,206.75) --
	( 82.29,206.78) --
	( 82.28,206.89) --
	( 82.28,206.91) --
	( 82.27,206.93) --
	( 82.28,206.95) --
	( 82.28,206.98) --
	( 82.28,207.00) --
	( 82.27,207.07) --
	( 82.27,207.11) --
	( 82.27,207.17) --
	( 82.25,207.23) --
	( 82.25,207.26) --
	( 82.26,207.31) --
	( 82.27,207.33) --
	( 82.30,207.39) --
	( 82.36,207.47) --
	( 82.40,207.53) --
	( 82.46,207.59) --
	( 82.50,207.64) --
	( 82.51,207.66) --
	( 82.51,207.68) --
	( 82.51,207.72) --
	( 82.50,207.75) --
	( 82.50,207.78) --
	( 82.49,207.88) --
	( 82.51,207.92) --
	( 82.52,207.97) --
	( 82.55,208.04) --
	( 82.59,208.09) --
	( 82.66,208.20) --
	( 82.72,208.28) --
	( 82.80,208.41) --
	( 82.82,208.46) --
	( 82.85,208.50) --
	( 82.87,208.51) --
	( 82.93,208.57) --
	( 82.97,208.61) --
	( 83.04,208.66) --
	( 83.09,208.69) --
	( 83.14,208.72) --
	( 83.20,208.76) --
	( 83.25,208.78) --
	( 83.45,208.83) --
	( 83.50,208.83) --
	( 83.55,208.83) --
	( 83.57,208.83) --
	( 83.60,208.84) --
	( 83.69,208.85) --
	( 83.74,208.86) --
	( 83.84,208.87) --
	( 83.88,208.87) --
	( 83.96,208.89) --
	( 84.07,208.94) --
	( 84.09,208.96) --
	( 84.11,208.98) --
	( 84.12,209.00) --
	( 84.11,209.02) --
	( 84.10,209.05) --
	( 84.09,209.12) --
	( 84.07,209.16) --
	( 84.06,209.25) --
	( 84.06,209.27) --
	( 84.07,209.32) --
	( 84.11,209.38) --
	( 84.18,209.51) --
	( 84.22,209.57) --
	( 84.32,209.66) --
	( 84.35,209.69) --
	( 84.37,209.71) --
	( 84.44,209.75) --
	( 84.47,209.77) --
	( 84.53,209.82) --
	( 84.58,209.84) --
	( 84.63,209.85) --
	( 84.77,209.89) --
	( 84.87,209.91) --
	( 84.94,209.94) --
	( 85.08,209.97) --
	( 85.20,209.99) --
	( 85.29,209.99) --
	( 85.44,210.02) --
	( 85.68,210.02) --
	( 85.98,210.08) --
	( 86.03,210.09) --
	( 86.14,210.13) --
	( 86.27,210.20) --
	( 86.31,210.22) --
	( 86.40,210.24) --
	( 86.45,210.24) --
	( 86.51,210.24) --
	( 86.54,210.24) --
	( 86.58,210.25) --
	( 86.61,210.25) --
	( 86.65,210.24) --
	( 86.69,210.25) --
	( 86.72,210.26) --
	( 86.83,210.26) --
	( 86.92,210.27) --
	( 87.05,210.28) --
	( 87.08,210.29) --
	( 87.13,210.30) --
	( 87.17,210.31) --
	( 87.19,210.32) --
	( 87.21,210.33) --
	( 87.25,210.34) --
	( 87.28,210.36) --
	( 87.32,210.36) --
	( 87.37,210.38) --
	( 87.45,210.40) --
	( 87.49,210.43) --
	( 87.54,210.47) --
	( 87.60,210.50) --
	( 87.66,210.53) --
	( 87.68,210.53) --
	( 87.74,210.56) --
	( 87.80,210.60) --
	( 87.88,210.68) --
	( 87.97,210.76) --
	( 88.02,210.79) --
	( 88.07,210.86) --
	( 88.09,210.88) --
	( 88.12,210.90) --
	( 88.16,210.91) --
	( 88.26,210.95) --
	( 88.49,211.04) --
	( 88.57,211.07) --
	( 88.62,211.10) --
	( 88.69,211.14) --
	( 88.76,211.21) --
	( 88.89,211.27) --
	( 88.99,211.34) --
	( 89.05,211.40) --
	( 89.09,211.42) --
	( 89.13,211.45) --
	( 89.37,211.58) --
	( 89.43,211.60) --
	( 89.58,211.62) --
	( 89.63,211.62) --
	( 89.67,211.62) --
	( 89.77,211.64) --
	( 89.90,211.66) --
	( 89.99,211.68) --
	( 90.03,211.70) --
	( 90.07,211.73) --
	( 90.10,211.77) --
	( 90.14,211.84) --
	( 90.15,211.86) --
	( 90.15,211.88) --
	( 90.15,211.92) --
	( 90.14,211.95) --
	( 90.12,212.00) --
	( 90.09,212.02) --
	( 90.07,212.03) --
	( 90.03,212.06) --
	( 89.90,212.12) --
	( 89.77,212.18) --
	( 89.54,212.35) --
	( 89.49,212.39) --
	( 89.46,212.41) --
	( 89.43,212.45) --
	( 89.36,212.50) --
	( 89.28,212.58) --
	( 89.23,212.65) --
	( 89.22,212.70) --
	( 89.22,212.72) --
	( 89.24,212.74) --
	( 89.25,212.75) --
	( 89.28,212.76) --
	( 89.32,212.77) --
	( 89.40,212.78) --
	( 89.43,212.79) --
	( 89.51,212.80) --
	( 89.54,212.81) --
	( 89.59,212.81) --
	( 89.63,212.81) --
	( 89.72,212.82) --
	( 89.78,212.82) --
	( 89.86,212.81) --
	( 89.98,212.82) --
	( 90.13,212.83) --
	( 90.17,212.83) --
	( 90.25,212.86) --
	( 90.31,212.90) --
	( 90.34,212.93) --
	( 90.36,212.96) --
	( 90.38,213.00) --
	( 90.38,213.06) --
	( 90.38,213.08) --
	( 90.36,213.11) --
	( 90.34,213.14) --
	( 90.30,213.17) --
	( 90.28,213.21) --
	( 90.25,213.27) --
	( 90.26,213.31) --
	( 90.24,213.38) --
	( 90.23,213.40) --
	( 90.22,213.41) --
	( 90.20,213.43) --
	( 90.16,213.48) --
	( 90.11,213.57) --
	( 90.07,213.62) --
	( 90.05,213.68) --
	( 90.03,213.73) --
	( 90.03,213.76) --
	( 90.03,213.85) --
	( 90.05,213.91) --
	( 90.05,213.94) --
	( 90.06,213.96) --
	( 90.07,213.98) --
	( 90.09,214.01) --
	( 90.11,214.06) --
	( 90.12,214.08) --
	( 90.13,214.10) --
	( 90.15,214.14) --
	( 90.16,214.17) --
	( 90.28,214.32) --
	( 90.29,214.35) --
	( 90.38,214.52) --
	( 90.38,214.57) --
	( 90.44,214.86) --
	( 90.45,214.91) --
	( 90.51,214.98) --
	( 90.56,215.05) --
	( 90.60,215.09) --
	( 90.63,215.13) --
	( 90.65,215.17) --
	( 90.68,215.22) --
	( 90.80,215.35) --
	( 90.86,215.40) --
	( 90.91,215.45) --
	( 90.98,215.51) --
	( 90.98,215.52) --
	( 90.98,215.56) --
	( 91.00,215.57) --
	( 91.02,215.58) --
	( 91.03,215.61) --
	( 91.02,215.63) --
	( 91.00,215.68) --
	( 90.99,215.69) --
	( 90.97,215.71) --
	( 90.95,215.73) --
	( 90.94,215.75) --
	( 90.94,215.77) --
	( 90.94,215.80) --
	( 90.94,215.82) --
	( 90.96,215.86) --
	( 90.97,215.88) --
	( 90.95,215.90) --
	( 90.98,215.91) --
	( 91.02,215.93) --
	( 91.03,215.94) --
	( 91.06,215.96) --
	( 91.09,216.00) --
	( 91.16,216.12) --
	( 91.18,216.15) --
	( 91.19,216.18) --
	( 91.22,216.22) --
	( 91.25,216.25) --
	( 91.31,216.28) --
	( 91.52,216.33) --
	( 91.57,216.35) --
	( 91.61,216.38) --
	( 91.67,216.42) --
	( 91.70,216.44) --
	( 91.84,216.48) --
	( 91.86,216.49) --
	( 91.88,216.52) --
	( 91.90,216.53) --
	( 91.92,216.56) --
	( 91.95,216.59) --
	( 91.96,216.59) --
	( 92.02,216.58) --
	( 92.05,216.58) --
	( 92.18,216.58) --
	( 92.22,216.59) --
	( 92.32,216.62) --
	( 92.35,216.63) --
	( 92.39,216.64) --
	( 92.42,216.64) --
	( 92.45,216.63) --
	( 92.47,216.62) --
	( 92.51,216.59) --
	( 92.55,216.57) --
	( 92.58,216.55) --
	( 92.62,216.52) --
	( 92.67,216.49) --
	( 92.78,216.43) --
	( 92.79,216.43) --
	( 92.88,216.35) --
	( 92.89,216.33) --
	( 92.91,216.28) --
	( 92.92,216.27) --
	( 92.95,216.25) --
	( 92.99,216.25) --
	( 93.03,216.25) --
	( 93.09,216.25) --
	( 93.14,216.25) --
	( 93.21,216.27) --
	( 93.29,216.29) --
	( 93.34,216.32) --
	( 93.49,216.36) --
	( 93.52,216.38) --
	( 93.58,216.42) --
	( 93.64,216.46) --
	( 93.66,216.49) --
	( 93.70,216.51) --
	( 93.75,216.53) --
	( 93.81,216.58) --
	( 93.87,216.64) --
	( 93.92,216.69) --
	( 93.95,216.73) --
	( 93.99,216.76) --
	( 94.14,216.84) --
	( 94.21,216.86) --
	( 94.28,216.90) --
	( 94.30,216.92) --
	( 94.34,216.95) --
	( 94.36,216.96) --
	( 94.40,216.97) --
	( 94.45,217.00) --
	( 94.53,217.06) --
	( 94.61,217.12) --
	( 94.66,217.14) --
	( 94.71,217.17) --
	( 94.76,217.18) --
	( 94.80,217.19) --
	( 94.82,217.19) --
	( 94.96,217.14) --
	( 95.02,217.11) --
	( 95.05,217.11) --
	( 95.07,217.11) --
	( 95.16,217.12) --
	( 95.19,217.12) --
	( 95.27,217.11) --
	( 95.31,217.13) --
	( 95.33,217.14) --
	( 95.42,217.19) --
	( 95.51,217.24) --
	( 95.62,217.30) --
	( 95.67,217.30) --
	( 95.72,217.29) --
	( 95.79,217.27) --
	( 95.84,217.25) --
	( 96.00,217.19) --
	( 96.09,217.14) --
	( 96.14,217.13) --
	( 96.24,217.08) --
	( 96.25,217.18) --
	( 96.28,217.26) --
	( 96.35,217.41) --
	( 96.37,217.45) --
	( 96.37,217.47) --
	( 96.37,217.52) --
	( 96.39,217.61) --
	( 96.40,217.78) --
	( 96.42,217.84) --
	( 96.43,217.92) --
	( 96.42,218.00) --
	( 96.43,218.09) --
	( 96.48,218.21) --
	( 96.46,218.26) --
	( 96.46,218.39) --
	( 96.47,218.45) --
	( 96.46,218.47) --
	( 96.46,218.51) --
	( 96.46,218.57) --
	( 96.45,218.61) --
	( 96.43,218.62) --
	( 96.43,218.64) --
	( 96.40,218.77) --
	( 96.39,218.85) --
	( 96.36,218.92) --
	( 96.33,218.99) --
	( 96.23,219.17) --
	( 96.15,219.30) --
	( 96.11,219.41) --
	( 96.06,219.47) --
	( 96.02,219.53) --
	( 95.97,219.58) --
	( 95.95,219.60) --
	( 95.91,219.65) --
	( 95.83,219.72) --
	( 95.80,219.75) --
	( 95.77,219.78) --
	( 95.72,219.83) --
	( 95.68,219.86) --
	( 95.64,219.93) --
	( 95.64,219.96) --
	( 95.64,219.98) --
	( 95.66,220.00) --
	( 95.68,220.02) --
	( 95.72,220.05) --
	( 95.74,220.07) --
	( 95.90,220.14) --
	( 95.93,220.16) --
	( 95.95,220.18) --
	( 95.99,220.22) --
	( 96.03,220.30) --
	( 96.05,220.31) --
	( 96.07,220.34) --
	( 96.08,220.35) --
	( 96.08,220.39) --
	( 96.07,220.42) --
	( 96.06,220.46) --
	( 96.05,220.50) --
	( 96.05,220.52) --
	( 96.06,220.58) --
	( 96.11,220.73) --
	( 96.14,220.79) --
	( 96.15,220.85) --
	( 96.18,220.96) --
	( 96.18,220.98) --
	( 96.19,221.04) --
	( 96.21,221.08) --
	( 96.21,221.18) --
	( 96.21,221.20) --
	( 96.20,221.22) --
	( 96.19,221.26) --
	( 96.19,221.36) --
	( 96.20,221.38) --
	( 96.20,221.42) --
	( 96.19,221.45) --
	( 96.19,221.49) --
	( 96.21,221.58) --
	( 96.27,221.68) --
	( 96.31,221.72) --
	( 96.43,221.81) --
	( 96.46,221.84) --
	( 96.53,221.90) --
	( 96.61,221.94) --
	( 96.66,221.97) --
	( 96.73,222.02) --
	( 96.75,222.06) --
	( 96.77,222.11) --
	( 96.74,222.18) --
	( 96.72,222.21) --
	( 96.65,222.32) --
	( 96.62,222.35) --
	( 96.53,222.43) --
	( 96.50,222.46) --
	( 96.47,222.49) --
	( 96.45,222.50) --
	( 96.18,222.67) --
	( 96.12,222.72) --
	( 96.08,222.75) --
	( 96.07,222.77) --
	( 96.02,222.81) --
	( 95.94,222.90) --
	( 95.91,222.96) --
	( 95.86,223.05) --
	( 95.82,223.10) --
	( 95.80,223.13) --
	( 95.60,223.29) --
	( 95.57,223.33) --
	( 95.54,223.42) --
	( 95.55,223.42) --
	( 95.55,223.42) --
	( 95.56,223.42) --
	( 95.56,223.42) --
	( 95.57,223.42) --
	( 95.57,223.42) --
	( 95.58,223.42) --
	( 95.58,223.42) --
	( 95.59,223.42) --
	( 95.59,223.42) --
	( 95.60,223.42) --
	( 95.60,223.53) --
	( 95.60,223.62) --
	( 95.59,223.80) --
	( 95.54,224.06) --
	( 95.52,224.15) --
	( 95.49,224.21) --
	( 95.47,224.29) --
	( 95.43,224.36) --
	( 95.39,224.42) --
	( 95.36,224.47) --
	( 95.31,224.56) --
	( 95.26,224.62) --
	( 95.20,224.68) --
	( 95.15,224.76) --
	( 95.11,224.81) --
	( 95.09,224.84) --
	( 95.05,224.92) --
	( 95.02,225.01) --
	( 95.01,225.09) --
	( 95.03,225.26) --
	( 95.05,225.35) --
	( 95.07,225.40) --
	( 95.11,225.51) --
	( 95.13,225.55) --
	( 95.15,225.58) --
	( 95.22,225.70) --
	( 95.25,225.74) --
	( 95.28,225.77) --
	( 95.33,225.81) --
	( 95.45,225.90) --
	( 95.59,226.00) --
	( 95.62,226.06) --
	( 95.69,226.13) --
	( 95.72,226.21) --
	( 95.77,226.29) --
	( 95.79,226.34) --
	( 95.81,226.43) --
	( 95.81,226.48) --
	( 95.80,226.53) --
	( 95.78,226.59) --
	( 95.73,226.70) --
	( 95.66,226.81) --
	( 95.64,226.86) --
	( 95.70,227.03) --
	( 95.74,227.10) --
	( 95.78,227.16) --
	( 95.81,227.21) --
	( 95.82,227.25) --
	( 95.82,227.29) --
	( 95.81,227.33) --
	( 95.79,227.39) --
	( 95.73,227.48) --
	( 95.68,227.53) --
	( 95.62,227.57) --
	( 95.44,227.76) --
	( 95.41,227.83) --
	( 95.39,227.87) --
	( 95.37,227.95) --
	( 95.36,228.04) --
	( 95.33,228.10) --
	( 95.30,228.14) --
	( 95.27,228.16) --
	( 95.24,228.17) --
	( 95.21,228.17) --
	( 95.17,228.17) --
	( 95.11,228.16) --
	( 95.05,228.15) --
	( 94.96,228.12) --
	( 94.83,228.08) --
	( 94.68,228.05) --
	( 94.50,228.01) --
	( 94.41,228.00) --
	( 94.34,228.01) --
	( 94.28,228.03) --
	( 94.16,228.12) --
	( 94.08,228.21) --
	( 94.03,228.29) --
	( 94.00,228.31) --
	( 93.95,228.33) --
	( 93.87,228.39) --
	( 93.83,228.42) --
	( 93.80,228.47) --
	( 93.77,228.53) --
	( 93.73,228.64) --
	( 93.70,228.77) --
	( 93.68,228.84) --
	( 93.68,228.93) --
	( 93.66,229.04) --
	( 93.62,229.17) --
	( 93.61,229.28) --
	( 93.60,229.33) --
	( 93.58,229.41) --
	( 93.57,229.49) --
	( 93.55,229.54) --
	( 93.50,229.64) --
	( 93.46,229.75) --
	( 93.43,229.86) --
	( 93.41,229.99) --
	( 93.37,230.20) --
	( 93.34,230.23) --
	( 93.29,230.26) --
	( 93.23,230.31) --
	( 93.18,230.34) --
	( 93.12,230.35) --
	( 93.00,230.37) --
	( 92.61,230.43) --
	( 92.57,230.44) --
	( 92.52,230.47) --
	( 92.42,230.56) --
	( 92.32,230.71) --
	( 92.28,230.90) --
	( 92.28,230.98) --
	( 92.30,231.02) --
	( 92.33,231.06) --
	( 92.35,231.09) --
	( 92.40,231.14) --
	( 92.53,231.23) --
	( 92.57,231.27) --
	( 92.60,231.30) --
	( 92.64,231.32) --
	( 92.68,231.35) --
	( 92.74,231.40) --
	( 92.77,231.43) --
	( 92.79,231.45) --
	( 92.83,231.48) --
	( 92.86,231.50) --
	( 92.88,231.52) --
	( 92.95,231.56) --
	( 93.00,231.60) --
	( 93.03,231.64) --
	( 93.06,231.68) --
	( 93.08,231.71) --
	( 93.08,231.76) --
	( 93.08,231.81) --
	( 93.05,231.90) --
	( 93.04,231.95) --
	( 93.00,232.01) --
	( 92.97,232.06) --
	( 92.92,232.10) --
	( 92.85,232.17) --
	( 92.79,232.26) --
	( 92.72,232.36) --
	( 92.67,232.43) --
	( 92.59,232.48) --
	( 92.52,232.51) --
	( 92.45,232.52) --
	( 92.37,232.54) --
	( 92.31,232.54) --
	( 92.25,232.53) --
	( 92.07,232.53) --
	( 91.98,232.54) --
	( 91.82,232.54) --
	( 91.69,232.54) --
	( 91.61,232.55) --
	( 91.55,232.57) --
	( 91.47,232.59) --
	( 91.36,232.64) --
	( 91.33,232.67) --
	( 91.30,232.72) --
	( 91.29,232.76) --
	( 91.29,232.81) --
	( 91.25,232.95) --
	( 91.22,233.00) --
	( 91.17,233.06) --
	( 91.10,233.13) --
	( 91.00,233.24) --
	( 90.96,233.30) --
	( 90.93,233.36) --
	( 90.92,233.40) --
	( 90.93,233.44) --
	( 90.94,233.49) --
	( 90.95,233.50) --
	( 90.97,233.51) --
	( 91.02,233.54) --
	( 91.23,233.63) --
	( 91.29,233.67) --
	( 91.33,233.71) --
	( 91.37,233.75) --
	( 91.39,233.77) --
	( 91.43,233.88) --
	( 91.44,233.98) --
	( 91.43,234.03) --
	( 91.36,234.30) --
	( 91.36,234.36) --
	( 91.36,234.41) --
	( 91.38,234.48) --
	( 91.38,234.52) --
	( 91.37,234.60) --
	( 91.37,234.65) --
	( 91.38,234.69) --
	( 91.37,234.74) --
	( 91.36,234.81) --
	( 91.34,234.86) --
	( 91.34,234.92) --
	( 91.34,234.97) --
	( 91.35,235.02) --
	( 91.37,235.05) --
	( 91.39,235.07) --
	( 91.41,235.08) --
	( 91.42,235.11) --
	( 91.44,235.13) --
	( 91.51,235.18) --
	( 91.61,235.24) --
	( 91.67,235.27) --
	( 91.69,235.30) --
	( 91.70,235.32) --
	( 91.73,235.36) --
	( 91.74,235.39) --
	( 91.75,235.42) --
	( 91.75,235.45) --
	( 91.72,235.52) --
	( 91.69,235.59) --
	( 91.65,235.66) --
	( 91.62,235.70) --
	( 91.52,235.78) --
	( 91.43,235.84) --
	( 91.34,235.89) --
	( 91.27,235.92) --
	( 91.21,235.95) --
	( 91.14,236.00) --
	( 91.06,236.07) --
	( 91.02,236.11) --
	( 90.96,236.20) --
	( 90.92,236.27) --
	( 90.88,236.39) --
	( 90.87,236.45) --
	( 90.86,236.51) --
	( 90.87,236.56) --
	( 90.88,236.62) --
	( 90.90,236.68) --
	( 90.94,236.75) --
	( 90.96,236.80) --
	( 91.02,236.85) --
	( 91.06,236.90) --
	( 91.10,236.94) --
	( 91.14,236.98) --
	( 91.18,237.01) --
	( 91.21,237.08) --
	( 91.23,237.12) --
	( 91.23,237.16) --
	( 91.21,237.20) --
	( 91.18,237.25) --
	( 91.13,237.30) --
	( 91.09,237.34) --
	( 91.05,237.37) --
	( 91.01,237.39) --
	( 90.89,237.45) --
	( 90.85,237.48) --
	( 90.80,237.53) --
	( 90.77,237.56) --
	( 90.70,237.63) --
	( 90.65,237.71) --
	( 90.58,237.83) --
	( 90.53,237.94) --
	( 90.51,238.02) --
	( 90.50,238.14) --
	( 90.50,238.20) --
	( 90.50,238.26) --
	( 90.52,238.41) --
	( 90.53,238.46) --
	( 90.55,238.49) --
	( 90.57,238.56) --
	( 90.57,238.60) --
	( 90.57,238.64) --
	( 90.56,238.67) --
	( 90.54,238.74) --
	( 90.47,238.92) --
	( 90.43,238.98) --
	( 90.42,239.04) --
	( 90.42,239.10) --
	( 90.44,239.24) --
	( 90.46,239.30) --
	( 90.50,239.35) --
	( 90.51,239.40) --
	( 90.60,239.58) --
	( 90.64,239.64) --
	( 90.72,239.72) --
	( 90.81,239.79) --
	( 91.01,240.02) --
	( 91.15,240.18) --
	( 91.24,240.26) --
	( 91.31,240.33) --
	( 91.39,240.38) --
	( 91.50,240.46) --
	( 91.64,240.58) --
	( 91.83,240.72) --
	( 91.90,240.75) --
	( 92.01,240.79) --
	( 92.09,240.80) --
	( 92.13,240.81) --
	( 92.26,240.84) --
	( 92.33,240.85) --
	( 92.40,240.85) --
	( 92.47,240.85) --
	( 92.56,240.83) --
	( 92.62,240.83) --
	( 92.69,240.82) --
	( 92.77,240.79) --
	( 93.05,240.70) --
	( 93.35,240.64) --
	( 93.45,240.61) --
	( 93.59,240.56) --
	( 93.68,240.55) --
	( 93.79,240.52) --
	( 93.85,240.50) --
	( 93.97,240.47) --
	( 94.10,240.45) --
	( 94.20,240.43) --
	( 94.26,240.41) --
	( 94.34,240.38) --
	( 94.42,240.36) --
	( 94.48,240.34) --
	( 94.53,240.32) --
	( 94.58,240.29) --
	( 94.65,240.27) --
	( 94.71,240.24) --
	( 94.78,240.23) --
	( 94.88,240.20) --
	( 94.95,240.18) --
	( 95.16,240.16) --
	( 95.43,240.14) --
	( 95.57,240.15) --
	( 95.65,240.16) --
	( 95.70,240.18) --
	( 95.77,240.19) --
	( 95.84,240.21) --
	( 95.91,240.23) --
	( 95.97,240.26) --
	( 96.06,240.30) --
	( 96.17,240.36) --
	( 96.23,240.40) --
	( 96.28,240.44) --
	( 96.44,240.53) --
	( 96.54,240.62) --
	( 96.70,240.75) --
	( 96.75,240.79) --
	( 96.86,240.89) --
	( 96.91,240.94) --
	( 96.96,241.00) --
	( 97.01,241.04) --
	( 97.06,241.11) --
	( 97.09,241.14) --
	( 97.18,241.21) --
	( 97.27,241.28) --
	( 97.30,241.30) --
	( 97.33,241.33) --
	( 97.39,241.38) --
	( 97.46,241.44) --
	( 97.54,241.48) --
	( 97.66,241.56) --
	( 97.70,241.59) --
	( 97.77,241.68) --
	( 97.79,241.72) --
	( 97.81,241.77) --
	( 97.82,241.81) --
	( 97.82,241.88) --
	( 97.81,242.02) --
	( 97.80,242.09) --
	( 97.80,242.13) --
	( 97.78,242.16) --
	( 97.77,242.18) --
	( 97.76,242.20) --
	( 97.74,242.25) --
	( 97.69,242.31) --
	( 97.63,242.42) --
	( 97.60,242.52) --
	( 97.56,242.71) --
	( 97.56,242.81) --
	( 97.56,242.85) --
	( 97.56,242.90) --
	( 97.57,242.96) --
	( 97.57,243.03) --
	( 97.56,243.17) --
	( 97.55,243.37) --
	( 97.56,243.43) --
	( 97.57,243.49) --
	( 97.59,243.53) --
	( 97.64,243.58) --
	( 97.70,243.65) --
	( 97.75,243.71) --
	( 97.77,243.74) --
	( 97.81,243.78) --
	( 97.83,243.80) --
	( 97.88,243.84) --
	( 97.90,243.87) --
	( 97.91,243.89) --
	( 97.93,243.90) --
	( 97.95,243.93) --
	( 97.97,243.97) --
	( 98.02,244.02) --
	( 98.05,244.04) --
	( 98.06,244.07) --
	( 98.07,244.09) --
	( 98.08,244.13) --
	( 98.07,244.23) --
	( 98.05,244.32) --
	( 97.99,244.43) --
	( 97.96,244.48) --
	( 97.94,244.52) --
	( 97.89,244.58) --
	( 97.83,244.65) --
	( 97.77,244.73) --
	( 97.73,244.79) --
	( 97.61,244.90) --
	( 97.56,244.94) --
	( 97.47,245.03) --
	( 97.39,245.08) --
	( 97.35,245.12) --
	( 97.30,245.16) --
	( 97.24,245.20) --
	( 97.15,245.24) --
	( 97.04,245.28) --
	( 96.97,245.30) --
	( 96.94,245.30) --
	( 96.88,245.33) --
	( 96.83,245.36) --
	( 96.77,245.39) --
	( 96.73,245.42) --
	( 96.70,245.46) --
	( 96.67,245.50) --
	( 96.65,245.54) --
	( 96.61,245.70) --
	( 96.55,245.79) --
	( 96.51,245.84) --
	( 96.45,245.87) --
	( 96.42,245.88) --
	( 96.37,245.89) --
	( 96.26,245.89) --
	( 96.16,245.88) --
	( 96.10,245.88) --
	( 96.02,245.89) --
	( 95.98,245.90) --
	( 95.93,245.91) --
	( 95.90,245.92) --
	( 95.80,245.95) --
	( 95.74,245.97) --
	( 95.67,246.01) --
	( 95.61,246.04) --
	( 95.52,246.11) --
	( 95.44,246.17) --
	( 95.19,246.26) --
	( 95.14,246.29) --
	( 95.10,246.31) --
	( 95.02,246.36) --
	( 94.99,246.40) --
	( 94.95,246.44) --
	( 94.89,246.47) --
	( 94.83,246.51) --
	( 94.80,246.53) --
	( 94.74,246.56) --
	( 94.62,246.63) --
	( 94.53,246.67) --
	( 94.44,246.71) --
	( 94.34,246.75) --
	( 94.18,246.82) --
	( 94.12,246.84) --
	( 94.09,246.86) --
	( 94.06,246.87) --
	( 94.03,246.88) --
	( 93.96,246.92) --
	( 93.92,246.96) --
	( 93.87,247.00) --
	( 93.80,247.08) --
	( 93.78,247.12) --
	( 93.74,247.21) --
	( 93.72,247.27) --
	( 93.70,247.34) --
	( 93.71,247.42) --
	( 93.72,247.46) --
	( 93.73,247.50) --
	( 93.75,247.56) --
	( 93.76,247.60) --
	( 93.77,247.66) --
	( 93.77,247.71) --
	( 93.77,247.76) --
	( 93.76,247.86) --
	( 93.76,247.90) --
	( 93.72,248.06) --
	( 93.72,248.08) --
	( 93.72,248.08) --
	( 93.71,248.10) --
	( 93.72,248.12) --
	( 93.73,248.15) --
	( 93.74,248.18) --
	( 93.75,248.23) --
	( 93.74,248.28) --
	( 93.74,248.31) --
	( 93.72,248.38) --
	( 93.69,248.45) --
	( 93.63,248.51) --
	( 93.59,248.57) --
	( 93.55,248.63) --
	( 93.53,248.68) --
	( 93.53,248.71) --
	( 93.54,248.75) --
	( 93.55,248.77) --
	( 93.56,248.81) --
	( 93.55,248.89) --
	( 93.54,248.94) --
	( 93.54,248.96) --
	( 93.54,248.99) --
	( 93.54,249.01) --
	( 93.56,249.04) --
	( 93.59,249.06) --
	( 93.63,249.11) --
	( 93.65,249.15) --
	( 93.65,249.17) --
	( 93.64,249.24) --
	( 93.64,249.29) --
	( 93.63,249.36) --
	( 93.63,249.41) --
	( 93.64,249.51) --
	( 93.64,249.53) --
	( 93.63,249.55) --
	( 93.66,249.58) --
	( 93.66,249.60) --
	( 93.66,249.62) --
	( 93.66,249.68) --
	( 93.64,249.72) --
	( 93.63,249.75) --
	( 93.61,249.79) --
	( 93.57,249.82) --
	( 93.54,249.86) --
	( 93.49,249.97) --
	( 93.47,250.05) --
	( 93.47,250.10) --
	( 93.46,250.16) --
	( 93.46,250.20) --
	( 93.47,250.24) --
	( 93.49,250.28) --
	( 93.54,250.34) --
	( 93.64,250.41) --
	( 93.70,250.49) --
	( 93.71,250.51) --
	( 93.72,250.53) --
	( 93.71,250.55) --
	( 93.69,250.57) --
	( 93.66,250.61) --
	( 93.63,250.68) --
	( 93.63,250.74) --
	( 93.59,250.81) --
	( 93.60,250.84) --
	( 93.61,250.87) --
	( 93.67,250.97) --
	( 93.68,251.01) --
	( 93.68,251.04) --
	( 93.67,251.07) --
	( 93.67,251.12) --
	( 93.67,251.17) --
	( 93.68,251.21) --
	( 93.68,251.24) --
	( 93.69,251.30) --
	( 93.68,251.34) --
	( 93.66,251.38) --
	( 93.65,251.40) --
	( 93.63,251.42) --
	( 93.60,251.46) --
	( 93.60,251.50) --
	( 93.61,251.53) --
	( 93.64,251.59) --
	( 93.67,251.62) --
	( 93.71,251.65) --
	( 93.73,251.67) --
	( 93.78,251.71) --
	( 93.83,251.73) --
	( 93.92,251.79) --
	( 93.94,251.82) --
	( 93.94,251.84) --
	( 93.91,251.88) --
	( 93.89,251.90) --
	( 93.87,251.90) --
	( 93.75,251.91) --
	( 93.72,251.92) --
	( 93.60,251.97) --
	( 93.53,252.01) --
	( 93.49,252.02) --
	( 93.46,252.04) --
	( 93.43,252.05) --
	( 93.40,252.08) --
	( 93.39,252.10) --
	( 93.39,252.12) --
	( 93.41,252.14) --
	( 93.43,252.16) --
	( 93.46,252.20) --
	( 93.46,252.22) --
	( 93.45,252.26) --
	( 93.43,252.29) --
	( 93.41,252.31) --
	( 93.38,252.33) --
	( 93.32,252.34) --
	( 93.28,252.35) --
	( 93.24,252.37) --
	( 93.20,252.39) --
	( 93.20,252.41) --
	( 93.20,252.43) --
	( 93.21,252.45) --
	( 93.28,252.52) --
	( 93.32,252.58) --
	( 93.33,252.62) --
	( 93.39,252.69) --
	( 93.41,252.73) --
	( 93.43,252.83) --
	( 93.43,252.89) --
	( 93.43,252.95) --
	( 93.40,253.03) --
	( 93.40,253.07) --
	( 93.40,253.11) --
	( 93.40,253.18) --
	( 93.41,253.31) --
	( 93.42,253.37) --
	( 93.43,253.41) --
	( 93.43,253.51) --
	( 93.42,253.54) --
	( 93.40,253.62) --
	( 93.40,253.74) --
	( 93.41,253.80) --
	( 93.42,253.82) --
	( 93.43,253.85) --
	( 93.44,253.89) --
	( 93.45,253.95) --
	( 93.45,253.99) --
	( 93.40,254.11) --
	( 93.39,254.14) --
	( 93.40,254.20) --
	( 93.40,254.25) --
	( 93.39,254.28) --
	( 93.38,254.32) --
	( 93.37,254.36) --
	( 93.35,254.42) --
	( 93.35,254.45) --
	( 93.34,254.53) --
	( 93.35,254.60) --
	( 93.35,254.66) --
	( 93.37,254.75) --
	( 93.39,254.81) --
	( 93.42,254.91) --
	( 93.45,254.98) --
	( 93.52,255.14) --
	( 93.55,255.19) --
	( 93.59,255.28) --
	( 93.60,255.34) --
	( 93.61,255.38) --
	( 93.62,255.41) --
	( 93.62,255.45) --
	( 93.61,255.49) --
	( 93.60,255.62) --
	( 93.60,255.68) --
	( 93.60,255.73) --
	( 93.61,255.78) --
	( 93.62,255.88) --
	( 93.62,255.92) --
	( 93.65,255.97) --
	( 93.66,256.02) --
	( 93.69,256.06) --
	( 93.71,256.09) --
	( 93.74,256.16) --
	( 93.79,256.22) --
	( 93.88,256.29) --
	( 93.96,256.35) --
	( 93.99,256.38) --
	( 94.03,256.41) --
	( 94.08,256.44) --
	( 94.12,256.47) --
	( 94.17,256.50) --
	( 94.24,256.53) --
	( 94.29,256.55) --
	( 94.34,256.56) --
	( 94.38,256.56) --
	( 94.43,256.56) --
	( 94.46,256.55) --
	( 94.50,256.54) --
	( 94.61,256.49) --
	( 94.67,256.46) --
	( 94.73,256.41) --
	( 94.78,256.38) --
	( 94.80,256.35) --
	( 94.85,256.30) --
	( 94.87,256.28) --
	( 94.92,256.26) --
	( 94.97,256.25) --
	( 95.02,256.24) --
	( 95.06,256.24) --
	( 95.12,256.26) --
	( 95.20,256.28) --
	( 95.27,256.31) --
	( 95.32,256.33) --
	( 95.40,256.37) --
	( 95.44,256.38) --
	( 95.48,256.40) --
	( 95.62,256.46) --
	( 95.80,256.58) --
	( 95.85,256.62) --
	( 95.93,256.69) --
	( 96.03,256.78) --
	( 96.08,256.82) --
	( 96.11,256.83) --
	( 96.21,256.83) --
	( 96.26,256.83) --
	( 96.30,256.83) --
	( 96.35,256.83) --
	( 96.41,256.84) --
	( 96.44,256.85) --
	( 96.46,256.86) --
	( 96.54,256.88) --
	( 96.56,256.90) --
	( 96.58,256.91) --
	( 96.64,256.92) --
	( 96.68,256.93) --
	( 96.80,257.00) --
	( 96.83,257.00) --
	( 96.86,257.00) --
	( 96.95,257.01) --
	( 97.01,257.03) --
	( 97.11,257.06) --
	( 97.18,257.10) --
	( 97.21,257.11) --
	( 97.23,257.11) --
	( 97.24,257.11) --
	( 97.30,257.09) --
	( 97.33,257.09) --
	( 97.35,257.08) --
	( 97.37,257.07) --
	( 97.42,257.06) --
	( 97.45,257.06) --
	( 97.48,257.04) --
	( 97.55,257.09) --
	( 97.58,257.12) --
	( 97.58,257.15) --
	( 97.58,257.19) --
	( 97.58,257.23) --
	( 97.58,257.26) --
	( 97.59,257.29) --
	( 97.58,257.31) --
	( 97.57,257.34) --
	( 97.56,257.36) --
	( 97.50,257.48) --
	( 97.49,257.53) --
	( 97.49,257.58) --
	( 97.45,257.66) --
	( 97.43,257.72) --
	( 97.41,257.79) --
	( 97.39,257.88) --
	( 97.36,257.96) --
	( 97.35,258.01) --
	( 97.35,258.10) --
	( 97.35,258.16) --
	( 97.33,258.24) --
	( 97.29,258.30) --
	( 97.27,258.38) --
	( 97.23,258.47) --
	( 97.20,258.50) --
	( 97.13,258.51) --
	( 97.07,258.52) --
	( 97.03,258.53) --
	( 97.00,258.56) --
	( 96.96,258.59) --
	( 96.91,258.66) --
	( 96.83,258.76) --
	( 96.76,258.85) --
	( 96.73,258.90) --
	( 96.71,258.98) --
	( 96.68,259.08) --
	( 96.66,259.14) --
	( 96.64,259.18) --
	( 96.63,259.21) --
	( 96.60,259.25) --
	( 96.58,259.28) --
	( 96.45,259.35) --
	( 96.39,259.42) --
	( 96.35,259.50) --
	( 96.33,259.57) --
	( 96.32,259.65) --
	( 96.31,259.70) --
	( 96.31,259.74) --
	( 96.31,259.77) --
	( 96.32,259.81) --
	( 96.31,259.84) --
	( 96.29,259.88) --
	( 96.27,259.91) --
	( 96.24,259.94) --
	( 96.20,259.97) --
	( 96.18,259.98) --
	( 96.12,259.98) --
	( 96.08,259.97) --
	( 96.01,259.94) --
	( 95.99,259.92) --
	( 95.98,259.92) --
	( 95.95,259.98) --
	( 95.92,260.03) --
	( 95.90,260.08) --
	( 95.89,260.12) --
	( 95.89,260.15) --
	( 95.89,260.19) --
	( 95.89,260.27) --
	( 95.90,260.30) --
	( 95.90,260.40) --
	( 95.90,260.49) --
	( 95.90,260.59) --
	( 95.91,260.62) --
	( 95.92,260.65) --
	( 95.92,260.67) --
	( 95.92,260.69) --
	( 95.91,260.72) --
	( 95.86,260.76) --
	( 95.82,260.80) --
	( 95.79,260.85) --
	( 95.76,260.87) --
	( 95.75,260.90) --
	( 95.73,260.92) --
	( 95.72,260.96) --
	( 95.72,261.01) --
	( 95.72,261.06) --
	( 95.68,261.11) --
	( 95.64,261.16) --
	( 95.55,261.27) --
	( 95.49,261.34) --
	( 95.47,261.40) --
	( 95.45,261.43) --
	( 95.43,261.46) --
	( 95.39,261.58) --
	( 95.36,261.76) --
	( 95.35,261.95) --
	( 95.33,262.04) --
	( 95.33,262.09) --
	( 95.31,262.19) --
	( 95.30,262.24) --
	( 95.27,262.31) --
	( 95.26,262.35) --
	( 95.27,262.38) --
	( 95.27,262.41) --
	( 95.26,262.43) --
	( 95.24,262.44) --
	( 95.20,262.53) --
	( 95.17,262.57) --
	( 95.12,262.66) --
	( 95.14,262.66) --
	( 95.19,262.66) --
	( 95.31,262.66) --
	( 95.35,262.65) --
	( 95.39,262.64) --
	( 95.41,262.63) --
	( 95.45,262.59) --
	( 95.47,262.54) --
	( 95.49,262.46) --
	( 95.49,262.40) --
	( 95.51,262.36) --
	( 95.51,262.34) --
	( 95.57,262.29) --
	( 95.58,262.27) --
	( 95.59,262.25) --
	( 95.58,262.21) --
	( 95.62,262.13) --
	( 95.61,262.09) --
	( 95.62,262.06) --
	( 95.62,262.01) --
	( 95.62,261.95) --
	( 95.64,261.79) --
	( 95.66,261.73) --
	( 95.67,261.68) --
	( 95.69,261.62) --
	( 95.75,261.51) --
	( 95.87,261.36) --
	( 95.98,261.24) --
	( 96.08,261.13) --
	( 96.11,261.08) --
	( 96.14,261.07) --
	( 96.16,261.06) --
	( 96.24,261.05) --
	( 96.27,261.04) --
	( 96.29,261.05) --
	( 96.33,261.05) --
	( 96.42,261.06) --
	( 96.48,261.08) --
	( 96.50,261.09) --
	( 96.51,261.10) --
	( 96.49,261.12) --
	( 96.48,261.14) --
	( 96.47,261.17) --
	( 96.48,261.22) --
	( 96.48,261.24) --
	( 96.55,261.27) --
	( 96.64,261.30) --
	( 96.68,261.31) --
	( 96.70,261.32) --
	( 96.74,261.32) --
	( 96.77,261.31) --
	( 96.80,261.30) --
	( 96.82,261.29) --
	( 96.84,261.25) --
	( 96.86,261.23) --
	( 96.88,261.20) --
	( 96.89,261.17) --
	( 96.91,261.13) --
	( 96.92,261.10) --
	( 96.94,261.09) --
	( 96.96,261.09) --
	( 96.98,261.09) --
	( 97.01,261.09) --
	( 97.05,261.10) --
	( 97.06,261.10) --
	( 97.10,261.09) --
	( 97.21,261.13) --
	( 97.34,261.17) --
	( 97.38,261.18) --
	( 97.44,261.18) --
	( 97.48,261.18) --
	( 97.58,261.21) --
	( 97.67,261.24) --
	( 97.70,261.24) --
	( 97.73,261.24) --
	( 97.77,261.25) --
	( 97.81,261.28) --
	( 97.83,261.30) --
	( 97.88,261.31) --
	( 97.91,261.31) --
	( 97.93,261.33) --
	( 97.99,261.37) --
	( 98.02,261.38) --
	( 98.07,261.40) --
	( 98.08,261.41) --
	( 98.09,261.43) --
	( 98.10,261.45) --
	( 98.10,261.48) --
	( 98.10,261.50) --
	( 98.11,261.53) --
	( 98.11,261.57) --
	( 98.11,261.61) --
	( 98.11,261.77) --
	( 98.08,261.83) --
	( 98.08,261.85) --
	( 98.09,261.95) --
	( 98.08,261.97) --
	( 98.08,262.01) --
	( 98.06,262.04) --
	( 98.03,262.11) --
	( 98.03,262.15) --
	( 98.04,262.18) --
	( 98.06,262.21) --
	( 98.06,262.23) --
	( 98.08,262.26) --
	( 98.08,262.31) --
	( 98.09,262.34) --
	( 98.08,262.36) --
	( 98.06,262.37) --
	( 98.01,262.40) --
	( 97.96,262.40) --
	( 97.90,262.41) --
	( 97.83,262.41) --
	( 97.80,262.42) --
	( 97.77,262.46) --
	( 97.75,262.48) --
	( 97.73,262.51) --
	( 97.72,262.58) --
	( 97.76,262.65) --
	( 97.79,262.71) --
	( 97.81,262.81) --
	( 97.84,262.84) --
	( 97.86,262.87) --
	( 97.87,262.91) --
	( 97.90,263.00) --
	( 97.90,263.01) --
	( 97.84,263.07) --
	( 97.82,263.12) --
	( 97.81,263.14) --
	( 97.80,263.16) --
	( 97.80,263.21) --
	( 97.80,263.24) --
	( 97.83,263.34) --
	( 97.83,263.35) --
	( 97.83,263.37) --
	( 97.78,263.45) --
	( 97.76,263.49) --
	( 97.76,263.59) --
	( 97.76,263.61) --
	( 97.75,263.64) --
	( 97.73,263.67) --
	( 97.71,263.71) --
	( 97.66,263.76) --
	( 97.65,263.81) --
	( 97.66,263.83) --
	( 97.68,263.85) --
	( 97.70,263.86) --
	( 97.75,263.87) --
	( 97.83,263.90) --
	( 97.91,263.92) --
	( 97.93,263.93) --
	( 97.96,263.95) --
	( 98.00,263.98) --
	( 98.03,263.99) --
	( 98.05,264.03) --
	( 98.06,264.07) --
	( 98.09,264.12) --
	( 98.11,264.18) --
	( 98.11,264.25) --
	( 98.10,264.30) --
	( 98.09,264.33) --
	( 98.05,264.38) --
	( 98.03,264.44) --
	( 98.00,264.49) --
	( 97.98,264.52) --
	( 97.98,264.54) --
	( 98.00,264.57) --
	( 98.13,264.69) --
	( 98.15,264.74) --
	( 98.16,264.76) --
	( 98.16,264.80) --
	( 98.14,264.84) --
	( 98.12,264.86) --
	( 98.10,264.87) --
	( 98.06,264.88) --
	( 98.03,264.88) --
	( 97.86,264.85) --
	( 97.80,264.85) --
	( 97.75,264.84) --
	( 97.70,264.83) --
	( 97.63,264.83) --
	( 97.49,264.82) --
	( 97.48,264.83) --
	( 97.47,264.84) --
	( 97.46,264.86) --
	( 97.45,264.88) --
	( 97.46,264.92) --
	( 97.46,264.94) --
	( 97.47,264.99) --
	( 97.49,265.04) --
	( 97.49,265.06) --
	( 97.48,265.08) --
	( 97.47,265.11) --
	( 97.44,265.20) --
	( 97.42,265.30) --
	( 97.42,265.34) --
	( 97.42,265.37) --
	( 97.42,265.40) --
	( 97.42,265.45) --
	( 97.43,265.48) --
	( 97.43,265.50) --
	( 97.41,265.53) --
	( 97.41,265.57) --
	( 97.41,265.59) --
	( 97.42,265.61) --
	( 97.44,265.63) --
	( 97.46,265.65) --
	( 97.52,265.67) --
	( 97.63,265.69) --
	( 97.70,265.71) --
	( 97.80,265.76) --
	( 97.85,265.76) --
	( 97.89,265.78) --
	( 97.95,265.82) --
	( 97.96,265.84) --
	( 97.96,265.86) --
	( 97.96,265.87) --
	( 97.94,265.89) --
	( 97.88,265.93) --
	( 97.86,265.96) --
	( 97.83,266.01) --
	( 97.78,266.07) --
	( 97.71,266.12) --
	( 97.65,266.16) --
	( 97.61,266.21) --
	( 97.60,266.23) --
	( 97.61,266.27) --
	( 97.60,266.29) --
	( 97.58,266.31) --
	( 97.58,266.35) --
	( 97.57,266.38) --
	( 97.59,266.43) --
	( 97.58,266.45) --
	( 97.59,266.48) --
	( 97.60,266.55) --
	( 97.61,266.64) --
	( 97.61,266.66) --
	( 97.60,266.68) --
	( 97.59,266.70) --
	( 97.58,266.73) --
	( 97.58,266.79) --
	( 97.58,266.81) --
	( 97.63,266.87) --
	( 97.65,266.90) --
	( 97.65,266.93) --
	( 97.66,266.95) --
	( 97.66,266.98) --
	( 97.66,267.01) --
	( 97.64,267.03) --
	( 97.62,267.08) --
	( 97.53,267.19) --
	( 97.52,267.22) --
	( 97.51,267.24) --
	( 97.52,267.28) --
	( 97.54,267.33) --
	( 97.54,267.35) --
	( 97.52,267.39) --
	( 97.51,267.42) --
	( 97.51,267.45) --
	( 97.51,267.50) --
	( 97.52,267.66) --
	( 97.50,267.83) --
	( 97.47,267.86) --
	( 97.49,267.87) --
	( 97.54,267.88) --
	( 97.57,267.88) --
	( 97.60,267.88) --
	( 97.63,267.88) --
	( 97.67,267.87) --
	( 97.70,267.87) --
	( 97.71,267.86) --
	( 97.73,267.86) --
	( 97.76,267.84) --
	( 97.84,267.79) --
	( 97.87,267.78) --
	( 97.91,267.77) --
	( 97.95,267.77) --
	( 97.98,267.76) --
	( 98.05,267.73) --
	( 98.10,267.72) --
	( 98.17,267.71) --
	( 98.19,267.72) --
	( 98.25,267.72) --
	( 98.27,267.71) --
	( 98.35,267.68) --
	( 98.37,267.68) --
	( 98.40,267.67) --
	( 98.57,267.68) --
	( 98.61,267.68) --
	( 98.64,267.68) --
	( 98.68,267.66) --
	( 98.78,267.63) --
	( 98.81,267.62) --
	( 98.84,267.62) --
	( 98.87,267.63) --
	( 98.92,267.63) --
	( 98.95,267.63) --
	( 98.97,267.65) --
	( 98.98,267.67) --
	( 98.98,267.80) --
	( 98.99,267.88) --
	( 99.02,267.92) --
	( 99.03,267.94) --
	( 99.03,267.95) --
	( 99.02,267.99) --
	( 99.01,268.02) --
	( 99.02,268.05) --
	( 99.02,268.08) --
	( 99.02,268.11) --
	( 99.03,268.15) --
	( 99.03,268.23) --
	( 99.01,268.29) --
	( 98.96,268.39) --
	( 98.94,268.42) --
	( 98.95,268.44) --
	( 98.98,268.50) --
	( 99.00,268.53) --
	( 99.00,268.56) --
	( 99.00,268.59) --
	( 98.99,268.62) --
	( 98.95,268.69) --
	( 98.93,268.73) --
	( 98.92,268.76) --
	( 98.91,268.85) --
	( 98.91,268.97) --
	( 98.91,268.99) --
	( 98.89,269.02) --
	( 98.86,269.04) --
	( 98.80,269.09) --
	( 98.79,269.10) --
	( 98.73,269.13) --
	( 98.71,269.15) --
	( 98.67,269.21) --
	( 98.66,269.25) --
	( 98.67,269.27) --
	( 98.67,269.29) --
	( 98.66,269.35) --
	( 98.66,269.39) --
	( 98.65,269.43) --
	( 98.63,269.45) --
	( 98.60,269.48) --
	( 98.58,269.52) --
	( 98.57,269.56) --
	( 98.57,269.58) --
	( 98.55,269.66) --
	( 98.55,269.70) --
	( 98.54,269.71) --
	( 98.53,269.74) --
	( 98.51,269.78) --
	( 98.51,269.82) --
	( 98.53,269.89) --
	( 98.53,269.93) --
	( 98.51,269.95) --
	( 98.50,269.97) --
	( 98.48,269.98) --
	( 98.44,269.99) --
	( 98.41,270.01) --
	( 98.39,270.02) --
	( 98.38,270.05) --
	( 98.36,270.07) --
	( 98.34,270.12) --
	( 98.32,270.18) --
	( 98.30,270.24) --
	( 98.28,270.27) --
	( 98.25,270.30) --
	( 98.22,270.32) --
	( 98.18,270.36) --
	( 98.14,270.39) --
	( 98.10,270.41) --
	( 98.07,270.43) --
	( 98.03,270.48) --
	( 98.03,270.51) --
	( 98.03,270.53) --
	( 98.04,270.56) --
	( 98.05,270.57) --
	( 98.11,270.59) --
	( 98.15,270.60) --
	( 98.18,270.62) --
	( 98.21,270.65) --
	( 98.28,270.71) --
	( 98.33,270.77) --
	( 98.37,270.82) --
	( 98.41,270.88) --
	( 98.45,270.95) --
	( 98.47,271.02) --
	( 98.49,271.10) --
	( 98.49,271.14) --
	( 98.47,271.17) --
	( 98.46,271.19) --
	( 98.44,271.24) --
	( 98.38,271.32) --
	( 98.36,271.39) --
	( 98.32,271.45) --
	( 98.30,271.47) --
	( 98.27,271.49) --
	( 98.23,271.50) --
	( 98.20,271.50) --
	( 98.17,271.49) --
	( 98.14,271.49) --
	( 98.10,271.51) --
	( 98.07,271.53) --
	( 97.97,271.60) --
	( 97.91,271.66) --
	( 97.88,271.68) --
	( 97.85,271.70) --
	( 97.76,271.71) --
	( 97.73,271.73) --
	( 97.72,271.76) --
	( 97.72,271.79) --
	( 97.73,271.83) --
	( 97.73,271.90) --
	( 97.73,271.95) --
	( 97.73,272.00) --
	( 97.65,272.12) --
	( 97.64,272.16) --
	( 97.62,272.19) --
	( 97.60,272.23) --
	( 97.55,272.27) --
	( 97.53,272.29) --
	( 97.44,272.41) --
	( 97.42,272.46) --
	( 97.41,272.52) --
	( 97.40,272.58) --
	( 97.39,272.61) --
	( 97.37,272.66) --
	( 97.38,272.67) --
	( 97.42,272.67) --
	( 97.45,272.67) --
	( 97.48,272.68) --
	( 97.52,272.68) --
	( 97.58,272.67) --
	( 97.67,272.66) --
	( 97.75,272.66) --
	( 97.79,272.66) --
	( 97.83,272.66) --
	( 97.89,272.68) --
	( 97.92,272.70) --
	( 97.94,272.71) --
	( 97.98,272.74) --
	( 97.99,272.75) --
	( 97.98,272.75) --
	( 97.98,272.75) --
	( 97.97,272.75) --
	( 97.97,272.75) --
	( 97.98,272.75) --
	( 98.07,272.79) --
	( 98.10,272.80) --
	( 98.12,272.82) --
	( 98.16,272.84) --
	( 98.20,272.84) --
	( 98.24,272.81) --
	( 98.26,272.81) --
	( 98.27,272.80) --
	( 98.28,272.79) --
	( 98.33,272.76) --
	( 98.34,272.75) --
	( 98.35,272.75) --
	( 98.35,272.75) --
	( 98.36,272.75) --
	( 98.36,272.75) --
	( 98.37,272.75) --
	( 98.37,272.75) --
	( 98.38,272.75) --
	( 98.42,272.72) --
	( 98.47,272.70) --
	( 98.51,272.68) --
	( 98.55,272.67) --
	( 98.59,272.66) --
	( 98.63,272.67) --
	( 98.71,272.69) --
	( 98.77,272.73) --
	( 98.80,272.73) --
	( 98.83,272.72) --
	( 98.88,272.70) --
	( 98.97,272.68) --
	( 99.02,272.68) --
	( 99.10,272.68) --
	( 99.13,272.67) --
	( 99.15,272.67) --
	( 99.16,272.65) --
	( 99.18,272.62) --
	( 99.18,272.57) --
	( 99.17,272.44) --
	( 99.17,272.41) --
	( 99.17,272.38) --
	( 99.19,272.27) --
	( 99.24,272.16) --
	( 99.26,272.14) --
	( 99.29,272.09) --
	( 99.35,272.03) --
	( 99.39,271.99) --
	( 99.42,271.97) --
	( 99.45,271.96) --
	( 99.49,271.95) --
	( 99.51,271.95) --
	( 99.63,271.99) --
	( 99.69,272.00) --
	( 99.74,272.01) --
	( 99.77,272.03) --
	( 99.81,272.05) --
	( 99.83,272.09) --
	( 99.85,272.10) --
	( 99.89,272.11) --
	( 99.91,272.12) --
	( 99.95,272.12) --
	( 99.99,272.12) --
	(100.02,272.13) --
	(100.05,272.14) --
	(100.10,272.17) --
	(100.13,272.18) --
	(100.16,272.19) --
	(100.21,272.25) --
	(100.24,272.27) --
	(100.27,272.28) --
	(100.30,272.29) --
	(100.46,272.33) --
	(100.51,272.36) --
	(100.54,272.37) --
	(100.55,272.39) --
	(100.56,272.42) --
	(100.56,272.49) --
	(100.57,272.52) --
	(100.58,272.56) --
	(100.60,272.61) --
	(100.62,272.63) --
	(100.66,272.65) --
	(100.70,272.66) --
	(100.89,272.74) --
	(100.92,272.75) --
	(100.92,272.75) --
	(100.93,272.75) --
	(100.93,272.75) --
	(100.94,272.75) --
	(100.94,272.75) --
	(100.95,272.75) --
	(100.95,272.75) --
	(100.96,272.75) --
	(100.96,272.75) --
	(100.97,272.75) --
	(100.97,272.75) --
	(100.98,272.75) --
	(100.99,272.76) --
	(101.07,272.80) --
	(101.10,272.84) --
	(101.11,272.86) --
	(101.12,272.90) --
	(101.14,272.93) --
	(101.17,272.96) --
	(101.20,272.98) --
	(101.22,273.02) --
	(101.28,273.07) --
	(101.32,273.13) --
	(101.36,273.17) --
	(101.50,273.26) --
	(101.52,273.28) --
	(101.54,273.30) --
	(101.57,273.34) --
	(101.59,273.38) --
	(101.60,273.42) --
	(101.63,273.44) --
	(101.64,273.45) --
	(101.68,273.49) --
	(101.69,273.50) --
	(101.70,273.52) --
	(101.71,273.54) --
	(101.71,273.59) --
	(101.70,273.62) --
	(101.70,273.64) --
	(101.65,273.70) --
	(101.66,273.80) --
	(101.68,273.85) --
	(101.71,273.87) --
	(101.74,273.91) --
	(101.76,273.92) --
	(101.82,273.95) --
	(101.85,273.96) --
	(101.89,273.96) --
	(101.93,273.96) --
	(101.98,273.95) --
	(102.09,273.95) --
	(102.15,273.95) --
	(102.21,273.95) --
	(102.26,273.95) --
	(102.36,273.94) --
	(102.45,273.94) --
	(102.65,273.92) --
	(102.70,273.91) --
	(102.77,273.91) --
	(102.85,273.92) --
	(102.90,273.92) --
	(102.97,273.93) --
	(103.04,273.93) --
	(103.07,273.94) --
	(103.11,273.95) --
	(103.16,273.96) --
	(103.23,273.99) --
	(103.27,274.01) --
	(103.30,274.03) --
	(103.35,274.05) --
	(103.43,274.08) --
	(103.50,274.10) --
	(103.56,274.13) --
	(103.68,274.18) --
	(103.73,274.19) --
	(103.83,274.25) --
	(103.89,274.26) --
	(103.92,274.27) --
	(103.97,274.29) --
	(104.00,274.30) --
	(104.08,274.34) --
	(104.13,274.36) --
	(104.19,274.37) --
	(104.27,274.41) --
	(104.35,274.47) --
	(104.37,274.48) --
	(104.47,274.51) --
	(104.51,274.52) --
	(104.55,274.52) --
	(104.69,274.52) --
	(104.82,274.50) --
	(104.90,274.49) --
	(105.06,274.49) --
	(105.11,274.48) --
	(105.19,274.46) --
	(105.35,274.42) --
	(105.41,274.41) --
	(105.51,274.38) --
	(105.63,274.37) --
	(105.68,274.37) --
	(105.78,274.38) --
	(105.95,274.43) --
	(105.98,274.44) --
	(106.06,274.47) --
	(106.12,274.49) --
	(106.15,274.51) --
	(106.23,274.56) --
	(106.26,274.58) --
	(106.28,274.61) --
	(106.30,274.65) --
	(106.31,274.67) --
	(106.30,274.72) --
	(106.29,274.74) --
	(106.29,274.75) --
	(106.28,274.77) --
	(106.29,274.78) --
	(106.31,274.80) --
	(106.32,274.81) --
	(106.38,274.82) --
	(106.41,274.82) --
	(106.46,274.81) --
	(106.48,274.81) --
	(106.51,274.82) --
	(106.54,274.83) --
	(106.54,274.84) --
	(106.54,274.85) --
	(106.54,274.87) --
	(106.52,274.88) --
	(106.48,274.95) --
	(106.46,274.98) --
	(106.45,275.02) --
	(106.44,275.04) --
	(106.45,275.06) --
	(106.47,275.07) --
	(106.48,275.08) --
	(106.50,275.11) --
	(106.53,275.13) --
	(106.55,275.14) --
	(106.58,275.16) --
	(106.63,275.17) --
	(106.67,275.19) --
	(106.70,275.19) --
	(106.73,275.20) --
	(106.79,275.20) --
	(106.85,275.20) --
	(106.89,275.20) --
	(106.97,275.21) --
	(107.00,275.21) --
	(107.07,275.22) --
	(107.16,275.27) --
	(107.18,275.28) --
	(107.37,275.33) --
	(107.44,275.35) --
	(107.48,275.36) --
	(107.49,275.37) --
	(107.53,275.38) --
	(107.57,275.38) --
	(107.61,275.39) --
	(107.66,275.40) --
	(107.71,275.40) --
	(107.79,275.41) --
	(107.81,275.42) --
	(107.86,275.43) --
	(107.96,275.43) --
	(107.97,275.42) --
	(107.99,275.42) --
	(108.00,275.42) --
	(108.02,275.44) --
	(108.07,275.49) --
	(108.15,275.55) --
	(108.18,275.59) --
	(108.21,275.65) --
	(108.21,275.67) --
	(108.18,275.74) --
	(108.16,275.77) --
	(108.15,275.80) --
	(108.15,275.84) --
	(108.17,275.89) --
	(108.20,275.95) --
	(108.21,275.98) --
	(108.22,276.00) --
	(108.28,276.04) --
	(108.28,276.05) --
	(108.31,276.09) --
	(108.33,276.16) --
	(108.34,276.18) --
	(108.36,276.20) --
	(108.37,276.21) --
	(108.38,276.22) --
	(108.43,276.24) --
	(108.46,276.27) --
	(108.52,276.31) --
	(108.53,276.32) --
	(108.51,276.35) --
	(108.48,276.39) --
	(108.47,276.40) --
	(108.45,276.42) --
	(108.43,276.44) --
	(108.41,276.47) --
	(108.40,276.50) --
	(108.40,276.52) --
	(108.37,276.57) --
	(108.35,276.61) --
	(108.35,276.62) --
	(108.35,276.65) --
	(108.35,276.67) --
	(108.37,276.74) --
	(108.39,276.76) --
	(108.43,276.86) --
	(108.42,276.90) --
	(108.41,276.93) --
	(108.39,276.99) --
	(108.38,277.01) --
	(108.38,277.04) --
	(108.38,277.09) --
	(108.37,277.12) --
	(108.37,277.17) --
	(108.38,277.20) --
	(108.39,277.23) --
	(108.39,277.26) --
	(108.39,277.30) --
	(108.36,277.32) --
	(108.35,277.35) --
	(108.36,277.38) --
	(108.41,277.48) --
	(108.43,277.50) --
	(108.43,277.51) --
	(108.42,277.58) --
	(108.43,277.61) --
	(108.45,277.67) --
	(108.46,277.72) --
	(108.50,277.79) --
	(108.54,277.84) --
	(108.58,277.93) --
	(108.57,277.96) --
	(108.56,278.00) --
	(108.54,278.04) --
	(108.53,278.06) --
	(108.53,278.08) --
	(108.50,278.11) --
	(108.49,278.12) --
	(108.53,278.15) --
	(108.55,278.16) --
	(108.57,278.17) --
	(108.58,278.18) --
	(108.59,278.21) --
	(108.61,278.22) --
	(108.66,278.24) --
	(108.72,278.24) --
	(108.76,278.26) --
	(108.84,278.28) --
	(108.89,278.28) --
	(108.91,278.26) --
	(108.94,278.26) --
	(108.95,278.26) --
	(108.96,278.26) --
	(108.99,278.26) --
	(109.10,278.23) --
	(109.12,278.24) --
	(109.16,278.25) --
	(109.19,278.28) --
	(109.21,278.32) --
	(109.22,278.33) --
	(109.23,278.35) --
	(109.25,278.42) --
	(109.25,278.46) --
	(109.24,278.49) --
	(109.24,278.53) --
	(109.23,278.62) --
	(109.20,278.68) --
	(109.19,278.71) --
	(109.14,278.74) --
	(109.13,278.77) --
	(109.12,278.78) --
	(109.11,278.80) --
	(109.11,278.89) --
	(109.12,278.95) --
	(109.11,278.97) --
	(109.07,279.03) --
	(109.00,279.09) --
	(108.98,279.09) --
	(108.97,279.10) --
	(108.96,279.11) --
	(108.95,279.12) --
	(108.94,279.14) --
	(108.92,279.16) --
	(108.91,279.18) --
	(108.84,279.27) --
	(108.83,279.29) --
	(108.82,279.35) --
	(108.81,279.36) --
	(108.80,279.40) --
	(108.79,279.42) --
	(108.78,279.43) --
	(108.76,279.44) --
	(108.73,279.47) --
	(108.70,279.49) --
	(108.70,279.51) --
	(108.69,279.51) --
	(108.60,279.53) --
	(108.57,279.54) --
	(108.55,279.54) --
	(108.50,279.55) --
	(108.43,279.56) --
	(108.33,279.58) --
	(108.21,279.60) --
	(108.13,279.62) --
	(107.91,279.67) --
	(107.80,279.70) --
	(107.73,279.71) --
	(107.70,279.70) --
	(107.68,279.71) --
	(107.65,279.71) --
	(107.58,279.72) --
	(107.48,279.73) --
	(107.43,279.73) --
	(107.37,279.74) --
	(107.30,279.75) --
	(107.27,279.76) --
	(107.20,279.78) --
	(107.15,279.81) --
	(107.11,279.85) --
	(107.09,279.88) --
	(107.08,279.91) --
	(107.07,279.93) --
	(107.06,279.98) --
	(107.03,280.07) --
	(107.03,280.11) --
	(106.99,280.21) --
	(106.97,280.24) --
	(106.96,280.27) --
	(106.95,280.30) --
	(106.92,280.32) --
	(106.90,280.34) --
	(106.87,280.37) --
	(106.86,280.38) --
	(106.81,280.44) --
	(106.81,280.47) --
	(106.80,280.51) --
	(106.78,280.54) --
	(106.75,280.59) --
	(106.72,280.62) --
	(106.70,280.66) --
	(106.70,280.69) --
	(106.70,280.73) --
	(106.74,280.79) --
	(106.74,280.82) --
	(106.75,280.85) --
	(106.77,280.93) --
	(106.80,280.97) --
	(106.82,280.98) --
	(106.85,281.01) --
	(106.90,281.04) --
	(106.92,281.04) --
	(106.96,281.05) --
	(106.98,281.07) --
	(107.01,281.12) --
	(107.03,281.15) --
	(107.05,281.17) --
	(107.08,281.19) --
	(107.11,281.24) --
	(107.15,281.26) --
	(107.19,281.29) --
	(107.21,281.31) --
	(107.28,281.37) --
	(107.33,281.40) --
	(107.42,281.45) --
	(107.45,281.46) --
	(107.48,281.48) --
	(107.59,281.51) --
	(107.63,281.51) --
	(107.70,281.52) --
	(107.73,281.52) --
	(107.75,281.52) --
	(107.79,281.50) --
	(107.83,281.49) --
	(107.87,281.48) --
	(107.91,281.48) --
	(108.10,281.49) --
	(108.18,281.50) --
	(108.27,281.52) --
	(108.32,281.54) --
	(108.36,281.56) --
	(108.45,281.63) --
	(108.49,281.65) --
	(108.50,281.67) --
	(108.51,281.68) --
	(108.52,281.70) --
	(108.51,281.76) --
	(108.51,281.77) --
	(108.50,281.78) --
	(108.48,281.83) --
	(108.48,281.85) --
	(108.49,281.86) --
	(108.50,281.87) --
	(108.51,281.88) --
	(108.55,281.90) --
	(108.60,281.90) --
	(108.63,281.90) --
	(108.71,281.93) --
	(108.75,281.95) --
	(108.81,281.99) --
	(108.90,282.04) --
	(108.92,282.05) --
	(108.95,282.06) --
	(108.98,282.07) --
	(108.99,282.09) --
	(109.01,282.10) --
	(109.02,282.11) --
	(109.06,282.14) --
	(109.09,282.15) --
	(109.17,282.20) --
	(109.30,282.21) --
	(109.36,282.24) --
	(109.43,282.26) --
	(109.50,282.30) --
	(109.56,282.34) --
	(109.59,282.38) --
	(109.64,282.43) --
	(109.66,282.45) --
	(109.68,282.47) --
	(109.72,282.50) --
	(109.73,282.51) --
	(109.76,282.52) --
	(109.77,282.53) --
	(109.83,282.57) --
	(109.85,282.59) --
	(109.90,282.67) --
	(109.97,282.71) --
	(110.00,282.72) --
	(110.03,282.73) --
	(110.07,282.76) --
	(110.08,282.77) --
	(110.11,282.79) --
	(110.12,282.79) --
	(110.15,282.80) --
	(110.18,282.79) --
	(110.23,282.75) --
	(110.25,282.74) --
	(110.30,282.71) --
	(110.31,282.70) --
	(110.34,282.68) --
	(110.37,282.68) --
	(110.40,282.67) --
	(110.41,282.67) --
	(110.43,282.67) --
	(110.44,282.67) --
	(110.47,282.69) --
	(110.49,282.70) --
	(110.51,282.71) --
	(110.56,282.73) --
	(110.59,282.74) --
	(110.63,282.74) --
	(110.67,282.73) --
	(110.78,282.67) --
	(110.79,282.66) --
	(110.82,282.64) --
	(110.85,282.67) --
	(110.92,282.70) --
	(111.02,282.74) --
	(111.04,282.75) --
	(111.06,282.75) --
	(111.08,282.76) --
	(111.16,282.77) --
	(111.17,282.77) --
	(111.19,282.79) --
	(111.20,282.80) --
	(111.20,282.82) --
	(111.20,282.85) --
	(111.18,282.88) --
	(111.15,282.92) --
	(111.09,282.98) --
	(111.06,283.00) --
	(110.99,283.08) --
	(110.95,283.13) --
	(110.94,283.16) --
	(110.92,283.18) --
	(110.91,283.21) --
	(110.90,283.23) --
	(110.90,283.25) --
	(110.89,283.27) --
	(110.85,283.35) --
	(110.84,283.36) --
	(110.83,283.38) --
	(110.83,283.41) --
	(110.84,283.43) --
	(110.86,283.46) --
	(110.89,283.50) --
	(110.90,283.51) --
	(111.01,283.59) --
	(111.06,283.62) --
	(111.09,283.63) --
	(111.14,283.65) --
	(111.16,283.67) --
	(111.16,283.67) --
	(111.24,283.73) --
	(111.29,283.76) --
	(111.44,283.81) --
	(111.46,283.82) --
	(111.47,283.83) --
	(111.51,283.85) --
	(111.53,283.86) --
	(111.60,283.88) --
	(111.62,283.89) --
	(111.63,283.90) --
	(111.66,283.92) --
	(111.69,283.93) --
	(111.71,283.95) --
	(111.73,283.96) --
	(111.79,284.01) --
	(111.83,284.03) --
	(111.86,284.04) --
	(111.92,284.05) --
	(111.95,284.06) --
	(111.98,284.07) --
	(112.02,284.08) --
	(112.07,284.08) --
	(112.12,284.08) --
	(112.12,284.08);
\end{scope}
\begin{scope}
\path[clip] (  0.00,  0.00) rectangle (252.94,361.35);
\definecolor[named]{drawColor}{rgb}{0.00,0.00,0.00}

\node[color=drawColor,anchor=base,inner sep=0pt, outer sep=0pt, scale=  1.00] at (138.47, 13.20) {Longitude%
};

\node[rotate= 90.00,color=drawColor,anchor=base,inner sep=0pt, outer sep=0pt, scale=  1.00] at ( 13.20,186.67) {Latitude%
};
\end{scope}
\begin{scope}
\path[clip] ( 49.20, 61.20) rectangle (227.75,312.15);
\definecolor[named]{drawColor}{rgb}{0.68,0.85,0.90}

\draw[color=drawColor,line cap=round,line join=round,fill opacity=0.00,] (111.27,260.74) --
	(111.32,260.60) --
	(111.34,260.45) --
	(111.36,260.29) --
	(111.38,260.23) --
	(111.41,260.16) --
	(111.48,260.11) --
	(111.51,260.01) --
	(111.50,259.87) --
	(111.47,259.69) --
	(111.46,259.62) --
	(111.40,259.53) --
	(111.29,259.30) --
	(111.17,259.18) --
	(110.99,258.92) --
	(110.89,258.70) --
	(110.79,258.46) --
	(110.67,258.38) --
	(110.48,257.99) --
	(110.49,257.36) --
	(110.19,257.02) --
	(110.13,256.91) --
	(110.03,256.79) --
	(109.77,256.59) --
	(109.73,256.54) --
	(109.73,256.51) --
	(109.70,256.44) --
	(109.53,256.11) --
	(109.44,256.00) --
	(109.43,255.95) --
	(109.32,255.85) --
	(109.25,255.67) --
	(109.13,255.50) --
	(109.00,255.45) --
	(108.88,255.40) --
	(108.83,255.37) --
	(108.79,255.33) --
	(108.72,255.21) --
	(108.66,255.12) --
	(108.65,255.01) --
	(108.66,254.91) --
	(108.70,254.70) --
	(108.62,254.45) --
	(108.85,253.98) --
	(109.12,253.93) --
	(109.30,253.62) --
	(109.22,253.38);

\draw[color=drawColor,line cap=round,line join=round,fill opacity=0.00,] (111.48,260.94) --
	(111.56,261.03) --
	(111.85,261.10) --
	(112.49,261.20) --
	(113.20,261.07) --
	(113.27,261.14) --
	(113.30,261.26) --
	(113.23,261.38) --
	(113.41,261.75) --
	(113.72,261.74) --
	(113.94,261.55) --
	(114.08,261.51) --
	(114.17,261.50) --
	(114.24,261.49) --
	(114.62,261.81) --
	(114.74,261.97) --
	(114.77,262.07) --
	(114.75,262.09) --
	(114.72,262.12) --
	(114.70,262.16) --
	(114.64,262.27) --
	(114.65,262.52) --
	(114.64,262.57) --
	(114.58,262.72) --
	(114.52,262.76) --
	(114.53,262.79) --
	(114.50,262.89) --
	(114.46,262.96) --
	(114.46,263.06) --
	(114.47,263.20) --
	(114.45,263.43) --
	(114.37,263.55) --
	(114.30,263.77) --
	(114.24,263.92) --
	(114.13,263.99) --
	(114.08,264.17) --
	(114.03,264.27) --
	(113.90,264.31) --
	(113.73,264.44) --
	(113.62,264.55) --
	(113.63,264.68) --
	(113.66,264.80) --
	(113.70,265.05) --
	(113.76,265.14) --
	(113.92,265.39) --
	(114.01,265.52) --
	(114.09,265.75) --
	(114.15,265.87) --
	(114.21,266.06) --
	(114.19,266.18) --
	(114.10,266.44) --
	(114.02,266.58) --
	(113.75,267.40) --
	(113.77,267.51) --
	(113.80,267.58) --
	(113.76,267.88) --
	(113.77,267.97) --
	(113.78,268.18) --
	(113.78,268.20) --
	(113.87,268.29) --
	(113.93,268.39) --
	(113.92,268.49) --
	(113.92,268.59) --
	(113.98,268.70) --
	(113.94,268.83) --
	(113.81,269.07) --
	(113.82,269.46) --
	(113.60,269.59) --
	(113.23,269.64) --
	(113.10,269.71) --
	(112.71,269.74) --
	(112.38,269.76) --
	(112.31,269.71) --
	(112.30,269.69) --
	(112.18,269.63) --
	(112.13,269.63) --
	(111.89,269.62) --
	(111.51,269.64) --
	(111.25,269.73) --
	(111.08,269.82) --
	(111.00,269.88) --
	(110.92,269.96) --
	(110.73,270.04) --
	(110.61,270.07) --
	(110.39,270.13) --
	(110.04,270.28) --
	(109.75,270.32) --
	(109.66,270.38) --
	(109.55,270.44) --
	(109.50,270.53) --
	(109.45,270.59) --
	(109.48,270.68) --
	(109.52,270.78) --
	(109.55,270.87) --
	(109.56,270.95) --
	(109.56,271.04) --
	(109.61,271.21) --
	(109.69,271.30) --
	(109.81,271.41) --
	(109.80,271.55) --
	(109.82,271.63) --
	(109.97,271.74) --
	(110.03,271.82) --
	(110.05,271.95) --
	(110.02,272.09) --
	(110.02,272.23) --
	(110.06,272.30) --
	(110.13,272.42) --
	(110.10,272.56) --
	(109.99,272.70) --
	(109.92,272.75) --
	(109.93,272.86) --
	(109.99,273.01) --
	(110.04,273.06) --
	(110.19,273.15) --
	(110.39,273.27) --
	(110.46,273.36) --
	(110.60,273.48) --
	(110.76,273.60) --
	(110.90,273.74) --
	(110.95,273.88) --
	(111.01,274.02) --
	(111.52,274.74) --
	(111.53,274.81) --
	(111.86,275.72) --
	(111.44,275.80) --
	(111.53,275.96) --
	(111.57,276.02) --
	(111.95,276.34) --
	(112.51,276.63) --
	(112.79,276.78) --
	(112.96,277.36) --
	(113.05,277.84) --
	(113.00,277.95) --
	(112.93,278.08) --
	(112.87,278.19) --
	(112.80,278.34) --
	(112.78,278.46) --
	(112.67,278.74) --
	(112.63,278.91) --
	(112.66,278.96) --
	(112.69,279.12) --
	(112.68,279.30) --
	(112.59,279.63) --
	(112.65,279.83) --
	(112.77,280.01) --
	(112.95,280.17) --
	(112.92,280.34) --
	(112.90,280.40) --
	(112.89,280.62) --
	(112.90,280.71) --
	(112.97,281.08) --
	(113.05,281.44) --
	(113.30,281.66) --
	(113.77,281.84) --
	(114.19,281.87) --
	(114.77,281.79) --
	(115.18,281.59) --
	(115.51,281.49) --
	(115.96,281.36) --
	(116.31,281.20) --
	(116.55,281.01) --
	(116.83,280.41);

\draw[color=drawColor,line cap=round,line join=round,fill opacity=0.00,] (119.68,277.55) --
	(119.13,278.22) --
	(118.68,278.53) --
	(118.30,278.64) --
	(118.06,278.83) --
	(117.77,278.92) --
	(117.64,279.14) --
	(117.54,279.37) --
	(117.58,279.52) --
	(117.40,279.81);

\draw[color=drawColor,line cap=round,line join=round,fill opacity=0.00,] (111.48,260.94) --
	(111.39,260.84) --
	(111.36,260.81) --
	(111.27,260.74);

\draw[color=drawColor,line cap=round,line join=round,fill opacity=0.00,] (108.74,252.03) --
	(108.69,252.07) --
	(108.58,252.14) --
	(108.55,252.20) --
	(108.55,252.29) --
	(108.65,252.32) --
	(109.03,252.52) --
	(109.22,252.68) --
	(109.28,252.75) --
	(109.39,253.17) --
	(109.22,253.38);

\draw[color=drawColor,line cap=round,line join=round,fill opacity=0.00,] (128.16,235.01) --
	(128.07,235.08) --
	(128.00,235.16) --
	(127.87,235.22) --
	(127.71,235.20) --
	(127.51,235.42) --
	(127.31,235.55) --
	(127.14,235.50) --
	(126.99,235.46) --
	(126.90,235.50) --
	(126.84,235.54) --
	(126.89,235.83) --
	(126.80,236.13) --
	(126.76,236.26) --
	(126.63,236.28) --
	(126.35,236.41) --
	(126.05,236.60) --
	(125.92,236.82) --
	(125.69,236.99) --
	(125.26,237.05) --
	(124.93,237.11) --
	(124.76,237.35) --
	(124.59,237.61) --
	(124.20,237.87) --
	(124.11,238.03) --
	(124.10,238.24) --
	(123.66,238.51) --
	(123.12,238.68) --
	(123.21,238.95) --
	(123.20,239.11) --
	(122.99,239.34) --
	(122.59,239.43) --
	(122.36,239.47) --
	(122.17,239.41) --
	(121.99,239.36) --
	(121.66,239.46) --
	(121.66,239.62) --
	(121.72,239.73) --
	(121.82,239.93) --
	(121.92,240.03) --
	(121.94,240.12) --
	(121.79,240.14) --
	(121.65,240.15) --
	(121.50,240.21) --
	(121.43,240.30) --
	(121.37,240.34) --
	(121.25,240.38) --
	(121.16,240.45) --
	(121.06,240.55) --
	(120.91,240.60) --
	(120.77,240.59) --
	(120.56,240.65) --
	(120.44,240.75) --
	(120.30,240.91) --
	(120.14,241.15) --
	(120.00,241.35) --
	(119.90,241.49) --
	(119.84,241.60) --
	(119.78,241.95) --
	(119.75,242.15) --
	(119.68,242.38) --
	(119.57,242.54) --
	(119.43,242.68) --
	(119.32,242.75) --
	(119.14,242.80) --
	(118.99,242.87) --
	(118.77,242.87) --
	(118.58,242.88) --
	(118.42,242.95) --
	(118.27,243.09) --
	(118.17,243.30) --
	(118.17,243.45) --
	(118.21,243.54) --
	(118.22,243.79) --
	(118.24,244.05) --
	(118.26,244.17) --
	(118.39,244.22) --
	(118.26,244.42) --
	(118.14,244.60) --
	(118.15,244.76) --
	(118.12,244.87) --
	(118.10,244.98) --
	(117.89,245.18) --
	(117.65,245.36) --
	(117.54,245.45) --
	(117.50,245.57) --
	(117.38,245.67) --
	(116.97,245.83) --
	(116.44,246.15) --
	(116.25,246.33) --
	(116.07,246.38) --
	(115.74,246.21) --
	(115.49,246.19) --
	(115.36,246.12) --
	(115.28,246.10) --
	(115.21,246.09) --
	(115.17,246.24) --
	(115.17,246.33) --
	(115.11,246.38) --
	(114.96,246.42) --
	(114.71,246.37) --
	(114.55,246.39) --
	(114.36,246.47) --
	(114.18,246.58) --
	(113.87,246.76) --
	(113.82,246.81) --
	(114.01,247.00) --
	(113.92,247.15) --
	(113.82,247.25) --
	(113.67,247.27) --
	(113.38,247.14) --
	(113.21,247.18) --
	(113.03,247.24) --
	(112.75,247.35) --
	(112.58,247.52) --
	(112.40,247.54) --
	(112.12,247.50) --
	(112.00,247.57) --
	(112.04,247.77) --
	(111.87,247.84) --
	(111.87,247.94) --
	(111.86,248.03) --
	(112.06,248.14) --
	(112.02,248.39) --
	(112.03,248.51) --
	(111.99,248.59) --
	(111.99,248.59) --
	(111.76,248.62) --
	(111.69,248.75);

\draw[color=drawColor,line cap=round,line join=round,fill opacity=0.00,] (151.22,209.94) --
	(151.36,210.03) --
	(151.51,210.07) --
	(151.57,210.16) --
	(151.65,210.25) --
	(151.66,210.42) --
	(151.84,210.53) --
	(151.92,210.63) --
	(151.91,210.68) --
	(151.85,210.75) --
	(151.86,210.87) --
	(151.93,210.92) --
	(152.02,210.88) --
	(152.06,210.93) --
	(152.01,211.05) --
	(152.08,211.27) --
	(152.09,211.33) --
	(152.03,211.38) --
	(151.91,211.46) --
	(151.88,211.54) --
	(151.96,211.66) --
	(152.07,211.72) --
	(152.02,211.84) --
	(152.00,211.89) --
	(152.06,212.03) --
	(152.01,212.13) --
	(152.05,212.24) --
	(152.05,212.24) --
	(152.03,212.40) --
	(152.04,212.54) --
	(152.13,212.59) --
	(152.26,212.60) --
	(152.33,212.62) --
	(152.36,212.73) --
	(152.40,212.78) --
	(152.55,212.83) --
	(152.61,212.92) --
	(152.74,212.95) --
	(152.87,212.98) --
	(152.96,213.10) --
	(153.00,213.23) --
	(153.06,213.25) --
	(153.24,213.26) --
	(153.34,213.30) --
	(153.45,213.35) --
	(153.52,213.34) --
	(153.54,213.27) --
	(153.60,213.25) --
	(153.71,213.32) --
	(153.92,213.36) --
	(154.02,213.45) --
	(154.20,213.45) --
	(154.26,213.52) --
	(154.21,213.59) --
	(154.11,213.71) --
	(154.19,213.77) --
	(154.36,213.73) --
	(154.45,213.74) --
	(154.49,213.88) --
	(154.58,213.92) --
	(154.76,214.11) --
	(154.91,214.15) --
	(155.06,214.29) --
	(155.07,214.37) --
	(155.08,214.51) --
	(155.24,214.72) --
	(155.43,214.80) --
	(155.62,214.81) --
	(155.68,214.89) --
	(155.66,215.01) --
	(155.79,215.09) --
	(155.97,215.05) --
	(156.08,215.11) --
	(156.15,215.20) --
	(156.30,215.24) --
	(156.36,215.28) --
	(156.35,215.43) --
	(156.56,215.49) --
	(156.64,215.64) --
	(156.78,215.65) --
	(156.82,215.85) --
	(156.86,215.92) --
	(157.04,215.88) --
	(157.13,215.90) --
	(157.23,216.03) --
	(157.29,216.13) --
	(157.25,216.26) --
	(157.33,216.30) --
	(157.48,216.29) --
	(157.50,216.31) --
	(157.44,216.39) --
	(157.46,216.44) --
	(157.55,216.46) --
	(157.66,216.49) --
	(157.63,216.60) --
	(157.66,216.68) --
	(157.63,216.75) --
	(157.62,216.84) --
	(157.70,216.92) --
	(157.76,216.90) --
	(157.89,216.93) --
	(157.84,217.03) --
	(157.73,217.12) --
	(157.75,217.21) --
	(157.92,217.26) --
	(157.95,217.33) --
	(157.92,217.43) --
	(157.98,217.52) --
	(157.99,217.65) --
	(158.04,217.82) --
	(158.11,217.93) --
	(158.19,217.93) --
	(158.28,217.89) --
	(158.32,217.94) --
	(158.29,218.08) --
	(158.33,218.14) --
	(158.41,218.14) --
	(158.47,218.08) --
	(158.54,218.03) --
	(158.62,218.06) --
	(158.60,218.13) --
	(158.63,218.18) --
	(158.72,218.20) --
	(158.82,218.31) --
	(159.00,218.36) --
	(159.04,218.41) --
	(158.98,218.45) --
	(158.86,218.47) --
	(158.84,218.52) --
	(158.84,218.56) --
	(158.98,218.64) --
	(158.91,218.70) --
	(158.93,218.76) --
	(158.99,218.81) --
	(159.03,218.87) --
	(159.01,218.97) --
	(158.94,218.99) --
	(158.89,219.04) --
	(158.90,219.12) --
	(158.88,219.17) --
	(158.84,219.21) --
	(158.72,219.22) --
	(158.70,219.34) --
	(158.74,219.44) --
	(158.82,219.46) --
	(158.84,219.51) --
	(158.77,219.57) --
	(158.75,219.63) --
	(158.82,219.68) --
	(158.85,219.85) --
	(159.07,219.96) --
	(159.11,220.08) --
	(159.08,220.17) --
	(159.21,220.28) --
	(159.21,220.34) --
	(159.12,220.41) --
	(159.14,220.50) --
	(159.25,220.49) --
	(159.35,220.58) --
	(159.43,220.57) --
	(159.50,220.63) --
	(159.51,220.81) --
	(159.65,221.00) --
	(159.76,220.98) --
	(159.82,220.94) --
	(159.87,220.84) --
	(159.97,220.85) --
	(160.06,221.03) --
	(160.17,221.20) --
	(160.26,221.43) --
	(160.36,221.83) --
	(160.53,221.92) --
	(160.83,221.91) --
	(160.96,221.95) --
	(161.02,222.11) --
	(161.15,222.14) --
	(161.27,222.19) --
	(161.45,222.19) --
	(161.56,222.06) --
	(161.83,222.02) --
	(162.04,222.12) --
	(162.07,222.16) --
	(162.15,222.45) --
	(162.22,222.48) --
	(162.32,222.60) --
	(162.35,222.74) --
	(162.44,222.78) --
	(162.50,222.79) --
	(162.63,222.74) --
	(162.68,222.79) --
	(162.69,222.91) --
	(162.73,222.94) --
	(162.86,222.92) --
	(162.90,222.98) --
	(162.87,223.07) --
	(162.92,223.11) --
	(163.00,223.14) --
	(163.09,223.11) --
	(163.13,223.14) --
	(163.19,223.15) --
	(163.24,223.13) --
	(163.30,223.10) --
	(163.45,223.11) --
	(163.53,223.12) --
	(163.55,223.18) --
	(163.60,223.16) --
	(163.61,223.14) --
	(163.62,223.06) --
	(163.66,223.05) --
	(163.72,223.14) --
	(163.76,223.15) --
	(163.84,223.08) --
	(163.92,223.07) --
	(164.01,223.12) --
	(164.15,223.28) --
	(164.30,223.24) --
	(164.36,223.24) --
	(164.38,223.28) --
	(164.33,223.37) --
	(164.33,223.41) --
	(164.32,223.46) --
	(164.31,223.49) --
	(164.34,223.51) --
	(164.52,223.65) --
	(164.61,223.72) --
	(165.14,223.81) --
	(165.21,223.94) --
	(165.42,223.97) --
	(165.57,224.03) --
	(165.81,224.05) --
	(166.01,223.99) --
	(166.36,223.95) --
	(166.69,223.85) --
	(166.96,223.71) --
	(167.29,223.55) --
	(167.30,223.50) --
	(167.30,223.48) --
	(167.30,223.46) --
	(167.30,223.42) --
	(167.32,223.36) --
	(167.42,223.38) --
	(167.48,223.35) --
	(167.52,223.28) --
	(167.56,223.30) --
	(167.63,223.39) --
	(167.68,223.39) --
	(167.74,223.34) --
	(167.77,223.28) --
	(167.91,223.28) --
	(167.98,223.23) --
	(168.06,223.22) --
	(168.12,223.30) --
	(168.18,223.32) --
	(168.25,223.27) --
	(168.30,223.19) --
	(168.26,223.10) --
	(168.24,223.06) --
	(168.30,223.04) --
	(168.41,223.07) --
	(168.65,223.11) --
	(168.75,223.09) --
	(168.89,223.09) --
	(168.94,223.20) --
	(169.00,223.23) --
	(169.09,223.19) --
	(169.21,223.24) --
	(169.29,223.25) --
	(169.38,223.21) --
	(169.44,223.25) --
	(169.45,223.36) --
	(169.47,223.46) --
	(169.48,223.49) --
	(169.49,223.50) --
	(169.55,223.56) --
	(169.73,223.57) --
	(169.76,223.63) --
	(169.79,223.70) --
	(169.87,223.71) --
	(170.03,223.63) --
	(170.05,223.70) --
	(170.07,223.81) --
	(170.02,223.86) --
	(170.01,223.95) --
	(170.05,224.03) --
	(170.19,224.13) --
	(170.28,224.19) --
	(170.47,224.38) --
	(171.03,224.39) --
	(171.15,224.38) --
	(171.25,224.36) --
	(171.40,224.37) --
	(171.51,224.33) --
	(171.63,224.32) --
	(171.78,224.30) --
	(171.93,224.22) --
	(172.05,224.15) --
	(172.17,224.14) --
	(172.32,224.15) --
	(172.46,224.22) --
	(172.65,224.26) --
	(172.83,224.35) --
	(173.09,224.36) --
	(173.25,224.36) --
	(173.46,224.30) --
	(173.58,224.15) --
	(173.59,224.14) --
	(173.94,224.06) --
	(174.29,224.10) --
	(175.05,224.03) --
	(175.58,223.74) --
	(175.77,223.73) --
	(176.00,223.76) --
	(176.15,223.78) --
	(176.21,223.80) --
	(176.64,223.96) --
	(177.03,223.86) --
	(177.21,223.75) --
	(177.33,223.61) --
	(177.41,223.57) --
	(177.51,223.51) --
	(177.51,223.51) --
	(177.54,223.47) --
	(177.55,223.46) --
	(177.71,223.33) --
	(178.04,223.40) --
	(178.36,223.40) --
	(178.54,223.41) --
	(178.79,223.38) --
	(178.90,223.33) --
	(178.97,223.27) --
	(179.06,223.27) --
	(179.15,223.36) --
	(179.29,223.43) --
	(179.50,223.44) --
	(179.63,223.40) --
	(179.86,223.38) --
	(180.16,223.38) --
	(180.32,223.35) --
	(180.58,223.41) --
	(180.76,223.42) --
	(181.31,223.36) --
	(181.78,223.42) --
	(182.15,223.41) --
	(182.43,223.32) --
	(182.73,223.20) --
	(183.06,223.23) --
	(183.57,223.35) --
	(183.88,223.39) --
	(184.42,223.41) --
	(185.05,223.42) --
	(185.62,223.41) --
	(185.83,223.32) --
	(186.06,223.30);

\draw[color=drawColor,line cap=round,line join=round,fill opacity=0.00,] (138.29,211.80) --
	(138.27,211.94) --
	(138.29,212.10) --
	(138.38,212.15) --
	(138.50,212.15) --
	(138.72,212.20) --
	(138.81,212.33) --
	(138.85,212.45) --
	(138.76,212.61) --
	(138.66,212.68) --
	(138.49,212.82) --
	(138.39,212.94) --
	(138.31,213.11) --
	(138.29,213.22) --
	(138.29,213.31) --
	(138.37,213.40) --
	(138.50,213.46) --
	(138.71,213.47) --
	(138.82,213.45) --
	(138.91,213.48) --
	(139.09,213.67) --
	(139.24,213.78) --
	(139.23,213.86) --
	(139.21,214.01) --
	(139.33,214.36) --
	(139.49,214.50) --
	(139.56,214.66) --
	(139.66,214.86) --
	(140.02,215.12) --
	(140.23,215.15) --
	(140.33,215.24) --
	(140.28,215.46) --
	(140.33,215.95) --
	(140.50,216.29) --
	(140.64,216.77) --
	(140.73,216.88) --
	(140.75,216.98) --
	(140.67,217.22) --
	(140.75,217.36) --
	(140.72,217.52) --
	(140.61,217.51) --
	(140.49,217.42) --
	(140.42,217.39) --
	(140.34,217.43) --
	(140.33,217.52) --
	(140.29,217.60) --
	(140.24,217.72) --
	(140.28,217.80) --
	(140.43,217.85) --
	(140.49,217.88) --
	(140.44,217.95) --
	(140.38,217.98) --
	(140.28,217.98) --
	(140.13,218.02) --
	(139.97,218.10) --
	(139.91,218.17) --
	(139.89,218.26) --
	(139.90,218.44) --
	(139.92,218.58) --
	(139.89,218.61) --
	(139.80,218.64) --
	(139.65,218.64) --
	(139.53,218.60) --
	(139.43,218.46) --
	(139.38,218.42) --
	(139.21,218.41) --
	(139.12,218.43) --
	(139.12,218.49) --
	(139.20,218.58) --
	(139.20,218.66) --
	(139.17,218.73) --
	(139.09,218.80) --
	(138.85,218.85) --
	(138.71,218.93) --
	(138.64,219.06) --
	(138.53,219.18) --
	(138.38,219.25) --
	(138.23,219.28) --
	(138.04,219.25) --
	(137.70,219.15) --
	(137.53,219.09) --
	(137.31,219.10) --
	(137.04,219.19) --
	(136.81,219.29) --
	(136.70,219.31) --
	(136.68,219.31) --
	(136.64,219.40) --
	(136.60,219.50) --
	(136.47,219.61) --
	(136.43,219.66) --
	(136.37,219.71) --
	(136.23,219.73) --
	(136.16,219.72) --
	(136.10,219.69) --
	(136.07,219.67) --
	(136.02,219.65) --
	(135.85,219.54) --
	(135.60,219.54) --
	(135.28,219.52) --
	(134.99,219.51) --
	(134.80,219.47) --
	(134.78,219.48) --
	(134.75,219.49) --
	(134.74,219.51) --
	(134.74,219.55) --
	(134.78,219.65) --
	(134.81,219.72) --
	(134.80,219.77) --
	(134.63,219.85) --
	(134.58,219.87) --
	(134.56,219.90) --
	(134.55,219.95) --
	(134.52,220.01) --
	(134.46,220.06) --
	(134.29,220.18) --
	(134.29,220.21) --
	(134.29,220.23) --
	(134.32,220.24) --
	(134.37,220.25) --
	(134.45,220.20) --
	(134.53,220.21) --
	(134.62,220.24) --
	(134.65,220.29) --
	(134.65,220.31) --
	(134.65,220.33) --
	(134.64,220.37) --
	(134.60,220.39) --
	(134.33,220.46) --
	(134.26,220.50) --
	(134.17,220.58) --
	(134.15,220.59) --
	(134.14,220.62) --
	(134.16,220.63) --
	(134.16,220.66) --
	(134.21,220.74) --
	(134.28,220.83) --
	(134.32,220.87) --
	(134.32,220.91) --
	(134.32,220.93) --
	(134.27,220.96) --
	(134.23,220.97) --
	(134.20,220.96) --
	(134.15,220.95) --
	(134.08,220.89) --
	(134.01,220.86) --
	(133.78,220.88) --
	(133.67,220.87) --
	(133.47,220.83) --
	(133.25,220.86) --
	(133.07,220.79) --
	(132.91,220.73) --
	(132.82,220.67) --
	(132.78,220.65) --
	(132.72,220.63) --
	(132.66,220.64) --
	(132.37,220.64) --
	(132.34,220.65) --
	(132.25,220.70) --
	(132.07,220.70) --
	(131.94,220.74) --
	(131.88,220.76) --
	(131.84,220.79) --
	(131.81,220.84) --
	(131.81,220.93) --
	(131.76,221.05) --
	(131.64,221.10) --
	(131.46,221.16) --
	(131.33,221.19) --
	(131.15,221.21) --
	(130.94,221.22) --
	(130.82,221.24) --
	(130.67,221.25) --
	(130.56,221.25) --
	(130.44,221.23) --
	(130.23,221.22) --
	(130.10,221.20) --
	(130.04,221.16) --
	(130.01,221.14) --
	(129.95,221.15) --
	(129.77,221.21) --
	(129.67,221.28) --
	(129.62,221.28) --
	(129.58,221.29) --
	(129.53,221.28) --
	(129.47,221.26) --
	(129.35,221.17) --
	(129.24,221.09) --
	(129.19,221.01) --
	(129.02,220.90) --
	(128.99,220.88) --
	(128.96,220.87) --
	(128.92,220.87) --
	(128.86,220.87) --
	(128.83,220.88) --
	(128.81,220.91) --
	(128.77,221.00) --
	(128.76,221.03) --
	(128.71,221.05) --
	(128.68,221.08) --
	(128.58,221.05) --
	(128.47,221.02) --
	(128.29,220.96) --
	(128.23,220.95) --
	(128.14,220.98) --
	(128.07,221.01) --
	(127.93,221.05) --
	(127.75,221.10) --
	(127.58,221.13) --
	(127.53,221.14) --
	(127.49,221.19) --
	(127.45,221.38);

\draw[color=drawColor,line cap=round,line join=round,fill opacity=0.00,] (121.16,200.33) --
	(121.05,200.48) --
	(121.00,200.66) --
	(121.09,200.71) --
	(121.09,200.74) --
	(121.10,200.76) --
	(121.16,200.81) --
	(121.55,200.82) --
	(121.80,200.92) --
	(121.88,201.03) --
	(121.82,201.08) --
	(121.80,201.10) --
	(121.61,201.19) --
	(121.58,201.21) --
	(121.56,201.24) --
	(121.58,201.30) --
	(121.58,201.36) --
	(121.63,201.39) --
	(121.67,201.42) --
	(121.72,201.44) --
	(122.15,201.42) --
	(122.19,201.42) --
	(122.24,201.44) --
	(122.51,201.58) --
	(122.52,201.59) --
	(122.54,201.63) --
	(122.53,201.83) --
	(122.44,202.14) --
	(122.44,202.17) --
	(122.49,202.33) --
	(122.50,202.35) --
	(122.53,202.38) --
	(122.76,202.51) --
	(122.99,202.58) --
	(123.22,202.54) --
	(123.30,202.53) --
	(123.36,202.54) --
	(123.51,202.58) --
	(123.58,202.62) --
	(123.68,202.75) --
	(123.86,202.97) --
	(123.93,203.11) --
	(123.93,203.16) --
	(123.90,203.20) --
	(123.83,203.35) --
	(123.79,203.36) --
	(123.73,203.36) --
	(123.65,203.37) --
	(123.28,203.27) --
	(123.02,203.24) --
	(122.82,203.27) --
	(122.73,203.29) --
	(122.70,203.32) --
	(122.67,203.36) --
	(122.67,203.39) --
	(122.66,203.43) --
	(122.69,203.47) --
	(122.72,203.50) --
	(122.78,203.55) --
	(122.86,203.57) --
	(122.93,203.59) --
	(122.98,203.59) --
	(123.09,203.59) --
	(123.32,203.59) --
	(123.38,203.61) --
	(123.41,203.65) --
	(123.53,203.77) --
	(123.53,203.79) --
	(123.55,203.81) --
	(123.56,204.03) --
	(123.46,204.27) --
	(123.27,204.50) --
	(123.25,204.59) --
	(123.25,204.62) --
	(123.25,204.66) --
	(123.27,204.72) --
	(123.28,204.74) --
	(123.31,204.76) --
	(123.33,204.76) --
	(123.36,204.75) --
	(123.55,204.72) --
	(123.67,204.72) --
	(123.83,204.75) --
	(123.89,204.80) --
	(123.91,204.81) --
	(123.91,204.82) --
	(123.90,204.87) --
	(123.88,204.91) --
	(123.87,204.90) --
	(123.64,204.88) --
	(123.61,204.90) --
	(123.40,205.06) --
	(123.31,205.17) --
	(123.30,205.20) --
	(123.27,205.25) --
	(123.27,205.29) --
	(123.29,205.33) --
	(123.33,205.36) --
	(123.38,205.39) --
	(123.79,205.60) --
	(124.19,205.71) --
	(124.45,205.86) --
	(124.47,205.87) --
	(124.49,205.86) --
	(124.53,205.87) --
	(124.54,205.85) --
	(124.55,205.83) --
	(124.57,205.81) --
	(124.57,205.79) --
	(124.56,205.73) --
	(124.51,205.56) --
	(124.53,205.54) --
	(124.53,205.49) --
	(124.65,205.31) --
	(124.68,205.29) --
	(124.68,205.25) --
	(124.70,205.22) --
	(124.68,205.13) --
	(124.69,205.08) --
	(124.70,205.01) --
	(124.78,204.94) --
	(124.91,204.84) --
	(124.96,204.82) --
	(124.99,204.80) --
	(125.08,204.78) --
	(125.16,204.78) --
	(125.34,204.79) --
	(125.40,204.82) --
	(125.44,204.84) --
	(125.47,204.87) --
	(125.50,204.90) --
	(125.61,205.10) --
	(125.76,205.37) --
	(125.77,205.41) --
	(125.76,205.49) --
	(125.75,205.56) --
	(125.71,205.68) --
	(125.69,205.72) --
	(125.58,205.77) --
	(125.52,205.79) --
	(125.43,205.78) --
	(124.96,205.67) --
	(124.92,205.69) --
	(124.91,205.70) --
	(124.89,205.71) --
	(124.89,205.74) --
	(124.91,205.75) --
	(124.92,205.76) --
	(125.10,205.86) --
	(125.14,205.94) --
	(125.04,206.13) --
	(125.07,206.16) --
	(125.10,206.20) --
	(125.19,206.23) --
	(125.28,206.23) --
	(125.44,206.15) --
	(125.63,206.11) --
	(125.69,206.12) --
	(125.78,206.16) --
	(125.91,206.42) --
	(125.93,206.45) --
	(125.94,206.47) --
	(125.96,206.49) --
	(125.99,206.53) --
	(126.02,206.53) --
	(126.07,206.54) --
	(126.11,206.52) --
	(126.18,206.49) --
	(126.40,206.40) --
	(126.64,206.41) --
	(126.77,206.39) --
	(126.91,206.32) --
	(127.08,206.20) --
	(127.37,206.11) --
	(127.60,206.05) --
	(127.64,206.03) --
	(127.68,206.03) --
	(127.73,206.04) --
	(127.74,206.06) --
	(128.10,206.53) --
	(128.24,206.64) --
	(128.38,206.69) --
	(128.56,206.82) --
	(128.74,206.86) --
	(129.01,206.89) --
	(129.05,206.91) --
	(129.09,206.93) --
	(129.07,206.97) --
	(129.06,207.02) --
	(129.03,207.08) --
	(128.99,207.13) --
	(128.90,207.23) --
	(128.93,207.27) --
	(128.98,207.29) --
	(129.13,207.32) --
	(129.19,207.33) --
	(129.33,207.43) --
	(129.40,207.52) --
	(129.51,207.58) --
	(129.59,207.68) --
	(129.63,207.87) --
	(129.66,208.00) --
	(129.74,208.06) --
	(129.97,208.22) --
	(129.94,208.31) --
	(129.93,208.35) --
	(129.88,208.38) --
	(129.81,208.43) --
	(129.75,208.46) --
	(129.71,208.52) --
	(129.68,208.59) --
	(129.67,208.66) --
	(129.67,208.70) --
	(129.70,208.78) --
	(129.75,208.79) --
	(129.81,208.82) --
	(129.84,208.85) --
	(129.86,208.89) --
	(129.84,208.92) --
	(129.84,209.10) --
	(129.79,209.41) --
	(129.85,209.50) --
	(129.98,209.41) --
	(130.17,209.39) --
	(130.37,209.35) --
	(130.59,209.26) --
	(130.69,209.20) --
	(130.78,209.14) --
	(130.80,209.00) --
	(130.72,208.85) --
	(130.79,208.71) --
	(130.89,208.61) --
	(130.95,208.53) --
	(131.12,208.45) --
	(131.33,208.44) --
	(131.29,208.52) --
	(131.31,208.99) --
	(131.31,209.00) --
	(131.46,209.04) --
	(131.64,209.07) --
	(131.65,209.07) --
	(131.72,208.93) --
	(131.72,208.79) --
	(131.78,208.78) --
	(131.83,208.77) --
	(131.89,208.74) --
	(131.98,208.73) --
	(132.15,208.76) --
	(132.04,208.89) --
	(132.02,208.92) --
	(131.99,208.98) --
	(131.92,209.09) --
	(131.84,209.21) --
	(131.77,209.28) --
	(131.76,209.45) --
	(131.77,209.52) --
	(131.72,209.75) --
	(131.70,209.79) --
	(131.67,209.81) --
	(131.64,209.82) --
	(131.33,209.91) --
	(131.30,209.93) --
	(131.30,209.96) --
	(131.30,210.00) --
	(131.33,210.02) --
	(131.39,210.04) --
	(131.57,210.04) --
	(131.79,210.04) --
	(131.92,210.10) --
	(132.14,210.15) --
	(132.41,210.23) --
	(132.61,210.24) --
	(132.71,210.27) --
	(132.86,210.27) --
	(133.13,210.26) --
	(133.27,210.30) --
	(133.40,210.34) --
	(133.51,210.49) --
	(133.53,210.56) --
	(133.54,210.75) --
	(133.55,210.80) --
	(133.79,211.02) --
	(133.91,211.08) --
	(133.98,211.10) --
	(134.19,211.11) --
	(134.31,211.11) --
	(134.34,211.12) --
	(134.39,211.14) --
	(134.39,211.17) --
	(134.40,211.18) --
	(134.42,211.22) --
	(134.42,211.26) --
	(134.41,211.31) --
	(134.38,211.32) --
	(134.04,211.43) --
	(134.01,211.44) --
	(134.00,211.47) --
	(134.02,211.53) --
	(134.03,211.58) --
	(134.08,211.61) --
	(134.14,211.65) --
	(134.18,211.66) --
	(134.23,211.64) --
	(134.32,211.62) --
	(134.59,211.44) --
	(134.80,211.37) --
	(134.93,211.35) --
	(134.96,211.34) --
	(135.06,211.28) --
	(135.12,211.27) --
	(135.15,211.27) --
	(135.31,211.44) --
	(135.34,211.46) --
	(135.46,211.49) --
	(135.54,211.51) --
	(135.66,211.59) --
	(135.69,211.62) --
	(135.73,211.85) --
	(135.76,211.90) --
	(135.91,211.98) --
	(136.08,212.03) --
	(136.20,212.05) --
	(136.47,212.09) --
	(136.62,212.07) --
	(136.65,212.07) --
	(136.68,212.06) --
	(136.88,212.00) --
	(137.05,211.95) --
	(137.20,211.93) --
	(137.30,211.97) --
	(137.40,212.02) --
	(137.59,212.15) --
	(137.69,212.18) --
	(137.78,212.17) --
	(137.87,212.14) --
	(137.98,212.09) --
	(138.03,212.03) --
	(138.07,211.96) --
	(138.11,211.88) --
	(138.11,211.76) --
	(138.10,211.69) --
	(138.13,211.64) --
	(138.19,211.64) --
	(138.22,211.64) --
	(138.28,211.73) --
	(138.29,211.80);

\draw[color=drawColor,line cap=round,line join=round,fill opacity=0.00,] (138.29,211.80) --
	(138.43,211.78) --
	(138.52,211.77) --
	(138.66,211.80) --
	(138.74,211.80) --
	(138.80,211.77) --
	(138.83,211.71) --
	(138.86,211.65) --
	(138.90,211.61) --
	(138.96,211.60) --
	(139.01,211.62) --
	(139.07,211.64) --
	(139.13,211.70) --
	(139.25,211.78) --
	(139.37,211.78) --
	(139.59,211.84) --
	(139.71,211.83) --
	(139.83,211.72) --
	(139.89,211.66) --
	(139.91,211.64) --
	(139.91,211.59) --
	(139.87,211.57) --
	(139.78,211.59) --
	(139.63,211.58) --
	(139.63,211.53) --
	(139.71,211.50) --
	(139.80,211.48) --
	(139.81,211.43) --
	(139.76,211.36) --
	(139.76,211.30) --
	(139.82,211.25) --
	(139.87,211.24) --
	(139.89,211.24) --
	(139.91,211.19) --
	(139.90,211.17) --
	(139.92,211.10) --
	(139.95,211.04) --
	(140.02,211.01) --
	(140.07,211.00) --
	(140.10,210.99) --
	(140.16,210.95) --
	(140.19,210.91) --
	(140.12,210.83) --
	(140.12,210.79) --
	(140.11,210.68) --
	(140.13,210.56) --
	(140.16,210.49) --
	(140.23,210.45) --
	(140.30,210.40) --
	(140.42,210.43) --
	(140.49,210.49) --
	(140.63,210.53) --
	(140.74,210.61) --
	(140.81,210.77) --
	(140.86,210.90) --
	(140.94,210.95) --
	(141.04,210.98) --
	(141.09,210.96) --
	(141.04,210.86) --
	(141.03,210.77) --
	(141.06,210.72) --
	(141.15,210.75) --
	(141.26,210.76) --
	(141.35,210.78) --
	(141.41,210.76) --
	(141.44,210.72) --
	(141.38,210.70) --
	(141.26,210.66) --
	(141.22,210.65) --
	(141.23,210.61) --
	(141.35,210.57) --
	(141.36,210.53) --
	(141.30,210.51) --
	(141.19,210.51) --
	(141.15,210.50) --
	(141.12,210.43) --
	(141.24,210.40) --
	(141.33,210.36) --
	(141.30,210.31) --
	(141.25,210.27) --
	(141.34,210.27) --
	(141.41,210.30) --
	(141.47,210.32) --
	(141.51,210.28) --
	(141.58,210.25) --
	(141.65,210.24) --
	(141.74,210.26) --
	(141.93,210.61) --
	(141.99,210.65) --
	(142.07,210.69) --
	(142.17,210.71) --
	(142.29,210.67) --
	(142.35,210.54) --
	(142.46,210.54) --
	(142.53,210.63) --
	(142.54,210.71) --
	(142.48,210.83) --
	(142.50,210.93) --
	(142.59,211.00) --
	(142.84,211.04) --
	(143.04,211.03) --
	(143.17,211.06) --
	(143.33,211.16) --
	(143.43,211.14) --
	(143.48,211.11) --
	(143.50,211.02) --
	(143.50,210.91) --
	(143.55,210.77) --
	(143.63,210.70) --
	(143.71,210.68) --
	(143.95,210.83) --
	(144.10,210.89) --
	(144.19,210.90) --
	(144.23,210.89) --
	(144.28,210.84) --
	(144.39,210.70) --
	(144.48,210.62) --
	(144.63,210.58) --
	(144.79,210.51) --
	(144.95,210.49) --
	(145.09,210.50) --
	(145.22,210.44) --
	(145.42,210.38) --
	(145.60,210.36) --
	(145.76,210.29) --
	(145.87,210.26) --
	(146.05,210.27) --
	(146.11,210.24) --
	(146.13,210.16) --
	(146.22,210.09) --
	(146.32,210.10) --
	(146.45,210.22) --
	(146.57,210.33) --
	(146.69,210.34) --
	(146.79,210.30) --
	(146.87,210.23) --
	(146.92,210.09) --
	(147.06,210.05) --
	(147.13,209.95) --
	(147.31,209.92) --
	(147.33,209.87) --
	(147.35,209.78) --
	(147.39,209.76) --
	(147.44,209.77) --
	(147.51,209.84) --
	(147.58,209.88) --
	(147.78,209.89) --
	(147.89,209.95) --
	(148.01,210.04) --
	(148.14,210.10) --
	(148.18,210.14) --
	(148.24,210.23) --
	(148.30,210.27) --
	(148.45,210.31) --
	(148.55,210.37) --
	(148.68,210.40) --
	(148.77,210.42) --
	(148.82,210.38) --
	(148.82,210.34) --
	(148.82,210.24) --
	(148.90,210.17) --
	(149.03,210.15) --
	(149.15,210.16) --
	(149.20,210.14) --
	(149.23,210.05) --
	(149.31,209.97) --
	(149.44,209.93) --
	(149.63,209.89) --
	(149.80,209.87) --
	(149.88,209.82) --
	(149.96,209.71) --
	(150.02,209.63) --
	(150.11,209.62) --
	(150.33,209.67) --
	(150.60,209.72) --
	(150.75,209.67) --
	(150.87,209.63) --
	(150.95,209.63) --
	(150.99,209.72) --
	(151.01,209.85) --
	(151.08,209.92) --
	(151.22,209.94) --
	(151.59,209.96) --
	(151.68,210.03) --
	(151.85,210.10) --
	(152.12,210.09) --
	(152.26,210.06) --
	(152.35,209.98) --
	(152.39,210.00) --
	(152.44,210.08) --
	(152.60,210.07) --
	(152.73,210.07) --
	(152.85,210.07) --
	(152.89,210.01) --
	(152.89,209.95) --
	(153.13,209.95) --
	(153.24,210.01) --
	(153.44,210.24) --
	(153.51,210.35) --
	(153.54,210.38) --
	(153.78,210.36) --
	(153.89,210.38) --
	(154.02,210.42) --
	(154.14,210.48) --
	(154.24,210.60) --
	(154.41,210.64) --
	(154.47,210.70) --
	(154.44,210.77) --
	(154.29,210.85) --
	(154.24,210.90) --
	(154.25,210.98) --
	(154.33,210.99) --
	(154.42,210.99) --
	(154.54,210.99) --
	(154.64,211.07) --
	(154.73,211.08) --
	(154.89,211.04) --
	(154.95,211.06) --
	(154.99,211.16) --
	(155.03,211.31) --
	(155.18,211.46) --
	(155.28,211.50) --
	(155.40,211.62) --
	(155.52,211.65) --
	(155.64,211.59) --
	(155.76,211.58) --
	(155.77,211.67) --
	(155.84,211.69) --
	(155.93,211.67) --
	(156.14,211.67) --
	(156.41,211.77) --
	(156.52,211.92) --
	(156.69,212.16) --
	(156.78,212.21) --
	(156.92,212.22) --
	(157.05,212.25) --
	(157.20,212.33) --
	(157.34,212.34) --
	(157.53,212.28) --
	(157.78,212.29) --
	(158.02,212.30) --
	(158.24,212.37) --
	(158.35,212.51) --
	(158.46,212.53) --
	(158.61,212.45) --
	(158.79,212.41) --
	(158.97,212.29) --
	(159.06,212.25) --
	(159.18,212.23) --
	(159.25,212.17) --
	(159.31,212.16) --
	(159.39,212.26) --
	(159.46,212.33) --
	(159.52,212.34) --
	(159.57,212.31) --
	(159.54,212.24) --
	(159.51,212.17) --
	(159.59,212.14) --
	(159.69,212.16) --
	(159.75,212.14) --
	(159.75,212.09) --
	(159.67,211.97) --
	(159.71,211.93) --
	(159.82,211.98) --
	(159.89,212.01) --
	(159.94,211.98) --
	(159.95,211.92) --
	(160.09,211.91) --
	(160.19,211.93) --
	(160.32,212.01) --
	(160.46,212.00) --
	(160.55,211.98) --
	(160.58,211.91) --
	(160.60,211.80) --
	(160.69,211.79) --
	(160.78,211.83) --
	(160.88,211.91) --
	(161.01,211.91) --
	(160.99,211.80) --
	(160.82,211.73) --
	(160.78,211.66) --
	(160.83,211.58) --
	(160.91,211.52) --
	(160.98,211.52) --
	(161.01,211.55) --
	(161.04,211.57) --
	(161.11,211.56) --
	(161.22,211.49) --
	(161.37,211.40) --
	(161.43,211.41) --
	(161.50,211.48) --
	(161.60,211.51) --
	(161.61,211.51) --
	(161.61,211.51) --
	(161.67,211.49) --
	(161.71,211.44) --
	(161.73,211.28) --
	(161.72,211.24) --
	(161.72,211.22) --
	(161.65,211.12) --
	(161.68,211.07) --
	(161.79,210.93) --
	(161.86,210.90) --
	(161.91,210.94) --
	(161.91,210.99) --
	(161.93,211.06) --
	(161.99,211.07) --
	(162.10,211.02) --
	(162.21,210.78) --
	(162.35,210.61) --
	(162.43,210.48) --
	(162.56,210.41) --
	(162.71,210.42) --
	(162.80,210.36) --
	(162.92,210.36) --
	(163.09,210.37) --
	(163.22,210.44) --
	(163.31,210.48) --
	(163.40,210.48) --
	(163.43,210.39) --
	(163.56,210.31) --
	(163.73,210.30) --
	(163.88,210.25) --
	(163.98,210.27) --
	(164.03,210.25) --
	(164.08,210.16) --
	(164.09,210.10) --
	(164.15,210.05) --
	(164.17,209.99) --
	(164.10,209.96) --
	(164.05,209.91) --
	(164.08,209.85) --
	(164.19,209.84) --
	(164.37,209.91) --
	(164.50,209.90) --
	(164.55,209.83) --
	(164.52,209.77) --
	(164.40,209.70) --
	(164.37,209.60) --
	(164.42,209.54) --
	(164.51,209.53) --
	(164.63,209.58) --
	(164.85,209.77) --
	(165.04,209.81) --
	(165.25,209.80) --
	(165.28,209.76) --
	(165.25,209.71) --
	(165.16,209.65) --
	(165.15,209.61) --
	(165.21,209.57) --
	(165.31,209.53) --
	(165.32,209.47) --
	(165.24,209.33) --
	(165.29,209.30) --
	(165.41,209.34) --
	(165.51,209.35) --
	(165.63,209.33) --
	(165.68,209.28) --
	(165.60,209.24) --
	(165.61,209.16) --
	(165.68,209.11) --
	(165.79,209.12) --
	(165.83,209.14) --
	(165.81,209.29) --
	(165.87,209.43) --
	(165.84,209.49) --
	(165.72,209.55) --
	(165.71,209.64) --
	(165.77,209.69) --
	(165.95,209.68) --
	(166.19,209.76) --
	(166.39,209.92) --
	(166.51,209.94) --
	(166.57,209.87) --
	(166.54,209.80) --
	(166.25,209.61) --
	(166.12,209.51) --
	(166.09,209.37) --
	(166.16,209.24) --
	(166.23,209.22) --
	(166.29,209.27) --
	(166.39,209.43) --
	(166.51,209.48) --
	(166.58,209.43) --
	(166.56,209.33) --
	(166.54,209.24) --
	(166.62,209.16) --
	(166.74,209.15) --
	(166.83,209.24) --
	(166.97,209.39) --
	(167.08,209.41) --
	(167.20,209.39) --
	(167.33,209.47) --
	(167.40,209.47) --
	(167.42,209.40) --
	(167.47,209.24) --
	(167.56,209.19) --
	(167.65,209.14) --
	(167.66,209.09) --
	(167.56,209.04) --
	(167.49,209.01) --
	(167.46,208.89) --
	(167.52,208.80) --
	(167.60,208.79) --
	(167.67,208.85) --
	(167.79,208.98) --
	(167.91,209.00) --
	(168.01,209.06) --
	(168.20,209.22) --
	(168.33,209.21) --
	(168.33,209.14) --
	(168.25,208.94) --
	(168.25,208.74) --
	(168.30,208.65) --
	(168.50,208.51) --
	(168.62,208.48) --
	(168.72,208.55) --
	(168.89,208.76) --
	(169.03,208.75) --
	(169.22,208.64) --
	(169.49,208.65) --
	(169.61,208.78) --
	(169.71,208.92) --
	(169.75,208.99) --
	(169.70,209.10) --
	(169.70,209.20) --
	(169.78,209.26) --
	(169.90,209.23) --
	(169.96,209.15) --
	(170.06,209.09) --
	(170.14,208.95) --
	(170.22,208.85) --
	(170.46,208.76) --
	(170.58,208.63) --
	(170.68,208.68) --
	(170.68,208.77) --
	(170.65,208.83) --
	(170.54,208.90) --
	(170.50,208.98) --
	(170.54,209.09) --
	(170.69,209.17) --
	(170.86,209.31) --
	(170.88,209.39) --
	(170.90,209.46) --
	(171.01,209.50) --
	(171.06,209.60) --
	(171.17,209.65) --
	(171.20,209.69) --
	(171.25,209.74) --
	(171.40,209.79) --
	(171.43,209.84) --
	(171.35,209.90) --
	(171.28,209.94) --
	(171.34,210.11) --
	(171.39,210.27) --
	(171.46,210.32) --
	(171.51,210.28) --
	(171.57,210.25) --
	(171.61,210.27) --
	(171.60,210.37) --
	(171.61,210.43) --
	(171.69,210.45) --
	(171.79,210.44) --
	(171.86,210.52) --
	(171.99,210.61) --
	(172.11,210.63) --
	(172.20,210.71) --
	(172.29,210.70) --
	(172.38,210.66) --
	(172.45,210.71) --
	(172.57,210.79) --
	(172.59,210.84) --
	(172.52,210.90) --
	(172.58,210.95) --
	(172.69,210.95) --
	(172.73,211.02) --
	(172.79,211.07) --
	(172.83,211.07) --
	(172.89,211.02) --
	(172.96,210.98) --
	(173.01,211.04) --
	(173.10,211.07) --
	(173.15,211.16) --
	(173.22,211.20) --
	(173.27,211.14) --
	(173.33,211.10) --
	(173.44,211.12) --
	(173.58,211.14) --
	(173.71,211.12) --
	(173.86,211.13) --
	(174.19,211.17) --
	(174.34,211.19) --
	(174.46,211.12) --
	(174.53,211.16) --
	(174.59,211.17) --
	(174.65,211.11) --
	(174.68,211.05) --
	(174.77,211.09) --
	(174.82,211.17) --
	(174.84,211.20) --
	(174.96,211.24) --
	(175.16,211.25) --
	(175.33,211.30) --
	(175.53,211.39) --
	(175.66,211.45) --
	(175.79,211.52) --
	(175.94,211.55) --
	(176.01,211.67) --
	(176.09,211.70) --
	(176.22,211.71) --
	(176.33,211.67) --
	(176.52,211.66) --
	(176.76,211.57) --
	(176.88,211.58) --
	(176.98,211.62) --
	(177.06,211.64) --
	(177.15,211.65) --
	(177.22,211.61) --
	(177.28,211.55) --
	(177.37,211.51) --
	(177.43,211.44) --
	(177.50,211.34) --
	(177.62,211.30) --
	(177.68,211.21) --
	(177.75,211.09) --
	(177.85,211.03) --
	(177.89,210.88) --
	(177.97,210.78) --
	(178.06,210.75) --
	(178.22,210.77) --
	(178.43,210.82) --
	(178.61,210.81) --
	(178.73,210.79) --
	(178.86,210.74) --
	(179.04,210.75) --
	(179.51,210.79) --
	(179.70,210.80) --
	(179.86,210.82) --
	(180.16,211.03) --
	(180.33,211.09) --
	(180.45,211.18) --
	(180.57,211.21) --
	(180.74,211.19) --
	(180.84,211.20) --
	(180.96,211.28) --
	(181.03,211.37) --
	(181.12,211.38) --
	(181.24,211.35) --
	(181.28,211.39) --
	(181.36,211.50) --
	(181.47,211.56) --
	(181.70,211.52) --
	(181.91,211.49) --
	(182.06,211.46) --
	(182.13,211.46) --
	(182.15,211.53) --
	(182.22,211.59) --
	(182.34,211.55) --
	(182.51,211.39) --
	(182.70,211.25) --
	(182.79,211.19) --
	(182.93,211.22) --
	(183.08,211.18) --
	(183.18,211.12) --
	(183.26,211.04) --
	(183.25,210.92) --
	(183.25,210.82) --
	(183.32,210.76) --
	(183.44,210.67) --
	(183.62,210.50) --
	(183.81,210.39) --
	(183.95,210.38) --
	(184.13,210.47) --
	(184.23,210.56) --
	(184.35,210.63) --
	(184.46,210.66) --
	(184.59,210.62) --
	(184.82,210.61) --
	(185.05,210.63) --
	(185.34,210.63) --
	(185.70,210.59) --
	(185.87,210.56) --
	(185.95,210.56) --
	(186.08,210.60) --
	(186.22,210.61) --
	(186.40,210.58) --
	(186.52,210.56) --
	(186.69,210.54) --
	(186.92,210.52) --
	(187.02,210.54) --
	(187.16,210.59) --
	(187.33,210.66) --
	(187.48,210.68) --
	(187.72,210.61) --
	(187.90,210.63) --
	(188.03,210.68) --
	(188.21,210.73) --
	(188.30,210.74) --
	(188.35,210.79) --
	(188.42,210.82) --
	(188.52,210.86) --
	(188.60,210.91) --
	(188.73,210.96) --
	(188.82,210.98) --
	(189.03,211.05) --
	(189.22,211.11) --
	(189.42,211.11) --
	(189.52,211.17) --
	(189.59,211.21) --
	(189.70,211.23) --
	(189.82,211.26) --
	(189.92,211.26) --
	(190.03,211.22) --
	(190.17,211.23) --
	(190.25,211.20) --
	(190.41,211.16) --
	(190.61,211.12) --
	(190.92,210.94) --
	(191.24,210.76) --
	(191.36,210.70) --
	(191.43,210.63) --
	(191.48,210.54) --
	(191.50,210.38) --
	(191.50,210.13) --
	(191.53,210.02) --
	(191.61,209.90) --
	(191.64,209.69) --
	(191.59,209.61) --
	(191.49,209.55) --
	(191.45,209.40) --
	(191.39,209.29) --
	(191.26,209.14) --
	(191.23,209.07) --
	(191.27,209.01) --
	(191.38,208.52) --
	(191.37,208.36) --
	(191.43,208.07) --
	(191.61,207.96) --
	(191.85,207.90) --
	(192.00,207.85) --
	(192.09,207.76) --
	(192.11,207.58) --
	(192.05,207.47) --
	(191.92,207.34) --
	(191.92,207.20) --
	(191.91,207.09) --
	(191.82,206.96) --
	(191.72,206.87) --
	(191.72,206.73) --
	(191.73,206.58) --
	(191.84,206.46) --
	(191.89,206.34) --
	(191.95,206.17) --
	(191.95,206.10) --
	(191.89,206.01) --
	(191.87,205.96) --
	(191.71,205.89) --
	(191.12,205.75) --
	(191.00,205.67) --
	(190.91,205.64) --
	(190.67,205.59) --
	(190.56,205.55) --
	(190.41,205.52) --
	(190.29,205.47) --
	(190.08,205.40) --
	(189.91,205.25) --
	(189.67,205.22) --
	(189.39,205.15) --
	(189.21,205.13) --
	(189.09,205.07) --
	(188.92,204.90) --
	(188.78,204.68) --
	(188.79,204.48) --
	(188.70,204.25) --
	(188.66,204.08) --
	(188.78,203.86) --
	(188.99,203.58) --
	(189.16,203.40) --
	(189.41,203.22) --
	(189.50,203.00) --
	(189.64,202.67);

\draw[color=drawColor,line cap=round,line join=round,fill opacity=0.00,] (121.16,200.81) --
	(121.21,200.86) --
	(121.17,200.89) --
	(121.14,200.90) --
	(120.98,200.95) --
	(120.96,200.97) --
	(120.92,201.12) --
	(120.84,201.31) --
	(120.81,201.38) --
	(120.77,201.40) --
	(120.65,201.48) --
	(120.51,201.60) --
	(120.37,201.66) --
	(120.18,201.70) --
	(120.15,201.73) --
	(120.16,201.76) --
	(120.14,201.78) --
	(120.08,201.84) --
	(120.09,201.87) --
	(120.10,201.91) --
	(120.10,201.93) --
	(120.07,201.97) --
	(119.84,202.17) --
	(119.83,202.20) --
	(119.79,202.30) --
	(119.72,202.42) --
	(119.65,202.62) --
	(119.65,202.66) --
	(119.56,202.74) --
	(119.55,202.78) --
	(119.48,202.94) --
	(119.48,202.97) --
	(119.38,203.05) --
	(119.37,203.07) --
	(119.38,203.15) --
	(119.38,203.16) --
	(119.37,203.18) --
	(119.28,203.20) --
	(119.28,203.22) --
	(119.31,203.24) --
	(119.28,203.25) --
	(119.20,203.40) --
	(119.15,203.44) --
	(119.13,203.47) --
	(119.13,203.57) --
	(119.11,203.59) --
	(119.09,203.60) --
	(118.98,203.65) --
	(118.81,203.73) --
	(118.68,203.91) --
	(118.68,203.93) --
	(118.70,203.95) --
	(118.78,203.95) --
	(118.78,203.97) --
	(118.79,203.99) --
	(118.76,204.00) --
	(118.55,204.07) --
	(118.55,204.07) --
	(118.51,204.05) --
	(118.27,204.00) --
	(118.16,203.94) --
	(118.10,203.93) --
	(118.04,203.93) --
	(117.92,203.96) --
	(117.89,203.96) --
	(117.85,203.95) --
	(117.83,203.93) --
	(117.70,203.91) --
	(117.64,203.87) --
	(117.62,203.87) --
	(117.51,203.90) --
	(117.47,203.87) --
	(117.45,203.87) --
	(117.28,203.88) --
	(117.22,203.89) --
	(117.13,203.91) --
	(117.00,203.93) --
	(116.97,203.92) --
	(116.92,203.84) --
	(116.91,203.84) --
	(116.89,203.84) --
	(116.78,203.87) --
	(116.60,203.88) --
	(116.57,203.88) --
	(116.42,203.84) --
	(116.29,203.84) --
	(116.26,203.83) --
	(116.20,203.80) --
	(116.18,203.81) --
	(116.14,203.84) --
	(116.11,203.86) --
	(116.02,203.78) --
	(116.00,203.76) --
	(115.88,203.75) --
	(115.85,203.74) --
	(115.81,203.69) --
	(115.76,203.68) --
	(115.37,203.64) --
	(115.34,203.64) --
	(115.31,203.63) --
	(115.30,203.60) --
	(115.28,203.56) --
	(115.27,203.55) --
	(115.21,203.50) --
	(115.18,203.49) --
	(115.15,203.48) --
	(115.12,203.48) --
	(115.10,203.48) --
	(115.10,203.49) --
	(115.06,203.53) --
	(115.03,203.52) --
	(115.01,203.50) --
	(114.98,203.48) --
	(114.95,203.46) --
	(114.75,203.46) --
	(114.73,203.45) --
	(114.76,203.39) --
	(114.74,203.37) --
	(114.51,203.41) --
	(114.36,203.39) --
	(114.33,203.39) --
	(114.31,203.40) --
	(114.27,203.45) --
	(114.25,203.44) --
	(114.24,203.45) --
	(114.22,203.44) --
	(114.16,203.41) --
	(114.07,203.40) --
	(113.96,203.37) --
	(113.93,203.38) --
	(113.88,203.44) --
	(113.85,203.46) --
	(113.82,203.46) --
	(113.81,203.45) --
	(113.76,203.41) --
	(113.70,203.39) --
	(113.67,203.41) --
	(113.65,203.43) --
	(113.60,203.48) --
	(113.57,203.49) --
	(113.51,203.47) --
	(113.50,203.48) --
	(113.44,203.51) --
	(113.42,203.51) --
	(113.38,203.50) --
	(113.35,203.44) --
	(113.33,203.43) --
	(113.30,203.44) --
	(113.23,203.56) --
	(113.10,203.66) --
	(113.07,203.66) --
	(113.04,203.65) --
	(113.01,203.58) --
	(112.98,203.56) --
	(112.95,203.57) --
	(112.89,203.61) --
	(112.79,203.66) --
	(112.48,203.75) --
	(112.27,203.75) --
	(112.24,203.77) --
	(112.22,203.83) --
	(112.22,203.84) --
	(112.20,203.84) --
	(112.17,203.83) --
	(112.08,203.79) --
	(112.07,203.78) --
	(111.92,203.79) --
	(111.71,203.78) --
	(111.69,203.78) --
	(111.69,203.77) --
	(111.72,203.75) --
	(111.69,203.74) --
	(111.66,203.73) --
	(111.58,203.77) --
	(111.53,203.78) --
	(111.35,203.79) --
	(111.03,203.76) --
	(111.01,203.75) --
	(110.99,203.72) --
	(110.97,203.71) --
	(110.88,203.72) --
	(110.80,203.72) --
	(110.79,203.69) --
	(110.77,203.65) --
	(110.71,203.68) --
	(110.68,203.67) --
	(110.68,203.66) --
	(110.68,203.64) --
	(110.69,203.59) --
	(110.68,203.58) --
	(110.66,203.58) --
	(110.61,203.61) --
	(110.56,203.62) --
	(110.53,203.61) --
	(110.50,203.58) --
	(110.48,203.50) --
	(110.47,203.48) --
	(110.41,203.43) --
	(110.39,203.41) --
	(110.39,203.33) --
	(110.36,203.32) --
	(110.32,203.32) --
	(110.26,203.38) --
	(110.23,203.38) --
	(110.20,203.38) --
	(110.18,203.35) --
	(110.15,203.30) --
	(110.16,203.17) --
	(110.15,203.16) --
	(110.13,203.14) --
	(109.93,203.14) --
	(109.90,203.14) --
	(109.85,203.12) --
	(109.78,203.11) --
	(109.75,203.09) --
	(109.72,203.06) --
	(109.69,203.05) --
	(109.63,203.05) --
	(109.60,203.04) --
	(109.56,202.97) --
	(109.55,202.96) --
	(109.52,202.95) --
	(109.51,202.96) --
	(109.49,202.99) --
	(109.45,203.00) --
	(109.43,202.96) --
	(109.41,202.92) --
	(109.39,202.92) --
	(109.37,202.91) --
	(109.28,202.91) --
	(109.25,202.91) --
	(109.24,202.97) --
	(109.19,202.98) --
	(109.15,202.97) --
	(109.15,202.96) --
	(109.13,202.90) --
	(109.12,202.89) --
	(109.01,202.89) --
	(108.77,202.85) --
	(108.77,202.84) --
	(108.76,202.80) --
	(108.74,202.78) --
	(108.70,202.81) --
	(108.67,202.82) --
	(108.64,202.80) --
	(108.58,202.75) --
	(108.55,202.73) --
	(108.48,202.81) --
	(108.45,202.81) --
	(108.42,202.81) --
	(108.39,202.81) --
	(108.39,202.76) --
	(108.37,202.75) --
	(108.28,202.75) --
	(108.27,202.79) --
	(108.19,202.80) --
	(108.17,202.80) --
	(108.13,202.77) --
	(108.09,202.77) --
	(108.05,202.82) --
	(108.00,202.82) --
	(107.96,202.83) --
	(107.88,202.85) --
	(107.82,202.84) --
	(107.75,202.83) --
	(107.72,202.83) --
	(107.69,202.81) --
	(107.61,202.72) --
	(107.58,202.69) --
	(107.54,202.69) --
	(107.46,202.70) --
	(107.42,202.73) --
	(107.39,202.74) --
	(107.26,202.72) --
	(107.24,202.71) --
	(107.23,202.72) --
	(107.20,202.76) --
	(107.17,202.77) --
	(107.01,202.74) --
	(106.86,202.74) --
	(106.83,202.76) --
	(106.74,202.82) --
	(106.73,202.84) --
	(106.65,202.84) --
	(106.62,202.86) --
	(106.61,202.89) --
	(106.57,202.93) --
	(106.46,202.97) --
	(106.18,203.02) --
	(106.02,203.08) --
	(105.95,203.08) --
	(105.89,203.06) --
	(105.87,203.06) --
	(105.80,203.00) --
	(105.77,202.99) --
	(105.74,203.01) --
	(105.51,203.01) --
	(105.50,203.02) --
	(105.43,203.05) --
	(105.41,203.07) --
	(105.29,203.05) --
	(105.26,203.03) --
	(105.22,203.05) --
	(105.10,203.11) --
	(105.07,203.10) --
	(104.94,203.09) --
	(104.89,203.10) --
	(104.86,203.15) --
	(104.83,203.16) --
	(104.70,203.18) --
	(104.67,203.18) --
	(104.64,203.15) --
	(104.58,203.08) --
	(104.56,203.05) --
	(104.47,203.04) --
	(104.46,203.06) --
	(104.45,203.09) --
	(104.42,203.12) --
	(104.39,203.14) --
	(104.24,203.10) --
	(104.22,203.09) --
	(104.22,203.03) --
	(104.21,202.99) --
	(104.13,202.96) --
	(104.10,202.94) --
	(104.10,202.90) --
	(104.07,202.89) --
	(103.98,202.88) --
	(103.94,202.85) --
	(103.91,202.79) --
	(103.85,202.77) --
	(103.65,202.73) --
	(103.39,202.87) --
	(103.29,202.95) --
	(103.18,203.04) --
	(103.18,203.06) --
	(103.17,203.11) --
	(103.20,203.17) --
	(103.20,203.20) --
	(103.17,203.25) --
	(103.16,203.27) --
	(103.20,203.31) --
	(103.22,203.33) --
	(103.20,203.35) --
	(103.18,203.69) --
	(103.32,203.81) --
	(103.32,203.83) --
	(103.34,203.91) --
	(103.35,204.00) --
	(103.37,204.10) --
	(103.37,204.12) --
	(103.36,204.18) --
	(103.36,204.21) --
	(103.41,204.33) --
	(103.41,204.39) --
	(103.38,204.42) --
	(103.35,204.46) --
	(103.31,204.50) --
	(103.29,204.53) --
	(103.31,204.55) --
	(103.29,204.57) --
	(103.23,204.65) --
	(103.22,204.65) --
	(103.19,204.65) --
	(103.15,204.66) --
	(103.12,204.68) --
	(103.12,204.72) --
	(103.13,204.75) --
	(103.15,204.78) --
	(103.12,204.81) --
	(103.07,204.81) --
	(103.06,204.82) --
	(103.05,204.85) --
	(102.94,204.98) --
	(102.84,205.06) --
	(102.78,205.10) --
	(102.61,205.24) --
	(102.56,205.25) --
	(102.55,205.27) --
	(102.52,205.33) --
	(102.51,205.34) --
	(102.43,205.35) --
	(102.39,205.36) --
	(102.32,205.47) --
	(102.24,205.57) --
	(102.21,205.60) --
	(102.18,205.61) --
	(102.08,205.61) --
	(102.02,205.63) --
	(101.88,205.72) --
	(101.85,205.73) --
	(101.77,205.75) --
	(101.59,205.75) --
	(101.56,205.75) --
	(101.49,205.81) --
	(101.48,205.82) --
	(101.43,205.83) --
	(101.31,205.82) --
	(101.13,205.85) --
	(101.11,205.86) --
	(101.00,205.95) --
	(100.83,206.11) --
	(100.80,206.13) --
	(100.77,206.14) --
	(100.40,206.24) --
	(100.35,206.24) --
	( 99.80,206.23) --
	( 99.70,206.21) --
	( 99.65,206.18) --
	( 99.61,206.17) --
	( 99.58,206.16) --
	( 99.24,206.17) --
	( 98.81,206.20) --
	( 98.39,206.14) --
	( 98.26,206.13) --
	( 98.23,206.13) --
	( 98.17,206.15) --
	( 98.14,206.15) --
	( 98.00,206.13) --
	( 97.97,206.14) --
	( 97.96,206.19) --
	( 97.93,206.21) --
	( 97.87,206.23) --
	( 97.60,206.29) --
	( 97.47,206.33) --
	( 97.41,206.37) --
	( 97.31,206.43) --
	( 97.13,206.49) --
	( 97.07,206.52) --
	( 96.66,206.79) --
	( 96.29,207.00) --
	( 96.18,207.11) --
	( 96.02,207.20) --
	( 95.74,207.36) --
	( 95.55,207.57) --
	( 95.43,207.72) --
	( 95.27,207.92) --
	( 95.20,207.98) --
	( 95.09,208.04) --
	( 95.08,208.04) --
	( 95.03,208.12) --
	( 95.02,208.14) --
	( 94.74,208.39) --
	( 94.30,208.69) --
	( 94.08,208.83) --
	( 93.87,209.01) --
	( 93.83,209.03) --
	( 93.74,209.03) --
	( 93.52,209.19) --
	( 93.49,209.22) --
	( 93.42,209.35) --
	( 93.32,209.48) --
	( 93.26,209.52) --
	( 93.11,209.58) --
	( 93.06,209.62) --
	( 93.04,209.66) --
	( 92.93,209.86) --
	( 92.80,209.96) --
	( 92.80,209.99) --
	( 92.83,210.05) --
	( 92.80,210.10) --
	( 92.69,210.16) --
	( 92.72,210.28) --
	( 92.78,210.38) --
	( 92.78,210.43) --
	( 92.75,210.48) --
	( 92.69,210.55) --
	( 92.57,210.64) --
	( 92.53,210.67) --
	( 92.53,210.72) --
	( 92.50,210.82) --
	( 92.46,210.85) --
	( 92.36,210.94) --
	( 92.34,210.98) --
	( 92.31,211.13) --
	( 92.30,211.17) --
	( 92.26,211.19) --
	( 92.12,211.30) --
	( 92.12,211.34) --
	( 92.11,211.41) --
	( 92.08,211.44) --
	( 92.05,211.47) --
	( 91.98,211.50) --
	( 91.96,211.53) --
	( 91.90,211.64) --
	( 91.82,211.75) --
	( 91.77,211.83) --
	( 91.70,211.92) --
	( 91.63,212.00) --
	( 91.61,212.02) --
	( 91.60,212.12) --
	( 91.57,212.15) --
	( 91.36,212.29) --
	( 91.33,212.35) --
	( 91.30,212.42) --
	( 91.32,212.48) --
	( 91.31,212.52) --
	( 91.26,212.60) --
	( 91.26,212.63) --
	( 91.25,212.66) --
	( 91.25,212.88) --
	( 91.23,213.36) --
	( 91.23,213.40) --
	( 91.29,213.57) --
	( 91.28,213.88) --
	( 91.31,213.98) --
	( 91.30,214.17) --
	( 91.33,214.23) --
	( 91.33,214.26) --
	( 91.31,214.37) --
	( 91.31,214.42) --
	( 91.32,214.45) --
	( 91.37,214.53) --
	( 91.50,214.69) --
	( 91.61,214.76) --
	( 91.88,214.91) --
	( 92.08,214.95) --
	( 92.08,214.97) --
	( 92.08,215.03) --
	( 92.11,215.06) --
	( 92.18,215.15) --
	( 92.19,215.18) --
	( 92.17,215.37) --
	( 92.16,215.47) --
	( 92.12,215.63) --
	( 92.10,215.75) --
	( 92.00,215.87) --
	( 91.97,215.95);

\draw[color=drawColor,line cap=round,line join=round,fill opacity=0.00,] (215.50,204.19) --
	(215.44,204.21) --
	(215.32,204.20) --
	(215.24,204.24) --
	(215.15,204.25) --
	(215.17,204.45) --
	(215.17,204.61) --
	(215.15,204.75) --
	(215.14,204.87) --
	(215.17,205.18) --
	(215.32,205.25) --
	(215.51,205.37) --
	(215.59,205.49) --
	(215.65,205.65) --
	(215.65,205.78) --
	(215.62,205.89) --
	(215.57,205.95) --
	(215.43,206.08) --
	(215.43,206.37) --
	(215.40,206.46) --
	(215.44,206.64) --
	(215.43,206.76) --
	(215.40,206.96) --
	(215.34,207.19) --
	(215.32,207.50) --
	(215.37,207.75) --
	(215.35,207.99) --
	(215.40,208.07) --
	(215.56,208.37) --
	(215.58,208.56) --
	(215.51,208.88) --
	(215.55,209.20) --
	(215.52,209.61) --
	(215.61,209.98) --
	(215.70,210.09) --
	(215.84,210.18);

\draw[color=drawColor,line cap=round,line join=round,fill opacity=0.00,] (216.18,203.55) --
	(216.15,203.60) --
	(216.09,203.65) --
	(215.87,203.74) --
	(215.84,203.92) --
	(215.78,203.96);

\draw[color=drawColor,line cap=round,line join=round,fill opacity=0.00,] (171.54,198.12) --
	(171.42,198.05) --
	(171.33,198.00) --
	(171.26,197.93) --
	(171.07,197.73) --
	(170.92,197.69) --
	(170.65,197.59) --
	(170.39,197.54) --
	(170.20,197.57) --
	(169.98,197.50) --
	(169.80,197.46) --
	(169.63,197.48) --
	(169.51,197.47) --
	(169.36,197.39) --
	(169.27,197.39) --
	(169.23,197.44) --
	(169.14,197.45) --
	(169.08,197.41) --
	(169.04,197.33) --
	(168.86,197.30) --
	(168.73,197.29) --
	(168.65,197.32) --
	(168.59,197.39) --
	(168.53,197.41) --
	(168.18,197.39) --
	(167.83,197.52) --
	(167.58,197.66) --
	(167.30,198.00) --
	(167.09,198.28) --
	(166.95,198.52) --
	(166.92,198.70) --
	(166.81,198.81) --
	(166.78,198.98) --
	(166.70,199.18) --
	(166.50,199.48) --
	(166.38,199.59) --
	(166.26,199.61) --
	(166.02,199.59) --
	(165.89,199.63) --
	(165.80,199.64) --
	(165.66,199.63) --
	(165.53,199.68) --
	(165.40,199.63) --
	(164.97,199.58) --
	(164.69,199.59) --
	(164.45,199.51) --
	(164.20,199.44) --
	(163.98,199.34) --
	(163.91,199.24) --
	(163.81,199.22) --
	(163.59,199.24) --
	(163.44,199.24) --
	(163.36,199.29) --
	(163.26,199.27) --
	(163.20,199.30) --
	(163.12,199.34) --
	(163.01,199.26) --
	(162.92,199.26) --
	(162.75,199.33) --
	(162.62,199.31) --
	(162.49,199.33) --
	(162.31,199.40) --
	(162.19,199.36) --
	(162.08,199.44) --
	(161.93,199.47) --
	(161.73,199.48) --
	(161.58,199.45) --
	(161.49,199.43) --
	(161.39,199.45) --
	(161.24,199.41) --
	(161.12,199.37) --
	(161.02,199.28) --
	(160.92,199.22) --
	(160.71,199.18) --
	(160.57,199.13) --
	(160.29,199.11) --
	(159.93,199.16) --
	(159.67,199.26) --
	(159.40,199.44) --
	(159.04,199.73) --
	(158.90,199.87) --
	(158.79,199.91) --
	(158.72,200.01) --
	(158.59,200.17) --
	(158.34,200.40) --
	(158.14,200.55) --
	(157.99,200.60) --
	(157.84,200.62) --
	(157.64,200.75) --
	(157.40,200.85) --
	(157.27,200.97) --
	(157.10,201.04) --
	(156.95,201.12) --
	(156.83,201.14) --
	(156.68,201.11) --
	(156.59,201.10) --
	(156.43,201.01) --
	(156.28,201.04) --
	(156.27,201.04) --
	(156.15,201.04) --
	(156.09,201.15) --
	(155.85,201.32) --
	(155.74,201.48) --
	(155.57,201.60) --
	(155.49,201.75) --
	(155.30,201.83) --
	(154.96,201.99) --
	(154.86,202.03) --
	(154.80,202.11) --
	(154.69,202.29) --
	(154.58,202.37) --
	(154.49,202.42) --
	(154.27,202.45) --
	(154.16,202.52) --
	(153.99,202.68) --
	(153.90,202.76) --
	(153.88,202.89) --
	(153.83,202.97) --
	(153.65,203.06) --
	(153.41,203.13) --
	(153.10,203.21) --
	(152.90,203.23) --
	(152.83,203.19) --
	(152.68,203.13) --
	(152.54,203.05) --
	(152.24,203.10) --
	(152.03,203.22) --
	(151.91,203.34) --
	(151.87,203.44) --
	(151.79,203.61) --
	(151.76,203.65) --
	(151.69,203.64) --
	(151.52,203.62) --
	(151.33,203.61) --
	(150.98,203.58) --
	(150.81,203.57) --
	(150.69,203.66) --
	(150.60,203.68) --
	(150.54,203.68) --
	(150.47,203.65) --
	(150.37,203.64) --
	(150.32,203.63) --
	(150.32,203.59) --
	(150.29,203.56) --
	(150.15,203.49) --
	(150.07,203.44) --
	(149.84,203.40) --
	(149.75,203.37) --
	(149.65,203.35) --
	(149.54,203.37) --
	(149.16,203.31) --
	(149.00,203.24) --
	(148.85,203.19) --
	(148.45,203.17) --
	(148.30,203.20) --
	(148.24,203.20) --
	(148.09,203.16) --
	(147.92,203.12) --
	(147.79,203.07) --
	(147.62,203.03) --
	(147.48,203.02) --
	(147.34,202.97) --
	(147.26,202.96) --
	(147.21,202.95) --
	(147.20,202.91) --
	(147.13,202.88) --
	(147.05,202.85) --
	(146.99,202.80) --
	(146.82,202.80) --
	(146.71,202.81) --
	(146.65,202.80) --
	(146.61,202.75) --
	(146.55,202.74) --
	(146.46,202.81) --
	(146.10,202.79) --
	(145.99,202.79) --
	(145.82,202.74) --
	(145.71,202.70) --
	(145.57,202.70) --
	(145.41,202.64) --
	(145.28,202.57) --
	(145.12,202.52) --
	(145.06,202.49) --
	(145.03,202.41) --
	(145.02,202.35) --
	(144.95,202.25) --
	(144.95,202.19) --
	(144.99,202.11) --
	(144.95,202.04) --
	(144.98,201.98) --
	(145.05,201.93) --
	(145.06,201.87) --
	(144.93,201.60) --
	(144.94,201.46) --
	(144.90,201.37) --
	(144.81,201.27) --
	(144.69,201.25) --
	(144.51,201.27) --
	(144.25,201.26) --
	(144.05,201.27) --
	(143.76,201.23) --
	(143.62,201.27) --
	(143.42,201.26) --
	(143.35,201.20) --
	(143.36,201.11) --
	(143.32,201.00) --
	(143.12,200.97) --
	(143.03,201.00) --
	(143.01,201.10) --
	(142.84,201.13) --
	(142.52,201.13) --
	(142.40,201.08) --
	(142.20,200.93) --
	(141.89,200.82) --
	(141.60,200.71) --
	(141.44,200.60) --
	(141.32,200.61) --
	(140.80,200.65) --
	(140.68,200.67) --
	(140.49,200.59) --
	(140.37,200.61) --
	(140.21,200.54) --
	(140.11,200.44) --
	(140.02,200.45) --
	(139.89,200.50) --
	(139.80,200.52) --
	(139.73,200.46) --
	(139.37,200.47) --
	(139.25,200.41) --
	(139.08,200.38) --
	(138.89,200.31) --
	(138.70,200.25) --
	(138.59,200.27) --
	(138.51,200.21) --
	(138.37,200.20) --
	(138.22,200.25) --
	(138.08,200.36) --
	(137.69,200.29) --
	(137.60,200.34) --
	(137.46,200.42) --
	(137.42,200.39) --
	(137.45,200.27) --
	(137.68,200.01) --
	(137.60,199.92) --
	(137.57,199.83) --
	(137.53,199.72) --
	(137.46,199.63) --
	(137.39,199.59) --
	(137.36,199.55) --
	(137.32,199.49) --
	(137.19,199.45) --
	(137.04,199.43) --
	(136.97,199.45) --
	(136.80,199.43) --
	(136.74,199.45) --
	(136.69,199.44) --
	(136.64,199.42) --
	(136.60,199.41) --
	(136.59,199.39) --
	(136.53,199.34) --
	(136.51,199.31) --
	(136.48,199.21) --
	(136.46,199.16) --
	(136.43,199.14) --
	(136.38,199.12) --
	(136.28,199.11) --
	(136.25,199.11) --
	(136.11,199.06) --
	(136.08,199.05) --
	(135.97,199.06) --
	(135.92,199.05) --
	(135.85,199.03) --
	(135.83,199.03) --
	(135.77,199.04) --
	(135.71,199.05) --
	(135.68,199.02) --
	(135.67,199.01) --
	(135.64,198.97) --
	(135.65,198.89) --
	(135.66,198.72) --
	(135.70,198.61) --
	(135.68,198.60) --
	(135.67,198.58) --
	(135.62,198.54) --
	(135.59,198.53) --
	(135.54,198.46) --
	(135.51,198.45) --
	(135.48,198.45) --
	(135.38,198.48) --
	(135.35,198.49) --
	(135.32,198.50) --
	(135.26,198.45) --
	(135.02,198.43) --
	(134.98,198.42) --
	(134.92,198.35) --
	(134.89,198.34) --
	(134.75,198.28) --
	(134.72,198.27) --
	(134.72,198.24) --
	(134.73,198.07) --
	(134.74,198.04) --
	(134.72,198.02) --
	(134.69,197.98) --
	(134.65,197.97) --
	(134.47,197.94) --
	(134.35,197.95) --
	(134.28,197.98) --
	(134.16,197.98) --
	(133.94,198.00) --
	(133.90,198.01) --
	(133.85,198.04) --
	(133.82,198.07) --
	(133.74,198.23) --
	(133.73,198.24) --
	(133.70,198.26) --
	(133.67,198.26) --
	(133.67,198.26) --
	(133.64,198.24) --
	(133.59,198.14) --
	(133.51,198.08) --
	(133.50,198.04) --
	(133.51,197.97) --
	(133.52,197.90) --
	(133.52,197.88) --
	(133.50,197.83) --
	(133.47,197.75) --
	(133.45,197.73) --
	(133.35,197.71) --
	(133.32,197.68) --
	(133.32,197.66) --
	(133.29,197.63) --
	(133.18,197.61) --
	(133.12,197.60) --
	(133.08,197.63) --
	(133.05,197.64) --
	(133.01,197.66) --
	(132.96,197.66) --
	(132.95,197.68) --
	(132.94,197.70) --
	(132.87,197.82) --
	(132.84,197.86) --
	(132.81,197.87) --
	(132.78,197.86) --
	(132.76,197.86) --
	(132.72,197.85) --
	(132.69,197.80) --
	(132.67,197.77) --
	(132.68,197.67) --
	(132.68,197.64) --
	(132.68,197.61) --
	(132.66,197.59) --
	(132.66,197.57) --
	(132.68,197.55) --
	(132.69,197.54) --
	(132.77,197.52) --
	(132.82,197.51) --
	(132.84,197.47) --
	(132.85,197.42) --
	(132.85,197.38) --
	(132.80,197.31) --
	(132.77,197.29) --
	(132.73,197.27) --
	(132.70,197.27) --
	(132.67,197.28) --
	(132.64,197.29) --
	(132.53,197.28) --
	(132.49,197.31) --
	(132.48,197.33) --
	(132.45,197.40) --
	(132.41,197.43) --
	(132.38,197.44) --
	(132.24,197.46) --
	(132.19,197.48) --
	(132.11,197.54) --
	(132.06,197.64) --
	(132.03,197.66) --
	(131.97,197.69) --
	(131.96,197.71) --
	(131.93,197.77) --
	(131.92,197.79) --
	(131.87,197.81) --
	(131.84,197.83) --
	(131.79,197.82) --
	(131.73,197.80) --
	(131.69,197.78) --
	(131.63,197.76) --
	(131.62,197.75) --
	(131.60,197.73) --
	(131.60,197.67) --
	(131.59,197.65) --
	(131.54,197.64) --
	(131.50,197.66) --
	(131.43,197.75) --
	(131.40,197.77) --
	(131.28,197.75) --
	(131.26,197.75) --
	(131.22,197.73) --
	(131.20,197.62) --
	(131.17,197.60) --
	(131.08,197.55) --
	(131.05,197.52) --
	(131.06,197.48) --
	(131.03,197.44) --
	(130.98,197.39) --
	(130.88,197.30) --
	(130.76,197.29) --
	(130.74,197.27) --
	(130.73,197.23) --
	(130.77,197.09) --
	(130.76,197.06) --
	(130.84,197.01) --
	(130.85,196.97) --
	(130.85,196.95) --
	(130.82,196.92) --
	(130.80,196.81) --
	(130.77,196.78) --
	(130.74,196.76) --
	(130.71,196.76) --
	(130.66,196.78) --
	(130.62,196.79) --
	(130.61,196.81) --
	(130.59,196.84) --
	(130.66,197.04) --
	(130.68,197.05) --
	(130.66,197.09) --
	(130.63,197.11) --
	(130.50,197.17) --
	(130.42,197.22) --
	(130.36,197.23) --
	(130.28,197.24) --
	(130.21,197.23) --
	(130.18,197.21) --
	(130.16,197.18) --
	(130.11,197.04) --
	(130.10,197.03) --
	(130.02,196.99) --
	(129.99,196.97) --
	(129.93,196.96) --
	(129.89,196.97) --
	(129.84,196.99) --
	(129.82,197.02) --
	(129.72,197.12) --
	(129.67,197.15) --
	(129.64,197.18) --
	(129.65,197.23) --
	(129.62,197.24) --
	(129.59,197.26) --
	(129.48,197.26) --
	(129.44,197.26) --
	(129.40,197.27) --
	(129.34,197.26) --
	(129.32,197.26) --
	(129.28,197.23) --
	(129.21,197.16) --
	(129.21,197.14) --
	(129.17,197.14) --
	(129.05,197.15) --
	(128.95,197.16) --
	(128.85,197.22) --
	(128.76,197.25) --
	(128.72,197.29) --
	(128.70,197.32) --
	(128.67,197.35) --
	(128.66,197.40) --
	(128.62,197.42) --
	(128.56,197.42) --
	(128.51,197.42) --
	(128.42,197.38) --
	(128.32,197.34) --
	(128.27,197.33) --
	(128.18,197.33) --
	(128.07,197.34) --
	(127.99,197.36) --
	(127.96,197.36) --
	(127.95,197.39) --
	(127.95,197.43) --
	(127.97,197.47) --
	(128.15,197.62) --
	(128.27,197.69) --
	(128.27,197.71) --
	(128.23,197.81) --
	(128.18,197.89) --
	(128.13,197.94) --
	(128.12,197.95) --
	(127.99,197.98) --
	(127.90,197.99) --
	(127.78,198.03) --
	(127.87,198.34) --
	(127.84,198.38) --
	(127.78,198.38) --
	(127.75,198.40) --
	(127.69,198.48) --
	(127.65,198.49) --
	(127.55,198.53) --
	(127.52,198.54) --
	(127.47,198.59) --
	(127.46,198.63) --
	(127.46,198.68) --
	(127.49,198.73) --
	(127.51,198.78) --
	(127.56,198.82) --
	(127.59,198.86) --
	(127.59,198.91) --
	(127.59,198.93) --
	(127.58,198.94) --
	(127.53,198.96) --
	(127.50,198.93) --
	(127.39,198.84) --
	(127.38,198.82) --
	(127.35,198.84) --
	(127.31,198.84) --
	(127.22,198.88) --
	(127.13,198.90) --
	(127.07,198.90) --
	(126.97,198.84) --
	(126.92,198.84) --
	(126.88,198.86) --
	(126.88,198.89) --
	(126.88,198.94) --
	(126.94,199.07) --
	(126.94,199.10) --
	(126.92,199.15) --
	(126.89,199.19) --
	(126.82,199.22) --
	(126.70,199.24) --
	(126.60,199.27) --
	(126.48,199.29) --
	(126.47,199.42) --
	(126.46,199.50) --
	(126.44,199.54) --
	(126.47,199.55) --
	(126.49,199.56) --
	(126.58,199.59) --
	(126.59,199.59) --
	(126.61,199.61) --
	(126.62,199.63) --
	(126.61,199.66) --
	(126.60,199.68) --
	(126.58,199.70) --
	(126.52,199.71) --
	(126.48,199.72) --
	(126.45,199.69) --
	(126.39,199.65) --
	(126.36,199.65) --
	(126.25,199.64) --
	(126.15,199.71) --
	(126.17,199.74) --
	(126.15,199.76) --
	(126.08,199.78) --
	(125.99,199.83) --
	(125.98,199.86) --
	(125.97,199.89) --
	(125.95,199.98) --
	(125.95,199.99) --
	(125.94,200.02) --
	(125.87,200.05) --
	(125.87,200.07) --
	(125.85,200.09) --
	(125.85,200.11) --
	(125.87,200.12) --
	(125.96,200.16) --
	(125.90,200.27) --
	(125.88,200.29) --
	(125.76,200.29) --
	(125.76,200.30) --
	(125.74,200.31) --
	(125.71,200.35) --
	(125.70,200.36) --
	(125.69,200.37) --
	(125.66,200.38) --
	(125.63,200.36) --
	(125.58,200.34) --
	(125.55,200.36) --
	(125.54,200.36) --
	(125.54,200.38) --
	(125.56,200.47) --
	(125.54,200.49) --
	(125.53,200.49) --
	(125.42,200.50) --
	(125.41,200.51) --
	(125.34,200.58) --
	(125.29,200.59) --
	(125.26,200.57) --
	(125.25,200.51) --
	(125.22,200.50) --
	(125.14,200.47) --
	(125.04,200.45) --
	(125.02,200.46) --
	(125.01,200.48) --
	(124.96,200.49) --
	(124.81,200.43) --
	(124.62,200.41) --
	(124.57,200.40) --
	(124.43,200.48) --
	(124.38,200.48) --
	(124.33,200.48) --
	(124.28,200.46) --
	(124.24,200.44) --
	(124.22,200.40) --
	(124.22,200.36) --
	(124.25,200.34) --
	(124.25,200.32) --
	(124.25,200.31) --
	(124.17,200.30) --
	(124.05,200.31) --
	(123.93,200.31) --
	(123.90,200.31) --
	(123.82,200.37) --
	(123.79,200.37) --
	(123.67,200.37) --
	(123.63,200.40) --
	(123.55,200.43) --
	(123.52,200.44) --
	(123.39,200.40) --
	(123.31,200.40) --
	(123.27,200.42) --
	(123.21,200.44) --
	(123.14,200.51) --
	(123.11,200.52) --
	(123.10,200.52) --
	(123.02,200.49) --
	(122.94,200.42) --
	(122.93,200.41) --
	(122.88,200.41) --
	(122.71,200.48) --
	(122.66,200.47) --
	(122.62,200.47) --
	(122.58,200.47) --
	(122.56,200.46) --
	(122.54,200.37) --
	(122.53,200.36) --
	(122.50,200.35) --
	(122.45,200.35) --
	(122.38,200.43) --
	(122.35,200.42) --
	(122.34,200.41) --
	(122.29,200.39) --
	(122.26,200.37) --
	(122.23,200.38) --
	(122.19,200.40) --
	(122.16,200.39) --
	(122.02,200.38) --
	(121.99,200.38) --
	(122.00,200.42) --
	(121.97,200.42) --
	(121.80,200.41) --
	(121.76,200.40) --
	(121.68,200.32) --
	(121.64,200.31) --
	(121.64,200.33) --
	(121.64,200.37) --
	(121.62,200.39) --
	(121.61,200.40) --
	(121.42,200.37) --
	(121.16,200.33);

\draw[color=drawColor,line cap=round,line join=round,fill opacity=0.00,] (213.95,201.77) --
	(214.13,201.78) --
	(214.29,201.77) --
	(214.53,201.88) --
	(214.70,201.98) --
	(214.85,202.03) --
	(214.93,202.09) --
	(214.97,202.17) --
	(215.05,202.25) --
	(215.12,202.34) --
	(215.20,202.50);

\draw[color=drawColor,line cap=round,line join=round,fill opacity=0.00,] (213.95,201.77) --
	(213.59,201.68) --
	(213.43,201.62) --
	(213.27,201.49) --
	(213.19,201.43) --
	(213.04,201.41) --
	(212.85,201.43) --
	(212.72,201.40) --
	(212.60,201.32) --
	(212.48,201.25);

\draw[color=drawColor,line cap=round,line join=round,fill opacity=0.00,] (207.26,200.17) --
	(207.23,200.25) --
	(207.10,200.28) --
	(206.74,200.30) --
	(206.49,200.29) --
	(206.25,200.31) --
	(206.11,200.32) --
	(205.88,200.28) --
	(205.71,200.26) --
	(205.32,200.29) --
	(205.23,200.28) --
	(205.07,200.21) --
	(204.71,200.19) --
	(204.39,200.16) --
	(204.28,200.16);

\draw[color=drawColor,line cap=round,line join=round,fill opacity=0.00,] (207.26,200.17) --
	(207.44,200.06) --
	(207.77,199.96) --
	(208.12,199.95) --
	(208.36,199.95) --
	(208.57,199.97) --
	(208.60,200.01) --
	(208.54,200.09) --
	(208.51,200.15) --
	(208.58,200.22) --
	(208.85,200.38) --
	(209.11,200.60) --
	(209.43,200.75) --
	(209.71,200.81) --
	(210.14,200.82) --
	(210.23,200.87) --
	(210.37,201.01) --
	(210.58,201.10) --
	(210.74,201.16) --
	(210.91,201.20) --
	(211.12,201.27) --
	(211.34,201.36) --
	(211.78,201.44) --
	(212.11,201.44) --
	(212.30,201.34) --
	(212.48,201.25);

\draw[color=drawColor,line cap=round,line join=round,fill opacity=0.00,] (204.28,200.16) --
	(204.17,200.12) --
	(204.09,200.03) --
	(204.03,200.02) --
	(203.97,200.07) --
	(203.93,200.06) --
	(203.90,199.96) --
	(203.83,199.91) --
	(203.74,199.93) --
	(203.71,200.02) --
	(203.62,200.04) --
	(203.56,199.96) --
	(203.49,199.94) --
	(203.35,199.98) --
	(203.23,199.97) --
	(203.10,199.91) --
	(202.92,199.96) --
	(202.82,199.96) --
	(202.66,199.88) --
	(202.58,199.87) --
	(202.54,199.91) --
	(202.54,199.96) --
	(202.46,200.00) --
	(202.37,200.00) --
	(202.31,199.97) --
	(202.29,199.90) --
	(202.21,199.87) --
	(202.14,199.92) --
	(202.06,199.92);

\draw[color=drawColor,line cap=round,line join=round,fill opacity=0.00,] (110.61,168.57) --
	(110.47,168.66) --
	(110.39,168.72) --
	(110.33,168.80) --
	(110.30,168.84) --
	(110.27,168.92) --
	(110.28,168.98) --
	(110.29,169.01) --
	(110.38,169.07) --
	(110.47,169.11) --
	(110.54,169.14) --
	(110.59,169.16) --
	(110.63,169.20) --
	(110.65,169.24) --
	(110.65,169.31) --
	(110.61,169.41) --
	(110.58,169.49) --
	(110.52,169.55) --
	(110.48,169.58) --
	(110.22,169.66) --
	(110.16,169.68) --
	(110.15,169.71) --
	(110.13,169.73) --
	(110.30,169.87) --
	(110.31,169.91) --
	(110.30,169.94) --
	(110.26,169.99) --
	(110.21,170.00) --
	(109.82,169.99) --
	(109.73,170.03) --
	(109.69,170.07) --
	(109.66,170.13) --
	(109.64,170.18) --
	(109.64,170.23) --
	(109.79,170.54) --
	(109.85,170.72) --
	(109.85,170.76) --
	(109.83,170.81) --
	(109.78,170.84) --
	(109.73,170.87) --
	(109.67,170.86) --
	(109.61,170.83) --
	(109.41,170.67) --
	(109.34,170.65) --
	(109.30,170.67) --
	(109.24,170.70) --
	(109.21,170.78) --
	(109.20,170.85) --
	(109.21,171.03) --
	(109.27,171.31) --
	(109.29,171.45) --
	(109.28,171.51) --
	(109.25,171.54) --
	(109.19,171.59) --
	(108.83,171.78) --
	(108.82,171.82) --
	(108.82,171.85) --
	(108.85,171.91) --
	(108.85,171.94) --
	(108.82,172.00) --
	(108.84,172.10) --
	(108.85,172.15) --
	(108.88,172.20) --
	(108.96,172.24) --
	(109.02,172.27) --
	(109.08,172.28) --
	(109.63,172.22) --
	(109.70,172.23) --
	(109.80,172.24) --
	(109.88,172.28) --
	(109.91,172.33) --
	(109.93,172.41) --
	(109.91,172.44) --
	(109.90,172.50) --
	(109.84,172.59) --
	(109.79,172.69) --
	(109.72,172.74) --
	(109.49,172.87) --
	(109.46,172.89) --
	(109.47,172.94) --
	(109.49,172.96) --
	(109.83,173.07) --
	(109.84,173.10) --
	(109.84,173.13) --
	(109.81,173.17) --
	(109.49,173.49) --
	(109.47,173.51) --
	(109.46,173.64) --
	(109.51,173.75) --
	(109.53,173.77) --
	(109.63,173.87) --
	(109.70,173.99) --
	(109.78,174.13) --
	(109.78,174.17) --
	(109.75,174.22) --
	(109.73,174.30) --
	(109.67,174.32) --
	(109.45,174.37) --
	(109.41,174.39) --
	(109.38,174.44) --
	(109.38,174.48) --
	(109.40,174.52) --
	(109.41,174.55) --
	(109.66,174.68) --
	(109.66,174.71) --
	(109.64,174.76) --
	(109.64,174.78) --
	(109.59,174.86) --
	(109.42,174.91) --
	(109.35,174.99) --
	(109.37,175.07) --
	(109.54,175.15) --
	(109.75,175.67) --
	(109.76,175.74) --
	(109.75,175.78) --
	(109.64,175.97) --
	(109.65,176.01) --
	(109.73,176.03) --
	(109.71,176.14) --
	(109.70,176.21) --
	(109.80,176.22) --
	(109.97,176.23) --
	(110.13,176.27) --
	(110.26,176.38) --
	(110.34,176.47) --
	(110.36,176.64) --
	(110.37,176.74) --
	(110.43,176.85) --
	(110.42,176.88) --
	(110.42,176.95) --
	(110.37,177.02) --
	(110.43,177.21) --
	(110.58,177.27) --
	(110.71,177.27) --
	(110.81,177.35) --
	(110.80,177.51) --
	(110.76,177.56) --
	(110.70,177.62) --
	(110.54,177.66) --
	(110.39,177.78) --
	(110.39,177.97) --
	(110.54,178.05) --
	(110.70,178.12) --
	(110.73,178.11) --
	(110.76,178.15) --
	(110.79,178.15) --
	(110.78,178.17) --
	(110.64,178.38) --
	(110.61,178.53) --
	(110.69,178.61) --
	(110.84,178.77) --
	(110.88,178.82) --
	(111.22,178.86) --
	(111.29,178.87) --
	(111.32,178.91) --
	(111.39,179.05) --
	(111.39,179.26) --
	(111.39,179.32) --
	(111.34,179.48) --
	(111.34,179.69) --
	(111.36,179.71) --
	(111.40,179.73) --
	(111.46,179.72) --
	(111.56,179.68) --
	(111.69,179.62) --
	(111.80,179.62) --
	(111.86,179.67) --
	(111.90,179.71) --
	(111.89,179.79) --
	(111.76,180.07) --
	(111.68,180.13) --
	(111.66,180.17) --
	(111.67,180.23) --
	(111.73,180.39) --
	(111.75,180.54) --
	(111.57,180.64) --
	(111.37,180.74) --
	(111.30,180.86) --
	(111.41,180.99) --
	(111.47,181.16) --
	(111.58,181.29) --
	(111.58,181.30) --
	(111.81,181.36) --
	(111.95,181.54) --
	(111.98,181.59) --
	(111.94,181.60) --
	(111.72,181.66) --
	(111.83,181.90) --
	(111.82,181.96) --
	(111.78,182.17) --
	(112.02,182.12) --
	(112.07,182.11) --
	(112.36,182.42) --
	(112.42,182.64) --
	(112.66,182.75) --
	(112.66,182.91) --
	(112.74,183.00) --
	(112.66,183.23) --
	(112.64,183.31) --
	(112.63,183.36) --
	(112.66,183.38) --
	(112.71,183.41) --
	(112.81,183.49) --
	(112.80,183.53) --
	(112.77,183.59) --
	(112.71,183.75) --
	(112.87,184.01) --
	(113.05,184.20) --
	(112.92,184.37) --
	(112.81,184.39) --
	(112.68,184.46) --
	(112.71,184.58) --
	(112.71,184.68) --
	(112.69,184.82) --
	(112.68,185.00) --
	(112.69,185.02) --
	(112.80,185.13) --
	(112.74,185.28) --
	(112.69,185.39) --
	(112.67,185.46) --
	(112.67,185.50) --
	(112.65,185.74) --
	(112.73,185.93) --
	(112.87,186.02) --
	(112.85,186.17) --
	(112.77,186.29) --
	(112.74,186.34) --
	(112.70,186.43) --
	(112.69,186.50) --
	(112.69,186.54) --
	(112.65,186.72) --
	(112.64,186.92) --
	(112.41,186.96) --
	(112.26,187.01) --
	(112.22,187.05) --
	(112.15,187.14) --
	(112.12,187.24) --
	(112.12,187.25) --
	(112.35,187.26) --
	(112.50,187.34) --
	(112.57,187.49) --
	(112.54,187.73) --
	(112.36,187.92) --
	(112.32,188.14) --
	(112.47,188.28) --
	(112.68,188.35) --
	(112.80,188.42) --
	(112.80,188.57) --
	(112.67,188.61) --
	(112.61,188.70) --
	(112.57,188.77) --
	(112.56,188.88) --
	(112.59,188.91) --
	(112.68,189.01) --
	(112.73,189.05) --
	(112.86,189.16) --
	(112.97,189.23) --
	(113.02,189.43) --
	(113.22,189.43) --
	(113.36,189.47) --
	(113.33,189.56) --
	(113.23,189.63) --
	(113.10,189.68) --
	(113.01,189.72) --
	(112.97,189.81) --
	(112.96,189.85) --
	(112.99,189.97) --
	(113.09,190.05) --
	(113.20,190.03) --
	(113.34,190.01) --
	(113.45,190.09) --
	(113.48,190.17) --
	(113.59,190.30) --
	(113.62,190.34) --
	(113.64,190.50) --
	(113.64,190.55) --
	(113.67,190.64) --
	(113.70,190.73) --
	(113.67,190.78) --
	(113.65,190.83) --
	(113.62,190.89) --
	(113.46,190.96) --
	(113.24,190.92) --
	(113.11,191.04) --
	(113.32,191.13) --
	(113.46,191.42) --
	(113.59,191.43) --
	(113.80,191.43) --
	(113.99,191.42) --
	(113.99,191.53) --
	(113.96,191.56) --
	(113.89,191.62) --
	(113.77,191.68) --
	(113.75,191.76) --
	(113.75,191.80) --
	(113.73,191.84) --
	(113.62,191.91) --
	(113.55,191.96) --
	(113.47,192.08) --
	(113.43,192.13) --
	(113.40,192.17) --
	(113.43,192.21) --
	(113.50,192.30) --
	(113.56,192.31) --
	(113.75,192.35) --
	(113.84,192.37) --
	(114.01,192.36) --
	(114.10,192.36) --
	(114.16,192.36) --
	(114.17,192.27) --
	(114.20,192.05) --
	(114.21,191.99) --
	(114.25,191.93) --
	(114.34,191.79) --
	(114.42,191.75) --
	(114.47,191.76) --
	(114.60,191.82) --
	(114.59,191.91) --
	(114.48,191.99) --
	(114.47,192.14) --
	(114.45,192.21) --
	(114.45,192.47) --
	(114.71,192.47) --
	(114.98,192.48) --
	(115.05,192.53) --
	(115.04,192.62) --
	(115.04,192.72) --
	(115.09,192.91) --
	(115.16,193.06) --
	(115.20,193.18) --
	(115.18,193.30) --
	(115.15,193.33) --
	(115.02,193.43) --
	(115.01,193.62) --
	(115.07,193.72) --
	(115.25,193.70) --
	(115.38,193.62) --
	(115.42,193.52) --
	(115.89,193.53) --
	(115.92,193.51) --
	(116.24,193.35) --
	(116.32,193.42) --
	(116.04,193.68) --
	(116.03,193.70) --
	(116.03,193.80) --
	(116.04,193.90) --
	(115.89,193.93) --
	(115.92,193.96) --
	(115.95,193.99) --
	(116.18,194.13) --
	(116.34,194.23) --
	(116.45,194.34) --
	(116.48,194.39) --
	(116.48,194.43) --
	(116.48,194.50) --
	(116.47,194.58) --
	(116.44,194.61) --
	(116.41,194.66) --
	(116.31,194.71) --
	(115.97,194.71) --
	(115.66,194.70) --
	(115.59,194.71) --
	(115.40,194.85) --
	(115.37,194.89) --
	(115.39,194.91) --
	(115.40,194.93) --
	(115.43,194.95) --
	(115.60,194.98) --
	(115.67,195.02) --
	(115.70,195.04) --
	(115.76,195.08) --
	(115.78,195.13) --
	(115.89,195.31) --
	(115.93,195.35) --
	(115.99,195.36) --
	(116.55,195.41) --
	(116.66,195.43) --
	(116.84,195.51) --
	(116.96,195.60) --
	(117.02,195.67) --
	(117.02,195.71) --
	(117.04,195.74) --
	(117.04,195.82) --
	(117.01,195.88) --
	(116.82,196.04) --
	(116.79,196.05) --
	(116.53,196.22) --
	(116.52,196.24) --
	(116.51,196.29) --
	(116.48,196.53) --
	(116.49,196.62) --
	(116.50,196.70) --
	(116.53,196.76) --
	(116.58,196.82) --
	(116.63,196.84) --
	(116.69,196.85) --
	(116.71,196.84) --
	(116.98,196.75) --
	(117.08,196.74) --
	(117.16,196.75) --
	(117.29,196.78) --
	(117.62,196.93) --
	(117.67,196.95) --
	(117.71,197.02) --
	(117.73,197.07) --
	(117.73,197.14) --
	(117.75,197.22) --
	(117.73,197.26) --
	(117.69,197.32) --
	(117.42,197.58) --
	(117.39,197.63) --
	(117.38,197.67) --
	(117.38,197.74) --
	(117.41,197.79) --
	(117.47,197.88) --
	(117.52,197.91) --
	(117.58,197.94) --
	(117.65,197.96) --
	(117.74,197.93) --
	(118.19,197.77) --
	(118.24,197.76) --
	(118.30,197.76) --
	(118.36,197.75) --
	(118.42,197.78) --
	(118.49,197.81) --
	(118.52,197.85) --
	(118.55,197.88) --
	(118.55,197.91) --
	(118.55,197.93) --
	(118.51,197.98) --
	(118.48,198.00) --
	(118.41,198.03) --
	(118.02,198.06) --
	(117.98,198.09) --
	(117.94,198.11) --
	(117.89,198.15) --
	(117.88,198.19) --
	(117.88,198.22) --
	(117.91,198.25) --
	(117.93,198.29) --
	(117.99,198.31) --
	(118.33,198.37) --
	(118.42,198.44) --
	(118.44,198.49) --
	(118.46,198.65) --
	(118.46,198.73) --
	(118.46,198.78) --
	(118.49,198.84) --
	(118.54,198.89) --
	(118.63,198.95) --
	(118.67,198.96) --
	(118.72,198.97) --
	(118.89,198.91) --
	(118.97,198.90) --
	(119.01,198.88) --
	(119.06,198.89) --
	(119.12,198.92) --
	(119.32,199.09) --
	(119.50,199.24) --
	(119.54,199.30) --
	(119.59,199.37) --
	(119.59,199.40) --
	(119.60,199.56) --
	(119.74,199.79) --
	(119.85,200.08) --
	(119.90,200.25) --
	(120.00,200.45) --
	(120.13,200.50) --
	(120.19,200.41) --
	(120.20,200.37) --
	(120.31,200.24) --
	(120.49,200.26) --
	(121.04,200.16) --
	(121.16,200.33);

\draw[color=drawColor,line cap=round,line join=round,fill opacity=0.00,] (189.52,195.57) --
	(189.67,195.64) --
	(189.94,195.67) --
	(190.06,195.63) --
	(190.19,195.72) --
	(190.22,195.78) --
	(190.13,195.87) --
	(190.27,195.96) --
	(190.48,195.91) --
	(190.60,195.85) --
	(190.64,196.00) --
	(190.71,196.07) --
	(190.96,196.22) --
	(191.17,196.13) --
	(191.28,196.08) --
	(191.56,196.07) --
	(191.93,196.17) --
	(192.37,196.28) --
	(192.50,196.56) --
	(192.57,196.62) --
	(192.66,196.56) --
	(192.80,196.45) --
	(193.03,196.43) --
	(193.15,196.44) --
	(193.39,196.44) --
	(193.51,196.44) --
	(193.65,196.40) --
	(193.82,196.36);

\draw[color=drawColor,line cap=round,line join=round,fill opacity=0.00,] (202.06,199.92) --
	(202.02,199.91) --
	(201.99,199.87) --
	(201.93,199.85) --
	(201.87,199.90) --
	(201.83,200.00) --
	(201.75,200.02) --
	(201.71,200.01) --
	(201.71,199.94) --
	(201.74,199.87) --
	(201.71,199.82) --
	(201.65,199.82) --
	(201.41,199.92) --
	(201.31,199.90) --
	(201.19,199.81) --
	(200.98,199.67) --
	(200.82,199.58) --
	(200.50,199.32) --
	(200.11,199.08) --
	(199.98,199.00) --
	(199.77,198.90) --
	(199.51,198.71) --
	(199.24,198.46) --
	(199.10,198.20) --
	(199.05,198.06) --
	(198.97,197.97) --
	(198.63,197.91) --
	(198.46,197.82) --
	(198.33,197.69) --
	(198.20,197.63) --
	(198.11,197.44) --
	(198.02,197.36) --
	(197.86,197.30) --
	(197.73,197.22) --
	(197.66,196.99) --
	(197.58,196.70) --
	(197.42,196.61) --
	(197.16,196.59) --
	(196.95,196.47) --
	(196.91,196.44) --
	(196.85,196.23) --
	(196.70,196.07) --
	(196.23,195.69) --
	(196.10,195.51) --
	(196.10,195.34) --
	(196.00,195.23);

\draw[color=drawColor,line cap=round,line join=round,fill opacity=0.00,] (193.82,196.36) --
	(193.86,196.34) --
	(194.02,196.36) --
	(194.11,196.33) --
	(194.16,196.29) --
	(194.14,196.16) --
	(194.16,196.11) --
	(194.19,196.09) --
	(194.23,196.11) --
	(194.34,196.19) --
	(194.39,196.22) --
	(194.42,196.21) --
	(194.44,196.15) --
	(194.41,196.03) --
	(194.52,195.91) --
	(194.64,195.80) --
	(194.76,195.54) --
	(194.86,195.43) --
	(195.01,195.30) --
	(195.25,195.20) --
	(195.48,195.13) --
	(195.74,195.18) --
	(195.94,195.25) --
	(196.00,195.23);

\draw[color=drawColor,line cap=round,line join=round,fill opacity=0.00,] (189.52,195.57) --
	(189.47,195.47) --
	(189.25,195.35) --
	(188.89,195.17) --
	(188.79,195.02) --
	(188.53,194.85) --
	(188.34,194.67) --
	(188.34,194.48) --
	(188.27,194.38) --
	(188.30,194.15) --
	(188.13,194.01) --
	(187.95,193.94) --
	(187.73,193.68) --
	(187.54,193.57) --
	(187.29,193.48) --
	(187.09,193.22) --
	(187.00,193.01) --
	(186.89,192.83) --
	(186.71,192.65) --
	(186.55,192.57) --
	(186.55,192.48) --
	(186.66,192.42) --
	(186.69,192.37) --
	(186.58,192.29) --
	(186.60,192.10) --
	(186.53,191.97) --
	(186.46,191.92) --
	(186.38,191.93) --
	(186.34,191.89);

\draw[color=drawColor,line cap=round,line join=round,fill opacity=0.00,] (186.34,191.89) --
	(186.33,191.78) --
	(186.25,191.70) --
	(186.09,191.66) --
	(186.05,191.64) --
	(186.03,191.59) --
	(186.02,191.51) --
	(185.96,191.48) --
	(185.74,191.46) --
	(185.43,191.40) --
	(185.39,191.34) --
	(185.41,191.16) --
	(185.41,190.99) --
	(185.44,190.87) --
	(185.52,190.79) --
	(185.53,190.70) --
	(185.52,190.61);

\draw[color=drawColor,line cap=round,line join=round,fill opacity=0.00,] (185.52,190.61) --
	(185.42,190.52) --
	(185.42,190.43) --
	(185.43,190.35) --
	(185.55,190.30) --
	(185.61,190.21) --
	(185.66,190.14) --
	(185.70,190.04);

\draw[color=drawColor,line cap=round,line join=round,fill opacity=0.00,] (176.12,188.16) --
	(175.96,188.10) --
	(175.69,188.12) --
	(175.47,188.14) --
	(175.15,188.01) --
	(174.92,187.92) --
	(174.47,187.94) --
	(174.22,187.99) --
	(174.02,188.01) --
	(173.87,188.00);

\draw[color=drawColor,line cap=round,line join=round,fill opacity=0.00,] (185.70,190.04) --
	(185.57,189.96) --
	(185.32,190.01) --
	(185.16,189.98) --
	(184.97,189.84) --
	(184.78,189.79) --
	(184.53,189.57) --
	(184.24,189.54) --
	(184.11,189.47) --
	(183.90,189.34) --
	(183.57,189.21) --
	(183.32,189.09) --
	(183.16,188.99) --
	(182.92,188.91) --
	(182.67,188.72) --
	(182.43,188.64) --
	(182.12,188.44) --
	(181.81,188.29) --
	(181.49,188.11) --
	(181.16,187.94) --
	(180.88,187.95) --
	(180.57,188.06) --
	(180.30,188.00) --
	(179.48,187.80) --
	(179.32,187.66) --
	(179.26,187.61) --
	(179.06,187.61);

\draw[color=drawColor,line cap=round,line join=round,fill opacity=0.00,] (179.06,187.61) --
	(178.89,187.83) --
	(178.74,187.88) --
	(178.60,187.94) --
	(178.20,187.95) --
	(177.64,187.90) --
	(177.35,187.87) --
	(177.06,187.98) --
	(176.57,188.16) --
	(176.30,188.17) --
	(176.12,188.16);

\draw[color=drawColor,line cap=round,line join=round,fill opacity=0.00,] (173.87,188.00) --
	(173.65,187.87) --
	(173.54,187.76) --
	(173.21,187.70) --
	(172.88,187.61) --
	(172.29,187.50) --
	(172.06,187.44) --
	(171.92,187.46);

\draw[color=drawColor,line cap=round,line join=round,fill opacity=0.00,] (171.92,187.46) --
	(171.75,187.48) --
	(171.32,187.42) --
	(171.10,187.40) --
	(170.98,187.37) --
	(170.69,187.35) --
	(170.44,187.30) --
	(170.18,187.22) --
	(169.67,187.17) --
	(169.44,187.19) --
	(169.08,187.14) --
	(168.71,187.03) --
	(168.49,186.94) --
	(168.13,186.89) --
	(167.92,186.87) --
	(167.36,186.91) --
	(166.98,186.93) --
	(166.69,186.88) --
	(166.50,186.83);

\draw[color=drawColor,line cap=round,line join=round,fill opacity=0.00,] (166.50,186.83) --
	(165.90,186.61) --
	(165.80,186.52) --
	(165.66,186.30) --
	(165.50,186.14) --
	(165.40,186.08) --
	(165.31,186.09) --
	(165.05,186.34) --
	(164.95,186.36) --
	(164.79,186.32) --
	(164.67,186.20) --
	(164.47,186.28) --
	(164.27,186.29) --
	(163.90,186.12) --
	(163.75,185.99) --
	(163.62,185.99) --
	(163.56,186.00) --
	(163.49,185.91) --
	(163.39,185.84) --
	(163.01,185.76) --
	(162.69,185.64) --
	(162.34,185.60) --
	(162.16,185.48) --
	(161.92,185.41) --
	(161.80,185.36) --
	(161.60,185.14) --
	(161.40,185.07);

\draw[color=drawColor,line cap=round,line join=round,fill opacity=0.00,] (157.27,182.18) --
	(157.46,182.27) --
	(157.51,182.34) --
	(157.63,182.40) --
	(157.77,182.46) --
	(157.87,182.47) --
	(157.92,182.50) --
	(157.93,182.59) --
	(158.04,182.69) --
	(158.20,182.76) --
	(158.44,182.80) --
	(158.55,182.89) --
	(158.78,183.01) --
	(159.06,183.16) --
	(159.27,183.41) --
	(159.46,183.57) --
	(159.63,183.63) --
	(159.83,183.73) --
	(159.93,183.80) --
	(160.05,183.85) --
	(160.24,183.97) --
	(160.27,184.02) --
	(160.28,184.09) --
	(160.34,184.17) --
	(160.48,184.29) --
	(160.62,184.35) --
	(160.70,184.39) --
	(160.69,184.51) --
	(160.76,184.59) --
	(160.84,184.63) --
	(160.92,184.66) --
	(161.09,184.66) --
	(161.11,184.69) --
	(161.10,184.76) --
	(161.09,184.87) --
	(161.14,184.93) --
	(161.24,184.96) --
	(161.34,185.02) --
	(161.40,185.07);

\draw[color=drawColor,line cap=round,line join=round,fill opacity=0.00,] (157.27,182.18) --
	(157.04,181.98) --
	(156.95,181.85) --
	(156.85,181.79) --
	(156.80,181.77) --
	(156.74,181.69) --
	(156.79,181.57) --
	(156.76,181.43) --
	(156.78,181.30) --
	(156.74,181.13) --
	(156.60,180.98) --
	(156.43,180.87) --
	(156.31,180.80) --
	(156.23,180.69) --
	(156.22,180.60) --
	(156.12,180.48) --
	(156.11,180.37) --
	(156.21,180.19) --
	(156.26,180.08) --
	(156.18,180.02) --
	(155.97,180.00) --
	(155.95,179.97) --
	(155.91,179.94) --
	(155.90,179.84) --
	(156.02,179.77) --
	(156.16,179.69) --
	(156.19,179.52) --
	(156.17,179.30) --
	(156.04,179.11) --
	(155.87,179.02) --
	(155.65,178.92) --
	(155.60,178.73);

\draw[color=drawColor,line cap=round,line join=round,fill opacity=0.00,] (155.60,178.73) --
	(155.39,178.52) --
	(155.09,178.27) --
	(155.05,178.09);

\draw[color=drawColor,line cap=round,line join=round,fill opacity=0.00,] (130.92,169.56) --
	(130.91,169.64) --
	(130.87,169.82) --
	(130.84,169.98) --
	(130.84,170.02) --
	(130.85,170.05) --
	(130.85,170.07) --
	(130.99,170.25) --
	(130.99,170.27) --
	(130.99,170.29) --
	(130.95,170.38) --
	(130.88,170.50) --
	(130.84,170.59) --
	(130.96,170.64) --
	(131.07,170.72) --
	(131.07,170.74) --
	(131.13,170.81) --
	(131.22,170.96) --
	(131.34,171.11) --
	(131.39,171.19) --
	(131.53,171.37) --
	(131.60,171.43) --
	(131.69,171.47) --
	(131.75,171.48) --
	(131.80,171.48) --
	(131.87,171.47) --
	(131.91,171.49) --
	(131.97,171.54) --
	(132.03,171.56) --
	(132.14,171.59) --
	(132.24,171.65) --
	(132.26,171.67) --
	(132.27,171.70) --
	(132.24,171.76) --
	(132.03,171.83) --
	(131.90,171.88) --
	(131.83,171.94) --
	(131.80,171.98) --
	(131.80,172.01) --
	(131.80,172.05) --
	(131.82,172.09) --
	(131.85,172.13) --
	(131.92,172.15) --
	(132.12,172.23) --
	(132.26,172.29) --
	(132.34,172.37) --
	(132.36,172.41) --
	(132.37,172.44) --
	(132.36,172.48) --
	(132.35,172.52) --
	(132.29,172.60) --
	(132.23,172.65) --
	(132.20,172.82) --
	(132.15,172.86) --
	(132.00,172.97) --
	(131.99,173.02) --
	(131.96,173.10) --
	(131.93,173.15) --
	(131.76,173.22) --
	(131.73,173.29) --
	(131.74,173.34) --
	(131.81,173.42) --
	(131.87,173.45) --
	(132.02,173.47) --
	(132.23,173.42) --
	(132.51,173.29) --
	(132.55,173.21) --
	(132.63,173.13) --
	(132.69,173.09) --
	(132.76,173.06) --
	(132.83,173.06) --
	(132.91,173.06) --
	(132.95,173.08) --
	(133.16,173.24) --
	(133.45,173.49) --
	(133.57,173.64) --
	(133.60,173.65) --
	(133.78,173.66) --
	(133.81,173.66) --
	(133.84,173.70) --
	(133.84,173.88) --
	(133.86,173.92) --
	(133.91,173.95) --
	(133.96,173.95) --
	(134.01,173.96) --
	(134.08,173.95) --
	(134.18,173.91) --
	(134.22,173.90) --
	(134.27,173.91) --
	(134.31,173.92) --
	(134.36,173.96) --
	(134.39,174.04) --
	(134.44,174.10) --
	(134.56,174.17) --
	(134.56,174.21) --
	(134.53,174.25) --
	(134.39,174.39) --
	(134.33,174.42) --
	(134.32,174.47) --
	(134.34,174.54) --
	(134.37,174.58) --
	(134.53,174.63) --
	(134.59,174.66) --
	(134.69,174.71) --
	(134.77,174.77) --
	(134.82,174.84) --
	(134.83,174.88) --
	(134.82,174.97) --
	(134.80,175.04) --
	(134.64,175.23) --
	(134.64,175.29) --
	(134.66,175.33) --
	(134.79,175.57) --
	(134.82,175.60) --
	(134.83,175.62) --
	(134.92,175.63) --
	(135.10,175.60) --
	(135.15,175.61) --
	(135.28,175.74) --
	(135.26,175.78) --
	(135.19,175.86) --
	(135.21,175.92) --
	(135.39,176.07) --
	(135.42,176.32) --
	(135.55,176.45) --
	(135.76,176.55) --
	(135.82,176.59) --
	(135.85,176.64) --
	(135.87,176.78) --
	(135.89,176.87) --
	(135.95,176.95) --
	(136.07,177.01) --
	(136.21,177.05) --
	(136.34,177.06) --
	(136.41,177.06) --
	(136.49,177.05) --
	(136.55,177.00) --
	(136.57,176.98) --
	(136.60,176.97) --
	(136.69,176.96) --
	(136.77,176.96) --
	(136.87,176.96) --
	(136.95,176.98) --
	(137.01,177.04) --
	(137.11,177.13) --
	(137.22,177.25) --
	(137.29,177.32) --
	(137.36,177.49) --
	(137.49,177.55) --
	(137.63,177.55) --
	(137.81,177.60) --
	(137.91,177.67) --
	(137.96,177.80) --
	(138.03,177.93) --
	(138.13,177.95) --
	(138.41,177.95) --
	(138.58,177.92) --
	(138.80,177.96) --
	(138.95,178.02) --
	(138.99,178.14) --
	(138.96,178.32) --
	(139.06,178.38) --
	(139.37,178.42) --
	(139.48,178.48) --
	(139.53,178.55) --
	(139.55,178.65) --
	(139.53,178.91) --
	(139.60,179.14) --
	(139.72,179.24) --
	(139.85,179.27) --
	(139.95,179.24) --
	(139.97,179.14) --
	(139.95,178.94) --
	(139.95,178.81) --
	(140.03,178.73) --
	(140.11,178.69) --
	(140.25,178.73) --
	(140.47,178.78) --
	(140.58,178.79) --
	(140.79,178.77) --
	(140.93,178.71) --
	(141.02,178.66) --
	(141.08,178.57) --
	(141.13,178.39) --
	(141.19,178.26) --
	(141.24,178.22) --
	(141.33,178.22) --
	(141.54,178.25) --
	(141.63,178.23) --
	(141.78,178.14) --
	(141.91,178.14) --
	(142.23,178.18) --
	(142.38,178.23) --
	(142.51,178.35) --
	(142.62,178.39) --
	(142.77,178.35) --
	(142.90,178.28) --
	(143.03,178.19) --
	(143.13,178.18) --
	(143.20,178.18) --
	(143.38,178.02) --
	(143.69,177.91) --
	(143.93,177.92) --
	(144.05,177.83) --
	(144.35,177.60) --
	(144.62,177.40) --
	(144.87,177.29) --
	(145.07,177.26) --
	(145.14,177.22) --
	(145.34,177.08) --
	(145.44,177.05) --
	(145.51,177.09) --
	(145.57,177.12) --
	(146.11,176.83) --
	(146.27,176.73) --
	(146.42,176.61) --
	(146.57,176.58) --
	(146.72,176.56) --
	(146.88,176.47) --
	(147.24,176.20) --
	(147.37,176.23) --
	(147.46,176.23) --
	(147.53,176.19) --
	(147.59,176.12) --
	(147.68,176.11) --
	(147.79,176.09) --
	(148.04,175.93) --
	(148.19,175.80) --
	(148.22,175.70) --
	(148.47,175.50);

\draw[color=drawColor,line cap=round,line join=round,fill opacity=0.00,] (110.61,168.57) --
	(110.54,168.49) --
	(110.51,168.46) --
	(110.48,168.44) --
	(110.44,168.44) --
	(110.36,168.43) --
	(110.33,168.41) --
	(110.26,168.27) --
	(110.24,168.26) --
	(110.23,168.25) --
	(110.21,168.24) --
	(110.11,168.19) --
	(110.04,168.14) --
	(110.02,168.13) --
	(109.69,168.14) --
	(109.67,168.12) --
	(109.66,168.10) --
	(109.64,168.09) --
	(109.64,168.07) --
	(109.65,167.97) --
	(109.65,167.93) --
	(109.63,167.81) --
	(109.47,167.82) --
	(109.42,167.81) --
	(109.39,167.81) --
	(109.36,167.77) --
	(109.33,167.76) --
	(109.30,167.76) --
	(109.27,167.75) --
	(109.24,167.75) --
	(109.21,167.78) --
	(109.15,167.81) --
	(109.04,167.94) --
	(109.03,167.96) --
	(108.98,167.96) --
	(108.95,167.95) --
	(108.91,167.94) --
	(108.89,167.93) --
	(108.88,167.89) --
	(108.88,167.87) --
	(108.92,167.81) --
	(108.96,167.73) --
	(108.98,167.72) --
	(108.95,167.72) --
	(108.80,167.74) --
	(108.71,167.64) --
	(108.71,167.54) --
	(108.71,167.53) --
	(108.70,167.52) --
	(108.68,167.50) --
	(108.62,167.51) --
	(108.61,167.49) --
	(108.58,167.55) --
	(108.58,167.56) --
	(108.55,167.57) --
	(108.54,167.58) --
	(108.52,167.60) --
	(108.51,167.59) --
	(108.48,167.57) --
	(108.39,167.54) --
	(108.32,167.51) --
	(108.24,167.47) --
	(108.24,167.45) --
	(108.21,167.46) --
	(108.15,167.46) --
	(108.13,167.46) --
	(108.11,167.47) --
	(108.10,167.48) --
	(108.08,167.49) --
	(108.04,167.56) --
	(108.03,167.57) --
	(108.01,167.57) --
	(107.92,167.55) --
	(107.91,167.56) --
	(107.88,167.64) --
	(107.87,167.65) --
	(107.84,167.66) --
	(107.83,167.66) --
	(107.81,167.64) --
	(107.77,167.61) --
	(107.75,167.61) --
	(107.65,167.64) --
	(107.62,167.63) --
	(107.58,167.60) --
	(107.55,167.59) --
	(107.53,167.59) --
	(107.43,167.63) --
	(107.40,167.65) --
	(107.39,167.65) --
	(107.37,167.64) --
	(107.24,167.53) --
	(107.21,167.50) --
	(107.18,167.50) --
	(107.14,167.50) --
	(106.98,167.52) --
	(106.95,167.51) --
	(106.92,167.51) --
	(106.90,167.50) --
	(106.80,167.43) --
	(106.76,167.40) --
	(106.73,167.40) --
	(106.71,167.40) --
	(106.70,167.40) --
	(106.54,167.48) --
	(106.51,167.47) --
	(106.48,167.47) --
	(106.42,167.39) --
	(106.38,167.39) --
	(106.29,167.40) --
	(106.16,167.46) --
	(106.06,167.62) --
	(106.02,167.64) --
	(105.92,167.66) --
	(105.89,167.66) --
	(105.78,167.80) --
	(105.76,167.80) --
	(105.62,167.79) --
	(105.60,167.81) --
	(105.60,167.82) --
	(105.58,167.93) --
	(105.55,167.96) --
	(105.52,167.97) --
	(105.40,168.04) --
	(105.33,168.05) --
	(105.22,168.09) --
	(105.04,168.16) --
	(105.03,168.19) --
	(104.99,168.37) --
	(104.96,168.38) --
	(104.86,168.39) --
	(104.84,168.40) --
	(104.84,168.44) --
	(104.83,168.48) --
	(104.82,168.51) --
	(104.77,168.53) --
	(104.77,168.55) --
	(104.76,168.60) --
	(104.76,168.66) --
	(104.79,168.70) --
	(104.84,168.78) --
	(104.85,168.81) --
	(104.84,168.82) --
	(104.81,168.82) --
	(104.78,168.96) --
	(104.77,168.97) --
	(104.70,168.98) --
	(104.65,169.01) --
	(104.58,169.07) --
	(104.58,169.09) --
	(104.57,169.18) --
	(104.57,169.19) --
	(104.56,169.21) --
	(104.53,169.22) --
	(104.38,169.24) --
	(104.35,169.26) --
	(104.33,169.29) --
	(104.31,169.33) --
	(104.31,169.35) --
	(104.41,169.38) --
	(104.41,169.39) --
	(104.42,169.40) --
	(104.36,169.48) --
	(104.31,169.49) --
	(104.26,169.48) --
	(104.21,169.49) --
	(104.17,169.52) --
	(104.04,169.58) --
	(104.03,169.59) --
	(104.06,169.67) --
	(104.06,169.69) --
	(103.99,169.86) --
	(103.97,169.92) --
	(104.01,169.97) --
	(104.17,170.06) --
	(104.20,170.10) --
	(104.18,170.15) --
	(104.17,170.22) --
	(104.16,170.27) --
	(104.14,170.30) --
	(104.03,170.45) --
	(104.02,170.49) --
	(104.03,170.50) --
	(104.11,170.53) --
	(104.11,170.54) --
	(104.11,170.55) --
	(104.11,170.60) --
	(104.08,170.61) --
	(104.06,170.61) --
	(104.05,170.61) --
	(104.00,170.59) --
	(103.98,170.58) --
	(103.95,170.59) --
	(103.93,170.61) --
	(103.98,170.71) --
	(104.05,170.79) --
	(104.05,170.82) --
	(104.04,170.93) --
	(104.06,170.98) --
	(104.09,171.08) --
	(104.12,171.10) --
	(104.13,171.13) --
	(104.19,171.17) --
	(104.22,171.21) --
	(104.22,171.24) --
	(104.19,171.49) --
	(104.17,171.51) --
	(104.14,171.54) --
	(104.11,171.57) --
	(104.10,171.58) --
	(103.98,171.58) --
	(103.94,171.60) --
	(103.90,171.62) --
	(103.91,171.65) --
	(103.93,171.67) --
	(103.96,171.71) --
	(103.96,171.73) --
	(103.94,171.74) --
	(103.86,171.74) --
	(103.83,171.75) --
	(103.81,171.82) --
	(103.78,171.85) --
	(103.66,171.97) --
	(103.63,171.99) --
	(103.66,172.06) --
	(103.66,172.08) --
	(103.64,172.11) --
	(103.59,172.14) --
	(103.28,172.23) --
	(103.22,172.24) --
	(103.24,172.29) --
	(103.33,172.49) --
	(103.35,172.50) --
	(103.32,172.63) --
	(103.31,172.64) --
	(103.35,172.67) --
	(103.35,172.69) --
	(103.32,172.79) --
	(103.34,172.81) --
	(103.35,172.82) --
	(103.38,172.85) --
	(103.40,172.87) --
	(103.40,172.92) --
	(103.41,172.93) --
	(103.53,172.95) --
	(103.55,172.98) --
	(103.59,173.03) --
	(103.58,173.05) --
	(103.58,173.08) --
	(103.54,173.10) --
	(103.52,173.13) --
	(103.52,173.14) --
	(103.51,173.14) --
	(103.48,173.16) --
	(103.46,173.14) --
	(103.42,173.08) --
	(103.35,173.09) --
	(103.32,173.09) --
	(103.27,173.08) --
	(103.20,173.00) --
	(103.17,172.98) --
	(103.14,172.98) --
	(103.11,173.00) --
	(103.10,173.00) --
	(103.02,173.09) --
	(102.88,173.23) --
	(102.75,173.22) --
	(102.69,173.24) --
	(102.66,173.27) --
	(102.66,173.29) --
	(102.68,173.32) --
	(102.71,173.38) --
	(102.71,173.42) --
	(102.70,173.43) --
	(102.67,173.45) --
	(102.45,173.38) --
	(102.40,173.39) --
	(102.37,173.39) --
	(102.35,173.43) --
	(102.32,173.43) --
	(102.18,173.41) --
	(102.16,173.44) --
	(102.14,173.45) --
	(102.13,173.45) --
	(102.10,173.44) --
	(102.07,173.40) --
	(102.04,173.37) --
	(101.99,173.36) --
	(101.94,173.45) --
	(101.91,173.45) --
	(101.86,173.47) --
	(101.85,173.49) --
	(101.65,173.63) --
	(101.62,173.64) --
	(101.55,173.66) --
	(101.50,173.69) --
	(101.49,173.72) --
	(101.49,173.74) --
	(101.51,173.77) --
	(101.60,173.89) --
	(101.63,173.92) --
	(101.74,173.94) --
	(101.95,174.00) --
	(102.05,174.02) --
	(102.07,174.05) --
	(102.08,174.09) --
	(102.08,174.11) --
	(102.26,174.30) --
	(102.26,174.34) --
	(102.24,174.46) --
	(102.27,174.53) --
	(102.27,174.57) --
	(102.25,174.57) --
	(102.18,174.59) --
	(102.17,174.59) --
	(102.04,174.68) --
	(101.99,174.68) --
	(101.93,174.68) --
	(101.85,174.65) --
	(101.66,174.63) --
	(101.63,174.62) --
	(101.39,174.49) --
	(101.36,174.47) --
	(101.37,174.38) --
	(101.33,174.37) --
	(101.31,174.36) --
	(101.24,174.38) --
	(101.21,174.37) --
	(101.18,174.36) --
	(101.11,174.28) --
	(101.09,174.27) --
	(101.05,174.26) --
	(101.00,174.26) --
	(100.99,174.27) --
	(100.85,174.35) --
	(100.82,174.35) --
	(100.57,174.31) --
	(100.52,174.28) --
	(100.51,174.25) --
	(100.51,174.22) --
	(100.54,174.20) --
	(100.58,174.18) --
	(100.61,174.15) --
	(100.61,174.13) --
	(100.61,174.11) --
	(100.46,174.03) --
	(100.46,174.02) --
	(100.50,173.96) --
	(100.52,173.94) --
	(100.49,173.92) --
	(100.36,173.94) --
	(100.34,173.93) --
	(100.24,173.89) --
	(100.21,173.89) --
	(100.19,173.90) --
	(100.15,174.05) --
	(100.14,174.17) --
	(100.13,174.19) --
	(100.07,174.18) --
	(100.09,174.04) --
	(100.08,174.02) --
	(100.05,174.03) --
	( 99.93,174.02) --
	( 99.79,174.02) --
	( 99.78,174.15) --
	( 99.76,174.16) --
	( 99.75,174.17) --
	( 99.70,174.20) --
	( 99.67,174.19) --
	( 99.64,174.19) --
	( 99.63,174.19) --
	( 99.60,174.06) --
	( 99.58,174.04) --
	( 99.55,174.04) --
	( 99.51,174.05) --
	( 99.50,174.08) --
	( 99.50,174.09) --
	( 99.51,174.20) --
	( 99.51,174.22) --
	( 99.48,174.23) --
	( 99.46,174.23) --
	( 99.44,174.23) --
	( 99.40,174.19) --
	( 99.32,174.16) --
	( 99.29,174.16) --
	( 99.28,174.17) --
	( 99.26,174.18) --
	( 99.27,174.20) --
	( 99.24,174.36) --
	( 99.22,174.38) --
	( 99.21,174.40) --
	( 99.15,174.44) --
	( 99.10,174.47) --
	( 98.97,174.56) --
	( 98.86,174.63) --
	( 98.85,174.65) --
	( 98.82,174.71) --
	( 98.82,174.80) --
	( 98.82,174.83) --
	( 98.81,174.86) --
	( 98.74,174.87) --
	( 98.74,174.88) --
	( 98.73,174.98) --
	( 98.70,175.00) --
	( 98.67,175.03) --
	( 98.61,175.01) --
	( 98.49,175.02) --
	( 98.46,175.02) --
	( 98.43,175.08) --
	( 98.41,175.10) --
	( 98.33,175.13) --
	( 98.32,175.14) --
	( 98.25,175.22) --
	( 98.22,175.23) --
	( 98.12,175.22) --
	( 98.09,175.23) --
	( 98.03,175.32) --
	( 98.03,175.33) --
	( 98.02,175.35) --
	( 97.93,175.41) --
	( 97.84,175.49) --
	( 97.81,175.50) --
	( 97.77,175.53) --
	( 97.64,175.55) --
	( 97.61,175.55) --
	( 97.50,175.73) --
	( 97.48,175.74) --
	( 97.40,175.81) --
	( 97.38,175.82) --
	( 97.25,176.00) --
	( 97.24,176.00) --
	( 97.21,176.03) --
	( 97.12,176.04) --
	( 97.02,176.07) --
	( 97.01,176.08) --
	( 96.99,176.08) --
	( 96.90,176.07) --
	( 96.88,176.07) --
	( 96.85,176.09) --
	( 96.63,176.13) --
	( 96.61,176.13) --
	( 96.57,176.10) --
	( 96.55,176.10) --
	( 96.52,176.09) --
	( 96.38,176.10) --
	( 96.35,176.10) --
	( 96.34,176.12) --
	( 96.31,176.14) --
	( 96.25,176.15) --
	( 96.16,176.19) --
	( 96.12,176.20) --
	( 96.06,176.19) --
	( 95.98,176.16) --
	( 95.94,176.17) --
	( 95.93,176.16) --
	( 95.90,176.17) --
	( 95.84,176.25) --
	( 95.71,176.24) --
	( 95.58,176.31) --
	( 95.55,176.30) --
	( 95.50,176.26) --
	( 95.47,176.24) --
	( 95.45,176.25) --
	( 95.45,176.27) --
	( 95.42,176.36) --
	( 95.40,176.38) --
	( 95.30,176.38) --
	( 95.29,176.39) --
	( 95.26,176.41) --
	( 95.26,176.44) --
	( 95.30,176.54) --
	( 95.29,176.57) --
	( 95.28,176.58) --
	( 95.25,176.59) --
	( 95.20,176.61) --
	( 95.11,176.52) --
	( 94.92,176.40) --
	( 94.85,176.39) --
	( 94.85,176.37) --
	( 94.74,176.29) --
	( 94.71,176.27) --
	( 94.57,176.28) --
	( 94.54,176.27) --
	( 94.51,176.22) --
	( 94.49,176.21) --
	( 94.46,176.20) --
	( 94.45,176.21) --
	( 94.36,176.23) --
	( 94.33,176.23) --
	( 94.29,176.23) --
	( 94.25,176.21) --
	( 94.20,176.22) --
	( 94.17,176.21) --
	( 94.03,176.28) --
	( 94.00,176.28) --
	( 93.98,176.27) --
	( 93.95,176.23) --
	( 93.94,176.22) --
	( 93.92,176.22) --
	( 93.90,176.22) --
	( 93.79,176.32) --
	( 93.75,176.33) --
	( 93.69,176.34) --
	( 93.65,176.34) --
	( 93.58,176.33) --
	( 93.55,176.31) --
	( 93.50,176.25) --
	( 93.46,176.22) --
	( 93.40,176.20) --
	( 93.34,176.22) --
	( 93.28,176.23) --
	( 93.22,176.28) --
	( 92.96,176.45) --
	( 92.95,176.47) --
	( 92.92,176.64) --
	( 92.91,176.65) --
	( 92.83,176.65) --
	( 92.82,176.66) --
	( 92.69,176.75) --
	( 92.62,176.78) --
	( 92.60,176.79) --
	( 92.57,176.85) --
	( 92.55,176.87) --
	( 92.53,176.88) --
	( 92.47,176.88) --
	( 92.46,176.88) --
	( 92.40,176.95) --
	( 92.39,176.95) --
	( 92.27,176.92) --
	( 92.25,176.94) --
	( 92.17,177.00) --
	( 92.14,177.02) --
	( 91.98,177.01) --
	( 91.96,177.01) --
	( 91.92,177.03) --
	( 91.90,177.05) --
	( 91.83,177.05) --
	( 91.80,177.06) --
	( 91.76,177.11) --
	( 91.70,177.11) --
	( 91.69,177.12) --
	( 91.61,177.22) --
	( 91.60,177.23) --
	( 91.54,177.23) --
	( 91.51,177.24) --
	( 91.50,177.26) --
	( 91.47,177.30) --
	( 91.44,177.39) --
	( 91.44,177.41) --
	( 91.43,177.42) --
	( 91.34,177.44) --
	( 91.33,177.45) --
	( 91.30,177.55) --
	( 91.30,177.57) --
	( 91.28,177.58) --
	( 91.15,177.56) --
	( 91.14,177.57) --
	( 91.11,177.65) --
	( 91.09,177.66) --
	( 90.99,177.63) --
	( 90.98,177.63) --
	( 90.96,177.64) --
	( 90.95,177.65) --
	( 90.96,177.75) --
	( 90.96,177.76) --
	( 90.95,177.79) --
	( 90.92,177.80) --
	( 90.91,177.81) --
	( 90.88,177.79) --
	( 90.83,177.76) --
	( 90.82,177.77) --
	( 90.73,177.77) --
	( 90.76,177.82) --
	( 90.78,177.85) --
	( 90.76,177.89);

\draw[color=drawColor,line cap=round,line join=round,fill opacity=0.00,] (148.47,175.50) --
	(148.67,175.38) --
	(148.79,175.38) --
	(149.16,175.39) --
	(149.35,175.39) --
	(149.54,175.34) --
	(149.86,175.27) --
	(150.07,175.31) --
	(150.15,175.33) --
	(150.24,175.42) --
	(150.31,175.46) --
	(150.60,175.54) --
	(150.71,175.56) --
	(150.83,175.51) --
	(150.95,175.54) --
	(151.01,175.51) --
	(151.10,175.46) --
	(151.20,175.47) --
	(151.33,175.51) --
	(151.52,175.64) --
	(151.88,175.74) --
	(151.98,175.84) --
	(152.09,175.87) --
	(152.17,175.91) --
	(152.28,176.06) --
	(152.45,176.16) --
	(152.68,176.27) --
	(152.74,176.32) --
	(152.81,176.40) --
	(152.92,176.45) --
	(153.11,176.46) --
	(153.27,176.56) --
	(153.43,176.57) --
	(153.66,176.60) --
	(153.80,176.51) --
	(153.88,176.55) --
	(153.93,176.65) --
	(153.98,176.79) --
	(154.06,176.86) --
	(154.17,176.88) --
	(154.42,176.86) --
	(154.50,176.88) --
	(154.59,176.95) --
	(154.61,177.05) --
	(154.59,177.24) --
	(154.62,177.40) --
	(154.81,177.74) --
	(154.90,177.84) --
	(154.98,177.90) --
	(155.04,177.99) --
	(155.05,178.09);

\draw[color=drawColor,line cap=round,line join=round,fill opacity=0.00,] (148.47,175.50) --
	(148.36,175.26) --
	(148.36,175.13) --
	(148.46,174.95) --
	(148.58,174.85) --
	(148.67,174.74) --
	(148.76,174.68) --
	(148.83,174.74) --
	(148.93,174.80) --
	(149.12,174.82) --
	(149.29,174.82) --
	(149.35,174.77) --
	(149.25,174.69) --
	(149.15,174.63) --
	(149.08,174.55) --
	(149.17,174.50) --
	(149.41,174.42) --
	(149.59,174.37) --
	(149.72,174.23) --
	(149.89,174.14) --
	(149.96,174.13) --
	(150.00,174.13) --
	(150.07,174.19) --
	(150.10,174.19) --
	(150.15,174.13) --
	(150.17,174.05) --
	(150.21,173.99) --
	(150.20,173.87) --
	(150.29,173.79) --
	(150.49,173.68) --
	(150.66,173.62) --
	(150.88,173.71) --
	(151.01,173.72) --
	(151.10,173.63) --
	(151.16,173.64) --
	(151.21,173.74) --
	(151.29,173.79) --
	(151.38,173.74) --
	(151.38,173.71) --
	(151.38,173.53) --
	(151.21,173.12) --
	(151.18,172.94) --
	(151.25,172.81) --
	(151.53,172.58) --
	(151.77,172.46) --
	(151.73,172.36) --
	(151.51,172.25) --
	(151.39,172.06) --
	(151.42,171.89) --
	(151.55,171.72) --
	(151.64,171.63) --
	(151.71,171.46) --
	(151.83,171.36) --
	(151.98,171.31) --
	(152.06,171.23) --
	(152.06,171.08) --
	(152.11,170.95) --
	(152.23,170.85) --
	(152.44,170.83) --
	(152.64,170.84) --
	(152.73,170.81) --
	(152.73,170.74) --
	(152.65,170.65) --
	(152.69,170.54) --
	(152.84,170.53) --
	(153.00,170.58) --
	(153.13,170.55) --
	(153.25,170.46) --
	(153.33,170.32) --
	(153.26,170.23) --
	(153.16,170.09) --
	(153.16,169.98) --
	(153.22,169.90) --
	(153.34,169.89) --
	(153.41,169.96) --
	(153.35,170.06) --
	(153.33,170.13) --
	(153.43,170.22) --
	(153.58,170.10) --
	(153.82,169.93) --
	(153.96,169.80) --
	(154.05,169.69) --
	(154.11,169.64) --
	(154.18,169.63) --
	(154.25,169.68) --
	(154.32,169.76) --
	(154.38,169.78) --
	(154.44,169.74) --
	(154.44,169.69) --
	(154.43,169.58) --
	(154.45,169.50) --
	(154.54,169.36) --
	(154.57,169.28) --
	(154.53,169.14) --
	(154.52,169.08) --
	(154.52,168.99) --
	(154.58,168.91) --
	(154.74,168.86) --
	(155.08,168.74) --
	(155.23,168.59) --
	(155.37,168.38) --
	(155.52,168.28) --
	(155.65,168.19) --
	(155.79,168.12) --
	(155.92,168.12) --
	(156.03,168.13) --
	(156.12,168.13) --
	(156.27,168.11) --
	(156.34,168.05) --
	(156.41,168.05) --
	(156.49,168.07) --
	(156.60,168.08) --
	(156.71,168.08) --
	(156.79,168.02) --
	(156.88,167.91) --
	(156.95,167.79) --
	(157.04,167.74) --
	(157.12,167.75) --
	(157.25,167.80) --
	(157.44,167.82) --
	(157.53,167.84) --
	(157.57,167.81) --
	(157.54,167.76) --
	(157.49,167.66) --
	(157.54,167.59) --
	(157.73,167.51) --
	(157.80,167.42) --
	(158.10,167.33) --
	(158.28,167.37) --
	(158.38,167.53) --
	(158.51,167.63) --
	(158.71,167.74) --
	(158.85,167.76) --
	(158.94,167.85) --
	(159.11,167.84) --
	(159.22,167.79) --
	(159.28,167.77) --
	(159.38,167.78) --
	(159.52,167.87) --
	(159.68,168.02) --
	(159.75,168.08) --
	(159.91,168.06) --
	(160.03,168.02) --
	(160.17,168.05) --
	(160.29,168.10) --
	(160.34,168.09) --
	(160.49,168.07) --
	(160.75,168.05) --
	(160.90,168.06) --
	(161.11,168.13) --
	(161.24,168.17) --
	(161.34,168.27) --
	(161.53,168.42) --
	(161.60,168.45) --
	(161.79,168.49) --
	(161.89,168.54) --
	(161.98,168.58) --
	(162.05,168.56) --
	(162.34,168.51) --
	(162.65,168.48) --
	(162.80,168.44) --
	(162.96,168.42) --
	(163.09,168.45) --
	(163.27,168.57) --
	(163.43,168.62) --
	(163.58,168.57) --
	(163.72,168.53) --
	(163.90,168.58) --
	(164.10,168.71) --
	(164.16,168.73) --
	(164.20,168.73) --
	(164.27,168.67) --
	(164.50,168.50) --
	(164.61,168.52) --
	(164.70,168.51) --
	(164.78,168.37) --
	(164.85,168.37) --
	(164.95,168.47) --
	(165.09,168.56) --
	(165.21,168.56) --
	(165.30,168.46) --
	(165.48,168.34) --
	(165.62,168.33) --
	(165.82,168.47) --
	(166.00,168.48) --
	(166.11,168.48) --
	(166.15,168.60) --
	(166.17,168.76) --
	(166.24,168.83) --
	(166.35,168.84) --
	(166.46,168.80) --
	(166.55,168.71) --
	(166.56,168.56) --
	(166.62,168.45) --
	(166.70,168.35) --
	(166.74,168.29) --
	(166.76,168.20) --
	(166.72,168.12) --
	(166.61,168.03) --
	(166.58,167.95) --
	(166.57,167.66) --
	(166.48,167.43) --
	(166.30,167.31) --
	(166.35,167.15) --
	(166.49,167.01) --
	(166.55,166.90) --
	(166.60,166.70) --
	(166.58,166.61) --
	(166.44,166.47) --
	(166.32,166.28) --
	(166.31,166.06) --
	(166.27,165.90) --
	(166.15,165.76) --
	(166.07,165.68) --
	(166.00,165.40) --
	(166.09,165.24) --
	(166.20,165.10) --
	(166.36,165.02) --
	(166.51,164.97) --
	(166.61,164.91) --
	(166.75,164.73) --
	(166.92,164.66) --
	(167.17,164.50) --
	(167.35,164.45) --
	(167.53,164.21) --
	(167.75,164.10) --
	(168.40,163.81) --
	(168.50,163.67) --
	(168.65,163.60) --
	(168.88,163.62) --
	(169.14,163.68) --
	(169.47,163.82) --
	(169.60,163.79) --
	(169.88,163.59) --
	(170.03,163.45) --
	(170.06,163.35) --
	(170.17,163.30) --
	(170.27,163.25) --
	(170.42,163.13) --
	(170.46,162.97) --
	(170.58,162.83) --
	(170.74,162.71) --
	(170.96,162.66) --
	(171.12,162.61) --
	(171.18,162.55) --
	(171.23,162.47) --
	(171.22,162.35) --
	(171.28,162.28) --
	(171.47,162.23) --
	(171.66,162.21) --
	(171.73,162.15) --
	(171.81,162.00) --
	(171.85,161.93) --
	(171.94,161.85) --
	(172.10,161.78) --
	(172.19,161.68) --
	(172.38,161.66) --
	(172.54,161.64) --
	(172.64,161.57) --
	(172.73,161.40) --
	(172.85,161.33) --
	(173.03,161.27) --
	(173.07,161.23) --
	(173.16,161.11) --
	(173.26,161.01) --
	(173.38,160.85) --
	(173.51,160.76) --
	(173.66,160.69);

\draw[color=drawColor,line cap=round,line join=round,fill opacity=0.00,] (193.31,174.42) --
	(193.45,174.39) --
	(193.53,174.23) --
	(193.70,174.19) --
	(193.92,174.17) --
	(194.11,174.11) --
	(194.19,174.07) --
	(194.33,174.06) --
	(194.47,174.02) --
	(194.63,173.83) --
	(194.86,173.58) --
	(195.10,173.40) --
	(195.36,173.23) --
	(195.52,173.18) --
	(195.67,173.18) --
	(195.77,173.23) --
	(195.86,173.23) --
	(195.89,173.19) --
	(195.99,173.12) --
	(196.08,173.12) --
	(196.14,173.09) --
	(196.21,173.03) --
	(196.34,173.00) --
	(196.37,172.95) --
	(196.40,172.86) --
	(196.62,172.71) --
	(196.71,172.64) --
	(196.84,172.60) --
	(196.93,172.54) --
	(196.98,172.43) --
	(197.02,172.31) --
	(197.09,172.22) --
	(197.20,172.13) --
	(197.24,172.00) --
	(197.27,171.92) --
	(197.49,171.67) --
	(197.68,171.55) --
	(197.84,171.43) --
	(197.92,171.29) --
	(197.99,171.12) --
	(197.99,170.95) --
	(197.91,170.74) --
	(197.84,170.57) --
	(197.69,170.47);

\draw[color=drawColor,line cap=round,line join=round,fill opacity=0.00,] (149.34,136.58) --
	(149.46,136.47) --
	(149.52,136.37) --
	(149.65,136.33) --
	(150.01,136.35) --
	(150.13,136.40) --
	(150.26,136.41) --
	(150.36,136.43) --
	(150.53,136.48) --
	(151.00,136.49) --
	(151.56,136.47) --
	(151.75,136.38) --
	(151.99,136.41) --
	(152.17,136.65) --
	(152.31,136.83) --
	(152.45,136.91) --
	(152.69,137.07) --
	(152.91,137.27) --
	(152.96,137.41) --
	(153.13,137.57) --
	(153.26,137.83) --
	(153.31,138.03) --
	(153.35,138.24) --
	(153.36,138.46) --
	(153.40,138.54) --
	(153.45,138.62) --
	(153.49,138.69) --
	(153.48,138.83) --
	(153.60,139.05) --
	(153.71,139.20) --
	(153.80,139.28) --
	(153.93,139.32) --
	(154.06,139.31) --
	(154.18,139.26) --
	(154.39,139.27) --
	(154.59,139.32) --
	(154.95,139.24) --
	(155.13,139.11) --
	(155.47,138.84) --
	(155.73,138.74) --
	(156.06,138.75) --
	(156.38,138.84) --
	(156.70,138.93) --
	(157.05,138.95) --
	(157.51,138.93) --
	(157.90,138.95) --
	(158.32,139.09) --
	(158.62,139.18) --
	(158.98,139.29) --
	(159.19,139.34) --
	(159.45,139.37) --
	(159.56,139.44) --
	(159.76,139.56) --
	(160.05,139.71) --
	(160.24,139.75) --
	(160.44,139.84) --
	(160.55,139.97) --
	(160.76,140.05) --
	(160.86,140.13) --
	(160.97,140.48) --
	(161.24,140.95) --
	(161.36,141.14) --
	(161.52,141.34) --
	(161.56,141.73) --
	(161.78,142.05) --
	(161.74,142.37) --
	(161.81,142.59) --
	(161.83,142.77) --
	(162.06,143.05) --
	(162.21,143.17) --
	(162.53,143.50) --
	(162.74,143.65) --
	(162.97,143.74) --
	(163.23,143.80) --
	(163.43,143.87) --
	(163.64,143.94) --
	(163.93,143.96) --
	(164.13,143.91) --
	(164.33,143.78) --
	(164.56,143.57) --
	(164.66,143.33) --
	(164.77,143.01) --
	(164.83,142.88) --
	(165.02,142.73) --
	(165.21,142.56);

\draw[color=drawColor,line cap=round,line join=round,fill opacity=0.00,] (141.37,157.74) --
	(141.47,157.68) --
	(141.45,157.59) --
	(141.36,157.53) --
	(141.21,157.46) --
	(141.12,157.41) --
	(141.07,157.34) --
	(141.17,157.29) --
	(141.27,157.27) --
	(141.29,157.19) --
	(141.19,157.19) --
	(141.04,157.21) --
	(140.94,157.17) --
	(140.87,157.18) --
	(140.85,157.27) --
	(140.77,157.26) --
	(140.68,157.15) --
	(140.61,157.10) --
	(140.54,157.05) --
	(140.50,157.01) --
	(140.49,156.89) --
	(140.38,156.76) --
	(140.39,156.64) --
	(140.45,156.58) --
	(140.46,156.50) --
	(140.28,156.49) --
	(139.99,156.45) --
	(139.83,156.37) --
	(139.76,156.36) --
	(139.67,156.35) --
	(139.60,156.35) --
	(139.58,156.30) --
	(139.58,156.25) --
	(139.68,156.16) --
	(139.71,156.11) --
	(139.70,156.04) --
	(139.69,155.98) --
	(139.73,155.94) --
	(139.75,155.86) --
	(139.68,155.81) --
	(139.59,155.81) --
	(139.48,155.87) --
	(139.39,155.83) --
	(139.39,155.76) --
	(139.48,155.71) --
	(139.53,155.64) --
	(139.53,155.60) --
	(139.49,155.57) --
	(139.36,155.62) --
	(139.30,155.63) --
	(139.27,155.58) --
	(139.30,155.48) --
	(139.34,155.43) --
	(139.49,155.43) --
	(139.57,155.42) --
	(139.63,155.36) --
	(139.64,155.28) --
	(139.56,155.24) --
	(139.45,155.34) --
	(139.35,155.36) --
	(139.33,155.32) --
	(139.30,155.23) --
	(139.30,155.15) --
	(139.38,155.08) --
	(139.28,155.01) --
	(139.22,154.94) --
	(139.30,154.80) --
	(139.39,154.78) --
	(139.49,154.74) --
	(139.64,154.75) --
	(139.87,154.76) --
	(139.96,154.68) --
	(139.93,154.62) --
	(139.82,154.60) --
	(139.79,154.49) --
	(139.69,154.49) --
	(139.62,154.46) --
	(139.64,154.31) --
	(139.61,154.22) --
	(139.69,154.19) --
	(139.77,154.29) --
	(139.86,154.34) --
	(139.91,154.37) --
	(140.08,154.46) --
	(140.13,154.39) --
	(140.00,154.30) --
	(139.95,154.18) --
	(139.97,154.07) --
	(140.07,154.11) --
	(140.11,154.23) --
	(140.21,154.25) --
	(140.30,154.18) --
	(140.40,154.18) --
	(140.48,154.19) --
	(140.51,154.13) --
	(140.40,154.06) --
	(140.35,153.97) --
	(140.29,153.96) --
	(140.21,154.03) --
	(140.14,154.04) --
	(140.10,153.99) --
	(140.13,153.92) --
	(140.22,153.87) --
	(140.21,153.81) --
	(140.08,153.80) --
	(140.02,153.77) --
	(140.04,153.68) --
	(140.10,153.65) --
	(140.19,153.65) --
	(140.30,153.59) --
	(140.38,153.59) --
	(140.39,153.65) --
	(140.39,153.71) --
	(140.41,153.77) --
	(140.49,153.75) --
	(140.52,153.68) --
	(140.59,153.65) --
	(140.63,153.70) --
	(140.66,153.77) --
	(140.73,153.77) --
	(140.82,153.81) --
	(140.90,153.85) --
	(140.96,153.82) --
	(140.93,153.72) --
	(140.99,153.66) --
	(141.12,153.68) --
	(141.20,153.64) --
	(141.22,153.54) --
	(141.07,153.50) --
	(140.89,153.48) --
	(140.77,153.50) --
	(140.69,153.43) --
	(140.62,153.36) --
	(140.63,153.31) --
	(140.81,153.27) --
	(140.77,153.21) --
	(140.61,153.14) --
	(140.57,153.07) --
	(140.74,152.89) --
	(140.80,152.81) --
	(140.97,152.74) --
	(141.06,152.65) --
	(141.07,152.53) --
	(141.01,152.42) --
	(140.96,152.32) --
	(141.01,152.25) --
	(141.11,152.21) --
	(141.23,152.19) --
	(141.23,152.09) --
	(141.03,151.98) --
	(140.78,151.87) --
	(140.66,151.76) --
	(140.35,151.66) --
	(140.28,151.56) --
	(140.24,151.47) --
	(140.13,151.44) --
	(139.95,151.21) --
	(139.99,151.04) --
	(140.03,150.97) --
	(140.15,150.91) --
	(140.10,150.83) --
	(140.12,150.62) --
	(140.30,150.50) --
	(140.50,150.52) --
	(140.60,150.43) --
	(140.65,150.18) --
	(140.76,150.11) --
	(140.86,150.11) --
	(140.94,150.19) --
	(141.06,150.20) --
	(141.15,150.15) --
	(141.16,150.05) --
	(141.34,149.96) --
	(141.41,149.98) --
	(141.56,150.13) --
	(141.65,150.12) --
	(141.63,150.04) --
	(141.56,149.91) --
	(141.65,149.80) --
	(141.87,149.78) --
	(141.80,149.61) --
	(141.73,149.54) --
	(141.76,149.45) --
	(141.94,149.42) --
	(141.97,149.37) --
	(141.97,149.31) --
	(141.84,149.23) --
	(141.78,149.09) --
	(141.72,149.00) --
	(141.61,148.96) --
	(141.56,148.87) --
	(141.63,148.77) --
	(141.72,148.74) --
	(141.76,148.77) --
	(141.80,148.88) --
	(141.90,148.91) --
	(142.05,148.89) --
	(142.09,148.83) --
	(142.08,148.68) --
	(142.05,148.46) --
	(141.92,148.42) --
	(141.79,148.37) --
	(141.72,148.25) --
	(141.74,148.17) --
	(141.85,148.00) --
	(142.04,147.97) --
	(142.17,147.93) --
	(142.36,147.77) --
	(142.55,147.62) --
	(142.73,147.57) --
	(142.85,147.68) --
	(142.95,147.71) --
	(143.00,147.65) --
	(142.95,147.57) --
	(142.74,147.39) --
	(142.52,147.13) --
	(142.42,146.95) --
	(142.29,146.52) --
	(142.29,146.35) --
	(142.52,146.00) --
	(142.61,145.81) --
	(142.63,145.56) --
	(142.63,145.34) --
	(142.53,145.19) --
	(142.42,145.02) --
	(142.38,144.71) --
	(142.28,144.57) --
	(142.20,144.46) --
	(142.14,144.25) --
	(142.20,144.19) --
	(142.36,144.31) --
	(142.62,144.64) --
	(142.82,144.82) --
	(142.96,144.92) --
	(143.15,144.91) --
	(143.24,144.82) --
	(143.27,144.58) --
	(143.40,144.26) --
	(143.57,144.09) --
	(143.70,143.99) --
	(143.75,143.87) --
	(143.96,143.79) --
	(144.11,143.67) --
	(144.23,143.51) --
	(144.32,143.47) --
	(144.45,143.42) --
	(144.50,143.30) --
	(144.53,143.20) --
	(144.59,143.06) --
	(144.50,142.97) --
	(144.46,142.79) --
	(144.51,142.60) --
	(144.45,142.42) --
	(144.51,142.33) --
	(144.64,142.21) --
	(144.73,142.08) --
	(144.69,142.04) --
	(144.56,141.99) --
	(144.48,141.85) --
	(144.49,141.71) --
	(144.40,141.65) --
	(144.35,141.53) --
	(144.40,141.46) --
	(144.51,141.38) --
	(144.55,141.26) --
	(144.49,141.19) --
	(144.41,141.12) --
	(144.37,141.06) --
	(144.39,140.94) --
	(144.33,140.84) --
	(144.38,140.72) --
	(144.47,140.57) --
	(144.56,140.50) --
	(144.69,140.47) --
	(144.90,140.51) --
	(145.00,140.49) --
	(145.14,140.39) --
	(145.32,140.36) --
	(145.47,140.31) --
	(145.67,140.23) --
	(145.83,140.16) --
	(146.11,140.14) --
	(146.30,140.07) --
	(146.66,140.01) --
	(146.85,139.91) --
	(146.98,139.82) --
	(147.07,139.71) --
	(147.23,139.63) --
	(147.51,139.49) --
	(147.76,139.30) --
	(147.88,139.12) --
	(148.04,139.03) --
	(148.18,138.98) --
	(148.34,139.01);

\draw[color=drawColor,line cap=round,line join=round,fill opacity=0.00,] (141.37,157.74) --
	(141.30,157.80) --
	(141.34,157.85) --
	(141.53,158.08) --
	(141.82,158.23) --
	(142.12,158.33) --
	(142.43,158.41) --
	(142.44,158.45) --
	(142.41,158.48) --
	(142.33,158.52) --
	(142.34,158.59) --
	(142.38,158.66) --
	(142.49,158.71) --
	(142.58,158.70) --
	(142.66,158.58) --
	(142.74,158.59) --
	(142.80,158.67) --
	(142.78,158.84) --
	(142.63,158.81) --
	(142.55,158.83) --
	(142.53,158.91) --
	(142.55,159.01) --
	(142.58,159.10) --
	(142.53,159.23) --
	(142.46,159.26) --
	(142.34,159.25) --
	(142.30,159.28) --
	(142.33,159.46) --
	(142.37,159.62) --
	(142.31,159.70) --
	(142.20,159.83) --
	(142.21,159.96) --
	(142.18,160.01) --
	(142.03,160.05) --
	(141.87,160.20) --
	(141.76,160.28) --
	(141.69,160.35) --
	(141.69,160.42) --
	(141.76,160.49) --
	(141.75,160.56) --
	(141.65,160.54) --
	(141.51,160.47) --
	(141.41,160.47) --
	(141.34,160.51) --
	(141.33,160.63) --
	(141.34,160.67) --
	(141.28,160.76) --
	(141.12,160.77) --
	(140.98,160.73) --
	(140.95,160.80) --
	(141.02,160.84) --
	(141.15,160.89) --
	(141.20,160.99) --
	(141.15,161.06) --
	(141.02,161.10) --
	(140.88,161.13) --
	(140.82,161.20) --
	(140.83,161.27) --
	(140.97,161.38) --
	(140.95,161.43) --
	(140.81,161.44) --
	(140.77,161.52) --
	(140.76,161.61) --
	(140.65,161.67) --
	(140.53,161.81) --
	(140.50,161.90) --
	(140.61,162.01) --
	(140.55,162.08) --
	(140.52,162.17) --
	(140.63,162.31) --
	(140.64,162.38) --
	(140.50,162.55) --
	(140.45,162.63) --
	(140.46,162.75) --
	(140.37,162.88) --
	(140.25,163.07) --
	(140.05,163.18) --
	(139.95,163.29) --
	(139.92,163.37) --
	(139.86,163.61) --
	(139.68,163.90) --
	(139.66,164.04) --
	(139.66,164.27) --
	(139.57,164.40) --
	(139.43,164.54) --
	(139.40,164.76) --
	(139.36,164.90) --
	(139.23,164.99) --
	(139.22,165.17) --
	(139.13,165.27) --
	(139.07,165.38) --
	(139.07,165.55) --
	(139.05,165.68) --
	(138.90,165.84) --
	(138.77,165.90) --
	(138.62,165.99) --
	(138.51,166.16) --
	(138.42,166.20) --
	(138.36,166.23) --
	(138.23,166.27) --
	(138.14,166.35) --
	(137.95,166.44) --
	(137.81,166.49) --
	(137.78,166.56) --
	(137.76,166.75) --
	(137.70,166.86) --
	(137.68,166.96) --
	(137.64,167.08) --
	(137.55,167.15) --
	(137.43,167.22) --
	(137.37,167.34) --
	(137.41,167.44) --
	(137.35,167.51) --
	(137.27,167.55) --
	(137.16,167.62) --
	(137.01,167.65) --
	(136.95,167.77) --
	(136.91,167.81) --
	(136.52,167.90) --
	(136.49,167.96) --
	(136.46,168.01) --
	(136.43,168.01) --
	(136.37,168.02) --
	(136.27,168.03) --
	(136.15,168.05) --
	(136.01,168.07) --
	(135.89,168.03) --
	(135.80,168.05) --
	(135.71,167.97) --
	(135.68,167.94) --
	(135.56,167.91) --
	(135.47,167.80) --
	(135.43,167.74) --
	(135.26,167.62) --
	(135.13,167.54) --
	(135.02,167.52) --
	(134.95,167.54) --
	(134.91,167.68) --
	(134.69,167.81) --
	(134.47,167.89) --
	(134.30,168.00) --
	(134.28,168.14) --
	(134.28,168.21) --
	(134.42,168.36) --
	(134.45,168.43) --
	(134.42,168.48) --
	(134.38,168.61) --
	(134.46,168.68) --
	(134.50,168.74) --
	(134.49,168.81) --
	(134.32,168.79) --
	(134.17,168.79) --
	(134.13,168.87) --
	(134.12,168.95) --
	(134.03,169.00) --
	(133.93,169.00) --
	(133.77,169.02) --
	(133.56,169.17) --
	(133.44,169.17) --
	(133.38,168.99) --
	(133.32,168.99) --
	(133.18,169.07) --
	(133.16,169.08) --
	(133.13,169.08) --
	(133.11,169.07) --
	(132.97,169.04) --
	(132.83,169.16) --
	(132.86,169.29) --
	(132.90,169.38) --
	(132.81,169.44) --
	(132.87,169.55) --
	(132.69,169.66) --
	(132.49,169.78) --
	(132.29,169.77) --
	(132.14,169.79) --
	(131.98,169.67) --
	(131.87,169.55) --
	(131.84,169.54) --
	(131.71,169.52) --
	(131.58,169.55) --
	(131.47,169.46) --
	(131.45,169.36) --
	(131.32,169.35) --
	(131.22,169.44) --
	(131.22,169.64) --
	(131.01,169.64) --
	(130.98,169.61) --
	(130.92,169.56);

\draw[color=drawColor,line cap=round,line join=round,fill opacity=0.00,] (136.81,160.23) --
	(136.75,160.19) --
	(136.74,160.11) --
	(136.76,160.04) --
	(136.89,159.95) --
	(136.95,159.91) --
	(137.03,159.72) --
	(137.13,159.55) --
	(137.22,159.43) --
	(137.42,159.21) --
	(137.67,159.04) --
	(137.82,158.92) --
	(138.10,158.85) --
	(138.20,158.83) --
	(138.23,158.80) --
	(138.32,158.69) --
	(138.40,158.63) --
	(138.57,158.56) --
	(138.66,158.47) --
	(138.74,158.41) --
	(138.93,158.27) --
	(139.37,158.09) --
	(139.59,158.05) --
	(139.68,157.97) --
	(139.80,157.86) --
	(139.98,157.75) --
	(140.36,157.61) --
	(140.48,157.60) --
	(140.59,157.63) --
	(140.69,157.64) --
	(140.97,157.62) --
	(141.12,157.61) --
	(141.24,157.64) --
	(141.37,157.74);

\draw[color=drawColor,line cap=round,line join=round,fill opacity=0.00,] (196.69,169.57) --
	(196.68,169.55) --
	(196.46,169.33) --
	(196.36,169.21) --
	(196.32,169.06) --
	(196.26,168.91) --
	(196.29,168.66) --
	(196.21,168.59) --
	(196.18,168.50) --
	(196.14,168.44) --
	(196.02,168.41) --
	(195.73,168.36) --
	(195.66,168.34) --
	(195.53,168.24) --
	(195.25,168.22) --
	(195.12,168.23) --
	(194.96,168.17) --
	(194.92,168.10) --
	(194.93,168.04) --
	(194.89,168.00) --
	(194.67,167.98) --
	(194.51,167.90) --
	(194.25,167.87) --
	(194.10,167.80) --
	(193.94,167.75) --
	(193.81,167.71) --
	(193.70,167.62) --
	(193.61,167.54) --
	(193.31,167.49) --
	(193.15,167.43) --
	(192.96,167.36) --
	(192.78,167.24);

\draw[color=drawColor,line cap=round,line join=round,fill opacity=0.00,] (116.08,154.13) --
	(116.18,154.25) --
	(116.29,154.29) --
	(116.36,154.31) --
	(116.49,154.32) --
	(116.62,154.35) --
	(116.72,154.40) --
	(116.84,154.63) --
	(116.93,154.75) --
	(116.98,154.76) --
	(117.14,154.74) --
	(117.30,154.74) --
	(117.50,154.98) --
	(117.59,155.04) --
	(117.65,155.06) --
	(118.12,155.15) --
	(118.15,155.18) --
	(118.18,155.24) --
	(118.16,155.29) --
	(118.12,155.33) --
	(118.08,155.37) --
	(118.04,155.37) --
	(117.99,155.36) --
	(117.93,155.32) --
	(117.77,155.20) --
	(117.70,155.18) --
	(117.45,155.16) --
	(117.35,155.18) --
	(117.27,155.21) --
	(117.24,155.25) --
	(117.20,155.31) --
	(117.18,155.37) --
	(117.20,155.43) --
	(117.26,155.46) --
	(117.30,155.49) --
	(117.35,155.49) --
	(117.78,155.46) --
	(117.84,155.48) --
	(117.96,155.56) --
	(117.99,155.62) --
	(117.97,155.68) --
	(117.93,155.80) --
	(117.95,155.85) --
	(117.99,155.87) --
	(118.11,155.92) --
	(118.30,156.00) --
	(118.36,156.05) --
	(118.38,156.11) --
	(118.38,156.16) --
	(118.35,156.21) --
	(118.05,156.44) --
	(118.04,156.50) --
	(118.04,156.54) --
	(118.06,156.58) --
	(118.30,156.60) --
	(118.39,156.62) --
	(118.45,156.68) --
	(118.50,156.73) --
	(118.57,156.85) --
	(118.59,156.89) --
	(118.65,156.91) --
	(119.01,156.87) --
	(119.07,156.90) --
	(119.13,156.95) --
	(119.13,156.97) --
	(119.13,157.01) --
	(118.90,157.30) --
	(118.91,157.34) --
	(118.88,157.42) --
	(118.91,157.50) --
	(118.94,157.59) --
	(118.97,157.70) --
	(118.98,157.78) --
	(118.96,157.83) --
	(118.95,157.88) --
	(118.91,157.93) --
	(118.78,158.01) --
	(118.66,158.15) --
	(118.61,158.24) --
	(118.59,158.29) --
	(118.57,158.35) --
	(118.61,158.42) --
	(118.66,158.46) --
	(118.75,158.51) --
	(118.88,158.54) --
	(118.96,158.51) --
	(119.01,158.48) --
	(119.03,158.46) --
	(119.14,158.24) --
	(119.18,158.20) --
	(119.24,158.18) --
	(119.34,158.15) --
	(119.51,158.15) --
	(119.66,158.17) --
	(119.71,158.20) --
	(119.79,158.25) --
	(119.80,158.32) --
	(119.82,158.37) --
	(119.81,158.44) --
	(119.77,158.52) --
	(119.73,158.63) --
	(119.66,158.75) --
	(119.52,158.93) --
	(119.49,158.98) --
	(119.49,159.06) --
	(119.51,159.10) --
	(119.53,159.14) --
	(119.54,159.24) --
	(119.53,159.35) --
	(119.49,159.48) --
	(119.48,159.51) --
	(119.49,159.54) --
	(119.63,159.69) --
	(119.73,159.79) --
	(119.79,159.84) --
	(119.80,159.88) --
	(119.80,159.91) --
	(119.77,159.93) --
	(119.74,159.98) --
	(119.68,160.03) --
	(119.63,160.08) --
	(119.19,160.33) --
	(119.13,160.38) --
	(119.09,160.43) --
	(119.06,160.49) --
	(119.06,160.54) --
	(119.06,160.58) --
	(119.06,160.63) --
	(119.09,160.68) --
	(119.11,160.71) --
	(119.15,160.74) --
	(119.23,160.75) --
	(119.27,160.73) --
	(119.31,160.71) --
	(119.35,160.65) --
	(119.41,160.41) --
	(119.45,160.36) --
	(119.48,160.34) --
	(119.56,160.34) --
	(119.62,160.35) --
	(119.92,160.59) --
	(119.93,160.60) --
	(120.14,160.59) --
	(120.22,160.63) --
	(120.30,160.62) --
	(120.34,160.61) --
	(120.38,160.58) --
	(120.41,160.52) --
	(120.46,160.28) --
	(120.49,160.25) --
	(120.52,160.24) --
	(120.60,160.24) --
	(120.66,160.25) --
	(120.71,160.27) --
	(120.79,160.31) --
	(120.84,160.32) --
	(120.90,160.31) --
	(121.04,160.23) --
	(121.11,160.21) --
	(121.24,160.23) --
	(121.30,160.24) --
	(121.34,160.26) --
	(121.37,160.31) --
	(121.39,160.39) --
	(121.41,160.50) --
	(121.40,160.64) --
	(121.39,160.74) --
	(121.40,160.79) --
	(121.43,160.84) --
	(121.46,160.87) --
	(121.66,160.98) --
	(121.69,161.03) --
	(121.82,161.22) --
	(121.85,161.28) --
	(121.88,161.69) --
	(121.88,161.72) --
	(121.71,161.84) --
	(121.71,161.87) --
	(121.71,161.91) --
	(121.76,161.95) --
	(121.92,162.06) --
	(121.92,162.11) --
	(121.88,162.44) --
	(121.86,162.50) --
	(121.75,162.69) --
	(121.76,162.72) --
	(121.77,162.74) --
	(121.82,162.79) --
	(122.02,162.92) --
	(122.05,162.97) --
	(122.07,163.02) --
	(122.03,163.05) --
	(121.81,163.20) --
	(121.77,163.24) --
	(121.76,163.31) --
	(121.75,163.38) --
	(121.76,163.45) --
	(121.80,163.50) --
	(121.83,163.57) --
	(121.89,163.60) --
	(121.92,163.64) --
	(122.05,163.68) --
	(122.13,163.69) --
	(122.21,163.67) --
	(122.28,163.62) --
	(122.32,163.54) --
	(122.34,163.44) --
	(122.37,163.22) --
	(122.31,162.78) --
	(122.33,162.68) --
	(122.39,162.61) --
	(122.45,162.57) --
	(122.58,162.50) --
	(122.66,162.47) --
	(122.75,162.46) --
	(122.85,162.47) --
	(122.91,162.49) --
	(123.01,162.56) --
	(123.14,162.80) --
	(123.28,163.10) --
	(123.41,163.23) --
	(123.55,163.38) --
	(123.59,163.46) --
	(123.70,163.55) --
	(123.66,163.67) --
	(123.52,163.83) --
	(123.45,163.96) --
	(123.43,164.05) --
	(123.41,164.10) --
	(123.43,164.15) --
	(123.47,164.19) --
	(123.62,164.26) --
	(123.68,164.29) --
	(123.73,164.35) --
	(123.97,164.33) --
	(124.09,164.35) --
	(124.09,164.39) --
	(124.11,164.51) --
	(124.12,164.54) --
	(124.17,164.56) --
	(124.24,164.57) --
	(124.33,164.57) --
	(124.54,164.46) --
	(124.57,164.45) --
	(124.60,164.44) --
	(124.63,164.48) --
	(124.91,164.67) --
	(125.12,164.79) --
	(125.33,164.93) --
	(125.61,165.03) --
	(125.86,165.10) --
	(125.99,165.17) --
	(126.08,165.24) --
	(126.08,165.29) --
	(126.10,165.34) --
	(126.10,165.38) --
	(126.10,165.41) --
	(126.09,165.43) --
	(126.06,165.46) --
	(126.00,165.48) --
	(125.96,165.47) --
	(125.93,165.45) --
	(125.68,165.35) --
	(125.63,165.33) --
	(125.56,165.32) --
	(125.50,165.32) --
	(125.46,165.34) --
	(125.45,165.37) --
	(125.43,165.42) --
	(125.43,165.46) --
	(125.45,165.50) --
	(125.48,165.60) --
	(125.58,165.77) --
	(125.57,165.95) --
	(125.57,165.98) --
	(125.61,166.01) --
	(125.73,166.07) --
	(125.85,166.15) --
	(126.04,166.27) --
	(126.16,166.31) --
	(126.28,166.35) --
	(126.53,166.37) --
	(126.69,166.35) --
	(126.79,166.31) --
	(126.89,166.29) --
	(127.00,166.29) --
	(127.21,166.31) --
	(127.28,166.34) --
	(127.36,166.37) --
	(127.49,166.43) --
	(127.53,166.48) --
	(127.67,166.62) --
	(127.73,166.63) --
	(127.83,166.65) --
	(127.88,166.65) --
	(127.95,166.74) --
	(127.97,166.76) --
	(128.01,166.78) --
	(128.20,166.82) --
	(128.20,166.83) --
	(128.23,166.85) --
	(128.26,166.89) --
	(128.26,166.91) --
	(128.27,167.01) --
	(128.25,167.03) --
	(128.27,167.04) --
	(128.56,167.18) --
	(128.59,167.19) --
	(128.64,167.24) --
	(128.67,167.27) --
	(128.77,167.29) --
	(128.80,167.37) --
	(128.85,167.42) --
	(128.95,167.51) --
	(129.15,167.65) --
	(129.34,167.75) --
	(129.39,167.78) --
	(129.45,167.86) --
	(129.49,167.90) --
	(129.72,168.05) --
	(129.76,168.12) --
	(129.81,168.16) --
	(129.93,168.24) --
	(129.94,168.27) --
	(129.89,168.41) --
	(129.86,168.44) --
	(129.73,168.57) --
	(129.67,168.72) --
	(129.64,168.81) --
	(129.63,168.85) --
	(129.56,168.93) --
	(129.46,169.12) --
	(129.45,169.13) --
	(129.45,169.16) --
	(129.47,169.21) --
	(129.50,169.22) --
	(129.53,169.23) --
	(129.60,169.24) --
	(129.63,169.25) --
	(129.64,169.24) --
	(129.74,169.15) --
	(129.78,169.11) --
	(129.83,169.10) --
	(129.88,169.09) --
	(129.91,169.10) --
	(130.06,169.14) --
	(130.09,169.17) --
	(130.11,169.18) --
	(130.12,169.19) --
	(130.14,169.22) --
	(130.11,169.33) --
	(130.13,169.35) --
	(130.14,169.40) --
	(130.17,169.41) --
	(130.20,169.42) --
	(130.25,169.44) --
	(130.29,169.44) --
	(130.63,169.43) --
	(130.70,169.45) --
	(130.88,169.49) --
	(130.92,169.52) --
	(130.92,169.56);

\draw[color=drawColor,line cap=round,line join=round,fill opacity=0.00,] (192.78,167.24) --
	(192.64,167.11) --
	(192.58,167.07) --
	(192.22,166.99) --
	(192.02,166.91) --
	(191.90,166.83) --
	(191.79,166.62) --
	(191.61,166.53) --
	(191.53,166.44) --
	(191.49,166.31) --
	(191.47,166.19) --
	(191.40,166.15) --
	(191.28,166.11) --
	(191.24,166.04) --
	(191.21,165.90) --
	(191.13,165.86) --
	(191.01,165.79);

\draw[color=drawColor,line cap=round,line join=round,fill opacity=0.00,] (116.08,154.13) --
	(115.66,154.07) --
	(115.61,154.08) --
	(115.60,154.11) --
	(115.57,154.13) --
	(115.59,154.19) --
	(115.60,154.22) --
	(115.69,154.30) --
	(116.00,154.47) --
	(116.00,154.50) --
	(115.99,154.53) --
	(115.96,154.54) --
	(115.90,154.55) --
	(115.73,154.56) --
	(115.40,154.66) --
	(115.27,154.74) --
	(115.22,154.80) --
	(115.19,154.87) --
	(115.19,154.92) --
	(115.21,154.98) --
	(115.28,155.03) --
	(115.38,155.05) --
	(115.54,155.06) --
	(115.95,155.07) --
	(115.96,155.09) --
	(115.98,155.11) --
	(115.72,155.41) --
	(115.70,155.42) --
	(115.45,155.32) --
	(115.40,155.31) --
	(115.35,155.33) --
	(115.30,155.37) --
	(115.28,155.41) --
	(115.29,155.45) --
	(115.34,155.50) --
	(115.38,155.55) --
	(115.59,155.63) --
	(115.63,155.70) --
	(115.63,155.80) --
	(115.64,155.90) --
	(115.61,155.98) --
	(115.55,156.04) --
	(115.50,156.06) --
	(115.43,156.07) --
	(115.37,156.06) --
	(115.04,155.82) --
	(115.01,155.81) --
	(114.94,155.82) --
	(114.86,155.85) --
	(114.83,155.87) --
	(114.80,155.93) --
	(114.80,155.95) --
	(114.80,155.99) --
	(114.92,156.19) --
	(114.98,156.28) --
	(114.98,156.34) --
	(114.94,156.38) --
	(114.71,156.51) --
	(114.50,156.59) --
	(114.46,156.62) --
	(114.44,156.66) --
	(114.45,156.71) --
	(114.46,156.76) --
	(114.49,156.89) --
	(114.54,156.95) --
	(114.56,157.00) --
	(114.54,157.06) --
	(114.53,157.09) --
	(114.51,157.12) --
	(114.16,157.31) --
	(114.12,157.36) --
	(114.07,157.40) --
	(114.06,157.45) --
	(114.06,157.49) --
	(114.08,157.54) --
	(114.09,157.59) --
	(114.14,157.61) --
	(114.20,157.66) --
	(114.46,157.74) --
	(114.50,157.77) --
	(114.53,157.82) --
	(114.56,157.87) --
	(114.70,158.07) --
	(114.69,158.11) --
	(114.67,158.14) --
	(114.65,158.16) --
	(114.63,158.18) --
	(114.62,158.19) --
	(114.56,158.20) --
	(114.49,158.21) --
	(114.46,158.19) --
	(114.43,158.18) --
	(114.33,157.92) --
	(114.29,157.87) --
	(114.25,157.84) --
	(114.20,157.82) --
	(114.16,157.81) --
	(114.11,157.81) --
	(114.06,157.83) --
	(113.99,157.86) --
	(113.86,157.96) --
	(113.58,158.23) --
	(113.59,158.26) --
	(113.61,158.28) --
	(113.71,158.48) --
	(113.91,158.84) --
	(113.91,158.90) --
	(113.88,158.94) --
	(113.84,158.98) --
	(113.80,158.99) --
	(113.74,159.02) --
	(113.70,159.01) --
	(113.64,159.00) --
	(113.61,158.97) --
	(113.34,158.68) --
	(113.13,158.46) --
	(113.10,158.44) --
	(112.99,158.38) --
	(112.96,158.38) --
	(112.93,158.39) --
	(112.89,158.41) --
	(112.87,158.44) --
	(112.96,158.55) --
	(112.98,158.61) --
	(112.99,158.64) --
	(112.98,158.67) --
	(113.00,158.72) --
	(112.97,158.76) --
	(112.95,158.77) --
	(112.90,158.81) --
	(112.85,158.82) --
	(112.68,158.83) --
	(112.58,158.86) --
	(112.54,158.89) --
	(112.51,158.90) --
	(112.43,159.05) --
	(112.43,159.07) --
	(112.71,159.22) --
	(112.73,159.25) --
	(112.74,159.28) --
	(112.75,159.30) --
	(112.75,159.36) --
	(112.77,159.38) --
	(112.83,159.42) --
	(113.00,159.50) --
	(113.11,159.58) --
	(113.12,159.60) --
	(113.11,159.64) --
	(113.01,159.90) --
	(112.99,159.95) --
	(112.80,160.14) --
	(112.64,160.39) --
	(112.59,160.42) --
	(112.56,160.43) --
	(112.26,160.41) --
	(112.23,160.43) --
	(112.17,160.45) --
	(112.16,160.49) --
	(112.13,160.52) --
	(112.12,160.55) --
	(112.12,160.63) --
	(112.15,160.68) --
	(112.18,160.69) --
	(112.18,160.70) --
	(112.21,160.72) --
	(112.27,160.72) --
	(112.30,160.74) --
	(112.34,160.73) --
	(112.40,160.73) --
	(112.43,160.70) --
	(112.48,160.68) --
	(112.63,160.56) --
	(112.65,160.52) --
	(112.67,160.47) --
	(112.72,160.43) --
	(112.77,160.43) --
	(112.82,160.42) --
	(112.87,160.43) --
	(112.91,160.44) --
	(112.94,160.48) --
	(112.97,160.52) --
	(112.97,160.56) --
	(112.97,160.62) --
	(112.96,160.71) --
	(112.95,160.74) --
	(112.98,160.80) --
	(113.02,160.85) --
	(113.07,160.94) --
	(113.15,161.05) --
	(113.15,161.08) --
	(113.11,161.38) --
	(113.10,161.39) --
	(113.05,161.42) --
	(113.04,161.43) --
	(112.93,161.48) --
	(112.88,161.50) --
	(112.83,161.56) --
	(112.66,161.72) --
	(112.61,161.75) --
	(112.57,161.78) --
	(112.23,161.88) --
	(112.15,161.91) --
	(112.11,161.95) --
	(112.10,161.98) --
	(112.10,162.01) --
	(112.10,162.05) --
	(112.11,162.07) --
	(112.13,162.11) --
	(112.19,162.18) --
	(112.23,162.21) --
	(112.26,162.23) --
	(112.28,162.23) --
	(112.38,162.19) --
	(112.69,162.06) --
	(112.75,162.06) --
	(112.81,162.09) --
	(112.86,162.11) --
	(112.87,162.14) --
	(112.89,162.17) --
	(112.89,162.25) --
	(112.90,162.28) --
	(112.95,162.32) --
	(112.98,162.33) --
	(113.34,162.32) --
	(113.37,162.33) --
	(113.40,162.34) --
	(113.41,162.38) --
	(113.42,162.42) --
	(113.40,162.46) --
	(113.40,162.47) --
	(113.16,162.57) --
	(113.12,162.62) --
	(113.07,162.68) --
	(113.06,162.75) --
	(113.06,162.83) --
	(113.09,162.90) --
	(113.12,162.94) --
	(113.51,163.21) --
	(113.53,163.25) --
	(113.53,163.29) --
	(113.51,163.35) --
	(113.50,163.37) --
	(113.47,163.40) --
	(113.40,163.42) --
	(113.36,163.43) --
	(113.31,163.44) --
	(113.25,163.42) --
	(113.21,163.40) --
	(113.00,163.27) --
	(112.97,163.26) --
	(112.92,163.25) --
	(112.86,163.26) --
	(112.81,163.28) --
	(112.80,163.30) --
	(112.77,163.34) --
	(112.76,163.35) --
	(112.73,163.54) --
	(112.71,163.59) --
	(112.68,163.61) --
	(112.64,163.62) --
	(112.59,163.64) --
	(112.56,163.63) --
	(112.53,163.63) --
	(112.49,163.61) --
	(112.46,163.58) --
	(112.43,163.57) --
	(112.28,163.54) --
	(112.13,163.66) --
	(112.10,163.67) --
	(111.88,163.69) --
	(111.84,163.70) --
	(111.80,163.74) --
	(111.80,163.78) --
	(111.80,163.81) --
	(111.84,163.90) --
	(111.87,163.94) --
	(112.29,164.30) --
	(112.36,164.35) --
	(112.39,164.35) --
	(112.45,164.36) --
	(112.50,164.36) --
	(112.52,164.36) --
	(112.58,164.33) --
	(112.83,163.95) --
	(112.86,163.92) --
	(112.89,163.90) --
	(112.93,163.91) --
	(113.02,163.90) --
	(113.09,163.93) --
	(113.34,164.06) --
	(113.34,164.08) --
	(113.32,164.34) --
	(113.26,164.57) --
	(113.23,164.62) --
	(113.09,164.75) --
	(113.06,164.78) --
	(112.99,164.80) --
	(112.91,164.81) --
	(112.86,164.80) --
	(112.81,164.80) --
	(112.78,164.77) --
	(112.75,164.74) --
	(112.74,164.68) --
	(112.67,164.50) --
	(112.64,164.48) --
	(112.27,164.46) --
	(112.22,164.44) --
	(112.06,164.39) --
	(111.97,164.37) --
	(111.81,164.38) --
	(111.77,164.38) --
	(111.70,164.39) --
	(111.64,164.43) --
	(111.60,164.45) --
	(111.55,164.50) --
	(111.53,164.55) --
	(111.51,164.61) --
	(111.54,164.65) --
	(111.57,164.68) --
	(111.65,164.70) --
	(111.69,164.69) --
	(111.83,164.66) --
	(112.04,164.64) --
	(112.11,164.63) --
	(112.14,164.66) --
	(112.17,164.67) --
	(112.16,164.69) --
	(112.17,164.75) --
	(112.03,165.06) --
	(112.02,165.09) --
	(112.04,165.16) --
	(112.10,165.26) --
	(112.11,165.29) --
	(112.19,165.34) --
	(112.25,165.39) --
	(112.28,165.43) --
	(112.31,165.47) --
	(112.31,165.50) --
	(112.30,165.56) --
	(112.30,165.59) --
	(112.25,165.65) --
	(112.23,165.66) --
	(112.20,165.67) --
	(112.15,165.68) --
	(112.11,165.67) --
	(112.08,165.66) --
	(112.06,165.64) --
	(112.02,165.44) --
	(112.00,165.41) --
	(111.96,165.36) --
	(111.93,165.35) --
	(111.91,165.33) --
	(111.85,165.34) --
	(111.82,165.37) --
	(111.77,165.47) --
	(111.77,165.49) --
	(111.82,165.67) --
	(111.82,165.70) --
	(111.82,165.71) --
	(111.81,165.72) --
	(111.78,165.74) --
	(111.76,165.74) --
	(111.67,165.69) --
	(111.51,165.59) --
	(111.48,165.59) --
	(111.45,165.61) --
	(111.42,165.63) --
	(111.40,165.66) --
	(111.40,165.67) --
	(111.41,165.71) --
	(111.55,165.91) --
	(111.55,165.94) --
	(111.54,165.97) --
	(111.49,165.99) --
	(111.41,166.02) --
	(111.35,166.02) --
	(111.30,166.01) --
	(111.26,166.00) --
	(111.25,165.99) --
	(111.10,165.84) --
	(111.07,165.81) --
	(111.02,165.81) --
	(110.98,165.80) --
	(110.94,165.83) --
	(110.89,165.84) --
	(110.86,165.88) --
	(110.84,165.92) --
	(110.84,165.94) --
	(110.85,165.96) --
	(111.12,166.14) --
	(111.12,166.17) --
	(111.11,166.19) --
	(111.05,166.24) --
	(110.83,166.40) --
	(110.76,166.51) --
	(110.77,166.56) --
	(110.78,166.59) --
	(111.08,166.84) --
	(111.09,166.88) --
	(111.08,166.91) --
	(111.05,166.94) --
	(111.00,167.00) --
	(110.95,167.01) --
	(110.88,167.01) --
	(110.79,166.99) --
	(110.67,166.94) --
	(110.51,166.85) --
	(110.44,166.83) --
	(110.34,166.82) --
	(110.28,166.83) --
	(110.22,166.87) --
	(110.19,166.89) --
	(110.18,166.92) --
	(110.18,166.95) --
	(110.21,166.96) --
	(110.25,166.98) --
	(110.54,167.03) --
	(110.57,167.05) --
	(110.59,167.06) --
	(110.58,167.10) --
	(110.44,167.39) --
	(110.41,167.42) --
	(110.41,167.46) --
	(110.41,167.48) --
	(110.44,167.52) --
	(110.62,167.67) --
	(110.63,167.69) --
	(110.67,167.80) --
	(110.68,167.82) --
	(110.74,167.86) --
	(110.79,167.89) --
	(110.90,167.91) --
	(111.14,167.98) --
	(111.20,168.01) --
	(111.20,168.05) --
	(111.21,168.08) --
	(111.20,168.11) --
	(111.19,168.14) --
	(111.14,168.17) --
	(111.07,168.26) --
	(110.95,168.44) --
	(110.90,168.48) --
	(110.82,168.51) --
	(110.70,168.55) --
	(110.61,168.57);

\draw[color=drawColor,line cap=round,line join=round,fill opacity=0.00,] (191.01,165.79) --
	(190.96,165.70) --
	(190.85,165.68) --
	(190.81,165.61) --
	(190.68,165.50) --
	(190.60,165.38) --
	(190.57,165.24) --
	(190.61,165.18) --
	(190.66,165.12) --
	(190.64,165.06) --
	(190.59,164.97) --
	(190.49,164.85) --
	(190.41,164.76) --
	(190.31,164.69) --
	(190.17,164.55) --
	(190.10,164.49) --
	(189.84,164.37) --
	(189.74,164.28) --
	(189.59,164.16) --
	(189.29,163.91) --
	(189.12,163.77) --
	(188.99,163.72) --
	(188.86,163.60) --
	(188.72,163.50) --
	(188.63,163.42);

\draw[color=drawColor,line cap=round,line join=round,fill opacity=0.00,] (188.63,163.42) --
	(188.62,163.08) --
	(188.52,163.03) --
	(188.46,162.96) --
	(188.49,162.82) --
	(188.48,162.68) --
	(188.45,162.57) --
	(188.34,162.46) --
	(188.27,162.30) --
	(188.04,162.20);

\draw[color=drawColor,line cap=round,line join=round,fill opacity=0.00,] (186.70,162.08) --
	(186.85,162.17) --
	(187.05,162.19) --
	(187.15,162.22) --
	(187.57,162.24) --
	(188.04,162.20);

\draw[color=drawColor,line cap=round,line join=round,fill opacity=0.00,] (173.66,160.69) --
	(173.75,160.65);

\draw[color=drawColor,line cap=round,line join=round,fill opacity=0.00,] (177.35,160.59) --
	(177.52,160.59) --
	(177.66,160.56) --
	(177.82,160.55) --
	(178.08,160.59) --
	(178.35,160.59) --
	(178.61,160.57) --
	(178.70,160.57) --
	(178.71,160.57) --
	(178.71,160.57) --
	(178.95,160.58) --
	(179.16,160.64);

\draw[color=drawColor,line cap=round,line join=round,fill opacity=0.00,] (104.18,146.70) --
	(104.44,146.73) --
	(104.76,146.72) --
	(104.96,146.72) --
	(105.09,146.70) --
	(105.18,146.70) --
	(105.45,146.68) --
	(105.72,146.71) --
	(105.81,146.68) --
	(105.89,146.61) --
	(106.06,146.14) --
	(106.18,146.12) --
	(106.31,146.15) --
	(106.40,146.20) --
	(106.35,146.48) --
	(106.37,146.58) --
	(106.43,146.75) --
	(106.49,146.83) --
	(106.60,146.88) --
	(106.75,146.91) --
	(106.98,146.90) --
	(107.14,146.86) --
	(107.32,146.83) --
	(107.44,146.85) --
	(107.52,146.84) --
	(107.64,146.82) --
	(107.84,146.78) --
	(107.98,146.79) --
	(108.07,146.82) --
	(108.11,146.85) --
	(108.14,146.89) --
	(108.18,147.01) --
	(108.16,147.13) --
	(108.18,147.16) --
	(108.30,147.28) --
	(108.30,147.32) --
	(108.12,147.79) --
	(108.11,147.86) --
	(108.13,147.92) --
	(108.16,147.98) --
	(108.20,148.03) --
	(108.42,148.20) --
	(108.98,148.50) --
	(109.09,148.60) --
	(109.18,148.75) --
	(109.22,148.88) --
	(109.26,148.96) --
	(109.57,149.19) --
	(109.66,149.28) --
	(109.72,149.39) --
	(109.73,149.44) --
	(109.79,149.50) --
	(109.84,149.54) --
	(110.06,149.64) --
	(110.17,149.65) --
	(110.33,149.65) --
	(110.44,149.62) --
	(110.57,149.54) --
	(110.73,149.43) --
	(110.93,149.29) --
	(111.01,149.26) --
	(111.13,149.26) --
	(111.26,149.35) --
	(111.32,149.42) --
	(111.38,149.52) --
	(111.37,149.67) --
	(111.33,149.83) --
	(111.35,149.92) --
	(111.38,150.02) --
	(111.45,150.20) --
	(111.62,150.50) --
	(111.72,150.57) --
	(111.90,150.61) --
	(112.04,150.63) --
	(112.10,150.71) --
	(112.15,150.80) --
	(112.11,151.17) --
	(112.17,151.28) --
	(112.26,151.42) --
	(112.80,152.02) --
	(112.92,152.10) --
	(113.02,152.12) --
	(113.08,152.10) --
	(113.37,151.90) --
	(113.46,151.89) --
	(113.53,151.90) --
	(113.68,152.07) --
	(113.76,152.13) --
	(114.06,152.25) --
	(114.12,152.31) --
	(114.27,152.55) --
	(114.35,152.61) --
	(114.46,152.68) --
	(114.80,152.77) --
	(114.96,152.87) --
	(115.04,153.00) --
	(115.04,153.08) --
	(114.96,153.25) --
	(114.99,153.34) --
	(115.30,153.53) --
	(115.44,153.68) --
	(115.53,153.74) --
	(115.62,153.81) --
	(115.97,153.93) --
	(116.00,153.95) --
	(116.08,154.13);

\draw[color=drawColor,line cap=round,line join=round,fill opacity=0.00,] ( 62.43,139.75) --
	( 62.42,139.72) --
	( 62.43,139.64) --
	( 62.43,139.61) --
	( 62.39,139.49) --
	( 62.40,139.37) --
	( 62.42,139.31) --
	( 62.53,139.18) --
	( 62.60,139.07) --
	( 62.60,139.00) --
	( 62.57,138.86) --
	( 62.59,138.79) --
	( 62.62,138.71) --
	( 62.69,138.64) --
	( 62.73,138.55) --
	( 62.78,138.38) --
	( 62.82,138.28) --
	( 62.83,138.22) --
	( 62.82,138.15) --
	( 62.73,137.74) --
	( 62.72,137.54) --
	( 62.73,137.48) --
	( 62.76,137.44) --
	( 62.81,137.37) --
	( 62.83,137.36) --
	( 62.99,137.29) --
	( 63.01,137.28) --
	( 63.05,137.15) --
	( 63.08,137.10) --
	( 63.21,136.97) --
	( 63.26,136.75) --
	( 63.23,136.54) --
	( 63.09,136.24) --
	( 63.05,136.07);

\draw[color=drawColor,line cap=round,line join=round,fill opacity=0.00,] ( 63.01,135.19) --
	( 63.04,135.26) --
	( 63.03,135.30) --
	( 63.01,135.41) --
	( 63.01,135.44) --
	( 63.02,135.49) --
	( 63.03,135.52) --
	( 63.03,135.55) --
	( 63.02,135.59) --
	( 63.01,135.61) --
	( 63.02,135.64) --
	( 63.11,135.78) --
	( 63.11,135.80) --
	( 63.08,135.88) --
	( 63.06,135.96) --
	( 63.05,136.07);

\draw[color=drawColor,line cap=round,line join=round,fill opacity=0.00,] (108.05,132.50) --
	(107.92,132.37) --
	(107.85,132.22) --
	(107.76,132.13) --
	(107.67,132.07) --
	(107.56,132.04) --
	(107.46,132.00) --
	(107.37,131.99) --
	(107.14,132.04) --
	(106.96,132.08) --
	(106.86,132.07) --
	(106.76,132.04) --
	(106.70,132.00) --
	(106.61,131.93) --
	(106.51,131.86) --
	(106.40,131.80) --
	(106.22,131.77) --
	(106.02,131.76);

\draw[color=drawColor,line cap=round,line join=round,fill opacity=0.00,] (106.02,131.76) --
	(105.94,131.81) --
	(105.90,131.86) --
	(105.77,131.96) --
	(105.68,132.09) --
	(105.57,132.26);

\draw[color=drawColor,line cap=round,line join=round,fill opacity=0.00,] (105.57,132.26) --
	(105.47,132.26) --
	(105.42,132.25) --
	(105.28,132.19) --
	(104.99,132.09) --
	(104.73,131.97) --
	(104.63,131.92) --
	(104.57,131.76) --
	(104.50,131.73) --
	(103.93,131.87) --
	(103.76,131.87) --
	(103.30,132.00) --
	(103.20,131.99) --
	(103.13,131.99) --
	(103.06,131.96) --
	(103.00,131.90) --
	(102.96,131.85) --
	(102.96,131.77);

\draw[color=drawColor,line cap=round,line join=round,fill opacity=0.00,] (126.36,131.76) --
	(126.12,131.76) --
	(125.92,131.72) --
	(125.83,131.67) --
	(125.74,131.64) --
	(125.65,131.59) --
	(125.55,131.54) --
	(125.49,131.51) --
	(125.34,131.56) --
	(125.28,131.57) --
	(125.02,131.53) --
	(124.95,131.53) --
	(124.88,131.54) --
	(124.60,131.66) --
	(124.52,131.68) --
	(124.42,131.69) --
	(124.30,131.66) --
	(124.21,131.63) --
	(124.11,131.53);

\draw[color=drawColor,line cap=round,line join=round,fill opacity=0.00,] (129.57,131.42) --
	(129.40,131.40) --
	(129.33,131.39) --
	(129.27,131.36) --
	(129.20,131.32) --
	(129.17,131.31) --
	(129.04,131.30);

\draw[color=drawColor,line cap=round,line join=round,fill opacity=0.00,] (129.04,131.30) --
	(128.90,131.32) --
	(128.57,131.44) --
	(128.22,131.58) --
	(127.89,131.73) --
	(127.84,131.73) --
	(127.80,131.73) --
	(127.54,131.61) --
	(127.51,131.59) --
	(126.98,131.56) --
	(126.92,131.56) --
	(126.85,131.58) --
	(126.75,131.62) --
	(126.60,131.68) --
	(126.52,131.72) --
	(126.36,131.76);

\draw[color=drawColor,line cap=round,line join=round,fill opacity=0.00,] ( 96.70,130.60) --
	( 96.54,130.65) --
	( 96.37,130.66) --
	( 96.19,130.70) --
	( 95.95,130.76) --
	( 95.81,130.86) --
	( 95.75,130.95) --
	( 95.71,131.05) --
	( 95.74,131.12) --
	( 95.78,131.21) --
	( 95.87,131.30) --
	( 95.99,131.33) --
	( 96.07,131.32) --
	( 96.16,131.28) --
	( 96.33,131.18) --
	( 96.54,130.99) --
	( 96.61,130.97) --
	( 97.08,131.00) --
	( 97.12,130.98) --
	( 97.13,130.95);

\draw[color=drawColor,line cap=round,line join=round,fill opacity=0.00,] (124.11,131.53) --
	(123.95,131.38) --
	(123.89,131.36) --
	(123.45,131.28) --
	(123.39,131.24) --
	(123.30,131.22) --
	(123.02,131.22) --
	(122.95,131.21) --
	(122.89,131.18) --
	(122.81,131.13) --
	(122.65,130.99) --
	(122.61,130.93) --
	(122.59,130.88) --
	(122.59,130.83) --
	(122.66,130.72) --
	(122.66,130.69) --
	(122.64,130.68) --
	(122.60,130.67) --
	(122.36,130.75) --
	(122.33,130.75) --
	(121.96,130.71) --
	(121.87,130.69) --
	(121.80,130.64) --
	(121.53,130.58);

\draw[color=drawColor,line cap=round,line join=round,fill opacity=0.00,] ( 65.59,130.41) --
	( 65.35,130.60) --
	( 65.15,130.76) --
	( 65.03,130.91) --
	( 64.95,131.06) --
	( 64.93,131.09) --
	( 64.85,131.12) --
	( 64.69,131.20) --
	( 64.66,131.22) --
	( 64.63,131.26) --
	( 64.59,131.38) --
	( 64.58,131.40) --
	( 64.39,131.59) --
	( 64.14,131.88) --
	( 64.14,131.90) --
	( 64.09,132.03) --
	( 64.09,132.05) --
	( 64.02,132.07) --
	( 64.01,132.10) --
	( 63.99,132.11) --
	( 63.98,132.16) --
	( 63.99,132.19) --
	( 64.01,132.26) --
	( 63.99,132.28) --
	( 63.99,132.30) --
	( 64.01,132.33) --
	( 64.04,132.36) --
	( 64.04,132.41) --
	( 64.02,132.43) --
	( 64.01,132.46) --
	( 63.98,132.51) --
	( 63.92,132.54) --
	( 63.86,132.57) --
	( 63.81,132.62) --
	( 63.79,132.64) --
	( 63.78,132.66) --
	( 63.76,132.71) --
	( 63.74,132.79) --
	( 63.74,132.83) --
	( 63.76,133.01) --
	( 63.78,133.03) --
	( 63.75,133.07) --
	( 63.66,133.20) --
	( 63.66,133.22) --
	( 63.64,133.26) --
	( 63.64,133.31) --
	( 63.64,133.37) --
	( 63.61,133.40) --
	( 63.54,133.53) --
	( 63.52,133.55) --
	( 63.44,133.60) --
	( 63.39,133.65) --
	( 63.35,133.69) --
	( 63.31,133.78) --
	( 63.28,133.84) --
	( 63.22,134.04) --
	( 63.19,134.17) --
	( 63.21,134.21) --
	( 63.29,134.35) --
	( 63.32,134.38) --
	( 63.32,134.42) --
	( 63.31,134.45) --
	( 63.29,134.47) --
	( 63.19,134.52) --
	( 63.18,134.55) --
	( 63.12,134.70) --
	( 63.09,134.78) --
	( 63.09,134.79) --
	( 63.08,134.81) --
	( 63.09,134.88) --
	( 63.09,134.90) --
	( 63.08,134.94) --
	( 63.06,134.98) --
	( 63.02,135.06) --
	( 63.01,135.08) --
	( 63.01,135.11) --
	( 63.01,135.19);

\draw[color=drawColor,line cap=round,line join=round,fill opacity=0.00,] (100.39,130.58) --
	(100.38,130.49) --
	(100.22,130.31) --
	(100.16,130.22);

\draw[color=drawColor,line cap=round,line join=round,fill opacity=0.00,] (121.53,130.58) --
	(121.47,130.54) --
	(121.45,130.53) --
	(121.37,130.42) --
	(121.28,130.37) --
	(121.13,130.20) --
	(121.09,130.16) --
	(121.03,130.15) --
	(120.79,130.14) --
	(120.71,130.12) --
	(120.66,130.11) --
	(120.53,130.07);

\draw[color=drawColor,line cap=round,line join=round,fill opacity=0.00,] (133.09,130.05) --
	(133.00,130.06) --
	(132.92,130.10) --
	(132.88,130.14) --
	(132.71,130.29) --
	(132.69,130.32) --
	(132.49,130.35) --
	(132.44,130.38) --
	(132.28,130.48) --
	(132.01,130.60) --
	(131.73,130.69) --
	(131.66,130.71) --
	(131.60,130.76) --
	(131.47,130.94) --
	(131.41,130.98) --
	(131.34,131.02) --
	(131.28,131.02) --
	(131.16,131.00) --
	(131.11,131.02) --
	(130.97,131.08) --
	(130.89,131.10) --
	(130.67,131.09) --
	(130.55,131.08) --
	(130.47,131.09) --
	(130.35,131.13) --
	(130.22,131.18) --
	(129.90,131.32) --
	(129.87,131.33) --
	(129.81,131.35) --
	(129.70,131.36) --
	(129.66,131.36) --
	(129.57,131.42);

\draw[color=drawColor,line cap=round,line join=round,fill opacity=0.00,] ( 98.80,130.00) --
	( 98.58,129.93) --
	( 98.51,129.92) --
	( 98.43,129.91);

\draw[color=drawColor,line cap=round,line join=round,fill opacity=0.00,] (120.53,130.07) --
	(120.40,130.06) --
	(120.37,130.03) --
	(120.28,129.92) --
	(120.25,129.85) --
	(120.26,129.82) --
	(120.29,129.76) --
	(120.29,129.73) --
	(120.29,129.67);

\draw[color=drawColor,line cap=round,line join=round,fill opacity=0.00,] ( 32.13,129.78) --
	( 32.05,129.75) --
	( 31.97,129.67) --
	( 31.96,129.67) --
	( 31.93,129.67) --
	( 31.60,129.63) --
	( 31.57,129.63) --
	( 31.44,129.70) --
	( 31.41,129.71) --
	( 31.39,129.70) --
	( 31.31,129.68) --
	( 31.27,129.66) --
	( 31.16,129.63) --
	( 31.12,129.62) --
	( 31.07,129.61) --
	( 30.98,129.54);

\draw[color=drawColor,line cap=round,line join=round,fill opacity=0.00,] ( 32.13,129.78) --
	( 32.21,129.63) --
	( 32.32,129.50) --
	( 32.36,129.49) --
	( 32.43,129.47) --
	( 32.50,129.47) --
	( 32.60,129.48) --
	( 32.69,129.54) --
	( 32.73,129.56) --
	( 32.93,129.57) --
	( 33.00,129.63) --
	( 33.07,129.65) --
	( 33.11,129.64) --
	( 33.24,129.61) --
	( 33.30,129.61) --
	( 33.34,129.61) --
	( 33.41,129.66) --
	( 33.44,129.66) --
	( 33.70,129.65) --
	( 33.72,129.66) --
	( 33.74,129.71) --
	( 33.87,129.69) --
	( 33.95,129.68) --
	( 34.07,129.65) --
	( 34.11,129.64) --
	( 34.25,129.69) --
	( 34.31,129.69) --
	( 34.45,129.63) --
	( 34.52,129.61) --
	( 34.56,129.62) --
	( 34.62,129.63) --
	( 34.64,129.66) --
	( 34.71,129.72) --
	( 34.75,129.74) --
	( 34.85,129.70);

\draw[color=drawColor,line cap=round,line join=round,fill opacity=0.00,] (120.29,129.67) --
	(120.24,129.67) --
	(120.17,129.65) --
	(120.13,129.61) --
	(120.05,129.53) --
	(120.01,129.51) --
	(119.97,129.50) --
	(119.91,129.50) --
	(119.88,129.50) --
	(119.86,129.52) --
	(119.84,129.53) --
	(119.84,129.56) --
	(119.84,129.62) --
	(119.82,129.65) --
	(119.79,129.64) --
	(119.70,129.58) --
	(119.67,129.58) --
	(119.66,129.59) --
	(119.64,129.59) --
	(119.64,129.60) --
	(119.63,129.63) --
	(119.67,129.67) --
	(119.70,129.70) --
	(119.72,129.73) --
	(119.71,129.75) --
	(119.65,129.86) --
	(119.56,130.02) --
	(119.54,130.05) --
	(119.50,130.07) --
	(119.47,130.07) --
	(119.44,130.07) --
	(119.43,130.04) --
	(119.41,130.03) --
	(119.40,129.99) --
	(119.36,129.79) --
	(119.35,129.78) --
	(119.32,129.77) --
	(119.12,129.75) --
	(119.11,129.74) --
	(119.08,129.73) --
	(119.06,129.70) --
	(118.96,129.55) --
	(118.93,129.52) --
	(118.93,129.48) --
	(118.93,129.46) --
	(118.91,129.45) --
	(118.94,129.43) --
	(118.97,129.40) --
	(119.08,129.37) --
	(119.09,129.34) --
	(119.11,129.30) --
	(119.09,129.28) --
	(119.05,129.25) --
	(118.98,129.23) --
	(118.92,129.23) --
	(118.88,129.23) --
	(118.82,129.25) --
	(118.69,129.32) --
	(118.67,129.33) --
	(118.57,129.31) --
	(118.55,129.31) --
	(118.55,129.39) --
	(118.54,129.40) --
	(118.53,129.41) --
	(118.41,129.43) --
	(118.30,129.46) --
	(118.24,129.46) --
	(118.09,129.45) --
	(118.04,129.44) --
	(117.98,129.41) --
	(117.85,129.23) --
	(117.82,129.21) --
	(117.79,129.19) --
	(117.76,129.19) --
	(117.69,129.18) --
	(117.59,129.19) --
	(117.52,129.20) --
	(117.45,129.21) --
	(117.30,129.30);

\draw[color=drawColor,line cap=round,line join=round,fill opacity=0.00,] (134.71,128.71) --
	(134.38,128.80) --
	(134.06,128.85) --
	(133.96,128.90) --
	(133.77,129.09) --
	(133.74,129.14) --
	(133.62,129.51) --
	(133.64,129.56) --
	(133.70,129.66) --
	(133.73,129.71) --
	(133.72,129.74) --
	(133.66,129.80) --
	(133.53,129.92) --
	(133.50,129.95) --
	(133.33,130.03) --
	(133.27,130.06) --
	(133.09,130.05);

\draw[color=drawColor,line cap=round,line join=round,fill opacity=0.00,] ( 38.92,128.76) --
	( 38.82,128.70) --
	( 38.78,128.69) --
	( 38.74,128.67) --
	( 38.70,128.68) --
	( 38.63,128.69) --
	( 38.58,128.68) --
	( 38.36,128.65) --
	( 38.30,128.65) --
	( 38.16,128.67) --
	( 37.94,128.67) --
	( 37.90,128.67) --
	( 37.80,128.71) --
	( 37.61,128.75) --
	( 37.46,128.76) --
	( 37.37,128.79) --
	( 37.27,128.84) --
	( 37.18,128.93) --
	( 37.15,128.94) --
	( 37.07,128.94) --
	( 36.97,128.95) --
	( 36.93,128.94) --
	( 36.83,128.87) --
	( 36.79,128.86) --
	( 36.74,128.86) --
	( 36.69,128.87) --
	( 36.57,128.94) --
	( 36.46,128.96) --
	( 36.41,128.97) --
	( 36.18,129.17) --
	( 36.12,129.18) --
	( 36.03,129.19) --
	( 35.95,129.21) --
	( 35.72,129.35) --
	( 35.67,129.38) --
	( 35.58,129.39) --
	( 35.55,129.41) --
	( 35.52,129.49) --
	( 35.49,129.50) --
	( 35.38,129.51) --
	( 35.34,129.52) --
	( 35.19,129.57) --
	( 34.99,129.65) --
	( 34.85,129.70);

\draw[color=drawColor,line cap=round,line join=round,fill opacity=0.00,] (116.73,128.36) --
	(116.69,128.37) --
	(116.63,128.40) --
	(116.56,128.42) --
	(116.52,128.45) --
	(116.38,128.57) --
	(116.36,128.58) --
	(116.31,128.57) --
	(116.29,128.56) --
	(116.22,128.46) --
	(116.20,128.44) --
	(116.18,128.45) --
	(116.13,128.45) --
	(116.12,128.48) --
	(116.12,128.50) --
	(116.11,128.62) --
	(116.10,128.64) --
	(116.05,128.65) --
	(116.01,128.66) --
	(115.97,128.65) --
	(115.92,128.64) --
	(115.85,128.57) --
	(115.84,128.55) --
	(115.79,128.57) --
	(115.78,128.59) --
	(115.78,128.62) --
	(115.78,128.64) --
	(115.80,128.67) --
	(115.83,128.69) --
	(116.01,128.79) --
	(116.06,128.82) --
	(116.09,128.84) --
	(116.09,128.86) --
	(116.09,128.90) --
	(116.06,128.92) --
	(115.91,129.05) --
	(115.89,129.08) --
	(115.90,129.10) --
	(115.88,129.11) --
	(115.91,129.13) --
	(115.94,129.16) --
	(116.04,129.21) --
	(116.06,129.23) --
	(116.03,129.27) --
	(116.01,129.28) --
	(115.97,129.29) --
	(115.93,129.28) --
	(115.86,129.26) --
	(115.84,129.24) --
	(115.74,129.05) --
	(115.71,129.04) --
	(115.68,129.03) --
	(115.62,129.04) --
	(115.59,129.04) --
	(115.58,129.06) --
	(115.56,129.21) --
	(115.54,129.22) --
	(115.51,129.21) --
	(115.49,129.21) --
	(115.47,129.19) --
	(115.45,129.15) --
	(115.42,128.77) --
	(115.41,128.74) --
	(115.40,128.74) --
	(115.37,128.72) --
	(115.31,128.72) --
	(115.24,128.73) --
	(115.19,128.74) --
	(115.19,128.77) --
	(115.17,128.79) --
	(115.18,128.87) --
	(115.19,128.91) --
	(115.18,128.94) --
	(115.16,128.95) --
	(115.15,128.96) --
	(115.12,128.96) --
	(115.09,128.95) --
	(115.08,128.93) --
	(115.05,128.88) --
	(115.01,128.62) --
	(115.00,128.59) --
	(114.98,128.58) --
	(114.95,128.57) --
	(114.91,128.57) --
	(114.87,128.56) --
	(114.80,128.59) --
	(114.77,128.60) --
	(114.76,128.62) --
	(114.70,128.66) --
	(114.70,128.69) --
	(114.67,128.73) --
	(114.67,128.75) --
	(114.69,128.79) --
	(114.70,128.81) --
	(114.75,128.85) --
	(114.85,128.91) --
	(114.97,128.97) --
	(114.98,128.98) --
	(114.98,129.01) --
	(114.97,129.03) --
	(114.94,129.06) --
	(114.90,129.07) --
	(114.84,129.06) --
	(114.78,129.05) --
	(114.72,129.01) --
	(114.67,128.95) --
	(114.59,128.90) --
	(114.56,128.90) --
	(114.52,128.91) --
	(114.49,128.91) --
	(114.45,128.95) --
	(114.37,129.04) --
	(114.36,129.08) --
	(114.33,129.14) --
	(114.33,129.15) --
	(114.33,129.20) --
	(114.32,129.21) --
	(114.23,129.31) --
	(114.23,129.33) --
	(114.29,129.41) --
	(114.29,129.44) --
	(114.28,129.53) --
	(114.26,129.58) --
	(114.19,129.63) --
	(114.12,129.70) --
	(114.07,129.70) --
	(114.03,129.70) --
	(113.80,129.62) --
	(113.77,129.62) --
	(113.76,129.63) --
	(113.73,129.68) --
	(113.72,129.73) --
	(113.72,129.80) --
	(113.73,129.83) --
	(113.78,129.95) --
	(113.78,129.99) --
	(113.74,130.02) --
	(113.71,130.03) --
	(113.64,130.05) --
	(113.60,130.06) --
	(113.57,130.04) --
	(113.54,130.01) --
	(113.38,129.76) --
	(113.35,129.74) --
	(113.30,129.72) --
	(113.25,129.71) --
	(113.20,129.72) --
	(113.16,129.72) --
	(113.13,129.74) --
	(113.11,129.78) --
	(113.09,129.81) --
	(113.09,129.84) --
	(113.26,130.04) --
	(113.26,130.07) --
	(113.26,130.10) --
	(113.07,130.17) --
	(113.06,130.20) --
	(112.99,130.34) --
	(112.98,130.36) --
	(112.95,130.36) --
	(112.84,130.35) --
	(112.79,130.32) --
	(112.75,130.30) --
	(112.72,130.26) --
	(112.63,130.09) --
	(112.62,130.07) --
	(112.57,130.07) --
	(112.53,130.06) --
	(112.49,130.09) --
	(112.44,130.10) --
	(112.42,130.13) --
	(112.39,130.22) --
	(112.35,130.37) --
	(112.32,130.40) --
	(112.20,130.52) --
	(112.19,130.54) --
	(112.20,130.56) --
	(112.27,130.64) --
	(112.26,130.66) --
	(112.26,130.69) --
	(112.23,130.75) --
	(112.19,130.80) --
	(112.14,130.82) --
	(112.08,130.84) --
	(112.01,130.84) --
	(111.96,130.82) --
	(111.92,130.81) --
	(111.73,130.69) --
	(111.72,130.68) --
	(111.68,130.67) --
	(111.53,130.67) --
	(111.49,130.67) --
	(111.46,130.70) --
	(111.45,130.73) --
	(111.42,130.75) --
	(111.40,130.95) --
	(111.38,130.99) --
	(111.33,131.05) --
	(111.23,131.10) --
	(110.95,131.15) --
	(110.93,131.16) --
	(110.90,131.20) --
	(110.83,131.44) --
	(110.81,131.48) --
	(110.65,131.68) --
	(110.49,131.85) --
	(110.41,131.92) --
	(110.37,131.94) --
	(110.23,131.98) --
	(110.18,131.99) --
	(109.97,132.12) --
	(109.94,132.13) --
	(109.50,132.21) --
	(109.48,132.23) --
	(109.45,132.25) --
	(109.43,132.28) --
	(109.42,132.32) --
	(109.44,132.34) --
	(109.46,132.36) --
	(109.51,132.43) --
	(109.61,132.56) --
	(109.61,132.59) --
	(109.60,132.63) --
	(109.53,132.66) --
	(109.46,132.67) --
	(109.43,132.66) --
	(109.27,132.59) --
	(109.23,132.59) --
	(109.08,132.56) --
	(108.86,132.57) --
	(108.50,132.64) --
	(108.37,132.66) --
	(108.29,132.66) --
	(108.24,132.65) --
	(108.21,132.63) --
	(108.05,132.50);

\draw[color=drawColor,line cap=round,line join=round,fill opacity=0.00,] (117.30,129.30) --
	(117.14,129.38) --
	(117.12,129.42) --
	(117.03,129.56) --
	(117.02,129.59) --
	(116.98,129.57) --
	(116.92,129.55) --
	(116.92,129.53) --
	(116.88,129.49) --
	(116.86,129.45) --
	(116.86,129.39) --
	(116.87,129.35) --
	(116.87,129.31) --
	(116.91,129.26) --
	(116.95,129.23) --
	(117.19,129.08) --
	(117.30,128.99) --
	(117.44,128.94) --
	(117.50,128.92) --
	(117.51,128.89) --
	(117.50,128.86) --
	(117.51,128.85) --
	(117.50,128.83) --
	(117.09,128.74) --
	(117.08,128.73) --
	(117.04,128.70) --
	(117.04,128.70) --
	(116.98,128.53) --
	(116.90,128.41) --
	(116.86,128.38) --
	(116.83,128.36) --
	(116.80,128.36) --
	(116.73,128.36);

\draw[color=drawColor,line cap=round,line join=round,fill opacity=0.00,] (186.08,127.80) --
	(185.95,127.89) --
	(185.85,127.99) --
	(185.80,128.09) --
	(185.79,128.22) --
	(185.83,128.40) --
	(185.82,128.51) --
	(185.95,128.63) --
	(186.15,128.77) --
	(186.23,128.86) --
	(186.30,129.08) --
	(186.38,129.19) --
	(186.51,129.23) --
	(186.58,129.36) --
	(186.51,129.42) --
	(186.49,129.51) --
	(186.63,129.67) --
	(186.66,129.87) --
	(186.66,130.05) --
	(186.67,130.12) --
	(186.82,130.30) --
	(186.91,130.47) --
	(186.95,130.78) --
	(186.94,131.10) --
	(186.90,131.23) --
	(186.97,131.36) --
	(187.21,131.48) --
	(187.58,131.70) --
	(187.83,132.15) --
	(187.90,132.27) --
	(187.85,132.39) --
	(187.89,132.44) --
	(187.98,132.43) --
	(188.01,132.49) --
	(187.96,132.64) --
	(187.98,132.68) --
	(188.08,132.73) --
	(188.12,132.81) --
	(188.19,132.88) --
	(188.50,132.98) --
	(188.57,133.04) --
	(188.66,133.07) --
	(188.76,133.06) --
	(188.87,133.09) --
	(189.00,133.20) --
	(189.14,133.30) --
	(189.41,133.55) --
	(189.72,133.83);

\draw[color=drawColor,line cap=round,line join=round,fill opacity=0.00,] ( 30.98,129.54) --
	( 30.87,129.68) --
	( 30.83,129.70) --
	( 30.79,129.68) --
	( 30.76,129.66) --
	( 30.72,129.60) --
	( 30.62,129.58) --
	( 30.60,129.57) --
	( 30.52,129.46) --
	( 30.48,129.43) --
	( 30.45,129.43) --
	( 30.31,129.53) --
	( 30.26,129.54) --
	( 30.22,129.53) --
	( 30.19,129.51) --
	( 30.11,129.32) --
	( 30.05,129.26) --
	( 29.98,129.22) --
	( 29.83,129.05) --
	( 29.76,129.00) --
	( 29.72,128.97) --
	( 29.68,128.89) --
	( 29.63,128.87) --
	( 29.51,128.82) --
	( 29.31,128.62) --
	( 29.21,128.50) --
	( 29.13,128.42) --
	( 29.06,128.34) --
	( 28.93,128.22) --
	( 28.82,128.13) --
	( 28.79,128.10) --
	( 28.75,127.99) --
	( 28.73,127.96) --
	( 28.64,127.89) --
	( 28.61,127.86) --
	( 28.59,127.75) --
	( 28.56,127.69) --
	( 28.49,127.63) --
	( 28.43,127.63) --
	( 28.38,127.63) --
	( 28.12,127.76) --
	( 28.06,127.77) --
	( 28.03,127.77) --
	( 27.99,127.75) --
	( 27.98,127.72) --
	( 27.90,127.70) --
	( 27.86,127.72) --
	( 27.79,127.75) --
	( 27.70,127.79) --
	( 27.65,127.81) --
	( 27.57,127.81) --
	( 27.38,127.79) --
	( 27.35,127.79) --
	( 27.23,127.82) --
	( 27.12,127.83) --
	( 27.03,127.82) --
	( 26.98,127.81) --
	( 26.94,127.76) --
	( 26.89,127.74) --
	( 26.79,127.68) --
	( 26.75,127.66) --
	( 26.70,127.54) --
	( 26.63,127.48) --
	( 26.59,127.45) --
	( 26.32,127.41) --
	( 26.27,127.39) --
	( 26.20,127.35) --
	( 26.15,127.27) --
	( 26.13,127.24) --
	( 26.04,127.06) --
	( 25.94,126.91) --
	( 25.90,126.89) --
	( 25.81,126.88) --
	( 25.70,126.88) --
	( 25.51,126.91) --
	( 25.44,126.91) --
	( 25.29,126.88) --
	( 25.25,126.88) --
	( 25.17,126.84) --
	( 25.12,126.79) --
	( 25.09,126.75) --
	( 25.05,126.70) --
	( 25.04,126.66) --
	( 24.99,126.65) --
	( 24.95,126.62) --
	( 24.90,126.57);

\draw[color=drawColor,line cap=round,line join=round,fill opacity=0.00,] (186.08,127.80) --
	(186.09,127.74) --
	(186.08,127.64) --
	(186.09,127.49) --
	(186.12,127.38) --
	(186.10,127.32) --
	(185.97,127.30) --
	(185.91,127.27) --
	(185.87,127.17) --
	(185.86,127.03) --
	(185.77,126.95) --
	(185.62,126.89) --
	(185.45,126.77) --
	(185.23,126.57) --
	(184.97,126.29) --
	(184.79,126.16) --
	(184.61,126.15) --
	(184.17,126.10) --
	(183.95,126.05) --
	(183.78,125.96) --
	(183.64,125.84) --
	(183.60,125.75) --
	(183.60,125.68) --
	(183.50,125.60) --
	(183.41,125.55) --
	(183.34,125.46);

\draw[color=drawColor,line cap=round,line join=round,fill opacity=0.00,] (182.46,125.80) --
	(182.62,125.76) --
	(182.75,125.53) --
	(182.97,125.38) --
	(183.11,125.38) --
	(183.34,125.46);

\draw[color=drawColor,line cap=round,line join=round,fill opacity=0.00,] (136.80,125.35) --
	(136.76,125.39) --
	(136.56,125.59) --
	(136.43,125.86) --
	(136.39,125.90) --
	(136.28,125.94) --
	(136.24,125.96) --
	(136.21,125.99) --
	(136.17,126.03) --
	(136.07,126.06) --
	(135.93,126.10) --
	(135.72,126.21) --
	(135.64,126.29) --
	(135.63,126.35) --
	(135.63,126.41) --
	(135.66,126.47) --
	(135.85,126.70) --
	(135.87,126.76) --
	(135.87,126.83) --
	(135.84,126.87) --
	(135.50,127.14) --
	(135.40,127.23) --
	(135.38,127.27) --
	(135.38,127.30) --
	(135.39,127.37) --
	(135.47,127.51) --
	(135.53,127.58) --
	(135.63,127.69) --
	(135.62,127.73) --
	(135.61,127.76) --
	(135.50,127.85) --
	(135.18,127.99) --
	(135.15,128.02) --
	(135.11,128.08) --
	(135.05,128.40) --
	(135.03,128.52) --
	(135.01,128.57) --
	(134.99,128.58) --
	(134.71,128.71);

\draw[color=drawColor,line cap=round,line join=round,fill opacity=0.00,] ( 67.37,124.82) --
	( 67.37,124.84) --
	( 67.28,125.03) --
	( 67.35,125.28) --
	( 67.35,125.31) --
	( 67.30,125.36) --
	( 67.26,125.43) --
	( 67.22,125.45) --
	( 67.12,125.49) --
	( 67.11,125.53) --
	( 67.11,125.55) --
	( 67.15,125.62) --
	( 67.16,125.64) --
	( 67.13,125.86) --
	( 67.12,125.88) --
	( 67.11,125.90) --
	( 67.05,125.92) --
	( 67.01,125.94) --
	( 66.99,125.97) --
	( 66.98,126.02) --
	( 66.96,126.06) --
	( 66.99,126.12) --
	( 67.01,126.15) --
	( 66.99,126.18) --
	( 66.91,126.23) --
	( 66.88,126.25) --
	( 66.85,126.33) --
	( 66.82,126.38) --
	( 66.62,126.56) --
	( 66.60,126.59) --
	( 66.58,126.62) --
	( 66.60,126.75) --
	( 66.58,126.80) --
	( 66.57,126.86) --
	( 66.47,126.97) --
	( 66.38,127.03) --
	( 66.24,127.13) --
	( 66.20,127.16) --
	( 66.17,127.21) --
	( 66.20,127.34) --
	( 66.18,127.37) --
	( 66.13,127.45) --
	( 66.08,127.54) --
	( 66.07,127.60) --
	( 66.08,127.65) --
	( 66.10,127.70) --
	( 66.10,127.75) --
	( 66.07,127.81) --
	( 66.06,127.83) --
	( 65.94,127.89) --
	( 65.91,127.92) --
	( 65.93,128.04) --
	( 65.91,128.10) --
	( 65.89,128.14) --
	( 65.79,128.23) --
	( 65.77,128.27) --
	( 65.84,128.35) --
	( 65.92,128.50) --
	( 66.01,128.77) --
	( 66.03,128.87) --
	( 66.02,128.96) --
	( 66.02,129.07) --
	( 65.99,129.16) --
	( 65.94,129.21) --
	( 65.83,129.26) --
	( 65.82,129.27) --
	( 65.82,129.51) --
	( 65.82,129.61) --
	( 65.79,129.68) --
	( 65.75,129.73) --
	( 65.72,129.79) --
	( 65.72,129.86) --
	( 65.73,130.07) --
	( 65.73,130.21) --
	( 65.70,130.26) --
	( 65.62,130.34) --
	( 65.59,130.41);

\draw[color=drawColor,line cap=round,line join=round,fill opacity=0.00,] ( 24.90,126.57) --
	( 24.83,126.49) --
	( 24.77,126.42) --
	( 24.71,126.37) --
	( 24.59,126.26) --
	( 24.49,126.19) --
	( 24.46,126.13) --
	( 24.44,126.06) --
	( 24.40,125.85) --
	( 24.39,125.80) --
	( 24.37,125.78) --
	( 24.36,125.77) --
	( 24.32,125.75) --
	( 23.89,125.73) --
	( 23.86,125.73) --
	( 23.79,125.70) --
	( 23.76,125.68) --
	( 23.67,125.57) --
	( 23.58,125.47) --
	( 23.47,125.39) --
	( 23.46,125.38) --
	( 23.44,125.35) --
	( 23.41,125.37) --
	( 23.31,125.43) --
	( 23.28,125.43) --
	( 23.28,125.42) --
	( 23.37,125.20) --
	( 23.38,125.17) --
	( 23.35,125.16) --
	( 23.32,125.17) --
	( 23.29,125.20) --
	( 23.20,125.28) --
	( 23.10,125.38) --
	( 23.06,125.41) --
	( 23.06,125.42) --
	( 23.03,125.43) --
	( 23.00,125.40) --
	( 22.99,125.39) --
	( 22.93,125.28) --
	( 22.90,125.28) --
	( 22.68,125.27) --
	( 22.65,125.25) --
	( 22.62,125.25) --
	( 22.62,125.24) --
	( 22.61,125.22) --
	( 22.58,125.15) --
	( 22.57,125.14) --
	( 22.54,125.12) --
	( 22.42,125.12) --
	( 22.41,125.11) --
	( 22.40,125.09) --
	( 22.38,125.04) --
	( 22.37,125.02) --
	( 22.20,125.00) --
	( 22.11,125.00) --
	( 22.03,124.99) --
	( 22.00,124.98) --
	( 21.97,124.94) --
	( 21.96,124.93) --
	( 21.96,124.91) --
	( 21.97,124.88) --
	( 22.02,124.80) --
	( 22.02,124.79);

\draw[color=drawColor,line cap=round,line join=round,fill opacity=0.00,] (178.90,125.24) --
	(179.20,125.15) --
	(179.45,125.13) --
	(179.60,125.08) --
	(179.72,125.08) --
	(179.86,125.06) --
	(180.09,125.05) --
	(180.30,124.95) --
	(180.49,124.92) --
	(180.59,124.86) --
	(180.66,124.76) --
	(180.67,124.73);

\draw[color=drawColor,line cap=round,line join=round,fill opacity=0.00,] (181.71,124.73) --
	(181.72,124.75) --
	(181.74,124.85) --
	(181.72,125.05) --
	(181.75,125.17) --
	(181.89,125.19) --
	(181.93,125.24) --
	(181.99,125.32) --
	(182.00,125.42) --
	(182.17,125.49) --
	(182.28,125.50) --
	(182.29,125.73) --
	(182.46,125.80);

\draw[color=drawColor,line cap=round,line join=round,fill opacity=0.00,] (136.80,125.35) --
	(136.81,125.33) --
	(136.96,125.14) --
	(137.02,124.93) --
	(137.21,124.84) --
	(137.38,124.85) --
	(137.69,124.85) --
	(137.79,124.74) --
	(137.81,124.71);

\draw[color=drawColor,line cap=round,line join=round,fill opacity=0.00,] (181.71,124.73) --
	(181.64,124.71) --
	(181.47,124.65) --
	(181.32,124.64) --
	(181.19,124.62) --
	(180.99,124.60) --
	(180.88,124.61) --
	(180.80,124.63) --
	(180.73,124.67) --
	(180.69,124.71) --
	(180.67,124.73);

\draw[color=drawColor,line cap=round,line join=round,fill opacity=0.00,] (137.81,124.71) --
	(137.82,124.70) --
	(137.84,124.66) --
	(137.95,124.60) --
	(138.10,124.58) --
	(138.23,124.53) --
	(138.34,124.47) --
	(138.41,124.43) --
	(138.42,124.37) --
	(138.40,124.25);

\draw[color=drawColor,line cap=round,line join=round,fill opacity=0.00,] (139.48,122.45) --
	(139.38,122.54) --
	(139.33,122.63) --
	(139.34,122.71) --
	(139.31,122.80) --
	(139.31,122.89) --
	(139.26,122.91) --
	(139.18,122.95) --
	(139.08,122.98) --
	(139.01,122.96) --
	(138.86,122.98) --
	(138.83,123.03) --
	(138.75,123.07) --
	(138.76,123.13) --
	(138.67,123.24) --
	(138.61,123.28) --
	(138.51,123.34) --
	(138.41,123.39) --
	(138.31,123.41) --
	(138.13,123.47) --
	(138.05,123.54) --
	(137.97,123.59) --
	(137.91,123.66) --
	(137.90,123.73) --
	(137.98,123.87) --
	(138.04,123.96) --
	(138.14,124.06) --
	(138.26,124.14) --
	(138.40,124.25);

\draw[color=drawColor,line cap=round,line join=round,fill opacity=0.00,] (139.80,122.22) --
	(139.63,122.33) --
	(139.50,122.39) --
	(139.48,122.45);

\draw[color=drawColor,line cap=round,line join=round,fill opacity=0.00,] (141.47,122.09) --
	(141.30,122.12) --
	(141.21,122.14);

\draw[color=drawColor,line cap=round,line join=round,fill opacity=0.00,] (141.21,122.14) --
	(141.18,122.24) --
	(141.09,122.32) --
	(141.03,122.32) --
	(140.89,122.28) --
	(140.70,122.18) --
	(140.67,122.06);

\draw[color=drawColor,line cap=round,line join=round,fill opacity=0.00,] ( 16.49,122.05) --
	( 16.63,122.26) --
	( 16.67,122.35) --
	( 16.76,122.44) --
	( 16.89,122.52) --
	( 16.97,122.55) --
	( 16.99,122.51) --
	( 17.03,122.51) --
	( 17.10,122.50) --
	( 17.28,122.62) --
	( 17.40,122.69) --
	( 17.42,122.75) --
	( 17.48,122.74) --
	( 17.49,122.78) --
	( 17.52,122.80) --
	( 17.58,122.80) --
	( 18.16,122.84) --
	( 18.20,122.83) --
	( 18.27,122.82) --
	( 18.33,122.78) --
	( 18.46,122.60) --
	( 18.58,122.56) --
	( 18.66,122.66) --
	( 18.73,122.69) --
	( 18.75,122.66) --
	( 18.80,122.74) --
	( 18.85,122.81) --
	( 18.83,122.84) --
	( 18.92,122.88) --
	( 18.95,122.95) --
	( 19.02,123.01) --
	( 19.07,123.19) --
	( 19.13,123.25) --
	( 19.34,123.32) --
	( 19.37,123.32) --
	( 19.43,123.30) --
	( 19.54,123.29) --
	( 19.55,123.26) --
	( 19.62,123.28) --
	( 19.65,123.30) --
	( 19.64,123.40) --
	( 19.69,123.42) --
	( 19.81,123.41) --
	( 19.86,123.41) --
	( 19.92,123.41) --
	( 19.90,123.56) --
	( 19.91,123.64) --
	( 19.99,123.65) --
	( 20.11,123.62) --
	( 20.34,123.59) --
	( 20.40,123.59) --
	( 20.44,123.67) --
	( 20.55,123.67) --
	( 20.62,123.72) --
	( 20.67,123.72) --
	( 20.69,123.75) --
	( 20.71,123.83) --
	( 20.72,123.93) --
	( 20.77,123.97) --
	( 20.83,123.94) --
	( 20.90,123.90) --
	( 20.95,123.86) --
	( 20.99,123.86) --
	( 21.07,123.87) --
	( 21.16,123.88) --
	( 21.22,123.89) --
	( 21.29,123.89) --
	( 21.34,123.91) --
	( 21.41,123.97) --
	( 21.42,124.18) --
	( 21.46,124.22) --
	( 21.62,124.33) --
	( 21.66,124.36) --
	( 21.70,124.37) --
	( 21.77,124.50) --
	( 21.80,124.52) --
	( 21.87,124.55) --
	( 21.92,124.63) --
	( 21.98,124.69) --
	( 22.00,124.75) --
	( 22.02,124.79);

\draw[color=drawColor,line cap=round,line join=round,fill opacity=0.00,] (141.21,122.14) --
	(140.93,122.13) --
	(140.73,122.03) --
	(140.67,122.06);

\draw[color=drawColor,line cap=round,line join=round,fill opacity=0.00,] (142.66,122.03) --
	(142.43,122.16) --
	(142.36,122.17) --
	(142.29,122.15) --
	(142.25,122.10) --
	(142.18,122.09) --
	(142.11,122.08) --
	(142.05,122.06) --
	(142.02,122.06) --
	(141.98,122.02);

\draw[color=drawColor,line cap=round,line join=round,fill opacity=0.00,] (141.98,122.02) --
	(141.82,122.15) --
	(141.76,122.15) --
	(141.71,122.16) --
	(141.57,122.15) --
	(141.50,122.12) --
	(141.47,122.09);

\draw[color=drawColor,line cap=round,line join=round,fill opacity=0.00,] (141.98,122.02) --
	(141.81,122.01) --
	(141.68,122.01) --
	(141.61,122.02) --
	(141.53,122.06) --
	(141.47,122.09);

\draw[color=drawColor,line cap=round,line join=round,fill opacity=0.00,] (140.42,121.99) --
	(140.39,122.18) --
	(140.30,122.22) --
	(140.10,122.26) --
	(140.03,122.28) --
	(139.91,122.33) --
	(139.72,122.40) --
	(139.65,122.43) --
	(139.55,122.44) --
	(139.48,122.45);

\draw[color=drawColor,line cap=round,line join=round,fill opacity=0.00,] (140.42,121.99) --
	(140.18,122.00) --
	(140.05,122.05) --
	(140.01,122.07) --
	(139.89,122.15) --
	(139.80,122.22);

\draw[color=drawColor,line cap=round,line join=round,fill opacity=0.00,] (140.67,122.06) --
	(140.51,121.97) --
	(140.42,121.99);

\draw[color=drawColor,line cap=round,line join=round,fill opacity=0.00,] (142.07,121.91) --
	(142.03,121.98) --
	(141.98,122.02);

\draw[color=drawColor,line cap=round,line join=round,fill opacity=0.00,] (142.53,121.88) --
	(142.60,121.95) --
	(142.66,122.03);

\draw[color=drawColor,line cap=round,line join=round,fill opacity=0.00,] (142.53,121.88) --
	(142.36,121.91) --
	(142.20,121.90) --
	(142.07,121.91);

\draw[color=drawColor,line cap=round,line join=round,fill opacity=0.00,] (143.13,121.62) --
	(143.17,121.73) --
	(143.13,121.81) --
	(143.00,121.86) --
	(142.88,121.94) --
	(142.78,121.96) --
	(142.74,121.99) --
	(142.66,122.03);

\draw[color=drawColor,line cap=round,line join=round,fill opacity=0.00,] (143.13,121.62) --
	(142.99,121.64) --
	(142.86,121.66) --
	(142.73,121.70) --
	(142.61,121.83) --
	(142.53,121.88);

\draw[color=drawColor,line cap=round,line join=round,fill opacity=0.00,] (143.39,121.49) --
	(143.31,121.51) --
	(143.25,121.51) --
	(143.18,121.54) --
	(143.13,121.62);

\draw[color=drawColor,line cap=round,line join=round,fill opacity=0.00,] (143.79,121.39) --
	(143.75,121.48) --
	(143.71,121.49) --
	(143.62,121.51) --
	(143.51,121.50) --
	(143.39,121.49);

\draw[color=drawColor,line cap=round,line join=round,fill opacity=0.00,] ( 67.37,124.82) --
	( 67.37,124.80) --
	( 67.36,124.55) --
	( 67.37,124.43) --
	( 67.40,124.38) --
	( 67.42,124.28) --
	( 67.47,124.24) --
	( 67.52,124.22) --
	( 67.57,124.24) --
	( 67.70,124.28) --
	( 67.76,124.27) --
	( 67.83,124.28) --
	( 67.86,124.27) --
	( 67.96,124.26) --
	( 68.08,124.25) --
	( 68.18,124.25) --
	( 68.24,124.27) --
	( 68.34,124.25) --
	( 68.37,124.29) --
	( 68.45,124.34) --
	( 68.51,124.39) --
	( 68.60,124.44) --
	( 68.65,124.42) --
	( 68.71,124.42) --
	( 68.92,124.37) --
	( 69.04,124.35) --
	( 69.16,124.34) --
	( 69.26,124.32) --
	( 69.31,124.31) --
	( 69.36,124.26) --
	( 69.42,124.19) --
	( 69.41,124.09) --
	( 69.38,124.01) --
	( 69.43,123.98) --
	( 69.50,123.89) --
	( 69.60,123.79) --
	( 69.59,123.74) --
	( 69.66,123.77) --
	( 69.73,123.78) --
	( 69.76,123.77) --
	( 69.80,123.72) --
	( 69.83,123.67) --
	( 69.89,123.67) --
	( 69.93,123.64) --
	( 70.01,123.60) --
	( 70.07,123.60) --
	( 70.16,123.58) --
	( 70.31,123.50) --
	( 70.43,123.45) --
	( 70.58,123.42) --
	( 70.68,123.43) --
	( 70.77,123.41) --
	( 70.85,123.38) --
	( 70.97,123.33) --
	( 71.04,123.30) --
	( 71.09,123.32) --
	( 71.18,123.37) --
	( 71.28,123.40) --
	( 71.33,123.42) --
	( 71.42,123.36) --
	( 71.58,123.33) --
	( 71.69,123.31) --
	( 71.77,123.32) --
	( 71.85,123.30) --
	( 71.92,123.27) --
	( 71.99,123.21) --
	( 72.09,123.02) --
	( 72.16,122.93) --
	( 72.24,122.87) --
	( 72.40,122.76) --
	( 72.65,122.59) --
	( 72.81,122.48) --
	( 72.93,122.39) --
	( 72.99,122.32) --
	( 73.09,122.29) --
	( 73.19,122.23) --
	( 73.25,122.23) --
	( 73.30,122.23) --
	( 73.33,122.20) --
	( 73.37,122.11) --
	( 73.42,122.02) --
	( 73.44,122.01) --
	( 73.57,121.92) --
	( 73.61,121.84) --
	( 73.60,121.78) --
	( 73.58,121.74) --
	( 73.63,121.71) --
	( 73.67,121.65) --
	( 73.77,121.60) --
	( 73.99,121.55) --
	( 74.02,121.47) --
	( 74.02,121.45) --
	( 73.99,121.34) --
	( 73.96,121.29);

\draw[color=drawColor,line cap=round,line join=round,fill opacity=0.00,] (143.80,121.22) --
	(143.78,121.31) --
	(143.79,121.39);

\draw[color=drawColor,line cap=round,line join=round,fill opacity=0.00,] (143.80,121.22) --
	(143.69,121.25) --
	(143.54,121.29) --
	(143.45,121.37) --
	(143.38,121.42) --
	(143.37,121.45) --
	(143.39,121.49);

\draw[color=drawColor,line cap=round,line join=round,fill opacity=0.00,] (144.11,120.97) --
	(144.16,121.08) --
	(144.19,121.18) --
	(144.14,121.24) --
	(144.08,121.28) --
	(144.00,121.33) --
	(143.88,121.38) --
	(143.79,121.39);

\draw[color=drawColor,line cap=round,line join=round,fill opacity=0.00,] (144.11,120.97) --
	(144.00,120.91) --
	(143.92,120.94) --
	(143.85,120.96) --
	(143.84,121.03) --
	(143.80,121.22);

\draw[color=drawColor,line cap=round,line join=round,fill opacity=0.00,] ( 73.96,121.29) --
	( 74.08,121.32) --
	( 74.19,121.39) --
	( 74.25,121.38) --
	( 74.28,121.33) --
	( 74.31,121.28) --
	( 74.29,121.21) --
	( 74.29,121.13) --
	( 74.33,120.97) --
	( 74.39,120.94) --
	( 74.46,120.87) --
	( 74.54,120.83) --
	( 74.63,120.85) --
	( 74.74,120.86) --
	( 74.84,120.86) --
	( 74.93,120.87) --
	( 75.10,120.85) --
	( 75.17,120.90) --
	( 75.14,120.96) --
	( 75.13,121.00) --
	( 75.00,121.11) --
	( 74.86,121.23) --
	( 74.72,121.34) --
	( 74.62,121.42) --
	( 74.60,121.48) --
	( 74.60,121.52) --
	( 74.65,121.57) --
	( 74.72,121.60) --
	( 74.79,121.54) --
	( 74.87,121.53) --
	( 74.93,121.54) --
	( 74.93,121.55) --
	( 75.02,121.60) --
	( 75.10,121.65) --
	( 75.16,121.66) --
	( 75.19,121.66) --
	( 75.23,121.65) --
	( 75.29,121.66) --
	( 75.34,121.72) --
	( 75.39,121.73) --
	( 75.42,121.84) --
	( 75.42,121.93) --
	( 75.44,121.97) --
	( 75.53,121.95) --
	( 75.60,122.00) --
	( 75.66,122.05) --
	( 75.74,122.14) --
	( 75.76,122.20) --
	( 75.84,122.23) --
	( 75.99,122.23) --
	( 76.11,122.21) --
	( 76.17,122.26) --
	( 76.20,122.36) --
	( 76.24,122.50) --
	( 76.29,122.62) --
	( 76.39,122.64) --
	( 76.51,122.71) --
	( 76.61,122.78) --
	( 76.69,122.84) --
	( 76.69,122.93) --
	( 76.69,123.01) --
	( 76.59,123.11) --
	( 76.49,123.20);

\draw[color=drawColor,line cap=round,line join=round,fill opacity=0.00,] (144.31,120.61) --
	(144.22,120.76) --
	(144.11,120.97);

\draw[color=drawColor,line cap=round,line join=round,fill opacity=0.00,] (144.54,120.30) --
	(144.50,120.41) --
	(144.38,120.50) --
	(144.31,120.61);

\draw[color=drawColor,line cap=round,line join=round,fill opacity=0.00,] (144.54,120.30) --
	(144.40,120.29) --
	(144.30,120.30) --
	(144.24,120.36) --
	(144.16,120.45) --
	(144.17,120.52) --
	(144.17,120.56) --
	(144.24,120.60) --
	(144.31,120.61);

\draw[color=drawColor,line cap=round,line join=round,fill opacity=0.00,] (145.28,119.88) --
	(145.19,120.08) --
	(145.14,120.18) --
	(145.12,120.33) --
	(145.10,120.48) --
	(145.02,120.53) --
	(144.87,120.61) --
	(144.76,120.63) --
	(144.65,120.64) --
	(144.45,120.62) --
	(144.35,120.62) --
	(144.31,120.61);

\draw[color=drawColor,line cap=round,line join=round,fill opacity=0.00,] ( 13.35,119.81) --
	( 13.38,119.81) --
	( 13.79,119.87) --
	( 14.16,119.97) --
	( 14.40,120.01) --
	( 14.58,120.09) --
	( 14.68,120.12) --
	( 14.74,120.15) --
	( 14.79,120.23) --
	( 14.85,120.31) --
	( 14.87,120.40) --
	( 14.88,120.44) --
	( 14.91,120.52) --
	( 14.95,120.52) --
	( 15.01,120.56) --
	( 15.08,120.60) --
	( 15.11,120.66) --
	( 15.18,120.71) --
	( 15.19,120.81) --
	( 15.23,120.83) --
	( 15.27,120.85) --
	( 15.34,120.86) --
	( 15.37,120.88) --
	( 15.43,120.90) --
	( 15.48,120.93) --
	( 15.59,121.08) --
	( 15.72,121.15) --
	( 15.79,121.18) --
	( 15.84,121.23) --
	( 15.86,121.28) --
	( 15.84,121.32) --
	( 15.94,121.32) --
	( 16.07,121.41) --
	( 16.12,121.54) --
	( 16.10,121.58) --
	( 16.17,121.64) --
	( 16.30,121.75) --
	( 16.30,121.85) --
	( 16.39,121.97) --
	( 16.44,122.01) --
	( 16.49,122.05);

\draw[color=drawColor,line cap=round,line join=round,fill opacity=0.00,] (172.22,119.81) --
	(172.15,119.84) --
	(171.85,120.14) --
	(171.68,120.21) --
	(171.59,120.33) --
	(171.46,120.39) --
	(171.33,120.45) --
	(171.16,120.50) --
	(171.02,120.48) --
	(170.86,120.42) --
	(170.77,120.35) --
	(170.68,120.25) --
	(170.61,120.23) --
	(170.53,120.20) --
	(170.46,120.21) --
	(170.34,120.23) --
	(170.23,120.27) --
	(170.17,120.26) --
	(170.05,120.21) --
	(170.03,120.14) --
	(170.01,120.07) --
	(170.02,120.01) --
	(169.95,119.98) --
	(169.88,119.96) --
	(169.80,119.98) --
	(169.76,119.97) --
	(169.70,119.97) --
	(169.68,119.96);

\draw[color=drawColor,line cap=round,line join=round,fill opacity=0.00,] (145.45,119.70) --
	(145.43,119.81) --
	(145.40,119.85) --
	(145.34,119.88) --
	(145.28,119.88);

\draw[color=drawColor,line cap=round,line join=round,fill opacity=0.00,] (145.28,119.88) --
	(145.13,119.84) --
	(145.03,119.84) --
	(144.99,119.80) --
	(144.77,119.68);

\draw[color=drawColor,line cap=round,line join=round,fill opacity=0.00,] (144.77,119.68) --
	(144.75,119.69) --
	(144.65,119.70) --
	(144.58,119.74) --
	(144.51,119.76) --
	(144.48,119.83) --
	(144.47,119.89) --
	(144.47,119.95) --
	(144.51,120.07) --
	(144.54,120.15) --
	(144.54,120.30);

\draw[color=drawColor,line cap=round,line join=round,fill opacity=0.00,] (145.45,119.70) --
	(145.19,119.67) --
	(145.01,119.68) --
	(144.77,119.68);

\draw[color=drawColor,line cap=round,line join=round,fill opacity=0.00,] (169.68,119.96) --
	(169.60,119.89) --
	(169.54,119.84) --
	(169.52,119.75) --
	(169.43,119.72) --
	(169.39,119.68) --
	(169.31,119.61) --
	(169.25,119.57);

\draw[color=drawColor,line cap=round,line join=round,fill opacity=0.00,] (147.13,119.24) --
	(147.13,119.30) --
	(147.10,119.37) --
	(147.01,119.42) --
	(146.86,119.47) --
	(146.77,119.45) --
	(146.67,119.45) --
	(146.63,119.43);

\draw[color=drawColor,line cap=round,line join=round,fill opacity=0.00,] (147.13,119.24) --
	(146.99,119.22) --
	(146.82,119.33) --
	(146.73,119.37) --
	(146.63,119.43);

\draw[color=drawColor,line cap=round,line join=round,fill opacity=0.00,] (146.63,119.43) --
	(146.53,119.42) --
	(146.46,119.42) --
	(146.38,119.40) --
	(146.12,119.09);

\draw[color=drawColor,line cap=round,line join=round,fill opacity=0.00,] (146.12,119.09) --
	(145.95,119.19) --
	(145.90,119.21) --
	(145.84,119.23) --
	(145.80,119.26) --
	(145.80,119.41) --
	(145.79,119.54) --
	(145.72,119.63) --
	(145.62,119.66) --
	(145.52,119.69) --
	(145.45,119.70);

\draw[color=drawColor,line cap=round,line join=round,fill opacity=0.00,] (169.25,119.57) --
	(169.20,119.45) --
	(169.16,119.34) --
	(169.10,119.21) --
	(169.04,119.10) --
	(168.95,118.98) --
	(168.86,118.97) --
	(168.73,118.94);

\draw[color=drawColor,line cap=round,line join=round,fill opacity=0.00,] (168.73,118.94) --
	(168.63,118.99) --
	(168.51,119.02) --
	(168.34,119.04) --
	(168.23,119.01) --
	(168.19,118.97) --
	(168.18,118.92);

\draw[color=drawColor,line cap=round,line join=round,fill opacity=0.00,] (168.18,118.92) --
	(168.06,118.86) --
	(167.98,118.82);

\draw[color=drawColor,line cap=round,line join=round,fill opacity=0.00,] (168.18,118.92) --
	(168.23,118.84) --
	(168.38,118.80) --
	(168.45,118.79) --
	(168.53,118.78) --
	(168.63,118.79) --
	(168.65,118.83) --
	(168.70,118.90) --
	(168.73,118.94);

\draw[color=drawColor,line cap=round,line join=round,fill opacity=0.00,] ( 10.18,118.67) --
	( 10.58,118.99) --
	( 11.02,119.27) --
	( 11.34,119.39) --
	( 11.43,119.49) --
	( 11.58,119.50) --
	( 11.84,119.56) --
	( 11.97,119.63) --
	( 12.21,119.59) --
	( 12.45,119.60) --
	( 12.58,119.62) --
	( 12.67,119.66) --
	( 12.74,119.72) --
	( 12.83,119.77) --
	( 13.03,119.81) --
	( 13.07,119.83) --
	( 13.16,119.86) --
	( 13.32,119.82) --
	( 13.35,119.81);

\draw[color=drawColor,line cap=round,line join=round,fill opacity=0.00,] (167.98,118.82) --
	(167.77,118.79) --
	(167.67,118.70) --
	(167.63,118.64) --
	(167.66,118.61);

\draw[color=drawColor,line cap=round,line join=round,fill opacity=0.00,] (167.66,118.61) --
	(167.74,118.60) --
	(167.82,118.61) --
	(167.87,118.64) --
	(167.93,118.68) --
	(167.97,118.72) --
	(167.97,118.79) --
	(167.98,118.82);

\draw[color=drawColor,line cap=round,line join=round,fill opacity=0.00,] (150.92,118.48) --
	(150.83,118.58) --
	(150.71,118.63) --
	(150.61,118.65) --
	(150.53,118.65);

\draw[color=drawColor,line cap=round,line join=round,fill opacity=0.00,] (150.92,118.48) --
	(150.80,118.42) --
	(150.70,118.47) --
	(150.64,118.53) --
	(150.57,118.57) --
	(150.53,118.65);

\draw[color=drawColor,line cap=round,line join=round,fill opacity=0.00,] (150.53,118.65) --
	(150.46,118.66) --
	(150.26,118.64) --
	(150.12,118.51) --
	(149.71,118.39) --
	(149.29,118.33) --
	(148.78,118.28) --
	(148.10,118.65) --
	(147.86,118.85) --
	(147.81,118.90) --
	(147.48,119.00) --
	(147.26,119.21) --
	(147.13,119.24);

\draw[color=drawColor,line cap=round,line join=round,fill opacity=0.00,] (153.49,118.08) --
	(153.39,118.06) --
	(153.28,118.03) --
	(153.14,118.01) --
	(153.01,117.98);

\draw[color=drawColor,line cap=round,line join=round,fill opacity=0.00,] (153.01,117.98) --
	(153.10,118.06) --
	(153.26,118.18) --
	(153.38,118.20) --
	(153.48,118.19) --
	(153.52,118.14) --
	(153.49,118.08);

\draw[color=drawColor,line cap=round,line join=round,fill opacity=0.00,] (150.92,118.48) --
	(151.06,118.41) --
	(151.22,118.36) --
	(151.39,118.35) --
	(151.53,118.33) --
	(151.74,118.34) --
	(151.90,118.33) --
	(152.08,118.42) --
	(152.23,118.45) --
	(152.33,118.52) --
	(152.36,118.55) --
	(152.46,118.56) --
	(152.57,118.50) --
	(152.59,118.46) --
	(152.66,118.36) --
	(152.69,118.23) --
	(152.76,118.10) --
	(152.84,118.04) --
	(152.90,117.99) --
	(153.01,117.98);

\draw[color=drawColor,line cap=round,line join=round,fill opacity=0.00,] (154.22,118.06) --
	(154.16,118.05) --
	(154.06,118.08) --
	(153.93,118.01) --
	(153.88,117.97) --
	(153.82,117.99) --
	(153.72,118.00) --
	(153.58,118.05) --
	(153.49,118.08);

\draw[color=drawColor,line cap=round,line join=round,fill opacity=0.00,] (154.46,117.94) --
	(154.43,118.03) --
	(154.40,118.07) --
	(154.34,118.09) --
	(154.23,118.07) --
	(154.22,118.06);

\draw[color=drawColor,line cap=round,line join=round,fill opacity=0.00,] (156.93,117.93) --
	(156.63,117.97) --
	(156.48,118.11) --
	(156.49,118.22) --
	(156.43,118.23) --
	(156.39,118.24) --
	(156.33,118.27);

\draw[color=drawColor,line cap=round,line join=round,fill opacity=0.00,] (156.33,118.27) --
	(156.16,118.31) --
	(156.02,118.26) --
	(155.91,118.23) --
	(155.80,118.18) --
	(155.71,118.16) --
	(155.57,118.11) --
	(155.37,118.01) --
	(155.20,117.97) --
	(155.13,117.95) --
	(155.03,117.94) --
	(154.97,117.94) --
	(154.87,117.95) --
	(154.83,117.95) --
	(154.76,117.98) --
	(154.73,117.99) --
	(154.66,117.96) --
	(154.55,117.93) --
	(154.46,117.94);

\draw[color=drawColor,line cap=round,line join=round,fill opacity=0.00,] (154.46,117.94) --
	(154.35,117.93) --
	(154.26,117.96) --
	(154.22,118.02) --
	(154.22,118.06);

\draw[color=drawColor,line cap=round,line join=round,fill opacity=0.00,] (167.66,118.61) --
	(167.48,118.47) --
	(167.27,118.48) --
	(167.19,118.25) --
	(167.17,118.14) --
	(167.09,118.00) --
	(167.04,117.90) --
	(167.02,117.89);

\draw[color=drawColor,line cap=round,line join=round,fill opacity=0.00,] (157.23,117.71) --
	(157.11,117.75) --
	(157.07,117.78) --
	(157.03,117.90) --
	(156.93,117.93);

\draw[color=drawColor,line cap=round,line join=round,fill opacity=0.00,] (167.02,117.89) --
	(166.90,117.80) --
	(166.84,117.72) --
	(166.76,117.69) --
	(166.65,117.61) --
	(166.61,117.59);

\draw[color=drawColor,line cap=round,line join=round,fill opacity=0.00,] (166.61,117.59) --
	(166.48,117.65) --
	(166.35,117.65) --
	(166.25,117.61) --
	(166.11,117.54) --
	(165.93,117.47) --
	(165.92,117.44) --
	(165.89,117.40);

\draw[color=drawColor,line cap=round,line join=round,fill opacity=0.00,] (165.89,117.40) --
	(165.95,117.35) --
	(166.03,117.34) --
	(166.09,117.39) --
	(166.14,117.41) --
	(166.22,117.42) --
	(166.29,117.42) --
	(166.37,117.45) --
	(166.41,117.45) --
	(166.48,117.48) --
	(166.52,117.52) --
	(166.58,117.57) --
	(166.61,117.59);

\draw[color=drawColor,line cap=round,line join=round,fill opacity=0.00,] ( 71.00,117.26) --
	( 71.10,117.42) --
	( 71.13,117.58) --
	( 71.17,117.70) --
	( 71.18,117.78) --
	( 71.30,117.93) --
	( 71.37,118.00) --
	( 71.41,118.05) --
	( 71.46,118.14) --
	( 71.51,118.22) --
	( 71.57,118.30) --
	( 71.61,118.34) --
	( 71.67,118.35) --
	( 71.75,118.43) --
	( 71.90,118.50) --
	( 72.01,118.58) --
	( 72.11,118.65) --
	( 72.17,118.82) --
	( 72.27,118.97) --
	( 72.35,119.10) --
	( 72.41,119.18) --
	( 72.52,119.25) --
	( 72.59,119.34) --
	( 72.62,119.43) --
	( 72.65,119.55) --
	( 72.68,119.72) --
	( 72.72,119.77) --
	( 72.74,119.82) --
	( 72.78,119.88) --
	( 72.84,119.95) --
	( 72.87,120.02) --
	( 72.90,120.08) --
	( 73.02,120.21) --
	( 73.05,120.30) --
	( 73.12,120.36) --
	( 73.18,120.39) --
	( 73.27,120.42) --
	( 73.25,120.47) --
	( 73.27,120.54) --
	( 73.31,120.66) --
	( 73.30,120.73) --
	( 73.35,120.80) --
	( 73.40,120.88) --
	( 73.52,120.93) --
	( 73.53,120.94) --
	( 73.58,120.97) --
	( 73.67,121.04) --
	( 73.75,121.19) --
	( 73.82,121.22) --
	( 73.92,121.24) --
	( 73.96,121.29);

\draw[color=drawColor,line cap=round,line join=round,fill opacity=0.00,] (159.75,117.21) --
	(159.55,117.24) --
	(159.27,117.31) --
	(159.21,117.42) --
	(159.05,117.46) --
	(158.89,117.55) --
	(158.75,117.59) --
	(157.56,117.47) --
	(157.30,117.56) --
	(157.23,117.71);

\draw[color=drawColor,line cap=round,line join=round,fill opacity=0.00,] (159.75,117.21) --
	(159.75,117.06);

\draw[color=drawColor,line cap=round,line join=round,fill opacity=0.00,] (160.61,116.98) --
	(160.38,117.06) --
	(160.19,117.09) --
	(160.02,117.13) --
	(159.94,117.17) --
	(159.84,117.19) --
	(159.75,117.21);

\draw[color=drawColor,line cap=round,line join=round,fill opacity=0.00,] (165.89,117.40) --
	(165.75,117.36) --
	(165.58,117.31) --
	(165.50,117.29) --
	(165.41,117.26) --
	(165.28,117.33) --
	(165.23,117.35) --
	(165.11,117.40) --
	(164.91,117.52) --
	(164.67,117.54) --
	(164.19,117.50) --
	(164.06,117.60) --
	(163.95,117.56) --
	(163.83,117.56) --
	(163.66,117.51) --
	(163.54,117.47) --
	(163.38,117.43) --
	(163.27,117.43) --
	(163.16,117.41) --
	(162.97,117.39) --
	(162.80,117.39) --
	(162.66,117.29) --
	(161.99,117.11) --
	(160.90,116.98) --
	(160.61,116.98);

\draw[color=drawColor,line cap=round,line join=round,fill opacity=0.00,] (159.75,117.06) --
	(159.86,117.00) --
	(159.89,116.94) --
	(159.93,116.91) --
	(160.00,116.90) --
	(160.11,116.91) --
	(160.28,116.89) --
	(160.38,116.89) --
	(160.48,116.94) --
	(160.61,116.98);

\draw[color=drawColor,line cap=round,line join=round,fill opacity=0.00,] ( 71.00,117.26) --
	( 70.83,117.11) --
	( 70.59,116.98) --
	( 70.53,116.95) --
	( 70.43,116.87) --
	( 70.39,116.80) --
	( 70.36,116.74);

\draw[color=drawColor,line cap=round,line join=round,fill opacity=0.00,] ( 70.36,116.74) --
	( 70.33,116.56) --
	( 70.33,116.38) --
	( 70.23,116.27) --
	( 70.13,116.21) --
	( 70.05,116.18) --
	( 69.98,116.17) --
	( 69.91,116.12) --
	( 69.92,116.08) --
	( 69.89,115.98) --
	( 69.90,115.91) --
	( 69.88,115.80) --
	( 69.82,115.73) --
	( 69.82,115.66) --
	( 69.79,115.62);

\draw[color=drawColor,line cap=round,line join=round,fill opacity=0.00,] ( 69.79,115.62) --
	( 69.86,115.46) --
	( 69.87,115.30) --
	( 69.83,115.13) --
	( 69.79,114.94) --
	( 69.67,114.77) --
	( 69.49,114.60) --
	( 69.41,114.51) --
	( 69.28,114.43) --
	( 69.19,114.38) --
	( 69.11,114.36) --
	( 68.92,114.32) --
	( 68.75,114.28) --
	( 68.70,114.20) --
	( 68.61,114.11) --
	( 68.58,113.99) --
	( 68.51,113.88) --
	( 68.48,113.83) --
	( 68.43,113.79) --
	( 68.34,113.70) --
	( 68.34,113.63) --
	( 68.33,113.54) --
	( 68.30,113.44) --
	( 68.33,113.38) --
	( 68.26,113.32) --
	( 68.17,113.27) --
	( 68.10,113.14) --
	( 68.03,113.13);

\draw[color=drawColor,line cap=round,line join=round,fill opacity=0.00,] (  6.17,112.65) --
	(  6.16,112.77) --
	(  6.17,112.88) --
	(  6.19,112.97) --
	(  6.22,113.04) --
	(  6.20,113.17) --
	(  6.10,113.24) --
	(  6.12,113.35) --
	(  6.15,113.50) --
	(  6.21,113.55) --
	(  6.27,113.65) --
	(  6.33,113.79) --
	(  6.31,113.94) --
	(  6.24,114.01) --
	(  6.26,114.09) --
	(  6.30,114.23) --
	(  6.31,114.30) --
	(  6.36,114.41) --
	(  6.40,114.60) --
	(  6.43,114.81) --
	(  6.44,114.89) --
	(  6.48,114.96) --
	(  6.53,115.06) --
	(  6.52,115.14) --
	(  6.58,115.24) --
	(  6.63,115.32) --
	(  6.67,115.40) --
	(  6.74,115.43) --
	(  6.79,115.47) --
	(  6.82,115.59) --
	(  6.87,115.79) --
	(  6.95,115.93) --
	(  7.00,116.04) --
	(  7.03,116.20) --
	(  7.05,116.29) --
	(  7.01,116.43) --
	(  7.06,116.52) --
	(  7.28,116.64) --
	(  7.37,116.79) --
	(  7.39,116.94) --
	(  7.53,116.95) --
	(  7.63,116.98) --
	(  7.72,116.99) --
	(  7.86,117.08) --
	(  7.95,117.16) --
	(  8.09,117.20) --
	(  8.19,117.26) --
	(  8.24,117.36) --
	(  8.22,117.45) --
	(  8.24,117.55) --
	(  8.34,117.63) --
	(  8.42,117.67) --
	(  8.50,117.69) --
	(  8.54,117.74) --
	(  8.53,117.81) --
	(  8.56,117.84) --
	(  8.62,117.84) --
	(  8.93,117.88) --
	(  9.10,117.96) --
	(  9.21,118.13) --
	(  9.37,118.27) --
	(  9.48,118.36) --
	(  9.55,118.41) --
	(  9.62,118.36) --
	(  9.76,118.36) --
	(  9.87,118.42) --
	(  9.93,118.48) --
	( 10.02,118.55) --
	( 10.12,118.63) --
	( 10.18,118.67);

\draw[color=drawColor,line cap=round,line join=round,fill opacity=0.00,] ( 48.43,109.73) --
	( 48.23,109.70) --
	( 48.05,109.58) --
	( 47.98,109.34) --
	( 47.83,109.08) --
	( 47.74,108.99) --
	( 47.59,108.85) --
	( 47.45,108.69) --
	( 47.31,108.60) --
	( 47.20,108.56) --
	( 47.11,108.62) --
	( 47.07,108.74) --
	( 46.97,108.84) --
	( 46.73,108.83) --
	( 46.53,108.76) --
	( 46.45,108.73) --
	( 46.38,108.69) --
	( 46.30,108.64) --
	( 46.14,108.53) --
	( 46.13,108.42) --
	( 46.16,108.24) --
	( 46.16,108.19) --
	( 46.13,108.01) --
	( 46.05,107.99) --
	( 45.95,107.91) --
	( 45.72,108.03) --
	( 45.57,108.08) --
	( 45.44,108.06) --
	( 45.37,108.04) --
	( 45.18,107.86) --
	( 45.12,107.72) --
	( 45.06,107.70);

\draw[color=drawColor,line cap=round,line join=round,fill opacity=0.00,] ( 45.06,107.70) --
	( 44.97,107.67) --
	( 44.70,107.57) --
	( 44.46,107.50) --
	( 44.46,107.43) --
	( 44.39,107.30) --
	( 44.34,107.21) --
	( 44.18,107.15) --
	( 44.00,107.16) --
	( 43.84,107.23) --
	( 43.70,107.21) --
	( 43.52,107.19) --
	( 43.44,107.07) --
	( 43.35,107.03) --
	( 43.20,106.98) --
	( 42.85,107.06) --
	( 42.65,107.11) --
	( 42.52,107.07) --
	( 42.45,107.03) --
	( 42.41,106.94) --
	( 42.36,106.93) --
	( 42.30,106.88) --
	( 42.23,106.77) --
	( 42.08,106.53) --
	( 41.95,106.43) --
	( 41.87,106.40);

\draw[color=drawColor,line cap=round,line join=round,fill opacity=0.00,] ( 41.87,106.40) --
	( 41.74,106.37) --
	( 41.57,106.42) --
	( 41.40,106.47) --
	( 41.23,106.45) --
	( 40.99,106.17) --
	( 40.68,106.00) --
	( 39.88,105.86);

\draw[color=drawColor,line cap=round,line join=round,fill opacity=0.00,] ( 39.88,105.86) --
	( 39.68,105.82) --
	( 39.51,105.85) --
	( 39.30,105.90) --
	( 39.17,105.92) --
	( 39.06,105.97) --
	( 38.67,105.90) --
	( 38.54,105.82) --
	( 38.36,105.68) --
	( 38.10,105.41) --
	( 37.86,105.20) --
	( 37.57,105.11) --
	( 37.34,105.02) --
	( 37.23,104.98);

\draw[color=drawColor,line cap=round,line join=round,fill opacity=0.00,] ( 37.23,104.98) --
	( 37.08,104.91);

\draw[color=drawColor,line cap=round,line join=round,fill opacity=0.00,] ( 68.03,113.13) --
	( 67.86,112.97) --
	( 67.62,112.71) --
	( 67.58,112.68) --
	( 67.52,112.62) --
	( 67.46,112.55) --
	( 67.46,112.39) --
	( 67.76,112.27) --
	( 67.72,111.99) --
	( 67.86,111.67) --
	( 67.94,111.26) --
	( 67.93,111.16) --
	( 67.93,111.09) --
	( 67.87,111.07) --
	( 67.77,110.99) --
	( 67.65,110.83) --
	( 67.56,110.72) --
	( 67.49,110.72) --
	( 67.38,110.72) --
	( 67.26,110.70) --
	( 67.18,110.71) --
	( 67.12,110.67) --
	( 67.09,110.62) --
	( 67.15,110.50) --
	( 67.15,110.35) --
	( 67.19,110.21) --
	( 67.16,110.14) --
	( 67.05,110.07) --
	( 66.90,109.99) --
	( 66.84,109.93) --
	( 66.88,109.84) --
	( 66.94,109.81) --
	( 67.02,109.78) --
	( 67.06,109.76) --
	( 67.16,109.75) --
	( 67.28,109.76) --
	( 67.45,109.77) --
	( 67.54,109.78) --
	( 67.59,109.77) --
	( 67.61,109.77) --
	( 67.63,109.72) --
	( 67.64,109.64) --
	( 67.59,109.57) --
	( 67.56,109.55) --
	( 67.50,109.48) --
	( 67.36,109.43) --
	( 67.22,109.43) --
	( 66.92,109.41) --
	( 66.85,109.40) --
	( 66.74,109.36) --
	( 66.68,109.29) --
	( 66.68,109.18) --
	( 66.73,109.03) --
	( 66.88,108.86) --
	( 67.11,108.71) --
	( 67.29,108.55) --
	( 67.33,108.50) --
	( 67.34,108.41) --
	( 67.36,108.35) --
	( 67.37,108.25) --
	( 67.34,108.12) --
	( 67.27,107.94) --
	( 67.27,107.86) --
	( 67.27,107.83) --
	( 67.24,107.77) --
	( 67.27,107.74) --
	( 67.29,107.72) --
	( 67.33,107.70) --
	( 67.39,107.68) --
	( 67.43,107.66) --
	( 67.46,107.58) --
	( 67.44,107.43) --
	( 67.44,107.39) --
	( 67.54,107.33) --
	( 67.64,107.29) --
	( 67.69,107.24) --
	( 67.74,107.19) --
	( 67.76,107.02) --
	( 67.83,106.92) --
	( 67.85,106.91) --
	( 67.95,106.83) --
	( 68.12,106.74) --
	( 68.21,106.62) --
	( 68.31,106.47) --
	( 68.35,106.34) --
	( 68.41,106.22) --
	( 68.37,106.18) --
	( 68.33,106.11) --
	( 68.27,106.03) --
	( 68.25,105.97) --
	( 68.23,105.87) --
	( 68.28,105.82) --
	( 68.32,105.77) --
	( 68.38,105.72) --
	( 68.51,105.66) --
	( 68.59,105.61) --
	( 68.63,105.56) --
	( 68.63,105.47) --
	( 68.65,105.40) --
	( 68.69,105.30) --
	( 68.74,105.22) --
	( 68.84,105.13) --
	( 68.87,105.05) --
	( 68.91,104.94) --
	( 68.95,104.85) --
	( 68.94,104.84);

\draw[color=drawColor,line cap=round,line join=round,fill opacity=0.00,] ( 37.08,104.91) --
	( 36.70,104.75) --
	( 36.64,104.70) --
	( 36.55,104.62) --
	( 36.45,104.54) --
	( 36.38,104.49) --
	( 36.33,104.42) --
	( 36.16,104.26) --
	( 35.98,104.19) --
	( 35.95,104.18) --
	( 35.78,104.31) --
	( 35.40,104.34) --
	( 35.04,104.15) --
	( 34.98,104.09) --
	( 34.83,103.88) --
	( 34.62,103.78) --
	( 34.50,103.68) --
	( 34.47,103.62);

\draw[color=drawColor,line cap=round,line join=round,fill opacity=0.00,] ( 48.43,109.73) --
	( 48.67,109.75) --
	( 48.85,109.77) --
	( 49.07,109.88) --
	( 49.33,109.92) --
	( 49.55,109.85) --
	( 49.70,109.84) --
	( 49.99,109.90) --
	( 50.20,109.92) --
	( 50.33,109.96) --
	( 50.46,109.94) --
	( 50.63,109.90) --
	( 50.77,109.85) --
	( 50.92,109.78) --
	( 51.05,109.70) --
	( 51.39,109.66) --
	( 51.70,109.56) --
	( 51.87,109.45) --
	( 52.08,109.37) --
	( 52.35,109.21) --
	( 52.49,109.05) --
	( 52.55,108.98) --
	( 52.61,108.95) --
	( 52.63,108.77) --
	( 52.66,108.58) --
	( 52.64,108.39) --
	( 52.34,108.16) --
	( 52.22,108.02) --
	( 52.12,107.88) --
	( 51.96,107.76) --
	( 51.80,107.66) --
	( 51.64,107.56) --
	( 51.59,107.50) --
	( 51.56,107.44) --
	( 51.56,107.38) --
	( 51.57,107.28) --
	( 51.56,107.20) --
	( 51.50,107.14) --
	( 51.41,107.05) --
	( 51.20,106.95) --
	( 51.03,106.85) --
	( 50.87,106.71) --
	( 50.80,106.61) --
	( 50.83,106.49) --
	( 50.90,106.40) --
	( 51.00,106.32) --
	( 51.19,106.23) --
	( 51.37,106.16) --
	( 51.61,106.10) --
	( 51.88,106.06) --
	( 52.17,105.96) --
	( 52.48,105.88) --
	( 52.60,105.81) --
	( 52.61,105.75) --
	( 52.64,105.59) --
	( 52.63,105.41) --
	( 52.59,105.18) --
	( 52.60,105.14) --
	( 52.64,105.09) --
	( 52.73,105.05) --
	( 52.81,105.02) --
	( 52.92,105.04) --
	( 53.06,105.10) --
	( 53.18,105.21) --
	( 53.32,105.28) --
	( 53.37,105.41) --
	( 53.41,105.50) --
	( 53.48,105.61) --
	( 53.50,105.73) --
	( 53.57,105.86) --
	( 53.61,105.93) --
	( 53.82,106.04) --
	( 53.99,106.13) --
	( 54.28,106.15) --
	( 54.54,106.15) --
	( 54.76,106.11) --
	( 54.99,106.06) --
	( 55.13,106.04) --
	( 55.31,105.95) --
	( 55.38,105.94) --
	( 55.42,105.92) --
	( 55.48,105.86) --
	( 55.48,105.75) --
	( 55.51,105.58) --
	( 55.64,105.35) --
	( 55.81,105.14) --
	( 56.00,104.94) --
	( 56.16,104.76) --
	( 56.22,104.69) --
	( 56.20,104.57) --
	( 56.20,104.41) --
	( 56.20,104.23) --
	( 56.25,103.99) --
	( 56.36,103.89) --
	( 56.53,103.85) --
	( 56.84,103.78) --
	( 57.08,103.79) --
	( 57.30,103.76) --
	( 57.39,103.75) --
	( 57.46,103.75) --
	( 57.53,103.70) --
	( 57.67,103.59) --
	( 57.81,103.55) --
	( 57.89,103.49) --
	( 57.94,103.46) --
	( 57.94,103.40) --
	( 57.99,103.31) --
	( 58.03,103.18) --
	( 58.08,103.06) --
	( 58.16,102.89) --
	( 58.33,102.77) --
	( 58.50,102.69) --
	( 58.77,102.61) --
	( 59.12,102.59) --
	( 59.40,102.55) --
	( 59.60,102.58) --
	( 59.79,102.64) --
	( 59.93,102.68) --
	( 60.06,102.73) --
	( 60.20,102.73) --
	( 60.30,102.70) --
	( 60.35,102.66) --
	( 60.54,102.54) --
	( 60.71,102.42) --
	( 60.79,102.40) --
	( 60.86,102.41) --
	( 61.03,102.47) --
	( 61.20,102.56) --
	( 61.29,102.61) --
	( 61.39,102.62) --
	( 61.48,102.64);

\draw[color=drawColor,line cap=round,line join=round,fill opacity=0.00,] ( 34.47,103.62) --
	( 34.43,103.47) --
	( 34.50,103.27) --
	( 34.55,103.03) --
	( 34.53,102.96) --
	( 34.48,102.93) --
	( 34.40,102.87) --
	( 34.34,102.85) --
	( 34.19,102.72) --
	( 33.94,102.58) --
	( 33.92,102.54) --
	( 33.87,102.46) --
	( 33.83,102.44) --
	( 33.80,102.42) --
	( 33.77,102.40) --
	( 33.58,102.35) --
	( 33.32,102.38) --
	( 33.00,102.32) --
	( 32.42,102.41) --
	( 32.17,102.48) --
	( 32.01,102.44) --
	( 31.86,102.46) --
	( 31.73,102.51) --
	( 31.68,102.54) --
	( 31.63,102.53) --
	( 31.57,102.43) --
	( 31.29,102.13);

\draw[color=drawColor,line cap=round,line join=round,fill opacity=0.00,] ( 61.48,102.64) --
	( 61.52,102.62) --
	( 61.58,102.59) --
	( 61.65,102.56) --
	( 61.77,102.58) --
	( 61.97,102.58) --
	( 62.01,102.58) --
	( 62.18,102.54) --
	( 62.28,102.49) --
	( 62.38,102.39) --
	( 62.49,102.29) --
	( 62.59,102.27) --
	( 62.77,102.22) --
	( 62.94,102.19) --
	( 62.99,102.17) --
	( 63.09,102.12) --
	( 63.15,102.02) --
	( 63.28,101.95) --
	( 63.36,101.85) --
	( 63.46,101.83) --
	( 63.51,101.82) --
	( 63.51,101.79) --
	( 63.63,101.75) --
	( 63.65,101.71) --
	( 63.72,101.66) --
	( 63.85,101.65) --
	( 64.01,101.60) --
	( 64.08,101.62) --
	( 64.29,101.56) --
	( 64.66,101.46) --
	( 65.07,101.34) --
	( 65.34,101.25) --
	( 65.45,101.22) --
	( 65.53,101.22) --
	( 65.66,101.27) --
	( 65.77,101.30) --
	( 65.89,101.34) --
	( 65.97,101.34) --
	( 66.08,101.34) --
	( 66.19,101.33) --
	( 66.25,101.36) --
	( 66.36,101.43) --
	( 66.43,101.48) --
	( 66.56,101.66) --
	( 66.63,101.76) --
	( 66.67,101.81) --
	( 66.80,101.83) --
	( 66.90,101.86) --
	( 66.98,101.86) --
	( 67.02,101.87) --
	( 67.04,101.90) --
	( 67.12,102.01) --
	( 67.14,102.11) --
	( 67.16,102.24) --
	( 67.23,102.29) --
	( 67.33,102.30) --
	( 67.46,102.34) --
	( 67.52,102.37) --
	( 67.61,102.43) --
	( 67.70,102.48) --
	( 67.81,102.49) --
	( 67.94,102.49) --
	( 68.02,102.52) --
	( 68.11,102.58) --
	( 68.23,102.62) --
	( 68.29,102.67) --
	( 68.32,102.66) --
	( 68.37,102.73) --
	( 68.44,102.95) --
	( 68.47,103.16) --
	( 68.50,103.30) --
	( 68.57,103.37) --
	( 68.61,103.42) --
	( 68.60,103.53) --
	( 68.57,103.62) --
	( 68.53,103.67) --
	( 68.54,103.75) --
	( 68.61,103.79) --
	( 68.69,103.90) --
	( 68.71,103.98) --
	( 68.73,104.08) --
	( 68.70,104.20) --
	( 68.67,104.38) --
	( 68.64,104.43) --
	( 68.64,104.47) --
	( 68.71,104.56) --
	( 68.84,104.63) --
	( 68.87,104.69) --
	( 68.94,104.74) --
	( 68.94,104.84);

\draw[color=drawColor,line cap=round,line join=round,fill opacity=0.00,] ( 68.94,104.84) --
	( 69.15,104.82) --
	( 69.22,104.79) --
	( 69.35,104.76) --
	( 69.40,104.75) --
	( 69.49,104.76) --
	( 69.55,104.81) --
	( 69.60,104.80) --
	( 69.67,104.82) --
	( 69.74,104.82) --
	( 69.83,104.86) --
	( 69.85,104.91) --
	( 69.91,104.94) --
	( 70.01,104.93) --
	( 70.07,104.95) --
	( 70.11,104.92) --
	( 70.18,104.91) --
	( 70.23,104.82) --
	( 70.32,104.74) --
	( 70.39,104.69) --
	( 70.52,104.67) --
	( 70.89,104.63) --
	( 71.05,104.61) --
	( 71.12,104.57) --
	( 71.18,104.53) --
	( 71.22,104.46) --
	( 71.23,104.41) --
	( 71.22,104.20) --
	( 71.19,103.82) --
	( 71.21,103.57) --
	( 71.24,103.44) --
	( 71.28,103.38) --
	( 71.45,103.25) --
	( 71.52,103.17) --
	( 71.53,103.09) --
	( 71.53,102.97) --
	( 71.56,102.90) --
	( 71.60,102.85) --
	( 71.66,102.76) --
	( 71.69,102.73) --
	( 71.66,102.63) --
	( 71.59,102.57) --
	( 71.53,102.49) --
	( 71.57,102.43) --
	( 71.67,102.38) --
	( 71.74,102.37) --
	( 71.80,102.35) --
	( 71.83,102.29) --
	( 71.88,102.25) --
	( 71.91,102.23) --
	( 71.97,102.23) --
	( 72.01,102.22) --
	( 72.12,102.19) --
	( 72.18,102.17) --
	( 72.22,102.09) --
	( 72.26,102.04) --
	( 72.36,102.02) --
	( 72.53,101.96) --
	( 72.47,101.84) --
	( 72.48,101.68) --
	( 72.48,101.58) --
	( 72.41,101.48) --
	( 72.31,101.42) --
	( 72.26,101.37) --
	( 72.22,101.34) --
	( 72.16,101.37) --
	( 72.03,101.36) --
	( 71.98,101.38) --
	( 71.89,101.36) --
	( 71.85,101.37) --
	( 71.81,101.29) --
	( 71.86,101.22) --
	( 71.91,101.10) --
	( 71.86,100.98) --
	( 71.88,100.90) --
	( 71.86,100.80) --
	( 71.97,100.76) --
	( 71.97,100.71) --
	( 71.95,100.62) --
	( 71.92,100.57) --
	( 71.93,100.54) --
	( 71.93,100.48) --
	( 71.96,100.42) --
	( 72.03,100.35) --
	( 72.07,100.32) --
	( 72.13,100.26) --
	( 72.07,100.25) --
	( 71.97,100.21) --
	( 71.94,100.12) --
	( 71.93,100.04) --
	( 71.94, 99.98) --
	( 72.00, 99.92) --
	( 72.05, 99.88) --
	( 72.07, 99.86) --
	( 72.01, 99.85) --
	( 71.89, 99.81) --
	( 71.83, 99.75) --
	( 71.79, 99.68) --
	( 71.83, 99.66) --
	( 71.87, 99.53) --
	( 71.86, 99.47) --
	( 71.84, 99.38) --
	( 71.81, 99.31) --
	( 71.81, 99.26) --
	( 71.81, 99.21) --
	( 71.87, 99.21) --
	( 72.12, 99.34) --
	( 72.35, 99.33) --
	( 72.40, 99.33) --
	( 72.47, 99.26) --
	( 72.54, 99.21) --
	( 72.61, 99.21) --
	( 72.65, 99.22) --
	( 72.74, 99.22) --
	( 72.79, 99.17) --
	( 72.79, 99.08) --
	( 72.64, 99.00) --
	( 72.54, 98.91) --
	( 72.54, 98.90) --
	( 72.47, 98.83) --
	( 72.44, 98.79) --
	( 72.48, 98.75) --
	( 72.57, 98.76) --
	( 72.69, 98.80) --
	( 72.74, 98.81) --
	( 72.78, 98.83) --
	( 72.95, 98.76) --
	( 73.02, 98.73) --
	( 73.11, 98.72) --
	( 73.25, 98.72) --
	( 73.34, 98.71) --
	( 73.38, 98.71) --
	( 73.56, 98.66) --
	( 73.76, 98.49) --
	( 73.84, 98.43) --
	( 73.88, 98.39) --
	( 73.98, 98.38) --
	( 74.04, 98.40) --
	( 74.16, 98.38) --
	( 74.16, 98.36) --
	( 74.22, 98.37) --
	( 74.29, 98.38) --
	( 74.33, 98.43) --
	( 74.46, 98.51) --
	( 74.58, 98.44) --
	( 74.62, 98.42) --
	( 74.67, 98.40) --
	( 74.75, 98.34) --
	( 74.77, 98.30) --
	( 74.79, 98.24) --
	( 75.02, 98.24) --
	( 75.07, 98.21) --
	( 75.10, 98.18) --
	( 75.21, 98.13) --
	( 75.35, 98.14) --
	( 75.44, 98.15) --
	( 75.52, 98.15) --
	( 75.56, 98.17) --
	( 75.59, 98.25) --
	( 75.68, 98.25) --
	( 75.76, 98.20) --
	( 75.83, 98.19) --
	( 75.89, 98.15) --
	( 76.03, 98.19) --
	( 76.10, 98.16) --
	( 76.17, 98.17) --
	( 76.26, 98.16) --
	( 76.35, 98.13) --
	( 76.38, 98.07) --
	( 76.43, 98.00) --
	( 76.54, 97.98) --
	( 76.60, 97.94) --
	( 76.67, 97.90) --
	( 76.71, 97.86) --
	( 76.86, 97.67) --
	( 77.31, 97.60) --
	( 77.52, 97.32);

\draw[color=drawColor,line cap=round,line join=round,fill opacity=0.00,] ( 77.52, 97.32) --
	( 77.52, 97.27) --
	( 77.67, 97.12) --
	( 77.80, 97.14);

\draw[color=drawColor,line cap=round,line join=round,fill opacity=0.00,] ( 31.29,102.13) --
	( 31.28,102.12) --
	( 31.24,102.04) --
	( 31.15,101.85) --
	( 30.71,101.41) --
	( 30.67,101.34) --
	( 30.63,101.26) --
	( 30.61,101.15) --
	( 30.61,101.04) --
	( 30.57,100.94) --
	( 30.71,100.74) --
	( 30.78,100.55) --
	( 30.87,100.37) --
	( 30.97,100.23) --
	( 30.91,100.07) --
	( 30.89, 99.98) --
	( 30.83, 99.84) --
	( 30.84, 99.74) --
	( 30.98, 99.60) --
	( 31.24, 99.47) --
	( 31.41, 99.39) --
	( 31.45, 99.29) --
	( 31.44, 99.18) --
	( 31.23, 98.92) --
	( 31.25, 98.81) --
	( 31.30, 98.53) --
	( 31.29, 98.37) --
	( 31.22, 98.24) --
	( 31.13, 98.17) --
	( 31.11, 98.08) --
	( 31.12, 98.01) --
	( 31.35, 97.69) --
	( 31.48, 97.57) --
	( 31.60, 97.47) --
	( 31.61, 96.82) --
	( 31.59, 96.67) --
	( 31.57, 96.62) --
	( 31.55, 96.57) --
	( 31.42, 96.43) --
	( 31.34, 96.28) --
	( 31.22, 95.98) --
	( 31.19, 95.87) --
	( 31.07, 95.71) --
	( 31.13, 95.40) --
	( 31.09, 95.28) --
	( 31.01, 94.99) --
	( 30.97, 94.94) --
	( 30.94, 94.82) --
	( 30.75, 94.63) --
	( 30.67, 94.55) --
	( 30.58, 94.48) --
	( 30.49, 94.41) --
	( 30.40, 94.38) --
	( 30.29, 94.31) --
	( 30.21, 94.22) --
	( 30.03, 94.07) --
	( 30.01, 94.01) --
	( 29.88, 94.02) --
	( 29.61, 94.04) --
	( 28.72, 93.99) --
	( 28.37, 94.08) --
	( 27.99, 94.26) --
	( 27.76, 94.28) --
	( 27.60, 94.28) --
	( 27.29, 94.33) --
	( 27.02, 94.35) --
	( 26.88, 94.34) --
	( 26.80, 94.36) --
	( 26.51, 94.57) --
	( 26.33, 94.74) --
	( 26.13, 94.92) --
	( 26.09, 94.94) --
	( 25.89, 95.14) --
	( 25.82, 95.15) --
	( 25.65, 95.20) --
	( 25.55, 95.25) --
	( 25.46, 95.34) --
	( 25.35, 95.49) --
	( 25.10, 95.83) --
	( 25.04, 95.90) --
	( 24.92, 95.99) --
	( 24.72, 96.09) --
	( 24.59, 96.16) --
	( 24.52, 96.26) --
	( 24.39, 96.34) --
	( 24.35, 96.37) --
	( 24.28, 96.35) --
	( 24.24, 96.19) --
	( 24.16, 95.99) --
	( 24.09, 95.84) --
	( 24.01, 95.80) --
	( 24.00, 95.77) --
	( 23.91, 95.75);

\draw[color=drawColor,line cap=round,line join=round,fill opacity=0.00,] ( 77.80, 97.14) --
	( 77.85, 97.18) --
	( 78.45, 96.96) --
	( 78.59, 96.89) --
	( 78.73, 96.70) --
	( 78.87, 96.69) --
	( 79.04, 96.70) --
	( 79.15, 96.61) --
	( 79.25, 96.51) --
	( 79.33, 96.39) --
	( 79.37, 96.27) --
	( 79.44, 96.21) --
	( 79.54, 96.13) --
	( 79.78, 96.05) --
	( 79.83, 95.98) --
	( 79.90, 95.96) --
	( 79.94, 95.87) --
	( 79.97, 95.78) --
	( 80.03, 95.63) --
	( 80.02, 95.50) --
	( 80.04, 95.42) --
	( 80.12, 95.33) --
	( 80.27, 95.21) --
	( 80.40, 95.09) --
	( 80.56, 94.90) --
	( 80.63, 94.72) --
	( 80.77, 94.42) --
	( 80.77, 94.30) --
	( 80.77, 94.20) --
	( 80.78, 94.11) --
	( 80.82, 94.03) --
	( 80.81, 93.92) --
	( 80.82, 93.80) --
	( 80.78, 93.75) --
	( 80.80, 93.66) --
	( 80.80, 93.56) --
	( 80.82, 93.46) --
	( 80.82, 93.37) --
	( 80.87, 93.29) --
	( 80.91, 93.23) --
	( 80.97, 93.16) --
	( 81.00, 93.10) --
	( 81.01, 92.99) --
	( 80.98, 92.81) --
	( 80.92, 92.66) --
	( 80.87, 92.57) --
	( 80.75, 92.49) --
	( 80.71, 92.49) --
	( 80.67, 92.46) --
	( 80.58, 92.34) --
	( 80.55, 92.25) --
	( 80.62, 92.19) --
	( 80.65, 92.15) --
	( 80.68, 92.02) --
	( 80.59, 91.97) --
	( 80.55, 91.86) --
	( 80.47, 91.72) --
	( 80.45, 91.65) --
	( 80.46, 91.56) --
	( 80.49, 91.47) --
	( 80.56, 91.41) --
	( 80.67, 91.29) --
	( 80.77, 91.15) --
	( 80.87, 91.06) --
	( 80.85, 90.95) --
	( 80.85, 90.84) --
	( 80.81, 90.82) --
	( 80.81, 90.74) --
	( 80.76, 90.65) --
	( 80.72, 90.55) --
	( 80.78, 90.50) --
	( 80.85, 90.43) --
	( 80.87, 90.39) --
	( 80.90, 90.32) --
	( 80.91, 90.21) --
	( 80.97, 90.10) --
	( 81.05, 90.01) --
	( 81.13, 89.96) --
	( 81.20, 89.91) --
	( 81.43, 89.78) --
	( 81.50, 89.76);

\draw[color=drawColor,line cap=round,line join=round,fill opacity=0.00,] ( 82.77, 87.20) --
	( 82.81, 87.25) --
	( 82.78, 87.38) --
	( 82.77, 87.47) --
	( 82.77, 87.58) --
	( 82.72, 87.64) --
	( 82.59, 87.74) --
	( 82.48, 87.78) --
	( 82.40, 87.90) --
	( 82.27, 88.01) --
	( 82.19, 88.15) --
	( 82.18, 88.32) --
	( 82.18, 88.50) --
	( 82.25, 88.65) --
	( 82.27, 88.77) --
	( 82.28, 88.84) --
	( 82.27, 88.92) --
	( 82.21, 88.98) --
	( 82.16, 89.07) --
	( 82.12, 89.14) --
	( 82.09, 89.21) --
	( 82.05, 89.31) --
	( 81.99, 89.45) --
	( 81.94, 89.54) --
	( 81.81, 89.61) --
	( 81.66, 89.70) --
	( 81.50, 89.76);

\draw[color=drawColor,line cap=round,line join=round,fill opacity=0.00,] ( 82.77, 87.20) --
	( 82.74, 87.00) --
	( 82.78, 86.83) --
	( 82.86, 86.65) --
	( 82.93, 86.49) --
	( 83.06, 86.39) --
	( 83.20, 86.28) --
	( 83.33, 86.20) --
	( 83.44, 86.14) --
	( 83.62, 86.11) --
	( 83.79, 86.11) --
	( 83.90, 86.11) --
	( 84.04, 86.13) --
	( 84.09, 86.08) --
	( 84.13, 86.03) --
	( 84.23, 85.88) --
	( 84.31, 85.73) --
	( 84.38, 85.69) --
	( 84.50, 85.62) --
	( 84.67, 85.59) --
	( 84.75, 85.55) --
	( 84.85, 85.53) --
	( 84.89, 85.47) --
	( 84.99, 85.37) --
	( 85.06, 85.31) --
	( 85.18, 85.21) --
	( 85.38, 85.15) --
	( 85.51, 85.14) --
	( 85.64, 85.18) --
	( 85.74, 85.24) --
	( 85.85, 85.30) --
	( 85.96, 85.33) --
	( 86.10, 85.35) --
	( 86.23, 85.35) --
	( 86.30, 85.28) --
	( 86.47, 85.09) --
	( 86.58, 84.98) --
	( 86.71, 84.91) --
	( 86.82, 84.90) --
	( 86.89, 84.82) --
	( 87.08, 84.68) --
	( 87.18, 84.61) --
	( 87.26, 84.56) --
	( 87.41, 84.51) --
	( 87.50, 84.47) --
	( 87.52, 84.44) --
	( 87.59, 84.41) --
	( 87.58, 84.32) --
	( 87.54, 84.25) --
	( 87.46, 84.14) --
	( 87.42, 84.10) --
	( 87.45, 84.02) --
	( 87.46, 83.97) --
	( 87.50, 83.91) --
	( 87.55, 83.93) --
	( 87.60, 83.89) --
	( 87.68, 83.88) --
	( 87.73, 83.89) --
	( 87.81, 83.81) --
	( 87.89, 83.77) --
	( 87.97, 83.80) --
	( 88.10, 83.78) --
	( 88.24, 83.74) --
	( 88.29, 83.71) --
	( 88.31, 83.65) --
	( 88.32, 83.59) --
	( 88.40, 83.54) --
	( 88.55, 83.47) --
	( 88.65, 83.42) --
	( 88.69, 83.33) --
	( 88.69, 83.23) --
	( 88.76, 83.16) --
	( 88.84, 83.12) --
	( 88.91, 83.04) --
	( 88.94, 82.98) --
	( 89.02, 82.81);

\draw[color=drawColor,line cap=round,line join=round,fill opacity=0.00,] ( 89.02, 82.81) --
	( 89.13, 82.62) --
	( 89.28, 82.45) --
	( 89.44, 82.39) --
	( 89.58, 82.36) --
	( 89.69, 82.33) --
	( 89.97, 82.30) --
	( 90.18, 82.28) --
	( 90.30, 82.29) --
	( 90.35, 82.29) --
	( 90.37, 82.33);

\draw[color=drawColor,line cap=round,line join=round,fill opacity=0.00,] (  0.00, 78.76) --
	(  0.13, 78.78) --
	(  0.23, 78.92) --
	(  0.24, 79.14) --
	(  0.30, 79.32) --
	(  0.35, 79.45) --
	(  0.33, 79.66) --
	(  0.35, 79.87) --
	(  0.34, 79.93) --
	(  0.32, 80.00) --
	(  0.32, 80.00) --
	(  0.32, 80.02) --
	(  0.29, 80.15) --
	(  0.30, 80.23) --
	(  0.30, 80.27) --
	(  0.22, 80.42);

\draw[color=drawColor,line cap=round,line join=round,fill opacity=0.00,] ( 90.37, 82.33) --
	( 90.41, 82.29) --
	( 90.58, 82.25) --
	( 90.72, 82.19) --
	( 90.89, 82.18) --
	( 90.95, 82.18) --
	( 91.01, 82.19) --
	( 91.04, 82.19) --
	( 91.08, 82.22) --
	( 91.09, 82.24) --
	( 91.17, 82.39) --
	( 91.18, 82.44) --
	( 91.25, 82.45) --
	( 91.32, 82.45) --
	( 91.39, 82.39) --
	( 91.46, 82.33) --
	( 91.50, 82.33) --
	( 91.55, 82.29) --
	( 91.61, 82.29) --
	( 91.71, 82.30) --
	( 91.85, 82.30) --
	( 91.93, 82.29) --
	( 91.96, 82.28) --
	( 92.00, 82.24) --
	( 92.00, 82.17) --
	( 91.99, 81.99) --
	( 91.98, 81.94) --
	( 92.01, 81.92) --
	( 92.06, 81.94) --
	( 92.19, 81.98) --
	( 92.26, 82.02) --
	( 92.27, 82.02) --
	( 92.33, 82.03) --
	( 92.43, 82.04) --
	( 92.52, 82.08) --
	( 92.58, 82.09) --
	( 92.62, 82.13) --
	( 92.72, 82.14) --
	( 92.73, 82.09) --
	( 92.77, 81.93) --
	( 92.80, 81.86) --
	( 92.85, 81.88) --
	( 92.91, 81.87) --
	( 92.98, 81.86) --
	( 93.02, 81.86) --
	( 93.03, 81.88) --
	( 93.09, 81.94) --
	( 93.16, 82.03) --
	( 93.22, 82.04) --
	( 93.26, 82.06) --
	( 93.29, 82.00) --
	( 93.41, 81.89) --
	( 93.55, 81.78) --
	( 93.66, 81.69) --
	( 93.84, 81.65) --
	( 93.92, 81.65) --
	( 93.97, 81.60) --
	( 94.04, 81.57) --
	( 94.06, 81.52) --
	( 94.04, 81.46) --
	( 94.09, 81.30) --
	( 94.16, 81.21) --
	( 94.23, 81.10) --
	( 94.30, 81.06) --
	( 94.35, 81.00) --
	( 94.37, 80.95) --
	( 94.34, 80.88) --
	( 94.36, 80.79) --
	( 94.35, 80.76) --
	( 94.43, 80.73) --
	( 94.51, 80.66) --
	( 94.53, 80.56) --
	( 94.59, 80.37) --
	( 94.63, 80.18) --
	( 94.66, 80.12) --
	( 94.74, 80.09) --
	( 94.82, 80.08) --
	( 94.89, 80.08) --
	( 95.00, 80.10) --
	( 95.10, 80.11) --
	( 95.24, 80.11) --
	( 95.45, 80.10) --
	( 95.67, 80.09) --
	( 95.84, 80.11) --
	( 95.94, 80.12) --
	( 96.02, 80.14) --
	( 96.10, 80.11) --
	( 96.14, 80.06) --
	( 96.20, 79.97) --
	( 96.21, 79.92) --
	( 96.29, 79.87) --
	( 96.36, 79.89) --
	( 96.46, 79.91) --
	( 96.51, 79.93) --
	( 96.60, 79.96) --
	( 96.70, 79.98) --
	( 96.70, 79.93) --
	( 96.71, 79.87) --
	( 96.68, 79.81) --
	( 96.66, 79.64) --
	( 96.63, 79.51) --
	( 96.62, 79.31) --
	( 96.63, 79.21) --
	( 96.68, 79.14) --
	( 96.79, 79.06) --
	( 96.85, 79.03) --
	( 96.83, 78.97) --
	( 96.86, 78.89) --
	( 96.91, 78.81) --
	( 96.97, 78.74) --
	( 97.05, 78.62) --
	( 97.16, 78.54) --
	( 97.21, 78.50) --
	( 97.18, 78.39) --
	( 97.14, 78.35) --
	( 97.10, 78.26) --
	( 97.01, 78.24) --
	( 96.86, 78.15) --
	( 96.82, 78.06) --
	( 96.84, 78.02) --
	( 96.87, 77.94) --
	( 96.72, 77.61) --
	( 96.66, 77.30) --
	( 96.87, 76.56) --
	( 96.86, 76.50) --
	( 96.90, 76.44) --
	( 96.74, 76.20) --
	( 97.05, 75.91) --
	( 97.12, 75.75) --
	( 97.26, 75.63) --
	( 97.27, 75.46);

\draw[color=drawColor,line cap=round,line join=round,fill opacity=0.00,] ( 97.43, 75.09) --
	( 97.27, 75.46);

\draw[color=drawColor,line cap=round,line join=round,fill opacity=0.00,] ( 97.43, 75.09) --
	( 97.63, 75.01) --
	( 97.73, 74.93) --
	( 97.83, 74.97) --
	( 97.89, 75.13) --
	( 98.09, 75.18);

\draw[color=drawColor,line cap=round,line join=round,fill opacity=0.00,] ( 98.09, 75.18) --
	( 98.25, 75.12) --
	( 98.27, 75.10) --
	( 98.33, 75.03) --
	( 98.48, 75.03) --
	( 98.56, 74.98) --
	( 98.81, 74.97) --
	( 99.21, 74.93) --
	( 99.29, 74.88) --
	( 99.36, 74.86) --
	( 99.42, 74.82) --
	( 99.44, 74.75) --
	( 99.50, 74.64) --
	( 99.55, 74.59) --
	( 99.63, 74.51) --
	( 99.67, 74.50) --
	( 99.74, 74.43) --
	( 99.78, 74.37) --
	( 99.86, 74.33) --
	( 99.99, 74.29) --
	(100.12, 74.22) --
	(100.39, 74.20) --
	(100.46, 74.19) --
	(100.51, 74.19);

\draw[color=drawColor,line cap=round,line join=round,fill opacity=0.00,] (100.51, 74.19) --
	(100.65, 74.15) --
	(100.75, 74.16) --
	(100.90, 74.19) --
	(101.02, 74.22) --
	(101.22, 74.29) --
	(101.41, 74.35) --
	(101.47, 74.34) --
	(102.11, 74.42);

\draw[color=drawColor,line cap=round,line join=round,fill opacity=0.00,] (108.01, 72.85) --
	(108.08, 72.86) --
	(108.23, 72.86) --
	(108.28, 72.86);

\draw[color=drawColor,line cap=round,line join=round,fill opacity=0.00,] (102.11, 74.42) --
	(102.26, 74.37) --
	(102.41, 74.34) --
	(102.98, 74.42) --
	(103.67, 74.35) --
	(104.36, 74.12) --
	(104.51, 74.14) --
	(104.63, 74.11) --
	(104.72, 74.09) --
	(104.82, 74.01) --
	(104.89, 73.97) --
	(104.95, 73.93) --
	(105.06, 73.93) --
	(105.17, 73.93) --
	(105.26, 73.90) --
	(105.33, 73.86) --
	(105.39, 73.83) --
	(105.43, 73.78) --
	(105.45, 73.71) --
	(105.58, 73.66) --
	(105.66, 73.68) --
	(105.73, 73.68) --
	(105.76, 73.64) --
	(105.92, 73.41) --
	(106.03, 73.35) --
	(106.18, 73.29) --
	(106.29, 73.28) --
	(106.37, 73.28) --
	(106.45, 73.22) --
	(106.52, 73.17) --
	(106.77, 72.98) --
	(106.87, 72.90) --
	(106.98, 72.88) --
	(107.07, 72.86) --
	(107.21, 72.86) --
	(107.33, 72.89) --
	(107.38, 72.87) --
	(107.44, 72.85) --
	(107.52, 72.79) --
	(107.58, 72.81) --
	(107.63, 72.78) --
	(107.76, 72.80) --
	(108.01, 72.85);

\draw[color=drawColor,line cap=round,line join=round,fill opacity=0.00,] (108.28, 72.86) --
	(108.39, 72.84) --
	(108.50, 72.80) --
	(108.60, 72.76) --
	(108.66, 72.80) --
	(108.81, 72.83) --
	(108.93, 72.84) --
	(109.10, 72.83) --
	(109.25, 72.83) --
	(109.51, 72.80) --
	(109.71, 72.81) --
	(109.86, 72.81) --
	(109.98, 72.82) --
	(110.04, 72.83) --
	(110.11, 72.83) --
	(110.16, 72.79);

\draw[color=drawColor,line cap=round,line join=round,fill opacity=0.00,] (110.16, 72.79) --
	(110.16, 72.69) --
	(110.30, 72.66) --
	(110.36, 72.60) --
	(110.48, 72.62) --
	(110.59, 72.61) --
	(110.64, 72.57) --
	(110.71, 72.52) --
	(110.79, 72.43) --
	(110.91, 72.37) --
	(111.13, 72.32) --
	(111.23, 72.24) --
	(111.27, 72.18) --
	(111.37, 72.03) --
	(111.45, 71.82) --
	(111.50, 71.74) --
	(111.61, 71.66) --
	(111.67, 71.57) --
	(111.69, 71.46) --
	(111.69, 71.37) --
	(111.67, 71.30) --
	(111.68, 71.21) --
	(111.67, 71.19) --
	(111.71, 71.11) --
	(111.78, 71.06) --
	(111.84, 70.92) --
	(111.96, 70.73) --
	(112.03, 70.59) --
	(112.09, 70.44) --
	(112.20, 70.42) --
	(112.27, 70.34) --
	(112.30, 70.25) --
	(112.32, 70.15) --
	(112.28, 70.10) --
	(112.17, 70.01) --
	(112.12, 69.94) --
	(112.04, 69.84) --
	(112.04, 69.82) --
	(112.10, 69.75) --
	(112.07, 69.69) --
	(112.10, 69.66) --
	(112.07, 69.59) --
	(112.06, 69.54) --
	(112.03, 69.49) --
	(112.04, 69.47) --
	(112.11, 69.46) --
	(112.18, 69.43) --
	(112.23, 69.42) --
	(112.25, 69.37) --
	(112.30, 69.26) --
	(112.31, 69.22);

\draw[color=drawColor,line cap=round,line join=round,fill opacity=0.00,] (  0.58, 80.59) --
	(  0.69, 80.51) --
	(  0.69, 80.24) --
	(  0.69, 80.24) --
	(  0.69, 80.15) --
	(  0.78, 79.92) --
	(  0.75, 79.66) --
	(  0.69, 79.27) --
	(  0.56, 78.65) --
	(  0.43, 78.45) --
	(  0.02, 78.30) --
	(  0.00, 78.29);

\draw[color=drawColor,line cap=round,line join=round,fill opacity=0.00,] (  0.00, 71.71) --
	(  0.06, 71.68) --
	(  0.33, 71.33) --
	(  0.56, 70.79) --
	(  0.71, 70.36) --
	(  0.96, 69.91) --
	(  1.05, 69.78) --
	(  1.40, 69.55) --
	(  1.75, 69.30) --
	(  1.93, 69.05) --
	(  2.23, 68.94);

\draw[color=drawColor,line cap=round,line join=round,fill opacity=0.00,] (  2.51, 68.80) --
	(  2.62, 68.66);

\draw[color=drawColor,line cap=round,line join=round,fill opacity=0.00,] (112.31, 69.22) --
	(112.43, 69.13) --
	(112.58, 69.14) --
	(112.72, 69.14) --
	(112.90, 69.14) --
	(113.11, 69.13) --
	(113.18, 69.11) --
	(113.30, 69.12) --
	(113.45, 69.08) --
	(113.59, 69.08) --
	(113.66, 69.07) --
	(113.69, 69.05) --
	(113.70, 68.96) --
	(113.75, 68.82) --
	(113.83, 68.59) --
	(113.92, 68.51) --
	(113.99, 68.48) --
	(114.07, 68.47) --
	(114.18, 68.46) --
	(114.31, 68.44) --
	(114.40, 68.43) --
	(114.47, 68.42) --
	(114.57, 68.43) --
	(114.63, 68.39) --
	(114.66, 68.39) --
	(114.70, 68.37) --
	(114.75, 68.40) --
	(114.87, 68.45) --
	(114.94, 68.49) --
	(115.00, 68.51) --
	(115.15, 68.49) --
	(115.23, 68.48) --
	(115.40, 68.50) --
	(115.51, 68.49) --
	(115.63, 68.48) --
	(115.70, 68.43) --
	(115.74, 68.37) --
	(115.83, 68.36) --
	(115.94, 68.34) --
	(116.00, 68.31) --
	(116.07, 68.26) --
	(116.15, 68.19) --
	(116.23, 68.13) --
	(116.31, 68.10);

\draw[color=drawColor,line cap=round,line join=round,fill opacity=0.00,] (118.87, 66.55) --
	(119.53, 66.81) --
	(119.56, 66.97) --
	(119.63, 66.98) --
	(119.71, 67.01) --
	(119.77, 67.03) --
	(119.85, 67.03) --
	(119.95, 67.03) --
	(120.08, 66.95) --
	(120.23, 66.89) --
	(120.38, 66.84) --
	(120.48, 66.81) --
	(120.55, 66.78) --
	(120.74, 66.82) --
	(120.95, 66.82) --
	(121.07, 66.83) --
	(121.13, 66.83) --
	(121.15, 66.80) --
	(121.27, 66.63);

\draw[color=drawColor,line cap=round,line join=round,fill opacity=0.00,] (116.31, 68.10) --
	(116.38, 67.97) --
	(116.59, 67.77) --
	(116.62, 67.70) --
	(116.63, 67.59) --
	(116.64, 67.51) --
	(116.65, 67.47) --
	(116.70, 67.43) --
	(116.76, 67.38) --
	(116.77, 67.35) --
	(116.81, 67.22) --
	(116.86, 67.14) --
	(116.94, 67.10) --
	(117.04, 67.08) --
	(117.11, 67.06) --
	(117.16, 66.98) --
	(117.23, 66.94) --
	(117.30, 66.91) --
	(117.35, 66.90) --
	(117.38, 66.82) --
	(117.40, 66.78) --
	(117.42, 66.75) --
	(117.43, 66.70) --
	(117.51, 66.68) --
	(117.58, 66.67) --
	(117.65, 66.68) --
	(117.72, 66.69) --
	(117.84, 66.69) --
	(117.91, 66.68) --
	(118.01, 66.65) --
	(118.06, 66.61) --
	(118.17, 66.56) --
	(118.25, 66.51) --
	(118.32, 66.49) --
	(118.37, 66.50) --
	(118.51, 66.52) --
	(118.58, 66.55) --
	(118.65, 66.56) --
	(118.71, 66.58) --
	(118.79, 66.57) --
	(118.87, 66.55);

\draw[color=drawColor,line cap=round,line join=round,fill opacity=0.00,] (  2.62, 68.66) --
	(  2.74, 68.54) --
	(  2.94, 68.51) --
	(  3.16, 68.35) --
	(  3.22, 68.20) --
	(  3.28, 67.84) --
	(  3.64, 67.24) --
	(  3.65, 67.08) --
	(  3.52, 67.00) --
	(  3.74, 66.83) --
	(  3.84, 66.65) --
	(  3.85, 66.60) --
	(  3.94, 66.18) --
	(  3.98, 66.12) --
	(  4.20, 65.97) --
	(  4.31, 65.80) --
	(  4.31, 65.79);

\draw[color=drawColor,line cap=round,line join=round,fill opacity=0.00,] (  0.00, 71.46) --
	(  0.02, 71.45) --
	(  0.05, 71.38) --
	(  0.13, 71.28) --
	(  0.20, 71.08) --
	(  0.27, 70.81) --
	(  0.37, 70.53) --
	(  0.59, 70.10) --
	(  0.66, 70.02) --
	(  1.04, 69.57) --
	(  1.11, 69.51) --
	(  1.18, 69.46) --
	(  1.31, 69.36) --
	(  1.40, 69.30) --
	(  1.52, 69.22) --
	(  1.56, 69.17) --
	(  1.63, 69.07) --
	(  1.74, 68.99) --
	(  1.90, 68.86) --
	(  1.99, 68.83) --
	(  2.28, 68.75) --
	(  2.42, 68.60) --
	(  2.51, 68.52) --
	(  2.65, 68.40) --
	(  2.76, 68.36) --
	(  2.83, 68.31) --
	(  2.85, 68.23) --
	(  2.90, 68.12) --
	(  2.99, 67.89) --
	(  3.18, 67.51) --
	(  3.28, 67.36) --
	(  3.17, 67.19) --
	(  3.20, 67.07) --
	(  3.29, 66.97) --
	(  3.33, 66.92) --
	(  3.42, 66.85) --
	(  3.47, 66.79) --
	(  3.49, 66.68) --
	(  3.53, 66.55) --
	(  3.58, 66.45) --
	(  3.58, 66.32) --
	(  3.62, 66.25) --
	(  3.71, 66.14) --
	(  3.83, 65.96) --
	(  3.87, 65.86) --
	(  3.86, 65.80) --
	(  3.83, 65.76);

\draw[color=drawColor,line cap=round,line join=round,fill opacity=0.00,] (126.60, 65.60) --
	(126.80, 65.71) --
	(126.92, 65.68) --
	(127.03, 65.70);

\draw[color=drawColor,line cap=round,line join=round,fill opacity=0.00,] (121.27, 66.63) --
	(121.36, 66.49) --
	(121.48, 66.37) --
	(121.61, 66.27) --
	(121.73, 66.21) --
	(121.89, 66.15) --
	(122.04, 66.13) --
	(122.15, 66.12) --
	(122.32, 66.11) --
	(123.51, 66.03) --
	(123.69, 65.95) --
	(124.09, 65.49);

\draw[color=drawColor,line cap=round,line join=round,fill opacity=0.00,] (124.09, 65.49) --
	(124.25, 65.40) --
	(125.16, 65.42) --
	(125.55, 65.42) --
	(125.72, 65.62) --
	(125.78, 65.67) --
	(125.84, 65.83) --
	(125.95, 65.80);

\draw[color=drawColor,line cap=round,line join=round,fill opacity=0.00,] (127.03, 65.70) --
	(127.08, 65.64) --
	(127.21, 65.59) --
	(127.44, 65.51) --
	(127.63, 65.37) --
	(127.73, 65.29) --
	(127.76, 65.23) --
	(127.77, 65.14);

\draw[color=drawColor,line cap=round,line join=round,fill opacity=0.00,] (127.77, 65.14) --
	(127.93, 65.07) --
	(128.01, 65.03) --
	(128.05, 64.98) --
	(128.06, 64.90) --
	(128.10, 64.83) --
	(128.16, 64.82) --
	(128.22, 64.76) --
	(128.25, 64.69) --
	(128.26, 64.61) --
	(128.33, 64.56) --
	(128.38, 64.51) --
	(128.46, 64.51) --
	(128.50, 64.47) --
	(128.57, 64.45) --
	(128.57, 64.35) --
	(128.57, 64.25) --
	(128.61, 64.20) --
	(128.63, 64.17) --
	(128.71, 64.14) --
	(128.80, 64.13) --
	(128.89, 64.12) --
	(128.95, 64.12) --
	(129.01, 64.06) --
	(129.13, 63.90) --
	(129.21, 63.80) --
	(129.24, 63.71) --
	(129.21, 63.64) --
	(129.13, 63.59) --
	(129.06, 63.54) --
	(129.03, 63.48) --
	(129.03, 63.42) --
	(129.06, 63.35) --
	(129.08, 63.24) --
	(129.09, 63.15) --
	(129.10, 63.06) --
	(129.06, 62.97) --
	(129.10, 62.94) --
	(129.18, 62.88) --
	(129.20, 62.78) --
	(129.20, 62.70) --
	(129.20, 62.63) --
	(129.20, 62.56);

\draw[color=drawColor,line cap=round,line join=round,fill opacity=0.00,] (129.20, 62.56) --
	(129.11, 62.47) --
	(129.12, 62.42) --
	(129.13, 62.25) --
	(129.13, 62.18) --
	(129.08, 62.10) --
	(129.00, 62.03) --
	(128.93, 61.99) --
	(128.94, 61.95) --
	(128.94, 61.86) --
	(128.88, 61.76) --
	(128.78, 61.68) --
	(128.73, 61.66) --
	(128.66, 61.65) --
	(128.57, 61.66) --
	(128.52, 61.73) --
	(128.37, 61.76) --
	(128.30, 61.82) --
	(128.20, 61.84) --
	(128.14, 61.84) --
	(128.07, 61.79) --
	(128.04, 61.76) --
	(127.98, 61.67) --
	(127.95, 61.64) --
	(127.90, 61.56) --
	(127.89, 61.48) --
	(127.92, 61.40) --
	(127.99, 61.34) --
	(128.02, 61.25) --
	(127.96, 61.11) --
	(127.92, 61.01) --
	(127.92, 60.92) --
	(128.09, 60.80) --
	(128.04, 60.44) --
	(128.07, 60.25) --
	(128.41, 60.10) --
	(128.45, 60.04);

\draw[color=drawColor,line cap=round,line join=round,fill opacity=0.00,] (129.28, 59.00) --
	(129.20, 58.95) --
	(129.03, 58.94) --
	(129.01, 58.81) --
	(128.92, 58.42) --
	(129.05, 58.23);

\draw[color=drawColor,line cap=round,line join=round,fill opacity=0.00,] ( 14.97, 57.07) --
	( 15.11, 57.10) --
	( 15.44, 57.12) --
	( 15.81, 57.09) --
	( 16.02, 57.04);

\draw[color=drawColor,line cap=round,line join=round,fill opacity=0.00,] ( 12.71, 57.97) --
	( 12.82, 57.93) --
	( 12.91, 57.91) --
	( 12.96, 57.83) --
	( 13.01, 57.76) --
	( 13.05, 57.76) --
	( 13.10, 57.68) --
	( 13.17, 57.64) --
	( 13.23, 57.62) --
	( 13.28, 57.61) --
	( 13.35, 57.56) --
	( 13.41, 57.49) --
	( 13.41, 57.39) --
	( 13.50, 57.32) --
	( 13.54, 57.32) --
	( 13.61, 57.33) --
	( 13.64, 57.31) --
	( 13.62, 57.21) --
	( 13.65, 57.18) --
	( 13.72, 57.16) --
	( 13.79, 57.16) --
	( 13.87, 57.22) --
	( 13.91, 57.29) --
	( 13.92, 57.33) --
	( 13.95, 57.35) --
	( 13.99, 57.36) --
	( 14.05, 57.33) --
	( 14.08, 57.29) --
	( 14.08, 57.24) --
	( 14.13, 57.18) --
	( 14.20, 57.12) --
	( 14.30, 57.06) --
	( 14.38, 57.03) --
	( 14.48, 57.04) --
	( 14.58, 57.04) --
	( 14.71, 57.05) --
	( 14.82, 57.04) --
	( 14.97, 57.07);

\draw[color=drawColor,line cap=round,line join=round,fill opacity=0.00,] ( 24.76, 56.75) --
	( 24.63, 56.67) --
	( 24.57, 56.64) --
	( 24.53, 56.61) --
	( 24.49, 56.58) --
	( 24.44, 56.56) --
	( 24.46, 56.51) --
	( 24.40, 56.49) --
	( 24.39, 56.48) --
	( 24.38, 56.48) --
	( 24.34, 56.45) --
	( 24.28, 56.42) --
	( 24.25, 56.46) --
	( 24.12, 56.45) --
	( 24.13, 56.42) --
	( 24.01, 56.36) --
	( 23.94, 56.36) --
	( 23.84, 56.38) --
	( 23.82, 56.38) --
	( 23.78, 56.40) --
	( 23.76, 56.39) --
	( 23.68, 56.33) --
	( 23.65, 56.32) --
	( 23.60, 56.28) --
	( 23.55, 56.30) --
	( 23.48, 56.26) --
	( 23.43, 56.36) --
	( 23.38, 56.37) --
	( 23.34, 56.34) --
	( 23.33, 56.32) --
	( 23.37, 56.27) --
	( 23.34, 56.25) --
	( 23.33, 56.22) --
	( 23.30, 56.19) --
	( 23.27, 56.22) --
	( 23.24, 56.24) --
	( 23.18, 56.20) --
	( 23.14, 56.15) --
	( 23.08, 56.11) --
	( 23.03, 56.09) --
	( 23.01, 56.11) --
	( 22.99, 56.11) --
	( 22.97, 56.11) --
	( 22.91, 56.04) --
	( 22.88, 56.02) --
	( 22.88, 56.00) --
	( 22.85, 55.99) --
	( 22.80, 56.01) --
	( 22.76, 56.01) --
	( 22.72, 56.02) --
	( 22.67, 55.99) --
	( 22.63, 56.00) --
	( 22.56, 56.01) --
	( 22.54, 55.99) --
	( 22.54, 55.97) --
	( 22.50, 55.94) --
	( 22.44, 55.93) --
	( 22.28, 55.93) --
	( 22.21, 55.94) --
	( 22.17, 55.94) --
	( 22.09, 55.94) --
	( 21.99, 55.96) --
	( 21.88, 55.97) --
	( 21.85, 55.98) --
	( 21.81, 55.99) --
	( 21.73, 56.10) --
	( 21.67, 56.09) --
	( 21.66, 56.11) --
	( 21.63, 56.12) --
	( 21.60, 56.14) --
	( 21.57, 56.22) --
	( 21.55, 56.24) --
	( 21.49, 56.23) --
	( 21.48, 56.22) --
	( 21.42, 56.23) --
	( 21.41, 56.24) --
	( 21.41, 56.27) --
	( 21.37, 56.32) --
	( 21.36, 56.34) --
	( 21.36, 56.37) --
	( 21.32, 56.48) --
	( 21.26, 56.58) --
	( 21.21, 56.69) --
	( 21.16, 56.70) --
	( 21.12, 56.71) --
	( 21.06, 56.70) --
	( 20.89, 56.69) --
	( 20.84, 56.67) --
	( 20.80, 56.65) --
	( 20.75, 56.64) --
	( 20.73, 56.66) --
	( 20.69, 56.63) --
	( 20.59, 56.61) --
	( 20.56, 56.59) --
	( 20.50, 56.58) --
	( 20.40, 56.51) --
	( 20.36, 56.51) --
	( 20.29, 56.50) --
	( 20.24, 56.47) --
	( 20.18, 56.45) --
	( 20.14, 56.40) --
	( 20.08, 56.39) --
	( 19.96, 56.41) --
	( 19.90, 56.43) --
	( 19.87, 56.48) --
	( 19.80, 56.49) --
	( 19.72, 56.52) --
	( 19.63, 56.55) --
	( 19.56, 56.56) --
	( 19.48, 56.59) --
	( 19.44, 56.58) --
	( 19.37, 56.54) --
	( 19.33, 56.52) --
	( 19.30, 56.52) --
	( 19.26, 56.47) --
	( 19.22, 56.48) --
	( 19.18, 56.47) --
	( 19.15, 56.50) --
	( 19.14, 56.54) --
	( 19.13, 56.56) --
	( 19.10, 56.57) --
	( 19.07, 56.55) --
	( 19.04, 56.53) --
	( 19.02, 56.50) --
	( 19.00, 56.45) --
	( 19.01, 56.42) --
	( 19.01, 56.39) --
	( 18.98, 56.38) --
	( 18.97, 56.35) --
	( 18.91, 56.33) --
	( 18.88, 56.33) --
	( 18.84, 56.29) --
	( 18.80, 56.29) --
	( 18.79, 56.28) --
	( 18.75, 56.28) --
	( 18.73, 56.28) --
	( 18.68, 56.26) --
	( 18.65, 56.26) --
	( 18.61, 56.26) --
	( 18.60, 56.26) --
	( 18.56, 56.28) --
	( 18.51, 56.31) --
	( 18.50, 56.38) --
	( 18.46, 56.47) --
	( 18.38, 56.59) --
	( 18.33, 56.63) --
	( 18.29, 56.69) --
	( 18.27, 56.75) --
	( 18.27, 56.82) --
	( 18.26, 56.83) --
	( 18.25, 56.91) --
	( 18.23, 56.95) --
	( 18.18, 56.97) --
	( 18.11, 56.99) --
	( 17.99, 57.01) --
	( 17.85, 57.03) --
	( 17.74, 57.05) --
	( 17.65, 57.04) --
	( 17.61, 57.03) --
	( 17.50, 56.98) --
	( 17.36, 56.96) --
	( 17.17, 56.90) --
	( 17.06, 56.88) --
	( 16.98, 56.88) --
	( 16.84, 56.89) --
	( 16.47, 56.91) --
	( 16.35, 56.90) --
	( 16.27, 56.91) --
	( 16.21, 56.96) --
	( 16.08, 57.02) --
	( 16.02, 57.04);

\draw[color=drawColor,line cap=round,line join=round,fill opacity=0.00,] (127.09, 51.68) --
	(127.07, 51.81) --
	(126.82, 52.19) --
	(126.72, 52.24) --
	(126.70, 52.44) --
	(127.20, 52.62) --
	(127.36, 52.69) --
	(127.49, 52.94) --
	(127.58, 52.98) --
	(127.88, 52.94) --
	(127.99, 52.98) --
	(128.12, 53.00) --
	(128.31, 53.05) --
	(128.49, 53.06) --
	(128.72, 53.11) --
	(128.94, 53.11) --
	(129.12, 53.18) --
	(129.21, 53.20) --
	(129.32, 53.23) --
	(129.50, 53.25) --
	(129.67, 53.25) --
	(129.80, 53.26) --
	(129.89, 53.28) --
	(130.01, 53.35) --
	(130.09, 53.43) --
	(130.25, 53.57) --
	(130.42, 53.76) --
	(130.56, 53.91) --
	(130.66, 54.04) --
	(130.69, 54.15) --
	(130.73, 54.23) --
	(130.79, 54.33) --
	(130.81, 54.41) --
	(130.83, 54.52) --
	(130.82, 54.65) --
	(130.75, 54.78) --
	(130.58, 54.95) --
	(130.47, 55.05) --
	(130.49, 55.16) --
	(130.46, 55.30) --
	(130.46, 55.37) --
	(130.45, 55.49) --
	(130.38, 55.56) --
	(130.44, 55.62) --
	(130.53, 55.71) --
	(130.53, 55.81) --
	(130.44, 55.90) --
	(130.44, 55.97) --
	(130.46, 56.07) --
	(130.42, 56.12) --
	(130.35, 56.19) --
	(130.30, 56.22) --
	(130.20, 56.27) --
	(130.11, 56.31) --
	(129.99, 56.56) --
	(129.90, 56.73) --
	(129.78, 56.86) --
	(129.69, 56.90) --
	(129.54, 56.98) --
	(129.34, 57.05) --
	(129.24, 57.08) --
	(129.19, 57.15) --
	(129.16, 57.21) --
	(129.17, 57.26) --
	(129.21, 57.40) --
	(129.30, 57.65) --
	(129.21, 58.01) --
	(129.05, 58.23);

\draw[color=drawColor,line cap=round,line join=round,fill opacity=0.00,] ( 11.30, 58.11) --
	( 11.34, 57.89) --
	( 11.34, 57.84);

\draw[color=drawColor,line cap=round,line join=round,fill opacity=0.00,] ( 11.34, 57.84) --
	( 11.35, 57.64) --
	( 11.42, 57.47) --
	( 11.46, 57.31) --
	( 11.43, 57.17) --
	( 11.28, 57.11) --
	( 11.06, 57.08) --
	( 10.82, 57.09) --
	( 10.71, 57.02) --
	( 10.67, 56.97) --
	( 10.53, 56.87) --
	( 10.43, 56.76) --
	( 10.30, 56.54) --
	(  9.93, 56.29) --
	(  9.49, 55.87) --
	(  9.18, 55.72) --
	(  9.16, 55.44) --
	(  9.08, 55.31) --
	(  8.85, 55.22) --
	(  8.38, 54.92) --
	(  7.99, 54.87) --
	(  7.71, 54.89) --
	(  7.63, 54.87) --
	(  7.52, 54.78) --
	(  7.38, 54.63) --
	(  7.10, 54.33) --
	(  6.81, 54.08) --
	(  6.50, 53.94) --
	(  5.98, 53.90) --
	(  5.79, 53.65) --
	(  5.63, 53.57) --
	(  5.25, 53.47) --
	(  5.05, 53.36) --
	(  4.64, 53.37) --
	(  4.45, 53.31) --
	(  4.13, 52.93) --
	(  3.99, 52.56) --
	(  3.91, 52.17) --
	(  4.02, 51.61) --
	(  3.90, 51.22) --
	(  3.68, 50.99) --
	(  3.48, 50.98) --
	(  3.30, 51.02) --
	(  3.21, 51.03) --
	(  3.21, 50.97) --
	(  3.39, 50.86) --
	(  3.45, 50.70) --
	(  3.45, 50.56) --
	(  3.36, 50.41) --
	(  3.16, 50.43) --
	(  3.09, 50.34) --
	(  2.76, 50.13) --
	(  2.71, 49.97) --
	(  2.44, 49.77) --
	(  2.08, 49.59) --
	(  1.84, 49.55) --
	(  1.75, 49.47) --
	(  1.56, 49.17) --
	(  1.59, 49.09) --
	(  1.67, 49.01) --
	(  1.78, 48.94) --
	(  2.04, 48.73) --
	(  2.19, 48.54) --
	(  2.18, 48.32) --
	(  2.04, 48.27) --
	(  1.87, 48.30) --
	(  1.73, 48.37) --
	(  1.72, 48.30) --
	(  1.88, 48.14) --
	(  2.20, 47.92) --
	(  2.30, 47.73) --
	(  2.38, 47.41) --
	(  2.22, 47.25) --
	(  2.02, 47.12) --
	(  1.84, 47.08) --
	(  1.64, 47.11) --
	(  1.62, 47.10) --
	(  1.65, 47.00) --
	(  1.86, 46.59) --
	(  1.92, 46.41) --
	(  1.91, 46.29) --
	(  1.78, 46.15) --
	(  1.78, 46.05) --
	(  1.82, 45.94) --
	(  1.96, 45.62) --
	(  2.07, 45.41) --
	(  2.18, 45.09) --
	(  2.33, 44.69) --
	(  2.15, 44.29) --
	(  2.18, 44.17) --
	(  2.25, 44.12) --
	(  2.26, 44.11) --
	(  2.28, 44.10) --
	(  2.35, 44.04) --
	(  2.52, 43.78) --
	(  2.56, 43.35) --
	(  2.32, 43.13) --
	(  1.83, 42.91) --
	(  1.67, 42.76) --
	(  1.66, 42.64) --
	(  1.75, 42.61) --
	(  1.99, 42.53) --
	(  2.06, 42.37) --
	(  1.94, 42.21) --
	(  1.65, 42.12) --
	(  1.65, 42.10) --
	(  1.82, 41.85) --
	(  1.82, 41.73) --
	(  1.68, 41.48) --
	(  1.68, 41.41) --
	(  1.84, 41.32) --
	(  1.80, 41.09) --
	(  1.41, 40.70) --
	(  1.40, 40.50) --
	(  1.42, 40.35) --
	(  1.55, 40.15) --
	(  1.66, 39.96) --
	(  1.85, 39.73) --
	(  1.68, 39.53) --
	(  1.29, 39.32) --
	(  0.98, 39.12) --
	(  0.94, 39.08) --
	(  0.94, 38.94) --
	(  0.87, 38.87) --
	(  0.86, 38.68) --
	(  0.74, 38.60);

\draw[color=drawColor,line cap=round,line join=round,fill opacity=0.00,] (  0.00, 38.65) --
	(  0.03, 38.70) --
	(  0.19, 38.78) --
	(  0.34, 38.87) --
	(  0.49, 38.91) --
	(  0.60, 39.05) --
	(  0.61, 39.25) --
	(  0.78, 39.40) --
	(  1.28, 39.68) --
	(  1.38, 39.78) --
	(  1.31, 39.92) --
	(  1.27, 40.05) --
	(  1.21, 40.17) --
	(  1.04, 40.30) --
	(  0.98, 40.34) --
	(  0.99, 40.45) --
	(  1.00, 40.54) --
	(  1.07, 40.68) --
	(  1.12, 40.79) --
	(  1.19, 40.91) --
	(  1.31, 41.08) --
	(  1.45, 41.30) --
	(  1.39, 41.42) --
	(  1.37, 41.56) --
	(  1.36, 41.69) --
	(  1.41, 41.84) --
	(  1.40, 41.98) --
	(  1.29, 42.10) --
	(  1.30, 42.20) --
	(  1.43, 42.38) --
	(  1.35, 42.58) --
	(  1.33, 42.72) --
	(  1.35, 42.92) --
	(  1.41, 43.04) --
	(  1.53, 43.17) --
	(  1.68, 43.23) --
	(  1.82, 43.33) --
	(  1.95, 43.40) --
	(  2.10, 43.47) --
	(  2.22, 43.60) --
	(  2.16, 43.76) --
	(  2.09, 43.90) --
	(  1.97, 44.00) --
	(  1.83, 44.13) --
	(  1.82, 44.29) --
	(  1.85, 44.49) --
	(  1.96, 44.68) --
	(  1.94, 44.89) --
	(  1.89, 45.09) --
	(  1.80, 45.27) --
	(  1.74, 45.47) --
	(  1.61, 45.78) --
	(  1.50, 46.02) --
	(  1.44, 46.35) --
	(  1.44, 46.50) --
	(  1.42, 46.68) --
	(  1.44, 46.80) --
	(  1.42, 46.94) --
	(  1.37, 47.14) --
	(  1.41, 47.33) --
	(  1.50, 47.44) --
	(  1.69, 47.49) --
	(  1.99, 47.52) --
	(  1.98, 47.70) --
	(  1.56, 47.97) --
	(  1.42, 48.17) --
	(  1.48, 48.30) --
	(  1.60, 48.43) --
	(  1.67, 48.46) --
	(  1.74, 48.53) --
	(  1.49, 48.70) --
	(  1.25, 48.89) --
	(  1.21, 49.16) --
	(  1.29, 49.36) --
	(  1.33, 49.48) --
	(  1.41, 49.58) --
	(  1.49, 49.70) --
	(  1.58, 49.81) --
	(  1.68, 49.91) --
	(  1.83, 49.95) --
	(  2.18, 50.06) --
	(  2.50, 50.28) --
	(  2.69, 50.50) --
	(  2.95, 50.67) --
	(  2.96, 50.81) --
	(  2.95, 50.90) --
	(  2.92, 51.04) --
	(  3.03, 51.17) --
	(  3.18, 51.24) --
	(  3.44, 51.23) --
	(  3.62, 51.32) --
	(  3.65, 51.41) --
	(  3.70, 51.55) --
	(  3.63, 52.15) --
	(  3.74, 52.77) --
	(  3.86, 53.10) --
	(  4.05, 53.32) --
	(  4.27, 53.55) --
	(  4.34, 53.59) --
	(  4.65, 53.67) --
	(  4.76, 53.67) --
	(  5.15, 53.75) --
	(  5.38, 53.80) --
	(  5.57, 53.90) --
	(  5.66, 53.98) --
	(  5.88, 54.23) --
	(  6.03, 54.28) --
	(  6.24, 54.33) --
	(  6.46, 54.36) --
	(  6.58, 54.43) --
	(  6.74, 54.55) --
	(  7.15, 54.93) --
	(  7.31, 55.10) --
	(  7.55, 55.20) --
	(  7.90, 55.28) --
	(  8.10, 55.29) --
	(  8.41, 55.36) --
	(  8.59, 55.47) --
	(  8.79, 55.58) --
	(  8.91, 55.65) --
	(  8.97, 55.81) --
	(  8.98, 55.90) --
	(  9.06, 55.98) --
	(  9.14, 56.08) --
	(  9.36, 56.18) --
	(  9.50, 56.29) --
	(  9.71, 56.46) --
	(  9.91, 56.65) --
	( 10.15, 56.87) --
	( 10.28, 57.03) --
	( 10.41, 57.08) --
	( 10.47, 57.15) --
	( 10.46, 57.26) --
	( 10.52, 57.34) --
	( 10.63, 57.35) --
	( 10.81, 57.32) --
	( 10.96, 57.36) --
	( 11.15, 57.36) --
	( 11.19, 57.62);

\draw[color=drawColor,line cap=round,line join=round,fill opacity=0.00,] (  1.82, 26.91) --
	(  1.39, 26.99) --
	(  0.64, 27.10) --
	(  0.43, 27.07) --
	(  0.17, 26.99) --
	(  0.07, 26.86) --
	(  0.00, 26.85);

\draw[color=drawColor,line cap=round,line join=round,fill opacity=0.00,] (  0.00, 38.10) --
	(  0.21, 38.36) --
	(  0.31, 38.41) --
	(  0.60, 38.52) --
	(  0.74, 38.60);

\draw[color=drawColor,line cap=round,line join=round,fill opacity=0.00,] (  1.82, 26.91) --
	(  2.21, 26.62) --
	(  2.31, 26.46);

\draw[color=drawColor,line cap=round,line join=round,fill opacity=0.00,] (  2.31, 26.46) --
	(  2.33, 26.41) --
	(  2.41, 26.19);

\draw[color=drawColor,line cap=round,line join=round,fill opacity=0.00,] (  2.41, 26.19) --
	(  2.43, 26.06);

\draw[color=drawColor,line cap=round,line join=round,fill opacity=0.00,] (  0.00, 26.58) --
	(  0.09, 26.58) --
	(  0.33, 26.77) --
	(  0.42, 26.88) --
	(  0.81, 26.81) --
	(  1.09, 26.77) --
	(  1.38, 26.75) --
	(  1.52, 26.74) --
	(  1.63, 26.74) --
	(  1.69, 26.69) --
	(  1.75, 26.64) --
	(  1.86, 26.52) --
	(  1.90, 26.40) --
	(  1.97, 26.24) --
	(  2.01, 26.15) --
	(  2.00, 26.01);

\draw[color=drawColor,line cap=round,line join=round,fill opacity=0.00,] (117.40,279.81) --
	(117.35,279.80) --
	(117.25,279.82) --
	(117.18,279.86) --
	(117.10,279.92) --
	(117.04,279.97) --
	(116.96,280.05) --
	(116.91,280.14) --
	(116.85,280.23) --
	(116.81,280.25) --
	(116.81,280.28) --
	(116.81,280.37) --
	(116.83,280.41) --
	(116.98,280.42) --
	(117.04,280.38) --
	(117.08,280.29) --
	(117.11,280.19) --
	(117.24,280.05) --
	(117.35,279.94) --
	(117.41,279.85) --
	(117.42,279.82) --
	(117.40,279.81);

\draw[color=drawColor,line cap=round,line join=round,fill opacity=0.00,] (109.09,251.41) --
	(109.28,251.37) --
	(109.49,251.45) --
	(109.84,251.43) --
	(110.03,251.32) --
	(110.12,251.25) --
	(110.31,251.01) --
	(110.39,250.87) --
	(110.37,250.70) --
	(110.31,250.58) --
	(110.27,250.48) --
	(110.18,250.36) --
	(110.14,250.27) --
	(110.20,250.18) --
	(110.32,250.10) --
	(110.52,250.05) --
	(110.82,250.00) --
	(111.14,249.93) --
	(111.31,249.85) --
	(111.51,249.71) --
	(111.79,249.51) --
	(111.98,249.42) --
	(112.06,249.32) --
	(112.09,249.25) --
	(112.14,249.20) --
	(112.22,249.06) --
	(111.69,248.75) --
	(111.52,248.89) --
	(111.39,249.02) --
	(111.38,249.09) --
	(111.34,249.31) --
	(111.31,249.43) --
	(111.21,249.55) --
	(111.03,249.56) --
	(110.80,249.59) --
	(110.71,249.58) --
	(110.67,249.64) --
	(110.58,249.66) --
	(110.47,249.68) --
	(110.38,249.62) --
	(110.34,249.57) --
	(110.23,249.56) --
	(110.08,249.60) --
	(109.97,249.69) --
	(109.85,249.80) --
	(109.81,249.86) --
	(109.78,249.94) --
	(109.78,250.08) --
	(109.77,250.11) --
	(109.71,250.16) --
	(109.60,250.21) --
	(109.68,250.28) --
	(109.80,250.41) --
	(109.88,250.59) --
	(109.92,250.71) --
	(109.94,250.83) --
	(109.92,250.91) --
	(109.89,250.97) --
	(109.76,251.05) --
	(109.62,251.10) --
	(109.44,251.14) --
	(109.33,251.14) --
	(109.18,251.13) --
	(109.06,251.13) --
	(108.96,251.17) --
	(108.87,251.17) --
	(108.78,251.22) --
	(108.75,251.28) --
	(108.68,251.36) --
	(108.71,251.48) --
	(108.74,251.55) --
	(108.73,251.66) --
	(108.62,251.79) --
	(108.65,251.84) --
	(108.67,251.88) --
	(108.68,252.03) --
	(108.74,252.03) --
	(108.78,252.04) --
	(108.90,251.83) --
	(108.93,251.62) --
	(109.02,251.49) --
	(109.09,251.41);

\draw[color=drawColor,line cap=round,line join=round,fill opacity=0.00,] (127.98,233.80) --
	(127.89,233.60) --
	(127.82,233.48) --
	(127.62,233.37) --
	(127.38,233.37) --
	(127.18,233.44) --
	(127.09,233.46) --
	(126.97,233.46) --
	(126.87,233.44) --
	(126.79,233.37) --
	(126.83,233.25) --
	(126.93,233.08) --
	(127.10,232.89) --
	(127.35,232.84) --
	(127.47,232.76) --
	(127.44,232.57) --
	(127.34,232.40) --
	(127.25,232.23) --
	(127.17,232.13) --
	(127.08,232.04) --
	(127.01,232.11) --
	(126.95,232.14) --
	(126.85,232.13) --
	(126.75,231.93) --
	(126.63,231.62) --
	(126.56,231.45) --
	(126.31,231.31) --
	(126.11,231.24) --
	(125.96,231.22) --
	(125.93,231.15) --
	(125.89,230.97) --
	(125.82,230.90) --
	(125.77,230.88) --
	(125.67,230.96) --
	(125.60,231.00) --
	(125.65,231.12) --
	(125.65,231.19) --
	(125.57,231.27) --
	(125.43,231.33) --
	(125.31,231.28) --
	(125.13,231.23) --
	(125.03,231.28) --
	(124.92,231.16) --
	(124.98,231.05) --
	(125.05,230.95) --
	(125.09,230.87) --
	(125.02,230.76) --
	(125.01,230.51) --
	(125.04,230.45) --
	(124.97,230.27) --
	(124.87,230.13) --
	(124.72,230.05) --
	(124.63,229.98) --
	(124.57,229.89) --
	(124.65,229.74) --
	(124.41,229.67) --
	(124.23,229.57) --
	(124.27,229.45) --
	(124.24,229.39) --
	(123.98,229.29) --
	(123.86,229.21) --
	(123.91,229.12) --
	(124.15,229.11) --
	(124.33,229.06) --
	(124.46,229.01) --
	(124.59,229.00) --
	(124.62,228.94) --
	(124.62,228.89) --
	(124.69,228.84) --
	(124.73,228.82) --
	(124.75,228.66) --
	(124.89,228.55) --
	(124.94,228.45) --
	(124.88,228.39) --
	(124.94,228.32) --
	(124.99,228.21) --
	(124.88,228.12) --
	(124.79,228.08) --
	(124.81,227.95) --
	(124.86,227.84) --
	(124.91,227.72) --
	(124.89,227.64) --
	(124.74,227.53) --
	(124.61,227.42) --
	(124.61,227.35) --
	(124.61,227.29) --
	(124.69,227.23) --
	(124.83,227.23) --
	(124.90,227.20) --
	(124.92,227.10) --
	(124.91,227.02) --
	(124.77,227.00) --
	(124.70,226.93) --
	(124.70,226.82) --
	(124.82,226.75) --
	(124.93,226.75) --
	(124.99,226.71) --
	(125.05,226.65) --
	(125.13,226.62) --
	(125.23,226.59) --
	(125.19,226.53) --
	(125.16,226.51) --
	(125.16,226.38) --
	(125.16,226.20) --
	(125.17,226.10) --
	(125.18,226.01) --
	(125.29,225.91) --
	(125.35,225.87) --
	(125.37,225.85) --
	(125.28,225.81) --
	(125.23,225.81) --
	(125.17,225.82) --
	(125.10,225.73) --
	(125.03,225.65) --
	(125.03,225.56) --
	(125.01,225.53) --
	(125.09,225.49) --
	(125.12,225.45) --
	(125.14,225.42) --
	(124.99,225.37) --
	(125.02,225.32) --
	(125.01,225.18) --
	(124.96,225.04) --
	(124.89,225.05) --
	(124.83,225.08) --
	(124.74,225.08) --
	(124.65,224.97) --
	(124.52,224.90) --
	(124.47,224.86) --
	(124.57,224.64) --
	(124.58,224.51) --
	(124.57,224.41) --
	(124.53,224.27) --
	(124.47,224.16) --
	(124.40,224.11) --
	(124.23,224.09) --
	(124.13,224.08) --
	(124.07,224.03) --
	(124.15,223.98) --
	(124.25,223.88) --
	(124.36,223.85) --
	(124.51,223.87) --
	(124.61,223.85) --
	(124.76,223.82) --
	(124.83,223.76) --
	(124.80,223.68) --
	(124.69,223.66) --
	(124.60,223.67) --
	(124.54,223.65) --
	(124.54,223.60) --
	(124.65,223.56) --
	(124.71,223.50) --
	(124.76,223.47) --
	(124.94,223.48) --
	(125.09,223.51) --
	(125.27,223.50) --
	(125.34,223.49) --
	(125.36,223.46) --
	(125.37,223.43) --
	(125.37,223.42) --
	(125.34,223.40) --
	(125.34,223.38) --
	(125.26,223.20) --
	(125.22,223.17) --
	(125.24,223.08) --
	(125.20,223.06) --
	(124.81,223.18) --
	(124.74,223.17) --
	(124.64,223.11) --
	(124.34,223.10) --
	(124.03,223.14) --
	(123.95,223.06) --
	(124.09,223.06) --
	(124.25,222.91) --
	(124.46,222.92) --
	(124.73,222.95) --
	(124.79,222.91) --
	(124.80,222.84) --
	(124.90,222.80) --
	(124.93,222.76) --
	(124.90,222.73) --
	(124.84,222.72) --
	(124.34,222.73) --
	(124.27,222.72) --
	(124.25,222.68) --
	(124.23,222.60) --
	(124.13,222.56) --
	(124.22,222.52) --
	(124.26,222.45) --
	(124.24,222.41) --
	(124.02,222.39) --
	(123.91,222.37) --
	(123.72,222.40) --
	(123.67,222.41) --
	(123.55,222.40) --
	(123.34,222.42) --
	(123.27,222.43) --
	(123.27,222.36) --
	(123.25,222.34) --
	(123.20,222.33) --
	(123.12,222.34) --
	(123.04,222.31) --
	(123.02,222.24) --
	(123.03,222.16) --
	(123.14,222.13) --
	(123.44,222.06) --
	(123.51,222.01) --
	(123.52,221.98) --
	(123.51,221.94) --
	(123.49,221.92) --
	(123.45,221.90) --
	(123.26,221.87) --
	(123.16,221.87) --
	(123.08,221.88) --
	(123.01,221.89) --
	(122.80,221.96) --
	(122.65,222.05) --
	(122.49,222.15) --
	(122.45,222.16) --
	(122.40,222.13) --
	(122.34,222.05) --
	(122.34,221.97) --
	(122.34,221.93) --
	(122.41,221.86) --
	(122.48,221.77) --
	(122.51,221.69) --
	(122.48,221.62) --
	(122.35,221.51) --
	(122.29,221.47) --
	(122.27,221.44) --
	(122.39,221.46) --
	(122.53,221.48) --
	(122.65,221.49) --
	(122.73,221.47) --
	(122.75,221.41) --
	(122.73,221.31) --
	(122.80,221.07) --
	(122.91,221.03) --
	(123.09,220.84) --
	(123.24,220.83) --
	(123.38,220.91) --
	(123.53,220.90) --
	(123.88,220.91) --
	(123.97,220.98) --
	(124.04,221.00) --
	(124.15,221.00) --
	(124.24,220.96) --
	(124.30,220.88) --
	(124.47,220.79) --
	(124.62,220.78) --
	(124.68,220.83) --
	(124.75,220.92) --
	(124.81,221.01) --
	(124.83,221.10) --
	(124.85,221.33) --
	(125.09,221.41) --
	(125.16,221.51) --
	(125.33,221.41) --
	(125.56,221.48) --
	(125.70,221.51) --
	(125.78,221.58) --
	(125.81,221.64) --
	(125.99,221.76) --
	(126.09,221.69) --
	(126.20,221.84) --
	(126.25,221.88) --
	(126.27,221.85) --
	(126.28,221.77) --
	(126.26,221.70) --
	(126.26,221.60) --
	(126.27,221.53) --
	(126.27,221.45) --
	(126.02,221.30) --
	(126.00,221.26) --
	(126.00,221.21) --
	(126.09,221.16) --
	(126.26,221.25) --
	(126.40,221.36) --
	(126.62,221.40) --
	(126.76,221.41) --
	(126.91,221.53) --
	(126.94,221.53) --
	(127.02,221.56) --
	(127.14,221.65) --
	(127.19,221.77) --
	(127.24,221.84) --
	(127.41,221.84) --
	(127.49,221.83) --
	(127.32,221.68) --
	(127.34,221.61) --
	(127.40,221.56) --
	(127.50,221.49) --
	(127.45,221.38) --
	(127.40,221.32) --
	(127.25,221.28) --
	(127.20,221.22) --
	(127.20,221.16) --
	(127.23,221.02) --
	(127.19,220.96) --
	(127.13,220.93) --
	(127.12,221.00) --
	(127.10,221.25) --
	(127.04,221.29) --
	(126.98,221.29) --
	(126.89,221.27) --
	(126.78,221.26) --
	(126.57,221.25) --
	(126.48,221.21) --
	(126.38,221.16) --
	(126.16,221.05) --
	(125.98,221.08) --
	(125.86,221.06) --
	(125.82,221.07) --
	(125.80,221.11) --
	(125.78,221.19) --
	(125.74,221.29) --
	(125.72,221.34) --
	(125.78,221.35) --
	(125.78,221.41) --
	(125.72,221.42) --
	(125.59,221.35) --
	(125.48,221.30) --
	(125.34,221.23) --
	(125.27,221.20) --
	(125.16,221.18) --
	(125.06,221.25) --
	(125.01,221.25) --
	(125.01,221.12) --
	(125.00,220.93) --
	(124.94,220.84) --
	(124.79,220.73) --
	(124.63,220.68) --
	(124.56,220.67) --
	(124.44,220.70) --
	(124.13,220.85) --
	(124.06,220.87) --
	(123.96,220.78) --
	(123.87,220.75) --
	(123.78,220.75) --
	(123.63,220.75) --
	(123.53,220.79) --
	(123.44,220.78) --
	(123.44,220.74) --
	(123.43,220.55) --
	(123.31,220.51) --
	(123.27,220.71) --
	(123.23,220.73) --
	(123.12,220.73) --
	(123.03,220.76) --
	(122.99,220.79) --
	(122.87,220.90) --
	(122.72,220.97) --
	(122.65,221.03) --
	(122.61,221.08) --
	(122.56,221.13) --
	(122.51,221.21) --
	(122.48,221.24) --
	(122.40,221.26) --
	(122.27,221.25) --
	(122.15,221.23) --
	(122.03,221.24) --
	(121.97,221.27) --
	(121.91,221.34) --
	(121.88,221.41) --
	(121.93,221.45) --
	(122.07,221.46) --
	(122.07,221.50) --
	(122.05,221.52) --
	(122.01,221.55) --
	(121.93,221.59) --
	(121.86,221.61) --
	(121.62,221.59) --
	(121.38,221.61) --
	(121.50,221.71) --
	(121.56,221.72) --
	(121.76,221.74) --
	(122.22,221.63) --
	(122.27,221.65) --
	(122.20,221.91) --
	(122.17,221.93) --
	(122.11,221.92) --
	(122.02,221.87) --
	(121.93,221.81) --
	(121.84,221.83) --
	(121.82,221.89) --
	(121.86,221.97) --
	(121.92,222.07) --
	(121.98,222.18) --
	(122.05,222.28) --
	(122.11,222.38) --
	(122.22,222.40) --
	(122.31,222.41) --
	(122.40,222.41) --
	(122.53,222.36) --
	(122.72,222.22) --
	(122.91,222.04) --
	(123.15,221.95) --
	(123.21,221.96) --
	(123.21,222.01) --
	(122.99,222.14) --
	(122.84,222.20) --
	(122.81,222.27) --
	(122.94,222.38) --
	(122.97,222.49) --
	(123.08,222.54) --
	(123.24,222.53) --
	(123.27,222.54) --
	(123.27,222.56) --
	(123.26,222.59) --
	(122.99,222.71) --
	(122.86,222.75) --
	(122.62,222.85) --
	(122.53,222.87) --
	(122.17,222.80) --
	(121.99,222.76) --
	(121.94,222.78) --
	(121.91,222.80) --
	(121.90,222.83) --
	(122.38,222.90) --
	(122.46,222.93) --
	(122.48,222.97) --
	(121.85,223.38) --
	(121.81,223.40) --
	(121.83,223.43) --
	(121.85,223.47) --
	(121.86,223.49) --
	(121.99,223.54) --
	(122.08,223.56) --
	(122.19,223.54) --
	(122.23,223.48) --
	(122.33,223.42) --
	(122.38,223.39) --
	(122.41,223.38) --
	(122.46,223.35) --
	(122.57,223.22) --
	(122.66,223.19) --
	(122.74,223.16) --
	(122.80,223.15) --
	(122.90,223.15) --
	(123.02,223.12) --
	(123.11,223.02) --
	(123.15,222.98) --
	(123.20,222.98) --
	(123.32,222.97) --
	(123.45,222.97) --
	(123.59,223.01) --
	(123.60,223.04) --
	(123.64,223.08) --
	(123.62,223.12) --
	(123.59,223.15) --
	(123.40,223.21) --
	(123.25,223.25) --
	(123.28,223.32) --
	(123.31,223.38) --
	(123.32,223.40) --
	(123.28,223.42) --
	(123.27,223.43) --
	(123.21,223.47) --
	(123.16,223.60) --
	(123.22,223.67) --
	(123.39,223.71) --
	(123.50,223.77) --
	(123.64,223.87) --
	(123.74,223.90) --
	(123.79,223.93) --
	(123.72,223.97) --
	(123.65,224.04) --
	(123.62,224.06) --
	(123.64,224.23) --
	(123.67,224.31) --
	(123.75,224.35) --
	(123.99,224.36) --
	(124.09,224.39) --
	(124.09,224.45) --
	(124.07,224.55) --
	(123.89,224.59) --
	(123.71,224.63) --
	(123.54,224.67) --
	(123.39,224.75) --
	(123.27,224.80) --
	(123.12,224.84) --
	(123.10,224.87) --
	(123.12,224.93) --
	(123.36,224.97) --
	(123.61,225.02) --
	(123.82,225.09) --
	(123.97,225.16) --
	(124.16,225.19) --
	(124.22,225.19) --
	(124.25,225.21) --
	(124.19,225.29) --
	(124.11,225.43) --
	(123.99,225.49) --
	(123.78,225.51) --
	(123.52,225.53) --
	(123.49,225.57) --
	(123.57,225.66) --
	(123.54,225.70) --
	(123.49,225.75) --
	(123.58,225.81) --
	(123.86,225.80) --
	(124.03,225.80) --
	(124.03,225.86) --
	(124.07,225.98) --
	(124.05,226.06) --
	(124.02,226.12) --
	(124.01,226.18) --
	(124.19,226.17) --
	(124.43,226.12) --
	(124.56,226.13) --
	(124.62,226.20) --
	(124.50,226.27) --
	(124.41,226.30) --
	(124.35,226.33) --
	(124.35,226.38) --
	(124.35,226.44) --
	(124.21,226.53) --
	(124.12,226.58) --
	(124.09,226.68) --
	(124.03,226.69) --
	(123.91,226.67) --
	(123.77,226.60) --
	(123.70,226.55) --
	(123.67,226.50) --
	(123.56,226.45) --
	(123.52,226.57) --
	(123.56,226.66) --
	(123.66,226.70) --
	(123.69,226.74) --
	(123.68,226.80) --
	(123.63,226.88) --
	(123.58,226.94) --
	(123.60,227.01) --
	(123.67,227.05) --
	(123.87,226.99) --
	(123.94,227.00) --
	(124.05,227.04) --
	(124.05,227.15) --
	(123.98,227.25) --
	(123.91,227.30) --
	(123.87,227.39) --
	(123.84,227.47) --
	(123.84,227.50) --
	(123.93,227.51) --
	(124.02,227.51) --
	(124.08,227.54) --
	(124.05,227.62) --
	(123.97,227.64) --
	(123.84,227.72) --
	(123.78,227.79) --
	(123.88,227.84) --
	(124.03,227.79) --
	(124.35,227.74) --
	(124.39,227.84) --
	(124.37,227.96) --
	(124.33,228.06) --
	(124.34,228.17) --
	(124.37,228.30) --
	(124.32,228.41) --
	(124.26,228.45) --
	(124.14,228.41) --
	(124.07,228.29) --
	(123.98,228.16) --
	(123.92,228.06) --
	(123.80,228.04) --
	(123.74,228.12) --
	(123.79,228.27) --
	(123.81,228.33) --
	(123.39,228.40) --
	(123.02,228.49) --
	(122.82,228.55) --
	(122.87,228.68) --
	(122.94,228.78) --
	(123.03,228.82) --
	(123.00,228.88) --
	(122.95,229.03) --
	(122.92,229.12) --
	(122.91,229.24) --
	(122.78,229.23) --
	(122.46,229.32) --
	(122.47,229.40) --
	(122.67,229.46) --
	(122.77,229.48) --
	(122.73,229.55) --
	(122.69,229.64) --
	(122.66,229.72) --
	(122.74,229.76) --
	(122.87,229.72) --
	(123.06,229.65) --
	(123.19,229.60) --
	(123.34,229.60) --
	(123.45,229.66) --
	(123.57,229.75) --
	(123.61,229.79) --
	(123.65,229.91) --
	(123.67,230.03) --
	(123.66,230.07) --
	(123.80,230.11) --
	(123.94,230.05) --
	(124.24,230.08) --
	(124.39,230.15) --
	(124.55,230.30) --
	(124.64,230.40) --
	(124.62,230.48) --
	(124.50,230.49) --
	(124.30,230.55) --
	(124.03,230.60) --
	(123.95,230.77) --
	(123.95,230.95) --
	(123.90,231.01) --
	(123.76,231.08) --
	(123.61,231.08) --
	(123.43,231.04) --
	(123.31,231.03) --
	(123.25,231.04) --
	(123.28,231.11) --
	(123.41,231.21) --
	(123.64,231.28) --
	(123.74,231.33) --
	(123.72,231.42) --
	(123.69,231.51) --
	(123.69,231.62) --
	(123.61,231.61) --
	(123.47,231.55) --
	(123.37,231.56) --
	(123.30,231.65) --
	(123.32,231.77) --
	(123.41,231.94) --
	(123.50,232.02) --
	(123.54,232.08) --
	(123.56,232.13) --
	(123.51,232.18) --
	(123.44,232.15) --
	(123.32,232.06) --
	(123.18,231.98) --
	(123.06,231.97) --
	(122.91,232.02) --
	(122.85,232.09) --
	(122.82,232.17) --
	(122.85,232.23) --
	(122.92,232.30) --
	(122.86,232.41) --
	(122.78,232.53) --
	(122.75,232.59) --
	(122.66,232.62) --
	(122.43,232.62) --
	(122.26,232.62) --
	(122.08,232.61) --
	(122.01,232.65) --
	(122.02,232.71) --
	(122.09,232.83) --
	(122.26,232.97) --
	(122.36,233.03) --
	(122.42,233.08) --
	(122.37,233.16) --
	(122.26,233.28) --
	(122.23,233.34) --
	(122.26,233.45) --
	(122.33,233.55) --
	(122.35,233.56) --
	(122.38,233.46) --
	(122.41,233.42) --
	(122.44,233.36) --
	(122.65,233.40) --
	(122.85,233.41) --
	(122.88,233.34) --
	(122.87,233.20) --
	(122.72,233.04) --
	(122.53,232.95) --
	(122.50,232.90) --
	(122.44,232.85) --
	(122.47,232.79) --
	(122.68,232.80) --
	(122.80,232.79) --
	(122.93,232.67) --
	(123.03,232.56) --
	(123.16,232.53) --
	(123.16,232.43) --
	(123.13,232.27) --
	(123.22,232.27) --
	(123.43,232.35) --
	(123.60,232.41) --
	(123.72,232.39) --
	(123.85,232.36) --
	(123.89,232.25) --
	(123.89,232.14) --
	(123.83,232.00) --
	(123.77,231.89) --
	(123.76,231.84) --
	(123.84,231.80) --
	(123.93,231.81) --
	(124.00,231.83) --
	(124.08,231.79) --
	(124.11,231.70) --
	(124.16,231.60) --
	(124.10,231.51) --
	(124.03,231.42) --
	(124.00,231.36) --
	(124.01,231.31) --
	(124.10,231.30) --
	(124.15,231.20) --
	(124.20,231.09) --
	(124.22,230.98) --
	(124.25,230.88) --
	(124.30,230.81) --
	(124.39,230.75) --
	(124.57,230.79) --
	(124.66,230.86) --
	(124.65,230.99) --
	(124.59,231.15) --
	(124.59,231.27) --
	(124.63,231.41) --
	(124.81,231.50) --
	(125.20,231.50) --
	(125.58,231.51) --
	(125.97,231.56) --
	(126.25,231.64) --
	(126.34,231.76) --
	(126.41,232.15) --
	(126.47,232.26) --
	(126.59,232.31) --
	(126.75,232.35) --
	(126.98,232.46) --
	(127.05,232.53) --
	(126.99,232.72) --
	(126.91,232.73) --
	(126.73,232.80) --
	(126.64,232.87) --
	(126.59,233.02) --
	(126.62,233.18) --
	(126.58,233.31) --
	(126.61,233.47) --
	(126.67,233.58) --
	(126.76,233.65) --
	(126.88,233.66) --
	(127.09,233.67) --
	(127.26,233.63) --
	(127.42,233.63) --
	(127.57,233.73) --
	(127.58,233.79) --
	(127.51,233.91) --
	(127.44,234.00) --
	(127.38,234.07) --
	(127.37,234.14) --
	(127.36,234.21) --
	(127.45,234.26) --
	(127.68,234.34) --
	(127.86,234.47) --
	(128.01,234.60) --
	(128.11,234.71) --
	(128.15,234.79) --
	(128.15,234.89) --
	(128.16,235.01) --
	(128.33,234.96) --
	(128.42,234.89) --
	(128.46,234.69) --
	(128.42,234.47) --
	(128.25,234.37) --
	(128.06,234.29) --
	(127.91,234.24) --
	(127.83,234.16) --
	(127.84,234.05) --
	(127.96,233.94) --
	(127.98,233.80);

\draw[color=drawColor,line cap=round,line join=round,fill opacity=0.00,] (215.78,203.96) --
	(215.58,203.94) --
	(215.52,204.00) --
	(215.48,204.02) --
	(215.47,204.04) --
	(215.45,204.06) --
	(215.45,204.11) --
	(215.50,204.19) --
	(215.48,204.26) --
	(215.56,204.40) --
	(215.63,204.45) --
	(215.69,204.54) --
	(215.70,204.63) --
	(215.72,204.72) --
	(215.76,204.79) --
	(215.78,204.83) --
	(215.87,204.91) --
	(215.96,204.90) --
	(215.99,204.88) --
	(215.99,204.84) --
	(215.99,204.76) --
	(215.96,204.68) --
	(215.91,204.60) --
	(215.90,204.55) --
	(215.88,204.53) --
	(215.88,204.43) --
	(215.88,204.22) --
	(215.78,204.11) --
	(215.78,203.96);

\draw[color=drawColor,line cap=round,line join=round,fill opacity=0.00,] (214.99,203.31) --
	(215.06,203.30) --
	(215.20,203.46) --
	(215.30,203.43) --
	(215.45,203.43) --
	(215.45,203.32) --
	(215.45,203.28) --
	(215.43,203.20) --
	(215.43,203.17) --
	(215.51,203.09) --
	(215.66,203.10) --
	(215.66,203.02) --
	(215.76,202.78) --
	(215.86,202.75) --
	(216.00,202.75) --
	(216.30,203.14) --
	(216.30,203.29) --
	(216.18,203.55) --
	(216.31,203.40) --
	(216.40,203.29) --
	(216.44,203.14) --
	(216.31,203.03) --
	(216.32,202.93) --
	(216.25,202.86) --
	(216.10,202.74) --
	(216.01,202.66) --
	(215.91,202.62) --
	(215.88,202.59) --
	(215.88,202.56) --
	(215.94,202.51) --
	(216.01,202.50) --
	(216.06,202.50) --
	(216.32,202.43) --
	(216.52,202.36) --
	(216.71,202.31) --
	(216.89,202.24) --
	(217.02,202.23) --
	(217.18,202.18) --
	(217.30,202.13) --
	(217.48,202.08) --
	(217.66,201.99) --
	(217.75,201.88) --
	(217.54,201.98) --
	(217.35,201.98) --
	(217.24,201.96) --
	(217.15,201.97) --
	(216.99,202.01) --
	(216.97,202.05) --
	(216.92,202.05) --
	(216.75,202.11) --
	(216.66,202.11) --
	(216.62,202.13) --
	(216.53,202.12) --
	(216.44,202.13) --
	(216.34,202.16) --
	(216.26,202.16) --
	(216.16,202.13) --
	(216.08,202.10) --
	(215.77,202.09) --
	(215.61,202.11) --
	(215.45,202.17) --
	(215.33,202.39) --
	(215.25,202.44) --
	(215.20,202.50) --
	(215.00,202.64) --
	(214.96,202.64) --
	(214.93,202.55) --
	(214.84,202.46) --
	(214.71,202.35) --
	(214.63,202.36) --
	(214.62,202.41) --
	(214.68,202.59) --
	(214.69,202.67) --
	(214.54,202.81) --
	(214.44,202.83) --
	(214.42,202.96) --
	(214.39,202.99) --
	(214.47,203.05) --
	(214.45,203.12) --
	(214.50,203.18) --
	(214.60,203.17) --
	(214.72,203.18) --
	(214.66,203.33) --
	(214.74,203.37) --
	(214.83,203.41) --
	(214.86,203.44) --
	(214.87,203.53) --
	(214.92,203.56) --
	(214.96,203.57) --
	(215.02,203.54) --
	(215.03,203.52) --
	(214.99,203.31);

\draw[color=drawColor,line cap=round,line join=round,fill opacity=0.00,] (197.55,169.74) --
	(197.52,169.59) --
	(197.47,169.58) --
	(197.41,169.59) --
	(197.35,169.60) --
	(197.23,169.61) --
	(197.13,169.61) --
	(196.98,169.58) --
	(196.69,169.57) --
	(196.60,169.72) --
	(196.74,169.79) --
	(196.82,169.85) --
	(197.07,170.03) --
	(197.26,170.27) --
	(197.30,170.32) --
	(197.33,170.34) --
	(197.52,170.43) --
	(197.69,170.47) --
	(197.72,170.40) --
	(197.74,170.33) --
	(197.71,170.24) --
	(197.71,170.22) --
	(197.78,170.17) --
	(197.80,170.11) --
	(197.78,170.05) --
	(197.70,169.96) --
	(197.61,169.88) --
	(197.55,169.80) --
	(197.55,169.74);

\draw[color=drawColor,line cap=round,line join=round,fill opacity=0.00,] (179.98,161.38) --
	(180.06,161.32) --
	(180.33,161.52) --
	(180.41,161.84) --
	(180.48,161.86) --
	(180.46,161.55) --
	(180.46,161.33) --
	(180.54,161.32) --
	(180.64,161.27) --
	(180.68,161.23) --
	(180.74,161.23) --
	(180.81,161.27) --
	(180.83,161.17) --
	(180.84,161.10) --
	(180.92,161.06) --
	(180.98,161.08) --
	(181.05,161.04) --
	(181.16,161.08) --
	(181.25,161.09) --
	(181.42,161.16) --
	(181.51,161.25) --
	(181.58,161.25) --
	(181.64,161.24) --
	(181.67,161.22) --
	(181.70,161.17) --
	(181.77,161.16) --
	(181.86,161.20) --
	(181.90,161.22) --
	(181.94,161.25) --
	(182.07,161.32) --
	(182.25,161.31) --
	(182.42,161.29) --
	(182.51,161.24) --
	(182.66,161.20) --
	(182.77,161.18) --
	(182.94,161.32) --
	(183.02,161.30) --
	(182.96,161.17) --
	(182.89,160.99) --
	(182.94,160.83) --
	(183.10,160.73) --
	(183.34,160.85) --
	(183.57,160.81) --
	(183.76,160.98) --
	(183.83,160.99) --
	(183.94,160.89) --
	(184.12,160.89) --
	(184.32,161.10) --
	(184.38,161.11) --
	(184.50,161.18) --
	(184.57,161.23) --
	(184.67,161.30) --
	(184.81,161.36) --
	(184.94,161.40) --
	(185.04,161.43) --
	(185.23,161.50) --
	(185.39,161.52) --
	(185.51,161.51) --
	(185.58,161.50) --
	(185.71,161.49) --
	(185.88,161.51) --
	(185.97,161.48) --
	(186.12,161.44) --
	(186.46,161.58) --
	(186.58,161.81) --
	(186.70,162.08) --
	(186.72,161.78) --
	(186.67,161.64) --
	(186.65,161.57) --
	(186.60,161.50) --
	(186.55,161.46) --
	(186.47,161.41) --
	(186.38,161.36) --
	(186.25,161.33) --
	(186.13,161.31) --
	(186.02,161.33) --
	(185.83,161.40) --
	(185.76,161.30) --
	(185.73,161.19) --
	(185.72,161.13) --
	(185.67,161.01) --
	(185.43,161.07) --
	(185.38,161.09) --
	(185.30,161.07) --
	(185.18,161.09) --
	(184.96,161.03) --
	(184.73,160.99) --
	(184.53,160.75) --
	(184.39,160.73) --
	(184.29,160.68) --
	(184.20,160.63) --
	(183.98,160.61) --
	(183.13,160.62) --
	(183.10,160.62) --
	(182.95,160.67) --
	(182.85,160.77) --
	(182.66,160.95) --
	(182.50,160.94) --
	(182.21,160.92) --
	(182.42,160.72) --
	(182.59,160.60) --
	(182.67,160.54) --
	(182.71,160.52) --
	(182.75,160.48) --
	(182.83,160.39) --
	(182.87,160.33) --
	(182.89,160.27) --
	(182.79,160.27) --
	(182.63,160.34) --
	(182.51,160.42) --
	(182.39,160.50) --
	(182.30,160.59) --
	(182.20,160.66) --
	(182.02,160.81) --
	(181.91,160.79) --
	(181.71,160.76) --
	(181.55,160.75) --
	(181.46,160.76) --
	(181.34,160.82) --
	(181.20,160.80) --
	(181.10,160.75) --
	(181.02,160.73) --
	(180.95,160.74) --
	(180.91,160.81) --
	(180.85,160.93) --
	(180.56,160.99) --
	(180.28,161.00) --
	(179.93,160.95) --
	(179.72,160.78) --
	(179.81,160.57) --
	(180.04,160.47) --
	(180.12,160.42) --
	(180.20,160.39) --
	(180.34,160.26) --
	(180.44,160.15) --
	(180.50,160.13) --
	(180.62,160.10) --
	(180.73,159.92) --
	(180.87,159.83) --
	(180.95,159.78) --
	(180.94,159.62) --
	(180.99,159.51) --
	(181.19,159.39) --
	(181.21,159.27) --
	(181.20,159.22) --
	(181.20,159.18) --
	(181.21,159.10) --
	(181.17,159.11) --
	(181.12,159.18) --
	(181.12,159.33) --
	(180.93,159.48) --
	(180.83,159.63) --
	(180.78,159.74) --
	(180.55,159.80) --
	(180.40,160.07) --
	(180.31,160.13) --
	(180.05,160.25) --
	(179.84,160.39) --
	(179.71,160.45) --
	(179.59,160.50) --
	(179.40,160.54) --
	(179.26,160.58) --
	(179.19,160.58) --
	(179.16,160.64) --
	(179.08,160.78) --
	(179.12,160.85) --
	(179.14,160.90) --
	(179.19,160.93) --
	(179.37,161.02) --
	(179.51,161.14) --
	(179.71,161.21) --
	(179.71,161.41) --
	(179.73,161.60) --
	(179.70,161.78) --
	(179.75,162.05) --
	(179.72,162.10) --
	(179.70,162.39) --
	(179.85,162.27) --
	(179.91,162.13) --
	(179.89,161.87) --
	(179.89,161.61) --
	(179.98,161.38);

\draw[color=drawColor,line cap=round,line join=round,fill opacity=0.00,] (174.91,160.53) --
	(175.75,160.53) --
	(175.94,160.55) --
	(176.19,160.54) --
	(176.46,160.52) --
	(176.78,160.53) --
	(176.93,160.56) --
	(177.05,160.62) --
	(177.21,160.62) --
	(177.35,160.59) --
	(177.17,160.52) --
	(177.06,160.48) --
	(176.99,160.47) --
	(176.75,160.39) --
	(176.34,160.35) --
	(175.81,160.37) --
	(175.39,160.30) --
	(174.91,160.38) --
	(174.33,160.42) --
	(173.71,160.58) --
	(173.75,160.65) --
	(173.81,160.71) --
	(174.21,160.66) --
	(174.46,160.58) --
	(174.91,160.53);

\draw[color=drawColor,line cap=round,line join=round,fill opacity=0.00,] (103.71,146.78) --
	(103.77,146.77) --
	(103.93,146.77) --
	(104.06,146.76) --
	(104.18,146.70) --
	(103.99,146.63) --
	(103.92,146.57) --
	(103.69,146.53) --
	(103.57,146.47) --
	(103.50,146.38) --
	(103.44,146.24) --
	(103.44,146.11) --
	(103.49,145.85) --
	(103.41,145.71) --
	(103.33,145.62) --
	(103.24,145.55) --
	(103.05,145.49) --
	(102.92,145.42) --
	(102.80,145.31) --
	(102.79,145.26) --
	(102.77,145.14) --
	(102.64,144.96) --
	(102.65,144.90) --
	(102.68,144.88) --
	(102.72,144.87) --
	(102.81,144.88) --
	(102.95,144.92) --
	(103.11,144.96) --
	(103.17,144.94) --
	(103.17,144.88) --
	(102.95,144.72) --
	(102.92,144.67) --
	(102.93,144.63) --
	(102.97,144.58) --
	(103.04,144.55) --
	(103.17,144.54) --
	(103.29,144.57) --
	(103.36,144.57) --
	(103.37,144.55) --
	(103.39,144.50) --
	(103.37,144.48) --
	(103.21,144.37) --
	(103.17,144.32) --
	(103.14,144.27) --
	(103.14,144.22) --
	(103.15,144.20) --
	(103.24,144.17) --
	(103.32,144.17) --
	(103.48,144.18) --
	(103.75,144.19) --
	(103.68,144.12) --
	(103.49,144.09) --
	(103.25,144.04) --
	(103.09,144.02) --
	(103.01,144.08) --
	(102.96,144.13) --
	(102.94,144.24) --
	(102.94,144.35) --
	(102.90,144.42) --
	(102.81,144.46) --
	(102.71,144.48) --
	(102.51,144.47) --
	(102.35,144.49) --
	(102.24,144.54) --
	(102.11,144.60) --
	(101.97,144.62) --
	(101.82,144.59) --
	(101.77,144.56) --
	(101.69,144.48) --
	(101.68,144.41) --
	(101.69,144.34) --
	(101.72,144.23) --
	(101.76,144.16) --
	(101.77,144.07) --
	(101.81,144.00) --
	(101.87,143.96) --
	(101.97,143.93) --
	(102.08,143.93) --
	(102.21,143.98) --
	(102.36,144.04) --
	(102.49,144.05) --
	(102.62,144.06) --
	(102.67,144.02) --
	(102.80,143.95) --
	(102.84,143.88) --
	(102.84,143.84) --
	(102.71,143.64) --
	(102.69,143.56) --
	(102.63,143.45) --
	(102.54,143.35) --
	(102.46,143.20) --
	(102.37,143.08) --
	(102.34,142.97) --
	(102.25,142.85) --
	(102.14,142.77) --
	(102.09,142.71) --
	(102.07,142.68) --
	(102.07,142.66) --
	(102.10,142.63) --
	(102.17,142.63) --
	(102.33,142.70) --
	(102.46,142.78) --
	(102.53,142.79) --
	(102.60,142.79) --
	(102.66,142.77) --
	(102.67,142.72) --
	(102.51,142.50) --
	(102.54,142.44) --
	(102.61,142.39) --
	(102.77,142.32) --
	(102.87,142.26) --
	(102.92,142.18) --
	(102.86,142.03) --
	(102.76,142.09) --
	(102.61,142.22) --
	(102.46,142.23) --
	(102.33,142.23) --
	(102.22,142.22) --
	(102.18,142.22) --
	(101.99,142.19) --
	(101.77,142.17) --
	(101.59,142.11) --
	(101.54,142.06) --
	(101.48,142.01) --
	(101.44,141.95) --
	(101.44,141.90) --
	(101.47,141.84) --
	(101.52,141.81) --
	(101.55,141.75) --
	(101.56,141.67) --
	(101.60,141.55) --
	(101.84,141.45) --
	(101.86,141.40) --
	(101.66,141.38) --
	(101.57,141.36) --
	(101.47,141.25) --
	(101.28,141.07) --
	(101.28,141.04) --
	(101.32,141.03) --
	(101.62,141.02) --
	(101.76,140.95) --
	(101.79,140.91) --
	(101.75,140.88) --
	(101.66,140.84) --
	(101.49,140.77) --
	(101.38,140.69) --
	(101.31,140.59) --
	(101.25,140.54) --
	(101.17,140.50) --
	(101.12,140.52) --
	(101.01,140.55) --
	(100.88,140.69) --
	(100.71,140.84) --
	(100.63,140.86) --
	(100.54,140.85) --
	(100.48,140.83) --
	(100.45,140.80) --
	(100.41,140.69) --
	(100.32,140.56) --
	(100.22,140.39) --
	(100.17,140.35) --
	(100.09,140.32) --
	(100.03,140.32) --
	( 99.91,140.31) --
	( 99.84,140.32) --
	( 99.73,140.39) --
	( 99.59,140.47) --
	( 99.54,140.46) --
	( 99.49,140.41) --
	( 99.47,140.37) --
	( 99.48,140.31) --
	( 99.47,140.23) --
	( 99.44,140.04) --
	( 99.38,139.96) --
	( 99.30,139.87) --
	( 99.19,139.80) --
	( 99.12,139.78) --
	( 99.01,139.74) --
	( 98.91,139.68) --
	( 98.84,139.63) --
	( 98.62,139.58) --
	( 98.55,139.52) --
	( 98.55,139.45) --
	( 98.56,139.33) --
	( 98.56,139.27) --
	( 98.57,139.22) --
	( 98.66,139.13) --
	( 98.77,139.09) --
	( 98.81,139.05) --
	( 98.83,138.99) --
	( 98.76,138.97) --
	( 98.70,138.93) --
	( 98.63,138.92) --
	( 98.54,138.84) --
	( 98.49,138.75) --
	( 98.41,138.68) --
	( 98.29,138.62) --
	( 98.23,138.59) --
	( 98.02,138.53) --
	( 98.02,138.48) --
	( 98.03,138.46) --
	( 98.13,138.45) --
	( 98.35,138.47) --
	( 98.45,138.46) --
	( 98.47,138.42) --
	( 98.39,138.38) --
	( 98.41,138.30) --
	( 98.31,138.26) --
	( 98.08,138.29) --
	( 97.93,138.37) --
	( 97.84,138.41) --
	( 97.79,138.40) --
	( 97.68,138.34) --
	( 97.63,138.28) --
	( 97.55,138.22) --
	( 97.52,138.15) --
	( 97.51,138.05) --
	( 97.51,137.93) --
	( 97.50,137.89) --
	( 97.58,137.84) --
	( 97.61,137.83) --
	( 98.02,137.66) --
	( 97.99,137.61) --
	( 97.77,137.59) --
	( 97.35,137.83) --
	( 97.26,137.82) --
	( 97.16,137.77) --
	( 97.06,137.62) --
	( 96.98,137.43) --
	( 96.91,137.26) --
	( 96.89,137.19) --
	( 96.92,137.15) --
	( 96.96,137.14) --
	( 97.06,137.20) --
	( 97.11,137.20) --
	( 97.29,137.09) --
	( 97.35,137.08) --
	( 97.42,137.05) --
	( 97.52,136.99) --
	( 97.55,136.71) --
	( 97.58,136.68) --
	( 97.91,136.82) --
	( 97.96,136.81) --
	( 97.97,136.77) --
	( 97.81,136.62) --
	( 97.82,136.56) --
	( 98.04,136.48) --
	( 98.24,136.56) --
	( 98.30,136.55) --
	( 98.31,136.54) --
	( 98.27,136.48) --
	( 98.26,136.45) --
	( 98.59,136.41) --
	( 98.65,136.41) --
	( 98.88,136.47) --
	( 98.94,136.47) --
	( 98.94,136.43) --
	( 98.91,136.34) --
	( 99.45,136.14) --
	( 99.23,136.08) --
	( 99.19,136.09) --
	( 99.13,136.15) --
	( 98.96,136.19) --
	( 98.55,136.29) --
	( 98.25,136.33) --
	( 98.08,136.30) --
	( 97.99,136.33) --
	( 97.65,136.52) --
	( 97.48,136.53) --
	( 97.21,136.51) --
	( 97.14,136.50) --
	( 96.92,136.44) --
	( 96.73,136.40) --
	( 96.58,136.41) --
	( 96.46,136.43) --
	( 96.38,136.41) --
	( 96.26,136.37) --
	( 96.23,136.37) --
	( 96.16,136.37) --
	( 96.06,136.39) --
	( 95.93,136.42) --
	( 95.63,136.41) --
	( 95.55,136.38) --
	( 95.52,136.30) --
	( 95.50,136.22) --
	( 95.47,136.09) --
	( 95.44,136.00) --
	( 95.18,135.92) --
	( 95.15,135.90) --
	( 95.18,135.81) --
	( 95.21,135.75) --
	( 95.25,135.72) --
	( 95.35,135.66) --
	( 95.45,135.68) --
	( 95.61,135.69) --
	( 95.73,135.67) --
	( 95.90,135.64) --
	( 96.05,135.59) --
	( 96.09,135.50) --
	( 96.06,135.42) --
	( 96.07,135.34) --
	( 96.13,135.33) --
	( 96.21,135.34) --
	( 96.34,135.45) --
	( 96.39,135.49) --
	( 96.65,135.49) --
	( 96.96,135.61) --
	( 96.93,135.52) --
	( 96.95,135.48) --
	( 96.99,135.45) --
	( 97.06,135.44) --
	( 97.24,135.36) --
	( 97.26,135.33) --
	( 97.24,135.31) --
	( 97.07,135.35) --
	( 96.93,135.39) --
	( 96.82,135.39) --
	( 96.69,135.32) --
	( 96.68,135.29) --
	( 96.75,135.20) --
	( 96.77,135.15) --
	( 96.72,135.13) --
	( 96.67,135.14) --
	( 96.57,135.29) --
	( 96.51,135.29) --
	( 96.44,135.21) --
	( 96.33,135.19) --
	( 96.15,135.13) --
	( 96.07,135.04) --
	( 95.93,134.90) --
	( 95.86,134.80) --
	( 95.71,134.60) --
	( 95.57,134.51) --
	( 95.44,134.50) --
	( 95.33,134.51) --
	( 95.21,134.56) --
	( 95.10,134.62) --
	( 94.96,134.64) --
	( 94.87,134.62) --
	( 94.86,134.57) --
	( 94.84,134.54) --
	( 94.88,134.47) --
	( 94.93,134.45) --
	( 95.00,134.40) --
	( 95.05,134.36) --
	( 95.21,134.27) --
	( 95.36,134.19) --
	( 95.45,134.11) --
	( 95.55,133.99) --
	( 95.56,133.89) --
	( 95.52,133.86) --
	( 95.43,133.70) --
	( 95.33,133.59) --
	( 95.18,133.37) --
	( 95.19,133.33) --
	( 95.23,133.26) --
	( 95.32,133.24) --
	( 95.39,133.22) --
	( 95.55,133.19) --
	( 95.55,133.12) --
	( 95.52,133.10) --
	( 95.39,133.10) --
	( 95.22,133.07) --
	( 95.02,133.09) --
	( 94.90,133.17) --
	( 94.82,133.20) --
	( 94.75,133.20) --
	( 94.75,133.19) --
	( 94.70,133.10) --
	( 94.63,132.95) --
	( 94.64,132.91) --
	( 94.67,132.83) --
	( 94.81,132.72) --
	( 94.88,132.57) --
	( 94.99,132.58) --
	( 94.99,132.55) --
	( 94.95,132.48) --
	( 94.95,132.43) --
	( 94.96,132.41) --
	( 95.06,132.34) --
	( 95.06,132.30) --
	( 95.06,132.28) --
	( 95.07,132.24) --
	( 95.23,132.22) --
	( 95.33,132.14) --
	( 95.23,132.05) --
	( 95.11,131.93) --
	( 95.06,131.96) --
	( 94.99,132.10) --
	( 94.90,132.19) --
	( 94.86,132.21) --
	( 94.82,132.46) --
	( 94.79,132.51) --
	( 94.62,132.52) --
	( 94.58,132.53) --
	( 94.65,132.57) --
	( 94.72,132.64) --
	( 94.70,132.68) --
	( 94.53,132.72) --
	( 94.44,132.73) --
	( 94.34,132.73) --
	( 94.28,132.70) --
	( 94.22,132.61) --
	( 94.08,132.60) --
	( 94.00,132.60) --
	( 93.94,132.58) --
	( 93.87,132.46) --
	( 93.84,132.47) --
	( 93.79,132.54) --
	( 93.75,132.54) --
	( 93.71,132.49) --
	( 93.53,132.34) --
	( 93.43,132.29) --
	( 93.36,132.29) --
	( 93.36,132.37) --
	( 93.35,132.46) --
	( 93.32,132.56) --
	( 93.13,132.65) --
	( 93.04,132.69) --
	( 92.57,132.92) --
	( 92.43,133.02) --
	( 92.08,133.23) --
	( 91.96,133.26) --
	( 91.88,133.25) --
	( 91.85,133.20) --
	( 91.79,133.11) --
	( 91.83,133.02) --
	( 91.86,132.97) --
	( 92.03,132.88) --
	( 92.13,132.85) --
	( 92.26,132.80) --
	( 92.41,132.76) --
	( 92.53,132.69) --
	( 92.64,132.52) --
	( 92.68,132.30) --
	( 92.66,132.20) --
	( 92.65,132.07) --
	( 92.56,131.94) --
	( 92.44,131.90) --
	( 92.23,131.85) --
	( 92.01,131.81) --
	( 91.93,131.79) --
	( 91.90,131.74) --
	( 91.88,131.63) --
	( 92.11,131.37) --
	( 92.13,131.28) --
	( 92.15,131.22) --
	( 92.10,131.14) --
	( 91.93,130.97) --
	( 91.83,130.91) --
	( 91.70,130.87) --
	( 91.66,130.82) --
	( 91.60,130.66) --
	( 91.62,130.51) --
	( 91.65,130.36) --
	( 91.63,130.31) --
	( 91.42,130.17) --
	( 91.29,130.11) --
	( 91.19,130.07) --
	( 91.17,130.00) --
	( 91.20,129.95) --
	( 91.37,129.76) --
	( 91.38,129.70) --
	( 91.38,129.63) --
	( 91.28,129.32) --
	( 91.31,129.29) --
	( 91.38,129.30) --
	( 91.52,129.43) --
	( 91.71,129.63) --
	( 91.72,129.64) --
	( 91.81,129.69) --
	( 91.90,129.70) --
	( 91.97,129.69) --
	( 92.04,129.62) --
	( 92.04,129.58) --
	( 92.02,129.51) --
	( 91.76,129.19) --
	( 91.74,129.12) --
	( 91.77,129.04) --
	( 91.81,129.01) --
	( 91.92,129.20) --
	( 91.97,129.23) --
	( 92.02,129.23) --
	( 92.05,129.20) --
	( 92.09,129.09) --
	( 92.13,129.06) --
	( 92.24,129.05) --
	( 92.33,129.06) --
	( 92.44,129.22) --
	( 92.47,129.27) --
	( 92.53,129.31) --
	( 92.63,129.29) --
	( 92.74,129.27) --
	( 92.76,129.24) --
	( 92.78,129.08) --
	( 92.85,129.03) --
	( 92.93,129.03) --
	( 93.03,129.08) --
	( 93.27,129.15) --
	( 93.48,129.15) --
	( 93.61,129.09) --
	( 93.68,129.09) --
	( 93.74,129.11) --
	( 93.81,129.14) --
	( 93.84,129.25) --
	( 93.89,129.31) --
	( 93.93,129.31) --
	( 94.01,129.29) --
	( 94.05,129.16) --
	( 94.10,129.12) --
	( 94.25,129.18) --
	( 94.30,129.13) --
	( 94.35,129.11) --
	( 94.41,129.11) --
	( 94.44,129.16) --
	( 94.43,129.29) --
	( 94.43,129.32) --
	( 94.46,129.36) --
	( 94.53,129.42) --
	( 94.63,129.44) --
	( 94.70,129.43) --
	( 94.76,129.38) --
	( 94.80,129.19) --
	( 94.81,129.17) --
	( 94.92,129.17) --
	( 94.98,129.21) --
	( 95.05,129.36) --
	( 95.14,129.46) --
	( 95.21,129.50) --
	( 95.31,129.55) --
	( 95.39,129.54) --
	( 95.51,129.53) --
	( 95.63,129.42) --
	( 95.67,129.41) --
	( 95.80,129.44) --
	( 96.01,129.47) --
	( 96.10,129.45) --
	( 96.18,129.40) --
	( 96.25,129.40) --
	( 96.32,129.40) --
	( 96.42,129.42) --
	( 96.47,129.43) --
	( 96.50,129.50) --
	( 96.52,129.52) --
	( 96.58,129.63) --
	( 96.66,129.68) --
	( 96.77,129.76) --
	( 96.77,129.84) --
	( 96.76,129.90) --
	( 96.72,129.96) --
	( 96.56,130.04) --
	( 96.53,130.09) --
	( 96.54,130.14) --
	( 96.75,130.38) --
	( 96.80,130.45) --
	( 96.80,130.49) --
	( 96.76,130.54) --
	( 96.70,130.60) --
	( 96.83,130.60) --
	( 96.94,130.53) --
	( 96.97,130.49) --
	( 96.98,130.43) --
	( 96.92,130.37) --
	( 96.74,130.15) --
	( 96.75,130.12) --
	( 96.86,130.12) --
	( 97.03,130.14) --
	( 97.15,130.16) --
	( 97.16,130.14) --
	( 97.15,129.98) --
	( 97.16,129.93) --
	( 97.23,129.80) --
	( 97.25,129.68) --
	( 97.24,129.63) --
	( 97.24,129.52) --
	( 97.23,129.39) --
	( 97.23,129.30) --
	( 97.22,129.23) --
	( 97.12,129.21) --
	( 96.90,129.26) --
	( 96.85,129.27) --
	( 96.78,129.25) --
	( 96.73,129.14) --
	( 96.59,129.06) --
	( 96.56,129.03) --
	( 96.52,128.97) --
	( 96.50,128.84) --
	( 96.47,128.84) --
	( 96.36,128.87) --
	( 96.17,128.97) --
	( 96.10,129.00) --
	( 96.00,128.95) --
	( 96.00,128.90) --
	( 95.98,128.71) --
	( 95.87,128.76) --
	( 95.79,128.81) --
	( 95.70,128.95) --
	( 95.66,128.98) --
	( 95.55,129.04) --
	( 95.51,129.09) --
	( 95.47,129.26) --
	( 95.42,129.33) --
	( 95.37,129.34) --
	( 95.30,129.32) --
	( 95.27,129.26) --
	( 95.22,129.14) --
	( 95.11,129.03) --
	( 95.02,129.00) --
	( 94.96,128.97) --
	( 94.91,128.98) --
	( 94.82,128.97) --
	( 94.75,129.04) --
	( 94.69,129.08) --
	( 94.64,129.26) --
	( 94.61,129.27) --
	( 94.57,129.26) --
	( 94.57,129.22) --
	( 94.59,129.02) --
	( 94.56,128.96) --
	( 94.41,128.91) --
	( 94.31,128.91) --
	( 94.24,128.94) --
	( 94.01,128.89) --
	( 93.93,128.83) --
	( 93.86,128.81) --
	( 93.81,128.81) --
	( 93.73,128.83) --
	( 93.65,128.88) --
	( 93.54,128.93) --
	( 93.42,128.93) --
	( 93.28,128.82) --
	( 93.24,128.77) --
	( 93.22,128.72) --
	( 93.19,128.67) --
	( 93.15,128.63) --
	( 93.09,128.64) --
	( 92.99,128.67) --
	( 92.93,128.69) --
	( 92.84,128.74) --
	( 92.75,128.79) --
	( 92.71,128.86) --
	( 92.68,128.97) --
	( 92.61,129.04) --
	( 92.56,129.03) --
	( 92.48,128.98) --
	( 92.34,128.90) --
	( 92.28,128.90) --
	( 92.21,128.91) --
	( 92.13,128.94) --
	( 92.00,129.06) --
	( 91.96,129.08) --
	( 91.93,129.06) --
	( 91.97,128.94) --
	( 91.87,128.91) --
	( 91.70,128.98) --
	( 91.57,129.05) --
	( 91.59,129.08) --
	( 91.72,129.37) --
	( 91.74,129.42) --
	( 91.69,129.42) --
	( 91.46,129.13) --
	( 91.36,129.09) --
	( 91.02,128.93) --
	( 90.90,128.91) --
	( 90.84,128.91) --
	( 90.73,128.92) --
	( 90.66,128.91) --
	( 90.69,128.83) --
	( 90.59,128.82) --
	( 90.53,128.80) --
	( 90.56,128.64) --
	( 90.51,128.60) --
	( 90.48,128.53) --
	( 90.50,128.42) --
	( 90.47,128.38) --
	( 90.41,128.36) --
	( 90.37,128.33) --
	( 90.32,128.32) --
	( 89.99,128.28) --
	( 89.95,128.26) --
	( 89.99,128.01) --
	( 89.99,127.63) --
	( 89.94,127.62) --
	( 89.86,127.63) --
	( 89.74,127.61) --
	( 89.73,127.65) --
	( 89.79,127.72) --
	( 89.80,127.78) --
	( 89.75,127.79) --
	( 89.70,127.77) --
	( 89.62,127.68) --
	( 89.52,127.63) --
	( 89.27,127.59) --
	( 89.23,127.53) --
	( 89.23,127.50) --
	( 89.23,127.43) --
	( 89.24,127.27) --
	( 89.22,127.05) --
	( 89.19,127.00) --
	( 89.15,126.96) --
	( 89.04,126.87) --
	( 88.98,126.85) --
	( 88.96,126.90) --
	( 88.98,126.91) --
	( 89.02,127.02) --
	( 89.05,127.10) --
	( 89.10,127.21) --
	( 89.10,127.39) --
	( 89.06,127.45) --
	( 89.00,127.44) --
	( 88.82,127.40) --
	( 88.72,127.40) --
	( 88.60,127.37) --
	( 88.53,127.37) --
	( 88.39,127.40) --
	( 88.30,127.42) --
	( 88.26,127.41) --
	( 88.23,127.35) --
	( 88.24,127.29) --
	( 88.30,127.23) --
	( 88.34,127.17) --
	( 88.40,126.89) --
	( 88.39,126.79) --
	( 88.41,126.71) --
	( 88.35,126.65) --
	( 88.31,126.64) --
	( 88.28,126.66) --
	( 88.25,126.69) --
	( 88.20,126.91) --
	( 88.17,127.05) --
	( 88.13,127.21) --
	( 88.10,127.24) --
	( 88.03,127.30) --
	( 87.96,127.30) --
	( 87.60,127.20) --
	( 87.60,127.16) --
	( 87.67,127.11) --
	( 87.67,127.03) --
	( 87.59,127.01) --
	( 87.51,127.00) --
	( 87.37,127.04) --
	( 87.23,127.05) --
	( 87.16,127.01) --
	( 87.13,126.99) --
	( 87.13,126.95) --
	( 87.14,126.90) --
	( 87.21,126.86) --
	( 87.21,126.79) --
	( 87.18,126.73) --
	( 87.14,126.68) --
	( 87.15,126.53) --
	( 87.15,126.33) --
	( 87.12,126.32) --
	( 87.07,126.36) --
	( 87.02,126.42) --
	( 86.91,126.69) --
	( 86.84,126.77) --
	( 86.82,126.82) --
	( 86.82,126.84) --
	( 86.90,126.95) --
	( 86.92,127.02) --
	( 86.92,127.06) --
	( 86.83,127.11) --
	( 86.73,127.14) --
	( 86.65,127.12) --
	( 86.55,127.13) --
	( 86.48,127.13) --
	( 86.43,127.13) --
	( 86.41,127.11) --
	( 86.32,127.10) --
	( 86.24,127.19) --
	( 86.09,127.18) --
	( 85.89,127.19) --
	( 85.78,127.18) --
	( 85.72,127.15) --
	( 85.71,127.10) --
	( 85.74,127.05) --
	( 85.76,126.98) --
	( 85.82,126.89) --
	( 85.99,126.74) --
	( 86.06,126.65) --
	( 86.07,126.59) --
	( 86.04,126.53) --
	( 85.99,126.49) --
	( 85.90,126.41) --
	( 85.84,126.36) --
	( 85.81,126.41) --
	( 85.80,126.47) --
	( 85.84,126.61) --
	( 85.80,126.71) --
	( 85.72,126.80) --
	( 85.62,126.87) --
	( 85.52,126.90) --
	( 85.48,126.90) --
	( 85.39,126.88) --
	( 85.35,126.86) --
	( 85.29,126.79) --
	( 85.29,126.76) --
	( 85.23,126.70) --
	( 85.13,126.72) --
	( 84.99,126.84) --
	( 84.97,126.87) --
	( 84.91,126.86) --
	( 84.88,126.83) --
	( 84.89,126.68) --
	( 84.86,126.63) --
	( 84.83,126.61) --
	( 84.74,126.63) --
	( 84.71,126.69) --
	( 84.65,126.80) --
	( 84.61,126.94) --
	( 84.57,127.04) --
	( 84.50,127.08) --
	( 84.46,127.07) --
	( 84.31,126.97) --
	( 84.27,126.94) --
	( 84.18,126.86) --
	( 84.10,126.81) --
	( 83.88,126.62) --
	( 83.77,126.55) --
	( 83.70,126.49) --
	( 83.71,126.46) --
	( 83.82,126.45) --
	( 83.87,126.42) --
	( 83.79,126.38) --
	( 83.79,126.31) --
	( 83.82,126.22) --
	( 83.86,126.14) --
	( 83.85,126.10) --
	( 83.96,126.04) --
	( 83.99,126.03) --
	( 83.97,125.85) --
	( 83.93,125.84) --
	( 83.86,125.82) --
	( 83.79,125.83) --
	( 83.65,125.92) --
	( 83.60,125.92) --
	( 83.56,125.90) --
	( 83.56,125.84) --
	( 83.59,125.73) --
	( 83.62,125.66) --
	( 83.77,125.60) --
	( 83.90,125.55) --
	( 84.13,125.48) --
	( 84.28,125.40) --
	( 84.39,125.38) --
	( 84.48,125.32) --
	( 84.55,125.24) --
	( 84.61,125.19) --
	( 84.66,125.17) --
	( 84.79,125.12) --
	( 84.86,125.09) --
	( 84.87,125.04) --
	( 84.86,125.01) --
	( 84.92,124.93) --
	( 85.03,124.87) --
	( 85.13,124.88) --
	( 85.15,124.87) --
	( 85.18,124.85) --
	( 85.20,124.83) --
	( 85.24,124.74) --
	( 85.38,124.63) --
	( 85.39,124.59) --
	( 85.45,124.55) --
	( 85.46,124.50) --
	( 85.49,124.47) --
	( 85.49,124.41) --
	( 85.48,124.39) --
	( 85.44,124.38) --
	( 85.34,124.45) --
	( 85.29,124.53) --
	( 85.24,124.56) --
	( 85.12,124.60) --
	( 85.03,124.61) --
	( 84.94,124.64) --
	( 84.91,124.72) --
	( 84.90,124.82) --
	( 84.90,124.84) --
	( 84.87,124.87) --
	( 84.79,124.93) --
	( 84.50,125.05) --
	( 84.46,125.08) --
	( 84.43,125.10) --
	( 84.38,125.15) --
	( 84.34,125.18) --
	( 84.28,125.22) --
	( 84.19,125.27) --
	( 84.08,125.34) --
	( 83.97,125.36) --
	( 83.91,125.36) --
	( 83.84,125.34) --
	( 83.80,125.31) --
	( 83.79,125.15) --
	( 83.75,125.15) --
	( 83.68,125.18) --
	( 83.64,125.38) --
	( 83.61,125.40) --
	( 83.43,125.42) --
	( 83.29,125.51) --
	( 83.24,125.58) --
	( 83.15,125.84) --
	( 83.11,125.88) --
	( 83.05,125.96) --
	( 82.94,126.05) --
	( 82.87,126.13) --
	( 82.82,126.13) --
	( 82.82,126.08) --
	( 82.98,125.86) --
	( 83.01,125.81) --
	( 83.05,125.63) --
	( 83.04,125.47) --
	( 82.96,125.26) --
	( 82.93,125.18) --
	( 82.87,125.12) --
	( 82.81,125.11) --
	( 82.71,125.12) --
	( 82.64,125.15) --
	( 82.53,125.17) --
	( 82.47,125.18) --
	( 82.43,125.17) --
	( 82.42,125.11) --
	( 82.47,125.02) --
	( 82.64,124.92) --
	( 82.72,124.87) --
	( 82.74,124.85) --
	( 82.76,124.82) --
	( 82.89,124.68) --
	( 82.95,124.63) --
	( 82.99,124.62) --
	( 83.05,124.58) --
	( 83.13,124.55) --
	( 83.15,124.52) --
	( 83.18,124.44) --
	( 83.16,124.39) --
	( 83.18,124.32) --
	( 83.26,124.28) --
	( 83.36,124.21) --
	( 83.40,124.13) --
	( 83.40,124.07) --
	( 83.36,124.03) --
	( 83.31,124.05) --
	( 83.10,124.20) --
	( 83.06,124.23) --
	( 82.98,124.23) --
	( 82.86,124.25) --
	( 82.76,124.23) --
	( 82.78,124.34) --
	( 82.72,124.43) --
	( 82.65,124.51) --
	( 82.55,124.60) --
	( 82.38,124.67) --
	( 82.26,124.73) --
	( 82.09,124.83) --
	( 82.06,124.85) --
	( 82.04,124.87) --
	( 81.97,124.94) --
	( 82.00,124.97) --
	( 81.99,125.02) --
	( 81.96,125.07) --
	( 81.95,125.11) --
	( 81.92,125.19) --
	( 81.91,125.25) --
	( 81.95,125.36) --
	( 81.94,125.44) --
	( 81.86,125.46) --
	( 81.82,125.51) --
	( 81.78,125.56) --
	( 81.74,125.59) --
	( 81.70,125.58) --
	( 81.64,125.51) --
	( 81.55,125.41) --
	( 81.48,125.32) --
	( 81.48,125.26) --
	( 81.46,125.05) --
	( 81.43,125.01) --
	( 81.36,125.03) --
	( 81.29,125.03) --
	( 81.26,125.02) --
	( 81.22,125.00) --
	( 81.21,124.96) --
	( 81.22,124.94) --
	( 81.22,124.92) --
	( 81.24,124.87) --
	( 81.26,124.84) --
	( 81.27,124.83) --
	( 81.35,124.70) --
	( 81.46,124.55) --
	( 81.64,124.32) --
	( 81.70,124.28) --
	( 81.65,124.25) --
	( 81.55,124.28) --
	( 81.38,124.34) --
	( 81.31,124.35) --
	( 81.19,124.34) --
	( 81.16,124.40) --
	( 81.13,124.44) --
	( 81.09,124.47) --
	( 80.93,124.54) --
	( 80.92,124.60) --
	( 80.95,124.65) --
	( 80.90,124.71) --
	( 80.83,124.83) --
	( 80.82,124.85) --
	( 80.81,124.88) --
	( 80.75,125.01) --
	( 80.71,125.03) --
	( 80.67,125.05) --
	( 80.62,125.06) --
	( 80.60,125.06) --
	( 80.52,125.02) --
	( 80.48,125.00) --
	( 80.44,125.00) --
	( 80.38,124.98) --
	( 80.32,124.96) --
	( 80.30,124.95) --
	( 80.25,124.94) --
	( 80.20,124.92) --
	( 80.15,124.89) --
	( 80.10,124.88) --
	( 80.04,124.85) --
	( 80.01,124.83) --
	( 79.85,124.74) --
	( 79.80,124.70) --
	( 79.67,124.69) --
	( 79.55,124.65) --
	( 79.47,124.65) --
	( 79.38,124.62) --
	( 79.28,124.65) --
	( 79.18,124.64) --
	( 79.11,124.60) --
	( 79.01,124.59) --
	( 78.90,124.59) --
	( 78.81,124.59) --
	( 78.70,124.59) --
	( 78.66,124.56) --
	( 78.64,124.51) --
	( 78.74,124.46) --
	( 78.83,124.42) --
	( 78.88,124.35) --
	( 79.00,124.30) --
	( 79.08,124.26) --
	( 79.10,124.28) --
	( 79.15,124.27) --
	( 79.18,124.22) --
	( 79.29,124.20) --
	( 79.35,124.16) --
	( 79.46,124.16) --
	( 79.46,124.12) --
	( 79.51,124.00) --
	( 79.58,123.98) --
	( 79.63,123.98) --
	( 79.70,123.97) --
	( 79.72,123.93) --
	( 79.80,123.76) --
	( 79.84,123.72) --
	( 79.92,123.70) --
	( 79.94,123.68) --
	( 80.01,123.64) --
	( 80.04,123.62) --
	( 80.11,123.61) --
	( 80.18,123.61) --
	( 80.34,123.58) --
	( 80.41,123.55) --
	( 80.48,123.46) --
	( 80.54,123.43) --
	( 80.62,123.36) --
	( 80.74,123.27) --
	( 80.79,123.24) --
	( 80.81,123.19) --
	( 80.83,123.13) --
	( 80.85,123.05) --
	( 80.87,122.97) --
	( 80.87,122.92) --
	( 80.93,122.87) --
	( 80.97,122.83) --
	( 80.97,122.76) --
	( 81.00,122.71) --
	( 81.07,122.64) --
	( 81.08,122.63) --
	( 81.14,122.63) --
	( 81.18,122.64) --
	( 81.25,122.63) --
	( 81.32,122.54) --
	( 81.38,122.53) --
	( 81.44,122.50) --
	( 81.57,122.48) --
	( 81.68,122.47) --
	( 81.72,122.48) --
	( 81.78,122.44) --
	( 81.85,122.37) --
	( 81.86,122.27) --
	( 81.90,122.20) --
	( 81.99,122.15) --
	( 82.12,122.13) --
	( 82.31,122.08) --
	( 82.37,122.04) --
	( 82.40,122.01) --
	( 82.47,121.89) --
	( 82.55,121.81) --
	( 82.64,121.77) --
	( 82.76,121.72) --
	( 82.85,121.67) --
	( 82.86,121.59) --
	( 82.83,121.56) --
	( 82.82,121.54) --
	( 82.69,121.50) --
	( 82.62,121.59) --
	( 82.57,121.64) --
	( 82.52,121.68) --
	( 82.45,121.71) --
	( 82.38,121.76) --
	( 82.25,121.79) --
	( 82.17,121.83) --
	( 82.11,121.85) --
	( 82.06,121.90) --
	( 81.96,121.92) --
	( 81.83,121.96) --
	( 81.75,122.05) --
	( 81.72,122.18) --
	( 81.72,122.29) --
	( 81.68,122.33) --
	( 81.62,122.36) --
	( 81.52,122.34) --
	( 81.35,122.31) --
	( 81.25,122.31) --
	( 81.18,122.31) --
	( 81.14,122.33) --
	( 81.05,122.39) --
	( 80.98,122.42) --
	( 80.93,122.46) --
	( 80.86,122.48) --
	( 80.73,122.44) --
	( 80.64,122.43) --
	( 80.61,122.46) --
	( 80.61,122.49) --
	( 80.64,122.52) --
	( 80.70,122.56) --
	( 80.73,122.59) --
	( 80.73,122.67) --
	( 80.72,122.72) --
	( 80.65,122.78) --
	( 80.62,122.80) --
	( 80.60,122.88) --
	( 80.69,122.95) --
	( 80.69,123.03) --
	( 80.65,123.10) --
	( 80.59,123.14) --
	( 80.52,123.19) --
	( 80.44,123.24) --
	( 80.44,123.31) --
	( 80.41,123.31) --
	( 80.32,123.31) --
	( 80.18,123.29) --
	( 80.07,123.30) --
	( 80.01,123.32) --
	( 79.98,123.36) --
	( 79.97,123.40) --
	( 79.94,123.44) --
	( 79.86,123.45) --
	( 79.80,123.52) --
	( 79.74,123.58) --
	( 79.63,123.69) --
	( 79.42,123.83) --
	( 79.29,123.89) --
	( 79.26,123.94) --
	( 79.19,123.91) --
	( 79.14,123.82) --
	( 79.13,123.73) --
	( 79.15,123.66) --
	( 79.11,123.62) --
	( 79.09,123.56) --
	( 79.01,123.53) --
	( 78.89,123.53) --
	( 78.88,123.49) --
	( 78.83,123.44) --
	( 78.81,123.43) --
	( 78.69,123.44) --
	( 78.58,123.43) --
	( 78.44,123.42) --
	( 78.37,123.43) --
	( 78.31,123.45) --
	( 78.29,123.49) --
	( 78.28,123.51) --
	( 78.27,123.53) --
	( 78.14,123.56) --
	( 78.04,123.61) --
	( 77.91,123.65) --
	( 77.78,123.66) --
	( 77.74,123.62) --
	( 77.74,123.54) --
	( 77.78,123.45) --
	( 77.84,123.42) --
	( 77.87,123.39) --
	( 77.90,123.35) --
	( 77.91,123.34) --
	( 77.91,123.29) --
	( 77.81,123.26) --
	( 77.64,123.24) --
	( 77.53,123.21) --
	( 77.45,123.15) --
	( 77.33,123.17) --
	( 77.26,123.17) --
	( 77.20,123.20) --
	( 77.13,123.26) --
	( 77.04,123.32) --
	( 76.73,123.21) --
	( 76.49,123.20) --
	( 76.19,123.19) --
	( 76.29,123.35) --
	( 76.33,123.45) --
	( 76.42,123.99) --
	( 76.41,124.09) --
	( 76.40,124.20) --
	( 76.45,124.29) --
	( 76.47,124.37) --
	( 76.47,124.41) --
	( 76.44,124.49) --
	( 76.43,124.53) --
	( 76.31,124.61) --
	( 76.16,124.65) --
	( 75.96,124.70) --
	( 75.75,124.73) --
	( 75.63,124.82) --
	( 75.60,124.84) --
	( 75.58,124.86) --
	( 75.36,125.05) --
	( 75.31,125.07) --
	( 75.28,125.10) --
	( 75.24,125.13) --
	( 75.19,125.16) --
	( 75.15,125.19) --
	( 75.14,125.21) --
	( 75.12,125.22) --
	( 75.07,125.26) --
	( 74.97,125.33) --
	( 74.91,125.39) --
	( 74.88,125.40) --
	( 74.83,125.37) --
	( 74.80,125.36) --
	( 74.77,125.36) --
	( 74.76,125.39) --
	( 74.74,125.45) --
	( 74.68,125.52) --
	( 74.64,125.57) --
	( 74.57,125.60) --
	( 74.50,125.61) --
	( 74.47,125.67) --
	( 74.64,125.80) --
	( 74.71,125.82) --
	( 74.79,125.84) --
	( 74.81,125.82) --
	( 74.97,125.69) --
	( 75.00,125.68) --
	( 75.03,125.71) --
	( 75.10,125.84) --
	( 75.13,125.87) --
	( 75.16,125.89) --
	( 75.21,125.91) --
	( 75.28,125.91) --
	( 75.30,125.90) --
	( 75.34,125.82) --
	( 75.38,125.81) --
	( 75.40,125.80) --
	( 75.43,125.77) --
	( 75.43,125.75) --
	( 75.31,125.61) --
	( 75.30,125.56) --
	( 75.31,125.54) --
	( 75.52,125.52) --
	( 75.72,125.54) --
	( 75.78,125.54) --
	( 75.88,125.61) --
	( 75.91,125.63) --
	( 75.94,125.62) --
	( 75.96,125.50) --
	( 76.09,125.38) --
	( 76.18,125.35) --
	( 76.20,125.32) --
	( 76.25,125.31) --
	( 76.30,125.30) --
	( 76.33,125.32) --
	( 76.38,125.33) --
	( 76.43,125.38) --
	( 76.46,125.39) --
	( 76.50,125.40) --
	( 76.53,125.39) --
	( 76.55,125.35) --
	( 76.57,125.29) --
	( 76.59,125.27) --
	( 76.77,125.22) --
	( 76.84,125.22) --
	( 76.90,125.23) --
	( 76.94,125.31) --
	( 76.97,125.32) --
	( 77.02,125.34) --
	( 77.06,125.34) --
	( 77.11,125.34) --
	( 77.16,125.32) --
	( 77.17,125.26) --
	( 77.16,125.23) --
	( 77.11,125.14) --
	( 77.13,125.09) --
	( 77.17,125.00) --
	( 77.21,124.91) --
	( 77.30,124.88) --
	( 77.37,124.92) --
	( 77.43,125.08) --
	( 77.58,125.40) --
	( 77.60,125.44) --
	( 77.57,125.49) --
	( 77.54,125.54) --
	( 77.50,125.60) --
	( 77.47,125.64) --
	( 77.42,125.76) --
	( 77.39,125.79) --
	( 77.40,125.84) --
	( 77.43,125.93) --
	( 77.50,126.08) --
	( 77.50,126.12) --
	( 77.49,126.17) --
	( 77.43,126.19) --
	( 77.38,126.21) --
	( 77.33,126.25) --
	( 77.31,126.31) --
	( 77.26,126.37) --
	( 77.22,126.38) --
	( 77.18,126.40) --
	( 77.14,126.45) --
	( 77.05,126.51) --
	( 77.00,126.57) --
	( 76.98,126.62) --
	( 77.00,126.65) --
	( 77.10,126.63) --
	( 77.15,126.65) --
	( 77.17,126.70) --
	( 77.05,126.84) --
	( 77.03,126.90) --
	( 77.01,126.98) --
	( 77.00,127.02) --
	( 77.01,127.10) --
	( 77.03,127.13) --
	( 77.07,127.10) --
	( 77.23,126.92) --
	( 77.32,126.78) --
	( 77.41,126.80) --
	( 77.47,126.60) --
	( 77.52,126.56) --
	( 77.56,126.55) --
	( 77.62,126.54) --
	( 77.66,126.56) --
	( 77.69,126.57) --
	( 77.74,126.60) --
	( 77.79,126.63) --
	( 77.88,126.67) --
	( 77.94,126.69) --
	( 77.96,126.68) --
	( 77.98,126.66) --
	( 77.91,126.58) --
	( 77.88,126.52) --
	( 77.86,126.40) --
	( 77.86,126.28) --
	( 77.88,126.25) --
	( 77.92,126.23) --
	( 77.99,126.24) --
	( 78.02,126.24) --
	( 78.05,126.23) --
	( 78.07,126.20) --
	( 78.06,126.18) --
	( 78.04,126.12) --
	( 78.04,126.11) --
	( 78.04,126.07) --
	( 78.02,126.02) --
	( 77.99,126.01) --
	( 77.93,126.01) --
	( 77.92,125.99) --
	( 77.93,125.94) --
	( 77.93,125.91) --
	( 77.97,125.90) --
	( 78.09,125.84) --
	( 78.18,125.82) --
	( 78.30,125.81) --
	( 78.36,125.82) --
	( 78.40,125.83) --
	( 78.43,125.84) --
	( 78.47,125.85) --
	( 78.54,125.87) --
	( 78.63,125.93) --
	( 78.67,125.98) --
	( 78.67,126.02) --
	( 78.71,126.10) --
	( 78.75,126.14) --
	( 78.80,126.12) --
	( 78.81,126.10) --
	( 78.86,126.06) --
	( 78.88,126.04) --
	( 78.94,126.05) --
	( 78.97,126.04) --
	( 79.07,125.89) --
	( 79.10,125.84) --
	( 79.12,125.81) --
	( 79.12,125.78) --
	( 79.14,125.74) --
	( 79.11,125.68) --
	( 79.05,125.59) --
	( 78.99,125.54) --
	( 78.95,125.51) --
	( 78.81,125.41) --
	( 78.75,125.37) --
	( 78.68,125.34) --
	( 78.65,125.29) --
	( 78.62,125.22) --
	( 78.59,125.16) --
	( 78.53,125.02) --
	( 78.53,125.00) --
	( 78.58,124.99) --
	( 78.69,124.99) --
	( 78.73,125.00) --
	( 78.82,125.04) --
	( 78.89,125.12) --
	( 78.95,125.13) --
	( 78.98,125.12) --
	( 79.00,125.02) --
	( 79.07,124.99) --
	( 79.13,124.98) --
	( 79.26,125.00) --
	( 79.39,124.97) --
	( 79.49,124.97) --
	( 79.60,124.97) --
	( 79.63,124.98) --
	( 79.67,125.01) --
	( 79.76,125.05) --
	( 79.81,125.09) --
	( 79.89,125.15) --
	( 79.97,125.21) --
	( 80.01,125.24) --
	( 80.09,125.26) --
	( 80.17,125.29) --
	( 80.24,125.30) --
	( 80.29,125.33) --
	( 80.33,125.37) --
	( 80.34,125.40) --
	( 80.33,125.45) --
	( 80.29,125.56) --
	( 80.25,125.62) --
	( 80.17,125.63) --
	( 80.16,125.65) --
	( 80.13,125.71) --
	( 80.12,125.73) --
	( 80.08,125.73) --
	( 80.00,125.73) --
	( 79.92,125.74) --
	( 79.91,125.77) --
	( 79.92,125.88) --
	( 79.91,125.90) --
	( 79.75,125.88) --
	( 79.71,125.89) --
	( 79.69,125.92) --
	( 79.69,125.95) --
	( 79.71,125.99) --
	( 79.74,126.02) --
	( 79.82,126.09) --
	( 79.84,126.12) --
	( 79.80,126.20) --
	( 79.77,126.27) --
	( 79.78,126.30) --
	( 79.77,126.36) --
	( 79.77,126.39) --
	( 79.85,126.41) --
	( 79.88,126.45) --
	( 79.87,126.48) --
	( 79.84,126.58) --
	( 79.84,126.60) --
	( 79.86,126.63) --
	( 79.88,126.63) --
	( 79.97,126.55) --
	( 80.01,126.47) --
	( 80.09,126.30) --
	( 80.14,126.27) --
	( 80.18,126.24) --
	( 80.22,126.20) --
	( 80.25,126.18) --
	( 80.28,126.13) --
	( 80.36,125.96) --
	( 80.53,125.82) --
	( 80.70,125.66) --
	( 80.76,125.62) --
	( 80.83,125.46) --
	( 80.84,125.41) --
	( 80.88,125.40) --
	( 80.93,125.40) --
	( 81.13,125.61) --
	( 81.16,125.67) --
	( 81.11,125.69) --
	( 80.94,125.66) --
	( 80.86,125.70) --
	( 80.73,125.78) --
	( 80.55,125.93) --
	( 80.49,126.11) --
	( 80.48,126.15) --
	( 80.42,126.22) --
	( 80.38,126.32) --
	( 80.27,126.36) --
	( 80.21,126.39) --
	( 80.18,126.42) --
	( 80.20,126.47) --
	( 80.21,126.58) --
	( 80.21,126.61) --
	( 80.15,126.66) --
	( 80.08,126.71) --
	( 80.06,126.78) --
	( 80.06,126.82) --
	( 80.09,126.87) --
	( 80.09,126.95) --
	( 80.09,126.98) --
	( 80.07,127.02) --
	( 80.06,127.08) --
	( 80.07,127.11) --
	( 80.12,127.12) --
	( 80.17,127.09) --
	( 80.20,127.06) --
	( 80.23,126.99) --
	( 80.27,126.90) --
	( 80.34,126.81) --
	( 80.38,126.75) --
	( 80.47,126.65) --
	( 80.55,126.56) --
	( 80.67,126.47) --
	( 80.71,126.45) --
	( 80.75,126.47) --
	( 80.79,126.50) --
	( 80.77,126.61) --
	( 80.77,126.63) --
	( 80.75,126.67) --
	( 80.68,126.70) --
	( 80.60,126.79) --
	( 80.58,126.83) --
	( 80.54,126.90) --
	( 80.50,126.95) --
	( 80.47,126.99) --
	( 80.46,127.05) --
	( 80.47,127.07) --
	( 80.50,127.08) --
	( 80.54,127.06) --
	( 80.58,127.04) --
	( 80.64,126.98) --
	( 80.70,126.92) --
	( 80.73,126.89) --
	( 80.77,126.90) --
	( 80.81,126.90) --
	( 80.81,126.96) --
	( 80.83,126.98) --
	( 80.87,126.99) --
	( 80.90,126.99) --
	( 80.94,126.98) --
	( 81.01,126.92) --
	( 81.11,126.92) --
	( 81.15,126.90) --
	( 81.27,126.80) --
	( 81.28,126.80) --
	( 81.32,126.81) --
	( 81.32,126.82) --
	( 81.34,126.83) --
	( 81.34,126.86) --
	( 81.25,126.93) --
	( 81.20,127.00) --
	( 81.14,127.10) --
	( 81.13,127.13) --
	( 81.11,127.15) --
	( 81.24,127.19) --
	( 81.26,127.24) --
	( 81.27,127.26) --
	( 81.30,127.27) --
	( 81.34,127.27) --
	( 81.37,127.22) --
	( 81.44,127.14) --
	( 81.50,127.02) --
	( 81.52,126.96) --
	( 81.55,126.90) --
	( 81.58,126.83) --
	( 81.61,126.73) --
	( 81.63,126.71) --
	( 81.68,126.67) --
	( 81.79,126.62) --
	( 81.89,126.59) --
	( 81.99,126.57) --
	( 82.05,126.55) --
	( 82.09,126.51) --
	( 82.09,126.49) --
	( 82.07,126.47) --
	( 82.09,126.46) --
	( 81.96,126.40) --
	( 81.93,126.39) --
	( 81.94,126.36) --
	( 81.96,126.31) --
	( 82.10,126.10) --
	( 82.39,125.69) --
	( 82.45,125.68) --
	( 82.48,125.69) --
	( 82.49,125.72) --
	( 82.37,125.92) --
	( 82.34,125.97) --
	( 82.35,126.03) --
	( 82.46,126.29) --
	( 82.46,126.34) --
	( 82.41,126.38) --
	( 82.39,126.41) --
	( 82.26,126.44) --
	( 82.20,126.48) --
	( 82.16,126.57) --
	( 82.10,126.70) --
	( 81.98,126.84) --
	( 81.95,126.88) --
	( 81.95,126.92) --
	( 81.96,126.99) --
	( 81.97,127.02) --
	( 81.91,127.08) --
	( 81.87,127.20) --
	( 81.81,127.25) --
	( 81.77,127.32) --
	( 81.76,127.44) --
	( 81.74,127.48) --
	( 81.71,127.52) --
	( 81.63,127.59) --
	( 81.59,127.59) --
	( 81.52,127.58) --
	( 81.40,127.56) --
	( 81.33,127.56) --
	( 81.30,127.58) --
	( 81.27,127.59) --
	( 81.27,127.64) --
	( 81.30,127.65) --
	( 81.46,127.67) --
	( 81.50,127.69) --
	( 81.53,127.76) --
	( 81.52,127.85) --
	( 81.60,127.96) --
	( 81.62,128.01) --
	( 81.60,128.05) --
	( 81.58,128.08) --
	( 81.45,128.07) --
	( 81.18,128.02) --
	( 81.13,128.02) --
	( 81.11,128.05) --
	( 81.11,128.07) --
	( 81.25,128.15) --
	( 81.42,128.25) --
	( 81.42,128.27) --
	( 81.39,128.33) --
	( 81.32,128.40) --
	( 81.25,128.43) --
	( 81.21,128.50) --
	( 81.18,128.56) --
	( 81.14,128.60) --
	( 80.88,128.71) --
	( 80.77,128.76) --
	( 80.66,128.81) --
	( 80.52,128.86) --
	( 80.39,128.88) --
	( 80.26,128.92) --
	( 80.17,128.94) --
	( 80.10,128.96) --
	( 79.95,129.03) --
	( 79.78,129.10) --
	( 79.75,129.12) --
	( 79.75,129.18) --
	( 79.79,129.18) --
	( 79.88,129.18) --
	( 79.96,129.15) --
	( 80.10,129.10) --
	( 80.16,129.05) --
	( 80.26,128.99) --
	( 80.35,128.96) --
	( 80.45,128.97) --
	( 80.56,128.99) --
	( 80.62,128.98) --
	( 80.72,128.94) --
	( 80.77,128.91) --
	( 80.87,128.86) --
	( 80.93,128.85) --
	( 80.99,128.85) --
	( 81.03,128.88) --
	( 81.07,128.97) --
	( 81.09,129.05) --
	( 81.12,129.11) --
	( 81.15,129.18) --
	( 81.19,129.27) --
	( 81.20,129.30) --
	( 81.25,129.36) --
	( 81.33,129.42) --
	( 81.42,129.27) --
	( 81.42,129.23) --
	( 81.33,129.16) --
	( 81.20,128.90) --
	( 81.21,128.87) --
	( 81.21,128.82) --
	( 81.31,128.75) --
	( 81.36,128.71) --
	( 81.38,128.65) --
	( 81.43,128.57) --
	( 81.48,128.51) --
	( 81.55,128.51) --
	( 81.67,128.62) --
	( 81.75,128.74) --
	( 81.84,128.82) --
	( 81.88,128.85) --
	( 81.94,128.87) --
	( 82.01,128.88) --
	( 82.04,128.88) --
	( 82.07,128.86) --
	( 82.08,128.84) --
	( 82.07,128.81) --
	( 81.97,128.72) --
	( 81.88,128.63) --
	( 81.85,128.56) --
	( 81.81,128.52) --
	( 81.77,128.48) --
	( 81.68,128.44) --
	( 81.67,128.42) --
	( 81.67,128.37) --
	( 81.66,128.26) --
	( 81.68,128.24) --
	( 81.73,128.24) --
	( 81.83,128.26) --
	( 82.06,128.34) --
	( 82.09,128.34) --
	( 82.12,128.32) --
	( 82.09,128.29) --
	( 81.96,128.17) --
	( 81.89,128.12) --
	( 81.85,128.07) --
	( 81.83,128.00) --
	( 81.83,127.92) --
	( 81.85,127.88) --
	( 81.86,127.87) --
	( 81.90,127.84) --
	( 82.02,127.84) --
	( 82.06,127.87) --
	( 82.17,127.91) --
	( 82.27,127.97) --
	( 82.33,127.97) --
	( 82.34,127.97) --
	( 82.36,127.94) --
	( 82.30,127.88) --
	( 82.26,127.83) --
	( 82.17,127.77) --
	( 82.19,127.75) --
	( 82.21,127.76) --
	( 82.24,127.74) --
	( 82.27,127.71) --
	( 82.26,127.68) --
	( 82.14,127.59) --
	( 82.13,127.55) --
	( 82.14,127.54) --
	( 82.18,127.53) --
	( 82.30,127.59) --
	( 82.33,127.59) --
	( 82.33,127.55) --
	( 82.20,127.34) --
	( 82.21,127.31) --
	( 82.22,127.29) --
	( 82.25,127.29) --
	( 82.35,127.34) --
	( 82.47,127.42) --
	( 82.54,127.45) --
	( 82.60,127.48) --
	( 82.65,127.48) --
	( 82.68,127.48) --
	( 82.72,127.45) --
	( 82.75,127.43) --
	( 82.72,127.37) --
	( 82.58,127.25) --
	( 82.49,127.17) --
	( 82.48,127.13) --
	( 82.41,126.99) --
	( 82.42,126.97) --
	( 82.46,126.96) --
	( 82.71,127.02) --
	( 82.75,127.01) --
	( 82.76,126.99) --
	( 82.76,126.97) --
	( 82.62,126.83) --
	( 82.57,126.78) --
	( 82.56,126.71) --
	( 82.57,126.69) --
	( 82.57,126.68) --
	( 82.60,126.66) --
	( 82.63,126.66) --
	( 82.76,126.79) --
	( 82.82,126.82) --
	( 82.85,126.82) --
	( 82.89,126.82) --
	( 82.90,126.80) --
	( 82.92,126.75) --
	( 82.87,126.63) --
	( 82.87,126.59) --
	( 82.90,126.59) --
	( 83.01,126.59) --
	( 83.04,126.56) --
	( 83.06,126.53) --
	( 83.08,126.51) --
	( 83.13,126.50) --
	( 83.14,126.46) --
	( 83.18,126.38) --
	( 83.20,126.38) --
	( 83.24,126.42) --
	( 83.31,126.59) --
	( 83.36,126.67) --
	( 83.40,126.73) --
	( 83.49,126.86) --
	( 83.57,126.98) --
	( 83.62,127.05) --
	( 83.60,127.09) --
	( 83.59,127.11) --
	( 83.55,127.14) --
	( 83.42,127.21) --
	( 83.40,127.22) --
	( 83.39,127.28) --
	( 83.41,127.37) --
	( 83.39,127.50) --
	( 83.35,127.65) --
	( 83.37,127.66) --
	( 83.42,127.67) --
	( 83.47,127.65) --
	( 83.54,127.57) --
	( 83.61,127.51) --
	( 83.65,127.45) --
	( 83.68,127.41) --
	( 83.73,127.32) --
	( 83.79,127.24) --
	( 83.83,127.22) --
	( 83.87,127.21) --
	( 83.93,127.22) --
	( 83.99,127.26) --
	( 84.02,127.28) --
	( 83.99,127.38) --
	( 84.00,127.41) --
	( 84.02,127.44) --
	( 84.13,127.52) --
	( 84.13,127.54) --
	( 84.09,127.64) --
	( 84.08,127.66) --
	( 84.21,127.73) --
	( 84.24,127.76) --
	( 84.24,127.79) --
	( 84.19,127.82) --
	( 84.14,127.86) --
	( 84.09,127.89) --
	( 84.07,127.97) --
	( 84.02,127.98) --
	( 83.97,128.00) --
	( 83.85,128.05) --
	( 83.84,128.07) --
	( 83.86,128.09) --
	( 83.86,128.19) --
	( 83.87,128.24) --
	( 83.84,128.38) --
	( 83.83,128.41) --
	( 83.79,128.48) --
	( 83.73,128.53) --
	( 83.68,128.57) --
	( 83.60,128.61) --
	( 83.51,128.65) --
	( 83.48,128.68) --
	( 83.49,128.72) --
	( 83.52,128.72) --
	( 83.58,128.73) --
	( 83.62,128.71) --
	( 83.68,128.65) --
	( 83.69,128.64) --
	( 83.75,128.65) --
	( 83.78,128.66) --
	( 83.79,128.68) --
	( 83.78,128.73) --
	( 83.76,128.75) --
	( 83.68,128.82) --
	( 83.58,128.90) --
	( 83.59,128.92) --
	( 83.65,128.99) --
	( 83.68,129.05) --
	( 83.70,129.14) --
	( 83.71,129.18) --
	( 83.77,129.22) --
	( 83.78,129.21) --
	( 83.82,129.15) --
	( 83.84,129.09) --
	( 83.84,129.01) --
	( 83.91,128.94) --
	( 83.94,128.89) --
	( 83.92,128.85) --
	( 83.95,128.80) --
	( 84.02,128.76) --
	( 84.05,128.73) --
	( 84.03,128.63) --
	( 84.06,128.50) --
	( 84.10,128.35) --
	( 84.13,128.29) --
	( 84.17,128.23) --
	( 84.21,128.22) --
	( 84.26,128.22) --
	( 84.31,128.24) --
	( 84.36,128.24) --
	( 84.38,128.23) --
	( 84.38,128.12) --
	( 84.38,128.02) --
	( 84.41,127.96) --
	( 84.47,127.94) --
	( 84.52,128.02) --
	( 84.57,128.11) --
	( 84.63,128.20) --
	( 84.70,128.25) --
	( 84.73,128.33) --
	( 84.73,128.42) --
	( 84.71,128.52) --
	( 84.71,128.59) --
	( 84.72,128.65) --
	( 84.73,128.70) --
	( 84.76,128.74) --
	( 84.80,128.75) --
	( 84.82,128.73) --
	( 84.83,128.68) --
	( 84.83,128.58) --
	( 84.84,128.45) --
	( 84.83,128.30) --
	( 84.80,128.20) --
	( 84.75,128.05) --
	( 84.75,128.04) --
	( 84.78,128.04) --
	( 84.88,128.08) --
	( 84.95,128.13) --
	( 84.99,128.16) --
	( 85.02,128.24) --
	( 85.05,128.26) --
	( 85.18,128.26) --
	( 85.45,128.39) --
	( 85.48,128.39) --
	( 85.50,128.37) --
	( 85.51,128.34) --
	( 85.49,128.28) --
	( 85.47,128.26) --
	( 85.41,128.24) --
	( 85.34,128.20) --
	( 85.27,128.18) --
	( 85.21,128.15) --
	( 85.17,128.09) --
	( 85.14,128.01) --
	( 85.08,127.98) --
	( 85.05,127.97) --
	( 84.99,127.96) --
	( 84.96,127.93) --
	( 84.92,127.89) --
	( 84.85,127.84) --
	( 84.78,127.80) --
	( 84.76,127.78) --
	( 84.88,127.72) --
	( 84.89,127.70) --
	( 84.89,127.69) --
	( 84.70,127.68) --
	( 84.63,127.68) --
	( 84.59,127.66) --
	( 84.53,127.57) --
	( 84.49,127.54) --
	( 84.49,127.52) --
	( 84.52,127.43) --
	( 84.55,127.41) --
	( 84.59,127.40) --
	( 84.76,127.41) --
	( 84.77,127.40) --
	( 84.76,127.32) --
	( 84.76,127.29) --
	( 84.79,127.24) --
	( 84.87,127.21) --
	( 85.10,127.10) --
	( 85.15,127.10) --
	( 85.34,127.25) --
	( 85.47,127.40) --
	( 85.47,127.45) --
	( 85.50,127.53) --
	( 85.54,127.63) --
	( 85.57,127.66) --
	( 85.60,127.66) --
	( 85.63,127.65) --
	( 85.69,127.53) --
	( 85.71,127.53) --
	( 85.76,127.52) --
	( 85.86,127.57) --
	( 86.04,127.62) --
	( 86.07,127.63) --
	( 86.04,127.86) --
	( 86.07,127.87) --
	( 86.10,127.88) --
	( 86.23,127.78) --
	( 86.26,127.78) --
	( 86.31,127.80) --
	( 86.37,127.86) --
	( 86.44,127.91) --
	( 86.47,127.93) --
	( 86.53,127.93) --
	( 86.56,127.89) --
	( 86.60,127.90) --
	( 86.83,128.06) --
	( 86.86,128.06) --
	( 86.89,128.03) --
	( 86.89,128.00) --
	( 86.83,127.91) --
	( 86.67,127.80) --
	( 86.70,127.78) --
	( 86.74,127.76) --
	( 86.77,127.71) --
	( 86.54,127.55) --
	( 86.42,127.46) --
	( 86.42,127.43) --
	( 86.42,127.41) --
	( 86.44,127.40) --
	( 86.51,127.41) --
	( 86.61,127.47) --
	( 86.65,127.49) --
	( 86.81,127.43) --
	( 87.06,127.41) --
	( 87.15,127.34) --
	( 87.26,127.44) --
	( 87.27,127.61) --
	( 87.27,127.71) --
	( 87.31,127.83) --
	( 87.34,127.89) --
	( 87.38,127.90) --
	( 87.40,127.86) --
	( 87.41,127.80) --
	( 87.38,127.49) --
	( 87.39,127.42) --
	( 87.41,127.42) --
	( 87.48,127.49) --
	( 87.52,127.52) --
	( 87.84,127.57) --
	( 87.86,127.58) --
	( 87.91,127.61) --
	( 87.94,127.66) --
	( 87.95,127.78) --
	( 87.97,127.84) --
	( 88.01,127.86) --
	( 88.05,127.87) --
	( 88.08,127.85) --
	( 88.06,127.72) --
	( 88.08,127.70) --
	( 88.11,127.67) --
	( 88.16,127.66) --
	( 88.21,127.65) --
	( 88.24,127.70) --
	( 88.29,127.78) --
	( 88.32,127.79) --
	( 88.48,127.72) --
	( 88.52,127.72) --
	( 88.55,127.73) --
	( 88.56,127.75) --
	( 88.55,127.92) --
	( 88.57,127.95) --
	( 88.61,127.96) --
	( 88.65,127.96) --
	( 88.67,127.94) --
	( 88.68,127.85) --
	( 88.69,127.73) --
	( 88.72,127.71) --
	( 88.78,127.72) --
	( 88.86,127.83) --
	( 88.92,127.91) --
	( 88.95,127.93) --
	( 88.99,127.96) --
	( 89.07,127.98) --
	( 89.33,127.94) --
	( 89.39,127.94) --
	( 89.42,127.97) --
	( 89.45,128.00) --
	( 89.45,128.07) --
	( 89.44,128.19) --
	( 89.44,128.28) --
	( 89.40,128.42) --
	( 89.37,128.50) --
	( 89.37,128.54) --
	( 89.39,128.57) --
	( 89.43,128.60) --
	( 89.44,128.57) --
	( 89.50,128.48) --
	( 89.51,128.46) --
	( 89.54,128.46) --
	( 89.57,128.46) --
	( 89.66,128.54) --
	( 89.70,128.57) --
	( 89.74,128.57) --
	( 89.81,128.57) --
	( 90.00,128.52) --
	( 90.04,128.52) --
	( 90.08,128.52) --
	( 90.10,128.54) --
	( 90.13,128.60) --
	( 90.11,128.63) --
	( 90.11,128.67) --
	( 90.11,128.70) --
	( 90.13,128.76) --
	( 90.16,128.82) --
	( 90.16,128.90) --
	( 90.16,128.96) --
	( 90.15,128.97) --
	( 90.09,129.03) --
	( 89.99,129.10) --
	( 89.96,129.14) --
	( 89.95,129.15) --
	( 89.96,129.18) --
	( 90.01,129.19) --
	( 90.04,129.18) --
	( 90.11,129.16) --
	( 90.18,129.13) --
	( 90.22,129.11) --
	( 90.29,129.14) --
	( 90.35,129.18) --
	( 90.38,129.20) --
	( 90.46,129.21) --
	( 90.54,129.22) --
	( 90.56,129.28) --
	( 90.57,129.32) --
	( 90.54,129.42) --
	( 90.51,129.48) --
	( 90.50,129.51) --
	( 90.51,129.54) --
	( 90.53,129.55) --
	( 90.58,129.54) --
	( 90.62,129.49) --
	( 90.68,129.42) --
	( 90.78,129.31) --
	( 90.82,129.31) --
	( 90.85,129.31) --
	( 90.94,129.42) --
	( 91.00,129.53) --
	( 91.13,129.70) --
	( 91.13,129.75) --
	( 91.07,129.90) --
	( 91.02,129.96) --
	( 90.97,130.00) --
	( 90.97,130.02) --
	( 91.00,130.12) --
	( 91.03,130.17) --
	( 91.05,130.18) --
	( 91.06,130.22) --
	( 91.13,130.24) --
	( 91.36,130.29) --
	( 91.41,130.32) --
	( 91.45,130.38) --
	( 91.45,130.44) --
	( 91.42,130.51) --
	( 91.31,130.61) --
	( 91.28,130.65) --
	( 91.27,130.71) --
	( 91.26,130.76) --
	( 91.21,130.83) --
	( 91.19,130.87) --
	( 91.20,130.92) --
	( 91.23,130.96) --
	( 91.29,131.02) --
	( 91.36,131.05) --
	( 91.55,131.11) --
	( 91.66,131.15) --
	( 91.69,131.17) --
	( 91.73,131.21) --
	( 91.73,131.28) --
	( 91.72,131.33) --
	( 91.58,131.54) --
	( 91.53,131.75) --
	( 91.50,131.99) --
	( 91.46,132.03) --
	( 91.39,132.04) --
	( 91.35,132.03) --
	( 91.24,131.90) --
	( 91.13,131.82) --
	( 91.09,131.85) --
	( 91.12,131.91) --
	( 91.17,132.00) --
	( 91.25,132.11) --
	( 91.23,132.12) --
	( 91.12,132.08) --
	( 91.09,132.11) --
	( 91.06,132.16) --
	( 91.02,132.26) --
	( 91.04,132.43) --
	( 90.87,132.35) --
	( 90.82,132.33) --
	( 90.77,132.34) --
	( 90.78,132.39) --
	( 90.88,132.45) --
	( 90.88,132.50) --
	( 90.84,132.54) --
	( 90.80,132.59) --
	( 90.73,132.68) --
	( 90.53,132.70) --
	( 90.47,132.71) --
	( 90.46,132.75) --
	( 90.43,132.84) --
	( 90.42,132.88) --
	( 90.42,132.97) --
	( 90.35,133.08) --
	( 90.43,133.09) --
	( 90.46,133.08) --
	( 90.57,132.91) --
	( 90.60,132.87) --
	( 90.66,132.83) --
	( 90.74,132.77) --
	( 90.89,132.70) --
	( 91.06,132.63) --
	( 91.08,132.57) --
	( 91.11,132.53) --
	( 91.20,132.44) --
	( 91.25,132.33) --
	( 91.28,132.29) --
	( 91.31,132.29) --
	( 91.42,132.25) --
	( 91.49,132.17) --
	( 91.62,132.12) --
	( 91.75,132.03) --
	( 91.89,131.96) --
	( 92.16,132.00) --
	( 92.29,132.06) --
	( 92.40,132.17) --
	( 92.45,132.25) --
	( 92.42,132.48) --
	( 92.37,132.56) --
	( 92.25,132.62) --
	( 92.03,132.68) --
	( 91.83,132.74) --
	( 91.74,132.78) --
	( 91.69,132.87) --
	( 91.63,132.97) --
	( 91.59,133.08) --
	( 91.58,133.18) --
	( 91.62,133.25) --
	( 91.64,133.31) --
	( 91.71,133.37) --
	( 92.00,133.53) --
	( 92.04,133.55) --
	( 92.08,133.54) --
	( 92.15,133.43) --
	( 92.21,133.42) --
	( 92.32,133.40) --
	( 92.41,133.38) --
	( 92.48,133.30) --
	( 92.51,133.22) --
	( 92.71,133.13) --
	( 92.82,133.05) --
	( 93.02,133.07) --
	( 93.12,133.05) --
	( 93.23,132.97) --
	( 93.30,132.91) --
	( 93.41,132.83) --
	( 93.54,132.83) --
	( 93.66,132.84) --
	( 93.97,132.86) --
	( 94.07,132.90) --
	( 94.19,133.01) --
	( 94.28,133.09) --
	( 94.42,133.23) --
	( 94.44,133.38) --
	( 94.48,133.46) --
	( 94.58,133.49) --
	( 94.68,133.51) --
	( 94.71,133.52) --
	( 94.80,133.51) --
	( 94.87,133.40) --
	( 94.95,133.42) --
	( 94.98,133.62) --
	( 95.09,133.73) --
	( 95.17,133.84) --
	( 95.17,133.95) --
	( 95.12,134.04) --
	( 95.02,134.10) --
	( 94.92,134.10) --
	( 94.83,134.07) --
	( 94.76,134.00) --
	( 94.70,133.87) --
	( 94.63,133.87) --
	( 94.56,133.87) --
	( 94.53,133.89) --
	( 94.55,133.96) --
	( 94.69,134.17) --
	( 94.78,134.24) --
	( 94.77,134.28) --
	( 94.53,134.50) --
	( 94.51,134.54) --
	( 94.50,134.61) --
	( 94.57,134.72) --
	( 94.63,134.76) --
	( 94.72,134.85) --
	( 94.81,134.88) --
	( 95.01,134.92) --
	( 95.03,134.90) --
	( 95.12,134.88) --
	( 95.35,134.76) --
	( 95.42,134.74) --
	( 95.53,134.79) --
	( 95.59,134.88) --
	( 95.61,134.98) --
	( 95.67,135.18) --
	( 95.66,135.38) --
	( 95.26,135.37) --
	( 95.20,135.41) --
	( 95.12,135.46) --
	( 95.03,135.50) --
	( 94.85,135.61) --
	( 94.76,135.72) --
	( 94.69,135.79) --
	( 94.65,136.03) --
	( 94.64,136.19) --
	( 94.66,136.27) --
	( 94.56,136.33) --
	( 94.33,136.42) --
	( 94.33,136.47) --
	( 94.42,136.52) --
	( 94.43,136.57) --
	( 94.41,136.62) --
	( 94.32,136.67) --
	( 94.15,136.76) --
	( 94.08,136.79) --
	( 94.08,136.87) --
	( 94.03,136.98) --
	( 93.95,137.01) --
	( 93.68,136.99) --
	( 93.62,137.02) --
	( 93.65,137.08) --
	( 93.84,137.19) --
	( 93.83,137.23) --
	( 93.79,137.25) --
	( 93.63,137.29) --
	( 93.52,137.39) --
	( 93.48,137.51) --
	( 93.59,137.55) --
	( 93.73,137.58) --
	( 93.83,137.57) --
	( 93.95,137.54) --
	( 94.02,137.50) --
	( 94.15,137.40) --
	( 94.25,137.31) --
	( 94.29,137.25) --
	( 94.34,137.20) --
	( 94.40,137.13) --
	( 94.58,136.93) --
	( 94.76,136.72) --
	( 94.76,136.69) --
	( 94.78,136.58) --
	( 94.78,136.50) --
	( 94.82,136.45) --
	( 94.93,136.44) --
	( 94.99,136.43) --
	( 95.02,136.41) --
	( 95.13,136.26) --
	( 95.17,136.21) --
	( 95.23,136.21) --
	( 95.24,136.24) --
	( 95.20,136.41) --
	( 95.09,136.58) --
	( 95.00,136.83) --
	( 94.77,137.11) --
	( 94.67,137.26) --
	( 94.63,137.28) --
	( 94.43,137.38) --
	( 94.40,137.46) --
	( 94.31,137.55) --
	( 94.28,137.61) --
	( 94.20,137.85) --
	( 94.13,137.98) --
	( 94.00,138.17) --
	( 93.85,138.34) --
	( 93.76,138.40) --
	( 93.73,138.46) --
	( 93.75,138.60) --
	( 93.76,138.86) --
	( 93.87,138.87) --
	( 93.90,138.86) --
	( 93.93,138.84) --
	( 94.04,138.76) --
	( 94.18,138.74) --
	( 94.34,138.72) --
	( 94.37,138.71) --
	( 94.32,138.57) --
	( 94.37,138.53) --
	( 94.47,138.54) --
	( 94.57,138.54) --
	( 94.65,138.53) --
	( 94.69,138.49) --
	( 94.71,138.45) --
	( 94.62,138.26) --
	( 94.60,138.23) --
	( 94.60,138.15) --
	( 94.64,138.05) --
	( 94.76,137.89) --
	( 94.80,137.89) --
	( 94.91,137.91) --
	( 94.99,137.92) --
	( 95.01,137.91) --
	( 95.08,137.74) --
	( 95.10,137.72) --
	( 95.21,137.71) --
	( 95.24,137.54) --
	( 95.39,137.47) --
	( 95.44,137.45) --
	( 95.48,137.39) --
	( 95.48,137.30) --
	( 95.57,137.19) --
	( 95.61,137.09) --
	( 95.64,137.06) --
	( 95.73,137.03) --
	( 95.80,137.04) --
	( 95.89,137.12) --
	( 95.94,137.11) --
	( 96.06,136.98) --
	( 96.08,136.93) --
	( 96.05,136.82) --
	( 96.08,136.78) --
	( 96.15,136.75) --
	( 96.25,136.79) --
	( 96.34,136.81) --
	( 96.40,136.85) --
	( 96.43,136.87) --
	( 96.47,136.86) --
	( 96.51,136.76) --
	( 96.54,136.70) --
	( 96.62,136.68) --
	( 96.72,136.68) --
	( 97.04,136.71) --
	( 97.13,136.72) --
	( 97.20,136.78) --
	( 97.01,136.77) --
	( 96.91,136.77) --
	( 96.74,136.84) --
	( 96.64,136.93) --
	( 96.60,137.05) --
	( 96.60,137.17) --
	( 96.61,137.24) --
	( 96.64,137.37) --
	( 96.68,137.53) --
	( 96.73,137.64) --
	( 96.79,137.75) --
	( 96.83,137.83) --
	( 96.92,137.91) --
	( 96.92,137.99) --
	( 96.88,138.08) --
	( 96.82,138.15) --
	( 96.54,138.46) --
	( 96.40,138.66) --
	( 96.32,138.80) --
	( 96.29,138.87) --
	( 96.34,139.02) --
	( 96.53,138.79) --
	( 96.64,138.67) --
	( 96.77,138.54) --
	( 96.84,138.42) --
	( 96.92,138.26) --
	( 97.02,138.18) --
	( 97.02,138.17) --
	( 97.08,138.14) --
	( 97.12,138.13) --
	( 97.17,138.16) --
	( 97.23,138.28) --
	( 97.27,138.34) --
	( 97.34,138.40) --
	( 97.47,138.52) --
	( 97.50,138.65) --
	( 97.52,138.79) --
	( 97.54,139.05) --
	( 97.53,139.26) --
	( 97.53,139.29) --
	( 97.57,139.31) --
	( 97.64,139.26) --
	( 97.68,139.07) --
	( 97.70,138.92) --
	( 97.72,138.82) --
	( 97.75,138.79) --
	( 97.84,138.75) --
	( 97.96,138.79) --
	( 98.17,138.85) --
	( 98.24,138.88) --
	( 98.28,138.96) --
	( 98.29,139.16) --
	( 98.26,139.33) --
	( 98.22,139.49) --
	( 98.22,139.64) --
	( 98.27,139.72) --
	( 98.30,139.83) --
	( 98.29,139.89) --
	( 98.30,139.91) --
	( 98.37,139.92) --
	( 98.47,139.88) --
	( 98.56,139.84) --
	( 98.74,139.88) --
	( 98.89,139.91) --
	( 99.12,140.01) --
	( 99.15,140.08) --
	( 99.15,140.20) --
	( 99.20,140.48) --
	( 99.26,140.59) --
	( 99.36,140.69) --
	( 99.45,140.73) --
	( 99.54,140.76) --
	( 99.62,140.72) --
	( 99.82,140.59) --
	( 99.88,140.56) --
	( 99.95,140.56) --
	( 99.98,140.57) --
	(100.11,140.76) --
	(100.18,140.86) --
	(100.30,140.94) --
	(100.39,141.06) --
	(100.43,141.11) --
	(100.49,141.14) --
	(100.56,141.14) --
	(100.78,141.09) --
	(100.86,141.10) --
	(100.89,141.13) --
	(100.92,141.20) --
	(100.98,141.37) --
	(101.01,141.45) --
	(101.03,141.58) --
	(100.99,141.65) --
	(100.89,141.76) --
	(100.77,141.98) --
	(100.52,142.07) --
	(100.49,142.10) --
	(100.47,142.16) --
	(100.47,142.19) --
	(100.48,142.19) --
	(100.51,142.19) --
	(100.71,142.10) --
	(100.83,142.14) --
	(100.87,142.13) --
	(100.94,142.08) --
	(101.01,142.09) --
	(101.10,142.13) --
	(101.17,142.17) --
	(101.36,142.36) --
	(101.45,142.49) --
	(101.48,142.65) --
	(101.50,142.74) --
	(101.56,142.82) --
	(101.63,142.89) --
	(101.78,142.96) --
	(101.90,142.98) --
	(102.01,142.99) --
	(102.04,143.06) --
	(102.05,143.19) --
	(102.04,143.30) --
	(102.07,143.39) --
	(102.15,143.53) --
	(102.50,143.74) --
	(102.50,143.79) --
	(102.47,143.82) --
	(102.44,143.82) --
	(102.00,143.65) --
	(101.82,143.65) --
	(101.66,143.73) --
	(101.58,143.74) --
	(101.49,143.74) --
	(101.39,143.73) --
	(101.19,143.72) --
	(101.17,143.75) --
	(101.23,143.79) --
	(101.36,143.84) --
	(101.46,143.88) --
	(101.49,143.91) --
	(101.51,144.00) --
	(101.48,144.10) --
	(101.38,144.22) --
	(101.33,144.27) --
	(101.26,144.30) --
	(101.07,144.35) --
	(101.03,144.36) --
	(101.03,144.41) --
	(101.27,144.44) --
	(101.36,144.46) --
	(101.49,144.67) --
	(101.60,144.80) --
	(101.70,144.89) --
	(101.70,144.94) --
	(101.67,144.99) --
	(101.57,145.04) --
	(101.50,145.07) --
	(101.36,145.14) --
	(101.26,145.20) --
	(101.23,145.24) --
	(101.21,145.48) --
	(101.31,145.37) --
	(101.38,145.33) --
	(101.54,145.33) --
	(101.58,145.31) --
	(101.89,145.17) --
	(101.94,145.16) --
	(102.01,145.04) --
	(102.03,144.97) --
	(102.06,144.93) --
	(102.15,144.93) --
	(102.17,144.90) --
	(102.23,144.80) --
	(102.26,144.74) --
	(102.33,144.70) --
	(102.39,144.70) --
	(102.44,144.74) --
	(102.49,144.85) --
	(102.51,145.14) --
	(102.53,145.35) --
	(102.56,145.39) --
	(102.76,145.56) --
	(103.04,145.70) --
	(103.17,145.82) --
	(103.22,145.93) --
	(103.25,146.03) --
	(103.21,146.17) --
	(103.11,146.27) --
	(103.00,146.34) --
	(102.68,146.53) --
	(102.93,146.50) --
	(103.08,146.42) --
	(103.13,146.45) --
	(103.26,146.58) --
	(103.61,146.69) --
	(103.61,146.76) --
	(103.48,146.84) --
	(103.43,146.95) --
	(103.43,146.99) --
	(103.51,147.03) --
	(103.58,146.91) --
	(103.71,146.78);

\draw[color=drawColor,line cap=round,line join=round,fill opacity=0.00,] ( 78.77,123.89) --
	( 78.65,124.03) --
	( 78.57,124.15) --
	( 78.53,124.26) --
	( 78.49,124.31) --
	( 78.46,124.35) --
	( 78.42,124.39) --
	( 78.39,124.41) --
	( 78.33,124.46) --
	( 78.27,124.53) --
	( 78.22,124.63) --
	( 78.22,124.73) --
	( 78.13,124.82) --
	( 78.12,124.84) --
	( 78.12,124.87) --
	( 78.12,124.90) --
	( 78.08,124.92) --
	( 77.99,124.95) --
	( 77.89,125.05) --
	( 77.85,125.04) --
	( 77.81,125.02) --
	( 77.77,124.97) --
	( 77.72,124.95) --
	( 77.68,124.90) --
	( 77.67,124.86) --
	( 77.67,124.84) --
	( 77.67,124.81) --
	( 77.68,124.71) --
	( 77.69,124.64) --
	( 77.61,124.62) --
	( 77.55,124.56) --
	( 77.38,124.53) --
	( 77.26,124.50) --
	( 77.28,124.44) --
	( 77.29,124.36) --
	( 77.26,124.31) --
	( 77.22,124.27) --
	( 77.09,124.23) --
	( 77.09,124.20) --
	( 77.09,124.14) --
	( 77.08,124.06) --
	( 77.09,124.05) --
	( 77.08,123.97) --
	( 77.09,123.95) --
	( 77.08,123.90) --
	( 76.99,123.85) --
	( 76.79,123.77) --
	( 76.68,123.69) --
	( 76.66,123.64) --
	( 76.69,123.58) --
	( 76.72,123.53) --
	( 76.83,123.49) --
	( 76.99,123.49) --
	( 77.17,123.45) --
	( 77.44,123.33) --
	( 77.48,123.33) --
	( 77.48,123.40) --
	( 77.47,123.50) --
	( 77.44,123.63) --
	( 77.37,123.71) --
	( 77.39,123.75) --
	( 77.47,123.77) --
	( 77.63,123.78) --
	( 77.74,123.78) --
	( 77.86,123.84) --
	( 77.93,123.87) --
	( 78.04,123.86) --
	( 78.08,123.87) --
	( 78.16,123.83) --
	( 78.15,123.73) --
	( 78.21,123.66) --
	( 78.27,123.68) --
	( 78.31,123.68) --
	( 78.40,123.67) --
	( 78.44,123.66) --
	( 78.48,123.63) --
	( 78.58,123.56) --
	( 78.61,123.55) --
	( 78.68,123.63) --
	( 78.69,123.68) --
	( 78.81,123.71) --
	( 78.81,123.76) --
	( 78.82,123.80) --
	( 78.77,123.89);

\draw[color=drawColor,line cap=round,line join=round,fill opacity=0.00,] (148.63,139.20) --
	(148.77,139.14) --
	(148.94,139.19) --
	(148.98,139.14) --
	(148.98,139.08) --
	(149.20,139.05) --
	(149.37,138.97) --
	(149.23,138.89) --
	(149.06,138.89) --
	(149.02,138.83) --
	(149.12,138.71) --
	(149.29,138.51) --
	(149.38,138.26) --
	(149.39,137.98) --
	(149.48,137.91) --
	(149.62,137.95) --
	(149.76,138.04) --
	(149.88,138.01) --
	(149.77,137.77) --
	(149.50,137.63) --
	(149.29,137.62) --
	(149.24,137.53) --
	(149.25,137.25) --
	(149.34,137.00) --
	(149.32,136.83) --
	(149.38,136.69) --
	(149.34,136.58) --
	(149.24,136.65) --
	(149.19,136.84) --
	(149.10,136.95) --
	(149.00,137.01) --
	(148.81,137.07) --
	(148.38,137.12) --
	(148.27,137.24) --
	(148.24,137.33) --
	(148.28,137.36) --
	(148.44,137.34) --
	(148.60,137.37) --
	(148.78,137.57) --
	(148.92,137.64) --
	(148.99,137.76) --
	(149.14,137.94) --
	(149.12,138.13) --
	(149.02,138.36) --
	(148.84,138.56) --
	(148.68,138.83) --
	(148.54,138.93) --
	(148.39,138.91) --
	(148.34,139.01) --
	(148.29,139.13) --
	(148.45,139.21) --
	(148.63,139.20);

\draw[color=drawColor,line cap=round,line join=round,fill opacity=0.00,] (102.90,131.28) --
	(102.85,131.23) --
	(102.76,131.21) --
	(102.62,131.24) --
	(102.53,131.22) --
	(102.46,131.20) --
	(102.39,131.17) --
	(102.07,131.11) --
	(101.99,131.13) --
	(101.89,131.22) --
	(101.82,131.21) --
	(101.76,131.16) --
	(101.62,131.08) --
	(101.53,131.04) --
	(101.40,131.01) --
	(101.29,131.04) --
	(101.23,131.09) --
	(101.19,131.11) --
	(101.05,131.08) --
	(100.92,131.06) --
	(100.73,131.02) --
	(100.63,130.97) --
	(100.59,130.89) --
	(100.53,130.75) --
	(100.51,130.67) --
	(100.48,130.63) --
	(100.39,130.58) --
	(100.40,130.75) --
	(100.38,130.84) --
	(100.44,130.94) --
	(100.50,131.01) --
	(100.60,131.05) --
	(100.79,131.10) --
	(100.86,131.14) --
	(101.10,131.18) --
	(101.18,131.19) --
	(101.23,131.17) --
	(101.44,131.10) --
	(101.50,131.10) --
	(101.65,131.16) --
	(101.72,131.23) --
	(101.81,131.27) --
	(101.91,131.28) --
	(102.08,131.26) --
	(102.18,131.22) --
	(102.40,131.25) --
	(102.71,131.34) --
	(102.82,131.38) --
	(102.94,131.46) --
	(102.95,131.50) --
	(102.96,131.77) --
	(103.04,131.56) --
	(103.04,131.48) --
	(102.99,131.34) --
	(102.90,131.28);

\draw[color=drawColor,line cap=round,line join=round,fill opacity=0.00,] ( 97.17,130.76) --
	( 97.20,130.73) --
	( 97.30,130.74) --
	( 97.37,130.73) --
	( 97.44,130.74) --
	( 97.54,130.73) --
	( 97.57,130.74) --
	( 97.69,130.75) --
	( 97.77,130.77) --
	( 97.80,130.77) --
	( 97.84,130.69) --
	( 97.90,130.58) --
	( 97.85,130.47) --
	( 97.98,130.33) --
	( 98.02,130.32) --
	( 98.11,130.31) --
	( 98.16,130.26) --
	( 98.19,130.20) --
	( 98.33,130.13) --
	( 98.39,130.09) --
	( 98.42,130.05) --
	( 98.43,129.91) --
	( 98.31,129.90) --
	( 98.18,129.89) --
	( 98.13,129.94) --
	( 98.10,129.96) --
	( 98.03,130.06) --
	( 97.99,130.11) --
	( 97.86,130.14) --
	( 97.79,130.18) --
	( 97.74,130.26) --
	( 97.68,130.37) --
	( 97.65,130.44) --
	( 97.61,130.50) --
	( 97.50,130.56) --
	( 97.38,130.59) --
	( 97.16,130.67) --
	( 97.10,130.72) --
	( 97.09,130.80) --
	( 97.13,130.95) --
	( 97.17,130.76);

\draw[color=drawColor,line cap=round,line join=round,fill opacity=0.00,] ( 99.72,130.22) --
	( 99.77,130.22) --
	( 99.80,130.23) --
	( 99.87,130.22) --
	(100.06,130.21) --
	(100.16,130.22) --
	(100.16,130.18) --
	(100.08,130.11) --
	(100.00,130.04) --
	( 99.90,130.00) --
	( 99.78,130.00) --
	( 99.63,130.06) --
	( 99.60,130.05) --
	( 99.50,129.93) --
	( 99.35,129.87) --
	( 99.21,129.83) --
	( 99.10,129.85) --
	( 99.03,129.88) --
	( 98.96,129.94) --
	( 98.90,129.98) --
	( 98.80,130.00) --
	( 98.89,130.11) --
	( 98.96,130.13) --
	( 99.06,130.14) --
	( 99.22,130.12) --
	( 99.33,130.14) --
	( 99.39,130.17) --
	( 99.44,130.19) --
	( 99.57,130.22) --
	( 99.63,130.25) --
	( 99.72,130.22);

\draw[color=drawColor,line cap=round,line join=round,fill opacity=0.00,] (177.20,125.72) --
	(177.26,125.67) --
	(177.37,125.68) --
	(177.47,125.60) --
	(177.55,125.46) --
	(177.58,125.36) --
	(177.69,125.35) --
	(177.76,125.38) --
	(177.86,125.46) --
	(178.15,125.49) --
	(178.37,125.45) --
	(178.49,125.43) --
	(178.69,125.43) --
	(178.79,125.38) --
	(178.86,125.38) --
	(178.89,125.34) --
	(178.90,125.24) --
	(178.76,125.20) --
	(178.58,125.21) --
	(178.36,125.22) --
	(178.28,125.18) --
	(178.22,125.17) --
	(178.15,125.23) --
	(178.08,125.25) --
	(177.99,125.22) --
	(177.94,125.15) --
	(177.78,125.13) --
	(177.73,125.08) --
	(177.74,125.04) --
	(177.74,124.98) --
	(177.71,124.96) --
	(177.46,124.96) --
	(177.45,124.85) --
	(177.30,124.81) --
	(177.12,124.78) --
	(177.08,124.76) --
	(177.04,124.73) --
	(177.03,124.71) --
	(177.01,124.52) --
	(176.97,124.40) --
	(176.96,124.31) --
	(176.93,124.24) --
	(176.89,124.12) --
	(176.86,124.09) --
	(176.78,124.03) --
	(176.72,123.98) --
	(176.72,123.91) --
	(176.61,123.86) --
	(176.61,123.80) --
	(176.44,123.70) --
	(176.42,123.54) --
	(176.59,123.51) --
	(176.42,123.34) --
	(176.37,123.24) --
	(176.37,123.19) --
	(176.40,123.07) --
	(176.44,123.05) --
	(176.49,123.04) --
	(176.56,123.04) --
	(176.59,123.04) --
	(176.63,123.03) --
	(176.67,123.00) --
	(176.63,122.92) --
	(176.53,122.82) --
	(176.52,122.73) --
	(176.51,122.59) --
	(176.46,122.42) --
	(176.36,122.26) --
	(176.25,122.17) --
	(176.13,122.11) --
	(175.99,122.07) --
	(175.81,122.05) --
	(175.70,122.03) --
	(175.60,121.97) --
	(175.50,121.89) --
	(175.40,121.80) --
	(175.36,121.72) --
	(175.28,121.72) --
	(175.21,121.70) --
	(175.12,121.67) --
	(175.08,121.64) --
	(175.02,121.63) --
	(174.92,121.62) --
	(174.80,121.62) --
	(174.68,121.59) --
	(174.56,121.50) --
	(174.50,121.46) --
	(174.55,121.22) --
	(174.41,121.19) --
	(174.34,121.19) --
	(174.28,121.14) --
	(174.11,121.00) --
	(173.93,120.93) --
	(173.70,120.93) --
	(173.60,120.94) --
	(173.56,120.90) --
	(173.46,120.83) --
	(173.32,120.79) --
	(173.28,120.79) --
	(173.20,120.78) --
	(173.15,120.74) --
	(173.07,120.62) --
	(172.96,120.54) --
	(172.92,120.54) --
	(172.79,120.56) --
	(172.72,120.49) --
	(172.70,120.32) --
	(172.88,120.16) --
	(172.88,120.13) --
	(172.79,120.08) --
	(172.67,120.01) --
	(172.60,119.95) --
	(172.60,119.88) --
	(172.51,119.74) --
	(172.37,119.70) --
	(172.24,119.70) --
	(172.22,119.81) --
	(172.22,119.93) --
	(172.26,119.96) --
	(172.29,120.01) --
	(172.29,120.07) --
	(172.27,120.09) --
	(172.23,120.16) --
	(172.06,120.35) --
	(172.00,120.43) --
	(172.00,120.51) --
	(172.02,120.55) --
	(172.06,120.64) --
	(172.15,120.77) --
	(172.23,120.95) --
	(172.24,121.05) --
	(172.22,121.23) --
	(172.13,121.34) --
	(172.15,121.45) --
	(172.19,121.61) --
	(172.17,121.76) --
	(172.15,121.89) --
	(172.13,121.99) --
	(172.05,122.11) --
	(171.94,122.28) --
	(171.90,122.47) --
	(171.91,122.64) --
	(171.98,122.67) --
	(172.09,122.73) --
	(172.17,122.81) --
	(172.23,122.89) --
	(172.28,122.97) --
	(172.39,123.15) --
	(172.42,123.17) --
	(172.46,123.20) --
	(172.60,123.27) --
	(172.70,123.29) --
	(172.75,123.24) --
	(172.75,123.21) --
	(172.73,123.14) --
	(172.67,122.96) --
	(172.53,122.81) --
	(172.43,122.72) --
	(172.42,122.66) --
	(172.39,122.55) --
	(172.42,122.45) --
	(172.45,122.41) --
	(172.50,122.38) --
	(172.48,122.33) --
	(172.48,122.22) --
	(172.50,122.12) --
	(172.54,122.06) --
	(172.59,122.01) --
	(172.62,121.96) --
	(172.59,121.94) --
	(172.52,121.81) --
	(172.52,121.66) --
	(172.53,121.50) --
	(172.62,121.38) --
	(172.63,121.28) --
	(172.61,121.13) --
	(172.53,120.93) --
	(172.53,120.82) --
	(172.54,120.76) --
	(172.60,120.73) --
	(172.70,120.80) --
	(172.76,120.81) --
	(172.77,120.86) --
	(172.87,120.89) --
	(173.00,120.88) --
	(173.05,120.92) --
	(173.09,120.96) --
	(173.19,121.04) --
	(173.37,121.16) --
	(173.54,121.26) --
	(173.68,121.22) --
	(173.78,121.19) --
	(173.88,121.20) --
	(173.95,121.23) --
	(174.05,121.31) --
	(174.09,121.36) --
	(174.10,121.47) --
	(174.21,121.62) --
	(174.34,121.75) --
	(174.42,121.81) --
	(174.54,121.82) --
	(174.62,121.82) --
	(174.71,121.85) --
	(174.76,121.89) --
	(174.86,121.95) --
	(174.97,121.97) --
	(175.06,121.96) --
	(175.16,122.00) --
	(175.23,122.09) --
	(175.30,122.19) --
	(175.37,122.26) --
	(175.44,122.29) --
	(175.54,122.28) --
	(175.66,122.26) --
	(175.76,122.26) --
	(175.88,122.30) --
	(176.00,122.38) --
	(176.08,122.50) --
	(176.11,122.61) --
	(176.07,122.73) --
	(175.95,122.87) --
	(175.92,122.96) --
	(175.95,123.04) --
	(176.00,123.14) --
	(175.98,123.26) --
	(176.00,123.41) --
	(175.99,123.62) --
	(176.00,123.70) --
	(176.05,123.81) --
	(176.09,123.85) --
	(176.17,123.91) --
	(176.26,123.96) --
	(176.28,124.00) --
	(176.28,124.05) --
	(176.22,124.07) --
	(176.15,124.08) --
	(176.10,124.12) --
	(176.12,124.18) --
	(176.22,124.24) --
	(176.26,124.27) --
	(176.26,124.35) --
	(176.25,124.44) --
	(176.32,124.47) --
	(176.40,124.45) --
	(176.45,124.38) --
	(176.46,124.27) --
	(176.44,124.16) --
	(176.45,124.07) --
	(176.52,124.06) --
	(176.58,124.11) --
	(176.61,124.17) --
	(176.63,124.25) --
	(176.63,124.32) --
	(176.63,124.44) --
	(176.73,124.55) --
	(176.75,124.71) --
	(176.75,124.73) --
	(176.74,124.76) --
	(176.62,124.97) --
	(176.69,125.00) --
	(176.82,124.95) --
	(176.85,124.85) --
	(176.93,124.96) --
	(177.16,124.99) --
	(177.24,125.04) --
	(177.24,125.25) --
	(177.31,125.29) --
	(177.32,125.31) --
	(177.27,125.34) --
	(177.18,125.37) --
	(177.09,125.45) --
	(177.03,125.57) --
	(177.02,125.63) --
	(177.00,125.73) --
	(176.97,125.88) --
	(177.01,126.05) --
	(176.98,126.14) --
	(177.02,126.21) --
	(177.09,126.18) --
	(177.15,126.08) --
	(177.14,125.91) --
	(177.16,125.87) --
	(177.20,125.80) --
	(177.20,125.72);

\draw[color=drawColor,line cap=round,line join=round,fill opacity=0.00,] (  0.00,103.75) --
	(  0.04,103.79) --
	(  0.01,103.87) --
	(  0.06,103.90) --
	(  0.17,103.86) --
	(  0.36,103.86) --
	(  0.45,103.90) --
	(  0.58,103.93) --
	(  0.75,103.93) --
	(  0.85,103.94) --
	(  0.95,103.97) --
	(  1.00,103.98) --
	(  1.02,104.05) --
	(  0.97,104.18) --
	(  0.90,104.21) --
	(  0.86,104.32) --
	(  0.89,104.39) --
	(  1.00,104.33) --
	(  1.14,104.18) --
	(  1.23,104.11) --
	(  1.29,104.06) --
	(  1.32,104.06) --
	(  1.34,104.04) --
	(  1.43,104.06) --
	(  1.46,104.14) --
	(  1.56,104.15) --
	(  1.64,104.12) --
	(  1.73,104.14) --
	(  1.80,104.24) --
	(  1.87,104.29) --
	(  1.98,104.24) --
	(  2.15,104.16) --
	(  2.26,104.17) --
	(  2.37,104.19) --
	(  2.46,104.17) --
	(  2.43,104.11) --
	(  2.41,104.05) --
	(  2.48,104.02) --
	(  2.56,104.02) --
	(  2.61,103.97) --
	(  2.66,103.93) --
	(  2.72,103.94) --
	(  2.87,103.97) --
	(  2.98,103.98) --
	(  3.11,103.98) --
	(  3.17,103.99) --
	(  3.28,104.07) --
	(  3.33,104.13) --
	(  3.44,104.16) --
	(  3.54,104.14) --
	(  3.76,104.08) --
	(  3.82,104.10) --
	(  3.78,104.22) --
	(  3.69,104.46) --
	(  3.67,104.56) --
	(  3.69,104.66) --
	(  3.71,104.75) --
	(  3.76,104.78) --
	(  3.78,104.91) --
	(  3.73,104.97) --
	(  3.64,104.98) --
	(  3.67,105.02) --
	(  3.74,105.03) --
	(  3.79,105.13) --
	(  3.84,105.22) --
	(  3.88,105.26) --
	(  3.97,105.43) --
	(  3.93,105.55) --
	(  4.09,105.55) --
	(  4.13,105.56) --
	(  4.19,105.69) --
	(  4.15,105.77) --
	(  4.24,105.82) --
	(  4.34,105.84) --
	(  4.37,105.91) --
	(  4.36,105.94) --
	(  4.36,106.02) --
	(  4.39,106.09) --
	(  4.48,106.13) --
	(  4.52,106.20) --
	(  4.50,106.30) --
	(  4.38,106.38) --
	(  4.38,106.45) --
	(  4.47,106.53) --
	(  4.47,106.60) --
	(  4.44,106.67) --
	(  4.46,106.76) --
	(  4.52,106.85) --
	(  4.51,106.95) --
	(  4.49,106.97) --
	(  4.53,107.01) --
	(  4.59,107.01) --
	(  4.72,106.98) --
	(  4.80,107.03) --
	(  4.86,107.16) --
	(  4.85,107.26) --
	(  4.77,107.30) --
	(  4.67,107.37) --
	(  4.56,107.33) --
	(  4.48,107.31) --
	(  4.41,107.30) --
	(  4.36,107.44) --
	(  4.34,107.50) --
	(  4.20,107.51) --
	(  4.13,107.48) --
	(  4.06,107.47) --
	(  4.00,107.46) --
	(  3.96,107.50) --
	(  3.92,107.56) --
	(  4.02,107.63) --
	(  4.19,107.76) --
	(  4.32,107.87) --
	(  4.41,107.96) --
	(  4.32,107.98) --
	(  4.20,108.03) --
	(  4.22,108.11) --
	(  4.34,108.19) --
	(  4.35,108.24) --
	(  4.31,108.28) --
	(  4.23,108.36) --
	(  4.22,108.45) --
	(  4.30,108.53) --
	(  4.42,108.54) --
	(  4.48,108.59) --
	(  4.53,108.68) --
	(  4.55,108.82) --
	(  4.66,108.87) --
	(  4.75,108.97) --
	(  4.75,109.04) --
	(  4.67,109.09) --
	(  4.65,109.23) --
	(  4.70,109.37) --
	(  4.71,109.47) --
	(  4.76,109.56) --
	(  4.78,109.58) --
	(  4.86,109.62) --
	(  4.92,109.63) --
	(  4.97,109.77) --
	(  4.98,109.82) --
	(  5.05,109.83) --
	(  5.09,109.80) --
	(  5.24,109.82) --
	(  5.33,109.94) --
	(  5.36,110.05) --
	(  5.35,110.18) --
	(  5.37,110.29) --
	(  5.43,110.39) --
	(  5.53,110.45) --
	(  5.66,110.45) --
	(  5.74,110.49) --
	(  5.76,110.55) --
	(  5.79,110.61) --
	(  5.85,110.65) --
	(  5.92,110.67) --
	(  6.01,110.77) --
	(  6.00,110.86) --
	(  5.96,110.94) --
	(  5.94,111.02) --
	(  5.85,111.03) --
	(  5.83,111.10) --
	(  5.76,111.20) --
	(  5.72,111.34) --
	(  5.68,111.38) --
	(  5.61,111.42) --
	(  5.52,111.43) --
	(  5.43,111.43) --
	(  5.36,111.42) --
	(  5.29,111.44) --
	(  5.31,111.53) --
	(  5.34,111.62) --
	(  5.30,111.74) --
	(  5.33,111.77) --
	(  5.41,111.86) --
	(  5.46,111.87) --
	(  5.57,111.83) --
	(  5.63,111.76) --
	(  5.68,111.70) --
	(  5.76,111.64) --
	(  5.85,111.67) --
	(  5.90,111.73) --
	(  6.00,111.81) --
	(  6.12,111.80) --
	(  6.17,111.85) --
	(  6.15,111.97) --
	(  6.15,112.14) --
	(  6.16,112.37) --
	(  6.18,112.54) --
	(  6.17,112.65) --
	(  6.25,112.51) --
	(  6.29,112.31) --
	(  6.32,112.19) --
	(  6.30,112.02) --
	(  6.31,111.87) --
	(  6.39,111.79) --
	(  6.33,111.72) --
	(  6.30,111.66) --
	(  6.36,111.62) --
	(  6.36,111.60) --
	(  6.25,111.55) --
	(  6.25,111.46) --
	(  6.45,111.43) --
	(  6.66,111.42) --
	(  6.70,111.38) --
	(  6.70,111.31) --
	(  6.64,111.26) --
	(  6.54,111.21) --
	(  6.33,111.23) --
	(  6.24,111.17) --
	(  6.32,111.05) --
	(  6.40,110.95) --
	(  6.42,110.88) --
	(  6.39,110.80) --
	(  6.35,110.74) --
	(  6.33,110.66) --
	(  6.35,110.48) --
	(  6.37,110.37) --
	(  6.41,110.28) --
	(  6.48,110.17) --
	(  6.48,110.11) --
	(  6.38,110.08) --
	(  6.29,109.99) --
	(  6.26,109.90) --
	(  6.21,109.75) --
	(  6.15,109.75) --
	(  6.12,109.68) --
	(  6.02,109.62) --
	(  5.89,109.59) --
	(  5.79,109.54) --
	(  5.81,109.45) --
	(  5.87,109.37) --
	(  5.89,109.25) --
	(  5.90,109.16) --
	(  5.94,109.09) --
	(  5.87,109.00) --
	(  5.82,108.94) --
	(  5.78,108.88) --
	(  5.66,108.91) --
	(  5.60,108.94) --
	(  5.52,108.94) --
	(  5.54,108.89) --
	(  5.59,108.83) --
	(  5.62,108.74) --
	(  5.61,108.66) --
	(  5.62,108.61) --
	(  5.69,108.54) --
	(  5.65,108.51) --
	(  5.52,108.49) --
	(  5.45,108.46) --
	(  5.38,108.40) --
	(  5.39,108.36) --
	(  5.44,108.27) --
	(  5.42,108.21) --
	(  5.34,108.13) --
	(  5.35,108.03) --
	(  5.41,107.83) --
	(  5.44,107.62) --
	(  5.42,107.53) --
	(  5.41,107.47) --
	(  5.32,107.43) --
	(  5.29,107.36) --
	(  5.29,107.25) --
	(  5.27,107.22) --
	(  5.35,107.18) --
	(  5.39,107.14) --
	(  5.41,107.11) --
	(  5.31,107.05) --
	(  5.22,107.05) --
	(  5.15,107.00) --
	(  5.16,106.96) --
	(  5.21,106.88) --
	(  5.11,106.77) --
	(  4.96,106.79) --
	(  4.87,106.75) --
	(  4.89,106.63) --
	(  4.86,106.52) --
	(  4.83,106.40) --
	(  4.93,106.38) --
	(  5.00,106.46) --
	(  5.13,106.54) --
	(  5.20,106.49) --
	(  5.25,106.38) --
	(  5.37,106.29) --
	(  5.43,106.20) --
	(  5.39,106.12) --
	(  5.31,106.05) --
	(  5.25,105.97) --
	(  5.27,105.90) --
	(  5.27,105.83) --
	(  5.25,105.74) --
	(  5.21,105.69) --
	(  5.20,105.59) --
	(  5.19,105.53) --
	(  5.17,105.50) --
	(  5.20,105.42) --
	(  5.22,105.38) --
	(  5.21,105.36) --
	(  5.12,105.32) --
	(  5.10,105.25) --
	(  5.14,105.15) --
	(  5.11,105.12) --
	(  5.00,105.08) --
	(  4.96,105.06) --
	(  4.95,104.99) --
	(  4.99,104.92) --
	(  5.06,104.90) --
	(  5.15,104.88) --
	(  5.19,104.87) --
	(  5.17,104.83) --
	(  5.18,104.67) --
	(  5.21,104.62) --
	(  5.28,104.62) --
	(  5.36,104.59) --
	(  5.31,104.50) --
	(  5.29,104.41) --
	(  5.33,104.30) --
	(  5.44,104.18) --
	(  5.51,104.09) --
	(  5.57,104.01) --
	(  5.66,104.02) --
	(  5.74,103.99) --
	(  5.77,103.89) --
	(  5.82,103.86) --
	(  5.88,103.84) --
	(  5.96,103.80) --
	(  6.00,103.68) --
	(  6.09,103.64) --
	(  6.19,103.63) --
	(  6.22,103.57) --
	(  6.36,103.34) --
	(  6.43,103.26) --
	(  6.48,103.20) --
	(  6.54,103.13) --
	(  6.62,103.10) --
	(  6.76,103.05) --
	(  6.90,103.02) --
	(  6.95,103.01) --
	(  7.02,102.97) --
	(  7.03,102.93) --
	(  7.06,102.87) --
	(  7.17,102.85) --
	(  7.24,102.82) --
	(  7.30,102.71) --
	(  7.27,102.67) --
	(  7.27,102.59) --
	(  7.32,102.51) --
	(  7.41,102.44) --
	(  7.42,102.37) --
	(  7.34,102.31) --
	(  7.27,102.24) --
	(  7.33,102.22) --
	(  7.41,102.24) --
	(  7.51,102.26) --
	(  7.57,102.23) --
	(  7.61,102.20) --
	(  7.73,102.21) --
	(  7.73,102.16) --
	(  7.70,102.11) --
	(  7.53,102.03) --
	(  7.41,101.98) --
	(  7.34,101.90) --
	(  7.34,101.86) --
	(  7.44,101.84) --
	(  7.45,101.78) --
	(  7.43,101.74) --
	(  7.39,101.69) --
	(  7.44,101.68) --
	(  7.63,101.73) --
	(  7.69,101.75) --
	(  7.81,101.79) --
	(  7.90,101.80) --
	(  7.91,101.75) --
	(  7.91,101.72) --
	(  7.95,101.68) --
	(  8.01,101.68) --
	(  8.05,101.70) --
	(  8.09,101.75) --
	(  8.13,101.76) --
	(  8.13,101.72) --
	(  8.16,101.66) --
	(  8.21,101.64) --
	(  8.28,101.60) --
	(  8.32,101.54) --
	(  8.38,101.53) --
	(  8.42,101.53) --
	(  8.40,101.47) --
	(  8.40,101.43) --
	(  8.41,101.39) --
	(  8.51,101.36) --
	(  8.58,101.30) --
	(  8.48,101.23) --
	(  8.35,101.25) --
	(  8.17,101.26) --
	(  8.07,101.27) --
	(  8.04,101.21) --
	(  8.15,101.15) --
	(  8.18,101.10) --
	(  8.16,101.02) --
	(  8.11,100.95) --
	(  8.20,100.91) --
	(  8.31,100.81) --
	(  8.36,100.81) --
	(  8.36,100.89) --
	(  8.37,100.95) --
	(  8.40,100.97) --
	(  8.48,100.93) --
	(  8.57,100.88) --
	(  8.66,100.88) --
	(  8.77,100.90) --
	(  8.83,100.93) --
	(  8.90,100.95) --
	(  8.94,100.95) --
	(  8.95,100.87) --
	(  8.87,100.83) --
	(  8.75,100.77) --
	(  8.69,100.69) --
	(  8.79,100.60) --
	(  8.99,100.53) --
	(  9.28,100.53) --
	(  9.57,100.63) --
	(  9.73,100.71) --
	(  9.92,100.77) --
	( 10.03,100.81) --
	( 10.15,100.85) --
	( 10.30,100.86) --
	( 10.40,100.83) --
	( 10.47,100.84) --
	( 10.51,100.93) --
	( 10.59,100.96) --
	( 10.67,100.95) --
	( 10.77,100.93) --
	( 10.80,101.01) --
	( 10.86,101.08) --
	( 10.97,101.11) --
	( 11.00,101.22) --
	( 11.02,101.33) --
	( 11.00,101.39) --
	( 10.89,101.42) --
	( 10.84,101.52) --
	( 10.87,101.62) --
	( 10.94,101.75) --
	( 10.99,101.89) --
	( 10.98,101.97) --
	( 10.95,102.09) --
	( 10.87,102.24) --
	( 10.87,102.33) --
	( 10.97,102.40) --
	( 11.07,102.37) --
	( 11.16,102.56) --
	( 11.18,102.75) --
	( 11.26,102.90) --
	( 11.38,102.93) --
	( 11.46,102.99) --
	( 11.52,103.04) --
	( 11.53,103.13) --
	( 11.60,103.22) --
	( 11.64,103.27) --
	( 11.62,103.33) --
	( 11.65,103.40) --
	( 11.69,103.41) --
	( 11.72,103.43) --
	( 11.72,103.48) --
	( 11.75,103.52) --
	( 11.81,103.51) --
	( 11.91,103.56) --
	( 11.90,103.70) --
	( 11.88,103.76) --
	( 11.95,103.79) --
	( 11.99,103.81) --
	( 12.02,103.92) --
	( 12.01,104.01) --
	( 12.11,104.01) --
	( 12.20,103.96) --
	( 12.31,103.94) --
	( 12.39,103.97) --
	( 12.48,104.08) --
	( 12.57,104.13) --
	( 12.62,104.23) --
	( 12.65,104.28) --
	( 12.68,104.34) --
	( 12.72,104.43) --
	( 12.79,104.64) --
	( 12.85,104.73) --
	( 13.01,104.81) --
	( 13.03,104.93) --
	( 13.07,105.05) --
	( 13.12,105.09) --
	( 13.22,105.12) --
	( 13.26,105.12) --
	( 13.34,105.12) --
	( 13.35,105.12) --
	( 13.39,105.14) --
	( 13.40,105.13) --
	( 13.57,105.10) --
	( 13.69,105.03) --
	( 13.80,105.01) --
	( 13.88,104.99) --
	( 14.00,105.04) --
	( 14.13,105.17) --
	( 14.20,105.24) --
	( 14.24,105.35) --
	( 14.19,105.50) --
	( 14.25,105.70) --
	( 14.30,105.86) --
	( 14.27,106.00) --
	( 14.43,106.06) --
	( 14.53,105.99) --
	( 14.52,105.96) --
	( 14.49,105.90) --
	( 14.44,105.78) --
	( 14.47,105.72) --
	( 14.56,105.66) --
	( 14.60,105.61) --
	( 14.63,105.61) --
	( 14.64,105.59) --
	( 14.65,105.47) --
	( 14.67,105.26) --
	( 14.70,105.09) --
	( 14.80,104.94) --
	( 14.90,104.75) --
	( 14.92,104.67) --
	( 15.02,104.51) --
	( 15.10,104.28) --
	( 15.22,104.05) --
	( 15.41,103.76) --
	( 15.55,103.60) --
	( 15.65,103.52) --
	( 15.74,103.48) --
	( 15.84,103.45) --
	( 15.94,103.45) --
	( 15.98,103.44) --
	( 16.05,103.43) --
	( 16.10,103.46) --
	( 16.23,103.46) --
	( 16.34,103.47) --
	( 16.53,103.44) --
	( 16.64,103.39) --
	( 16.67,103.35) --
	( 16.75,103.31) --
	( 16.76,103.18) --
	( 16.80,103.07) --
	( 16.86,102.84) --
	( 17.00,102.61) --
	( 17.04,102.60) --
	( 17.15,102.51) --
	( 17.28,102.44) --
	( 17.44,102.25) --
	( 17.58,102.12) --
	( 17.80,102.02) --
	( 17.94,101.93) --
	( 18.08,101.84) --
	( 18.34,101.75) --
	( 18.61,101.65) --
	( 18.86,101.62) --
	( 19.02,101.60) --
	( 19.27,101.60) --
	( 19.39,101.58) --
	( 19.45,101.56) --
	( 19.55,101.54) --
	( 19.65,101.51) --
	( 19.71,101.44) --
	( 19.75,101.39) --
	( 19.84,101.33) --
	( 19.92,101.18) --
	( 20.10,100.95) --
	( 20.16,100.89) --
	( 20.23,100.82) --
	( 20.26,100.77) --
	( 20.30,100.65) --
	( 20.36,100.40) --
	( 20.43,100.12) --
	( 20.52, 99.87) --
	( 20.59, 99.70) --
	( 20.73, 99.52) --
	( 20.90, 99.47) --
	( 21.06, 99.35) --
	( 21.23, 99.16) --
	( 21.29, 99.07) --
	( 21.31, 98.98) --
	( 21.32, 98.90) --
	( 21.44, 98.82) --
	( 21.44, 98.69) --
	( 21.43, 98.63) --
	( 21.50, 98.66) --
	( 21.54, 98.67) --
	( 21.58, 98.63) --
	( 21.64, 98.64) --
	( 21.68, 98.64) --
	( 21.73, 98.68) --
	( 21.79, 98.68) --
	( 21.83, 98.68) --
	( 21.88, 98.67) --
	( 21.88, 98.64) --
	( 21.88, 98.57) --
	( 21.91, 98.40) --
	( 21.87, 98.17) --
	( 21.90, 97.97) --
	( 21.94, 97.86) --
	( 21.99, 97.72) --
	( 22.01, 97.65) --
	( 22.04, 97.49) --
	( 22.11, 97.40) --
	( 22.20, 97.35) --
	( 22.21, 97.34) --
	( 22.37, 97.35) --
	( 22.55, 97.27) --
	( 22.65, 97.21) --
	( 22.71, 97.16) --
	( 22.79, 97.11) --
	( 22.86, 97.09) --
	( 22.96, 97.00) --
	( 23.11, 96.88) --
	( 23.22, 96.75) --
	( 23.31, 96.63) --
	( 23.34, 96.58) --
	( 23.38, 96.57) --
	( 23.46, 96.54) --
	( 23.49, 96.52) --
	( 23.46, 96.48) --
	( 23.48, 96.33) --
	( 23.43, 96.16) --
	( 23.43, 96.05) --
	( 23.43, 95.96) --
	( 23.48, 95.91) --
	( 23.59, 95.90) --
	( 23.67, 95.86) --
	( 23.77, 95.79) --
	( 23.89, 95.76) --
	( 23.91, 95.75) --
	( 23.90, 95.72) --
	( 23.85, 95.68) --
	( 23.79, 95.67) --
	( 23.68, 95.68) --
	( 23.61, 95.74) --
	( 23.48, 95.77) --
	( 23.39, 95.81) --
	( 23.30, 95.88) --
	( 23.25, 95.97) --
	( 23.28, 96.12) --
	( 23.30, 96.25) --
	( 23.26, 96.29) --
	( 23.13, 96.39) --
	( 23.00, 96.54) --
	( 22.90, 96.71) --
	( 22.81, 96.78) --
	( 22.59, 96.99) --
	( 22.47, 97.04) --
	( 22.34, 97.11) --
	( 22.01, 97.17) --
	( 21.92, 97.18) --
	( 21.82, 97.24) --
	( 21.74, 97.28) --
	( 21.71, 97.35) --
	( 21.67, 97.42) --
	( 21.66, 97.53) --
	( 21.67, 97.72) --
	( 21.68, 97.87) --
	( 21.72, 98.03) --
	( 21.71, 98.12) --
	( 21.65, 98.24) --
	( 21.58, 98.27) --
	( 21.52, 98.33) --
	( 21.29, 98.47) --
	( 21.13, 98.58) --
	( 21.09, 98.63) --
	( 21.11, 98.88) --
	( 21.12, 99.07) --
	( 20.84, 99.33) --
	( 20.74, 99.39) --
	( 20.49, 99.37) --
	( 20.23, 99.59) --
	( 20.14, 99.84) --
	( 19.97,100.21) --
	( 19.96,100.36) --
	( 19.97,100.45) --
	( 19.99,100.59) --
	( 19.98,100.66) --
	( 19.86,100.92) --
	( 19.76,101.02) --
	( 19.57,101.14) --
	( 19.18,101.30) --
	( 18.92,101.34) --
	( 18.90,101.34) --
	( 18.45,101.36) --
	( 18.11,101.42) --
	( 17.63,101.58) --
	( 17.44,101.69) --
	( 17.38,101.75) --
	( 17.23,102.08) --
	( 17.15,102.16) --
	( 16.93,102.36) --
	( 16.86,102.43) --
	( 16.82,102.41) --
	( 16.76,102.42) --
	( 16.59,102.53) --
	( 16.46,102.71) --
	( 16.39,102.88) --
	( 16.31,103.01) --
	( 16.27,103.10) --
	( 16.17,103.19) --
	( 16.10,103.21) --
	( 15.89,103.16) --
	( 15.72,103.13) --
	( 15.48,102.99) --
	( 15.17,102.71) --
	( 15.11,102.66) --
	( 14.85,102.55) --
	( 14.70,102.52) --
	( 14.49,102.49) --
	( 14.40,102.51) --
	( 14.34,102.57) --
	( 14.36,102.63) --
	( 14.38,102.69) --
	( 14.48,102.74) --
	( 14.60,102.88) --
	( 14.67,102.97) --
	( 14.67,102.99) --
	( 14.67,103.01) --
	( 14.61,103.02) --
	( 14.49,103.03) --
	( 14.44,103.05) --
	( 14.43,103.10) --
	( 14.47,103.14) --
	( 14.59,103.15) --
	( 14.76,103.20) --
	( 14.97,103.19) --
	( 15.06,103.22) --
	( 15.10,103.28) --
	( 15.12,103.36) --
	( 15.10,103.51) --
	( 15.10,103.64) --
	( 15.04,103.74) --
	( 14.97,103.84) --
	( 14.95,103.98) --
	( 14.82,104.21) --
	( 14.72,104.42) --
	( 14.63,104.59) --
	( 14.48,104.76) --
	( 14.48,104.81) --
	( 14.38,104.85) --
	( 14.31,104.85) --
	( 14.21,104.82) --
	( 14.16,104.72) --
	( 14.12,104.62) --
	( 14.00,104.55) --
	( 13.91,104.57) --
	( 13.89,104.63) --
	( 13.89,104.68) --
	( 13.85,104.74) --
	( 13.82,104.75) --
	( 13.71,104.73) --
	( 13.64,104.61) --
	( 13.60,104.51) --
	( 13.49,104.43) --
	( 13.44,104.43) --
	( 13.43,104.49) --
	( 13.43,104.53) --
	( 13.46,104.61) --
	( 13.46,104.69) --
	( 13.47,104.73) --
	( 13.38,104.79) --
	( 13.34,104.77) --
	( 13.23,104.68) --
	( 13.16,104.64) --
	( 12.95,104.38) --
	( 12.89,104.24) --
	( 12.85,104.19) --
	( 12.77,104.08) --
	( 12.76,103.98) --
	( 12.76,103.90) --
	( 12.71,103.86) --
	( 12.62,103.77) --
	( 12.57,103.68) --
	( 12.50,103.61) --
	( 12.52,103.44) --
	( 12.58,103.44) --
	( 12.57,103.36) --
	( 12.57,103.32) --
	( 12.50,103.27) --
	( 12.50,103.24) --
	( 12.56,103.16) --
	( 12.52,103.09) --
	( 12.55,103.03) --
	( 12.63,103.01) --
	( 12.61,102.92) --
	( 12.57,102.86) --
	( 12.54,102.85) --
	( 12.42,102.85) --
	( 12.31,102.91) --
	( 12.18,102.94) --
	( 12.09,102.93) --
	( 12.03,102.93) --
	( 12.00,102.79) --
	( 12.06,102.74) --
	( 12.19,102.69) --
	( 12.25,102.68) --
	( 12.30,102.67) --
	( 12.33,102.59) --
	( 12.42,102.57) --
	( 12.49,102.50) --
	( 12.44,102.48) --
	( 12.36,102.48) --
	( 12.28,102.47) --
	( 12.18,102.45) --
	( 12.11,102.43) --
	( 12.13,102.37) --
	( 12.16,102.29) --
	( 12.16,102.19) --
	( 12.09,102.14) --
	( 12.05,102.09) --
	( 12.00,102.07) --
	( 12.03,102.01) --
	( 12.04,101.94) --
	( 12.00,101.89) --
	( 11.96,101.83) --
	( 11.85,101.74) --
	( 11.80,101.66) --
	( 11.77,101.60) --
	( 11.80,101.54) --
	( 11.82,101.49) --
	( 11.83,101.45) --
	( 11.81,101.37) --
	( 11.75,101.30) --
	( 11.66,101.23) --
	( 11.59,101.14) --
	( 11.53,101.11) --
	( 11.56,101.06) --
	( 11.59,101.03) --
	( 11.61,100.99) --
	( 11.64,100.94) --
	( 11.61,100.86) --
	( 11.58,100.77) --
	( 11.64,100.68) --
	( 11.70,100.51) --
	( 11.75,100.40) --
	( 11.72,100.32) --
	( 11.68,100.25) --
	( 11.53,100.21) --
	( 11.51,100.49) --
	( 11.42,100.60) --
	( 11.34,100.62) --
	( 11.29,100.61) --
	( 11.27,100.59) --
	( 11.21,100.52) --
	( 11.17,100.40) --
	( 11.09,100.31) --
	( 10.98,100.33) --
	( 10.89,100.46) --
	( 10.83,100.57) --
	( 10.65,100.61) --
	( 10.41,100.59) --
	( 10.13,100.58) --
	( 10.05,100.53) --
	(  9.88,100.43) --
	(  9.83,100.37) --
	(  9.79,100.35) --
	(  9.70,100.36) --
	(  9.72,100.31) --
	(  9.61,100.24) --
	(  9.38,100.24) --
	(  9.28,100.32) --
	(  9.20,100.35) --
	(  9.11,100.32) --
	(  9.09,100.25) --
	(  9.05,100.23) --
	(  8.98,100.23) --
	(  8.93,100.24) --
	(  8.86,100.25) --
	(  8.72,100.25) --
	(  8.72,100.23) --
	(  8.67,100.18) --
	(  8.63,100.09) --
	(  8.55,100.03) --
	(  8.49,100.04) --
	(  8.49,100.17) --
	(  8.50,100.32) --
	(  8.49,100.35) --
	(  8.45,100.39) --
	(  8.40,100.40) --
	(  8.33,100.39) --
	(  8.28,100.42) --
	(  8.21,100.58) --
	(  8.16,100.62) --
	(  8.09,100.66) --
	(  8.02,100.62) --
	(  8.04,100.55) --
	(  8.00,100.49) --
	(  7.94,100.50) --
	(  7.90,100.48) --
	(  7.89,100.47) --
	(  7.79,100.45) --
	(  7.75,100.48) --
	(  7.74,100.57) --
	(  7.80,100.62) --
	(  7.73,100.65) --
	(  7.67,100.64) --
	(  7.67,100.67) --
	(  7.66,100.71) --
	(  7.63,100.77) --
	(  7.57,100.74) --
	(  7.49,100.73) --
	(  7.47,100.79) --
	(  7.47,100.85) --
	(  7.45,100.94) --
	(  7.35,100.96) --
	(  7.20,100.93) --
	(  7.16,100.91) --
	(  7.09,100.87) --
	(  6.98,100.81) --
	(  6.92,100.84) --
	(  6.88,100.89) --
	(  6.85,100.91) --
	(  6.78,100.93) --
	(  6.59,101.05) --
	(  6.51,101.09) --
	(  6.41,101.09) --
	(  6.31,101.02) --
	(  6.26,100.98) --
	(  6.20,100.95) --
	(  6.08,100.98) --
	(  6.06,101.01) --
	(  6.07,101.06) --
	(  6.13,101.11) --
	(  6.18,101.14) --
	(  6.25,101.22) --
	(  6.32,101.29) --
	(  6.42,101.36) --
	(  6.50,101.51) --
	(  6.52,101.66) --
	(  6.53,101.76) --
	(  6.54,101.86) --
	(  6.51,101.97) --
	(  6.52,102.01) --
	(  6.56,102.02) --
	(  6.62,102.02) --
	(  6.66,102.05) --
	(  6.72,102.06) --
	(  6.70,102.13) --
	(  6.63,102.17) --
	(  6.50,102.20) --
	(  6.42,102.23) --
	(  6.32,102.30) --
	(  6.25,102.32) --
	(  6.22,102.38) --
	(  6.27,102.43) --
	(  6.32,102.50) --
	(  6.39,102.56) --
	(  6.42,102.62) --
	(  6.35,102.68) --
	(  6.29,102.72) --
	(  6.23,102.73) --
	(  6.16,102.76) --
	(  6.01,102.71) --
	(  5.93,102.68) --
	(  5.81,102.62) --
	(  5.73,102.68) --
	(  5.73,102.72) --
	(  5.71,102.76) --
	(  5.71,102.89) --
	(  5.62,102.97) --
	(  5.53,103.08) --
	(  5.47,103.20) --
	(  5.34,103.32) --
	(  5.29,103.19) --
	(  5.32,103.06) --
	(  5.19,103.11) --
	(  5.03,103.10) --
	(  4.92,103.02) --
	(  4.86,102.91) --
	(  4.92,102.81) --
	(  4.88,102.69) --
	(  4.96,102.59) --
	(  4.71,102.44) --
	(  4.81,102.37) --
	(  4.76,102.28) --
	(  4.62,102.13) --
	(  4.48,102.09) --
	(  4.34,102.22) --
	(  4.18,102.19) --
	(  4.26,102.13) --
	(  4.17,102.09) --
	(  4.02,102.06) --
	(  4.01,101.96) --
	(  3.94,101.90) --
	(  3.93,101.79) --
	(  3.76,101.74) --
	(  3.70,101.57) --
	(  3.50,101.55) --
	(  3.37,101.48) --
	(  3.34,101.47) --
	(  3.29,101.28) --
	(  3.29,101.14) --
	(  3.21,101.10) --
	(  3.06,101.22) --
	(  3.02,101.42) --
	(  2.95,101.67) --
	(  2.99,101.84) --
	(  3.18,101.90) --
	(  3.28,102.01) --
	(  3.23,102.16) --
	(  3.10,102.21) --
	(  3.02,102.14) --
	(  2.88,102.15) --
	(  2.57,102.32) --
	(  2.42,102.40) --
	(  2.29,102.44) --
	(  2.24,102.48) --
	(  2.18,102.49) --
	(  2.07,102.51) --
	(  1.90,102.61) --
	(  1.75,102.75) --
	(  1.60,102.96) --
	(  1.46,103.37) --
	(  1.24,103.70) --
	(  1.13,103.76) --
	(  1.04,103.76) --
	(  0.97,103.77) --
	(  0.90,103.74) --
	(  0.82,103.76) --
	(  0.66,103.73) --
	(  0.57,103.63) --
	(  0.49,103.53) --
	(  0.42,103.49) --
	(  0.38,103.47) --
	(  0.38,103.52) --
	(  0.28,103.53) --
	(  0.21,103.53) --
	(  0.18,103.45) --
	(  0.14,103.43) --
	(  0.07,103.43) --
	(  0.04,103.42) --
	(  0.00,103.34) --
	(  0.00,103.34);

\draw[color=drawColor,line cap=round,line join=round,fill opacity=0.00,] (  0.00, 83.82) --
	(  0.01, 83.79) --
	(  0.06, 83.69) --
	(  0.09, 83.37) --
	(  0.06, 82.72) --
	(  0.34, 81.81) --
	(  0.34, 81.51) --
	(  0.42, 81.16) --
	(  0.43, 80.71) --
	(  0.58, 80.59) --
	(  0.36, 80.48) --
	(  0.31, 80.46) --
	(  0.22, 80.42) --
	(  0.14, 80.51) --
	(  0.09, 80.62) --
	(  0.08, 80.87) --
	(  0.08, 81.05) --
	(  0.07, 81.14) --
	(  0.02, 81.31) --
	(  0.02, 81.41) --
	(  0.04, 81.51) --
	(  0.04, 81.54) --
	(  0.02, 81.63) --
	(  0.00, 81.70) --
	(  0.00, 81.74);

\draw[color=drawColor,line cap=round,line join=round,fill opacity=0.00,] (126.52, 65.58) --
	(126.42, 65.56) --
	(126.38, 65.56) --
	(126.30, 65.64) --
	(126.16, 65.69) --
	(126.09, 65.72) --
	(126.05, 65.72) --
	(125.98, 65.75) --
	(125.95, 65.80) --
	(125.96, 65.84) --
	(126.06, 65.85) --
	(126.15, 65.86) --
	(126.20, 65.90) --
	(126.24, 65.95) --
	(126.28, 66.07) --
	(126.33, 66.10) --
	(126.37, 66.11) --
	(126.50, 66.03) --
	(126.53, 65.93) --
	(126.54, 65.91) --
	(126.54, 65.86) --
	(126.52, 65.81) --
	(126.51, 65.74) --
	(126.56, 65.68) --
	(126.59, 65.66) --
	(126.60, 65.64) --
	(126.60, 65.60) --
	(126.52, 65.58);

\draw[color=drawColor,line cap=round,line join=round,fill opacity=0.00,] ( 11.30, 58.11) --
	( 11.23, 58.10) --
	( 11.18, 58.10) --
	( 11.06, 58.10) --
	( 11.00, 58.26) --
	( 10.98, 58.38) --
	( 10.94, 58.45) --
	( 10.86, 58.53) --
	( 10.79, 58.61) --
	( 10.70, 58.67) --
	( 10.65, 58.74) --
	( 10.44, 58.90) --
	( 10.19, 59.05) --
	( 10.09, 59.24) --
	( 10.00, 59.39) --
	(  9.88, 59.44) --
	(  9.68, 59.49) --
	(  9.46, 59.56) --
	(  9.03, 59.68) --
	(  8.81, 59.91) --
	(  8.54, 60.16) --
	(  8.36, 60.30) --
	(  8.12, 60.56) --
	(  8.04, 60.61) --
	(  7.96, 60.64) --
	(  7.89, 60.65) --
	(  7.82, 60.62) --
	(  7.79, 60.63) --
	(  7.70, 60.66) --
	(  7.56, 60.67) --
	(  7.44, 60.75) --
	(  7.39, 60.79) --
	(  7.35, 60.83) --
	(  7.31, 60.90) --
	(  7.24, 60.97) --
	(  7.23, 61.08) --
	(  7.18, 61.18) --
	(  7.07, 61.24) --
	(  6.93, 61.32) --
	(  6.79, 61.37) --
	(  6.72, 61.42) --
	(  6.64, 61.47) --
	(  6.57, 61.51) --
	(  6.42, 61.55) --
	(  6.33, 61.58) --
	(  6.23, 61.72) --
	(  6.17, 61.85) --
	(  6.10, 61.84) --
	(  6.04, 61.82) --
	(  5.95, 61.76) --
	(  5.83, 61.77) --
	(  5.75, 61.76) --
	(  5.52, 61.71) --
	(  5.37, 61.74) --
	(  5.26, 61.75) --
	(  5.03, 61.93) --
	(  4.95, 62.02) --
	(  4.90, 62.10) --
	(  4.71, 62.28) --
	(  4.62, 62.45) --
	(  4.60, 62.49) --
	(  4.60, 62.52) --
	(  4.62, 62.58) --
	(  4.62, 62.66) --
	(  4.61, 62.70) --
	(  4.59, 62.77) --
	(  4.56, 62.82) --
	(  4.59, 62.85) --
	(  4.64, 62.90) --
	(  4.59, 62.96) --
	(  4.55, 62.99) --
	(  4.55, 63.02) --
	(  4.60, 63.10) --
	(  4.60, 63.14) --
	(  4.57, 63.18) --
	(  4.58, 63.21) --
	(  4.63, 63.24) --
	(  4.70, 63.28) --
	(  4.72, 63.36) --
	(  4.71, 63.40) --
	(  4.74, 63.42) --
	(  4.77, 63.45) --
	(  4.85, 63.44) --
	(  4.94, 63.52) --
	(  5.00, 63.58) --
	(  5.01, 63.70) --
	(  4.99, 63.79) --
	(  4.94, 63.88) --
	(  4.91, 63.94) --
	(  4.88, 64.06) --
	(  4.89, 64.17) --
	(  4.77, 64.32) --
	(  4.64, 64.44) --
	(  4.57, 64.55) --
	(  4.52, 64.62) --
	(  4.48, 64.73) --
	(  4.38, 64.87) --
	(  4.24, 65.01) --
	(  3.94, 65.22) --
	(  3.78, 65.39) --
	(  3.70, 65.45) --
	(  3.64, 65.53) --
	(  3.63, 65.59) --
	(  3.65, 65.69) --
	(  3.69, 65.73) --
	(  3.72, 65.75) --
	(  3.83, 65.76) --
	(  4.10, 65.76) --
	(  4.14, 65.76) --
	(  4.18, 65.76) --
	(  4.31, 65.79) --
	(  4.33, 65.52) --
	(  4.11, 65.43) --
	(  4.10, 65.39) --
	(  4.31, 65.21) --
	(  4.70, 64.97) --
	(  4.83, 64.78) --
	(  5.18, 64.16) --
	(  5.37, 63.84) --
	(  5.44, 63.56) --
	(  5.46, 63.49) --
	(  5.55, 63.33) --
	(  5.64, 63.10) --
	(  5.79, 62.66) --
	(  5.86, 62.59) --
	(  6.25, 62.04) --
	(  6.51, 61.84) --
	(  6.62, 61.67) --
	(  7.23, 61.39) --
	(  7.49, 61.13) --
	(  7.57, 60.87) --
	(  7.68, 60.85) --
	(  8.01, 60.85) --
	(  8.21, 60.73) --
	(  8.51, 60.49) --
	(  8.88, 60.18) --
	(  9.10, 59.89) --
	( 10.00, 59.67) --
	( 10.16, 59.55) --
	( 10.34, 59.37) --
	( 10.53, 59.09) --
	( 10.67, 58.98) --
	( 11.00, 58.73) --
	( 11.13, 58.52) --
	( 11.26, 58.41) --
	( 11.36, 58.38) --
	( 11.45, 58.36) --
	( 11.55, 58.33) --
	( 11.63, 58.29) --
	( 11.70, 58.26) --
	( 11.77, 58.20) --
	( 11.84, 58.17) --
	( 11.90, 58.16) --
	( 11.97, 58.16) --
	( 12.05, 58.18) --
	( 12.10, 58.19) --
	( 12.17, 58.19) --
	( 12.28, 58.18) --
	( 12.39, 58.14) --
	( 12.49, 58.08) --
	( 12.57, 58.05) --
	( 12.67, 58.00) --
	( 12.71, 57.97) --
	( 12.69, 57.93) --
	( 12.66, 57.92) --
	( 12.59, 57.91) --
	( 12.55, 57.92) --
	( 12.48, 57.93) --
	( 12.40, 57.95) --
	( 12.33, 57.98) --
	( 12.23, 58.01) --
	( 12.15, 58.02) --
	( 12.07, 58.02) --
	( 12.00, 58.02) --
	( 11.94, 58.01) --
	( 11.85, 57.99) --
	( 11.78, 57.99) --
	( 11.68, 58.01) --
	( 11.59, 58.06) --
	( 11.52, 58.12) --
	( 11.39, 58.20) --
	( 11.33, 58.20) --
	( 11.29, 58.17) --
	( 11.30, 58.11);

\draw[color=drawColor,line cap=round,line join=round,fill opacity=0.00,] ( 26.54, 59.38) --
	( 26.65, 59.30) --
	( 26.64, 59.25) --
	( 26.60, 59.19) --
	( 26.61, 59.15) --
	( 26.63, 59.09) --
	( 26.63, 58.94) --
	( 26.64, 58.72) --
	( 26.67, 58.63) --
	( 26.59, 58.51) --
	( 26.66, 58.48) --
	( 26.86, 58.47) --
	( 27.10, 58.36) --
	( 27.15, 58.31) --
	( 27.39, 58.27) --
	( 27.80, 58.29) --
	( 27.96, 58.18) --
	( 28.06, 58.15) --
	( 27.99, 58.08) --
	( 27.53, 58.08) --
	( 27.42, 58.08) --
	( 27.35, 58.02) --
	( 27.24, 58.06) --
	( 27.08, 58.02) --
	( 27.02, 57.94) --
	( 26.92, 58.05) --
	( 26.76, 58.12) --
	( 26.75, 57.99) --
	( 26.64, 58.01) --
	( 26.52, 58.25) --
	( 26.40, 58.30) --
	( 26.39, 58.20) --
	( 26.36, 58.11) --
	( 26.31, 58.03) --
	( 26.25, 57.98) --
	( 26.23, 57.93) --
	( 26.12, 57.94) --
	( 26.01, 57.95) --
	( 25.95, 57.89) --
	( 25.91, 57.79) --
	( 25.92, 57.71) --
	( 25.93, 57.59) --
	( 25.89, 57.52) --
	( 25.80, 57.46) --
	( 25.77, 57.35) --
	( 25.77, 57.29) --
	( 25.80, 57.26) --
	( 25.79, 57.23) --
	( 25.67, 57.17) --
	( 25.58, 57.11) --
	( 25.45, 57.10) --
	( 25.41, 57.03) --
	( 25.33, 56.93) --
	( 25.29, 56.83) --
	( 25.26, 56.76) --
	( 25.24, 56.72) --
	( 25.21, 56.72) --
	( 25.21, 56.69) --
	( 25.17, 56.66) --
	( 25.11, 56.68) --
	( 25.02, 56.68) --
	( 24.89, 56.66) --
	( 24.85, 56.69) --
	( 24.81, 56.71) --
	( 24.76, 56.75) --
	( 24.62, 56.85) --
	( 24.59, 56.90) --
	( 24.56, 56.93) --
	( 24.55, 57.00) --
	( 24.62, 57.12) --
	( 24.61, 57.22) --
	( 24.63, 57.29) --
	( 24.65, 57.36) --
	( 24.63, 57.46) --
	( 24.64, 57.51) --
	( 24.70, 57.57) --
	( 24.75, 57.58) --
	( 24.83, 57.60) --
	( 24.90, 57.67) --
	( 24.93, 57.77) --
	( 24.97, 57.85) --
	( 25.01, 57.94) --
	( 25.06, 58.01) --
	( 25.09, 58.02) --
	( 25.13, 58.06) --
	( 25.22, 58.09) --
	( 25.29, 58.10) --
	( 25.33, 58.14) --
	( 25.41, 58.18) --
	( 25.48, 58.21) --
	( 25.56, 58.34) --
	( 25.80, 58.50) --
	( 26.03, 58.66) --
	( 26.14, 58.68) --
	( 26.21, 58.65) --
	( 26.32, 58.71) --
	( 26.36, 58.72) --
	( 26.40, 58.83) --
	( 26.43, 58.87) --
	( 26.44, 58.96) --
	( 26.45, 59.14) --
	( 26.42, 59.22) --
	( 26.36, 59.31) --
	( 26.32, 59.33) --
	( 26.24, 59.36) --
	( 26.19, 59.33) --
	( 26.12, 59.34) --
	( 26.06, 59.38) --
	( 25.92, 59.46) --
	( 25.85, 59.51) --
	( 25.82, 59.61) --
	( 25.83, 59.75) --
	( 25.82, 59.93) --
	( 25.77, 60.02) --
	( 25.66, 60.12) --
	( 25.63, 60.20) --
	( 25.40, 60.38) --
	( 25.24, 60.43) --
	( 25.08, 60.51) --
	( 25.04, 60.54) --
	( 24.97, 60.58) --
	( 25.01, 60.61) --
	( 25.03, 60.68) --
	( 25.15, 60.75) --
	( 25.22, 60.69) --
	( 25.45, 60.52) --
	( 25.57, 60.41) --
	( 25.64, 60.32) --
	( 25.76, 60.25) --
	( 25.92, 60.23) --
	( 25.96, 60.15) --
	( 26.02, 60.09) --
	( 26.04, 59.96) --
	( 26.11, 59.83) --
	( 26.16, 59.75) --
	( 26.21, 59.71) --
	( 26.30, 59.62) --
	( 26.36, 59.54) --
	( 26.39, 59.51) --
	( 26.42, 59.49) --
	( 26.54, 59.38);

\draw[color=drawColor,line cap=round,line join=round,fill opacity=0.00,] (129.10, 59.70) --
	(129.12, 59.58) --
	(129.27, 59.60) --
	(129.42, 59.50) --
	(129.43, 59.39) --
	(129.30, 59.29) --
	(129.33, 59.21) --
	(129.41, 59.19) --
	(129.46, 59.11) --
	(129.42, 59.02) --
	(129.28, 59.00) --
	(129.26, 59.10) --
	(129.11, 59.24) --
	(129.17, 59.33) --
	(129.10, 59.45) --
	(128.93, 59.48) --
	(128.72, 59.49) --
	(128.68, 59.59) --
	(128.65, 59.74) --
	(128.52, 59.77) --
	(128.42, 59.87) --
	(128.42, 59.97) --
	(128.45, 60.04) --
	(128.52, 60.00) --
	(128.53, 59.92) --
	(128.57, 59.89) --
	(128.71, 59.87) --
	(129.02, 59.79) --
	(129.10, 59.70);

\draw[color=drawColor,line cap=round,line join=round,fill opacity=0.00,] (  2.90, 25.62) --
	(  2.89, 25.52) --
	(  2.89, 25.50) --
	(  2.87, 25.48) --
	(  2.75, 25.27) --
	(  2.63, 25.31) --
	(  2.52, 25.29) --
	(  2.46, 25.18) --
	(  2.13, 25.28) --
	(  2.27, 25.35) --
	(  2.13, 25.44) --
	(  2.08, 25.51) --
	(  2.03, 25.58) --
	(  2.00, 25.63) --
	(  2.01, 25.69) --
	(  1.96, 25.77) --
	(  1.87, 25.95) --
	(  1.96, 25.98) --
	(  1.99, 25.90) --
	(  2.11, 25.81) --
	(  2.25, 25.48) --
	(  2.24, 25.57) --
	(  2.20, 25.70) --
	(  2.19, 25.84) --
	(  2.07, 25.91) --
	(  2.05, 25.97) --
	(  2.00, 26.01) --
	(  2.32, 26.05) --
	(  2.32, 26.05) --
	(  2.37, 26.05) --
	(  2.43, 26.06) --
	(  2.57, 25.89) --
	(  2.68, 25.73) --
	(  2.77, 25.63) --
	(  2.90, 25.62);

\draw[color=drawColor,line cap=round,line join=round,fill opacity=0.00,] (  3.31, 23.30) --
	(  3.20, 23.29) --
	(  3.18, 23.28) --
	(  3.14, 23.29) --
	(  2.84, 23.31) --
	(  2.72, 23.69) --
	(  2.72, 23.77) --
	(  2.75, 23.85) --
	(  2.77, 23.86) --
	(  2.81, 24.03) --
	(  2.69, 24.17) --
	(  2.63, 24.26) --
	(  2.60, 24.34) --
	(  2.55, 24.44) --
	(  2.57, 24.53) --
	(  2.60, 24.64) --
	(  2.72, 24.81) --
	(  2.95, 25.23) --
	(  2.75, 25.27) --
	(  2.87, 25.48) --
	(  2.89, 25.50) --
	(  2.89, 25.52) --
	(  2.90, 25.62) --
	(  2.94, 25.61) --
	(  3.12, 25.52) --
	(  3.20, 25.25) --
	(  3.21, 25.01) --
	(  3.17, 24.87) --
	(  3.04, 24.74) --
	(  2.97, 24.64) --
	(  3.00, 24.56) --
	(  2.94, 24.53) --
	(  2.86, 24.51) --
	(  2.86, 24.39) --
	(  2.92, 24.26) --
	(  3.08, 24.16) --
	(  3.24, 23.99) --
	(  3.32, 23.81) --
	(  3.36, 23.61) --
	(  3.31, 23.30);

\draw[color=drawColor,line cap=round,line join=round,fill opacity=0.00,] (  3.31, 23.30) --
	(  3.28, 23.18) --
	(  3.16, 23.18) --
	(  3.09, 23.18) --
	(  3.05, 23.19) --
	(  2.98, 23.21) --
	(  2.88, 23.24) --
	(  2.85, 23.25) --
	(  2.85, 23.27) --
	(  2.84, 23.29) --
	(  2.84, 23.31) --
	(  3.14, 23.29) --
	(  3.18, 23.28) --
	(  3.20, 23.29) --
	(  3.31, 23.30);
\definecolor[named]{drawColor}{rgb}{0.75,0.75,0.75}

\draw[color=drawColor,line cap=round,line join=round,fill opacity=0.00,] (  0.00, 79.42) --
	(  0.40, 79.85) --
	(  0.00, 81.62);

\draw[color=drawColor,line cap=round,line join=round,fill opacity=0.00,] (  0.00,103.56) --
	(  4.78,103.45) --
	(  8.03,100.77) --
	( 11.14,101.05) --
	( 12.55,103.45) --
	( 13.54,124.65);

\draw[color=drawColor,line cap=round,line join=round,fill opacity=0.00,] (136.91,124.65) --
	( 82.37,124.65) --
	( 80.67,124.79) --
	( 13.54,124.65);

\draw[color=drawColor,line cap=round,line join=round,fill opacity=0.00,] (136.91,124.65) --
	(136.49, 45.23) --
	(136.49,  0.00);

\draw[color=drawColor,line cap=round,line join=round,fill opacity=0.00,] (  0.00, 12.46) --
	( 39.96,  0.00);

\draw[color=drawColor,line cap=round,line join=round,fill opacity=0.00,] (  0.00, 73.30) --
	(  0.54, 72.08) --
	(  0.40, 70.67) --
	(  3.51, 68.26) --
	(  5.35, 62.33) --
	( 11.14, 58.09) --
	(  8.88, 55.69) --
	(  4.22, 52.72) --
	(  1.25, 48.90) --
	(  1.67, 44.10) --
	(  1.53, 41.70) --
	(  0.00, 39.78);

\draw[color=drawColor,line cap=round,line join=round,fill opacity=0.00,] (  0.00, 27.45) --
	(  1.25, 27.14) --
	(  2.80, 24.46) --
	(  0.97, 20.22) --
	(  0.00, 20.07);

\draw[color=drawColor,line cap=round,line join=round,fill opacity=0.00,] (252.94,124.59) --
	(218.74,124.37) --
	(211.67,124.37) --
	(136.91,124.65);

\draw[color=drawColor,line cap=round,line join=round,fill opacity=0.00,] (136.63,223.29) --
	(136.49,183.02) --
	(136.91,124.65);

\draw[color=drawColor,line cap=round,line join=round,fill opacity=0.00,] (252.94,223.57) --
	(203.76,223.57) --
	(136.63,223.29);

\draw[color=drawColor,line cap=round,line join=round,fill opacity=0.00,] (  6.43,361.35) --
	(  6.05,359.38) --
	(  4.78,358.96) --
	(  5.35,357.26) --
	(  3.23,353.02) --
	(  3.65,350.90) --
	(  2.10,349.91) --
	(  3.23,348.50) --
	(  2.24,347.65) --
	(  4.50,346.10) --
	(  5.35,343.70) --
	(  1.53,342.14) --
	(  2.38,338.33) --
	(  0.68,337.48) --
	(  1.25,335.92) --
	(  6.34,333.38) --
	(  8.03,333.80) --
	(  8.88,335.22) --
	( 10.58,335.22) --
	( 14.81,339.46) --
	( 19.34,337.20) --
	( 19.05,335.50) --
	( 20.47,334.93) --
	( 20.47,333.38) --
	( 25.13,325.32) --
	( 28.24,323.49) --
	( 28.38,321.23) --
	( 27.11,320.52) --
	( 30.36,317.41) --
	( 33.61,317.69) --
	( 37.85,313.17) --
	( 37.28,312.32) --
	( 38.70,309.21) --
	( 43.08,306.53) --
	( 43.64,307.80) --
	( 46.19,309.92) --
	( 53.96,308.51) --
	( 55.37,308.79) --
	( 55.51,310.63) --
	( 57.92,311.62) --
	( 59.61,310.49) --
	( 67.24,311.33) --
	( 68.66,310.34) --
	( 72.33,311.05) --
	( 76.71,310.91) --
	( 75.86,312.75) --
	( 76.71,314.30) --
	( 79.11,316.28) --
	( 82.65,312.46) --
	( 87.17,309.50);

\draw[color=drawColor,line cap=round,line join=round,fill opacity=0.00,] ( 87.31,248.02) --
	( 87.31,279.96) --
	( 87.03,285.19) --
	( 87.17,309.50);

\draw[color=drawColor,line cap=round,line join=round,fill opacity=0.00,] ( 87.31,248.02) --
	( 61.17,248.16) --
	( 13.40,248.16);

\draw[color=drawColor,line cap=round,line join=round,fill opacity=0.00,] ( 13.40,248.16) --
	(  7.04,248.31) --
	(  0.00,248.31);

\draw[color=drawColor,line cap=round,line join=round,fill opacity=0.00,] (252.94,322.22) --
	(235.84,322.22) --
	(234.42,322.07) --
	(194.86,322.22) --
	(186.52,322.22) --
	( 87.45,321.93) --
	( 87.17,309.50);

\draw[color=drawColor,line cap=round,line join=round,fill opacity=0.00,] ( 13.40,248.16) --
	( 13.54,187.40) --
	( 13.40,165.92) --
	( 13.54,124.65);

\draw[color=drawColor,line cap=round,line join=round,fill opacity=0.00,] (252.94,  1.28) --
	(196.41,  1.28) --
	(196.17,  0.00);

\draw[color=drawColor,line cap=round,line join=round,fill opacity=0.00,] ( 87.31,248.02) --
	( 87.17,223.29) --
	(136.63,223.29);
\definecolor[named]{drawColor}{rgb}{0.00,0.00,0.00}

\draw[color=drawColor,line cap=round,line join=round,fill opacity=0.00,] (225.45,187.65) circle (  1.35);

\draw[color=drawColor,line cap=round,line join=round,fill opacity=0.00,] (179.57,135.36) circle (  1.35);

\draw[color=drawColor,line cap=round,line join=round,fill opacity=0.00,] (224.96,185.92) circle (  1.35);

\draw[color=drawColor,line cap=round,line join=round,fill opacity=0.00,] (182.04,185.43) circle (  1.35);

\draw[color=drawColor,line cap=round,line join=round,fill opacity=0.00,] (203.25,201.95) circle (  1.35);

\draw[color=drawColor,line cap=round,line join=round,fill opacity=0.00,] (178.34,187.15) circle (  1.35);

\draw[color=drawColor,line cap=round,line join=round,fill opacity=0.00,] (197.83,162.74) circle (  1.35);

\draw[color=drawColor,line cap=round,line join=round,fill opacity=0.00,] (175.38,134.37) circle (  1.35);

\draw[color=drawColor,line cap=round,line join=round,fill opacity=0.00,] (175.38,134.37) circle (  1.35);

\draw[color=drawColor,line cap=round,line join=round,fill opacity=0.00,] (203.01,204.67) circle (  1.35);

\draw[color=drawColor,line cap=round,line join=round,fill opacity=0.00,] (156.88,155.58) circle (  1.35);

\draw[color=drawColor,line cap=round,line join=round,fill opacity=0.00,] (162.31,131.17) circle (  1.35);

\draw[color=drawColor,line cap=round,line join=round,fill opacity=0.00,] (209.42,181.24) circle (  1.35);

\draw[color=drawColor,line cap=round,line join=round,fill opacity=0.00,] (186.48,201.21) circle (  1.35);

\draw[color=drawColor,line cap=round,line join=round,fill opacity=0.00,] (205.23,125.00) circle (  1.35);

\draw[color=drawColor,line cap=round,line join=round,fill opacity=0.00,] (224.46,204.91) circle (  1.35);

\draw[color=drawColor,line cap=round,line join=round,fill opacity=0.00,] (203.99,130.43) circle (  1.35);

\draw[color=drawColor,line cap=round,line join=round,fill opacity=0.00,] (173.41,136.59) circle (  1.35);

\draw[color=drawColor,line cap=round,line join=round,fill opacity=0.00,] (159.84,138.07) circle (  1.35);

\draw[color=drawColor,line cap=round,line join=round,fill opacity=0.00,] (201.03,210.59) circle (  1.35);

\draw[color=drawColor,line cap=round,line join=round,fill opacity=0.00,] (202.02,199.98) circle (  1.35);

\draw[color=drawColor,line cap=round,line join=round,fill opacity=0.00,] (159.84,130.92) circle (  1.35);

\draw[color=drawColor,line cap=round,line join=round,fill opacity=0.00,] (183.03,176.06) circle (  1.35);

\draw[color=drawColor,line cap=round,line join=round,fill opacity=0.00,] (203.99,182.96) circle (  1.35);

\draw[color=drawColor,line cap=round,line join=round,fill opacity=0.00,] (161.32,174.82) circle (  1.35);

\draw[color=drawColor,line cap=round,line join=round,fill opacity=0.00,] (186.73,133.88) circle (  1.35);

\draw[color=drawColor,line cap=round,line join=round,fill opacity=0.00,] (211.64,185.67) circle (  1.35);

\draw[color=drawColor,line cap=round,line join=round,fill opacity=0.00,] (176.37,137.33) circle (  1.35);

\draw[color=drawColor,line cap=round,line join=round,fill opacity=0.00,] (177.11,135.85) circle (  1.35);

\draw[color=drawColor,line cap=round,line join=round,fill opacity=0.00,] (203.01,178.52) circle (  1.35);

\draw[color=drawColor,line cap=round,line join=round,fill opacity=0.00,] (185.00,175.07) circle (  1.35);

\draw[color=drawColor,line cap=round,line join=round,fill opacity=0.00,] (177.85,175.32) circle (  1.35);

\draw[color=drawColor,line cap=round,line join=round,fill opacity=0.00,] (203.25,161.50) circle (  1.35);

\draw[color=drawColor,line cap=round,line join=round,fill opacity=0.00,] (174.40,174.58) circle (  1.35);

\draw[color=drawColor,line cap=round,line join=round,fill opacity=0.00,] (223.97,204.67) circle (  1.35);

\draw[color=drawColor,line cap=round,line join=round,fill opacity=0.00,] (206.95,156.57) circle (  1.35);

\draw[color=drawColor,line cap=round,line join=round,fill opacity=0.00,] (201.03,204.17) circle (  1.35);

\draw[color=drawColor,line cap=round,line join=round,fill opacity=0.00,] (176.62,138.07) circle (  1.35);

\draw[color=drawColor,line cap=round,line join=round,fill opacity=0.00,] (182.78,200.47) circle (  1.35);

\draw[color=drawColor,line cap=round,line join=round,fill opacity=0.00,] (186.48,174.33) circle (  1.35);

\draw[color=drawColor,line cap=round,line join=round,fill opacity=0.00,] (162.31,134.37) circle (  1.35);

\draw[color=drawColor,line cap=round,line join=round,fill opacity=0.00,] (184.26,139.30) circle (  1.35);

\draw[color=drawColor,line cap=round,line join=round,fill opacity=0.00,] (175.63,135.11) circle (  1.35);

\draw[color=drawColor,line cap=round,line join=round,fill opacity=0.00,] (222.98,202.20) circle (  1.35);

\draw[color=drawColor,line cap=round,line join=round,fill opacity=0.00,] (177.85,131.91) circle (  1.35);

\draw[color=drawColor,line cap=round,line join=round,fill opacity=0.00,] (209.42,184.69) circle (  1.35);

\draw[color=drawColor,line cap=round,line join=round,fill opacity=0.00,] (201.77,206.64) circle (  1.35);

\draw[color=drawColor,line cap=round,line join=round,fill opacity=0.00,] (183.77,198.75) circle (  1.35);

\draw[color=drawColor,line cap=round,line join=round,fill opacity=0.00,] (199.55,131.91) circle (  1.35);

\draw[color=drawColor,line cap=round,line join=round,fill opacity=0.00,] (206.21,183.21) circle (  1.35);

\draw[color=drawColor,line cap=round,line join=round,fill opacity=0.00,] (179.82,131.91) circle (  1.35);

\draw[color=drawColor,line cap=round,line join=round,fill opacity=0.00,] (210.41,203.93) circle (  1.35);

\draw[color=drawColor,line cap=round,line join=round,fill opacity=0.00,] (200.05,131.91) circle (  1.35);

\draw[color=drawColor,line cap=round,line join=round,fill opacity=0.00,] ( 79.93,159.53) circle (  1.35);

\draw[color=drawColor,line cap=round,line join=round,fill opacity=0.00,] (104.59,207.38) circle (  1.35);

\draw[color=drawColor,line cap=round,line join=round,fill opacity=0.00,] ( 82.15,176.06) circle (  1.35);

\draw[color=drawColor,line cap=round,line join=round,fill opacity=0.00,] (130.74,136.59) circle (  1.35);

\draw[color=drawColor,line cap=round,line join=round,fill opacity=0.00,] (113.23,210.09) circle (  1.35);

\draw[color=drawColor,line cap=round,line join=round,fill opacity=0.00,] ( 76.48,139.06) circle (  1.35);

\draw[color=drawColor,line cap=round,line join=round,fill opacity=0.00,] ( 87.08,204.17) circle (  1.35);

\draw[color=drawColor,line cap=round,line join=round,fill opacity=0.00,] ( 84.86,203.19) circle (  1.35);

\draw[color=drawColor,line cap=round,line join=round,fill opacity=0.00,] (106.81,207.63) circle (  1.35);

\draw[color=drawColor,line cap=round,line join=round,fill opacity=0.00,] (104.10,209.85) circle (  1.35);

\draw[color=drawColor,line cap=round,line join=round,fill opacity=0.00,] (133.94,156.57) circle (  1.35);

\draw[color=drawColor,line cap=round,line join=round,fill opacity=0.00,] ( 83.88,184.19) circle (  1.35);

\draw[color=drawColor,line cap=round,line join=round,fill opacity=0.00,] (124.82,207.87) circle (  1.35);

\draw[color=drawColor,line cap=round,line join=round,fill opacity=0.00,] ( 84.62,180.99) circle (  1.35);

\draw[color=drawColor,line cap=round,line join=round,fill opacity=0.00,] (103.11,206.89) circle (  1.35);

\draw[color=drawColor,line cap=round,line join=round,fill opacity=0.00,] ( 82.15,178.77) circle (  1.35);

\draw[color=drawColor,line cap=round,line join=round,fill opacity=0.00,] (105.83,212.31) circle (  1.35);

\draw[color=drawColor,line cap=round,line join=round,fill opacity=0.00,] ( 85.36,201.46) circle (  1.35);

\draw[color=drawColor,line cap=round,line join=round,fill opacity=0.00,] ( 99.17,208.86) circle (  1.35);

\draw[color=drawColor,line cap=round,line join=round,fill opacity=0.00,] (128.03,209.60) circle (  1.35);

\draw[color=drawColor,line cap=round,line join=round,fill opacity=0.00,] ( 98.67,188.39) circle (  1.35);

\draw[color=drawColor,line cap=round,line join=round,fill opacity=0.00,] ( 75.49,137.09) circle (  1.35);

\draw[color=drawColor,line cap=round,line join=round,fill opacity=0.00,] (183.27,224.15) circle (  1.35);

\draw[color=drawColor,line cap=round,line join=round,fill opacity=0.00,] (133.94,257.70) circle (  1.35);

\draw[color=drawColor,line cap=round,line join=round,fill opacity=0.00,] (104.35,262.14) circle (  1.35);

\draw[color=drawColor,line cap=round,line join=round,fill opacity=0.00,] (111.25,275.21) circle (  1.35);

\draw[color=drawColor,line cap=round,line join=round,fill opacity=0.00,] (123.83,274.47) circle (  1.35);

\draw[color=drawColor,line cap=round,line join=round,fill opacity=0.00,] (197.58,225.63) circle (  1.35);

\draw[color=drawColor,line cap=round,line join=round,fill opacity=0.00,] (184.75,225.14) circle (  1.35);

\draw[color=drawColor,line cap=round,line join=round,fill opacity=0.00,] (105.33,254.98) circle (  1.35);

\draw[color=drawColor,line cap=round,line join=round,fill opacity=0.00,] (103.36,255.97) circle (  1.35);

\draw[color=drawColor,line cap=round,line join=round,fill opacity=0.00,] (104.59,259.42) circle (  1.35);

\draw[color=drawColor,line cap=round,line join=round,fill opacity=0.00,] (198.57,223.41) circle (  1.35);
\end{scope}
\begin{scope}
\path[clip] (  0.00,  0.00) rectangle (252.94,361.35);
\definecolor[named]{drawColor}{rgb}{0.00,0.00,0.00}

\draw[color=drawColor,line cap=round,line join=round,fill opacity=0.00,] ( 63.90, 61.20) -- (211.89, 61.20);

\draw[color=drawColor,line cap=round,line join=round,fill opacity=0.00,] ( 63.90, 61.20) -- ( 63.90, 55.20);

\draw[color=drawColor,line cap=round,line join=round,fill opacity=0.00,] ( 88.56, 61.20) -- ( 88.56, 55.20);

\draw[color=drawColor,line cap=round,line join=round,fill opacity=0.00,] (113.23, 61.20) -- (113.23, 55.20);

\draw[color=drawColor,line cap=round,line join=round,fill opacity=0.00,] (137.89, 61.20) -- (137.89, 55.20);

\draw[color=drawColor,line cap=round,line join=round,fill opacity=0.00,] (162.56, 61.20) -- (162.56, 55.20);

\draw[color=drawColor,line cap=round,line join=round,fill opacity=0.00,] (187.22, 61.20) -- (187.22, 55.20);

\draw[color=drawColor,line cap=round,line join=round,fill opacity=0.00,] (211.89, 61.20) -- (211.89, 55.20);

\node[color=drawColor,anchor=base,inner sep=0pt, outer sep=0pt, scale=  1.00] at ( 63.90, 37.20) {-112%
};

\node[color=drawColor,anchor=base,inner sep=0pt, outer sep=0pt, scale=  1.00] at (113.23, 37.20) {-110%
};

\node[color=drawColor,anchor=base,inner sep=0pt, outer sep=0pt, scale=  1.00] at (162.56, 37.20) {-108%
};

\node[color=drawColor,anchor=base,inner sep=0pt, outer sep=0pt, scale=  1.00] at (211.89, 37.20) {-106%
};

\draw[color=drawColor,line cap=round,line join=round,fill opacity=0.00,] ( 49.20,100.09) -- ( 49.20,297.41);

\draw[color=drawColor,line cap=round,line join=round,fill opacity=0.00,] ( 49.20,100.09) -- ( 43.20,100.09);

\draw[color=drawColor,line cap=round,line join=round,fill opacity=0.00,] ( 49.20,149.42) -- ( 43.20,149.42);

\draw[color=drawColor,line cap=round,line join=round,fill opacity=0.00,] ( 49.20,198.75) -- ( 43.20,198.75);

\draw[color=drawColor,line cap=round,line join=round,fill opacity=0.00,] ( 49.20,248.08) -- ( 43.20,248.08);

\draw[color=drawColor,line cap=round,line join=round,fill opacity=0.00,] ( 49.20,297.41) -- ( 43.20,297.41);

\node[rotate= 90.00,color=drawColor,anchor=base,inner sep=0pt, outer sep=0pt, scale=  1.00] at ( 37.20,100.09) {36%
};

\node[rotate= 90.00,color=drawColor,anchor=base,inner sep=0pt, outer sep=0pt, scale=  1.00] at ( 37.20,149.42) {38%
};

\node[rotate= 90.00,color=drawColor,anchor=base,inner sep=0pt, outer sep=0pt, scale=  1.00] at ( 37.20,198.75) {40%
};

\node[rotate= 90.00,color=drawColor,anchor=base,inner sep=0pt, outer sep=0pt, scale=  1.00] at ( 37.20,248.08) {42%
};

\node[rotate= 90.00,color=drawColor,anchor=base,inner sep=0pt, outer sep=0pt, scale=  1.00] at ( 37.20,297.41) {44%
};
\end{scope}
\end{tikzpicture}

   %\includegraphics[width=.8\textwidth]{map_both.pdf} 
   \caption{Locations of snow data sites used in this study indicated by an open circle.}
   \label{fig:map-snow}
\end{figure}


%%%%%%%%%%%%%%%%%%%%%%%%%%%%%%%%%%%%%%%%%%%%%%%%%%%%%%%%%%%%
%%%%%%%%%%%%%%%%%%%%%%%%%%%%%%%%%%%%%%%%%%%%%%%%%%%%%%%%%%%%
\section{Methodology}

We use the same framework as \cite[see their section 4]{Bracken:2010cw} with some notable changes. Most importantly we (1) expand the disaggregation to include all 20 natural flow nodes in the UCRB and (2) we include the flexible disaggregation method of \cite{Nowak:2010ha} instead of computationally intensive the disaggregation method of \cite{Prairie:2007jf}.  As in \cite{Bracken:2010cw} we created an ``index'' gauge peak season flow which the sum of peak season flows at all the 20 locations; this is forecasted using large scale climate variables in a multi-model ensemble approach \cite{Bracken:2010cw,Regonda2006}; which is subsequently disaggregated to ensemble monthly flows at all the 20 locations.  

The nonparametric disaggregation method of \cite{Nowak:2010ha} is known as proportion disaggregation is a computationally simple approach that preserves the summability criteria of the network (intervening flows upstream sum to downstream total flow at Lees Ferry). The method is able to simultaneously conduct space and time disaggregation through the use of proportion matrices (matrices whose contents sum to unity). Figure \ref{fig:disag} shows schematically how the index gauge peak season flow is split simultaneously to in space (at all twenty sites) and time (for all the months in the peak season). In space-time disaggregation there is one proportion matrix for each year in the historical record.  For a given seasonal flow value to be disaggregated, the method identifies K nearest neighbors of the flow value and selects a historical proportion vector which is multiplied by the total flow to obtain space-time disaggregated values. For more details, the interested reader is referred to \cite{Nowak:2010ha}.

\begin{figure}[htbp] %  figure placement: here, top, bottom, or page
   \centering
   %\includegraphics[width=.4\textwidth]{figures/disag-schematic.pdf} 
   % !TEX root = disag-schematic.tex

\tikzset{paint/.style={ draw=#1!50!black, fill=#1!50 },decorate with/.style=
{decorate,decoration={shape backgrounds,shape=#1,shape size=1mm,anchor=west}}}

\begin{tikzpicture}
    [
    node distance=4mm and 9mm,
    scale=.8,
    block/.style ={
        rectangle, 
        draw=gray!80, 
        thick, 
        top color=gray!20, 
        bottom color=white,
        text badly centered, 
        text width=6.5em,
        text height=.7em,
        rounded corners
    },
    decision/.style={
        diamond, 
        draw=gray!80, 
        thick, 
        top color=gray!20, 
        bottom color=white,
	   % text width=5em, 
	    text centered, 
	    inner sep=0pt
	},
	a/.style={
		-stealth',
		draw=gray
	}
    ]

    \node (Index)[block,text width=3em,draw=black,very thick] at (3.5,0) {Index};
    \node (apr1) [block] at (4,5) {April 1 Site 1};
    \node (may1) [block] at (6,4) {May 1 Site 1};
    \node (jun1) [block] at (7,3) {June 1 Site 1};
    \node (jul1) [block] at (8,2) {July 1 Site 1};
    \node (apr20)[block] at (8,-2) {April 1 Site 20};
    \node (may20)[block] at (7,-3) {May 1 Site 20};
    \node (jun20)[block] at (6,-4) {June 1 Site 20};
    \node (jul20)[block] at (4,-5) {July 1 Site 20};
    \draw[decorate with=circle,paint=black] (jul1) -- (apr20);
    %\node (jul1x) [block] at (8,2) {July 1 Site 1};
    %\node (apr20x)[block] at (8,-2) {April 1 Site 20};
    
    \draw[a] (Index) edge (apr1);
    \draw[a] (Index) edge (may1.200);
    \draw[a] (Index) edge (jun1.200);
    \draw[a] (Index) edge (jul1);
    \draw[a] (Index) edge (apr20);
    \draw[a] (Index) edge (may20.160);
    \draw[a] (Index) edge (jun20.160);
    \draw[a] (Index) edge (jul20);
    \draw[decorate with=circle,paint=black] (jul1) -- (apr20);


\end{tikzpicture}
   \caption{Schematic of disaggregation process using proportion disaggregation.}
   \label{fig:disag}
\end{figure}

Below we describe the implementation steps briefly for the benefit of the reader: 
\begin{enumerate}

\item 1.	Create an ``index'' gauge flow by summing all the total seasonal flow at each site, as mentioned above. Identify large scale climate predictors of index gauge flow. This is done by correlating it with global climate variables at different lead times and creating average time series from regions of high correlation with physical reasons \citep{Grantz:2005ve,Bracken:2010cw}. 

\item 2.	Identify a set of the ``best'' set of models using  locally weighted polynomial (Loader 1999) for each lead time. The best models are defined as those that have low generalized cross validation value (GCV) (Craven and Wahba, 1979) and do not exhibit multicollinearity A suite of best models is selected using their GCV values - typically, all models within a small range of the lowest GCV value are selected (see \cite{Bracken:2010cw,Regonda2006}, for details). The functional form of the models is the typical regression form
\begin{equation}
y = f(\mathbf{x})+\varepsilon\label{eqn:reg}
\end{equation}
where $\mathbf{x}$ is the vector of predictor variables, $f$ is the function that is estimated locally using locally weighted polynomial approach and $\varepsilon$ is the error with the standard assumption of Normality and identical and independently distributed residuals (i.i.d) based on regression theory. The details of this approach and applications can be found in the aforementioned references. Also, see \cite{Rajagopalan:2010ur} and \cite{Lall:1995wk} for a general review of local functional estimation methods.

\item Identify a set of the best locally weighted polynomial models \cite{Loader:1999hx} for each lead time.  The best models are defined as those that minimize the generalized cross validation value (GCV) \citep{Craven1979} and do not exhibit multicollinearity \cite{Regonda2006}.

\item Generate ensemble forecasts of index gauge flow  by weighting the best models according to their GCV value (lower GCV gets higher weight).  Multimodel ensemble predictions are made by randomly selecting a model based on the GCV-based weights, obtaining the mean forecast from the model (of the form in equation \ref{eqn:reg} above) and generating a Normal deviate with the appropriate error variance.  This is repeated to generate the ensemble. 

\item Disaggregate the total seasonal flow at to the index gauge to  intervening monthly flow at all the 20 sites. 

\end{enumerate}

\section{Validation}

As mentioned we applied the methodology described above to generate monthly ensemble streamflow forecasts at 20 locations in the UCRB for the peak season (Apr-Jul) at different lead times from the first of each month starting November 1st. We validated the results using the same methods as \cite{Bracken:2010cw} -- Leave-one-out cross validation and ``retroactive'' forecasts. Leave-one-out cross validation removes the current year from the model, and then predicts that year using the remaining data. Retroactive verification only uses the data available prior to the forecast year, as would be the case in real operations. This way it is possible to compare directly to operational forecasts of the Colorado Basin River Forecast Center (CBRFC). The leave-one-out validation was performed for the period 1949-2007 and the retroactive for 1993-2007 during which the CBRFC forecasts are available for comparison.  The ranked probability skill score (RPSS) \cite{Wilks:1995ur} was computed for each year in the history and then the median value is presented. RPSS value of, 1 indicates a perfect categorical forecast, 0 indicates no different from climatology and negative values suggest worse than climatological forecast. The median correlation (MC), the correlation between the historical data and the median of the ensembles for each year,  was also computed for each site/month combination. 

\section{Results}

The predictors in the various ``best'' multi-models at all the lead times are shown in Table \label{tab:models}. It can be seen that the number of ``best'' models decrease with decrease in lead time. For example, for March 1st and April 1st forecast have two models while the earlier lead times have five models. This is consistent in that at long lead times when basin snow information is absent or partial at best, predictability is obtained mainly through climate variables, while at March and April 1st the snow information is complete and hence, the role of climate variables is minimal. Also of interest is that the PDSI enters the best models for shorter lead times, the reasoning is that it captures the role of antecedent land surface, such that, warmer preceding summer and fall conditions can enable increased losses and thus reduce streamflow. This role of land surface was also identified by \cite{Regonda2006} in the Gunnison River Basin.

We computed the skill scores for all the locations and at all lead times. Here we present results at three lead times Nov 1st, Jan 1st and Apr 1st. Table \ref{tab:models} shows the models used at each lead time. The RPSS values of the index gauge (Lees Ferry total flow) flow at these lead times for the two validation method are shown in Table \ref{tab:indexskill}. These are similar to the skills from \cite{Bracken:2010cw}. The skills are substantially higher than climatology at long lead time (e.g., Nov 1st) and progressively increase with decrease in lead time. This indicates that the methodology is able to provide useful information of the streamflow distribution well in advance. The retroactive skills are lower because of the use of less data.
 
\begin{table}[ht]
\centering
\caption{Selected Models for each lead time}\label{tab:models}
\begin{tabular}{rcccccc} 
\toprule
& GPH & ZNW & MDW & SST & PDSI & SWE \\
\midrule
& \multicolumn{6}{c}{November 1} \\
\midrule
1 & 0 & 0 & 1 & 1 & 0 & 0\\
2 & 0 & 1 & 0 & 0 & 0 & 0\\
3 & 0 & 0 & 0 & 0 & 1 & 0\\
4 & 0 & 0 & 0 & 1 & 0 & 0\\
5 & 0 & 0 & 1 & 0 & 0 & 0\\
6 & 1 & 0 & 0 & 0 & 0 & 0\\
\midrule
& \multicolumn{6}{c}{January 1} \\
\midrule
1 & 0 & 0 & 0 & 1 & 0 & 1\\
2 & 0 & 0 & 0 & 0 & 0 & 1\\
3 & 0 & 0 & 0 & 0 & 1 & 0\\
4 & 0 & 0 & 1 & 0 & 0 & 0\\
5 & 0 & 0 & 0 & 1 & 0 & 0\\
\midrule
& \multicolumn{6}{c}{February 1} \\
\midrule
1 & 0 & 0 & 0 & 1 & 0 & 1\\
2 & 0 & 0 & 0 & 0 & 0 & 1\\
3 & 0 & 0 & 0 & 0 & 1 & 0\\
\midrule
& \multicolumn{6}{c}{March 1} \\
\midrule
1 & 0 & 0 & 0 & 1 & 0 & 1\\
2 & 0 & 0 & 0 & 0 & 0 & 1\\
\midrule
& \multicolumn{6}{c}{April 1} \\
\midrule
1 & 0 & 0 & 0 & 1 & 0 & 1\\
2 & 0 & 0 & 0 & 0 & 0 & 1\\
\bottomrule
\end{tabular}
\end{table}
 
Table \ref{tab:dropone} shows the RPSS and MC for all the locations and all months in the peak season for the three lead times mentioned above. In general we see the skills decreasing as lead time increases, as to be expected. Some sites such as the Duchesne River Near Randlett, UT (14) have consistently positive skill back to the Nov 1 lead time. June had the highest overall skill of any month, likely because the average peak flow occurs in June. The sites included in \cite{Bracken:2010cw} -- ``key gauges'' (8, 16, 19 and 20, outlined in bold) represent the major contributions to Lees Ferry flow and thus important for planning purposes.  The skills at these gauges are high as they have a strong response to large-scale climate forcing.  Another trend we see is that the skills tend to be better as we move toward the outlet of the basin. That is consistent with the result that the larger tributaries perform better and that by aggregating the flows the impact of large scale climate signal emerges better. 

With the exception of few sites the skills of the disaggregated flows are positive from Nov 1st onwards with increasing magnitudes moving towards Apr 1st (Table \ref{tab:indexskill}). This indicates that the skill in forecasting streamflow at the index gauge is translated throughout the upper basin, an important contribution that will have significant impact on water resources planning and management. The locations with poor skills are mainly in smaller tributaries and this could be improved by disaggregating  in chunks based on regions (such as the San Juan basin) to better capture regional behavior. 
 
\begin{table}[ht]
\centering 
\caption{RPSS values for the index gauge for each lead time.}\label{tab:indexskill}
\begin{tabular}{ccccc}  
\toprule
& Apr1 & Feb1 & Jan1 & Nov1 \\
\midrule
Leave-one out & 0.85 & 0.74 & 0.49 & 0.30 \\
Retroactive   & 0.62 & 0.58 & 0.55 & 0.52 \\
\bottomrule
\end{tabular}
\end{table}

%\begin{table}[ht]
%\centering 
%\caption{RPSS values for the index gauge for each lead time.}\label{tab:indexskill}
%\begin{tabular}{cccccc}  
%\toprule
%& Apr1 & Feb1 & Jan1 & Nov1 \\
%\midrule
%Leave-one out & 0.85 & 0.74 & 0.49 & 0.30 & \includegraphics[width=40pt,height=10pt]{figures/rpss-spark-bar.pdf} \\
%Retroactive   & 0.62 & 0.58 & 0.55 & 0.52 & \includegraphics[width=40pt,height=10pt]{figures/rpss-spark-bar2.pdf} \\
%\bottomrule
%\end{tabular}
%\end{table}

{\small
% latex table generated in R 2.13.0 by xtable 1.5-6 package
% Thu Jul  7 11:37:48 2011
\begin{sidewaystable}[ht]
\begin{center}\footnotesize
\caption{RPSS and MC (in parentheses) after disaggregation and drop-one cross validation for each lead time. At 95\% confidence 0.21 is a significant correlation. Postive skills and significant correlations are shown in bold.}\label{tab:dropone}
\begin{tabular}{rrrrr|rrrr|rrrr}
  \toprule
  &\multicolumn{4}{c}{April 1} & \multicolumn{4}{c}{Jan 1} & \multicolumn{4}{c}{Nov 1}\\
  \midrule
Node & April & May & June & July & April & May & June & July & April & May & June & July \\ 
  \midrule
  1 & {\bf 0.14} ({\bf 0.24}) & {\bf 0.23} ({\bf 0.57}) & {\bf 0.57} ({\bf 0.75}) & {\bf 0.47} ({\bf 0.62}) & {\bf 0.01} (0.09) & {\bf 0.19} ({\bf 0.44}) & {\bf 0.43} ({\bf 0.61}) & {\bf 0.29} ({\bf 0.46}) & -0.14 (0.17) & {\bf 0.07} ({\bf 0.30}) & {\bf 0.13} ({\bf 0.33}) & {\bf 0.18} ({\bf 0.37}) \\ 
    2 & {\bf 0.22} ({\bf 0.28}) & {\bf 0.32} ({\bf 0.64}) & {\bf 0.69} ({\bf 0.77}) & {\bf 0.66} ({\bf 0.63}) & {\bf 0.11} (0.15) & {\bf 0.22} ({\bf 0.47}) & {\bf 0.47} ({\bf 0.67}) & {\bf 0.37} ({\bf 0.49}) & -0.04 ({\bf 0.33}) & {\bf 0.10} ({\bf 0.38}) & {\bf 0.03} ({\bf 0.37}) & {\bf 0.23} ({\bf 0.41}) \\ 
    3 & {\bf 0.13} (-0.00) & {\bf 0.26} ({\bf 0.58}) & {\bf 0.56} ({\bf 0.77}) & {\bf 0.62} ({\bf 0.61}) & -0.01 (-0.03) & {\bf 0.09} ({\bf 0.24}) & {\bf 0.26} ({\bf 0.63}) & {\bf 0.41} ({\bf 0.46}) & {\bf 0.01} (-0.04) & {\bf 0.01} (-0.21) & {\bf 0.08} ({\bf 0.28}) & {\bf 0.26} ({\bf 0.29}) \\ 
    4 & {\bf 0.15} ({\bf 0.45}) & {\bf 0.36} ({\bf 0.67}) & {\bf 0.80} ({\bf 0.78}) & {\bf 0.75} ({\bf 0.65}) & {\bf 0.12} ({\bf 0.39}) & {\bf 0.25} ({\bf 0.52}) & {\bf 0.50} ({\bf 0.64}) & {\bf 0.47} ({\bf 0.49}) & -0.03 (0.06) & {\bf 0.14} ({\bf 0.22}) & {\bf 0.01} ({\bf 0.35}) & {\bf 0.18} ({\bf 0.33}) \\ 
    5 & {\bf 0.26} ({\bf 0.48}) & {\bf 0.41} ({\bf 0.64}) & {\bf 0.74} ({\bf 0.77}) & {\bf 0.71} ({\bf 0.62}) & {\bf 0.21} ({\bf 0.42}) & {\bf 0.30} ({\bf 0.56}) & {\bf 0.49} ({\bf 0.64}) & {\bf 0.48} ({\bf 0.45}) & {\bf 0.07} (0.09) & {\bf 0.19} ({\bf 0.31}) & {\bf 0.05} ({\bf 0.37}) & {\bf 0.26} ({\bf 0.34}) \\ 
    6 & {\bf 0.13} ({\bf 0.45}) & {\bf 0.44} ({\bf 0.73}) & {\bf 0.77} ({\bf 0.75}) & {\bf 0.80} ({\bf 0.61}) & {\bf 0.05} ({\bf 0.47}) & {\bf 0.32} ({\bf 0.63}) & {\bf 0.42} ({\bf 0.59}) & {\bf 0.55} ({\bf 0.46}) & -0.05 ({\bf 0.33}) & {\bf 0.29} ({\bf 0.44}) & {\bf 0.15} ({\bf 0.39}) & {\bf 0.22} ({\bf 0.36}) \\ 
    7 & {\bf 0.40} ({\bf 0.66}) & {\bf 0.40} ({\bf 0.65}) & {\bf 0.51} ({\bf 0.67}) & {\bf 0.51} ({\bf 0.63}) & {\bf 0.21} ({\bf 0.47}) & {\bf 0.29} ({\bf 0.53}) & {\bf 0.32} ({\bf 0.44}) & {\bf 0.41} ({\bf 0.47}) & {\bf 0.09} ({\bf 0.27}) & {\bf 0.26} ({\bf 0.42}) & {\bf 0.31} ({\bf 0.40}) & {\bf 0.38} ({\bf 0.38}) \\ 
    8 & {\bf 0.33} ({\bf 0.52}) & {\bf 0.38} ({\bf 0.71}) & {\bf 0.78} ({\bf 0.78}) & {\bf 0.72} ({\bf 0.64}) & {\bf 0.15} ({\bf 0.51}) & {\bf 0.29} ({\bf 0.60}) & {\bf 0.48} ({\bf 0.65}) & {\bf 0.45} ({\bf 0.50}) & -0.03 ({\bf 0.34}) & {\bf 0.13} ({\bf 0.45}) & {\bf 0.08} ({\bf 0.40}) & {\bf 0.35} ({\bf 0.43}) \\ 
    9 & {\bf 0.35} ({\bf 0.55}) & {\bf 0.42} ({\bf 0.62}) & {\bf 0.34} ({\bf 0.50}) & {\bf 0.30} ({\bf 0.50}) & {\bf 0.21} ({\bf 0.41}) & {\bf 0.26} ({\bf 0.44}) & {\bf 0.09} (0.16) & -0.08 ({\bf 0.36}) & {\bf 0.07} (0.03) & {\bf 0.07} ({\bf 0.28}) & -0.08 (-0.08) & -0.08 (0.17) \\ 
    10 & {\bf 0.45} ({\bf 0.60}) & {\bf 0.37} ({\bf 0.63}) & {\bf 0.45} ({\bf 0.51}) & {\bf 0.40} ({\bf 0.53}) & {\bf 0.37} ({\bf 0.43}) & {\bf 0.18} ({\bf 0.44}) & {\bf 0.20} (0.18) & -0.02 ({\bf 0.36}) & {\bf 0.28} (0.10) & -0.04 ({\bf 0.28}) & {\bf 0.01} (-0.07) & -0.02 (0.16) \\ 
    11 & {\bf 0.33} ({\bf 0.61}) & {\bf 0.46} ({\bf 0.74}) & {\bf 0.39} ({\bf 0.57}) & {\bf 0.38} ({\bf 0.52}) & {\bf 0.27} ({\bf 0.48}) & {\bf 0.24} ({\bf 0.61}) & {\bf 0.15} ({\bf 0.28}) & -0.01 ({\bf 0.38}) & {\bf 0.18} (0.19) & {\bf 0.16} ({\bf 0.42}) & {\bf 0.00} (0.01) & -0.02 ({\bf 0.22}) \\ 
    12 & {\bf 0.23} ({\bf 0.52}) & {\bf 0.43} ({\bf 0.70}) & {\bf 0.68} ({\bf 0.72}) & {\bf 0.57} ({\bf 0.59}) & {\bf 0.03} ({\bf 0.38}) & {\bf 0.34} ({\bf 0.56}) & {\bf 0.42} ({\bf 0.59}) & {\bf 0.36} ({\bf 0.46}) & -0.11 ({\bf 0.30}) & {\bf 0.28} ({\bf 0.42}) & {\bf 0.08} ({\bf 0.37}) & {\bf 0.25} ({\bf 0.38}) \\ 
    13 & -0.02 ({\bf 0.22}) & {\bf 0.45} ({\bf 0.63}) & {\bf 0.41} ({\bf 0.68}) & {\bf 0.32} ({\bf 0.43}) & {\bf 0.01} (0.17) & {\bf 0.37} ({\bf 0.52}) & {\bf 0.28} ({\bf 0.66}) & {\bf 0.22} ({\bf 0.27}) & {\bf 0.06} ({\bf 0.30}) & {\bf 0.30} ({\bf 0.47}) & {\bf 0.12} ({\bf 0.45}) & {\bf 0.06} (0.04) \\ 
    14 & {\bf 0.40} ({\bf 0.60}) & {\bf 0.66} ({\bf 0.71}) & {\bf 0.48} ({\bf 0.72}) & {\bf 0.43} ({\bf 0.44}) & {\bf 0.24} ({\bf 0.56}) & {\bf 0.32} ({\bf 0.61}) & {\bf 0.30} ({\bf 0.57}) & {\bf 0.32} ({\bf 0.40}) & {\bf 0.19} ({\bf 0.38}) & {\bf 0.19} ({\bf 0.49}) & {\bf 0.25} ({\bf 0.39}) & {\bf 0.22} ({\bf 0.24}) \\ 
    15 & {\bf 0.04} ({\bf 0.40}) & {\bf 0.30} ({\bf 0.65}) & {\bf 0.63} ({\bf 0.76}) & {\bf 0.74} ({\bf 0.59}) & {\bf 0.10} ({\bf 0.35}) & {\bf 0.24} ({\bf 0.55}) & {\bf 0.36} ({\bf 0.66}) & {\bf 0.45} ({\bf 0.50}) & {\bf 0.10} ({\bf 0.39}) & {\bf 0.14} ({\bf 0.45}) & {\bf 0.10} ({\bf 0.46}) & {\bf 0.39} ({\bf 0.49}) \\ 
    16 & {\bf 0.38} ({\bf 0.60}) & {\bf 0.60} ({\bf 0.82}) & {\bf 0.50} ({\bf 0.77}) & {\bf 0.53} ({\bf 0.62}) & {\bf 0.20} ({\bf 0.50}) & {\bf 0.44} ({\bf 0.67}) & {\bf 0.33} ({\bf 0.61}) & {\bf 0.08} ({\bf 0.50}) & {\bf 0.06} ({\bf 0.31}) & {\bf 0.20} ({\bf 0.52}) & {\bf 0.14} ({\bf 0.35}) & {\bf 0.04} ({\bf 0.33}) \\ 
    17 & {\bf 0.35} ({\bf 0.52}) & {\bf 0.37} ({\bf 0.58}) & {\bf 0.80} ({\bf 0.73}) & {\bf 0.14} ({\bf 0.54}) & {\bf 0.07} ({\bf 0.45}) & {\bf 0.29} ({\bf 0.54}) & {\bf 0.37} ({\bf 0.60}) & {\bf 0.10} ({\bf 0.42}) & {\bf 0.00} ({\bf 0.28}) & {\bf 0.26} ({\bf 0.38}) & {\bf 0.19} ({\bf 0.40}) & {\bf 0.10} ({\bf 0.36}) \\ 
    18 & {\bf 0.24} ({\bf 0.54}) & {\bf 0.36} ({\bf 0.63}) & {\bf 0.36} ({\bf 0.68}) & {\bf 0.36} ({\bf 0.57}) & -0.06 ({\bf 0.41}) & {\bf 0.21} ({\bf 0.51}) & {\bf 0.11} ({\bf 0.46}) & {\bf 0.25} ({\bf 0.44}) & -0.33 (0.17) & {\bf 0.11} ({\bf 0.31}) & {\bf 0.01} ({\bf 0.25}) & {\bf 0.20} (0.18) \\ 
    19 & {\bf 0.35} ({\bf 0.56}) & {\bf 0.27} ({\bf 0.62}) & {\bf 0.38} ({\bf 0.69}) & {\bf 0.55} ({\bf 0.58}) & {\bf 0.04} ({\bf 0.39}) & {\bf 0.22} ({\bf 0.51}) & {\bf 0.27} ({\bf 0.43}) & {\bf 0.31} ({\bf 0.46}) & -0.24 ({\bf 0.22}) & {\bf 0.17} ({\bf 0.30}) & {\bf 0.22} ({\bf 0.30}) & {\bf 0.23} ({\bf 0.30}) \\ 
    20 & {\bf 0.27} ({\bf 0.58}) & {\bf 0.49} ({\bf 0.79}) & {\bf 0.81} ({\bf 0.82}) & {\bf 0.74} ({\bf 0.66}) & {\bf 0.23} ({\bf 0.55}) & {\bf 0.31} ({\bf 0.66}) & {\bf 0.55} ({\bf 0.69}) & {\bf 0.38} ({\bf 0.53}) & {\bf 0.17} ({\bf 0.34}) & {\bf 0.30} ({\bf 0.50}) & {\bf 0.10} ({\bf 0.45}) & {\bf 0.33} ({\bf 0.41}) \\
   \bottomrule
\end{tabular}
\end{center}
\end{sidewaystable}

}

As a representative sample, we display the results for the location Gunnison River near Grand Junction (Site 6), results from other locations are quite similar. Also this site was chosen because it was not presented in the four sites used by Bracken et al. (2010) and it is an important site for water resources planning in the UCRB. Figure \ref{fig:box-seas} shows the boxplot of ensemble forecast of the peak season volume at the three lead times along with the historical natural flow (in red). The skill scores are shown in the figure captions and it can be seen that there is a significant positive skill for the seasonal flow forecast issued on November 1st (about six months lead time) which increases as the lead time decreases. Figure \ref{fig:box} shows the ensemble forecast of June (the peak month in the season) streamflow at the same lead times and here too the skills for Nov 1st forecast are quite impressive.  These plots show the general pattern seen across almost all the sites, that is, ensemble variability increases and skill decreases as lead time increases. Furthermore, they also capture the high and low flow variability quite well, especially for Jan 1st and Apr 1st forecasts, which are very useful for management. Figure \ref{fig:box-retro} shows the forecasts in retroactive mode, where in the model fitting is updated for each year with the most recent data, this is the approach used by CBRFC in their forecasting framework using watershed models  Also shown in this figure are the CBRFC coordinated forecasts for comparison. The forecasts from our multi-model ensemble performs very well in comparison to the CBRFC forecasts, also, it captures the high and low flow years very well.  

 
\begin{figure*}[htbp] %  figure placement: here, top, bottom, or page
   \centering
   %\includegraphics[width=.9\textwidth]{boxplots.pdf}\\
   % Created by tikzDevice version 0.6.1 on 2011-07-07 10:48:54
% !TEX encoding = UTF-8 Unicode
\begin{tikzpicture}[x=1pt,y=1pt]
\definecolor[named]{drawColor}{rgb}{0.00,0.00,0.00}
\definecolor[named]{fillColor}{rgb}{1.00,1.00,1.00}
\fill[color=fillColor,] (0,0) rectangle (505.89,650.43);
\begin{scope}
\path[clip] ( 32.47,450.25) rectangle (489.26,625.88);
\definecolor[named]{drawColor}{rgb}{0.00,0.00,0.00}

\draw[color=drawColor,line width= 1.2pt,line join=round,fill opacity=0.00,] ( 50.06,544.22) -- ( 55.43,544.22);

\draw[color=drawColor,dash pattern=on 4pt off 4pt ,line cap=round,line join=round,fill opacity=0.00,] ( 52.75,520.31) -- ( 52.75,537.21);

\draw[color=drawColor,dash pattern=on 4pt off 4pt ,line cap=round,line join=round,fill opacity=0.00,] ( 52.75,576.73) -- ( 52.75,553.85);

\draw[color=drawColor,line cap=round,line join=round,fill opacity=0.00,] ( 51.40,520.31) -- ( 54.09,520.31);

\draw[color=drawColor,line cap=round,line join=round,fill opacity=0.00,] ( 51.40,576.73) -- ( 54.09,576.73);

\draw[color=drawColor,line cap=round,line join=round,fill opacity=0.00,] ( 50.06,537.21) --
	( 55.43,537.21) --
	( 55.43,553.85) --
	( 50.06,553.85) --
	( 50.06,537.21);

\draw[color=drawColor,line width= 1.2pt,line join=round,fill opacity=0.00,] ( 56.77,518.86) -- ( 62.15,518.86);

\draw[color=drawColor,dash pattern=on 4pt off 4pt ,line cap=round,line join=round,fill opacity=0.00,] ( 59.46,499.37) -- ( 59.46,511.56);

\draw[color=drawColor,dash pattern=on 4pt off 4pt ,line cap=round,line join=round,fill opacity=0.00,] ( 59.46,543.03) -- ( 59.46,527.90);

\draw[color=drawColor,line cap=round,line join=round,fill opacity=0.00,] ( 58.12,499.37) -- ( 60.80,499.37);

\draw[color=drawColor,line cap=round,line join=round,fill opacity=0.00,] ( 58.12,543.03) -- ( 60.80,543.03);

\draw[color=drawColor,line cap=round,line join=round,fill opacity=0.00,] ( 56.77,511.56) --
	( 62.15,511.56) --
	( 62.15,527.90) --
	( 56.77,527.90) --
	( 56.77,511.56);

\draw[color=drawColor,line width= 1.2pt,line join=round,fill opacity=0.00,] ( 63.49,511.23) -- ( 68.86,511.23);

\draw[color=drawColor,dash pattern=on 4pt off 4pt ,line cap=round,line join=round,fill opacity=0.00,] ( 66.17,493.51) -- ( 66.17,504.20);

\draw[color=drawColor,dash pattern=on 4pt off 4pt ,line cap=round,line join=round,fill opacity=0.00,] ( 66.17,539.47) -- ( 66.17,519.86);

\draw[color=drawColor,line cap=round,line join=round,fill opacity=0.00,] ( 64.83,493.51) -- ( 67.52,493.51);

\draw[color=drawColor,line cap=round,line join=round,fill opacity=0.00,] ( 64.83,539.47) -- ( 67.52,539.47);

\draw[color=drawColor,line cap=round,line join=round,fill opacity=0.00,] ( 63.49,504.20) --
	( 68.86,504.20) --
	( 68.86,519.86) --
	( 63.49,519.86) --
	( 63.49,504.20);

\draw[color=drawColor,line width= 1.2pt,line join=round,fill opacity=0.00,] ( 70.20,580.70) -- ( 75.57,580.70);

\draw[color=drawColor,dash pattern=on 4pt off 4pt ,line cap=round,line join=round,fill opacity=0.00,] ( 72.89,545.73) -- ( 72.89,569.75);

\draw[color=drawColor,dash pattern=on 4pt off 4pt ,line cap=round,line join=round,fill opacity=0.00,] ( 72.89,626.34) -- ( 72.89,593.45);

\draw[color=drawColor,line cap=round,line join=round,fill opacity=0.00,] ( 71.54,545.73) -- ( 74.23,545.73);

\draw[color=drawColor,line cap=round,line join=round,fill opacity=0.00,] ( 71.54,626.34) -- ( 74.23,626.34);

\draw[color=drawColor,line cap=round,line join=round,fill opacity=0.00,] ( 70.20,569.75) --
	( 75.57,569.75) --
	( 75.57,593.45) --
	( 70.20,593.45) --
	( 70.20,569.75);

\draw[color=drawColor,line width= 1.2pt,line join=round,fill opacity=0.00,] ( 76.92,504.04) -- ( 82.29,504.04);

\draw[color=drawColor,dash pattern=on 4pt off 4pt ,line cap=round,line join=round,fill opacity=0.00,] ( 79.60,490.74) -- ( 79.60,497.95);

\draw[color=drawColor,dash pattern=on 4pt off 4pt ,line cap=round,line join=round,fill opacity=0.00,] ( 79.60,515.27) -- ( 79.60,506.83);

\draw[color=drawColor,line cap=round,line join=round,fill opacity=0.00,] ( 78.26,490.74) -- ( 80.94,490.74);

\draw[color=drawColor,line cap=round,line join=round,fill opacity=0.00,] ( 78.26,515.27) -- ( 80.94,515.27);

\draw[color=drawColor,line cap=round,line join=round,fill opacity=0.00,] ( 76.92,497.95) --
	( 82.29,497.95) --
	( 82.29,506.83) --
	( 76.92,506.83) --
	( 76.92,497.95);

\draw[color=drawColor,line width= 1.2pt,line join=round,fill opacity=0.00,] ( 83.63,501.16) -- ( 89.00,501.16);

\draw[color=drawColor,dash pattern=on 4pt off 4pt ,line cap=round,line join=round,fill opacity=0.00,] ( 86.31,489.56) -- ( 86.31,497.36);

\draw[color=drawColor,dash pattern=on 4pt off 4pt ,line cap=round,line join=round,fill opacity=0.00,] ( 86.31,515.30) -- ( 86.31,505.20);

\draw[color=drawColor,line cap=round,line join=round,fill opacity=0.00,] ( 84.97,489.56) -- ( 87.66,489.56);

\draw[color=drawColor,line cap=round,line join=round,fill opacity=0.00,] ( 84.97,515.30) -- ( 87.66,515.30);

\draw[color=drawColor,line cap=round,line join=round,fill opacity=0.00,] ( 83.63,497.36) --
	( 89.00,497.36) --
	( 89.00,505.20) --
	( 83.63,505.20) --
	( 83.63,497.36);

\draw[color=drawColor,line width= 1.2pt,line join=round,fill opacity=0.00,] ( 90.34,500.31) -- ( 95.71,500.31);

\draw[color=drawColor,dash pattern=on 4pt off 4pt ,line cap=round,line join=round,fill opacity=0.00,] ( 93.03,489.35) -- ( 93.03,496.73);

\draw[color=drawColor,dash pattern=on 4pt off 4pt ,line cap=round,line join=round,fill opacity=0.00,] ( 93.03,513.98) -- ( 93.03,504.15);

\draw[color=drawColor,line cap=round,line join=round,fill opacity=0.00,] ( 91.69,489.35) -- ( 94.37,489.35);

\draw[color=drawColor,line cap=round,line join=round,fill opacity=0.00,] ( 91.69,513.98) -- ( 94.37,513.98);

\draw[color=drawColor,line cap=round,line join=round,fill opacity=0.00,] ( 90.34,496.73) --
	( 95.71,496.73) --
	( 95.71,504.15) --
	( 90.34,504.15) --
	( 90.34,496.73);

\draw[color=drawColor,line width= 1.2pt,line join=round,fill opacity=0.00,] ( 97.06,535.95) -- (102.43,535.95);

\draw[color=drawColor,dash pattern=on 4pt off 4pt ,line cap=round,line join=round,fill opacity=0.00,] ( 99.74,508.37) -- ( 99.74,527.43);

\draw[color=drawColor,dash pattern=on 4pt off 4pt ,line cap=round,line join=round,fill opacity=0.00,] ( 99.74,568.63) -- ( 99.74,543.92);

\draw[color=drawColor,line cap=round,line join=round,fill opacity=0.00,] ( 98.40,508.37) -- (101.08,508.37);

\draw[color=drawColor,line cap=round,line join=round,fill opacity=0.00,] ( 98.40,568.63) -- (101.08,568.63);

\draw[color=drawColor,line cap=round,line join=round,fill opacity=0.00,] ( 97.06,527.43) --
	(102.43,527.43) --
	(102.43,543.92) --
	( 97.06,543.92) --
	( 97.06,527.43);

\draw[color=drawColor,line width= 1.2pt,line join=round,fill opacity=0.00,] (103.77,535.70) -- (109.14,535.70);

\draw[color=drawColor,dash pattern=on 4pt off 4pt ,line cap=round,line join=round,fill opacity=0.00,] (106.45,523.88) -- (106.45,532.65);

\draw[color=drawColor,dash pattern=on 4pt off 4pt ,line cap=round,line join=round,fill opacity=0.00,] (106.45,547.57) -- (106.45,538.63);

\draw[color=drawColor,line cap=round,line join=round,fill opacity=0.00,] (105.11,523.88) -- (107.80,523.88);

\draw[color=drawColor,line cap=round,line join=round,fill opacity=0.00,] (105.11,547.57) -- (107.80,547.57);

\draw[color=drawColor,line cap=round,line join=round,fill opacity=0.00,] (103.77,532.65) --
	(109.14,532.65) --
	(109.14,538.63) --
	(103.77,538.63) --
	(103.77,532.65);

\draw[color=drawColor,line width= 1.2pt,line join=round,fill opacity=0.00,] (110.48,534.42) -- (115.85,534.42);

\draw[color=drawColor,dash pattern=on 4pt off 4pt ,line cap=round,line join=round,fill opacity=0.00,] (113.17,507.77) -- (113.17,525.75);

\draw[color=drawColor,dash pattern=on 4pt off 4pt ,line cap=round,line join=round,fill opacity=0.00,] (113.17,560.76) -- (113.17,539.82);

\draw[color=drawColor,line cap=round,line join=round,fill opacity=0.00,] (111.83,507.77) -- (114.51,507.77);

\draw[color=drawColor,line cap=round,line join=round,fill opacity=0.00,] (111.83,560.76) -- (114.51,560.76);

\draw[color=drawColor,line cap=round,line join=round,fill opacity=0.00,] (110.48,525.75) --
	(115.85,525.75) --
	(115.85,539.82) --
	(110.48,539.82) --
	(110.48,525.75);

\draw[color=drawColor,line width= 1.2pt,line join=round,fill opacity=0.00,] (117.20,500.10) -- (122.57,500.10);

\draw[color=drawColor,dash pattern=on 4pt off 4pt ,line cap=round,line join=round,fill opacity=0.00,] (119.88,489.60) -- (119.88,496.93);

\draw[color=drawColor,dash pattern=on 4pt off 4pt ,line cap=round,line join=round,fill opacity=0.00,] (119.88,515.24) -- (119.88,504.67);

\draw[color=drawColor,line cap=round,line join=round,fill opacity=0.00,] (118.54,489.60) -- (121.22,489.60);

\draw[color=drawColor,line cap=round,line join=round,fill opacity=0.00,] (118.54,515.24) -- (121.22,515.24);

\draw[color=drawColor,line cap=round,line join=round,fill opacity=0.00,] (117.20,496.93) --
	(122.57,496.93) --
	(122.57,504.67) --
	(117.20,504.67) --
	(117.20,496.93);

\draw[color=drawColor,line width= 1.2pt,line join=round,fill opacity=0.00,] (123.91,502.51) -- (129.28,502.51);

\draw[color=drawColor,dash pattern=on 4pt off 4pt ,line cap=round,line join=round,fill opacity=0.00,] (126.60,491.99) -- (126.60,497.78);

\draw[color=drawColor,dash pattern=on 4pt off 4pt ,line cap=round,line join=round,fill opacity=0.00,] (126.60,520.09) -- (126.60,507.39);

\draw[color=drawColor,line cap=round,line join=round,fill opacity=0.00,] (125.25,491.99) -- (127.94,491.99);

\draw[color=drawColor,line cap=round,line join=round,fill opacity=0.00,] (125.25,520.09) -- (127.94,520.09);

\draw[color=drawColor,line cap=round,line join=round,fill opacity=0.00,] (123.91,497.78) --
	(129.28,497.78) --
	(129.28,507.39) --
	(123.91,507.39) --
	(123.91,497.78);

\draw[color=drawColor,line width= 1.2pt,line join=round,fill opacity=0.00,] (130.62,489.34) -- (135.99,489.34);

\draw[color=drawColor,dash pattern=on 4pt off 4pt ,line cap=round,line join=round,fill opacity=0.00,] (133.31,474.91) -- (133.31,485.03);

\draw[color=drawColor,dash pattern=on 4pt off 4pt ,line cap=round,line join=round,fill opacity=0.00,] (133.31,507.65) -- (133.31,494.24);

\draw[color=drawColor,line cap=round,line join=round,fill opacity=0.00,] (131.97,474.91) -- (134.65,474.91);

\draw[color=drawColor,line cap=round,line join=round,fill opacity=0.00,] (131.97,507.65) -- (134.65,507.65);

\draw[color=drawColor,line cap=round,line join=round,fill opacity=0.00,] (130.62,485.03) --
	(135.99,485.03) --
	(135.99,494.24) --
	(130.62,494.24) --
	(130.62,485.03);

\draw[color=drawColor,line width= 1.2pt,line join=round,fill opacity=0.00,] (137.34,548.18) -- (142.71,548.18);

\draw[color=drawColor,dash pattern=on 4pt off 4pt ,line cap=round,line join=round,fill opacity=0.00,] (140.02,530.67) -- (140.02,543.42);

\draw[color=drawColor,dash pattern=on 4pt off 4pt ,line cap=round,line join=round,fill opacity=0.00,] (140.02,575.71) -- (140.02,561.74);

\draw[color=drawColor,line cap=round,line join=round,fill opacity=0.00,] (138.68,530.67) -- (141.36,530.67);

\draw[color=drawColor,line cap=round,line join=round,fill opacity=0.00,] (138.68,575.71) -- (141.36,575.71);

\draw[color=drawColor,line cap=round,line join=round,fill opacity=0.00,] (137.34,543.42) --
	(142.71,543.42) --
	(142.71,561.74) --
	(137.34,561.74) --
	(137.34,543.42);

\draw[color=drawColor,line width= 1.2pt,line join=round,fill opacity=0.00,] (144.05,495.54) -- (149.42,495.54);

\draw[color=drawColor,dash pattern=on 4pt off 4pt ,line cap=round,line join=round,fill opacity=0.00,] (146.74,482.56) -- (146.74,492.33);

\draw[color=drawColor,dash pattern=on 4pt off 4pt ,line cap=round,line join=round,fill opacity=0.00,] (146.74,508.52) -- (146.74,498.86);

\draw[color=drawColor,line cap=round,line join=round,fill opacity=0.00,] (145.39,482.56) -- (148.08,482.56);

\draw[color=drawColor,line cap=round,line join=round,fill opacity=0.00,] (145.39,508.52) -- (148.08,508.52);

\draw[color=drawColor,line cap=round,line join=round,fill opacity=0.00,] (144.05,492.33) --
	(149.42,492.33) --
	(149.42,498.86) --
	(144.05,498.86) --
	(144.05,492.33);

\draw[color=drawColor,line width= 1.2pt,line join=round,fill opacity=0.00,] (150.76,495.30) -- (156.13,495.30);

\draw[color=drawColor,dash pattern=on 4pt off 4pt ,line cap=round,line join=round,fill opacity=0.00,] (153.45,480.81) -- (153.45,491.22);

\draw[color=drawColor,dash pattern=on 4pt off 4pt ,line cap=round,line join=round,fill opacity=0.00,] (153.45,509.56) -- (153.45,498.58);

\draw[color=drawColor,line cap=round,line join=round,fill opacity=0.00,] (152.11,480.81) -- (154.79,480.81);

\draw[color=drawColor,line cap=round,line join=round,fill opacity=0.00,] (152.11,509.56) -- (154.79,509.56);

\draw[color=drawColor,line cap=round,line join=round,fill opacity=0.00,] (150.76,491.22) --
	(156.13,491.22) --
	(156.13,498.58) --
	(150.76,498.58) --
	(150.76,491.22);

\draw[color=drawColor,line width= 1.2pt,line join=round,fill opacity=0.00,] (157.48,549.40) -- (162.85,549.40);

\draw[color=drawColor,dash pattern=on 4pt off 4pt ,line cap=round,line join=round,fill opacity=0.00,] (160.16,521.24) -- (160.16,542.59);

\draw[color=drawColor,dash pattern=on 4pt off 4pt ,line cap=round,line join=round,fill opacity=0.00,] (160.16,582.75) -- (160.16,562.67);

\draw[color=drawColor,line cap=round,line join=round,fill opacity=0.00,] (158.82,521.24) -- (161.51,521.24);

\draw[color=drawColor,line cap=round,line join=round,fill opacity=0.00,] (158.82,582.75) -- (161.51,582.75);

\draw[color=drawColor,line cap=round,line join=round,fill opacity=0.00,] (157.48,542.59) --
	(162.85,542.59) --
	(162.85,562.67) --
	(157.48,562.67) --
	(157.48,542.59);

\draw[color=drawColor,line width= 1.2pt,line join=round,fill opacity=0.00,] (164.19,504.65) -- (169.56,504.65);

\draw[color=drawColor,dash pattern=on 4pt off 4pt ,line cap=round,line join=round,fill opacity=0.00,] (166.88,494.18) -- (166.88,502.69);

\draw[color=drawColor,dash pattern=on 4pt off 4pt ,line cap=round,line join=round,fill opacity=0.00,] (166.88,524.11) -- (166.88,511.41);

\draw[color=drawColor,line cap=round,line join=round,fill opacity=0.00,] (165.53,494.18) -- (168.22,494.18);

\draw[color=drawColor,line cap=round,line join=round,fill opacity=0.00,] (165.53,524.11) -- (168.22,524.11);

\draw[color=drawColor,line cap=round,line join=round,fill opacity=0.00,] (164.19,502.69) --
	(169.56,502.69) --
	(169.56,511.41) --
	(164.19,511.41) --
	(164.19,502.69);

\draw[color=drawColor,line width= 1.2pt,line join=round,fill opacity=0.00,] (170.90,512.10) -- (176.28,512.10);

\draw[color=drawColor,dash pattern=on 4pt off 4pt ,line cap=round,line join=round,fill opacity=0.00,] (173.59,497.27) -- (173.59,508.29);

\draw[color=drawColor,dash pattern=on 4pt off 4pt ,line cap=round,line join=round,fill opacity=0.00,] (173.59,536.38) -- (173.59,520.50);

\draw[color=drawColor,line cap=round,line join=round,fill opacity=0.00,] (172.25,497.27) -- (174.93,497.27);

\draw[color=drawColor,line cap=round,line join=round,fill opacity=0.00,] (172.25,536.38) -- (174.93,536.38);

\draw[color=drawColor,line cap=round,line join=round,fill opacity=0.00,] (170.90,508.29) --
	(176.28,508.29) --
	(176.28,520.50) --
	(170.90,520.50) --
	(170.90,508.29);

\draw[color=drawColor,line width= 1.2pt,line join=round,fill opacity=0.00,] (177.62,515.00) -- (182.99,515.00);

\draw[color=drawColor,dash pattern=on 4pt off 4pt ,line cap=round,line join=round,fill opacity=0.00,] (180.30,495.15) -- (180.30,508.31);

\draw[color=drawColor,dash pattern=on 4pt off 4pt ,line cap=round,line join=round,fill opacity=0.00,] (180.30,539.84) -- (180.30,524.06);

\draw[color=drawColor,line cap=round,line join=round,fill opacity=0.00,] (178.96,495.15) -- (181.65,495.15);

\draw[color=drawColor,line cap=round,line join=round,fill opacity=0.00,] (178.96,539.84) -- (181.65,539.84);

\draw[color=drawColor,line cap=round,line join=round,fill opacity=0.00,] (177.62,508.31) --
	(182.99,508.31) --
	(182.99,524.06) --
	(177.62,524.06) --
	(177.62,508.31);

\draw[color=drawColor,line width= 1.2pt,line join=round,fill opacity=0.00,] (184.33,548.35) -- (189.70,548.35);

\draw[color=drawColor,dash pattern=on 4pt off 4pt ,line cap=round,line join=round,fill opacity=0.00,] (187.02,531.31) -- (187.02,544.72);

\draw[color=drawColor,dash pattern=on 4pt off 4pt ,line cap=round,line join=round,fill opacity=0.00,] (187.02,574.73) -- (187.02,562.18);

\draw[color=drawColor,line cap=round,line join=round,fill opacity=0.00,] (185.67,531.31) -- (188.36,531.31);

\draw[color=drawColor,line cap=round,line join=round,fill opacity=0.00,] (185.67,574.73) -- (188.36,574.73);

\draw[color=drawColor,line cap=round,line join=round,fill opacity=0.00,] (184.33,544.72) --
	(189.70,544.72) --
	(189.70,562.18) --
	(184.33,562.18) --
	(184.33,544.72);

\draw[color=drawColor,line width= 1.2pt,line join=round,fill opacity=0.00,] (191.04,520.01) -- (196.42,520.01);

\draw[color=drawColor,dash pattern=on 4pt off 4pt ,line cap=round,line join=round,fill opacity=0.00,] (193.73,505.10) -- (193.73,511.52);

\draw[color=drawColor,dash pattern=on 4pt off 4pt ,line cap=round,line join=round,fill opacity=0.00,] (193.73,540.26) -- (193.73,530.99);

\draw[color=drawColor,line cap=round,line join=round,fill opacity=0.00,] (192.39,505.10) -- (195.07,505.10);

\draw[color=drawColor,line cap=round,line join=round,fill opacity=0.00,] (192.39,540.26) -- (195.07,540.26);

\draw[color=drawColor,line cap=round,line join=round,fill opacity=0.00,] (191.04,511.52) --
	(196.42,511.52) --
	(196.42,530.99) --
	(191.04,530.99) --
	(191.04,511.52);

\draw[color=drawColor,line width= 1.2pt,line join=round,fill opacity=0.00,] (197.76,528.02) -- (203.13,528.02);

\draw[color=drawColor,dash pattern=on 4pt off 4pt ,line cap=round,line join=round,fill opacity=0.00,] (200.44,507.64) -- (200.44,521.69);

\draw[color=drawColor,dash pattern=on 4pt off 4pt ,line cap=round,line join=round,fill opacity=0.00,] (200.44,548.18) -- (200.44,534.27);

\draw[color=drawColor,line cap=round,line join=round,fill opacity=0.00,] (199.10,507.64) -- (201.79,507.64);

\draw[color=drawColor,line cap=round,line join=round,fill opacity=0.00,] (199.10,548.18) -- (201.79,548.18);

\draw[color=drawColor,line cap=round,line join=round,fill opacity=0.00,] (197.76,521.69) --
	(203.13,521.69) --
	(203.13,534.27) --
	(197.76,534.27) --
	(197.76,521.69);

\draw[color=drawColor,line width= 1.2pt,line join=round,fill opacity=0.00,] (204.47,526.86) -- (209.84,526.86);

\draw[color=drawColor,dash pattern=on 4pt off 4pt ,line cap=round,line join=round,fill opacity=0.00,] (207.16,509.63) -- (207.16,521.36);

\draw[color=drawColor,dash pattern=on 4pt off 4pt ,line cap=round,line join=round,fill opacity=0.00,] (207.16,541.51) -- (207.16,529.42);

\draw[color=drawColor,line cap=round,line join=round,fill opacity=0.00,] (205.81,509.63) -- (208.50,509.63);

\draw[color=drawColor,line cap=round,line join=round,fill opacity=0.00,] (205.81,541.51) -- (208.50,541.51);

\draw[color=drawColor,line cap=round,line join=round,fill opacity=0.00,] (204.47,521.36) --
	(209.84,521.36) --
	(209.84,529.42) --
	(204.47,529.42) --
	(204.47,521.36);

\draw[color=drawColor,line width= 1.2pt,line join=round,fill opacity=0.00,] (211.19,529.71) -- (216.56,529.71);

\draw[color=drawColor,dash pattern=on 4pt off 4pt ,line cap=round,line join=round,fill opacity=0.00,] (213.87,511.11) -- (213.87,525.18);

\draw[color=drawColor,dash pattern=on 4pt off 4pt ,line cap=round,line join=round,fill opacity=0.00,] (213.87,548.34) -- (213.87,534.57);

\draw[color=drawColor,line cap=round,line join=round,fill opacity=0.00,] (212.53,511.11) -- (215.21,511.11);

\draw[color=drawColor,line cap=round,line join=round,fill opacity=0.00,] (212.53,548.34) -- (215.21,548.34);

\draw[color=drawColor,line cap=round,line join=round,fill opacity=0.00,] (211.19,525.18) --
	(216.56,525.18) --
	(216.56,534.57) --
	(211.19,534.57) --
	(211.19,525.18);

\draw[color=drawColor,line width= 1.2pt,line join=round,fill opacity=0.00,] (217.90,509.57) -- (223.27,509.57);

\draw[color=drawColor,dash pattern=on 4pt off 4pt ,line cap=round,line join=round,fill opacity=0.00,] (220.58,494.40) -- (220.58,503.99);

\draw[color=drawColor,dash pattern=on 4pt off 4pt ,line cap=round,line join=round,fill opacity=0.00,] (220.58,534.04) -- (220.58,516.02);

\draw[color=drawColor,line cap=round,line join=round,fill opacity=0.00,] (219.24,494.40) -- (221.93,494.40);

\draw[color=drawColor,line cap=round,line join=round,fill opacity=0.00,] (219.24,534.04) -- (221.93,534.04);

\draw[color=drawColor,line cap=round,line join=round,fill opacity=0.00,] (217.90,503.99) --
	(223.27,503.99) --
	(223.27,516.02) --
	(217.90,516.02) --
	(217.90,503.99);

\draw[color=drawColor,line width= 1.2pt,line join=round,fill opacity=0.00,] (224.61,529.72) -- (229.98,529.72);

\draw[color=drawColor,dash pattern=on 4pt off 4pt ,line cap=round,line join=round,fill opacity=0.00,] (227.30,516.58) -- (227.30,526.96);

\draw[color=drawColor,dash pattern=on 4pt off 4pt ,line cap=round,line join=round,fill opacity=0.00,] (227.30,544.24) -- (227.30,533.93);

\draw[color=drawColor,line cap=round,line join=round,fill opacity=0.00,] (225.95,516.58) -- (228.64,516.58);

\draw[color=drawColor,line cap=round,line join=round,fill opacity=0.00,] (225.95,544.24) -- (228.64,544.24);

\draw[color=drawColor,line cap=round,line join=round,fill opacity=0.00,] (224.61,526.96) --
	(229.98,526.96) --
	(229.98,533.93) --
	(224.61,533.93) --
	(224.61,526.96);

\draw[color=drawColor,line width= 1.2pt,line join=round,fill opacity=0.00,] (231.33,506.38) -- (236.70,506.38);

\draw[color=drawColor,dash pattern=on 4pt off 4pt ,line cap=round,line join=round,fill opacity=0.00,] (234.01,494.09) -- (234.01,503.44);

\draw[color=drawColor,dash pattern=on 4pt off 4pt ,line cap=round,line join=round,fill opacity=0.00,] (234.01,527.20) -- (234.01,513.22);

\draw[color=drawColor,line cap=round,line join=round,fill opacity=0.00,] (232.67,494.09) -- (235.35,494.09);

\draw[color=drawColor,line cap=round,line join=round,fill opacity=0.00,] (232.67,527.20) -- (235.35,527.20);

\draw[color=drawColor,line cap=round,line join=round,fill opacity=0.00,] (231.33,503.44) --
	(236.70,503.44) --
	(236.70,513.22) --
	(231.33,513.22) --
	(231.33,503.44);

\draw[color=drawColor,line width= 1.2pt,line join=round,fill opacity=0.00,] (238.04,470.40) -- (243.41,470.40);

\draw[color=drawColor,dash pattern=on 4pt off 4pt ,line cap=round,line join=round,fill opacity=0.00,] (240.72,455.93) -- (240.72,466.53);

\draw[color=drawColor,dash pattern=on 4pt off 4pt ,line cap=round,line join=round,fill opacity=0.00,] (240.72,486.01) -- (240.72,474.35);

\draw[color=drawColor,line cap=round,line join=round,fill opacity=0.00,] (239.38,455.93) -- (242.07,455.93);

\draw[color=drawColor,line cap=round,line join=round,fill opacity=0.00,] (239.38,486.01) -- (242.07,486.01);

\draw[color=drawColor,line cap=round,line join=round,fill opacity=0.00,] (238.04,466.53) --
	(243.41,466.53) --
	(243.41,474.35) --
	(238.04,474.35) --
	(238.04,466.53);

\draw[color=drawColor,line width= 1.2pt,line join=round,fill opacity=0.00,] (244.75,538.56) -- (250.12,538.56);

\draw[color=drawColor,dash pattern=on 4pt off 4pt ,line cap=round,line join=round,fill opacity=0.00,] (247.44,520.78) -- (247.44,535.60);

\draw[color=drawColor,dash pattern=on 4pt off 4pt ,line cap=round,line join=round,fill opacity=0.00,] (247.44,560.40) -- (247.44,545.57);

\draw[color=drawColor,line cap=round,line join=round,fill opacity=0.00,] (246.10,520.78) -- (248.78,520.78);

\draw[color=drawColor,line cap=round,line join=round,fill opacity=0.00,] (246.10,560.40) -- (248.78,560.40);

\draw[color=drawColor,line cap=round,line join=round,fill opacity=0.00,] (244.75,535.60) --
	(250.12,535.60) --
	(250.12,545.57) --
	(244.75,545.57) --
	(244.75,535.60);

\draw[color=drawColor,line width= 1.2pt,line join=round,fill opacity=0.00,] (251.47,538.03) -- (256.84,538.03);

\draw[color=drawColor,dash pattern=on 4pt off 4pt ,line cap=round,line join=round,fill opacity=0.00,] (254.15,522.36) -- (254.15,534.84);

\draw[color=drawColor,dash pattern=on 4pt off 4pt ,line cap=round,line join=round,fill opacity=0.00,] (254.15,557.02) -- (254.15,543.72);

\draw[color=drawColor,line cap=round,line join=round,fill opacity=0.00,] (252.81,522.36) -- (255.49,522.36);

\draw[color=drawColor,line cap=round,line join=round,fill opacity=0.00,] (252.81,557.02) -- (255.49,557.02);

\draw[color=drawColor,line cap=round,line join=round,fill opacity=0.00,] (251.47,534.84) --
	(256.84,534.84) --
	(256.84,543.72) --
	(251.47,543.72) --
	(251.47,534.84);

\draw[color=drawColor,line width= 1.2pt,line join=round,fill opacity=0.00,] (258.18,547.83) -- (263.55,547.83);

\draw[color=drawColor,dash pattern=on 4pt off 4pt ,line cap=round,line join=round,fill opacity=0.00,] (260.87,523.32) -- (260.87,544.19);

\draw[color=drawColor,dash pattern=on 4pt off 4pt ,line cap=round,line join=round,fill opacity=0.00,] (260.87,573.45) -- (260.87,561.69);

\draw[color=drawColor,line cap=round,line join=round,fill opacity=0.00,] (259.52,523.32) -- (262.21,523.32);

\draw[color=drawColor,line cap=round,line join=round,fill opacity=0.00,] (259.52,573.45) -- (262.21,573.45);

\draw[color=drawColor,line cap=round,line join=round,fill opacity=0.00,] (258.18,544.19) --
	(263.55,544.19) --
	(263.55,561.69) --
	(258.18,561.69) --
	(258.18,544.19);

\draw[color=drawColor,line width= 1.2pt,line join=round,fill opacity=0.00,] (264.89,479.52) -- (270.26,479.52);

\draw[color=drawColor,dash pattern=on 4pt off 4pt ,line cap=round,line join=round,fill opacity=0.00,] (267.58,469.52) -- (267.58,477.02);

\draw[color=drawColor,dash pattern=on 4pt off 4pt ,line cap=round,line join=round,fill opacity=0.00,] (267.58,490.46) -- (267.58,482.50);

\draw[color=drawColor,line cap=round,line join=round,fill opacity=0.00,] (266.24,469.52) -- (268.92,469.52);

\draw[color=drawColor,line cap=round,line join=round,fill opacity=0.00,] (266.24,490.46) -- (268.92,490.46);

\draw[color=drawColor,line cap=round,line join=round,fill opacity=0.00,] (264.89,477.02) --
	(270.26,477.02) --
	(270.26,482.50) --
	(264.89,482.50) --
	(264.89,477.02);

\draw[color=drawColor,line width= 1.2pt,line join=round,fill opacity=0.00,] (271.61,541.13) -- (276.98,541.13);

\draw[color=drawColor,dash pattern=on 4pt off 4pt ,line cap=round,line join=round,fill opacity=0.00,] (274.29,522.79) -- (274.29,536.61);

\draw[color=drawColor,dash pattern=on 4pt off 4pt ,line cap=round,line join=round,fill opacity=0.00,] (274.29,566.27) -- (274.29,548.57);

\draw[color=drawColor,line cap=round,line join=round,fill opacity=0.00,] (272.95,522.79) -- (275.63,522.79);

\draw[color=drawColor,line cap=round,line join=round,fill opacity=0.00,] (272.95,566.27) -- (275.63,566.27);

\draw[color=drawColor,line cap=round,line join=round,fill opacity=0.00,] (271.61,536.61) --
	(276.98,536.61) --
	(276.98,548.57) --
	(271.61,548.57) --
	(271.61,536.61);

\draw[color=drawColor,line width= 1.2pt,line join=round,fill opacity=0.00,] (278.32,545.18) -- (283.69,545.18);

\draw[color=drawColor,dash pattern=on 4pt off 4pt ,line cap=round,line join=round,fill opacity=0.00,] (281.01,515.04) -- (281.01,533.13);

\draw[color=drawColor,dash pattern=on 4pt off 4pt ,line cap=round,line join=round,fill opacity=0.00,] (281.01,579.44) -- (281.01,555.44);

\draw[color=drawColor,line cap=round,line join=round,fill opacity=0.00,] (279.66,515.04) -- (282.35,515.04);

\draw[color=drawColor,line cap=round,line join=round,fill opacity=0.00,] (279.66,579.44) -- (282.35,579.44);

\draw[color=drawColor,line cap=round,line join=round,fill opacity=0.00,] (278.32,533.13) --
	(283.69,533.13) --
	(283.69,555.44) --
	(278.32,555.44) --
	(278.32,533.13);

\draw[color=drawColor,line width= 1.2pt,line join=round,fill opacity=0.00,] (285.03,552.63) -- (290.40,552.63);

\draw[color=drawColor,dash pattern=on 4pt off 4pt ,line cap=round,line join=round,fill opacity=0.00,] (287.72,523.97) -- (287.72,545.34);

\draw[color=drawColor,dash pattern=on 4pt off 4pt ,line cap=round,line join=round,fill opacity=0.00,] (287.72,581.49) -- (287.72,563.71);

\draw[color=drawColor,line cap=round,line join=round,fill opacity=0.00,] (286.38,523.97) -- (289.06,523.97);

\draw[color=drawColor,line cap=round,line join=round,fill opacity=0.00,] (286.38,581.49) -- (289.06,581.49);

\draw[color=drawColor,line cap=round,line join=round,fill opacity=0.00,] (285.03,545.34) --
	(290.40,545.34) --
	(290.40,563.71) --
	(285.03,563.71) --
	(285.03,545.34);

\draw[color=drawColor,line width= 1.2pt,line join=round,fill opacity=0.00,] (291.75,523.49) -- (297.12,523.49);

\draw[color=drawColor,dash pattern=on 4pt off 4pt ,line cap=round,line join=round,fill opacity=0.00,] (294.43,507.21) -- (294.43,517.45);

\draw[color=drawColor,dash pattern=on 4pt off 4pt ,line cap=round,line join=round,fill opacity=0.00,] (294.43,541.95) -- (294.43,527.27);

\draw[color=drawColor,line cap=round,line join=round,fill opacity=0.00,] (293.09,507.21) -- (295.78,507.21);

\draw[color=drawColor,line cap=round,line join=round,fill opacity=0.00,] (293.09,541.95) -- (295.78,541.95);

\draw[color=drawColor,line cap=round,line join=round,fill opacity=0.00,] (291.75,517.45) --
	(297.12,517.45) --
	(297.12,527.27) --
	(291.75,527.27) --
	(291.75,517.45);

\draw[color=drawColor,line width= 1.2pt,line join=round,fill opacity=0.00,] (298.46,548.29) -- (303.83,548.29);

\draw[color=drawColor,dash pattern=on 4pt off 4pt ,line cap=round,line join=round,fill opacity=0.00,] (301.15,523.07) -- (301.15,540.12);

\draw[color=drawColor,dash pattern=on 4pt off 4pt ,line cap=round,line join=round,fill opacity=0.00,] (301.15,579.92) -- (301.15,562.00);

\draw[color=drawColor,line cap=round,line join=round,fill opacity=0.00,] (299.80,523.07) -- (302.49,523.07);

\draw[color=drawColor,line cap=round,line join=round,fill opacity=0.00,] (299.80,579.92) -- (302.49,579.92);

\draw[color=drawColor,line cap=round,line join=round,fill opacity=0.00,] (298.46,540.12) --
	(303.83,540.12) --
	(303.83,562.00) --
	(298.46,562.00) --
	(298.46,540.12);

\draw[color=drawColor,line width= 1.2pt,line join=round,fill opacity=0.00,] (305.17,504.45) -- (310.54,504.45);

\draw[color=drawColor,dash pattern=on 4pt off 4pt ,line cap=round,line join=round,fill opacity=0.00,] (307.86,490.76) -- (307.86,498.32);

\draw[color=drawColor,dash pattern=on 4pt off 4pt ,line cap=round,line join=round,fill opacity=0.00,] (307.86,518.13) -- (307.86,508.50);

\draw[color=drawColor,line cap=round,line join=round,fill opacity=0.00,] (306.52,490.76) -- (309.20,490.76);

\draw[color=drawColor,line cap=round,line join=round,fill opacity=0.00,] (306.52,518.13) -- (309.20,518.13);

\draw[color=drawColor,line cap=round,line join=round,fill opacity=0.00,] (305.17,498.32) --
	(310.54,498.32) --
	(310.54,508.50) --
	(305.17,508.50) --
	(305.17,498.32);

\draw[color=drawColor,line width= 1.2pt,line join=round,fill opacity=0.00,] (311.89,509.27) -- (317.26,509.27);

\draw[color=drawColor,dash pattern=on 4pt off 4pt ,line cap=round,line join=round,fill opacity=0.00,] (314.57,491.02) -- (314.57,504.75);

\draw[color=drawColor,dash pattern=on 4pt off 4pt ,line cap=round,line join=round,fill opacity=0.00,] (314.57,535.12) -- (314.57,516.98);

\draw[color=drawColor,line cap=round,line join=round,fill opacity=0.00,] (313.23,491.02) -- (315.92,491.02);

\draw[color=drawColor,line cap=round,line join=round,fill opacity=0.00,] (313.23,535.12) -- (315.92,535.12);

\draw[color=drawColor,line cap=round,line join=round,fill opacity=0.00,] (311.89,504.75) --
	(317.26,504.75) --
	(317.26,516.98) --
	(311.89,516.98) --
	(311.89,504.75);

\draw[color=drawColor,line width= 1.2pt,line join=round,fill opacity=0.00,] (318.60,501.77) -- (323.97,501.77);

\draw[color=drawColor,dash pattern=on 4pt off 4pt ,line cap=round,line join=round,fill opacity=0.00,] (321.29,489.81) -- (321.29,497.49);

\draw[color=drawColor,dash pattern=on 4pt off 4pt ,line cap=round,line join=round,fill opacity=0.00,] (321.29,517.16) -- (321.29,507.65);

\draw[color=drawColor,line cap=round,line join=round,fill opacity=0.00,] (319.94,489.81) -- (322.63,489.81);

\draw[color=drawColor,line cap=round,line join=round,fill opacity=0.00,] (319.94,517.16) -- (322.63,517.16);

\draw[color=drawColor,line cap=round,line join=round,fill opacity=0.00,] (318.60,497.49) --
	(323.97,497.49) --
	(323.97,507.65) --
	(318.60,507.65) --
	(318.60,497.49);

\draw[color=drawColor,line width= 1.2pt,line join=round,fill opacity=0.00,] (325.31,487.71) -- (330.69,487.71);

\draw[color=drawColor,dash pattern=on 4pt off 4pt ,line cap=round,line join=round,fill opacity=0.00,] (328.00,476.88) -- (328.00,483.52);

\draw[color=drawColor,dash pattern=on 4pt off 4pt ,line cap=round,line join=round,fill opacity=0.00,] (328.00,506.47) -- (328.00,493.68);

\draw[color=drawColor,line cap=round,line join=round,fill opacity=0.00,] (326.66,476.88) -- (329.34,476.88);

\draw[color=drawColor,line cap=round,line join=round,fill opacity=0.00,] (326.66,506.47) -- (329.34,506.47);

\draw[color=drawColor,line cap=round,line join=round,fill opacity=0.00,] (325.31,483.52) --
	(330.69,483.52) --
	(330.69,493.68) --
	(325.31,493.68) --
	(325.31,483.52);

\draw[color=drawColor,line width= 1.2pt,line join=round,fill opacity=0.00,] (332.03,500.84) -- (337.40,500.84);

\draw[color=drawColor,dash pattern=on 4pt off 4pt ,line cap=round,line join=round,fill opacity=0.00,] (334.71,489.39) -- (334.71,497.22);

\draw[color=drawColor,dash pattern=on 4pt off 4pt ,line cap=round,line join=round,fill opacity=0.00,] (334.71,514.74) -- (334.71,505.61);

\draw[color=drawColor,line cap=round,line join=round,fill opacity=0.00,] (333.37,489.39) -- (336.06,489.39);

\draw[color=drawColor,line cap=round,line join=round,fill opacity=0.00,] (333.37,514.74) -- (336.06,514.74);

\draw[color=drawColor,line cap=round,line join=round,fill opacity=0.00,] (332.03,497.22) --
	(337.40,497.22) --
	(337.40,505.61) --
	(332.03,505.61) --
	(332.03,497.22);

\draw[color=drawColor,line width= 1.2pt,line join=round,fill opacity=0.00,] (338.74,496.07) -- (344.11,496.07);

\draw[color=drawColor,dash pattern=on 4pt off 4pt ,line cap=round,line join=round,fill opacity=0.00,] (341.43,482.88) -- (341.43,493.11);

\draw[color=drawColor,dash pattern=on 4pt off 4pt ,line cap=round,line join=round,fill opacity=0.00,] (341.43,509.94) -- (341.43,499.98);

\draw[color=drawColor,line cap=round,line join=round,fill opacity=0.00,] (340.08,482.88) -- (342.77,482.88);

\draw[color=drawColor,line cap=round,line join=round,fill opacity=0.00,] (340.08,509.94) -- (342.77,509.94);

\draw[color=drawColor,line cap=round,line join=round,fill opacity=0.00,] (338.74,493.11) --
	(344.11,493.11) --
	(344.11,499.98) --
	(338.74,499.98) --
	(338.74,493.11);

\draw[color=drawColor,line width= 1.2pt,line join=round,fill opacity=0.00,] (345.45,547.48) -- (350.83,547.48);

\draw[color=drawColor,dash pattern=on 4pt off 4pt ,line cap=round,line join=round,fill opacity=0.00,] (348.14,523.30) -- (348.14,541.82);

\draw[color=drawColor,dash pattern=on 4pt off 4pt ,line cap=round,line join=round,fill opacity=0.00,] (348.14,575.47) -- (348.14,561.25);

\draw[color=drawColor,line cap=round,line join=round,fill opacity=0.00,] (346.80,523.30) -- (349.48,523.30);

\draw[color=drawColor,line cap=round,line join=round,fill opacity=0.00,] (346.80,575.47) -- (349.48,575.47);

\draw[color=drawColor,line cap=round,line join=round,fill opacity=0.00,] (345.45,541.82) --
	(350.83,541.82) --
	(350.83,561.25) --
	(345.45,561.25) --
	(345.45,541.82);

\draw[color=drawColor,line width= 1.2pt,line join=round,fill opacity=0.00,] (352.17,502.77) -- (357.54,502.77);

\draw[color=drawColor,dash pattern=on 4pt off 4pt ,line cap=round,line join=round,fill opacity=0.00,] (354.85,490.75) -- (354.85,497.93);

\draw[color=drawColor,dash pattern=on 4pt off 4pt ,line cap=round,line join=round,fill opacity=0.00,] (354.85,517.18) -- (354.85,507.29);

\draw[color=drawColor,line cap=round,line join=round,fill opacity=0.00,] (353.51,490.75) -- (356.20,490.75);

\draw[color=drawColor,line cap=round,line join=round,fill opacity=0.00,] (353.51,517.18) -- (356.20,517.18);

\draw[color=drawColor,line cap=round,line join=round,fill opacity=0.00,] (352.17,497.93) --
	(357.54,497.93) --
	(357.54,507.29) --
	(352.17,507.29) --
	(352.17,497.93);

\draw[color=drawColor,line width= 1.2pt,line join=round,fill opacity=0.00,] (358.88,525.55) -- (364.25,525.55);

\draw[color=drawColor,dash pattern=on 4pt off 4pt ,line cap=round,line join=round,fill opacity=0.00,] (361.57,508.29) -- (361.57,518.89);

\draw[color=drawColor,dash pattern=on 4pt off 4pt ,line cap=round,line join=round,fill opacity=0.00,] (361.57,543.74) -- (361.57,528.95);

\draw[color=drawColor,line cap=round,line join=round,fill opacity=0.00,] (360.22,508.29) -- (362.91,508.29);

\draw[color=drawColor,line cap=round,line join=round,fill opacity=0.00,] (360.22,543.74) -- (362.91,543.74);

\draw[color=drawColor,line cap=round,line join=round,fill opacity=0.00,] (358.88,518.89) --
	(364.25,518.89) --
	(364.25,528.95) --
	(358.88,528.95) --
	(358.88,518.89);

\draw[color=drawColor,line width= 1.2pt,line join=round,fill opacity=0.00,] (365.60,516.95) -- (370.97,516.95);

\draw[color=drawColor,dash pattern=on 4pt off 4pt ,line cap=round,line join=round,fill opacity=0.00,] (368.28,497.05) -- (368.28,510.14);

\draw[color=drawColor,dash pattern=on 4pt off 4pt ,line cap=round,line join=round,fill opacity=0.00,] (368.28,544.16) -- (368.28,526.80);

\draw[color=drawColor,line cap=round,line join=round,fill opacity=0.00,] (366.94,497.05) -- (369.62,497.05);

\draw[color=drawColor,line cap=round,line join=round,fill opacity=0.00,] (366.94,544.16) -- (369.62,544.16);

\draw[color=drawColor,line cap=round,line join=round,fill opacity=0.00,] (365.60,510.14) --
	(370.97,510.14) --
	(370.97,526.80) --
	(365.60,526.80) --
	(365.60,510.14);

\draw[color=drawColor,line width= 1.2pt,line join=round,fill opacity=0.00,] (372.31,556.94) -- (377.68,556.94);

\draw[color=drawColor,dash pattern=on 4pt off 4pt ,line cap=round,line join=round,fill opacity=0.00,] (374.99,535.00) -- (374.99,551.90);

\draw[color=drawColor,dash pattern=on 4pt off 4pt ,line cap=round,line join=round,fill opacity=0.00,] (374.99,585.26) -- (374.99,572.35);

\draw[color=drawColor,line cap=round,line join=round,fill opacity=0.00,] (373.65,535.00) -- (376.34,535.00);

\draw[color=drawColor,line cap=round,line join=round,fill opacity=0.00,] (373.65,585.26) -- (376.34,585.26);

\draw[color=drawColor,line cap=round,line join=round,fill opacity=0.00,] (372.31,551.90) --
	(377.68,551.90) --
	(377.68,572.35) --
	(372.31,572.35) --
	(372.31,551.90);

\draw[color=drawColor,line width= 1.2pt,line join=round,fill opacity=0.00,] (379.02,521.27) -- (384.39,521.27);

\draw[color=drawColor,dash pattern=on 4pt off 4pt ,line cap=round,line join=round,fill opacity=0.00,] (381.71,494.02) -- (381.71,506.64);

\draw[color=drawColor,dash pattern=on 4pt off 4pt ,line cap=round,line join=round,fill opacity=0.00,] (381.71,563.07) -- (381.71,532.48);

\draw[color=drawColor,line cap=round,line join=round,fill opacity=0.00,] (380.37,494.02) -- (383.05,494.02);

\draw[color=drawColor,line cap=round,line join=round,fill opacity=0.00,] (380.37,563.07) -- (383.05,563.07);

\draw[color=drawColor,line cap=round,line join=round,fill opacity=0.00,] (379.02,506.64) --
	(384.39,506.64) --
	(384.39,532.48) --
	(379.02,532.48) --
	(379.02,506.64);

\draw[color=drawColor,line width= 1.2pt,line join=round,fill opacity=0.00,] (385.74,503.51) -- (391.11,503.51);

\draw[color=drawColor,dash pattern=on 4pt off 4pt ,line cap=round,line join=round,fill opacity=0.00,] (388.42,490.90) -- (388.42,497.65);

\draw[color=drawColor,dash pattern=on 4pt off 4pt ,line cap=round,line join=round,fill opacity=0.00,] (388.42,517.60) -- (388.42,508.74);

\draw[color=drawColor,line cap=round,line join=round,fill opacity=0.00,] (387.08,490.90) -- (389.76,490.90);

\draw[color=drawColor,line cap=round,line join=round,fill opacity=0.00,] (387.08,517.60) -- (389.76,517.60);

\draw[color=drawColor,line cap=round,line join=round,fill opacity=0.00,] (385.74,497.65) --
	(391.11,497.65) --
	(391.11,508.74) --
	(385.74,508.74) --
	(385.74,497.65);

\draw[color=drawColor,line width= 1.2pt,line join=round,fill opacity=0.00,] (392.45,489.55) -- (397.82,489.55);

\draw[color=drawColor,dash pattern=on 4pt off 4pt ,line cap=round,line join=round,fill opacity=0.00,] (395.13,473.42) -- (395.13,482.22);

\draw[color=drawColor,dash pattern=on 4pt off 4pt ,line cap=round,line join=round,fill opacity=0.00,] (395.13,512.63) -- (395.13,496.46);

\draw[color=drawColor,line cap=round,line join=round,fill opacity=0.00,] (393.79,473.42) -- (396.48,473.42);

\draw[color=drawColor,line cap=round,line join=round,fill opacity=0.00,] (393.79,512.63) -- (396.48,512.63);

\draw[color=drawColor,line cap=round,line join=round,fill opacity=0.00,] (392.45,482.22) --
	(397.82,482.22) --
	(397.82,496.46) --
	(392.45,496.46) --
	(392.45,482.22);

\draw[color=drawColor,line width= 1.2pt,line join=round,fill opacity=0.00,] (399.16,504.13) -- (404.53,504.13);

\draw[color=drawColor,dash pattern=on 4pt off 4pt ,line cap=round,line join=round,fill opacity=0.00,] (401.85,490.79) -- (401.85,498.08);

\draw[color=drawColor,dash pattern=on 4pt off 4pt ,line cap=round,line join=round,fill opacity=0.00,] (401.85,515.76) -- (401.85,507.59);

\draw[color=drawColor,line cap=round,line join=round,fill opacity=0.00,] (400.51,490.79) -- (403.19,490.79);

\draw[color=drawColor,line cap=round,line join=round,fill opacity=0.00,] (400.51,515.76) -- (403.19,515.76);

\draw[color=drawColor,line cap=round,line join=round,fill opacity=0.00,] (399.16,498.08) --
	(404.53,498.08) --
	(404.53,507.59) --
	(399.16,507.59) --
	(399.16,498.08);

\draw[color=drawColor,line width= 1.2pt,line join=round,fill opacity=0.00,] (405.88,482.13) -- (411.25,482.13);

\draw[color=drawColor,dash pattern=on 4pt off 4pt ,line cap=round,line join=round,fill opacity=0.00,] (408.56,472.34) -- (408.56,479.53);

\draw[color=drawColor,dash pattern=on 4pt off 4pt ,line cap=round,line join=round,fill opacity=0.00,] (408.56,497.01) -- (408.56,486.58);

\draw[color=drawColor,line cap=round,line join=round,fill opacity=0.00,] (407.22,472.34) -- (409.90,472.34);

\draw[color=drawColor,line cap=round,line join=round,fill opacity=0.00,] (407.22,497.01) -- (409.90,497.01);

\draw[color=drawColor,line cap=round,line join=round,fill opacity=0.00,] (405.88,479.53) --
	(411.25,479.53) --
	(411.25,486.58) --
	(405.88,486.58) --
	(405.88,479.53);

\draw[color=drawColor,line width= 1.2pt,line join=round,fill opacity=0.00,] (412.59,500.72) -- (417.96,500.72);

\draw[color=drawColor,dash pattern=on 4pt off 4pt ,line cap=round,line join=round,fill opacity=0.00,] (415.28,488.77) -- (415.28,497.19);

\draw[color=drawColor,dash pattern=on 4pt off 4pt ,line cap=round,line join=round,fill opacity=0.00,] (415.28,515.32) -- (415.28,504.66);

\draw[color=drawColor,line cap=round,line join=round,fill opacity=0.00,] (413.93,488.77) -- (416.62,488.77);

\draw[color=drawColor,line cap=round,line join=round,fill opacity=0.00,] (413.93,515.32) -- (416.62,515.32);

\draw[color=drawColor,line cap=round,line join=round,fill opacity=0.00,] (412.59,497.19) --
	(417.96,497.19) --
	(417.96,504.66) --
	(412.59,504.66) --
	(412.59,497.19);

\draw[color=drawColor,line width= 1.2pt,line join=round,fill opacity=0.00,] (419.30,500.97) -- (424.67,500.97);

\draw[color=drawColor,dash pattern=on 4pt off 4pt ,line cap=round,line join=round,fill opacity=0.00,] (421.99,484.72) -- (421.99,497.02);

\draw[color=drawColor,dash pattern=on 4pt off 4pt ,line cap=round,line join=round,fill opacity=0.00,] (421.99,516.16) -- (421.99,505.99);

\draw[color=drawColor,line cap=round,line join=round,fill opacity=0.00,] (420.65,484.72) -- (423.33,484.72);

\draw[color=drawColor,line cap=round,line join=round,fill opacity=0.00,] (420.65,516.16) -- (423.33,516.16);

\draw[color=drawColor,line cap=round,line join=round,fill opacity=0.00,] (419.30,497.02) --
	(424.67,497.02) --
	(424.67,505.99) --
	(419.30,505.99) --
	(419.30,497.02);

\draw[color=drawColor,line width= 1.2pt,line join=round,fill opacity=0.00,] (426.02,533.90) -- (431.39,533.90);

\draw[color=drawColor,dash pattern=on 4pt off 4pt ,line cap=round,line join=round,fill opacity=0.00,] (428.70,520.59) -- (428.70,530.61);

\draw[color=drawColor,dash pattern=on 4pt off 4pt ,line cap=round,line join=round,fill opacity=0.00,] (428.70,547.17) -- (428.70,537.35);

\draw[color=drawColor,line cap=round,line join=round,fill opacity=0.00,] (427.36,520.59) -- (430.04,520.59);

\draw[color=drawColor,line cap=round,line join=round,fill opacity=0.00,] (427.36,547.17) -- (430.04,547.17);

\draw[color=drawColor,line cap=round,line join=round,fill opacity=0.00,] (426.02,530.61) --
	(431.39,530.61) --
	(431.39,537.35) --
	(426.02,537.35) --
	(426.02,530.61);

\draw[color=drawColor,line width= 1.2pt,line join=round,fill opacity=0.00,] (432.73,502.34) -- (438.10,502.34);

\draw[color=drawColor,dash pattern=on 4pt off 4pt ,line cap=round,line join=round,fill opacity=0.00,] (435.42,485.69) -- (435.42,497.17);

\draw[color=drawColor,dash pattern=on 4pt off 4pt ,line cap=round,line join=round,fill opacity=0.00,] (435.42,520.31) -- (435.42,506.90);

\draw[color=drawColor,line cap=round,line join=round,fill opacity=0.00,] (434.07,485.69) -- (436.76,485.69);

\draw[color=drawColor,line cap=round,line join=round,fill opacity=0.00,] (434.07,520.31) -- (436.76,520.31);

\draw[color=drawColor,line cap=round,line join=round,fill opacity=0.00,] (432.73,497.17) --
	(438.10,497.17) --
	(438.10,506.90) --
	(432.73,506.90) --
	(432.73,497.17);

\draw[color=drawColor,line width= 1.2pt,line join=round,fill opacity=0.00,] (439.44,498.98) -- (444.81,498.98);

\draw[color=drawColor,dash pattern=on 4pt off 4pt ,line cap=round,line join=round,fill opacity=0.00,] (442.13,486.28) -- (442.13,495.92);

\draw[color=drawColor,dash pattern=on 4pt off 4pt ,line cap=round,line join=round,fill opacity=0.00,] (442.13,512.27) -- (442.13,502.65);

\draw[color=drawColor,line cap=round,line join=round,fill opacity=0.00,] (440.79,486.28) -- (443.47,486.28);

\draw[color=drawColor,line cap=round,line join=round,fill opacity=0.00,] (440.79,512.27) -- (443.47,512.27);

\draw[color=drawColor,line cap=round,line join=round,fill opacity=0.00,] (439.44,495.92) --
	(444.81,495.92) --
	(444.81,502.65) --
	(439.44,502.65) --
	(439.44,495.92);

\draw[color=drawColor,line width= 1.2pt,line join=round,fill opacity=0.00,] (446.16,521.10) -- (451.53,521.10);

\draw[color=drawColor,dash pattern=on 4pt off 4pt ,line cap=round,line join=round,fill opacity=0.00,] (448.84,492.62) -- (448.84,508.57);

\draw[color=drawColor,dash pattern=on 4pt off 4pt ,line cap=round,line join=round,fill opacity=0.00,] (448.84,562.04) -- (448.84,534.04);

\draw[color=drawColor,line cap=round,line join=round,fill opacity=0.00,] (447.50,492.62) -- (450.19,492.62);

\draw[color=drawColor,line cap=round,line join=round,fill opacity=0.00,] (447.50,562.04) -- (450.19,562.04);

\draw[color=drawColor,line cap=round,line join=round,fill opacity=0.00,] (446.16,508.57) --
	(451.53,508.57) --
	(451.53,534.04) --
	(446.16,534.04) --
	(446.16,508.57);

\draw[color=drawColor,line width= 1.2pt,line join=round,fill opacity=0.00,] (452.87,508.73) -- (458.24,508.73);

\draw[color=drawColor,dash pattern=on 4pt off 4pt ,line cap=round,line join=round,fill opacity=0.00,] (455.56,491.33) -- (455.56,502.67);

\draw[color=drawColor,dash pattern=on 4pt off 4pt ,line cap=round,line join=round,fill opacity=0.00,] (455.56,532.36) -- (455.56,514.57);

\draw[color=drawColor,line cap=round,line join=round,fill opacity=0.00,] (454.21,491.33) -- (456.90,491.33);

\draw[color=drawColor,line cap=round,line join=round,fill opacity=0.00,] (454.21,532.36) -- (456.90,532.36);

\draw[color=drawColor,line cap=round,line join=round,fill opacity=0.00,] (452.87,502.67) --
	(458.24,502.67) --
	(458.24,514.57) --
	(452.87,514.57) --
	(452.87,502.67);

\draw[color=drawColor,line width= 1.2pt,line join=round,fill opacity=0.00,] (459.58,498.72) -- (464.96,498.72);

\draw[color=drawColor,dash pattern=on 4pt off 4pt ,line cap=round,line join=round,fill opacity=0.00,] (462.27,482.72) -- (462.27,495.44);

\draw[color=drawColor,dash pattern=on 4pt off 4pt ,line cap=round,line join=round,fill opacity=0.00,] (462.27,515.49) -- (462.27,505.11);

\draw[color=drawColor,line cap=round,line join=round,fill opacity=0.00,] (460.93,482.72) -- (463.61,482.72);

\draw[color=drawColor,line cap=round,line join=round,fill opacity=0.00,] (460.93,515.49) -- (463.61,515.49);

\draw[color=drawColor,line cap=round,line join=round,fill opacity=0.00,] (459.58,495.44) --
	(464.96,495.44) --
	(464.96,505.11) --
	(459.58,505.11) --
	(459.58,495.44);

\draw[color=drawColor,line width= 1.2pt,line join=round,fill opacity=0.00,] (466.30,525.02) -- (471.67,525.02);

\draw[color=drawColor,dash pattern=on 4pt off 4pt ,line cap=round,line join=round,fill opacity=0.00,] (468.98,486.55) -- (468.98,508.32);

\draw[color=drawColor,dash pattern=on 4pt off 4pt ,line cap=round,line join=round,fill opacity=0.00,] (468.98,572.80) -- (468.98,543.14);

\draw[color=drawColor,line cap=round,line join=round,fill opacity=0.00,] (467.64,486.55) -- (470.33,486.55);

\draw[color=drawColor,line cap=round,line join=round,fill opacity=0.00,] (467.64,572.80) -- (470.33,572.80);

\draw[color=drawColor,line cap=round,line join=round,fill opacity=0.00,] (466.30,508.32) --
	(471.67,508.32) --
	(471.67,543.14) --
	(466.30,543.14) --
	(466.30,508.32);
\end{scope}
\begin{scope}
\path[clip] (  0.00,  0.00) rectangle (505.89,650.43);
\definecolor[named]{drawColor}{rgb}{0.00,0.00,0.00}

\draw[color=drawColor,line cap=round,line join=round,fill opacity=0.00,] ( 52.75,450.25) -- (468.98,450.25);

\draw[color=drawColor,line cap=round,line join=round,fill opacity=0.00,] ( 52.75,450.25) -- ( 52.75,446.29);

\draw[color=drawColor,line cap=round,line join=round,fill opacity=0.00,] ( 59.46,450.25) -- ( 59.46,446.29);

\draw[color=drawColor,line cap=round,line join=round,fill opacity=0.00,] ( 66.17,450.25) -- ( 66.17,446.29);

\draw[color=drawColor,line cap=round,line join=round,fill opacity=0.00,] ( 72.89,450.25) -- ( 72.89,446.29);

\draw[color=drawColor,line cap=round,line join=round,fill opacity=0.00,] ( 79.60,450.25) -- ( 79.60,446.29);

\draw[color=drawColor,line cap=round,line join=round,fill opacity=0.00,] ( 86.31,450.25) -- ( 86.31,446.29);

\draw[color=drawColor,line cap=round,line join=round,fill opacity=0.00,] ( 93.03,450.25) -- ( 93.03,446.29);

\draw[color=drawColor,line cap=round,line join=round,fill opacity=0.00,] ( 99.74,450.25) -- ( 99.74,446.29);

\draw[color=drawColor,line cap=round,line join=round,fill opacity=0.00,] (106.45,450.25) -- (106.45,446.29);

\draw[color=drawColor,line cap=round,line join=round,fill opacity=0.00,] (113.17,450.25) -- (113.17,446.29);

\draw[color=drawColor,line cap=round,line join=round,fill opacity=0.00,] (119.88,450.25) -- (119.88,446.29);

\draw[color=drawColor,line cap=round,line join=round,fill opacity=0.00,] (126.60,450.25) -- (126.60,446.29);

\draw[color=drawColor,line cap=round,line join=round,fill opacity=0.00,] (133.31,450.25) -- (133.31,446.29);

\draw[color=drawColor,line cap=round,line join=round,fill opacity=0.00,] (140.02,450.25) -- (140.02,446.29);

\draw[color=drawColor,line cap=round,line join=round,fill opacity=0.00,] (146.74,450.25) -- (146.74,446.29);

\draw[color=drawColor,line cap=round,line join=round,fill opacity=0.00,] (153.45,450.25) -- (153.45,446.29);

\draw[color=drawColor,line cap=round,line join=round,fill opacity=0.00,] (160.16,450.25) -- (160.16,446.29);

\draw[color=drawColor,line cap=round,line join=round,fill opacity=0.00,] (166.88,450.25) -- (166.88,446.29);

\draw[color=drawColor,line cap=round,line join=round,fill opacity=0.00,] (173.59,450.25) -- (173.59,446.29);

\draw[color=drawColor,line cap=round,line join=round,fill opacity=0.00,] (180.30,450.25) -- (180.30,446.29);

\draw[color=drawColor,line cap=round,line join=round,fill opacity=0.00,] (187.02,450.25) -- (187.02,446.29);

\draw[color=drawColor,line cap=round,line join=round,fill opacity=0.00,] (193.73,450.25) -- (193.73,446.29);

\draw[color=drawColor,line cap=round,line join=round,fill opacity=0.00,] (200.44,450.25) -- (200.44,446.29);

\draw[color=drawColor,line cap=round,line join=round,fill opacity=0.00,] (207.16,450.25) -- (207.16,446.29);

\draw[color=drawColor,line cap=round,line join=round,fill opacity=0.00,] (213.87,450.25) -- (213.87,446.29);

\draw[color=drawColor,line cap=round,line join=round,fill opacity=0.00,] (220.58,450.25) -- (220.58,446.29);

\draw[color=drawColor,line cap=round,line join=round,fill opacity=0.00,] (227.30,450.25) -- (227.30,446.29);

\draw[color=drawColor,line cap=round,line join=round,fill opacity=0.00,] (234.01,450.25) -- (234.01,446.29);

\draw[color=drawColor,line cap=round,line join=round,fill opacity=0.00,] (240.72,450.25) -- (240.72,446.29);

\draw[color=drawColor,line cap=round,line join=round,fill opacity=0.00,] (247.44,450.25) -- (247.44,446.29);

\draw[color=drawColor,line cap=round,line join=round,fill opacity=0.00,] (254.15,450.25) -- (254.15,446.29);

\draw[color=drawColor,line cap=round,line join=round,fill opacity=0.00,] (260.87,450.25) -- (260.87,446.29);

\draw[color=drawColor,line cap=round,line join=round,fill opacity=0.00,] (267.58,450.25) -- (267.58,446.29);

\draw[color=drawColor,line cap=round,line join=round,fill opacity=0.00,] (274.29,450.25) -- (274.29,446.29);

\draw[color=drawColor,line cap=round,line join=round,fill opacity=0.00,] (281.01,450.25) -- (281.01,446.29);

\draw[color=drawColor,line cap=round,line join=round,fill opacity=0.00,] (287.72,450.25) -- (287.72,446.29);

\draw[color=drawColor,line cap=round,line join=round,fill opacity=0.00,] (294.43,450.25) -- (294.43,446.29);

\draw[color=drawColor,line cap=round,line join=round,fill opacity=0.00,] (301.15,450.25) -- (301.15,446.29);

\draw[color=drawColor,line cap=round,line join=round,fill opacity=0.00,] (307.86,450.25) -- (307.86,446.29);

\draw[color=drawColor,line cap=round,line join=round,fill opacity=0.00,] (314.57,450.25) -- (314.57,446.29);

\draw[color=drawColor,line cap=round,line join=round,fill opacity=0.00,] (321.29,450.25) -- (321.29,446.29);

\draw[color=drawColor,line cap=round,line join=round,fill opacity=0.00,] (328.00,450.25) -- (328.00,446.29);

\draw[color=drawColor,line cap=round,line join=round,fill opacity=0.00,] (334.71,450.25) -- (334.71,446.29);

\draw[color=drawColor,line cap=round,line join=round,fill opacity=0.00,] (341.43,450.25) -- (341.43,446.29);

\draw[color=drawColor,line cap=round,line join=round,fill opacity=0.00,] (348.14,450.25) -- (348.14,446.29);

\draw[color=drawColor,line cap=round,line join=round,fill opacity=0.00,] (354.85,450.25) -- (354.85,446.29);

\draw[color=drawColor,line cap=round,line join=round,fill opacity=0.00,] (361.57,450.25) -- (361.57,446.29);

\draw[color=drawColor,line cap=round,line join=round,fill opacity=0.00,] (368.28,450.25) -- (368.28,446.29);

\draw[color=drawColor,line cap=round,line join=round,fill opacity=0.00,] (374.99,450.25) -- (374.99,446.29);

\draw[color=drawColor,line cap=round,line join=round,fill opacity=0.00,] (381.71,450.25) -- (381.71,446.29);

\draw[color=drawColor,line cap=round,line join=round,fill opacity=0.00,] (388.42,450.25) -- (388.42,446.29);

\draw[color=drawColor,line cap=round,line join=round,fill opacity=0.00,] (395.13,450.25) -- (395.13,446.29);

\draw[color=drawColor,line cap=round,line join=round,fill opacity=0.00,] (401.85,450.25) -- (401.85,446.29);

\draw[color=drawColor,line cap=round,line join=round,fill opacity=0.00,] (408.56,450.25) -- (408.56,446.29);

\draw[color=drawColor,line cap=round,line join=round,fill opacity=0.00,] (415.28,450.25) -- (415.28,446.29);

\draw[color=drawColor,line cap=round,line join=round,fill opacity=0.00,] (421.99,450.25) -- (421.99,446.29);

\draw[color=drawColor,line cap=round,line join=round,fill opacity=0.00,] (428.70,450.25) -- (428.70,446.29);

\draw[color=drawColor,line cap=round,line join=round,fill opacity=0.00,] (435.42,450.25) -- (435.42,446.29);

\draw[color=drawColor,line cap=round,line join=round,fill opacity=0.00,] (442.13,450.25) -- (442.13,446.29);

\draw[color=drawColor,line cap=round,line join=round,fill opacity=0.00,] (448.84,450.25) -- (448.84,446.29);

\draw[color=drawColor,line cap=round,line join=round,fill opacity=0.00,] (455.56,450.25) -- (455.56,446.29);

\draw[color=drawColor,line cap=round,line join=round,fill opacity=0.00,] (462.27,450.25) -- (462.27,446.29);

\draw[color=drawColor,line cap=round,line join=round,fill opacity=0.00,] (468.98,450.25) -- (468.98,446.29);

\node[color=drawColor,anchor=base,inner sep=0pt, outer sep=0pt, scale=  0.66] at ( 52.75,434.41) {1949%
};

\node[color=drawColor,anchor=base,inner sep=0pt, outer sep=0pt, scale=  0.66] at ( 79.60,434.41) {1953%
};

\node[color=drawColor,anchor=base,inner sep=0pt, outer sep=0pt, scale=  0.66] at (106.45,434.41) {1957%
};

\node[color=drawColor,anchor=base,inner sep=0pt, outer sep=0pt, scale=  0.66] at (133.31,434.41) {1961%
};

\node[color=drawColor,anchor=base,inner sep=0pt, outer sep=0pt, scale=  0.66] at (160.16,434.41) {1965%
};

\node[color=drawColor,anchor=base,inner sep=0pt, outer sep=0pt, scale=  0.66] at (187.02,434.41) {1969%
};

\node[color=drawColor,anchor=base,inner sep=0pt, outer sep=0pt, scale=  0.66] at (213.87,434.41) {1973%
};

\node[color=drawColor,anchor=base,inner sep=0pt, outer sep=0pt, scale=  0.66] at (240.72,434.41) {1977%
};

\node[color=drawColor,anchor=base,inner sep=0pt, outer sep=0pt, scale=  0.66] at (267.58,434.41) {1981%
};

\node[color=drawColor,anchor=base,inner sep=0pt, outer sep=0pt, scale=  0.66] at (294.43,434.41) {1985%
};

\node[color=drawColor,anchor=base,inner sep=0pt, outer sep=0pt, scale=  0.66] at (321.29,434.41) {1989%
};

\node[color=drawColor,anchor=base,inner sep=0pt, outer sep=0pt, scale=  0.66] at (348.14,434.41) {1993%
};

\node[color=drawColor,anchor=base,inner sep=0pt, outer sep=0pt, scale=  0.66] at (374.99,434.41) {1997%
};

\node[color=drawColor,anchor=base,inner sep=0pt, outer sep=0pt, scale=  0.66] at (401.85,434.41) {2001%
};

\node[color=drawColor,anchor=base,inner sep=0pt, outer sep=0pt, scale=  0.66] at (428.70,434.41) {2005%
};

\node[color=drawColor,anchor=base,inner sep=0pt, outer sep=0pt, scale=  0.66] at (455.56,434.41) {2009%
};

\draw[color=drawColor,line cap=round,line join=round,fill opacity=0.00,] ( 32.47,456.76) -- ( 32.47,619.37);

\draw[color=drawColor,line cap=round,line join=round,fill opacity=0.00,] ( 32.47,456.76) -- ( 28.51,456.76);

\draw[color=drawColor,line cap=round,line join=round,fill opacity=0.00,] ( 32.47,497.41) -- ( 28.51,497.41);

\draw[color=drawColor,line cap=round,line join=round,fill opacity=0.00,] ( 32.47,538.06) -- ( 28.51,538.06);

\draw[color=drawColor,line cap=round,line join=round,fill opacity=0.00,] ( 32.47,578.72) -- ( 28.51,578.72);

\draw[color=drawColor,line cap=round,line join=round,fill opacity=0.00,] ( 32.47,619.37) -- ( 28.51,619.37);

\node[rotate= 90.00,color=drawColor,anchor=base,inner sep=0pt, outer sep=0pt, scale=  0.66] at ( 24.55,456.76) {0%
};

\node[rotate= 90.00,color=drawColor,anchor=base,inner sep=0pt, outer sep=0pt, scale=  0.66] at ( 24.55,497.41) {1000%
};

\node[rotate= 90.00,color=drawColor,anchor=base,inner sep=0pt, outer sep=0pt, scale=  0.66] at ( 24.55,538.06) {2000%
};

\node[rotate= 90.00,color=drawColor,anchor=base,inner sep=0pt, outer sep=0pt, scale=  0.66] at ( 24.55,578.72) {3000%
};

\node[rotate= 90.00,color=drawColor,anchor=base,inner sep=0pt, outer sep=0pt, scale=  0.66] at ( 24.55,619.37) {4000%
};

\draw[color=drawColor,line cap=round,line join=round,fill opacity=0.00,] ( 32.47,450.25) --
	(489.26,450.25) --
	(489.26,625.88) --
	( 32.47,625.88) --
	( 32.47,450.25);
\end{scope}
\begin{scope}
\path[clip] ( 32.47,450.25) rectangle (489.26,625.88);
\definecolor[named]{drawColor}{rgb}{1.00,0.00,0.00}

\draw[color=drawColor,line cap=round,line join=round,fill opacity=0.00,] ( 53.66,533.03) -- ( 58.55,512.29);

\draw[color=drawColor,line cap=round,line join=round,fill opacity=0.00,] ( 62.60,506.02) -- ( 63.04,505.68);

\draw[color=drawColor,line cap=round,line join=round,fill opacity=0.00,] ( 66.64,507.19) -- ( 72.42,555.90);

\draw[color=drawColor,line cap=round,line join=round,fill opacity=0.00,] ( 73.37,555.90) -- ( 79.11,509.47);

\draw[color=drawColor,line cap=round,line join=round,fill opacity=0.00,] ( 80.72,501.75) -- ( 85.20,486.52);

\draw[color=drawColor,line cap=round,line join=round,fill opacity=0.00,] ( 88.07,486.27) -- ( 91.27,492.73);

\draw[color=drawColor,line cap=round,line join=round,fill opacity=0.00,] ( 95.78,499.13) -- ( 96.99,500.38);

\draw[color=drawColor,line cap=round,line join=round,fill opacity=0.00,] (100.11,507.17) -- (106.09,571.58);

\draw[color=drawColor,line cap=round,line join=round,fill opacity=0.00,] (107.27,571.65) -- (112.36,547.30);

\draw[color=drawColor,line cap=round,line join=round,fill opacity=0.00,] (113.68,539.50) -- (119.37,496.27);

\draw[color=drawColor,line cap=round,line join=round,fill opacity=0.00,] (121.34,496.02) -- (125.13,505.58);

\draw[color=drawColor,line cap=round,line join=round,fill opacity=0.00,] (128.29,505.68) -- (131.62,498.64);

\draw[color=drawColor,line cap=round,line join=round,fill opacity=0.00,] (133.91,498.98) -- (139.42,535.06);

\draw[color=drawColor,line cap=round,line join=round,fill opacity=0.00,] (140.53,535.05) -- (146.23,490.59);

\draw[color=drawColor,line cap=round,line join=round,fill opacity=0.00,] (147.91,490.45) -- (152.27,504.46);

\draw[color=drawColor,line cap=round,line join=round,fill opacity=0.00,] (154.10,512.15) -- (159.51,544.76);

\draw[color=drawColor,line cap=round,line join=round,fill opacity=0.00,] (160.72,544.74) -- (166.32,505.38);

\draw[color=drawColor,line cap=round,line join=round,fill opacity=0.00,] (169.89,498.90) -- (170.57,498.32);

\draw[color=drawColor,line cap=round,line join=round,fill opacity=0.00,] (174.87,499.50) -- (179.02,511.64);

\draw[color=drawColor,line cap=round,line join=round,fill opacity=0.00,] (182.53,518.66) -- (184.79,521.99);

\draw[color=drawColor,line cap=round,line join=round,fill opacity=0.00,] (189.51,528.35) -- (191.24,530.49);

\draw[color=drawColor,line cap=round,line join=round,fill opacity=0.00,] (195.14,529.87) -- (199.03,519.66);

\draw[color=drawColor,line cap=round,line join=round,fill opacity=0.00,] (201.63,512.18) -- (205.97,498.36);

\draw[color=drawColor,line cap=round,line join=round,fill opacity=0.00,] (207.75,498.49) -- (213.28,534.84);

\draw[color=drawColor,line cap=round,line join=round,fill opacity=0.00,] (214.79,534.90) -- (219.67,514.45);

\draw[color=drawColor,line cap=round,line join=round,fill opacity=0.00,] (221.80,514.36) -- (226.08,527.68);

\draw[color=drawColor,line cap=round,line join=round,fill opacity=0.00,] (228.08,527.57) -- (233.23,502.13);

\draw[color=drawColor,line cap=round,line join=round,fill opacity=0.00,] (234.92,494.39) -- (239.81,473.73);

\draw[color=drawColor,line cap=round,line join=round,fill opacity=0.00,] (241.15,473.81) -- (247.02,528.57);

\draw[color=drawColor,line cap=round,line join=round,fill opacity=0.00,] (248.97,536.16) -- (252.62,544.88);

\draw[color=drawColor,line cap=round,line join=round,fill opacity=0.00,] (261.30,540.84) -- (267.14,487.95);

\draw[color=drawColor,line cap=round,line join=round,fill opacity=0.00,] (268.18,487.92) -- (273.69,523.58);

\draw[color=drawColor,line cap=round,line join=round,fill opacity=0.00,] (275.01,531.39) -- (280.29,560.24);

\draw[color=drawColor,line cap=round,line join=round,fill opacity=0.00,] (281.93,567.99) -- (286.80,588.30);

\draw[color=drawColor,line cap=round,line join=round,fill opacity=0.00,] (288.64,588.30) -- (293.52,567.79);

\draw[color=drawColor,line cap=round,line join=round,fill opacity=0.00,] (296.19,560.39) -- (299.39,553.89);

\draw[color=drawColor,line cap=round,line join=round,fill opacity=0.00,] (302.51,546.63) -- (306.49,535.83);

\draw[color=drawColor,line cap=round,line join=round,fill opacity=0.00,] (308.60,528.22) -- (313.83,500.67);

\draw[color=drawColor,line cap=round,line join=round,fill opacity=0.00,] (323.86,494.38) -- (325.43,492.54);

\draw[color=drawColor,line cap=round,line join=round,fill opacity=0.00,] (329.06,493.34) -- (333.65,509.82);

\draw[color=drawColor,line cap=round,line join=round,fill opacity=0.00,] (337.94,511.34) -- (338.20,511.16);

\draw[color=drawColor,line cap=round,line join=round,fill opacity=0.00,] (341.93,512.79) -- (347.64,557.18);

\draw[color=drawColor,line cap=round,line join=round,fill opacity=0.00,] (348.61,557.17) -- (354.38,509.23);

\draw[color=drawColor,line cap=round,line join=round,fill opacity=0.00,] (355.24,509.24) -- (361.18,569.38);

\draw[color=drawColor,line cap=round,line join=round,fill opacity=0.00,] (362.07,569.39) -- (367.77,525.13);

\draw[color=drawColor,line cap=round,line join=round,fill opacity=0.00,] (369.11,525.07) -- (374.17,548.71);

\draw[color=drawColor,line cap=round,line join=round,fill opacity=0.00,] (375.77,548.69) -- (380.93,522.75);

\draw[color=drawColor,line cap=round,line join=round,fill opacity=0.00,] (384.05,515.67) -- (386.08,512.89);

\draw[color=drawColor,line cap=round,line join=round,fill opacity=0.00,] (390.90,506.61) -- (392.66,504.42);

\draw[color=drawColor,line cap=round,line join=round,fill opacity=0.00,] (402.89,494.80) -- (407.52,477.79);

\draw[color=drawColor,line cap=round,line join=round,fill opacity=0.00,] (409.72,477.75) -- (414.12,492.18);

\draw[color=drawColor,line cap=round,line join=round,fill opacity=0.00,] (422.79,498.42) -- (427.91,523.33);

\draw[color=drawColor,line cap=round,line join=round,fill opacity=0.00,] (429.93,523.45) -- (434.19,510.35);

\draw[color=drawColor,line cap=round,line join=round,fill opacity=0.00,] (438.69,504.35) -- (438.86,504.24);

\draw[color=drawColor,line cap=round,line join=round,fill opacity=0.00,] ( 52.75,536.89) circle (  0.89);

\draw[color=drawColor,line cap=round,line join=round,fill opacity=0.00,] ( 59.46,508.44) circle (  0.89);

\draw[color=drawColor,line cap=round,line join=round,fill opacity=0.00,] ( 66.17,503.26) circle (  0.89);

\draw[color=drawColor,line cap=round,line join=round,fill opacity=0.00,] ( 72.89,559.83) circle (  0.89);

\draw[color=drawColor,line cap=round,line join=round,fill opacity=0.00,] ( 79.60,505.54) circle (  0.89);

\draw[color=drawColor,line cap=round,line join=round,fill opacity=0.00,] ( 86.31,482.72) circle (  0.89);

\draw[color=drawColor,line cap=round,line join=round,fill opacity=0.00,] ( 93.03,496.28) circle (  0.89);

\draw[color=drawColor,line cap=round,line join=round,fill opacity=0.00,] ( 99.74,503.23) circle (  0.89);

\draw[color=drawColor,line cap=round,line join=round,fill opacity=0.00,] (106.45,575.52) circle (  0.89);

\draw[color=drawColor,line cap=round,line join=round,fill opacity=0.00,] (113.17,543.43) circle (  0.89);

\draw[color=drawColor,line cap=round,line join=round,fill opacity=0.00,] (119.88,492.34) circle (  0.89);

\draw[color=drawColor,line cap=round,line join=round,fill opacity=0.00,] (126.60,509.26) circle (  0.89);

\draw[color=drawColor,line cap=round,line join=round,fill opacity=0.00,] (133.31,495.06) circle (  0.89);

\draw[color=drawColor,line cap=round,line join=round,fill opacity=0.00,] (140.02,538.97) circle (  0.89);

\draw[color=drawColor,line cap=round,line join=round,fill opacity=0.00,] (146.74,486.66) circle (  0.89);

\draw[color=drawColor,line cap=round,line join=round,fill opacity=0.00,] (153.45,508.24) circle (  0.89);

\draw[color=drawColor,line cap=round,line join=round,fill opacity=0.00,] (160.16,548.66) circle (  0.89);

\draw[color=drawColor,line cap=round,line join=round,fill opacity=0.00,] (166.88,501.46) circle (  0.89);

\draw[color=drawColor,line cap=round,line join=round,fill opacity=0.00,] (173.59,495.76) circle (  0.89);

\draw[color=drawColor,line cap=round,line join=round,fill opacity=0.00,] (180.30,515.39) circle (  0.89);

\draw[color=drawColor,line cap=round,line join=round,fill opacity=0.00,] (187.02,525.27) circle (  0.89);

\draw[color=drawColor,line cap=round,line join=round,fill opacity=0.00,] (193.73,533.57) circle (  0.89);

\draw[color=drawColor,line cap=round,line join=round,fill opacity=0.00,] (200.44,515.96) circle (  0.89);

\draw[color=drawColor,line cap=round,line join=round,fill opacity=0.00,] (207.16,494.58) circle (  0.89);

\draw[color=drawColor,line cap=round,line join=round,fill opacity=0.00,] (213.87,538.75) circle (  0.89);

\draw[color=drawColor,line cap=round,line join=round,fill opacity=0.00,] (220.58,510.59) circle (  0.89);

\draw[color=drawColor,line cap=round,line join=round,fill opacity=0.00,] (227.30,531.45) circle (  0.89);

\draw[color=drawColor,line cap=round,line join=round,fill opacity=0.00,] (234.01,498.24) circle (  0.89);

\draw[color=drawColor,line cap=round,line join=round,fill opacity=0.00,] (240.72,469.88) circle (  0.89);

\draw[color=drawColor,line cap=round,line join=round,fill opacity=0.00,] (247.44,532.50) circle (  0.89);

\draw[color=drawColor,line cap=round,line join=round,fill opacity=0.00,] (254.15,548.53) circle (  0.89);

\draw[color=drawColor,line cap=round,line join=round,fill opacity=0.00,] (260.87,544.77) circle (  0.89);

\draw[color=drawColor,line cap=round,line join=round,fill opacity=0.00,] (267.58,484.01) circle (  0.89);

\draw[color=drawColor,line cap=round,line join=round,fill opacity=0.00,] (274.29,527.49) circle (  0.89);

\draw[color=drawColor,line cap=round,line join=round,fill opacity=0.00,] (281.01,564.14) circle (  0.89);

\draw[color=drawColor,line cap=round,line join=round,fill opacity=0.00,] (287.72,592.15) circle (  0.89);

\draw[color=drawColor,line cap=round,line join=round,fill opacity=0.00,] (294.43,563.94) circle (  0.89);

\draw[color=drawColor,line cap=round,line join=round,fill opacity=0.00,] (301.15,550.34) circle (  0.89);

\draw[color=drawColor,line cap=round,line join=round,fill opacity=0.00,] (307.86,532.11) circle (  0.89);

\draw[color=drawColor,line cap=round,line join=round,fill opacity=0.00,] (314.57,496.78) circle (  0.89);

\draw[color=drawColor,line cap=round,line join=round,fill opacity=0.00,] (321.29,497.39) circle (  0.89);

\draw[color=drawColor,line cap=round,line join=round,fill opacity=0.00,] (328.00,489.53) circle (  0.89);

\draw[color=drawColor,line cap=round,line join=round,fill opacity=0.00,] (334.71,513.64) circle (  0.89);

\draw[color=drawColor,line cap=round,line join=round,fill opacity=0.00,] (341.43,508.87) circle (  0.89);

\draw[color=drawColor,line cap=round,line join=round,fill opacity=0.00,] (348.14,561.11) circle (  0.89);

\draw[color=drawColor,line cap=round,line join=round,fill opacity=0.00,] (354.85,505.30) circle (  0.89);

\draw[color=drawColor,line cap=round,line join=round,fill opacity=0.00,] (361.57,573.32) circle (  0.89);

\draw[color=drawColor,line cap=round,line join=round,fill opacity=0.00,] (368.28,521.20) circle (  0.89);

\draw[color=drawColor,line cap=round,line join=round,fill opacity=0.00,] (374.99,552.58) circle (  0.89);

\draw[color=drawColor,line cap=round,line join=round,fill opacity=0.00,] (381.71,518.86) circle (  0.89);

\draw[color=drawColor,line cap=round,line join=round,fill opacity=0.00,] (388.42,509.70) circle (  0.89);

\draw[color=drawColor,line cap=round,line join=round,fill opacity=0.00,] (395.13,501.33) circle (  0.89);

\draw[color=drawColor,line cap=round,line join=round,fill opacity=0.00,] (401.85,498.63) circle (  0.89);

\draw[color=drawColor,line cap=round,line join=round,fill opacity=0.00,] (408.56,473.96) circle (  0.89);

\draw[color=drawColor,line cap=round,line join=round,fill opacity=0.00,] (415.28,495.97) circle (  0.89);

\draw[color=drawColor,line cap=round,line join=round,fill opacity=0.00,] (421.99,494.54) circle (  0.89);

\draw[color=drawColor,line cap=round,line join=round,fill opacity=0.00,] (428.70,527.21) circle (  0.89);

\draw[color=drawColor,line cap=round,line join=round,fill opacity=0.00,] (435.42,506.59) circle (  0.89);

\draw[color=drawColor,line cap=round,line join=round,fill opacity=0.00,] (442.13,502.00) circle (  0.89);
\end{scope}
\begin{scope}
\path[clip] (  0.00,  0.00) rectangle (505.89,650.43);
\definecolor[named]{drawColor}{rgb}{0.00,0.00,0.00}

\node[color=drawColor,anchor=base,inner sep=0pt, outer sep=0pt, scale=  1.00] at (260.87,633.80) {(a) RPSS = 0.81 MC = 0.81%
};
\end{scope}
\begin{scope}
\path[clip] ( 32.47,233.44) rectangle (489.26,409.07);
\end{scope}
\begin{scope}
\path[clip] ( 32.47,233.44) rectangle (489.26,409.07);
\definecolor[named]{drawColor}{rgb}{0.00,0.00,0.00}

\draw[color=drawColor,line width= 1.2pt,line join=round,fill opacity=0.00,] ( 50.06,319.91) -- ( 55.43,319.91);

\draw[color=drawColor,dash pattern=on 4pt off 4pt ,line cap=round,line join=round,fill opacity=0.00,] ( 52.75,300.64) -- ( 52.75,315.25);

\draw[color=drawColor,dash pattern=on 4pt off 4pt ,line cap=round,line join=round,fill opacity=0.00,] ( 52.75,339.89) -- ( 52.75,325.43);

\draw[color=drawColor,line cap=round,line join=round,fill opacity=0.00,] ( 51.40,300.64) -- ( 54.09,300.64);

\draw[color=drawColor,line cap=round,line join=round,fill opacity=0.00,] ( 51.40,339.89) -- ( 54.09,339.89);

\draw[color=drawColor,line cap=round,line join=round,fill opacity=0.00,] ( 50.06,315.25) --
	( 55.43,315.25) --
	( 55.43,325.43) --
	( 50.06,325.43) --
	( 50.06,315.25);

\draw[color=drawColor,line width= 1.2pt,line join=round,fill opacity=0.00,] ( 56.77,294.07) -- ( 62.15,294.07);

\draw[color=drawColor,dash pattern=on 4pt off 4pt ,line cap=round,line join=round,fill opacity=0.00,] ( 59.46,276.19) -- ( 59.46,286.95);

\draw[color=drawColor,dash pattern=on 4pt off 4pt ,line cap=round,line join=round,fill opacity=0.00,] ( 59.46,338.70) -- ( 59.46,309.29);

\draw[color=drawColor,line cap=round,line join=round,fill opacity=0.00,] ( 58.12,276.19) -- ( 60.80,276.19);

\draw[color=drawColor,line cap=round,line join=round,fill opacity=0.00,] ( 58.12,338.70) -- ( 60.80,338.70);

\draw[color=drawColor,line cap=round,line join=round,fill opacity=0.00,] ( 56.77,286.95) --
	( 62.15,286.95) --
	( 62.15,309.29) --
	( 56.77,309.29) --
	( 56.77,286.95);

\draw[color=drawColor,line width= 1.2pt,line join=round,fill opacity=0.00,] ( 63.49,286.64) -- ( 68.86,286.64);

\draw[color=drawColor,dash pattern=on 4pt off 4pt ,line cap=round,line join=round,fill opacity=0.00,] ( 66.17,274.20) -- ( 66.17,281.84);

\draw[color=drawColor,dash pattern=on 4pt off 4pt ,line cap=round,line join=round,fill opacity=0.00,] ( 66.17,304.12) -- ( 66.17,291.39);

\draw[color=drawColor,line cap=round,line join=round,fill opacity=0.00,] ( 64.83,274.20) -- ( 67.52,274.20);

\draw[color=drawColor,line cap=round,line join=round,fill opacity=0.00,] ( 64.83,304.12) -- ( 67.52,304.12);

\draw[color=drawColor,line cap=round,line join=round,fill opacity=0.00,] ( 63.49,281.84) --
	( 68.86,281.84) --
	( 68.86,291.39) --
	( 63.49,291.39) --
	( 63.49,281.84);

\draw[color=drawColor,line width= 1.2pt,line join=round,fill opacity=0.00,] ( 70.20,344.76) -- ( 75.57,344.76);

\draw[color=drawColor,dash pattern=on 4pt off 4pt ,line cap=round,line join=round,fill opacity=0.00,] ( 72.89,307.33) -- ( 72.89,334.93);

\draw[color=drawColor,dash pattern=on 4pt off 4pt ,line cap=round,line join=round,fill opacity=0.00,] ( 72.89,381.55) -- ( 72.89,353.58);

\draw[color=drawColor,line cap=round,line join=round,fill opacity=0.00,] ( 71.54,307.33) -- ( 74.23,307.33);

\draw[color=drawColor,line cap=round,line join=round,fill opacity=0.00,] ( 71.54,381.55) -- ( 74.23,381.55);

\draw[color=drawColor,line cap=round,line join=round,fill opacity=0.00,] ( 70.20,334.93) --
	( 75.57,334.93) --
	( 75.57,353.58) --
	( 70.20,353.58) --
	( 70.20,334.93);

\draw[color=drawColor,line width= 1.2pt,line join=round,fill opacity=0.00,] ( 76.92,288.06) -- ( 82.29,288.06);

\draw[color=drawColor,dash pattern=on 4pt off 4pt ,line cap=round,line join=round,fill opacity=0.00,] ( 79.60,274.02) -- ( 79.60,281.61);

\draw[color=drawColor,dash pattern=on 4pt off 4pt ,line cap=round,line join=round,fill opacity=0.00,] ( 79.60,304.19) -- ( 79.60,292.20);

\draw[color=drawColor,line cap=round,line join=round,fill opacity=0.00,] ( 78.26,274.02) -- ( 80.94,274.02);

\draw[color=drawColor,line cap=round,line join=round,fill opacity=0.00,] ( 78.26,304.19) -- ( 80.94,304.19);

\draw[color=drawColor,line cap=round,line join=round,fill opacity=0.00,] ( 76.92,281.61) --
	( 82.29,281.61) --
	( 82.29,292.20) --
	( 76.92,292.20) --
	( 76.92,281.61);

\draw[color=drawColor,line width= 1.2pt,line join=round,fill opacity=0.00,] ( 83.63,283.70) -- ( 89.00,283.70);

\draw[color=drawColor,dash pattern=on 4pt off 4pt ,line cap=round,line join=round,fill opacity=0.00,] ( 86.31,266.07) -- ( 86.31,279.41);

\draw[color=drawColor,dash pattern=on 4pt off 4pt ,line cap=round,line join=round,fill opacity=0.00,] ( 86.31,298.69) -- ( 86.31,288.85);

\draw[color=drawColor,line cap=round,line join=round,fill opacity=0.00,] ( 84.97,266.07) -- ( 87.66,266.07);

\draw[color=drawColor,line cap=round,line join=round,fill opacity=0.00,] ( 84.97,298.69) -- ( 87.66,298.69);

\draw[color=drawColor,line cap=round,line join=round,fill opacity=0.00,] ( 83.63,279.41) --
	( 89.00,279.41) --
	( 89.00,288.85) --
	( 83.63,288.85) --
	( 83.63,279.41);

\draw[color=drawColor,line width= 1.2pt,line join=round,fill opacity=0.00,] ( 90.34,280.85) -- ( 95.71,280.85);

\draw[color=drawColor,dash pattern=on 4pt off 4pt ,line cap=round,line join=round,fill opacity=0.00,] ( 93.03,265.74) -- ( 93.03,277.73);

\draw[color=drawColor,dash pattern=on 4pt off 4pt ,line cap=round,line join=round,fill opacity=0.00,] ( 93.03,296.44) -- ( 93.03,285.73);

\draw[color=drawColor,line cap=round,line join=round,fill opacity=0.00,] ( 91.69,265.74) -- ( 94.37,265.74);

\draw[color=drawColor,line cap=round,line join=round,fill opacity=0.00,] ( 91.69,296.44) -- ( 94.37,296.44);

\draw[color=drawColor,line cap=round,line join=round,fill opacity=0.00,] ( 90.34,277.73) --
	( 95.71,277.73) --
	( 95.71,285.73) --
	( 90.34,285.73) --
	( 90.34,277.73);

\draw[color=drawColor,line width= 1.2pt,line join=round,fill opacity=0.00,] ( 97.06,313.13) -- (102.43,313.13);

\draw[color=drawColor,dash pattern=on 4pt off 4pt ,line cap=round,line join=round,fill opacity=0.00,] ( 99.74,277.88) -- ( 99.74,297.92);

\draw[color=drawColor,dash pattern=on 4pt off 4pt ,line cap=round,line join=round,fill opacity=0.00,] ( 99.74,359.09) -- ( 99.74,326.47);

\draw[color=drawColor,line cap=round,line join=round,fill opacity=0.00,] ( 98.40,277.88) -- (101.08,277.88);

\draw[color=drawColor,line cap=round,line join=round,fill opacity=0.00,] ( 98.40,359.09) -- (101.08,359.09);

\draw[color=drawColor,line cap=round,line join=round,fill opacity=0.00,] ( 97.06,297.92) --
	(102.43,297.92) --
	(102.43,326.47) --
	( 97.06,326.47) --
	( 97.06,297.92);

\draw[color=drawColor,line width= 1.2pt,line join=round,fill opacity=0.00,] (103.77,311.75) -- (109.14,311.75);

\draw[color=drawColor,dash pattern=on 4pt off 4pt ,line cap=round,line join=round,fill opacity=0.00,] (106.45,277.22) -- (106.45,295.92);

\draw[color=drawColor,dash pattern=on 4pt off 4pt ,line cap=round,line join=round,fill opacity=0.00,] (106.45,337.13) -- (106.45,317.33);

\draw[color=drawColor,line cap=round,line join=round,fill opacity=0.00,] (105.11,277.22) -- (107.80,277.22);

\draw[color=drawColor,line cap=round,line join=round,fill opacity=0.00,] (105.11,337.13) -- (107.80,337.13);

\draw[color=drawColor,line cap=round,line join=round,fill opacity=0.00,] (103.77,295.92) --
	(109.14,295.92) --
	(109.14,317.33) --
	(103.77,317.33) --
	(103.77,295.92);

\draw[color=drawColor,line width= 1.2pt,line join=round,fill opacity=0.00,] (110.48,321.37) -- (115.85,321.37);

\draw[color=drawColor,dash pattern=on 4pt off 4pt ,line cap=round,line join=round,fill opacity=0.00,] (113.17,291.84) -- (113.17,312.22);

\draw[color=drawColor,dash pattern=on 4pt off 4pt ,line cap=round,line join=round,fill opacity=0.00,] (113.17,353.21) -- (113.17,328.96);

\draw[color=drawColor,line cap=round,line join=round,fill opacity=0.00,] (111.83,291.84) -- (114.51,291.84);

\draw[color=drawColor,line cap=round,line join=round,fill opacity=0.00,] (111.83,353.21) -- (114.51,353.21);

\draw[color=drawColor,line cap=round,line join=round,fill opacity=0.00,] (110.48,312.22) --
	(115.85,312.22) --
	(115.85,328.96) --
	(110.48,328.96) --
	(110.48,312.22);

\draw[color=drawColor,line width= 1.2pt,line join=round,fill opacity=0.00,] (117.20,280.82) -- (122.57,280.82);

\draw[color=drawColor,dash pattern=on 4pt off 4pt ,line cap=round,line join=round,fill opacity=0.00,] (119.88,267.67) -- (119.88,277.82);

\draw[color=drawColor,dash pattern=on 4pt off 4pt ,line cap=round,line join=round,fill opacity=0.00,] (119.88,295.50) -- (119.88,285.12);

\draw[color=drawColor,line cap=round,line join=round,fill opacity=0.00,] (118.54,267.67) -- (121.22,267.67);

\draw[color=drawColor,line cap=round,line join=round,fill opacity=0.00,] (118.54,295.50) -- (121.22,295.50);

\draw[color=drawColor,line cap=round,line join=round,fill opacity=0.00,] (117.20,277.82) --
	(122.57,277.82) --
	(122.57,285.12) --
	(117.20,285.12) --
	(117.20,277.82);

\draw[color=drawColor,line width= 1.2pt,line join=round,fill opacity=0.00,] (123.91,285.48) -- (129.28,285.48);

\draw[color=drawColor,dash pattern=on 4pt off 4pt ,line cap=round,line join=round,fill opacity=0.00,] (126.60,270.05) -- (126.60,280.77);

\draw[color=drawColor,dash pattern=on 4pt off 4pt ,line cap=round,line join=round,fill opacity=0.00,] (126.60,301.89) -- (126.60,289.58);

\draw[color=drawColor,line cap=round,line join=round,fill opacity=0.00,] (125.25,270.05) -- (127.94,270.05);

\draw[color=drawColor,line cap=round,line join=round,fill opacity=0.00,] (125.25,301.89) -- (127.94,301.89);

\draw[color=drawColor,line cap=round,line join=round,fill opacity=0.00,] (123.91,280.77) --
	(129.28,280.77) --
	(129.28,289.58) --
	(123.91,289.58) --
	(123.91,280.77);

\draw[color=drawColor,line width= 1.2pt,line join=round,fill opacity=0.00,] (130.62,280.98) -- (135.99,280.98);

\draw[color=drawColor,dash pattern=on 4pt off 4pt ,line cap=round,line join=round,fill opacity=0.00,] (133.31,267.84) -- (133.31,278.26);

\draw[color=drawColor,dash pattern=on 4pt off 4pt ,line cap=round,line join=round,fill opacity=0.00,] (133.31,295.63) -- (133.31,285.28);

\draw[color=drawColor,line cap=round,line join=round,fill opacity=0.00,] (131.97,267.84) -- (134.65,267.84);

\draw[color=drawColor,line cap=round,line join=round,fill opacity=0.00,] (131.97,295.63) -- (134.65,295.63);

\draw[color=drawColor,line cap=round,line join=round,fill opacity=0.00,] (130.62,278.26) --
	(135.99,278.26) --
	(135.99,285.28) --
	(130.62,285.28) --
	(130.62,278.26);

\draw[color=drawColor,line width= 1.2pt,line join=round,fill opacity=0.00,] (137.34,319.71) -- (142.71,319.71);

\draw[color=drawColor,dash pattern=on 4pt off 4pt ,line cap=round,line join=round,fill opacity=0.00,] (140.02,298.87) -- (140.02,314.03);

\draw[color=drawColor,dash pattern=on 4pt off 4pt ,line cap=round,line join=round,fill opacity=0.00,] (140.02,339.26) -- (140.02,324.16);

\draw[color=drawColor,line cap=round,line join=round,fill opacity=0.00,] (138.68,298.87) -- (141.36,298.87);

\draw[color=drawColor,line cap=round,line join=round,fill opacity=0.00,] (138.68,339.26) -- (141.36,339.26);

\draw[color=drawColor,line cap=round,line join=round,fill opacity=0.00,] (137.34,314.03) --
	(142.71,314.03) --
	(142.71,324.16) --
	(137.34,324.16) --
	(137.34,314.03);

\draw[color=drawColor,line width= 1.2pt,line join=round,fill opacity=0.00,] (144.05,278.16) -- (149.42,278.16);

\draw[color=drawColor,dash pattern=on 4pt off 4pt ,line cap=round,line join=round,fill opacity=0.00,] (146.74,260.46) -- (146.74,273.03);

\draw[color=drawColor,dash pattern=on 4pt off 4pt ,line cap=round,line join=round,fill opacity=0.00,] (146.74,294.84) -- (146.74,281.96);

\draw[color=drawColor,line cap=round,line join=round,fill opacity=0.00,] (145.39,260.46) -- (148.08,260.46);

\draw[color=drawColor,line cap=round,line join=round,fill opacity=0.00,] (145.39,294.84) -- (148.08,294.84);

\draw[color=drawColor,line cap=round,line join=round,fill opacity=0.00,] (144.05,273.03) --
	(149.42,273.03) --
	(149.42,281.96) --
	(144.05,281.96) --
	(144.05,273.03);

\draw[color=drawColor,line width= 1.2pt,line join=round,fill opacity=0.00,] (150.76,278.65) -- (156.13,278.65);

\draw[color=drawColor,dash pattern=on 4pt off 4pt ,line cap=round,line join=round,fill opacity=0.00,] (153.45,261.55) -- (153.45,273.52);

\draw[color=drawColor,dash pattern=on 4pt off 4pt ,line cap=round,line join=round,fill opacity=0.00,] (153.45,296.76) -- (153.45,282.82);

\draw[color=drawColor,line cap=round,line join=round,fill opacity=0.00,] (152.11,261.55) -- (154.79,261.55);

\draw[color=drawColor,line cap=round,line join=round,fill opacity=0.00,] (152.11,296.76) -- (154.79,296.76);

\draw[color=drawColor,line cap=round,line join=round,fill opacity=0.00,] (150.76,273.52) --
	(156.13,273.52) --
	(156.13,282.82) --
	(150.76,282.82) --
	(150.76,273.52);

\draw[color=drawColor,line width= 1.2pt,line join=round,fill opacity=0.00,] (157.48,329.90) -- (162.85,329.90);

\draw[color=drawColor,dash pattern=on 4pt off 4pt ,line cap=round,line join=round,fill opacity=0.00,] (160.16,292.65) -- (160.16,319.12);

\draw[color=drawColor,dash pattern=on 4pt off 4pt ,line cap=round,line join=round,fill opacity=0.00,] (160.16,369.56) -- (160.16,343.54);

\draw[color=drawColor,line cap=round,line join=round,fill opacity=0.00,] (158.82,292.65) -- (161.51,292.65);

\draw[color=drawColor,line cap=round,line join=round,fill opacity=0.00,] (158.82,369.56) -- (161.51,369.56);

\draw[color=drawColor,line cap=round,line join=round,fill opacity=0.00,] (157.48,319.12) --
	(162.85,319.12) --
	(162.85,343.54) --
	(157.48,343.54) --
	(157.48,319.12);

\draw[color=drawColor,line width= 1.2pt,line join=round,fill opacity=0.00,] (164.19,314.99) -- (169.56,314.99);

\draw[color=drawColor,dash pattern=on 4pt off 4pt ,line cap=round,line join=round,fill opacity=0.00,] (166.88,293.01) -- (166.88,309.87);

\draw[color=drawColor,dash pattern=on 4pt off 4pt ,line cap=round,line join=round,fill opacity=0.00,] (166.88,357.49) -- (166.88,329.03);

\draw[color=drawColor,line cap=round,line join=round,fill opacity=0.00,] (165.53,293.01) -- (168.22,293.01);

\draw[color=drawColor,line cap=round,line join=round,fill opacity=0.00,] (165.53,357.49) -- (168.22,357.49);

\draw[color=drawColor,line cap=round,line join=round,fill opacity=0.00,] (164.19,309.87) --
	(169.56,309.87) --
	(169.56,329.03) --
	(164.19,329.03) --
	(164.19,309.87);

\draw[color=drawColor,line width= 1.2pt,line join=round,fill opacity=0.00,] (170.90,307.79) -- (176.28,307.79);

\draw[color=drawColor,dash pattern=on 4pt off 4pt ,line cap=round,line join=round,fill opacity=0.00,] (173.59,290.19) -- (173.59,300.85);

\draw[color=drawColor,dash pattern=on 4pt off 4pt ,line cap=round,line join=round,fill opacity=0.00,] (173.59,330.19) -- (173.59,315.13);

\draw[color=drawColor,line cap=round,line join=round,fill opacity=0.00,] (172.25,290.19) -- (174.93,290.19);

\draw[color=drawColor,line cap=round,line join=round,fill opacity=0.00,] (172.25,330.19) -- (174.93,330.19);

\draw[color=drawColor,line cap=round,line join=round,fill opacity=0.00,] (170.90,300.85) --
	(176.28,300.85) --
	(176.28,315.13) --
	(170.90,315.13) --
	(170.90,300.85);

\draw[color=drawColor,line width= 1.2pt,line join=round,fill opacity=0.00,] (177.62,297.96) -- (182.99,297.96);

\draw[color=drawColor,dash pattern=on 4pt off 4pt ,line cap=round,line join=round,fill opacity=0.00,] (180.30,277.47) -- (180.30,291.71);

\draw[color=drawColor,dash pattern=on 4pt off 4pt ,line cap=round,line join=round,fill opacity=0.00,] (180.30,326.34) -- (180.30,307.97);

\draw[color=drawColor,line cap=round,line join=round,fill opacity=0.00,] (178.96,277.47) -- (181.65,277.47);

\draw[color=drawColor,line cap=round,line join=round,fill opacity=0.00,] (178.96,326.34) -- (181.65,326.34);

\draw[color=drawColor,line cap=round,line join=round,fill opacity=0.00,] (177.62,291.71) --
	(182.99,291.71) --
	(182.99,307.97) --
	(177.62,307.97) --
	(177.62,291.71);

\draw[color=drawColor,line width= 1.2pt,line join=round,fill opacity=0.00,] (184.33,321.33) -- (189.70,321.33);

\draw[color=drawColor,dash pattern=on 4pt off 4pt ,line cap=round,line join=round,fill opacity=0.00,] (187.02,296.95) -- (187.02,315.62);

\draw[color=drawColor,dash pattern=on 4pt off 4pt ,line cap=round,line join=round,fill opacity=0.00,] (187.02,347.95) -- (187.02,328.63);

\draw[color=drawColor,line cap=round,line join=round,fill opacity=0.00,] (185.67,296.95) -- (188.36,296.95);

\draw[color=drawColor,line cap=round,line join=round,fill opacity=0.00,] (185.67,347.95) -- (188.36,347.95);

\draw[color=drawColor,line cap=round,line join=round,fill opacity=0.00,] (184.33,315.62) --
	(189.70,315.62) --
	(189.70,328.63) --
	(184.33,328.63) --
	(184.33,315.62);

\draw[color=drawColor,line width= 1.2pt,line join=round,fill opacity=0.00,] (191.04,311.36) -- (196.42,311.36);

\draw[color=drawColor,dash pattern=on 4pt off 4pt ,line cap=round,line join=round,fill opacity=0.00,] (193.73,289.45) -- (193.73,304.56);

\draw[color=drawColor,dash pattern=on 4pt off 4pt ,line cap=round,line join=round,fill opacity=0.00,] (193.73,332.16) -- (193.73,316.58);

\draw[color=drawColor,line cap=round,line join=round,fill opacity=0.00,] (192.39,289.45) -- (195.07,289.45);

\draw[color=drawColor,line cap=round,line join=round,fill opacity=0.00,] (192.39,332.16) -- (195.07,332.16);

\draw[color=drawColor,line cap=round,line join=round,fill opacity=0.00,] (191.04,304.56) --
	(196.42,304.56) --
	(196.42,316.58) --
	(191.04,316.58) --
	(191.04,304.56);

\draw[color=drawColor,line width= 1.2pt,line join=round,fill opacity=0.00,] (197.76,313.34) -- (203.13,313.34);

\draw[color=drawColor,dash pattern=on 4pt off 4pt ,line cap=round,line join=round,fill opacity=0.00,] (200.44,291.91) -- (200.44,308.35);

\draw[color=drawColor,dash pattern=on 4pt off 4pt ,line cap=round,line join=round,fill opacity=0.00,] (200.44,335.51) -- (200.44,319.36);

\draw[color=drawColor,line cap=round,line join=round,fill opacity=0.00,] (199.10,291.91) -- (201.79,291.91);

\draw[color=drawColor,line cap=round,line join=round,fill opacity=0.00,] (199.10,335.51) -- (201.79,335.51);

\draw[color=drawColor,line cap=round,line join=round,fill opacity=0.00,] (197.76,308.35) --
	(203.13,308.35) --
	(203.13,319.36) --
	(197.76,319.36) --
	(197.76,308.35);

\draw[color=drawColor,line width= 1.2pt,line join=round,fill opacity=0.00,] (204.47,325.46) -- (209.84,325.46);

\draw[color=drawColor,dash pattern=on 4pt off 4pt ,line cap=round,line join=round,fill opacity=0.00,] (207.16,290.80) -- (207.16,315.00);

\draw[color=drawColor,dash pattern=on 4pt off 4pt ,line cap=round,line join=round,fill opacity=0.00,] (207.16,357.32) -- (207.16,332.84);

\draw[color=drawColor,line cap=round,line join=round,fill opacity=0.00,] (205.81,290.80) -- (208.50,290.80);

\draw[color=drawColor,line cap=round,line join=round,fill opacity=0.00,] (205.81,357.32) -- (208.50,357.32);

\draw[color=drawColor,line cap=round,line join=round,fill opacity=0.00,] (204.47,315.00) --
	(209.84,315.00) --
	(209.84,332.84) --
	(204.47,332.84) --
	(204.47,315.00);

\draw[color=drawColor,line width= 1.2pt,line join=round,fill opacity=0.00,] (211.19,321.35) -- (216.56,321.35);

\draw[color=drawColor,dash pattern=on 4pt off 4pt ,line cap=round,line join=round,fill opacity=0.00,] (213.87,297.04) -- (213.87,315.83);

\draw[color=drawColor,dash pattern=on 4pt off 4pt ,line cap=round,line join=round,fill opacity=0.00,] (213.87,348.27) -- (213.87,328.85);

\draw[color=drawColor,line cap=round,line join=round,fill opacity=0.00,] (212.53,297.04) -- (215.21,297.04);

\draw[color=drawColor,line cap=round,line join=round,fill opacity=0.00,] (212.53,348.27) -- (215.21,348.27);

\draw[color=drawColor,line cap=round,line join=round,fill opacity=0.00,] (211.19,315.83) --
	(216.56,315.83) --
	(216.56,328.85) --
	(211.19,328.85) --
	(211.19,315.83);

\draw[color=drawColor,line width= 1.2pt,line join=round,fill opacity=0.00,] (217.90,300.29) -- (223.27,300.29);

\draw[color=drawColor,dash pattern=on 4pt off 4pt ,line cap=round,line join=round,fill opacity=0.00,] (220.58,276.85) -- (220.58,291.37);

\draw[color=drawColor,dash pattern=on 4pt off 4pt ,line cap=round,line join=round,fill opacity=0.00,] (220.58,336.66) -- (220.58,310.57);

\draw[color=drawColor,line cap=round,line join=round,fill opacity=0.00,] (219.24,276.85) -- (221.93,276.85);

\draw[color=drawColor,line cap=round,line join=round,fill opacity=0.00,] (219.24,336.66) -- (221.93,336.66);

\draw[color=drawColor,line cap=round,line join=round,fill opacity=0.00,] (217.90,291.37) --
	(223.27,291.37) --
	(223.27,310.57) --
	(217.90,310.57) --
	(217.90,291.37);

\draw[color=drawColor,line width= 1.2pt,line join=round,fill opacity=0.00,] (224.61,295.33) -- (229.98,295.33);

\draw[color=drawColor,dash pattern=on 4pt off 4pt ,line cap=round,line join=round,fill opacity=0.00,] (227.30,274.55) -- (227.30,289.42);

\draw[color=drawColor,dash pattern=on 4pt off 4pt ,line cap=round,line join=round,fill opacity=0.00,] (227.30,322.08) -- (227.30,305.34);

\draw[color=drawColor,line cap=round,line join=round,fill opacity=0.00,] (225.95,274.55) -- (228.64,274.55);

\draw[color=drawColor,line cap=round,line join=round,fill opacity=0.00,] (225.95,322.08) -- (228.64,322.08);

\draw[color=drawColor,line cap=round,line join=round,fill opacity=0.00,] (224.61,289.42) --
	(229.98,289.42) --
	(229.98,305.34) --
	(224.61,305.34) --
	(224.61,289.42);

\draw[color=drawColor,line width= 1.2pt,line join=round,fill opacity=0.00,] (231.33,286.81) -- (236.70,286.81);

\draw[color=drawColor,dash pattern=on 4pt off 4pt ,line cap=round,line join=round,fill opacity=0.00,] (234.01,274.84) -- (234.01,281.18);

\draw[color=drawColor,dash pattern=on 4pt off 4pt ,line cap=round,line join=round,fill opacity=0.00,] (234.01,307.66) -- (234.01,292.39);

\draw[color=drawColor,line cap=round,line join=round,fill opacity=0.00,] (232.67,274.84) -- (235.35,274.84);

\draw[color=drawColor,line cap=round,line join=round,fill opacity=0.00,] (232.67,307.66) -- (235.35,307.66);

\draw[color=drawColor,line cap=round,line join=round,fill opacity=0.00,] (231.33,281.18) --
	(236.70,281.18) --
	(236.70,292.39) --
	(231.33,292.39) --
	(231.33,281.18);

\draw[color=drawColor,line width= 1.2pt,line join=round,fill opacity=0.00,] (238.04,263.45) -- (243.41,263.45);

\draw[color=drawColor,dash pattern=on 4pt off 4pt ,line cap=round,line join=round,fill opacity=0.00,] (240.72,243.73) -- (240.72,258.40);

\draw[color=drawColor,dash pattern=on 4pt off 4pt ,line cap=round,line join=round,fill opacity=0.00,] (240.72,293.10) -- (240.72,272.32);

\draw[color=drawColor,line cap=round,line join=round,fill opacity=0.00,] (239.38,243.73) -- (242.07,243.73);

\draw[color=drawColor,line cap=round,line join=round,fill opacity=0.00,] (239.38,293.10) -- (242.07,293.10);

\draw[color=drawColor,line cap=round,line join=round,fill opacity=0.00,] (238.04,258.40) --
	(243.41,258.40) --
	(243.41,272.32) --
	(238.04,272.32) --
	(238.04,258.40);

\draw[color=drawColor,line width= 1.2pt,line join=round,fill opacity=0.00,] (244.75,309.59) -- (250.12,309.59);

\draw[color=drawColor,dash pattern=on 4pt off 4pt ,line cap=round,line join=round,fill opacity=0.00,] (247.44,275.27) -- (247.44,295.43);

\draw[color=drawColor,dash pattern=on 4pt off 4pt ,line cap=round,line join=round,fill opacity=0.00,] (247.44,331.67) -- (247.44,314.31);

\draw[color=drawColor,line cap=round,line join=round,fill opacity=0.00,] (246.10,275.27) -- (248.78,275.27);

\draw[color=drawColor,line cap=round,line join=round,fill opacity=0.00,] (246.10,331.67) -- (248.78,331.67);

\draw[color=drawColor,line cap=round,line join=round,fill opacity=0.00,] (244.75,295.43) --
	(250.12,295.43) --
	(250.12,314.31) --
	(244.75,314.31) --
	(244.75,295.43);

\draw[color=drawColor,line width= 1.2pt,line join=round,fill opacity=0.00,] (251.47,312.67) -- (256.84,312.67);

\draw[color=drawColor,dash pattern=on 4pt off 4pt ,line cap=round,line join=round,fill opacity=0.00,] (254.15,298.57) -- (254.15,309.66);

\draw[color=drawColor,dash pattern=on 4pt off 4pt ,line cap=round,line join=round,fill opacity=0.00,] (254.15,328.51) -- (254.15,317.21);

\draw[color=drawColor,line cap=round,line join=round,fill opacity=0.00,] (252.81,298.57) -- (255.49,298.57);

\draw[color=drawColor,line cap=round,line join=round,fill opacity=0.00,] (252.81,328.51) -- (255.49,328.51);

\draw[color=drawColor,line cap=round,line join=round,fill opacity=0.00,] (251.47,309.66) --
	(256.84,309.66) --
	(256.84,317.21) --
	(251.47,317.21) --
	(251.47,309.66);

\draw[color=drawColor,line width= 1.2pt,line join=round,fill opacity=0.00,] (258.18,308.79) -- (263.55,308.79);

\draw[color=drawColor,dash pattern=on 4pt off 4pt ,line cap=round,line join=round,fill opacity=0.00,] (260.87,283.08) -- (260.87,300.49);

\draw[color=drawColor,dash pattern=on 4pt off 4pt ,line cap=round,line join=round,fill opacity=0.00,] (260.87,340.70) -- (260.87,316.93);

\draw[color=drawColor,line cap=round,line join=round,fill opacity=0.00,] (259.52,283.08) -- (262.21,283.08);

\draw[color=drawColor,line cap=round,line join=round,fill opacity=0.00,] (259.52,340.70) -- (262.21,340.70);

\draw[color=drawColor,line cap=round,line join=round,fill opacity=0.00,] (258.18,300.49) --
	(263.55,300.49) --
	(263.55,316.93) --
	(258.18,316.93) --
	(258.18,300.49);

\draw[color=drawColor,line width= 1.2pt,line join=round,fill opacity=0.00,] (264.89,276.05) -- (270.26,276.05);

\draw[color=drawColor,dash pattern=on 4pt off 4pt ,line cap=round,line join=round,fill opacity=0.00,] (267.58,258.56) -- (267.58,268.60);

\draw[color=drawColor,dash pattern=on 4pt off 4pt ,line cap=round,line join=round,fill opacity=0.00,] (267.58,298.38) -- (267.58,281.85);

\draw[color=drawColor,line cap=round,line join=round,fill opacity=0.00,] (266.24,258.56) -- (268.92,258.56);

\draw[color=drawColor,line cap=round,line join=round,fill opacity=0.00,] (266.24,298.38) -- (268.92,298.38);

\draw[color=drawColor,line cap=round,line join=round,fill opacity=0.00,] (264.89,268.60) --
	(270.26,268.60) --
	(270.26,281.85) --
	(264.89,281.85) --
	(264.89,268.60);

\draw[color=drawColor,line width= 1.2pt,line join=round,fill opacity=0.00,] (271.61,315.59) -- (276.98,315.59);

\draw[color=drawColor,dash pattern=on 4pt off 4pt ,line cap=round,line join=round,fill opacity=0.00,] (274.29,293.07) -- (274.29,310.22);

\draw[color=drawColor,dash pattern=on 4pt off 4pt ,line cap=round,line join=round,fill opacity=0.00,] (274.29,339.51) -- (274.29,321.97);

\draw[color=drawColor,line cap=round,line join=round,fill opacity=0.00,] (272.95,293.07) -- (275.63,293.07);

\draw[color=drawColor,line cap=round,line join=round,fill opacity=0.00,] (272.95,339.51) -- (275.63,339.51);

\draw[color=drawColor,line cap=round,line join=round,fill opacity=0.00,] (271.61,310.22) --
	(276.98,310.22) --
	(276.98,321.97) --
	(271.61,321.97) --
	(271.61,310.22);

\draw[color=drawColor,line width= 1.2pt,line join=round,fill opacity=0.00,] (278.32,322.84) -- (283.69,322.84);

\draw[color=drawColor,dash pattern=on 4pt off 4pt ,line cap=round,line join=round,fill opacity=0.00,] (281.01,288.84) -- (281.01,312.64);

\draw[color=drawColor,dash pattern=on 4pt off 4pt ,line cap=round,line join=round,fill opacity=0.00,] (281.01,360.15) -- (281.01,331.96);

\draw[color=drawColor,line cap=round,line join=round,fill opacity=0.00,] (279.66,288.84) -- (282.35,288.84);

\draw[color=drawColor,line cap=round,line join=round,fill opacity=0.00,] (279.66,360.15) -- (282.35,360.15);

\draw[color=drawColor,line cap=round,line join=round,fill opacity=0.00,] (278.32,312.64) --
	(283.69,312.64) --
	(283.69,331.96) --
	(278.32,331.96) --
	(278.32,312.64);

\draw[color=drawColor,line width= 1.2pt,line join=round,fill opacity=0.00,] (285.03,345.05) -- (290.40,345.05);

\draw[color=drawColor,dash pattern=on 4pt off 4pt ,line cap=round,line join=round,fill opacity=0.00,] (287.72,307.22) -- (287.72,334.88);

\draw[color=drawColor,dash pattern=on 4pt off 4pt ,line cap=round,line join=round,fill opacity=0.00,] (287.72,385.94) -- (287.72,355.31);

\draw[color=drawColor,line cap=round,line join=round,fill opacity=0.00,] (286.38,307.22) -- (289.06,307.22);

\draw[color=drawColor,line cap=round,line join=round,fill opacity=0.00,] (286.38,385.94) -- (289.06,385.94);

\draw[color=drawColor,line cap=round,line join=round,fill opacity=0.00,] (285.03,334.88) --
	(290.40,334.88) --
	(290.40,355.31) --
	(285.03,355.31) --
	(285.03,334.88);

\draw[color=drawColor,line width= 1.2pt,line join=round,fill opacity=0.00,] (291.75,317.43) -- (297.12,317.43);

\draw[color=drawColor,dash pattern=on 4pt off 4pt ,line cap=round,line join=round,fill opacity=0.00,] (294.43,292.75) -- (294.43,310.59);

\draw[color=drawColor,dash pattern=on 4pt off 4pt ,line cap=round,line join=round,fill opacity=0.00,] (294.43,360.08) -- (294.43,330.54);

\draw[color=drawColor,line cap=round,line join=round,fill opacity=0.00,] (293.09,292.75) -- (295.78,292.75);

\draw[color=drawColor,line cap=round,line join=round,fill opacity=0.00,] (293.09,360.08) -- (295.78,360.08);

\draw[color=drawColor,line cap=round,line join=round,fill opacity=0.00,] (291.75,310.59) --
	(297.12,310.59) --
	(297.12,330.54) --
	(291.75,330.54) --
	(291.75,310.59);

\draw[color=drawColor,line width= 1.2pt,line join=round,fill opacity=0.00,] (298.46,317.21) -- (303.83,317.21);

\draw[color=drawColor,dash pattern=on 4pt off 4pt ,line cap=round,line join=round,fill opacity=0.00,] (301.15,291.54) -- (301.15,309.96);

\draw[color=drawColor,dash pattern=on 4pt off 4pt ,line cap=round,line join=round,fill opacity=0.00,] (301.15,350.67) -- (301.15,326.44);

\draw[color=drawColor,line cap=round,line join=round,fill opacity=0.00,] (299.80,291.54) -- (302.49,291.54);

\draw[color=drawColor,line cap=round,line join=round,fill opacity=0.00,] (299.80,350.67) -- (302.49,350.67);

\draw[color=drawColor,line cap=round,line join=round,fill opacity=0.00,] (298.46,309.96) --
	(303.83,309.96) --
	(303.83,326.44) --
	(298.46,326.44) --
	(298.46,309.96);

\draw[color=drawColor,line width= 1.2pt,line join=round,fill opacity=0.00,] (305.17,292.11) -- (310.54,292.11);

\draw[color=drawColor,dash pattern=on 4pt off 4pt ,line cap=round,line join=round,fill opacity=0.00,] (307.86,276.08) -- (307.86,286.51);

\draw[color=drawColor,dash pattern=on 4pt off 4pt ,line cap=round,line join=round,fill opacity=0.00,] (307.86,315.24) -- (307.86,298.07);

\draw[color=drawColor,line cap=round,line join=round,fill opacity=0.00,] (306.52,276.08) -- (309.20,276.08);

\draw[color=drawColor,line cap=round,line join=round,fill opacity=0.00,] (306.52,315.24) -- (309.20,315.24);

\draw[color=drawColor,line cap=round,line join=round,fill opacity=0.00,] (305.17,286.51) --
	(310.54,286.51) --
	(310.54,298.07) --
	(305.17,298.07) --
	(305.17,286.51);

\draw[color=drawColor,line width= 1.2pt,line join=round,fill opacity=0.00,] (311.89,301.24) -- (317.26,301.24);

\draw[color=drawColor,dash pattern=on 4pt off 4pt ,line cap=round,line join=round,fill opacity=0.00,] (314.57,275.84) -- (314.57,292.43);

\draw[color=drawColor,dash pattern=on 4pt off 4pt ,line cap=round,line join=round,fill opacity=0.00,] (314.57,337.02) -- (314.57,310.60);

\draw[color=drawColor,line cap=round,line join=round,fill opacity=0.00,] (313.23,275.84) -- (315.92,275.84);

\draw[color=drawColor,line cap=round,line join=round,fill opacity=0.00,] (313.23,337.02) -- (315.92,337.02);

\draw[color=drawColor,line cap=round,line join=round,fill opacity=0.00,] (311.89,292.43) --
	(317.26,292.43) --
	(317.26,310.60) --
	(311.89,310.60) --
	(311.89,292.43);

\draw[color=drawColor,line width= 1.2pt,line join=round,fill opacity=0.00,] (318.60,287.30) -- (323.97,287.30);

\draw[color=drawColor,dash pattern=on 4pt off 4pt ,line cap=round,line join=round,fill opacity=0.00,] (321.29,272.49) -- (321.29,281.63);

\draw[color=drawColor,dash pattern=on 4pt off 4pt ,line cap=round,line join=round,fill opacity=0.00,] (321.29,307.95) -- (321.29,292.16);

\draw[color=drawColor,line cap=round,line join=round,fill opacity=0.00,] (319.94,272.49) -- (322.63,272.49);

\draw[color=drawColor,line cap=round,line join=round,fill opacity=0.00,] (319.94,307.95) -- (322.63,307.95);

\draw[color=drawColor,line cap=round,line join=round,fill opacity=0.00,] (318.60,281.63) --
	(323.97,281.63) --
	(323.97,292.16) --
	(318.60,292.16) --
	(318.60,281.63);

\draw[color=drawColor,line width= 1.2pt,line join=round,fill opacity=0.00,] (325.31,266.89) -- (330.69,266.89);

\draw[color=drawColor,dash pattern=on 4pt off 4pt ,line cap=round,line join=round,fill opacity=0.00,] (328.00,251.66) -- (328.00,263.55);

\draw[color=drawColor,dash pattern=on 4pt off 4pt ,line cap=round,line join=round,fill opacity=0.00,] (328.00,286.82) -- (328.00,272.90);

\draw[color=drawColor,line cap=round,line join=round,fill opacity=0.00,] (326.66,251.66) -- (329.34,251.66);

\draw[color=drawColor,line cap=round,line join=round,fill opacity=0.00,] (326.66,286.82) -- (329.34,286.82);

\draw[color=drawColor,line cap=round,line join=round,fill opacity=0.00,] (325.31,263.55) --
	(330.69,263.55) --
	(330.69,272.90) --
	(325.31,272.90) --
	(325.31,263.55);

\draw[color=drawColor,line width= 1.2pt,line join=round,fill opacity=0.00,] (332.03,280.10) -- (337.40,280.10);

\draw[color=drawColor,dash pattern=on 4pt off 4pt ,line cap=round,line join=round,fill opacity=0.00,] (334.71,261.98) -- (334.71,275.77);

\draw[color=drawColor,dash pattern=on 4pt off 4pt ,line cap=round,line join=round,fill opacity=0.00,] (334.71,297.45) -- (334.71,285.18);

\draw[color=drawColor,line cap=round,line join=round,fill opacity=0.00,] (333.37,261.98) -- (336.06,261.98);

\draw[color=drawColor,line cap=round,line join=round,fill opacity=0.00,] (333.37,297.45) -- (336.06,297.45);

\draw[color=drawColor,line cap=round,line join=round,fill opacity=0.00,] (332.03,275.77) --
	(337.40,275.77) --
	(337.40,285.18) --
	(332.03,285.18) --
	(332.03,275.77);

\draw[color=drawColor,line width= 1.2pt,line join=round,fill opacity=0.00,] (338.74,285.08) -- (344.11,285.08);

\draw[color=drawColor,dash pattern=on 4pt off 4pt ,line cap=round,line join=round,fill opacity=0.00,] (341.43,269.21) -- (341.43,280.50);

\draw[color=drawColor,dash pattern=on 4pt off 4pt ,line cap=round,line join=round,fill opacity=0.00,] (341.43,301.23) -- (341.43,290.61);

\draw[color=drawColor,line cap=round,line join=round,fill opacity=0.00,] (340.08,269.21) -- (342.77,269.21);

\draw[color=drawColor,line cap=round,line join=round,fill opacity=0.00,] (340.08,301.23) -- (342.77,301.23);

\draw[color=drawColor,line cap=round,line join=round,fill opacity=0.00,] (338.74,280.50) --
	(344.11,280.50) --
	(344.11,290.61) --
	(338.74,290.61) --
	(338.74,280.50);

\draw[color=drawColor,line width= 1.2pt,line join=round,fill opacity=0.00,] (345.45,308.86) -- (350.83,308.86);

\draw[color=drawColor,dash pattern=on 4pt off 4pt ,line cap=round,line join=round,fill opacity=0.00,] (348.14,290.13) -- (348.14,302.06);

\draw[color=drawColor,dash pattern=on 4pt off 4pt ,line cap=round,line join=round,fill opacity=0.00,] (348.14,334.71) -- (348.14,315.22);

\draw[color=drawColor,line cap=round,line join=round,fill opacity=0.00,] (346.80,290.13) -- (349.48,290.13);

\draw[color=drawColor,line cap=round,line join=round,fill opacity=0.00,] (346.80,334.71) -- (349.48,334.71);

\draw[color=drawColor,line cap=round,line join=round,fill opacity=0.00,] (345.45,302.06) --
	(350.83,302.06) --
	(350.83,315.22) --
	(345.45,315.22) --
	(345.45,302.06);

\draw[color=drawColor,line width= 1.2pt,line join=round,fill opacity=0.00,] (352.17,286.08) -- (357.54,286.08);

\draw[color=drawColor,dash pattern=on 4pt off 4pt ,line cap=round,line join=round,fill opacity=0.00,] (354.85,271.09) -- (354.85,281.24);

\draw[color=drawColor,dash pattern=on 4pt off 4pt ,line cap=round,line join=round,fill opacity=0.00,] (354.85,303.51) -- (354.85,290.72);

\draw[color=drawColor,line cap=round,line join=round,fill opacity=0.00,] (353.51,271.09) -- (356.20,271.09);

\draw[color=drawColor,line cap=round,line join=round,fill opacity=0.00,] (353.51,303.51) -- (356.20,303.51);

\draw[color=drawColor,line cap=round,line join=round,fill opacity=0.00,] (352.17,281.24) --
	(357.54,281.24) --
	(357.54,290.72) --
	(352.17,290.72) --
	(352.17,281.24);

\draw[color=drawColor,line width= 1.2pt,line join=round,fill opacity=0.00,] (358.88,298.36) -- (364.25,298.36);

\draw[color=drawColor,dash pattern=on 4pt off 4pt ,line cap=round,line join=round,fill opacity=0.00,] (361.57,280.38) -- (361.57,292.13);

\draw[color=drawColor,dash pattern=on 4pt off 4pt ,line cap=round,line join=round,fill opacity=0.00,] (361.57,327.23) -- (361.57,309.77);

\draw[color=drawColor,line cap=round,line join=round,fill opacity=0.00,] (360.22,280.38) -- (362.91,280.38);

\draw[color=drawColor,line cap=round,line join=round,fill opacity=0.00,] (360.22,327.23) -- (362.91,327.23);

\draw[color=drawColor,line cap=round,line join=round,fill opacity=0.00,] (358.88,292.13) --
	(364.25,292.13) --
	(364.25,309.77) --
	(358.88,309.77) --
	(358.88,292.13);

\draw[color=drawColor,line width= 1.2pt,line join=round,fill opacity=0.00,] (365.60,291.43) -- (370.97,291.43);

\draw[color=drawColor,dash pattern=on 4pt off 4pt ,line cap=round,line join=round,fill opacity=0.00,] (368.28,274.77) -- (368.28,286.51);

\draw[color=drawColor,dash pattern=on 4pt off 4pt ,line cap=round,line join=round,fill opacity=0.00,] (368.28,311.47) -- (368.28,296.51);

\draw[color=drawColor,line cap=round,line join=round,fill opacity=0.00,] (366.94,274.77) -- (369.62,274.77);

\draw[color=drawColor,line cap=round,line join=round,fill opacity=0.00,] (366.94,311.47) -- (369.62,311.47);

\draw[color=drawColor,line cap=round,line join=round,fill opacity=0.00,] (365.60,286.51) --
	(370.97,286.51) --
	(370.97,296.51) --
	(365.60,296.51) --
	(365.60,286.51);

\draw[color=drawColor,line width= 1.2pt,line join=round,fill opacity=0.00,] (372.31,338.97) -- (377.68,338.97);

\draw[color=drawColor,dash pattern=on 4pt off 4pt ,line cap=round,line join=round,fill opacity=0.00,] (374.99,305.52) -- (374.99,329.91);

\draw[color=drawColor,dash pattern=on 4pt off 4pt ,line cap=round,line join=round,fill opacity=0.00,] (374.99,379.84) -- (374.99,351.79);

\draw[color=drawColor,line cap=round,line join=round,fill opacity=0.00,] (373.65,305.52) -- (376.34,305.52);

\draw[color=drawColor,line cap=round,line join=round,fill opacity=0.00,] (373.65,379.84) -- (376.34,379.84);

\draw[color=drawColor,line cap=round,line join=round,fill opacity=0.00,] (372.31,329.91) --
	(377.68,329.91) --
	(377.68,351.79) --
	(372.31,351.79) --
	(372.31,329.91);

\draw[color=drawColor,line width= 1.2pt,line join=round,fill opacity=0.00,] (379.02,301.85) -- (384.39,301.85);

\draw[color=drawColor,dash pattern=on 4pt off 4pt ,line cap=round,line join=round,fill opacity=0.00,] (381.71,275.01) -- (381.71,289.47);

\draw[color=drawColor,dash pattern=on 4pt off 4pt ,line cap=round,line join=round,fill opacity=0.00,] (381.71,349.86) -- (381.71,318.06);

\draw[color=drawColor,line cap=round,line join=round,fill opacity=0.00,] (380.37,275.01) -- (383.05,275.01);

\draw[color=drawColor,line cap=round,line join=round,fill opacity=0.00,] (380.37,349.86) -- (383.05,349.86);

\draw[color=drawColor,line cap=round,line join=round,fill opacity=0.00,] (379.02,289.47) --
	(384.39,289.47) --
	(384.39,318.06) --
	(379.02,318.06) --
	(379.02,289.47);

\draw[color=drawColor,line width= 1.2pt,line join=round,fill opacity=0.00,] (385.74,292.41) -- (391.11,292.41);

\draw[color=drawColor,dash pattern=on 4pt off 4pt ,line cap=round,line join=round,fill opacity=0.00,] (388.42,275.31) -- (388.42,286.15);

\draw[color=drawColor,dash pattern=on 4pt off 4pt ,line cap=round,line join=round,fill opacity=0.00,] (388.42,322.26) -- (388.42,300.60);

\draw[color=drawColor,line cap=round,line join=round,fill opacity=0.00,] (387.08,275.31) -- (389.76,275.31);

\draw[color=drawColor,line cap=round,line join=round,fill opacity=0.00,] (387.08,322.26) -- (389.76,322.26);

\draw[color=drawColor,line cap=round,line join=round,fill opacity=0.00,] (385.74,286.15) --
	(391.11,286.15) --
	(391.11,300.60) --
	(385.74,300.60) --
	(385.74,286.15);

\draw[color=drawColor,line width= 1.2pt,line join=round,fill opacity=0.00,] (392.45,271.61) -- (397.82,271.61);

\draw[color=drawColor,dash pattern=on 4pt off 4pt ,line cap=round,line join=round,fill opacity=0.00,] (395.13,242.19) -- (395.13,261.02);

\draw[color=drawColor,dash pattern=on 4pt off 4pt ,line cap=round,line join=round,fill opacity=0.00,] (395.13,298.03) -- (395.13,280.80);

\draw[color=drawColor,line cap=round,line join=round,fill opacity=0.00,] (393.79,242.19) -- (396.48,242.19);

\draw[color=drawColor,line cap=round,line join=round,fill opacity=0.00,] (393.79,298.03) -- (396.48,298.03);

\draw[color=drawColor,line cap=round,line join=round,fill opacity=0.00,] (392.45,261.02) --
	(397.82,261.02) --
	(397.82,280.80) --
	(392.45,280.80) --
	(392.45,261.02);

\draw[color=drawColor,line width= 1.2pt,line join=round,fill opacity=0.00,] (399.16,286.81) -- (404.53,286.81);

\draw[color=drawColor,dash pattern=on 4pt off 4pt ,line cap=round,line join=round,fill opacity=0.00,] (401.85,274.61) -- (401.85,281.46);

\draw[color=drawColor,dash pattern=on 4pt off 4pt ,line cap=round,line join=round,fill opacity=0.00,] (401.85,302.59) -- (401.85,290.89);

\draw[color=drawColor,line cap=round,line join=round,fill opacity=0.00,] (400.51,274.61) -- (403.19,274.61);

\draw[color=drawColor,line cap=round,line join=round,fill opacity=0.00,] (400.51,302.59) -- (403.19,302.59);

\draw[color=drawColor,line cap=round,line join=round,fill opacity=0.00,] (399.16,281.46) --
	(404.53,281.46) --
	(404.53,290.89) --
	(399.16,290.89) --
	(399.16,281.46);

\draw[color=drawColor,line width= 1.2pt,line join=round,fill opacity=0.00,] (405.88,279.32) -- (411.25,279.32);

\draw[color=drawColor,dash pattern=on 4pt off 4pt ,line cap=round,line join=round,fill opacity=0.00,] (408.56,263.52) -- (408.56,275.63);

\draw[color=drawColor,dash pattern=on 4pt off 4pt ,line cap=round,line join=round,fill opacity=0.00,] (408.56,295.09) -- (408.56,283.72);

\draw[color=drawColor,line cap=round,line join=round,fill opacity=0.00,] (407.22,263.52) -- (409.90,263.52);

\draw[color=drawColor,line cap=round,line join=round,fill opacity=0.00,] (407.22,295.09) -- (409.90,295.09);

\draw[color=drawColor,line cap=round,line join=round,fill opacity=0.00,] (405.88,275.63) --
	(411.25,275.63) --
	(411.25,283.72) --
	(405.88,283.72) --
	(405.88,275.63);

\draw[color=drawColor,line width= 1.2pt,line join=round,fill opacity=0.00,] (412.59,280.34) -- (417.96,280.34);

\draw[color=drawColor,dash pattern=on 4pt off 4pt ,line cap=round,line join=round,fill opacity=0.00,] (415.28,253.91) -- (415.28,271.80);

\draw[color=drawColor,dash pattern=on 4pt off 4pt ,line cap=round,line join=round,fill opacity=0.00,] (415.28,298.64) -- (415.28,285.60);

\draw[color=drawColor,line cap=round,line join=round,fill opacity=0.00,] (413.93,253.91) -- (416.62,253.91);

\draw[color=drawColor,line cap=round,line join=round,fill opacity=0.00,] (413.93,298.64) -- (416.62,298.64);

\draw[color=drawColor,line cap=round,line join=round,fill opacity=0.00,] (412.59,271.80) --
	(417.96,271.80) --
	(417.96,285.60) --
	(412.59,285.60) --
	(412.59,271.80);

\draw[color=drawColor,line width= 1.2pt,line join=round,fill opacity=0.00,] (419.30,289.08) -- (424.67,289.08);

\draw[color=drawColor,dash pattern=on 4pt off 4pt ,line cap=round,line join=round,fill opacity=0.00,] (421.99,272.25) -- (421.99,282.50);

\draw[color=drawColor,dash pattern=on 4pt off 4pt ,line cap=round,line join=round,fill opacity=0.00,] (421.99,309.38) -- (421.99,293.33);

\draw[color=drawColor,line cap=round,line join=round,fill opacity=0.00,] (420.65,272.25) -- (423.33,272.25);

\draw[color=drawColor,line cap=round,line join=round,fill opacity=0.00,] (420.65,309.38) -- (423.33,309.38);

\draw[color=drawColor,line cap=round,line join=round,fill opacity=0.00,] (419.30,282.50) --
	(424.67,282.50) --
	(424.67,293.33) --
	(419.30,293.33) --
	(419.30,282.50);

\draw[color=drawColor,line width= 1.2pt,line join=round,fill opacity=0.00,] (426.02,317.21) -- (431.39,317.21);

\draw[color=drawColor,dash pattern=on 4pt off 4pt ,line cap=round,line join=round,fill opacity=0.00,] (428.70,293.92) -- (428.70,310.87);

\draw[color=drawColor,dash pattern=on 4pt off 4pt ,line cap=round,line join=round,fill opacity=0.00,] (428.70,354.33) -- (428.70,328.49);

\draw[color=drawColor,line cap=round,line join=round,fill opacity=0.00,] (427.36,293.92) -- (430.04,293.92);

\draw[color=drawColor,line cap=round,line join=round,fill opacity=0.00,] (427.36,354.33) -- (430.04,354.33);

\draw[color=drawColor,line cap=round,line join=round,fill opacity=0.00,] (426.02,310.87) --
	(431.39,310.87) --
	(431.39,328.49) --
	(426.02,328.49) --
	(426.02,310.87);

\draw[color=drawColor,line width= 1.2pt,line join=round,fill opacity=0.00,] (432.73,291.47) -- (438.10,291.47);

\draw[color=drawColor,dash pattern=on 4pt off 4pt ,line cap=round,line join=round,fill opacity=0.00,] (435.42,272.75) -- (435.42,286.16);

\draw[color=drawColor,dash pattern=on 4pt off 4pt ,line cap=round,line join=round,fill opacity=0.00,] (435.42,314.72) -- (435.42,297.61);

\draw[color=drawColor,line cap=round,line join=round,fill opacity=0.00,] (434.07,272.75) -- (436.76,272.75);

\draw[color=drawColor,line cap=round,line join=round,fill opacity=0.00,] (434.07,314.72) -- (436.76,314.72);

\draw[color=drawColor,line cap=round,line join=round,fill opacity=0.00,] (432.73,286.16) --
	(438.10,286.16) --
	(438.10,297.61) --
	(432.73,297.61) --
	(432.73,286.16);

\draw[color=drawColor,line width= 1.2pt,line join=round,fill opacity=0.00,] (439.44,287.76) -- (444.81,287.76);

\draw[color=drawColor,dash pattern=on 4pt off 4pt ,line cap=round,line join=round,fill opacity=0.00,] (442.13,273.93) -- (442.13,281.67);

\draw[color=drawColor,dash pattern=on 4pt off 4pt ,line cap=round,line join=round,fill opacity=0.00,] (442.13,302.22) -- (442.13,292.01);

\draw[color=drawColor,line cap=round,line join=round,fill opacity=0.00,] (440.79,273.93) -- (443.47,273.93);

\draw[color=drawColor,line cap=round,line join=round,fill opacity=0.00,] (440.79,302.22) -- (443.47,302.22);

\draw[color=drawColor,line cap=round,line join=round,fill opacity=0.00,] (439.44,281.67) --
	(444.81,281.67) --
	(444.81,292.01) --
	(439.44,292.01) --
	(439.44,281.67);

\draw[color=drawColor,line width= 1.2pt,line join=round,fill opacity=0.00,] (446.16,298.34) -- (451.53,298.34);

\draw[color=drawColor,dash pattern=on 4pt off 4pt ,line cap=round,line join=round,fill opacity=0.00,] (448.84,271.00) -- (448.84,289.10);

\draw[color=drawColor,dash pattern=on 4pt off 4pt ,line cap=round,line join=round,fill opacity=0.00,] (448.84,327.63) -- (448.84,308.67);

\draw[color=drawColor,line cap=round,line join=round,fill opacity=0.00,] (447.50,271.00) -- (450.19,271.00);

\draw[color=drawColor,line cap=round,line join=round,fill opacity=0.00,] (447.50,327.63) -- (450.19,327.63);

\draw[color=drawColor,line cap=round,line join=round,fill opacity=0.00,] (446.16,289.10) --
	(451.53,289.10) --
	(451.53,308.67) --
	(446.16,308.67) --
	(446.16,289.10);

\draw[color=drawColor,line width= 1.2pt,line join=round,fill opacity=0.00,] (452.87,291.80) -- (458.24,291.80);

\draw[color=drawColor,dash pattern=on 4pt off 4pt ,line cap=round,line join=round,fill opacity=0.00,] (455.56,274.30) -- (455.56,284.31);

\draw[color=drawColor,dash pattern=on 4pt off 4pt ,line cap=round,line join=round,fill opacity=0.00,] (455.56,320.60) -- (455.56,299.06);

\draw[color=drawColor,line cap=round,line join=round,fill opacity=0.00,] (454.21,274.30) -- (456.90,274.30);

\draw[color=drawColor,line cap=round,line join=round,fill opacity=0.00,] (454.21,320.60) -- (456.90,320.60);

\draw[color=drawColor,line cap=round,line join=round,fill opacity=0.00,] (452.87,284.31) --
	(458.24,284.31) --
	(458.24,299.06) --
	(452.87,299.06) --
	(452.87,284.31);

\draw[color=drawColor,line width= 1.2pt,line join=round,fill opacity=0.00,] (459.58,284.48) -- (464.96,284.48);

\draw[color=drawColor,dash pattern=on 4pt off 4pt ,line cap=round,line join=round,fill opacity=0.00,] (462.27,264.55) -- (462.27,278.75);

\draw[color=drawColor,dash pattern=on 4pt off 4pt ,line cap=round,line join=round,fill opacity=0.00,] (462.27,307.41) -- (462.27,290.24);

\draw[color=drawColor,line cap=round,line join=round,fill opacity=0.00,] (460.93,264.55) -- (463.61,264.55);

\draw[color=drawColor,line cap=round,line join=round,fill opacity=0.00,] (460.93,307.41) -- (463.61,307.41);

\draw[color=drawColor,line cap=round,line join=round,fill opacity=0.00,] (459.58,278.75) --
	(464.96,278.75) --
	(464.96,290.24) --
	(459.58,290.24) --
	(459.58,278.75);

\draw[color=drawColor,line width= 1.2pt,line join=round,fill opacity=0.00,] (466.30,304.18) -- (471.67,304.18);

\draw[color=drawColor,dash pattern=on 4pt off 4pt ,line cap=round,line join=round,fill opacity=0.00,] (468.98,257.76) -- (468.98,289.06);

\draw[color=drawColor,dash pattern=on 4pt off 4pt ,line cap=round,line join=round,fill opacity=0.00,] (468.98,346.58) -- (468.98,318.00);

\draw[color=drawColor,line cap=round,line join=round,fill opacity=0.00,] (467.64,257.76) -- (470.33,257.76);

\draw[color=drawColor,line cap=round,line join=round,fill opacity=0.00,] (467.64,346.58) -- (470.33,346.58);

\draw[color=drawColor,line cap=round,line join=round,fill opacity=0.00,] (466.30,289.06) --
	(471.67,289.06) --
	(471.67,318.00) --
	(466.30,318.00) --
	(466.30,289.06);
\end{scope}
\begin{scope}
\path[clip] (  0.00,  0.00) rectangle (505.89,650.43);
\definecolor[named]{drawColor}{rgb}{0.00,0.00,0.00}

\draw[color=drawColor,line cap=round,line join=round,fill opacity=0.00,] ( 52.75,233.44) -- (468.98,233.44);

\draw[color=drawColor,line cap=round,line join=round,fill opacity=0.00,] ( 52.75,233.44) -- ( 52.75,229.48);

\draw[color=drawColor,line cap=round,line join=round,fill opacity=0.00,] ( 59.46,233.44) -- ( 59.46,229.48);

\draw[color=drawColor,line cap=round,line join=round,fill opacity=0.00,] ( 66.17,233.44) -- ( 66.17,229.48);

\draw[color=drawColor,line cap=round,line join=round,fill opacity=0.00,] ( 72.89,233.44) -- ( 72.89,229.48);

\draw[color=drawColor,line cap=round,line join=round,fill opacity=0.00,] ( 79.60,233.44) -- ( 79.60,229.48);

\draw[color=drawColor,line cap=round,line join=round,fill opacity=0.00,] ( 86.31,233.44) -- ( 86.31,229.48);

\draw[color=drawColor,line cap=round,line join=round,fill opacity=0.00,] ( 93.03,233.44) -- ( 93.03,229.48);

\draw[color=drawColor,line cap=round,line join=round,fill opacity=0.00,] ( 99.74,233.44) -- ( 99.74,229.48);

\draw[color=drawColor,line cap=round,line join=round,fill opacity=0.00,] (106.45,233.44) -- (106.45,229.48);

\draw[color=drawColor,line cap=round,line join=round,fill opacity=0.00,] (113.17,233.44) -- (113.17,229.48);

\draw[color=drawColor,line cap=round,line join=round,fill opacity=0.00,] (119.88,233.44) -- (119.88,229.48);

\draw[color=drawColor,line cap=round,line join=round,fill opacity=0.00,] (126.60,233.44) -- (126.60,229.48);

\draw[color=drawColor,line cap=round,line join=round,fill opacity=0.00,] (133.31,233.44) -- (133.31,229.48);

\draw[color=drawColor,line cap=round,line join=round,fill opacity=0.00,] (140.02,233.44) -- (140.02,229.48);

\draw[color=drawColor,line cap=round,line join=round,fill opacity=0.00,] (146.74,233.44) -- (146.74,229.48);

\draw[color=drawColor,line cap=round,line join=round,fill opacity=0.00,] (153.45,233.44) -- (153.45,229.48);

\draw[color=drawColor,line cap=round,line join=round,fill opacity=0.00,] (160.16,233.44) -- (160.16,229.48);

\draw[color=drawColor,line cap=round,line join=round,fill opacity=0.00,] (166.88,233.44) -- (166.88,229.48);

\draw[color=drawColor,line cap=round,line join=round,fill opacity=0.00,] (173.59,233.44) -- (173.59,229.48);

\draw[color=drawColor,line cap=round,line join=round,fill opacity=0.00,] (180.30,233.44) -- (180.30,229.48);

\draw[color=drawColor,line cap=round,line join=round,fill opacity=0.00,] (187.02,233.44) -- (187.02,229.48);

\draw[color=drawColor,line cap=round,line join=round,fill opacity=0.00,] (193.73,233.44) -- (193.73,229.48);

\draw[color=drawColor,line cap=round,line join=round,fill opacity=0.00,] (200.44,233.44) -- (200.44,229.48);

\draw[color=drawColor,line cap=round,line join=round,fill opacity=0.00,] (207.16,233.44) -- (207.16,229.48);

\draw[color=drawColor,line cap=round,line join=round,fill opacity=0.00,] (213.87,233.44) -- (213.87,229.48);

\draw[color=drawColor,line cap=round,line join=round,fill opacity=0.00,] (220.58,233.44) -- (220.58,229.48);

\draw[color=drawColor,line cap=round,line join=round,fill opacity=0.00,] (227.30,233.44) -- (227.30,229.48);

\draw[color=drawColor,line cap=round,line join=round,fill opacity=0.00,] (234.01,233.44) -- (234.01,229.48);

\draw[color=drawColor,line cap=round,line join=round,fill opacity=0.00,] (240.72,233.44) -- (240.72,229.48);

\draw[color=drawColor,line cap=round,line join=round,fill opacity=0.00,] (247.44,233.44) -- (247.44,229.48);

\draw[color=drawColor,line cap=round,line join=round,fill opacity=0.00,] (254.15,233.44) -- (254.15,229.48);

\draw[color=drawColor,line cap=round,line join=round,fill opacity=0.00,] (260.87,233.44) -- (260.87,229.48);

\draw[color=drawColor,line cap=round,line join=round,fill opacity=0.00,] (267.58,233.44) -- (267.58,229.48);

\draw[color=drawColor,line cap=round,line join=round,fill opacity=0.00,] (274.29,233.44) -- (274.29,229.48);

\draw[color=drawColor,line cap=round,line join=round,fill opacity=0.00,] (281.01,233.44) -- (281.01,229.48);

\draw[color=drawColor,line cap=round,line join=round,fill opacity=0.00,] (287.72,233.44) -- (287.72,229.48);

\draw[color=drawColor,line cap=round,line join=round,fill opacity=0.00,] (294.43,233.44) -- (294.43,229.48);

\draw[color=drawColor,line cap=round,line join=round,fill opacity=0.00,] (301.15,233.44) -- (301.15,229.48);

\draw[color=drawColor,line cap=round,line join=round,fill opacity=0.00,] (307.86,233.44) -- (307.86,229.48);

\draw[color=drawColor,line cap=round,line join=round,fill opacity=0.00,] (314.57,233.44) -- (314.57,229.48);

\draw[color=drawColor,line cap=round,line join=round,fill opacity=0.00,] (321.29,233.44) -- (321.29,229.48);

\draw[color=drawColor,line cap=round,line join=round,fill opacity=0.00,] (328.00,233.44) -- (328.00,229.48);

\draw[color=drawColor,line cap=round,line join=round,fill opacity=0.00,] (334.71,233.44) -- (334.71,229.48);

\draw[color=drawColor,line cap=round,line join=round,fill opacity=0.00,] (341.43,233.44) -- (341.43,229.48);

\draw[color=drawColor,line cap=round,line join=round,fill opacity=0.00,] (348.14,233.44) -- (348.14,229.48);

\draw[color=drawColor,line cap=round,line join=round,fill opacity=0.00,] (354.85,233.44) -- (354.85,229.48);

\draw[color=drawColor,line cap=round,line join=round,fill opacity=0.00,] (361.57,233.44) -- (361.57,229.48);

\draw[color=drawColor,line cap=round,line join=round,fill opacity=0.00,] (368.28,233.44) -- (368.28,229.48);

\draw[color=drawColor,line cap=round,line join=round,fill opacity=0.00,] (374.99,233.44) -- (374.99,229.48);

\draw[color=drawColor,line cap=round,line join=round,fill opacity=0.00,] (381.71,233.44) -- (381.71,229.48);

\draw[color=drawColor,line cap=round,line join=round,fill opacity=0.00,] (388.42,233.44) -- (388.42,229.48);

\draw[color=drawColor,line cap=round,line join=round,fill opacity=0.00,] (395.13,233.44) -- (395.13,229.48);

\draw[color=drawColor,line cap=round,line join=round,fill opacity=0.00,] (401.85,233.44) -- (401.85,229.48);

\draw[color=drawColor,line cap=round,line join=round,fill opacity=0.00,] (408.56,233.44) -- (408.56,229.48);

\draw[color=drawColor,line cap=round,line join=round,fill opacity=0.00,] (415.28,233.44) -- (415.28,229.48);

\draw[color=drawColor,line cap=round,line join=round,fill opacity=0.00,] (421.99,233.44) -- (421.99,229.48);

\draw[color=drawColor,line cap=round,line join=round,fill opacity=0.00,] (428.70,233.44) -- (428.70,229.48);

\draw[color=drawColor,line cap=round,line join=round,fill opacity=0.00,] (435.42,233.44) -- (435.42,229.48);

\draw[color=drawColor,line cap=round,line join=round,fill opacity=0.00,] (442.13,233.44) -- (442.13,229.48);

\draw[color=drawColor,line cap=round,line join=round,fill opacity=0.00,] (448.84,233.44) -- (448.84,229.48);

\draw[color=drawColor,line cap=round,line join=round,fill opacity=0.00,] (455.56,233.44) -- (455.56,229.48);

\draw[color=drawColor,line cap=round,line join=round,fill opacity=0.00,] (462.27,233.44) -- (462.27,229.48);

\draw[color=drawColor,line cap=round,line join=round,fill opacity=0.00,] (468.98,233.44) -- (468.98,229.48);

\node[color=drawColor,anchor=base,inner sep=0pt, outer sep=0pt, scale=  0.66] at ( 52.75,217.60) {1949%
};

\node[color=drawColor,anchor=base,inner sep=0pt, outer sep=0pt, scale=  0.66] at ( 79.60,217.60) {1953%
};

\node[color=drawColor,anchor=base,inner sep=0pt, outer sep=0pt, scale=  0.66] at (106.45,217.60) {1957%
};

\node[color=drawColor,anchor=base,inner sep=0pt, outer sep=0pt, scale=  0.66] at (133.31,217.60) {1961%
};

\node[color=drawColor,anchor=base,inner sep=0pt, outer sep=0pt, scale=  0.66] at (160.16,217.60) {1965%
};

\node[color=drawColor,anchor=base,inner sep=0pt, outer sep=0pt, scale=  0.66] at (187.02,217.60) {1969%
};

\node[color=drawColor,anchor=base,inner sep=0pt, outer sep=0pt, scale=  0.66] at (213.87,217.60) {1973%
};

\node[color=drawColor,anchor=base,inner sep=0pt, outer sep=0pt, scale=  0.66] at (240.72,217.60) {1977%
};

\node[color=drawColor,anchor=base,inner sep=0pt, outer sep=0pt, scale=  0.66] at (267.58,217.60) {1981%
};

\node[color=drawColor,anchor=base,inner sep=0pt, outer sep=0pt, scale=  0.66] at (294.43,217.60) {1985%
};

\node[color=drawColor,anchor=base,inner sep=0pt, outer sep=0pt, scale=  0.66] at (321.29,217.60) {1989%
};

\node[color=drawColor,anchor=base,inner sep=0pt, outer sep=0pt, scale=  0.66] at (348.14,217.60) {1993%
};

\node[color=drawColor,anchor=base,inner sep=0pt, outer sep=0pt, scale=  0.66] at (374.99,217.60) {1997%
};

\node[color=drawColor,anchor=base,inner sep=0pt, outer sep=0pt, scale=  0.66] at (401.85,217.60) {2001%
};

\node[color=drawColor,anchor=base,inner sep=0pt, outer sep=0pt, scale=  0.66] at (428.70,217.60) {2005%
};

\node[color=drawColor,anchor=base,inner sep=0pt, outer sep=0pt, scale=  0.66] at (455.56,217.60) {2009%
};

\draw[color=drawColor,line cap=round,line join=round,fill opacity=0.00,] ( 32.47,239.95) -- ( 32.47,402.56);

\draw[color=drawColor,line cap=round,line join=round,fill opacity=0.00,] ( 32.47,239.95) -- ( 28.51,239.95);

\draw[color=drawColor,line cap=round,line join=round,fill opacity=0.00,] ( 32.47,280.60) -- ( 28.51,280.60);

\draw[color=drawColor,line cap=round,line join=round,fill opacity=0.00,] ( 32.47,321.25) -- ( 28.51,321.25);

\draw[color=drawColor,line cap=round,line join=round,fill opacity=0.00,] ( 32.47,361.91) -- ( 28.51,361.91);

\draw[color=drawColor,line cap=round,line join=round,fill opacity=0.00,] ( 32.47,402.56) -- ( 28.51,402.56);

\node[rotate= 90.00,color=drawColor,anchor=base,inner sep=0pt, outer sep=0pt, scale=  0.66] at ( 24.55,239.95) {0%
};

\node[rotate= 90.00,color=drawColor,anchor=base,inner sep=0pt, outer sep=0pt, scale=  0.66] at ( 24.55,280.60) {1000%
};

\node[rotate= 90.00,color=drawColor,anchor=base,inner sep=0pt, outer sep=0pt, scale=  0.66] at ( 24.55,321.25) {2000%
};

\node[rotate= 90.00,color=drawColor,anchor=base,inner sep=0pt, outer sep=0pt, scale=  0.66] at ( 24.55,361.91) {3000%
};

\node[rotate= 90.00,color=drawColor,anchor=base,inner sep=0pt, outer sep=0pt, scale=  0.66] at ( 24.55,402.56) {4000%
};

\draw[color=drawColor,line cap=round,line join=round,fill opacity=0.00,] ( 32.47,233.44) --
	(489.26,233.44) --
	(489.26,409.07) --
	( 32.47,409.07) --
	( 32.47,233.44);
\end{scope}
\begin{scope}
\path[clip] ( 32.47,233.44) rectangle (489.26,409.07);
\definecolor[named]{drawColor}{rgb}{1.00,0.00,0.00}

\draw[color=drawColor,line cap=round,line join=round,fill opacity=0.00,] ( 53.66,316.22) -- ( 58.55,295.48);

\draw[color=drawColor,line cap=round,line join=round,fill opacity=0.00,] ( 62.60,289.21) -- ( 63.04,288.87);

\draw[color=drawColor,line cap=round,line join=round,fill opacity=0.00,] ( 66.64,290.38) -- ( 72.42,339.09);

\draw[color=drawColor,line cap=round,line join=round,fill opacity=0.00,] ( 73.37,339.09) -- ( 79.11,292.66);

\draw[color=drawColor,line cap=round,line join=round,fill opacity=0.00,] ( 80.72,284.94) -- ( 85.20,269.71);

\draw[color=drawColor,line cap=round,line join=round,fill opacity=0.00,] ( 88.07,269.46) -- ( 91.27,275.92);

\draw[color=drawColor,line cap=round,line join=round,fill opacity=0.00,] ( 95.78,282.32) -- ( 96.99,283.57);

\draw[color=drawColor,line cap=round,line join=round,fill opacity=0.00,] (100.11,290.36) -- (106.09,354.77);

\draw[color=drawColor,line cap=round,line join=round,fill opacity=0.00,] (107.27,354.84) -- (112.36,330.49);

\draw[color=drawColor,line cap=round,line join=round,fill opacity=0.00,] (113.68,322.69) -- (119.37,279.46);

\draw[color=drawColor,line cap=round,line join=round,fill opacity=0.00,] (121.34,279.21) -- (125.13,288.77);

\draw[color=drawColor,line cap=round,line join=round,fill opacity=0.00,] (128.29,288.87) -- (131.62,281.83);

\draw[color=drawColor,line cap=round,line join=round,fill opacity=0.00,] (133.91,282.17) -- (139.42,318.25);

\draw[color=drawColor,line cap=round,line join=round,fill opacity=0.00,] (140.53,318.24) -- (146.23,273.78);

\draw[color=drawColor,line cap=round,line join=round,fill opacity=0.00,] (147.91,273.64) -- (152.27,287.65);

\draw[color=drawColor,line cap=round,line join=round,fill opacity=0.00,] (154.10,295.34) -- (159.51,327.95);

\draw[color=drawColor,line cap=round,line join=round,fill opacity=0.00,] (160.72,327.93) -- (166.32,288.57);

\draw[color=drawColor,line cap=round,line join=round,fill opacity=0.00,] (169.89,282.09) -- (170.57,281.51);

\draw[color=drawColor,line cap=round,line join=round,fill opacity=0.00,] (174.87,282.69) -- (179.02,294.83);

\draw[color=drawColor,line cap=round,line join=round,fill opacity=0.00,] (182.53,301.85) -- (184.79,305.18);

\draw[color=drawColor,line cap=round,line join=round,fill opacity=0.00,] (189.51,311.54) -- (191.24,313.68);

\draw[color=drawColor,line cap=round,line join=round,fill opacity=0.00,] (195.14,313.06) -- (199.03,302.85);

\draw[color=drawColor,line cap=round,line join=round,fill opacity=0.00,] (201.63,295.37) -- (205.97,281.55);

\draw[color=drawColor,line cap=round,line join=round,fill opacity=0.00,] (207.75,281.68) -- (213.28,318.03);

\draw[color=drawColor,line cap=round,line join=round,fill opacity=0.00,] (214.79,318.09) -- (219.67,297.64);

\draw[color=drawColor,line cap=round,line join=round,fill opacity=0.00,] (221.80,297.55) -- (226.08,310.87);

\draw[color=drawColor,line cap=round,line join=round,fill opacity=0.00,] (228.08,310.76) -- (233.23,285.32);

\draw[color=drawColor,line cap=round,line join=round,fill opacity=0.00,] (234.92,277.58) -- (239.81,256.92);

\draw[color=drawColor,line cap=round,line join=round,fill opacity=0.00,] (241.15,257.00) -- (247.02,311.76);

\draw[color=drawColor,line cap=round,line join=round,fill opacity=0.00,] (248.97,319.35) -- (252.62,328.07);

\draw[color=drawColor,line cap=round,line join=round,fill opacity=0.00,] (261.30,324.03) -- (267.14,271.14);

\draw[color=drawColor,line cap=round,line join=round,fill opacity=0.00,] (268.18,271.11) -- (273.69,306.77);

\draw[color=drawColor,line cap=round,line join=round,fill opacity=0.00,] (275.01,314.58) -- (280.29,343.43);

\draw[color=drawColor,line cap=round,line join=round,fill opacity=0.00,] (281.93,351.18) -- (286.80,371.49);

\draw[color=drawColor,line cap=round,line join=round,fill opacity=0.00,] (288.64,371.49) -- (293.52,350.98);

\draw[color=drawColor,line cap=round,line join=round,fill opacity=0.00,] (296.19,343.58) -- (299.39,337.08);

\draw[color=drawColor,line cap=round,line join=round,fill opacity=0.00,] (302.51,329.82) -- (306.49,319.02);

\draw[color=drawColor,line cap=round,line join=round,fill opacity=0.00,] (308.60,311.41) -- (313.83,283.86);

\draw[color=drawColor,line cap=round,line join=round,fill opacity=0.00,] (323.86,277.57) -- (325.43,275.73);

\draw[color=drawColor,line cap=round,line join=round,fill opacity=0.00,] (329.06,276.53) -- (333.65,293.01);

\draw[color=drawColor,line cap=round,line join=round,fill opacity=0.00,] (337.94,294.53) -- (338.20,294.35);

\draw[color=drawColor,line cap=round,line join=round,fill opacity=0.00,] (341.93,295.98) -- (347.64,340.37);

\draw[color=drawColor,line cap=round,line join=round,fill opacity=0.00,] (348.61,340.36) -- (354.38,292.42);

\draw[color=drawColor,line cap=round,line join=round,fill opacity=0.00,] (355.24,292.43) -- (361.18,352.57);

\draw[color=drawColor,line cap=round,line join=round,fill opacity=0.00,] (362.07,352.58) -- (367.77,308.32);

\draw[color=drawColor,line cap=round,line join=round,fill opacity=0.00,] (369.11,308.26) -- (374.17,331.90);

\draw[color=drawColor,line cap=round,line join=round,fill opacity=0.00,] (375.77,331.88) -- (380.93,305.94);

\draw[color=drawColor,line cap=round,line join=round,fill opacity=0.00,] (384.05,298.86) -- (386.08,296.08);

\draw[color=drawColor,line cap=round,line join=round,fill opacity=0.00,] (390.90,289.80) -- (392.66,287.61);

\draw[color=drawColor,line cap=round,line join=round,fill opacity=0.00,] (402.89,277.99) -- (407.52,260.98);

\draw[color=drawColor,line cap=round,line join=round,fill opacity=0.00,] (409.72,260.94) -- (414.12,275.37);

\draw[color=drawColor,line cap=round,line join=round,fill opacity=0.00,] (422.79,281.61) -- (427.91,306.52);

\draw[color=drawColor,line cap=round,line join=round,fill opacity=0.00,] (429.93,306.64) -- (434.19,293.54);

\draw[color=drawColor,line cap=round,line join=round,fill opacity=0.00,] (438.69,287.54) -- (438.86,287.43);

\draw[color=drawColor,line cap=round,line join=round,fill opacity=0.00,] ( 52.75,320.08) circle (  0.89);

\draw[color=drawColor,line cap=round,line join=round,fill opacity=0.00,] ( 59.46,291.63) circle (  0.89);

\draw[color=drawColor,line cap=round,line join=round,fill opacity=0.00,] ( 66.17,286.45) circle (  0.89);

\draw[color=drawColor,line cap=round,line join=round,fill opacity=0.00,] ( 72.89,343.02) circle (  0.89);

\draw[color=drawColor,line cap=round,line join=round,fill opacity=0.00,] ( 79.60,288.73) circle (  0.89);

\draw[color=drawColor,line cap=round,line join=round,fill opacity=0.00,] ( 86.31,265.91) circle (  0.89);

\draw[color=drawColor,line cap=round,line join=round,fill opacity=0.00,] ( 93.03,279.47) circle (  0.89);

\draw[color=drawColor,line cap=round,line join=round,fill opacity=0.00,] ( 99.74,286.42) circle (  0.89);

\draw[color=drawColor,line cap=round,line join=round,fill opacity=0.00,] (106.45,358.71) circle (  0.89);

\draw[color=drawColor,line cap=round,line join=round,fill opacity=0.00,] (113.17,326.62) circle (  0.89);

\draw[color=drawColor,line cap=round,line join=round,fill opacity=0.00,] (119.88,275.53) circle (  0.89);

\draw[color=drawColor,line cap=round,line join=round,fill opacity=0.00,] (126.60,292.45) circle (  0.89);

\draw[color=drawColor,line cap=round,line join=round,fill opacity=0.00,] (133.31,278.25) circle (  0.89);

\draw[color=drawColor,line cap=round,line join=round,fill opacity=0.00,] (140.02,322.16) circle (  0.89);

\draw[color=drawColor,line cap=round,line join=round,fill opacity=0.00,] (146.74,269.85) circle (  0.89);

\draw[color=drawColor,line cap=round,line join=round,fill opacity=0.00,] (153.45,291.43) circle (  0.89);

\draw[color=drawColor,line cap=round,line join=round,fill opacity=0.00,] (160.16,331.85) circle (  0.89);

\draw[color=drawColor,line cap=round,line join=round,fill opacity=0.00,] (166.88,284.65) circle (  0.89);

\draw[color=drawColor,line cap=round,line join=round,fill opacity=0.00,] (173.59,278.95) circle (  0.89);

\draw[color=drawColor,line cap=round,line join=round,fill opacity=0.00,] (180.30,298.58) circle (  0.89);

\draw[color=drawColor,line cap=round,line join=round,fill opacity=0.00,] (187.02,308.46) circle (  0.89);

\draw[color=drawColor,line cap=round,line join=round,fill opacity=0.00,] (193.73,316.76) circle (  0.89);

\draw[color=drawColor,line cap=round,line join=round,fill opacity=0.00,] (200.44,299.15) circle (  0.89);

\draw[color=drawColor,line cap=round,line join=round,fill opacity=0.00,] (207.16,277.77) circle (  0.89);

\draw[color=drawColor,line cap=round,line join=round,fill opacity=0.00,] (213.87,321.94) circle (  0.89);

\draw[color=drawColor,line cap=round,line join=round,fill opacity=0.00,] (220.58,293.78) circle (  0.89);

\draw[color=drawColor,line cap=round,line join=round,fill opacity=0.00,] (227.30,314.64) circle (  0.89);

\draw[color=drawColor,line cap=round,line join=round,fill opacity=0.00,] (234.01,281.43) circle (  0.89);

\draw[color=drawColor,line cap=round,line join=round,fill opacity=0.00,] (240.72,253.07) circle (  0.89);

\draw[color=drawColor,line cap=round,line join=round,fill opacity=0.00,] (247.44,315.69) circle (  0.89);

\draw[color=drawColor,line cap=round,line join=round,fill opacity=0.00,] (254.15,331.72) circle (  0.89);

\draw[color=drawColor,line cap=round,line join=round,fill opacity=0.00,] (260.87,327.96) circle (  0.89);

\draw[color=drawColor,line cap=round,line join=round,fill opacity=0.00,] (267.58,267.20) circle (  0.89);

\draw[color=drawColor,line cap=round,line join=round,fill opacity=0.00,] (274.29,310.68) circle (  0.89);

\draw[color=drawColor,line cap=round,line join=round,fill opacity=0.00,] (281.01,347.33) circle (  0.89);

\draw[color=drawColor,line cap=round,line join=round,fill opacity=0.00,] (287.72,375.34) circle (  0.89);

\draw[color=drawColor,line cap=round,line join=round,fill opacity=0.00,] (294.43,347.13) circle (  0.89);

\draw[color=drawColor,line cap=round,line join=round,fill opacity=0.00,] (301.15,333.53) circle (  0.89);

\draw[color=drawColor,line cap=round,line join=round,fill opacity=0.00,] (307.86,315.30) circle (  0.89);

\draw[color=drawColor,line cap=round,line join=round,fill opacity=0.00,] (314.57,279.97) circle (  0.89);

\draw[color=drawColor,line cap=round,line join=round,fill opacity=0.00,] (321.29,280.58) circle (  0.89);

\draw[color=drawColor,line cap=round,line join=round,fill opacity=0.00,] (328.00,272.72) circle (  0.89);

\draw[color=drawColor,line cap=round,line join=round,fill opacity=0.00,] (334.71,296.83) circle (  0.89);

\draw[color=drawColor,line cap=round,line join=round,fill opacity=0.00,] (341.43,292.06) circle (  0.89);

\draw[color=drawColor,line cap=round,line join=round,fill opacity=0.00,] (348.14,344.30) circle (  0.89);

\draw[color=drawColor,line cap=round,line join=round,fill opacity=0.00,] (354.85,288.49) circle (  0.89);

\draw[color=drawColor,line cap=round,line join=round,fill opacity=0.00,] (361.57,356.51) circle (  0.89);

\draw[color=drawColor,line cap=round,line join=round,fill opacity=0.00,] (368.28,304.39) circle (  0.89);

\draw[color=drawColor,line cap=round,line join=round,fill opacity=0.00,] (374.99,335.77) circle (  0.89);

\draw[color=drawColor,line cap=round,line join=round,fill opacity=0.00,] (381.71,302.05) circle (  0.89);

\draw[color=drawColor,line cap=round,line join=round,fill opacity=0.00,] (388.42,292.89) circle (  0.89);

\draw[color=drawColor,line cap=round,line join=round,fill opacity=0.00,] (395.13,284.52) circle (  0.89);

\draw[color=drawColor,line cap=round,line join=round,fill opacity=0.00,] (401.85,281.82) circle (  0.89);

\draw[color=drawColor,line cap=round,line join=round,fill opacity=0.00,] (408.56,257.15) circle (  0.89);

\draw[color=drawColor,line cap=round,line join=round,fill opacity=0.00,] (415.28,279.16) circle (  0.89);

\draw[color=drawColor,line cap=round,line join=round,fill opacity=0.00,] (421.99,277.73) circle (  0.89);

\draw[color=drawColor,line cap=round,line join=round,fill opacity=0.00,] (428.70,310.40) circle (  0.89);

\draw[color=drawColor,line cap=round,line join=round,fill opacity=0.00,] (435.42,289.78) circle (  0.89);

\draw[color=drawColor,line cap=round,line join=round,fill opacity=0.00,] (442.13,285.19) circle (  0.89);
\end{scope}
\begin{scope}
\path[clip] (  0.00,  0.00) rectangle (505.89,650.43);
\definecolor[named]{drawColor}{rgb}{0.00,0.00,0.00}

\node[color=drawColor,anchor=base,inner sep=0pt, outer sep=0pt, scale=  1.00] at (260.87,416.99) {(b) RPSS = 0.64 MC = 0.75%
};
\end{scope}
\begin{scope}
\path[clip] ( 32.47, 16.63) rectangle (489.26,192.26);
\end{scope}
\begin{scope}
\path[clip] ( 32.47, 16.63) rectangle (489.26,192.26);
\definecolor[named]{drawColor}{rgb}{0.00,0.00,0.00}

\draw[color=drawColor,line width= 1.2pt,line join=round,fill opacity=0.00,] ( 50.06, 74.73) -- ( 55.43, 74.73);

\draw[color=drawColor,dash pattern=on 4pt off 4pt ,line cap=round,line join=round,fill opacity=0.00,] ( 52.75, 54.64) -- ( 52.75, 69.12);

\draw[color=drawColor,dash pattern=on 4pt off 4pt ,line cap=round,line join=round,fill opacity=0.00,] ( 52.75, 99.60) -- ( 52.75, 81.34);

\draw[color=drawColor,line cap=round,line join=round,fill opacity=0.00,] ( 51.40, 54.64) -- ( 54.09, 54.64);

\draw[color=drawColor,line cap=round,line join=round,fill opacity=0.00,] ( 51.40, 99.60) -- ( 54.09, 99.60);

\draw[color=drawColor,line cap=round,line join=round,fill opacity=0.00,] ( 50.06, 69.12) --
	( 55.43, 69.12) --
	( 55.43, 81.34) --
	( 50.06, 81.34) --
	( 50.06, 69.12);

\draw[color=drawColor,line width= 1.2pt,line join=round,fill opacity=0.00,] ( 56.77, 82.22) -- ( 62.15, 82.22);

\draw[color=drawColor,dash pattern=on 4pt off 4pt ,line cap=round,line join=round,fill opacity=0.00,] ( 59.46, 60.26) -- ( 59.46, 74.88);

\draw[color=drawColor,dash pattern=on 4pt off 4pt ,line cap=round,line join=round,fill opacity=0.00,] ( 59.46,112.92) -- ( 59.46, 92.78);

\draw[color=drawColor,line cap=round,line join=round,fill opacity=0.00,] ( 58.12, 60.26) -- ( 60.80, 60.26);

\draw[color=drawColor,line cap=round,line join=round,fill opacity=0.00,] ( 58.12,112.92) -- ( 60.80,112.92);

\draw[color=drawColor,line cap=round,line join=round,fill opacity=0.00,] ( 56.77, 74.88) --
	( 62.15, 74.88) --
	( 62.15, 92.78) --
	( 56.77, 92.78) --
	( 56.77, 74.88);

\draw[color=drawColor,line width= 1.2pt,line join=round,fill opacity=0.00,] ( 63.49, 72.81) -- ( 68.86, 72.81);

\draw[color=drawColor,dash pattern=on 4pt off 4pt ,line cap=round,line join=round,fill opacity=0.00,] ( 66.17, 48.25) -- ( 66.17, 65.86);

\draw[color=drawColor,dash pattern=on 4pt off 4pt ,line cap=round,line join=round,fill opacity=0.00,] ( 66.17,101.22) -- ( 66.17, 80.10);

\draw[color=drawColor,line cap=round,line join=round,fill opacity=0.00,] ( 64.83, 48.25) -- ( 67.52, 48.25);

\draw[color=drawColor,line cap=round,line join=round,fill opacity=0.00,] ( 64.83,101.22) -- ( 67.52,101.22);

\draw[color=drawColor,line cap=round,line join=round,fill opacity=0.00,] ( 63.49, 65.86) --
	( 68.86, 65.86) --
	( 68.86, 80.10) --
	( 63.49, 80.10) --
	( 63.49, 65.86);

\draw[color=drawColor,line width= 1.2pt,line join=round,fill opacity=0.00,] ( 70.20, 81.84) -- ( 75.57, 81.84);

\draw[color=drawColor,dash pattern=on 4pt off 4pt ,line cap=round,line join=round,fill opacity=0.00,] ( 72.89, 60.39) -- ( 72.89, 75.58);

\draw[color=drawColor,dash pattern=on 4pt off 4pt ,line cap=round,line join=round,fill opacity=0.00,] ( 72.89,110.00) -- ( 72.89, 91.58);

\draw[color=drawColor,line cap=round,line join=round,fill opacity=0.00,] ( 71.54, 60.39) -- ( 74.23, 60.39);

\draw[color=drawColor,line cap=round,line join=round,fill opacity=0.00,] ( 71.54,110.00) -- ( 74.23,110.00);

\draw[color=drawColor,line cap=round,line join=round,fill opacity=0.00,] ( 70.20, 75.58) --
	( 75.57, 75.58) --
	( 75.57, 91.58) --
	( 70.20, 91.58) --
	( 70.20, 75.58);

\draw[color=drawColor,line width= 1.2pt,line join=round,fill opacity=0.00,] ( 76.92, 68.37) -- ( 82.29, 68.37);

\draw[color=drawColor,dash pattern=on 4pt off 4pt ,line cap=round,line join=round,fill opacity=0.00,] ( 79.60, 44.78) -- ( 79.60, 62.31);

\draw[color=drawColor,dash pattern=on 4pt off 4pt ,line cap=round,line join=round,fill opacity=0.00,] ( 79.60, 91.00) -- ( 79.60, 74.23);

\draw[color=drawColor,line cap=round,line join=round,fill opacity=0.00,] ( 78.26, 44.78) -- ( 80.94, 44.78);

\draw[color=drawColor,line cap=round,line join=round,fill opacity=0.00,] ( 78.26, 91.00) -- ( 80.94, 91.00);

\draw[color=drawColor,line cap=round,line join=round,fill opacity=0.00,] ( 76.92, 62.31) --
	( 82.29, 62.31) --
	( 82.29, 74.23) --
	( 76.92, 74.23) --
	( 76.92, 62.31);

\draw[color=drawColor,line width= 1.2pt,line join=round,fill opacity=0.00,] ( 83.63, 71.35) -- ( 89.00, 71.35);

\draw[color=drawColor,dash pattern=on 4pt off 4pt ,line cap=round,line join=round,fill opacity=0.00,] ( 86.31, 57.27) -- ( 86.31, 66.54);

\draw[color=drawColor,dash pattern=on 4pt off 4pt ,line cap=round,line join=round,fill opacity=0.00,] ( 86.31, 89.44) -- ( 86.31, 75.99);

\draw[color=drawColor,line cap=round,line join=round,fill opacity=0.00,] ( 84.97, 57.27) -- ( 87.66, 57.27);

\draw[color=drawColor,line cap=round,line join=round,fill opacity=0.00,] ( 84.97, 89.44) -- ( 87.66, 89.44);

\draw[color=drawColor,line cap=round,line join=round,fill opacity=0.00,] ( 83.63, 66.54) --
	( 89.00, 66.54) --
	( 89.00, 75.99) --
	( 83.63, 75.99) --
	( 83.63, 66.54);

\draw[color=drawColor,line width= 1.2pt,line join=round,fill opacity=0.00,] ( 90.34, 70.54) -- ( 95.71, 70.54);

\draw[color=drawColor,dash pattern=on 4pt off 4pt ,line cap=round,line join=round,fill opacity=0.00,] ( 93.03, 49.83) -- ( 93.03, 65.00);

\draw[color=drawColor,dash pattern=on 4pt off 4pt ,line cap=round,line join=round,fill opacity=0.00,] ( 93.03, 87.83) -- ( 93.03, 75.43);

\draw[color=drawColor,line cap=round,line join=round,fill opacity=0.00,] ( 91.69, 49.83) -- ( 94.37, 49.83);

\draw[color=drawColor,line cap=round,line join=round,fill opacity=0.00,] ( 91.69, 87.83) -- ( 94.37, 87.83);

\draw[color=drawColor,line cap=round,line join=round,fill opacity=0.00,] ( 90.34, 65.00) --
	( 95.71, 65.00) --
	( 95.71, 75.43) --
	( 90.34, 75.43) --
	( 90.34, 65.00);

\draw[color=drawColor,line width= 1.2pt,line join=round,fill opacity=0.00,] ( 97.06, 70.18) -- (102.43, 70.18);

\draw[color=drawColor,dash pattern=on 4pt off 4pt ,line cap=round,line join=round,fill opacity=0.00,] ( 99.74, 50.49) -- ( 99.74, 64.91);

\draw[color=drawColor,dash pattern=on 4pt off 4pt ,line cap=round,line join=round,fill opacity=0.00,] ( 99.74, 90.51) -- ( 99.74, 75.18);

\draw[color=drawColor,line cap=round,line join=round,fill opacity=0.00,] ( 98.40, 50.49) -- (101.08, 50.49);

\draw[color=drawColor,line cap=round,line join=round,fill opacity=0.00,] ( 98.40, 90.51) -- (101.08, 90.51);

\draw[color=drawColor,line cap=round,line join=round,fill opacity=0.00,] ( 97.06, 64.91) --
	(102.43, 64.91) --
	(102.43, 75.18) --
	( 97.06, 75.18) --
	( 97.06, 64.91);

\draw[color=drawColor,line width= 1.2pt,line join=round,fill opacity=0.00,] (103.77, 93.97) -- (109.14, 93.97);

\draw[color=drawColor,dash pattern=on 4pt off 4pt ,line cap=round,line join=round,fill opacity=0.00,] (106.45, 51.30) -- (106.45, 78.97);

\draw[color=drawColor,dash pattern=on 4pt off 4pt ,line cap=round,line join=round,fill opacity=0.00,] (106.45,134.54) -- (106.45,101.26);

\draw[color=drawColor,line cap=round,line join=round,fill opacity=0.00,] (105.11, 51.30) -- (107.80, 51.30);

\draw[color=drawColor,line cap=round,line join=round,fill opacity=0.00,] (105.11,134.54) -- (107.80,134.54);

\draw[color=drawColor,line cap=round,line join=round,fill opacity=0.00,] (103.77, 78.97) --
	(109.14, 78.97) --
	(109.14,101.26) --
	(103.77,101.26) --
	(103.77, 78.97);

\draw[color=drawColor,line width= 1.2pt,line join=round,fill opacity=0.00,] (110.48, 95.63) -- (115.85, 95.63);

\draw[color=drawColor,dash pattern=on 4pt off 4pt ,line cap=round,line join=round,fill opacity=0.00,] (113.17, 63.05) -- (113.17, 86.17);

\draw[color=drawColor,dash pattern=on 4pt off 4pt ,line cap=round,line join=round,fill opacity=0.00,] (113.17,128.60) -- (113.17,103.16);

\draw[color=drawColor,line cap=round,line join=round,fill opacity=0.00,] (111.83, 63.05) -- (114.51, 63.05);

\draw[color=drawColor,line cap=round,line join=round,fill opacity=0.00,] (111.83,128.60) -- (114.51,128.60);

\draw[color=drawColor,line cap=round,line join=round,fill opacity=0.00,] (110.48, 86.17) --
	(115.85, 86.17) --
	(115.85,103.16) --
	(110.48,103.16) --
	(110.48, 86.17);

\draw[color=drawColor,line width= 1.2pt,line join=round,fill opacity=0.00,] (117.20, 75.99) -- (122.57, 75.99);

\draw[color=drawColor,dash pattern=on 4pt off 4pt ,line cap=round,line join=round,fill opacity=0.00,] (119.88, 57.63) -- (119.88, 70.96);

\draw[color=drawColor,dash pattern=on 4pt off 4pt ,line cap=round,line join=round,fill opacity=0.00,] (119.88, 99.32) -- (119.88, 82.30);

\draw[color=drawColor,line cap=round,line join=round,fill opacity=0.00,] (118.54, 57.63) -- (121.22, 57.63);

\draw[color=drawColor,line cap=round,line join=round,fill opacity=0.00,] (118.54, 99.32) -- (121.22, 99.32);

\draw[color=drawColor,line cap=round,line join=round,fill opacity=0.00,] (117.20, 70.96) --
	(122.57, 70.96) --
	(122.57, 82.30) --
	(117.20, 82.30) --
	(117.20, 70.96);

\draw[color=drawColor,line width= 1.2pt,line join=round,fill opacity=0.00,] (123.91, 77.81) -- (129.28, 77.81);

\draw[color=drawColor,dash pattern=on 4pt off 4pt ,line cap=round,line join=round,fill opacity=0.00,] (126.60, 58.22) -- (126.60, 70.81);

\draw[color=drawColor,dash pattern=on 4pt off 4pt ,line cap=round,line join=round,fill opacity=0.00,] (126.60,116.25) -- (126.60, 89.38);

\draw[color=drawColor,line cap=round,line join=round,fill opacity=0.00,] (125.25, 58.22) -- (127.94, 58.22);

\draw[color=drawColor,line cap=round,line join=round,fill opacity=0.00,] (125.25,116.25) -- (127.94,116.25);

\draw[color=drawColor,line cap=round,line join=round,fill opacity=0.00,] (123.91, 70.81) --
	(129.28, 70.81) --
	(129.28, 89.38) --
	(123.91, 89.38) --
	(123.91, 70.81);

\draw[color=drawColor,line width= 1.2pt,line join=round,fill opacity=0.00,] (130.62, 77.91) -- (135.99, 77.91);

\draw[color=drawColor,dash pattern=on 4pt off 4pt ,line cap=round,line join=round,fill opacity=0.00,] (133.31, 58.60) -- (133.31, 73.17);

\draw[color=drawColor,dash pattern=on 4pt off 4pt ,line cap=round,line join=round,fill opacity=0.00,] (133.31,106.44) -- (133.31, 86.77);

\draw[color=drawColor,line cap=round,line join=round,fill opacity=0.00,] (131.97, 58.60) -- (134.65, 58.60);

\draw[color=drawColor,line cap=round,line join=round,fill opacity=0.00,] (131.97,106.44) -- (134.65,106.44);

\draw[color=drawColor,line cap=round,line join=round,fill opacity=0.00,] (130.62, 73.17) --
	(135.99, 73.17) --
	(135.99, 86.77) --
	(130.62, 86.77) --
	(130.62, 73.17);

\draw[color=drawColor,line width= 1.2pt,line join=round,fill opacity=0.00,] (137.34, 88.58) -- (142.71, 88.58);

\draw[color=drawColor,dash pattern=on 4pt off 4pt ,line cap=round,line join=round,fill opacity=0.00,] (140.02, 63.07) -- (140.02, 81.62);

\draw[color=drawColor,dash pattern=on 4pt off 4pt ,line cap=round,line join=round,fill opacity=0.00,] (140.02,118.43) -- (140.02, 96.58);

\draw[color=drawColor,line cap=round,line join=round,fill opacity=0.00,] (138.68, 63.07) -- (141.36, 63.07);

\draw[color=drawColor,line cap=round,line join=round,fill opacity=0.00,] (138.68,118.43) -- (141.36,118.43);

\draw[color=drawColor,line cap=round,line join=round,fill opacity=0.00,] (137.34, 81.62) --
	(142.71, 81.62) --
	(142.71, 96.58) --
	(137.34, 96.58) --
	(137.34, 81.62);

\draw[color=drawColor,line width= 1.2pt,line join=round,fill opacity=0.00,] (144.05, 73.91) -- (149.42, 73.91);

\draw[color=drawColor,dash pattern=on 4pt off 4pt ,line cap=round,line join=round,fill opacity=0.00,] (146.74, 58.21) -- (146.74, 68.80);

\draw[color=drawColor,dash pattern=on 4pt off 4pt ,line cap=round,line join=round,fill opacity=0.00,] (146.74, 94.77) -- (146.74, 79.34);

\draw[color=drawColor,line cap=round,line join=round,fill opacity=0.00,] (145.39, 58.21) -- (148.08, 58.21);

\draw[color=drawColor,line cap=round,line join=round,fill opacity=0.00,] (145.39, 94.77) -- (148.08, 94.77);

\draw[color=drawColor,line cap=round,line join=round,fill opacity=0.00,] (144.05, 68.80) --
	(149.42, 68.80) --
	(149.42, 79.34) --
	(144.05, 79.34) --
	(144.05, 68.80);

\draw[color=drawColor,line width= 1.2pt,line join=round,fill opacity=0.00,] (150.76, 81.28) -- (156.13, 81.28);

\draw[color=drawColor,dash pattern=on 4pt off 4pt ,line cap=round,line join=round,fill opacity=0.00,] (153.45, 54.84) -- (153.45, 72.63);

\draw[color=drawColor,dash pattern=on 4pt off 4pt ,line cap=round,line join=round,fill opacity=0.00,] (153.45,118.34) -- (153.45, 93.48);

\draw[color=drawColor,line cap=round,line join=round,fill opacity=0.00,] (152.11, 54.84) -- (154.79, 54.84);

\draw[color=drawColor,line cap=round,line join=round,fill opacity=0.00,] (152.11,118.34) -- (154.79,118.34);

\draw[color=drawColor,line cap=round,line join=round,fill opacity=0.00,] (150.76, 72.63) --
	(156.13, 72.63) --
	(156.13, 93.48) --
	(150.76, 93.48) --
	(150.76, 72.63);

\draw[color=drawColor,line width= 1.2pt,line join=round,fill opacity=0.00,] (157.48, 70.54) -- (162.85, 70.54);

\draw[color=drawColor,dash pattern=on 4pt off 4pt ,line cap=round,line join=round,fill opacity=0.00,] (160.16, 56.24) -- (160.16, 64.83);

\draw[color=drawColor,dash pattern=on 4pt off 4pt ,line cap=round,line join=round,fill opacity=0.00,] (160.16, 87.93) -- (160.16, 75.03);

\draw[color=drawColor,line cap=round,line join=round,fill opacity=0.00,] (158.82, 56.24) -- (161.51, 56.24);

\draw[color=drawColor,line cap=round,line join=round,fill opacity=0.00,] (158.82, 87.93) -- (161.51, 87.93);

\draw[color=drawColor,line cap=round,line join=round,fill opacity=0.00,] (157.48, 64.83) --
	(162.85, 64.83) --
	(162.85, 75.03) --
	(157.48, 75.03) --
	(157.48, 64.83);

\draw[color=drawColor,line width= 1.2pt,line join=round,fill opacity=0.00,] (164.19, 70.39) -- (169.56, 70.39);

\draw[color=drawColor,dash pattern=on 4pt off 4pt ,line cap=round,line join=round,fill opacity=0.00,] (166.88, 33.98) -- (166.88, 62.41);

\draw[color=drawColor,dash pattern=on 4pt off 4pt ,line cap=round,line join=round,fill opacity=0.00,] (166.88,114.79) -- (166.88, 83.38);

\draw[color=drawColor,line cap=round,line join=round,fill opacity=0.00,] (165.53, 33.98) -- (168.22, 33.98);

\draw[color=drawColor,line cap=round,line join=round,fill opacity=0.00,] (165.53,114.79) -- (168.22,114.79);

\draw[color=drawColor,line cap=round,line join=round,fill opacity=0.00,] (164.19, 62.41) --
	(169.56, 62.41) --
	(169.56, 83.38) --
	(164.19, 83.38) --
	(164.19, 62.41);

\draw[color=drawColor,line width= 1.2pt,line join=round,fill opacity=0.00,] (170.90, 75.29) -- (176.28, 75.29);

\draw[color=drawColor,dash pattern=on 4pt off 4pt ,line cap=round,line join=round,fill opacity=0.00,] (173.59, 57.43) -- (173.59, 69.99);

\draw[color=drawColor,dash pattern=on 4pt off 4pt ,line cap=round,line join=round,fill opacity=0.00,] (173.59,106.05) -- (173.59, 84.43);

\draw[color=drawColor,line cap=round,line join=round,fill opacity=0.00,] (172.25, 57.43) -- (174.93, 57.43);

\draw[color=drawColor,line cap=round,line join=round,fill opacity=0.00,] (172.25,106.05) -- (174.93,106.05);

\draw[color=drawColor,line cap=round,line join=round,fill opacity=0.00,] (170.90, 69.99) --
	(176.28, 69.99) --
	(176.28, 84.43) --
	(170.90, 84.43) --
	(170.90, 69.99);

\draw[color=drawColor,line width= 1.2pt,line join=round,fill opacity=0.00,] (177.62, 71.75) -- (182.99, 71.75);

\draw[color=drawColor,dash pattern=on 4pt off 4pt ,line cap=round,line join=round,fill opacity=0.00,] (180.30, 55.58) -- (180.30, 65.75);

\draw[color=drawColor,dash pattern=on 4pt off 4pt ,line cap=round,line join=round,fill opacity=0.00,] (180.30, 91.37) -- (180.30, 76.02);

\draw[color=drawColor,line cap=round,line join=round,fill opacity=0.00,] (178.96, 55.58) -- (181.65, 55.58);

\draw[color=drawColor,line cap=round,line join=round,fill opacity=0.00,] (178.96, 91.37) -- (181.65, 91.37);

\draw[color=drawColor,line cap=round,line join=round,fill opacity=0.00,] (177.62, 65.75) --
	(182.99, 65.75) --
	(182.99, 76.02) --
	(177.62, 76.02) --
	(177.62, 65.75);

\draw[color=drawColor,line width= 1.2pt,line join=round,fill opacity=0.00,] (184.33, 87.28) -- (189.70, 87.28);

\draw[color=drawColor,dash pattern=on 4pt off 4pt ,line cap=round,line join=round,fill opacity=0.00,] (187.02, 63.41) -- (187.02, 79.41);

\draw[color=drawColor,dash pattern=on 4pt off 4pt ,line cap=round,line join=round,fill opacity=0.00,] (187.02,113.25) -- (187.02, 95.97);

\draw[color=drawColor,line cap=round,line join=round,fill opacity=0.00,] (185.67, 63.41) -- (188.36, 63.41);

\draw[color=drawColor,line cap=round,line join=round,fill opacity=0.00,] (185.67,113.25) -- (188.36,113.25);

\draw[color=drawColor,line cap=round,line join=round,fill opacity=0.00,] (184.33, 79.41) --
	(189.70, 79.41) --
	(189.70, 95.97) --
	(184.33, 95.97) --
	(184.33, 79.41);

\draw[color=drawColor,line width= 1.2pt,line join=round,fill opacity=0.00,] (191.04, 87.83) -- (196.42, 87.83);

\draw[color=drawColor,dash pattern=on 4pt off 4pt ,line cap=round,line join=round,fill opacity=0.00,] (193.73, 62.23) -- (193.73, 79.80);

\draw[color=drawColor,dash pattern=on 4pt off 4pt ,line cap=round,line join=round,fill opacity=0.00,] (193.73,114.82) -- (193.73, 96.07);

\draw[color=drawColor,line cap=round,line join=round,fill opacity=0.00,] (192.39, 62.23) -- (195.07, 62.23);

\draw[color=drawColor,line cap=round,line join=round,fill opacity=0.00,] (192.39,114.82) -- (195.07,114.82);

\draw[color=drawColor,line cap=round,line join=round,fill opacity=0.00,] (191.04, 79.80) --
	(196.42, 79.80) --
	(196.42, 96.07) --
	(191.04, 96.07) --
	(191.04, 79.80);

\draw[color=drawColor,line width= 1.2pt,line join=round,fill opacity=0.00,] (197.76, 87.43) -- (203.13, 87.43);

\draw[color=drawColor,dash pattern=on 4pt off 4pt ,line cap=round,line join=round,fill opacity=0.00,] (200.44, 59.40) -- (200.44, 77.64);

\draw[color=drawColor,dash pattern=on 4pt off 4pt ,line cap=round,line join=round,fill opacity=0.00,] (200.44,125.94) -- (200.44, 97.24);

\draw[color=drawColor,line cap=round,line join=round,fill opacity=0.00,] (199.10, 59.40) -- (201.79, 59.40);

\draw[color=drawColor,line cap=round,line join=round,fill opacity=0.00,] (199.10,125.94) -- (201.79,125.94);

\draw[color=drawColor,line cap=round,line join=round,fill opacity=0.00,] (197.76, 77.64) --
	(203.13, 77.64) --
	(203.13, 97.24) --
	(197.76, 97.24) --
	(197.76, 77.64);

\draw[color=drawColor,line width= 1.2pt,line join=round,fill opacity=0.00,] (204.47, 95.51) -- (209.84, 95.51);

\draw[color=drawColor,dash pattern=on 4pt off 4pt ,line cap=round,line join=round,fill opacity=0.00,] (207.16, 64.69) -- (207.16, 85.38);

\draw[color=drawColor,dash pattern=on 4pt off 4pt ,line cap=round,line join=round,fill opacity=0.00,] (207.16,126.41) -- (207.16,102.13);

\draw[color=drawColor,line cap=round,line join=round,fill opacity=0.00,] (205.81, 64.69) -- (208.50, 64.69);

\draw[color=drawColor,line cap=round,line join=round,fill opacity=0.00,] (205.81,126.41) -- (208.50,126.41);

\draw[color=drawColor,line cap=round,line join=round,fill opacity=0.00,] (204.47, 85.38) --
	(209.84, 85.38) --
	(209.84,102.13) --
	(204.47,102.13) --
	(204.47, 85.38);

\draw[color=drawColor,line width= 1.2pt,line join=round,fill opacity=0.00,] (211.19, 92.13) -- (216.56, 92.13);

\draw[color=drawColor,dash pattern=on 4pt off 4pt ,line cap=round,line join=round,fill opacity=0.00,] (213.87, 61.89) -- (213.87, 83.56);

\draw[color=drawColor,dash pattern=on 4pt off 4pt ,line cap=round,line join=round,fill opacity=0.00,] (213.87,120.63) -- (213.87, 98.66);

\draw[color=drawColor,line cap=round,line join=round,fill opacity=0.00,] (212.53, 61.89) -- (215.21, 61.89);

\draw[color=drawColor,line cap=round,line join=round,fill opacity=0.00,] (212.53,120.63) -- (215.21,120.63);

\draw[color=drawColor,line cap=round,line join=round,fill opacity=0.00,] (211.19, 83.56) --
	(216.56, 83.56) --
	(216.56, 98.66) --
	(211.19, 98.66) --
	(211.19, 83.56);

\draw[color=drawColor,line width= 1.2pt,line join=round,fill opacity=0.00,] (217.90, 87.69) -- (223.27, 87.69);

\draw[color=drawColor,dash pattern=on 4pt off 4pt ,line cap=round,line join=round,fill opacity=0.00,] (220.58, 52.46) -- (220.58, 75.61);

\draw[color=drawColor,dash pattern=on 4pt off 4pt ,line cap=round,line join=round,fill opacity=0.00,] (220.58,128.26) -- (220.58, 98.19);

\draw[color=drawColor,line cap=round,line join=round,fill opacity=0.00,] (219.24, 52.46) -- (221.93, 52.46);

\draw[color=drawColor,line cap=round,line join=round,fill opacity=0.00,] (219.24,128.26) -- (221.93,128.26);

\draw[color=drawColor,line cap=round,line join=round,fill opacity=0.00,] (217.90, 75.61) --
	(223.27, 75.61) --
	(223.27, 98.19) --
	(217.90, 98.19) --
	(217.90, 75.61);

\draw[color=drawColor,line width= 1.2pt,line join=round,fill opacity=0.00,] (224.61, 76.06) -- (229.98, 76.06);

\draw[color=drawColor,dash pattern=on 4pt off 4pt ,line cap=round,line join=round,fill opacity=0.00,] (227.30, 55.76) -- (227.30, 71.05);

\draw[color=drawColor,dash pattern=on 4pt off 4pt ,line cap=round,line join=round,fill opacity=0.00,] (227.30,103.07) -- (227.30, 83.92);

\draw[color=drawColor,line cap=round,line join=round,fill opacity=0.00,] (225.95, 55.76) -- (228.64, 55.76);

\draw[color=drawColor,line cap=round,line join=round,fill opacity=0.00,] (225.95,103.07) -- (228.64,103.07);

\draw[color=drawColor,line cap=round,line join=round,fill opacity=0.00,] (224.61, 71.05) --
	(229.98, 71.05) --
	(229.98, 83.92) --
	(224.61, 83.92) --
	(224.61, 71.05);

\draw[color=drawColor,line width= 1.2pt,line join=round,fill opacity=0.00,] (231.33, 94.19) -- (236.70, 94.19);

\draw[color=drawColor,dash pattern=on 4pt off 4pt ,line cap=round,line join=round,fill opacity=0.00,] (234.01, 60.81) -- (234.01, 78.10);

\draw[color=drawColor,dash pattern=on 4pt off 4pt ,line cap=round,line join=round,fill opacity=0.00,] (234.01,142.79) -- (234.01,104.39);

\draw[color=drawColor,line cap=round,line join=round,fill opacity=0.00,] (232.67, 60.81) -- (235.35, 60.81);

\draw[color=drawColor,line cap=round,line join=round,fill opacity=0.00,] (232.67,142.79) -- (235.35,142.79);

\draw[color=drawColor,line cap=round,line join=round,fill opacity=0.00,] (231.33, 78.10) --
	(236.70, 78.10) --
	(236.70,104.39) --
	(231.33,104.39) --
	(231.33, 78.10);

\draw[color=drawColor,line width= 1.2pt,line join=round,fill opacity=0.00,] (238.04, 73.38) -- (243.41, 73.38);

\draw[color=drawColor,dash pattern=on 4pt off 4pt ,line cap=round,line join=round,fill opacity=0.00,] (240.72, 55.82) -- (240.72, 67.90);

\draw[color=drawColor,dash pattern=on 4pt off 4pt ,line cap=round,line join=round,fill opacity=0.00,] (240.72, 97.28) -- (240.72, 79.69);

\draw[color=drawColor,line cap=round,line join=round,fill opacity=0.00,] (239.38, 55.82) -- (242.07, 55.82);

\draw[color=drawColor,line cap=round,line join=round,fill opacity=0.00,] (239.38, 97.28) -- (242.07, 97.28);

\draw[color=drawColor,line cap=round,line join=round,fill opacity=0.00,] (238.04, 67.90) --
	(243.41, 67.90) --
	(243.41, 79.69) --
	(238.04, 79.69) --
	(238.04, 67.90);

\draw[color=drawColor,line width= 1.2pt,line join=round,fill opacity=0.00,] (244.75, 74.18) -- (250.12, 74.18);

\draw[color=drawColor,dash pattern=on 4pt off 4pt ,line cap=round,line join=round,fill opacity=0.00,] (247.44, 53.21) -- (247.44, 69.11);

\draw[color=drawColor,dash pattern=on 4pt off 4pt ,line cap=round,line join=round,fill opacity=0.00,] (247.44, 95.40) -- (247.44, 79.73);

\draw[color=drawColor,line cap=round,line join=round,fill opacity=0.00,] (246.10, 53.21) -- (248.78, 53.21);

\draw[color=drawColor,line cap=round,line join=round,fill opacity=0.00,] (246.10, 95.40) -- (248.78, 95.40);

\draw[color=drawColor,line cap=round,line join=round,fill opacity=0.00,] (244.75, 69.11) --
	(250.12, 69.11) --
	(250.12, 79.73) --
	(244.75, 79.73) --
	(244.75, 69.11);

\draw[color=drawColor,line width= 1.2pt,line join=round,fill opacity=0.00,] (251.47, 69.93) -- (256.84, 69.93);

\draw[color=drawColor,dash pattern=on 4pt off 4pt ,line cap=round,line join=round,fill opacity=0.00,] (254.15, 48.58) -- (254.15, 64.47);

\draw[color=drawColor,dash pattern=on 4pt off 4pt ,line cap=round,line join=round,fill opacity=0.00,] (254.15, 90.51) -- (254.15, 75.11);

\draw[color=drawColor,line cap=round,line join=round,fill opacity=0.00,] (252.81, 48.58) -- (255.49, 48.58);

\draw[color=drawColor,line cap=round,line join=round,fill opacity=0.00,] (252.81, 90.51) -- (255.49, 90.51);

\draw[color=drawColor,line cap=round,line join=round,fill opacity=0.00,] (251.47, 64.47) --
	(256.84, 64.47) --
	(256.84, 75.11) --
	(251.47, 75.11) --
	(251.47, 64.47);

\draw[color=drawColor,line width= 1.2pt,line join=round,fill opacity=0.00,] (258.18, 74.50) -- (263.55, 74.50);

\draw[color=drawColor,dash pattern=on 4pt off 4pt ,line cap=round,line join=round,fill opacity=0.00,] (260.87, 55.60) -- (260.87, 69.03);

\draw[color=drawColor,dash pattern=on 4pt off 4pt ,line cap=round,line join=round,fill opacity=0.00,] (260.87, 96.50) -- (260.87, 80.05);

\draw[color=drawColor,line cap=round,line join=round,fill opacity=0.00,] (259.52, 55.60) -- (262.21, 55.60);

\draw[color=drawColor,line cap=round,line join=round,fill opacity=0.00,] (259.52, 96.50) -- (262.21, 96.50);

\draw[color=drawColor,line cap=round,line join=round,fill opacity=0.00,] (258.18, 69.03) --
	(263.55, 69.03) --
	(263.55, 80.05) --
	(258.18, 80.05) --
	(258.18, 69.03);

\draw[color=drawColor,line width= 1.2pt,line join=round,fill opacity=0.00,] (264.89, 72.11) -- (270.26, 72.11);

\draw[color=drawColor,dash pattern=on 4pt off 4pt ,line cap=round,line join=round,fill opacity=0.00,] (267.58, 41.48) -- (267.58, 64.34);

\draw[color=drawColor,dash pattern=on 4pt off 4pt ,line cap=round,line join=round,fill opacity=0.00,] (267.58,103.10) -- (267.58, 80.06);

\draw[color=drawColor,line cap=round,line join=round,fill opacity=0.00,] (266.24, 41.48) -- (268.92, 41.48);

\draw[color=drawColor,line cap=round,line join=round,fill opacity=0.00,] (266.24,103.10) -- (268.92,103.10);

\draw[color=drawColor,line cap=round,line join=round,fill opacity=0.00,] (264.89, 64.34) --
	(270.26, 64.34) --
	(270.26, 80.06) --
	(264.89, 80.06) --
	(264.89, 64.34);

\draw[color=drawColor,line width= 1.2pt,line join=round,fill opacity=0.00,] (271.61, 97.61) -- (276.98, 97.61);

\draw[color=drawColor,dash pattern=on 4pt off 4pt ,line cap=round,line join=round,fill opacity=0.00,] (274.29, 70.84) -- (274.29, 88.63);

\draw[color=drawColor,dash pattern=on 4pt off 4pt ,line cap=round,line join=round,fill opacity=0.00,] (274.29,128.31) -- (274.29,104.60);

\draw[color=drawColor,line cap=round,line join=round,fill opacity=0.00,] (272.95, 70.84) -- (275.63, 70.84);

\draw[color=drawColor,line cap=round,line join=round,fill opacity=0.00,] (272.95,128.31) -- (275.63,128.31);

\draw[color=drawColor,line cap=round,line join=round,fill opacity=0.00,] (271.61, 88.63) --
	(276.98, 88.63) --
	(276.98,104.60) --
	(271.61,104.60) --
	(271.61, 88.63);

\draw[color=drawColor,line width= 1.2pt,line join=round,fill opacity=0.00,] (278.32,101.19) -- (283.69,101.19);

\draw[color=drawColor,dash pattern=on 4pt off 4pt ,line cap=round,line join=round,fill opacity=0.00,] (281.01, 67.41) -- (281.01, 92.52);

\draw[color=drawColor,dash pattern=on 4pt off 4pt ,line cap=round,line join=round,fill opacity=0.00,] (281.01,144.26) -- (281.01,113.35);

\draw[color=drawColor,line cap=round,line join=round,fill opacity=0.00,] (279.66, 67.41) -- (282.35, 67.41);

\draw[color=drawColor,line cap=round,line join=round,fill opacity=0.00,] (279.66,144.26) -- (282.35,144.26);

\draw[color=drawColor,line cap=round,line join=round,fill opacity=0.00,] (278.32, 92.52) --
	(283.69, 92.52) --
	(283.69,113.35) --
	(278.32,113.35) --
	(278.32, 92.52);

\draw[color=drawColor,line width= 1.2pt,line join=round,fill opacity=0.00,] (285.03, 89.86) -- (290.40, 89.86);

\draw[color=drawColor,dash pattern=on 4pt off 4pt ,line cap=round,line join=round,fill opacity=0.00,] (287.72, 60.79) -- (287.72, 76.80);

\draw[color=drawColor,dash pattern=on 4pt off 4pt ,line cap=round,line join=round,fill opacity=0.00,] (287.72,132.72) -- (287.72,100.02);

\draw[color=drawColor,line cap=round,line join=round,fill opacity=0.00,] (286.38, 60.79) -- (289.06, 60.79);

\draw[color=drawColor,line cap=round,line join=round,fill opacity=0.00,] (286.38,132.72) -- (289.06,132.72);

\draw[color=drawColor,line cap=round,line join=round,fill opacity=0.00,] (285.03, 76.80) --
	(290.40, 76.80) --
	(290.40,100.02) --
	(285.03,100.02) --
	(285.03, 76.80);

\draw[color=drawColor,line width= 1.2pt,line join=round,fill opacity=0.00,] (291.75, 95.71) -- (297.12, 95.71);

\draw[color=drawColor,dash pattern=on 4pt off 4pt ,line cap=round,line join=round,fill opacity=0.00,] (294.43, 70.43) -- (294.43, 87.11);

\draw[color=drawColor,dash pattern=on 4pt off 4pt ,line cap=round,line join=round,fill opacity=0.00,] (294.43,127.63) -- (294.43,103.36);

\draw[color=drawColor,line cap=round,line join=round,fill opacity=0.00,] (293.09, 70.43) -- (295.78, 70.43);

\draw[color=drawColor,line cap=round,line join=round,fill opacity=0.00,] (293.09,127.63) -- (295.78,127.63);

\draw[color=drawColor,line cap=round,line join=round,fill opacity=0.00,] (291.75, 87.11) --
	(297.12, 87.11) --
	(297.12,103.36) --
	(291.75,103.36) --
	(291.75, 87.11);

\draw[color=drawColor,line width= 1.2pt,line join=round,fill opacity=0.00,] (298.46, 96.36) -- (303.83, 96.36);

\draw[color=drawColor,dash pattern=on 4pt off 4pt ,line cap=round,line join=round,fill opacity=0.00,] (301.15, 71.69) -- (301.15, 88.25);

\draw[color=drawColor,dash pattern=on 4pt off 4pt ,line cap=round,line join=round,fill opacity=0.00,] (301.15,127.76) -- (301.15,104.12);

\draw[color=drawColor,line cap=round,line join=round,fill opacity=0.00,] (299.80, 71.69) -- (302.49, 71.69);

\draw[color=drawColor,line cap=round,line join=round,fill opacity=0.00,] (299.80,127.76) -- (302.49,127.76);

\draw[color=drawColor,line cap=round,line join=round,fill opacity=0.00,] (298.46, 88.25) --
	(303.83, 88.25) --
	(303.83,104.12) --
	(298.46,104.12) --
	(298.46, 88.25);

\draw[color=drawColor,line width= 1.2pt,line join=round,fill opacity=0.00,] (305.17, 81.90) -- (310.54, 81.90);

\draw[color=drawColor,dash pattern=on 4pt off 4pt ,line cap=round,line join=round,fill opacity=0.00,] (307.86, 58.17) -- (307.86, 74.55);

\draw[color=drawColor,dash pattern=on 4pt off 4pt ,line cap=round,line join=round,fill opacity=0.00,] (307.86,119.85) -- (307.86, 93.78);

\draw[color=drawColor,line cap=round,line join=round,fill opacity=0.00,] (306.52, 58.17) -- (309.20, 58.17);

\draw[color=drawColor,line cap=round,line join=round,fill opacity=0.00,] (306.52,119.85) -- (309.20,119.85);

\draw[color=drawColor,line cap=round,line join=round,fill opacity=0.00,] (305.17, 74.55) --
	(310.54, 74.55) --
	(310.54, 93.78) --
	(305.17, 93.78) --
	(305.17, 74.55);

\draw[color=drawColor,line width= 1.2pt,line join=round,fill opacity=0.00,] (311.89, 77.04) -- (317.26, 77.04);

\draw[color=drawColor,dash pattern=on 4pt off 4pt ,line cap=round,line join=round,fill opacity=0.00,] (314.57, 52.63) -- (314.57, 70.14);

\draw[color=drawColor,dash pattern=on 4pt off 4pt ,line cap=round,line join=round,fill opacity=0.00,] (314.57,124.06) -- (314.57, 91.77);

\draw[color=drawColor,line cap=round,line join=round,fill opacity=0.00,] (313.23, 52.63) -- (315.92, 52.63);

\draw[color=drawColor,line cap=round,line join=round,fill opacity=0.00,] (313.23,124.06) -- (315.92,124.06);

\draw[color=drawColor,line cap=round,line join=round,fill opacity=0.00,] (311.89, 70.14) --
	(317.26, 70.14) --
	(317.26, 91.77) --
	(311.89, 91.77) --
	(311.89, 70.14);

\draw[color=drawColor,line width= 1.2pt,line join=round,fill opacity=0.00,] (318.60, 67.83) -- (323.97, 67.83);

\draw[color=drawColor,dash pattern=on 4pt off 4pt ,line cap=round,line join=round,fill opacity=0.00,] (321.29, 45.07) -- (321.29, 62.14);

\draw[color=drawColor,dash pattern=on 4pt off 4pt ,line cap=round,line join=round,fill opacity=0.00,] (321.29, 90.38) -- (321.29, 73.56);

\draw[color=drawColor,line cap=round,line join=round,fill opacity=0.00,] (319.94, 45.07) -- (322.63, 45.07);

\draw[color=drawColor,line cap=round,line join=round,fill opacity=0.00,] (319.94, 90.38) -- (322.63, 90.38);

\draw[color=drawColor,line cap=round,line join=round,fill opacity=0.00,] (318.60, 62.14) --
	(323.97, 62.14) --
	(323.97, 73.56) --
	(318.60, 73.56) --
	(318.60, 62.14);

\draw[color=drawColor,line width= 1.2pt,line join=round,fill opacity=0.00,] (325.31, 82.73) -- (330.69, 82.73);

\draw[color=drawColor,dash pattern=on 4pt off 4pt ,line cap=round,line join=round,fill opacity=0.00,] (328.00, 58.70) -- (328.00, 72.71);

\draw[color=drawColor,dash pattern=on 4pt off 4pt ,line cap=round,line join=round,fill opacity=0.00,] (328.00,130.01) -- (328.00, 95.98);

\draw[color=drawColor,line cap=round,line join=round,fill opacity=0.00,] (326.66, 58.70) -- (329.34, 58.70);

\draw[color=drawColor,line cap=round,line join=round,fill opacity=0.00,] (326.66,130.01) -- (329.34,130.01);

\draw[color=drawColor,line cap=round,line join=round,fill opacity=0.00,] (325.31, 72.71) --
	(330.69, 72.71) --
	(330.69, 95.98) --
	(325.31, 95.98) --
	(325.31, 72.71);

\draw[color=drawColor,line width= 1.2pt,line join=round,fill opacity=0.00,] (332.03, 77.00) -- (337.40, 77.00);

\draw[color=drawColor,dash pattern=on 4pt off 4pt ,line cap=round,line join=round,fill opacity=0.00,] (334.71, 50.89) -- (334.71, 70.03);

\draw[color=drawColor,dash pattern=on 4pt off 4pt ,line cap=round,line join=round,fill opacity=0.00,] (334.71,108.72) -- (334.71, 85.58);

\draw[color=drawColor,line cap=round,line join=round,fill opacity=0.00,] (333.37, 50.89) -- (336.06, 50.89);

\draw[color=drawColor,line cap=round,line join=round,fill opacity=0.00,] (333.37,108.72) -- (336.06,108.72);

\draw[color=drawColor,line cap=round,line join=round,fill opacity=0.00,] (332.03, 70.03) --
	(337.40, 70.03) --
	(337.40, 85.58) --
	(332.03, 85.58) --
	(332.03, 70.03);

\draw[color=drawColor,line width= 1.2pt,line join=round,fill opacity=0.00,] (338.74, 75.76) -- (344.11, 75.76);

\draw[color=drawColor,dash pattern=on 4pt off 4pt ,line cap=round,line join=round,fill opacity=0.00,] (341.43, 57.70) -- (341.43, 70.27);

\draw[color=drawColor,dash pattern=on 4pt off 4pt ,line cap=round,line join=round,fill opacity=0.00,] (341.43,101.98) -- (341.43, 82.98);

\draw[color=drawColor,line cap=round,line join=round,fill opacity=0.00,] (340.08, 57.70) -- (342.77, 57.70);

\draw[color=drawColor,line cap=round,line join=round,fill opacity=0.00,] (340.08,101.98) -- (342.77,101.98);

\draw[color=drawColor,line cap=round,line join=round,fill opacity=0.00,] (338.74, 70.27) --
	(344.11, 70.27) --
	(344.11, 82.98) --
	(338.74, 82.98) --
	(338.74, 70.27);

\draw[color=drawColor,line width= 1.2pt,line join=round,fill opacity=0.00,] (345.45, 78.33) -- (350.83, 78.33);

\draw[color=drawColor,dash pattern=on 4pt off 4pt ,line cap=round,line join=round,fill opacity=0.00,] (348.14, 59.82) -- (348.14, 71.60);

\draw[color=drawColor,dash pattern=on 4pt off 4pt ,line cap=round,line join=round,fill opacity=0.00,] (348.14,111.51) -- (348.14, 90.09);

\draw[color=drawColor,line cap=round,line join=round,fill opacity=0.00,] (346.80, 59.82) -- (349.48, 59.82);

\draw[color=drawColor,line cap=round,line join=round,fill opacity=0.00,] (346.80,111.51) -- (349.48,111.51);

\draw[color=drawColor,line cap=round,line join=round,fill opacity=0.00,] (345.45, 71.60) --
	(350.83, 71.60) --
	(350.83, 90.09) --
	(345.45, 90.09) --
	(345.45, 71.60);

\draw[color=drawColor,line width= 1.2pt,line join=round,fill opacity=0.00,] (352.17, 79.22) -- (357.54, 79.22);

\draw[color=drawColor,dash pattern=on 4pt off 4pt ,line cap=round,line join=round,fill opacity=0.00,] (354.85, 58.48) -- (354.85, 71.68);

\draw[color=drawColor,dash pattern=on 4pt off 4pt ,line cap=round,line join=round,fill opacity=0.00,] (354.85,122.63) -- (354.85, 92.22);

\draw[color=drawColor,line cap=round,line join=round,fill opacity=0.00,] (353.51, 58.48) -- (356.20, 58.48);

\draw[color=drawColor,line cap=round,line join=round,fill opacity=0.00,] (353.51,122.63) -- (356.20,122.63);

\draw[color=drawColor,line cap=round,line join=round,fill opacity=0.00,] (352.17, 71.68) --
	(357.54, 71.68) --
	(357.54, 92.22) --
	(352.17, 92.22) --
	(352.17, 71.68);

\draw[color=drawColor,line width= 1.2pt,line join=round,fill opacity=0.00,] (358.88, 85.39) -- (364.25, 85.39);

\draw[color=drawColor,dash pattern=on 4pt off 4pt ,line cap=round,line join=round,fill opacity=0.00,] (361.57, 62.17) -- (361.57, 77.12);

\draw[color=drawColor,dash pattern=on 4pt off 4pt ,line cap=round,line join=round,fill opacity=0.00,] (361.57,110.66) -- (361.57, 94.07);

\draw[color=drawColor,line cap=round,line join=round,fill opacity=0.00,] (360.22, 62.17) -- (362.91, 62.17);

\draw[color=drawColor,line cap=round,line join=round,fill opacity=0.00,] (360.22,110.66) -- (362.91,110.66);

\draw[color=drawColor,line cap=round,line join=round,fill opacity=0.00,] (358.88, 77.12) --
	(364.25, 77.12) --
	(364.25, 94.07) --
	(358.88, 94.07) --
	(358.88, 77.12);

\draw[color=drawColor,line width= 1.2pt,line join=round,fill opacity=0.00,] (365.60, 73.28) -- (370.97, 73.28);

\draw[color=drawColor,dash pattern=on 4pt off 4pt ,line cap=round,line join=round,fill opacity=0.00,] (368.28, 53.15) -- (368.28, 67.83);

\draw[color=drawColor,dash pattern=on 4pt off 4pt ,line cap=round,line join=round,fill opacity=0.00,] (368.28, 95.66) -- (368.28, 79.00);

\draw[color=drawColor,line cap=round,line join=round,fill opacity=0.00,] (366.94, 53.15) -- (369.62, 53.15);

\draw[color=drawColor,line cap=round,line join=round,fill opacity=0.00,] (366.94, 95.66) -- (369.62, 95.66);

\draw[color=drawColor,line cap=round,line join=round,fill opacity=0.00,] (365.60, 67.83) --
	(370.97, 67.83) --
	(370.97, 79.00) --
	(365.60, 79.00) --
	(365.60, 67.83);

\draw[color=drawColor,line width= 1.2pt,line join=round,fill opacity=0.00,] (372.31, 89.16) -- (377.68, 89.16);

\draw[color=drawColor,dash pattern=on 4pt off 4pt ,line cap=round,line join=round,fill opacity=0.00,] (374.99, 61.14) -- (374.99, 81.25);

\draw[color=drawColor,dash pattern=on 4pt off 4pt ,line cap=round,line join=round,fill opacity=0.00,] (374.99,120.77) -- (374.99, 98.26);

\draw[color=drawColor,line cap=round,line join=round,fill opacity=0.00,] (373.65, 61.14) -- (376.34, 61.14);

\draw[color=drawColor,line cap=round,line join=round,fill opacity=0.00,] (373.65,120.77) -- (376.34,120.77);

\draw[color=drawColor,line cap=round,line join=round,fill opacity=0.00,] (372.31, 81.25) --
	(377.68, 81.25) --
	(377.68, 98.26) --
	(372.31, 98.26) --
	(372.31, 81.25);

\draw[color=drawColor,line width= 1.2pt,line join=round,fill opacity=0.00,] (379.02,100.07) -- (384.39,100.07);

\draw[color=drawColor,dash pattern=on 4pt off 4pt ,line cap=round,line join=round,fill opacity=0.00,] (381.71, 63.25) -- (381.71, 90.81);

\draw[color=drawColor,dash pattern=on 4pt off 4pt ,line cap=round,line join=round,fill opacity=0.00,] (381.71,139.93) -- (381.71,110.47);

\draw[color=drawColor,line cap=round,line join=round,fill opacity=0.00,] (380.37, 63.25) -- (383.05, 63.25);

\draw[color=drawColor,line cap=round,line join=round,fill opacity=0.00,] (380.37,139.93) -- (383.05,139.93);

\draw[color=drawColor,line cap=round,line join=round,fill opacity=0.00,] (379.02, 90.81) --
	(384.39, 90.81) --
	(384.39,110.47) --
	(379.02,110.47) --
	(379.02, 90.81);

\draw[color=drawColor,line width= 1.2pt,line join=round,fill opacity=0.00,] (385.74, 97.87) -- (391.11, 97.87);

\draw[color=drawColor,dash pattern=on 4pt off 4pt ,line cap=round,line join=round,fill opacity=0.00,] (388.42, 71.20) -- (388.42, 88.90);

\draw[color=drawColor,dash pattern=on 4pt off 4pt ,line cap=round,line join=round,fill opacity=0.00,] (388.42,127.75) -- (388.42,104.47);

\draw[color=drawColor,line cap=round,line join=round,fill opacity=0.00,] (387.08, 71.20) -- (389.76, 71.20);

\draw[color=drawColor,line cap=round,line join=round,fill opacity=0.00,] (387.08,127.75) -- (389.76,127.75);

\draw[color=drawColor,line cap=round,line join=round,fill opacity=0.00,] (385.74, 88.90) --
	(391.11, 88.90) --
	(391.11,104.47) --
	(385.74,104.47) --
	(385.74, 88.90);

\draw[color=drawColor,line width= 1.2pt,line join=round,fill opacity=0.00,] (392.45, 69.40) -- (397.82, 69.40);

\draw[color=drawColor,dash pattern=on 4pt off 4pt ,line cap=round,line join=round,fill opacity=0.00,] (395.13, 44.81) -- (395.13, 63.00);

\draw[color=drawColor,dash pattern=on 4pt off 4pt ,line cap=round,line join=round,fill opacity=0.00,] (395.13, 92.92) -- (395.13, 75.22);

\draw[color=drawColor,line cap=round,line join=round,fill opacity=0.00,] (393.79, 44.81) -- (396.48, 44.81);

\draw[color=drawColor,line cap=round,line join=round,fill opacity=0.00,] (393.79, 92.92) -- (396.48, 92.92);

\draw[color=drawColor,line cap=round,line join=round,fill opacity=0.00,] (392.45, 63.00) --
	(397.82, 63.00) --
	(397.82, 75.22) --
	(392.45, 75.22) --
	(392.45, 63.00);

\draw[color=drawColor,line width= 1.2pt,line join=round,fill opacity=0.00,] (399.16, 78.63) -- (404.53, 78.63);

\draw[color=drawColor,dash pattern=on 4pt off 4pt ,line cap=round,line join=round,fill opacity=0.00,] (401.85, 57.76) -- (401.85, 71.89);

\draw[color=drawColor,dash pattern=on 4pt off 4pt ,line cap=round,line join=round,fill opacity=0.00,] (401.85,116.98) -- (401.85, 91.45);

\draw[color=drawColor,line cap=round,line join=round,fill opacity=0.00,] (400.51, 57.76) -- (403.19, 57.76);

\draw[color=drawColor,line cap=round,line join=round,fill opacity=0.00,] (400.51,116.98) -- (403.19,116.98);

\draw[color=drawColor,line cap=round,line join=round,fill opacity=0.00,] (399.16, 71.89) --
	(404.53, 71.89) --
	(404.53, 91.45) --
	(399.16, 91.45) --
	(399.16, 71.89);

\draw[color=drawColor,line width= 1.2pt,line join=round,fill opacity=0.00,] (405.88, 76.33) -- (411.25, 76.33);

\draw[color=drawColor,dash pattern=on 4pt off 4pt ,line cap=round,line join=round,fill opacity=0.00,] (408.56, 54.71) -- (408.56, 70.75);

\draw[color=drawColor,dash pattern=on 4pt off 4pt ,line cap=round,line join=round,fill opacity=0.00,] (408.56,103.43) -- (408.56, 83.97);

\draw[color=drawColor,line cap=round,line join=round,fill opacity=0.00,] (407.22, 54.71) -- (409.90, 54.71);

\draw[color=drawColor,line cap=round,line join=round,fill opacity=0.00,] (407.22,103.43) -- (409.90,103.43);

\draw[color=drawColor,line cap=round,line join=round,fill opacity=0.00,] (405.88, 70.75) --
	(411.25, 70.75) --
	(411.25, 83.97) --
	(405.88, 83.97) --
	(405.88, 70.75);

\draw[color=drawColor,line width= 1.2pt,line join=round,fill opacity=0.00,] (412.59, 74.78) -- (417.96, 74.78);

\draw[color=drawColor,dash pattern=on 4pt off 4pt ,line cap=round,line join=round,fill opacity=0.00,] (415.28, 54.53) -- (415.28, 68.97);

\draw[color=drawColor,dash pattern=on 4pt off 4pt ,line cap=round,line join=round,fill opacity=0.00,] (415.28, 98.30) -- (415.28, 80.71);

\draw[color=drawColor,line cap=round,line join=round,fill opacity=0.00,] (413.93, 54.53) -- (416.62, 54.53);

\draw[color=drawColor,line cap=round,line join=round,fill opacity=0.00,] (413.93, 98.30) -- (416.62, 98.30);

\draw[color=drawColor,line cap=round,line join=round,fill opacity=0.00,] (412.59, 68.97) --
	(417.96, 68.97) --
	(417.96, 80.71) --
	(412.59, 80.71) --
	(412.59, 68.97);

\draw[color=drawColor,line width= 1.2pt,line join=round,fill opacity=0.00,] (419.30, 69.27) -- (424.67, 69.27);

\draw[color=drawColor,dash pattern=on 4pt off 4pt ,line cap=round,line join=round,fill opacity=0.00,] (421.99, 46.03) -- (421.99, 63.04);

\draw[color=drawColor,dash pattern=on 4pt off 4pt ,line cap=round,line join=round,fill opacity=0.00,] (421.99, 92.30) -- (421.99, 74.76);

\draw[color=drawColor,line cap=round,line join=round,fill opacity=0.00,] (420.65, 46.03) -- (423.33, 46.03);

\draw[color=drawColor,line cap=round,line join=round,fill opacity=0.00,] (420.65, 92.30) -- (423.33, 92.30);

\draw[color=drawColor,line cap=round,line join=round,fill opacity=0.00,] (419.30, 63.04) --
	(424.67, 63.04) --
	(424.67, 74.76) --
	(419.30, 74.76) --
	(419.30, 63.04);

\draw[color=drawColor,line width= 1.2pt,line join=round,fill opacity=0.00,] (426.02, 93.75) -- (431.39, 93.75);

\draw[color=drawColor,dash pattern=on 4pt off 4pt ,line cap=round,line join=round,fill opacity=0.00,] (428.70, 63.88) -- (428.70, 84.49);

\draw[color=drawColor,dash pattern=on 4pt off 4pt ,line cap=round,line join=round,fill opacity=0.00,] (428.70,120.91) -- (428.70, 99.82);

\draw[color=drawColor,line cap=round,line join=round,fill opacity=0.00,] (427.36, 63.88) -- (430.04, 63.88);

\draw[color=drawColor,line cap=round,line join=round,fill opacity=0.00,] (427.36,120.91) -- (430.04,120.91);

\draw[color=drawColor,line cap=round,line join=round,fill opacity=0.00,] (426.02, 84.49) --
	(431.39, 84.49) --
	(431.39, 99.82) --
	(426.02, 99.82) --
	(426.02, 84.49);

\draw[color=drawColor,line width= 1.2pt,line join=round,fill opacity=0.00,] (432.73, 76.88) -- (438.10, 76.88);

\draw[color=drawColor,dash pattern=on 4pt off 4pt ,line cap=round,line join=round,fill opacity=0.00,] (435.42, 42.86) -- (435.42, 69.05);

\draw[color=drawColor,dash pattern=on 4pt off 4pt ,line cap=round,line join=round,fill opacity=0.00,] (435.42,118.78) -- (435.42, 89.43);

\draw[color=drawColor,line cap=round,line join=round,fill opacity=0.00,] (434.07, 42.86) -- (436.76, 42.86);

\draw[color=drawColor,line cap=round,line join=round,fill opacity=0.00,] (434.07,118.78) -- (436.76,118.78);

\draw[color=drawColor,line cap=round,line join=round,fill opacity=0.00,] (432.73, 69.05) --
	(438.10, 69.05) --
	(438.10, 89.43) --
	(432.73, 89.43) --
	(432.73, 69.05);

\draw[color=drawColor,line width= 1.2pt,line join=round,fill opacity=0.00,] (439.44, 83.18) -- (444.81, 83.18);

\draw[color=drawColor,dash pattern=on 4pt off 4pt ,line cap=round,line join=round,fill opacity=0.00,] (442.13, 58.52) -- (442.13, 75.98);

\draw[color=drawColor,dash pattern=on 4pt off 4pt ,line cap=round,line join=round,fill opacity=0.00,] (442.13,112.01) -- (442.13, 93.24);

\draw[color=drawColor,line cap=round,line join=round,fill opacity=0.00,] (440.79, 58.52) -- (443.47, 58.52);

\draw[color=drawColor,line cap=round,line join=round,fill opacity=0.00,] (440.79,112.01) -- (443.47,112.01);

\draw[color=drawColor,line cap=round,line join=round,fill opacity=0.00,] (439.44, 75.98) --
	(444.81, 75.98) --
	(444.81, 93.24) --
	(439.44, 93.24) --
	(439.44, 75.98);

\draw[color=drawColor,line width= 1.2pt,line join=round,fill opacity=0.00,] (446.16, 74.49) -- (451.53, 74.49);

\draw[color=drawColor,dash pattern=on 4pt off 4pt ,line cap=round,line join=round,fill opacity=0.00,] (448.84, 34.52) -- (448.84, 64.02);

\draw[color=drawColor,dash pattern=on 4pt off 4pt ,line cap=round,line join=round,fill opacity=0.00,] (448.84,115.53) -- (448.84, 84.81);

\draw[color=drawColor,line cap=round,line join=round,fill opacity=0.00,] (447.50, 34.52) -- (450.19, 34.52);

\draw[color=drawColor,line cap=round,line join=round,fill opacity=0.00,] (447.50,115.53) -- (450.19,115.53);

\draw[color=drawColor,line cap=round,line join=round,fill opacity=0.00,] (446.16, 64.02) --
	(451.53, 64.02) --
	(451.53, 84.81) --
	(446.16, 84.81) --
	(446.16, 64.02);

\draw[color=drawColor,line width= 1.2pt,line join=round,fill opacity=0.00,] (452.87, 70.43) -- (458.24, 70.43);

\draw[color=drawColor,dash pattern=on 4pt off 4pt ,line cap=round,line join=round,fill opacity=0.00,] (455.56, 45.46) -- (455.56, 63.99);

\draw[color=drawColor,dash pattern=on 4pt off 4pt ,line cap=round,line join=round,fill opacity=0.00,] (455.56, 97.34) -- (455.56, 77.33);

\draw[color=drawColor,line cap=round,line join=round,fill opacity=0.00,] (454.21, 45.46) -- (456.90, 45.46);

\draw[color=drawColor,line cap=round,line join=round,fill opacity=0.00,] (454.21, 97.34) -- (456.90, 97.34);

\draw[color=drawColor,line cap=round,line join=round,fill opacity=0.00,] (452.87, 63.99) --
	(458.24, 63.99) --
	(458.24, 77.33) --
	(452.87, 77.33) --
	(452.87, 63.99);

\draw[color=drawColor,line width= 1.2pt,line join=round,fill opacity=0.00,] (459.58, 75.36) -- (464.96, 75.36);

\draw[color=drawColor,dash pattern=on 4pt off 4pt ,line cap=round,line join=round,fill opacity=0.00,] (462.27, 43.80) -- (462.27, 66.60);

\draw[color=drawColor,dash pattern=on 4pt off 4pt ,line cap=round,line join=round,fill opacity=0.00,] (462.27,113.50) -- (462.27, 87.37);

\draw[color=drawColor,line cap=round,line join=round,fill opacity=0.00,] (460.93, 43.80) -- (463.61, 43.80);

\draw[color=drawColor,line cap=round,line join=round,fill opacity=0.00,] (460.93,113.50) -- (463.61,113.50);

\draw[color=drawColor,line cap=round,line join=round,fill opacity=0.00,] (459.58, 66.60) --
	(464.96, 66.60) --
	(464.96, 87.37) --
	(459.58, 87.37) --
	(459.58, 66.60);

\draw[color=drawColor,line width= 1.2pt,line join=round,fill opacity=0.00,] (466.30, 69.44) -- (471.67, 69.44);

\draw[color=drawColor,dash pattern=on 4pt off 4pt ,line cap=round,line join=round,fill opacity=0.00,] (468.98, 28.56) -- (468.98, 57.23);

\draw[color=drawColor,dash pattern=on 4pt off 4pt ,line cap=round,line join=round,fill opacity=0.00,] (468.98,103.54) -- (468.98, 76.38);

\draw[color=drawColor,line cap=round,line join=round,fill opacity=0.00,] (467.64, 28.56) -- (470.33, 28.56);

\draw[color=drawColor,line cap=round,line join=round,fill opacity=0.00,] (467.64,103.54) -- (470.33,103.54);

\draw[color=drawColor,line cap=round,line join=round,fill opacity=0.00,] (466.30, 57.23) --
	(471.67, 57.23) --
	(471.67, 76.38) --
	(466.30, 76.38) --
	(466.30, 57.23);
\end{scope}
\begin{scope}
\path[clip] (  0.00,  0.00) rectangle (505.89,650.43);
\definecolor[named]{drawColor}{rgb}{0.00,0.00,0.00}

\draw[color=drawColor,line cap=round,line join=round,fill opacity=0.00,] ( 52.75, 16.63) -- (468.98, 16.63);

\draw[color=drawColor,line cap=round,line join=round,fill opacity=0.00,] ( 52.75, 16.63) -- ( 52.75, 12.67);

\draw[color=drawColor,line cap=round,line join=round,fill opacity=0.00,] ( 59.46, 16.63) -- ( 59.46, 12.67);

\draw[color=drawColor,line cap=round,line join=round,fill opacity=0.00,] ( 66.17, 16.63) -- ( 66.17, 12.67);

\draw[color=drawColor,line cap=round,line join=round,fill opacity=0.00,] ( 72.89, 16.63) -- ( 72.89, 12.67);

\draw[color=drawColor,line cap=round,line join=round,fill opacity=0.00,] ( 79.60, 16.63) -- ( 79.60, 12.67);

\draw[color=drawColor,line cap=round,line join=round,fill opacity=0.00,] ( 86.31, 16.63) -- ( 86.31, 12.67);

\draw[color=drawColor,line cap=round,line join=round,fill opacity=0.00,] ( 93.03, 16.63) -- ( 93.03, 12.67);

\draw[color=drawColor,line cap=round,line join=round,fill opacity=0.00,] ( 99.74, 16.63) -- ( 99.74, 12.67);

\draw[color=drawColor,line cap=round,line join=round,fill opacity=0.00,] (106.45, 16.63) -- (106.45, 12.67);

\draw[color=drawColor,line cap=round,line join=round,fill opacity=0.00,] (113.17, 16.63) -- (113.17, 12.67);

\draw[color=drawColor,line cap=round,line join=round,fill opacity=0.00,] (119.88, 16.63) -- (119.88, 12.67);

\draw[color=drawColor,line cap=round,line join=round,fill opacity=0.00,] (126.60, 16.63) -- (126.60, 12.67);

\draw[color=drawColor,line cap=round,line join=round,fill opacity=0.00,] (133.31, 16.63) -- (133.31, 12.67);

\draw[color=drawColor,line cap=round,line join=round,fill opacity=0.00,] (140.02, 16.63) -- (140.02, 12.67);

\draw[color=drawColor,line cap=round,line join=round,fill opacity=0.00,] (146.74, 16.63) -- (146.74, 12.67);

\draw[color=drawColor,line cap=round,line join=round,fill opacity=0.00,] (153.45, 16.63) -- (153.45, 12.67);

\draw[color=drawColor,line cap=round,line join=round,fill opacity=0.00,] (160.16, 16.63) -- (160.16, 12.67);

\draw[color=drawColor,line cap=round,line join=round,fill opacity=0.00,] (166.88, 16.63) -- (166.88, 12.67);

\draw[color=drawColor,line cap=round,line join=round,fill opacity=0.00,] (173.59, 16.63) -- (173.59, 12.67);

\draw[color=drawColor,line cap=round,line join=round,fill opacity=0.00,] (180.30, 16.63) -- (180.30, 12.67);

\draw[color=drawColor,line cap=round,line join=round,fill opacity=0.00,] (187.02, 16.63) -- (187.02, 12.67);

\draw[color=drawColor,line cap=round,line join=round,fill opacity=0.00,] (193.73, 16.63) -- (193.73, 12.67);

\draw[color=drawColor,line cap=round,line join=round,fill opacity=0.00,] (200.44, 16.63) -- (200.44, 12.67);

\draw[color=drawColor,line cap=round,line join=round,fill opacity=0.00,] (207.16, 16.63) -- (207.16, 12.67);

\draw[color=drawColor,line cap=round,line join=round,fill opacity=0.00,] (213.87, 16.63) -- (213.87, 12.67);

\draw[color=drawColor,line cap=round,line join=round,fill opacity=0.00,] (220.58, 16.63) -- (220.58, 12.67);

\draw[color=drawColor,line cap=round,line join=round,fill opacity=0.00,] (227.30, 16.63) -- (227.30, 12.67);

\draw[color=drawColor,line cap=round,line join=round,fill opacity=0.00,] (234.01, 16.63) -- (234.01, 12.67);

\draw[color=drawColor,line cap=round,line join=round,fill opacity=0.00,] (240.72, 16.63) -- (240.72, 12.67);

\draw[color=drawColor,line cap=round,line join=round,fill opacity=0.00,] (247.44, 16.63) -- (247.44, 12.67);

\draw[color=drawColor,line cap=round,line join=round,fill opacity=0.00,] (254.15, 16.63) -- (254.15, 12.67);

\draw[color=drawColor,line cap=round,line join=round,fill opacity=0.00,] (260.87, 16.63) -- (260.87, 12.67);

\draw[color=drawColor,line cap=round,line join=round,fill opacity=0.00,] (267.58, 16.63) -- (267.58, 12.67);

\draw[color=drawColor,line cap=round,line join=round,fill opacity=0.00,] (274.29, 16.63) -- (274.29, 12.67);

\draw[color=drawColor,line cap=round,line join=round,fill opacity=0.00,] (281.01, 16.63) -- (281.01, 12.67);

\draw[color=drawColor,line cap=round,line join=round,fill opacity=0.00,] (287.72, 16.63) -- (287.72, 12.67);

\draw[color=drawColor,line cap=round,line join=round,fill opacity=0.00,] (294.43, 16.63) -- (294.43, 12.67);

\draw[color=drawColor,line cap=round,line join=round,fill opacity=0.00,] (301.15, 16.63) -- (301.15, 12.67);

\draw[color=drawColor,line cap=round,line join=round,fill opacity=0.00,] (307.86, 16.63) -- (307.86, 12.67);

\draw[color=drawColor,line cap=round,line join=round,fill opacity=0.00,] (314.57, 16.63) -- (314.57, 12.67);

\draw[color=drawColor,line cap=round,line join=round,fill opacity=0.00,] (321.29, 16.63) -- (321.29, 12.67);

\draw[color=drawColor,line cap=round,line join=round,fill opacity=0.00,] (328.00, 16.63) -- (328.00, 12.67);

\draw[color=drawColor,line cap=round,line join=round,fill opacity=0.00,] (334.71, 16.63) -- (334.71, 12.67);

\draw[color=drawColor,line cap=round,line join=round,fill opacity=0.00,] (341.43, 16.63) -- (341.43, 12.67);

\draw[color=drawColor,line cap=round,line join=round,fill opacity=0.00,] (348.14, 16.63) -- (348.14, 12.67);

\draw[color=drawColor,line cap=round,line join=round,fill opacity=0.00,] (354.85, 16.63) -- (354.85, 12.67);

\draw[color=drawColor,line cap=round,line join=round,fill opacity=0.00,] (361.57, 16.63) -- (361.57, 12.67);

\draw[color=drawColor,line cap=round,line join=round,fill opacity=0.00,] (368.28, 16.63) -- (368.28, 12.67);

\draw[color=drawColor,line cap=round,line join=round,fill opacity=0.00,] (374.99, 16.63) -- (374.99, 12.67);

\draw[color=drawColor,line cap=round,line join=round,fill opacity=0.00,] (381.71, 16.63) -- (381.71, 12.67);

\draw[color=drawColor,line cap=round,line join=round,fill opacity=0.00,] (388.42, 16.63) -- (388.42, 12.67);

\draw[color=drawColor,line cap=round,line join=round,fill opacity=0.00,] (395.13, 16.63) -- (395.13, 12.67);

\draw[color=drawColor,line cap=round,line join=round,fill opacity=0.00,] (401.85, 16.63) -- (401.85, 12.67);

\draw[color=drawColor,line cap=round,line join=round,fill opacity=0.00,] (408.56, 16.63) -- (408.56, 12.67);

\draw[color=drawColor,line cap=round,line join=round,fill opacity=0.00,] (415.28, 16.63) -- (415.28, 12.67);

\draw[color=drawColor,line cap=round,line join=round,fill opacity=0.00,] (421.99, 16.63) -- (421.99, 12.67);

\draw[color=drawColor,line cap=round,line join=round,fill opacity=0.00,] (428.70, 16.63) -- (428.70, 12.67);

\draw[color=drawColor,line cap=round,line join=round,fill opacity=0.00,] (435.42, 16.63) -- (435.42, 12.67);

\draw[color=drawColor,line cap=round,line join=round,fill opacity=0.00,] (442.13, 16.63) -- (442.13, 12.67);

\draw[color=drawColor,line cap=round,line join=round,fill opacity=0.00,] (448.84, 16.63) -- (448.84, 12.67);

\draw[color=drawColor,line cap=round,line join=round,fill opacity=0.00,] (455.56, 16.63) -- (455.56, 12.67);

\draw[color=drawColor,line cap=round,line join=round,fill opacity=0.00,] (462.27, 16.63) -- (462.27, 12.67);

\draw[color=drawColor,line cap=round,line join=round,fill opacity=0.00,] (468.98, 16.63) -- (468.98, 12.67);

\node[color=drawColor,anchor=base,inner sep=0pt, outer sep=0pt, scale=  0.66] at ( 52.75,  0.79) {1949%
};

\node[color=drawColor,anchor=base,inner sep=0pt, outer sep=0pt, scale=  0.66] at ( 79.60,  0.79) {1953%
};

\node[color=drawColor,anchor=base,inner sep=0pt, outer sep=0pt, scale=  0.66] at (106.45,  0.79) {1957%
};

\node[color=drawColor,anchor=base,inner sep=0pt, outer sep=0pt, scale=  0.66] at (133.31,  0.79) {1961%
};

\node[color=drawColor,anchor=base,inner sep=0pt, outer sep=0pt, scale=  0.66] at (160.16,  0.79) {1965%
};

\node[color=drawColor,anchor=base,inner sep=0pt, outer sep=0pt, scale=  0.66] at (187.02,  0.79) {1969%
};

\node[color=drawColor,anchor=base,inner sep=0pt, outer sep=0pt, scale=  0.66] at (213.87,  0.79) {1973%
};

\node[color=drawColor,anchor=base,inner sep=0pt, outer sep=0pt, scale=  0.66] at (240.72,  0.79) {1977%
};

\node[color=drawColor,anchor=base,inner sep=0pt, outer sep=0pt, scale=  0.66] at (267.58,  0.79) {1981%
};

\node[color=drawColor,anchor=base,inner sep=0pt, outer sep=0pt, scale=  0.66] at (294.43,  0.79) {1985%
};

\node[color=drawColor,anchor=base,inner sep=0pt, outer sep=0pt, scale=  0.66] at (321.29,  0.79) {1989%
};

\node[color=drawColor,anchor=base,inner sep=0pt, outer sep=0pt, scale=  0.66] at (348.14,  0.79) {1993%
};

\node[color=drawColor,anchor=base,inner sep=0pt, outer sep=0pt, scale=  0.66] at (374.99,  0.79) {1997%
};

\node[color=drawColor,anchor=base,inner sep=0pt, outer sep=0pt, scale=  0.66] at (401.85,  0.79) {2001%
};

\node[color=drawColor,anchor=base,inner sep=0pt, outer sep=0pt, scale=  0.66] at (428.70,  0.79) {2005%
};

\node[color=drawColor,anchor=base,inner sep=0pt, outer sep=0pt, scale=  0.66] at (455.56,  0.79) {2009%
};

\draw[color=drawColor,line cap=round,line join=round,fill opacity=0.00,] ( 32.47, 23.14) -- ( 32.47,185.75);

\draw[color=drawColor,line cap=round,line join=round,fill opacity=0.00,] ( 32.47, 23.14) -- ( 28.51, 23.14);

\draw[color=drawColor,line cap=round,line join=round,fill opacity=0.00,] ( 32.47, 63.79) -- ( 28.51, 63.79);

\draw[color=drawColor,line cap=round,line join=round,fill opacity=0.00,] ( 32.47,104.44) -- ( 28.51,104.44);

\draw[color=drawColor,line cap=round,line join=round,fill opacity=0.00,] ( 32.47,145.10) -- ( 28.51,145.10);

\draw[color=drawColor,line cap=round,line join=round,fill opacity=0.00,] ( 32.47,185.75) -- ( 28.51,185.75);

\node[rotate= 90.00,color=drawColor,anchor=base,inner sep=0pt, outer sep=0pt, scale=  0.66] at ( 24.55, 23.14) {0%
};

\node[rotate= 90.00,color=drawColor,anchor=base,inner sep=0pt, outer sep=0pt, scale=  0.66] at ( 24.55, 63.79) {1000%
};

\node[rotate= 90.00,color=drawColor,anchor=base,inner sep=0pt, outer sep=0pt, scale=  0.66] at ( 24.55,104.44) {2000%
};

\node[rotate= 90.00,color=drawColor,anchor=base,inner sep=0pt, outer sep=0pt, scale=  0.66] at ( 24.55,145.10) {3000%
};

\node[rotate= 90.00,color=drawColor,anchor=base,inner sep=0pt, outer sep=0pt, scale=  0.66] at ( 24.55,185.75) {4000%
};

\draw[color=drawColor,line cap=round,line join=round,fill opacity=0.00,] ( 32.47, 16.63) --
	(489.26, 16.63) --
	(489.26,192.26) --
	( 32.47,192.26) --
	( 32.47, 16.63);
\end{scope}
\begin{scope}
\path[clip] ( 32.47, 16.63) rectangle (489.26,192.26);
\definecolor[named]{drawColor}{rgb}{1.00,0.00,0.00}

\draw[color=drawColor,line cap=round,line join=round,fill opacity=0.00,] ( 53.66, 99.41) -- ( 58.55, 78.67);

\draw[color=drawColor,line cap=round,line join=round,fill opacity=0.00,] ( 62.60, 72.40) -- ( 63.04, 72.06);

\draw[color=drawColor,line cap=round,line join=round,fill opacity=0.00,] ( 66.64, 73.57) -- ( 72.42,122.28);

\draw[color=drawColor,line cap=round,line join=round,fill opacity=0.00,] ( 73.37,122.28) -- ( 79.11, 75.85);

\draw[color=drawColor,line cap=round,line join=round,fill opacity=0.00,] ( 80.72, 68.13) -- ( 85.20, 52.90);

\draw[color=drawColor,line cap=round,line join=round,fill opacity=0.00,] ( 88.07, 52.65) -- ( 91.27, 59.11);

\draw[color=drawColor,line cap=round,line join=round,fill opacity=0.00,] ( 95.78, 65.51) -- ( 96.99, 66.76);

\draw[color=drawColor,line cap=round,line join=round,fill opacity=0.00,] (100.11, 73.55) -- (106.09,137.96);

\draw[color=drawColor,line cap=round,line join=round,fill opacity=0.00,] (107.27,138.03) -- (112.36,113.68);

\draw[color=drawColor,line cap=round,line join=round,fill opacity=0.00,] (113.68,105.88) -- (119.37, 62.65);

\draw[color=drawColor,line cap=round,line join=round,fill opacity=0.00,] (121.34, 62.40) -- (125.13, 71.96);

\draw[color=drawColor,line cap=round,line join=round,fill opacity=0.00,] (128.29, 72.06) -- (131.62, 65.02);

\draw[color=drawColor,line cap=round,line join=round,fill opacity=0.00,] (133.91, 65.36) -- (139.42,101.44);

\draw[color=drawColor,line cap=round,line join=round,fill opacity=0.00,] (140.53,101.43) -- (146.23, 56.97);

\draw[color=drawColor,line cap=round,line join=round,fill opacity=0.00,] (147.91, 56.83) -- (152.27, 70.84);

\draw[color=drawColor,line cap=round,line join=round,fill opacity=0.00,] (154.10, 78.53) -- (159.51,111.14);

\draw[color=drawColor,line cap=round,line join=round,fill opacity=0.00,] (160.72,111.12) -- (166.32, 71.76);

\draw[color=drawColor,line cap=round,line join=round,fill opacity=0.00,] (169.89, 65.28) -- (170.57, 64.70);

\draw[color=drawColor,line cap=round,line join=round,fill opacity=0.00,] (174.87, 65.88) -- (179.02, 78.02);

\draw[color=drawColor,line cap=round,line join=round,fill opacity=0.00,] (182.53, 85.04) -- (184.79, 88.37);

\draw[color=drawColor,line cap=round,line join=round,fill opacity=0.00,] (189.51, 94.73) -- (191.24, 96.87);

\draw[color=drawColor,line cap=round,line join=round,fill opacity=0.00,] (195.14, 96.25) -- (199.03, 86.04);

\draw[color=drawColor,line cap=round,line join=round,fill opacity=0.00,] (201.63, 78.56) -- (205.97, 64.74);

\draw[color=drawColor,line cap=round,line join=round,fill opacity=0.00,] (207.75, 64.87) -- (213.28,101.22);

\draw[color=drawColor,line cap=round,line join=round,fill opacity=0.00,] (214.79,101.28) -- (219.67, 80.83);

\draw[color=drawColor,line cap=round,line join=round,fill opacity=0.00,] (221.80, 80.74) -- (226.08, 94.06);

\draw[color=drawColor,line cap=round,line join=round,fill opacity=0.00,] (228.08, 93.95) -- (233.23, 68.51);

\draw[color=drawColor,line cap=round,line join=round,fill opacity=0.00,] (234.92, 60.77) -- (239.81, 40.11);

\draw[color=drawColor,line cap=round,line join=round,fill opacity=0.00,] (241.15, 40.19) -- (247.02, 94.95);

\draw[color=drawColor,line cap=round,line join=round,fill opacity=0.00,] (248.97,102.54) -- (252.62,111.26);

\draw[color=drawColor,line cap=round,line join=round,fill opacity=0.00,] (261.30,107.22) -- (267.14, 54.33);

\draw[color=drawColor,line cap=round,line join=round,fill opacity=0.00,] (268.18, 54.30) -- (273.69, 89.96);

\draw[color=drawColor,line cap=round,line join=round,fill opacity=0.00,] (275.01, 97.77) -- (280.29,126.62);

\draw[color=drawColor,line cap=round,line join=round,fill opacity=0.00,] (281.93,134.37) -- (286.80,154.68);

\draw[color=drawColor,line cap=round,line join=round,fill opacity=0.00,] (288.64,154.68) -- (293.52,134.17);

\draw[color=drawColor,line cap=round,line join=round,fill opacity=0.00,] (296.19,126.77) -- (299.39,120.27);

\draw[color=drawColor,line cap=round,line join=round,fill opacity=0.00,] (302.51,113.01) -- (306.49,102.21);

\draw[color=drawColor,line cap=round,line join=round,fill opacity=0.00,] (308.60, 94.60) -- (313.83, 67.05);

\draw[color=drawColor,line cap=round,line join=round,fill opacity=0.00,] (323.86, 60.76) -- (325.43, 58.92);

\draw[color=drawColor,line cap=round,line join=round,fill opacity=0.00,] (329.06, 59.72) -- (333.65, 76.20);

\draw[color=drawColor,line cap=round,line join=round,fill opacity=0.00,] (337.94, 77.72) -- (338.20, 77.54);

\draw[color=drawColor,line cap=round,line join=round,fill opacity=0.00,] (341.93, 79.17) -- (347.64,123.56);

\draw[color=drawColor,line cap=round,line join=round,fill opacity=0.00,] (348.61,123.55) -- (354.38, 75.61);

\draw[color=drawColor,line cap=round,line join=round,fill opacity=0.00,] (355.24, 75.62) -- (361.18,135.76);

\draw[color=drawColor,line cap=round,line join=round,fill opacity=0.00,] (362.07,135.77) -- (367.77, 91.51);

\draw[color=drawColor,line cap=round,line join=round,fill opacity=0.00,] (369.11, 91.45) -- (374.17,115.09);

\draw[color=drawColor,line cap=round,line join=round,fill opacity=0.00,] (375.77,115.07) -- (380.93, 89.13);

\draw[color=drawColor,line cap=round,line join=round,fill opacity=0.00,] (384.05, 82.05) -- (386.08, 79.27);

\draw[color=drawColor,line cap=round,line join=round,fill opacity=0.00,] (390.90, 72.99) -- (392.66, 70.80);

\draw[color=drawColor,line cap=round,line join=round,fill opacity=0.00,] (402.89, 61.18) -- (407.52, 44.17);

\draw[color=drawColor,line cap=round,line join=round,fill opacity=0.00,] (409.72, 44.13) -- (414.12, 58.56);

\draw[color=drawColor,line cap=round,line join=round,fill opacity=0.00,] (422.79, 64.80) -- (427.91, 89.71);

\draw[color=drawColor,line cap=round,line join=round,fill opacity=0.00,] (429.93, 89.83) -- (434.19, 76.73);

\draw[color=drawColor,line cap=round,line join=round,fill opacity=0.00,] (438.69, 70.73) -- (438.86, 70.62);

\draw[color=drawColor,line cap=round,line join=round,fill opacity=0.00,] ( 52.75,103.27) circle (  0.89);

\draw[color=drawColor,line cap=round,line join=round,fill opacity=0.00,] ( 59.46, 74.82) circle (  0.89);

\draw[color=drawColor,line cap=round,line join=round,fill opacity=0.00,] ( 66.17, 69.64) circle (  0.89);

\draw[color=drawColor,line cap=round,line join=round,fill opacity=0.00,] ( 72.89,126.21) circle (  0.89);

\draw[color=drawColor,line cap=round,line join=round,fill opacity=0.00,] ( 79.60, 71.92) circle (  0.89);

\draw[color=drawColor,line cap=round,line join=round,fill opacity=0.00,] ( 86.31, 49.10) circle (  0.89);

\draw[color=drawColor,line cap=round,line join=round,fill opacity=0.00,] ( 93.03, 62.66) circle (  0.89);

\draw[color=drawColor,line cap=round,line join=round,fill opacity=0.00,] ( 99.74, 69.61) circle (  0.89);

\draw[color=drawColor,line cap=round,line join=round,fill opacity=0.00,] (106.45,141.90) circle (  0.89);

\draw[color=drawColor,line cap=round,line join=round,fill opacity=0.00,] (113.17,109.81) circle (  0.89);

\draw[color=drawColor,line cap=round,line join=round,fill opacity=0.00,] (119.88, 58.72) circle (  0.89);

\draw[color=drawColor,line cap=round,line join=round,fill opacity=0.00,] (126.60, 75.64) circle (  0.89);

\draw[color=drawColor,line cap=round,line join=round,fill opacity=0.00,] (133.31, 61.44) circle (  0.89);

\draw[color=drawColor,line cap=round,line join=round,fill opacity=0.00,] (140.02,105.35) circle (  0.89);

\draw[color=drawColor,line cap=round,line join=round,fill opacity=0.00,] (146.74, 53.04) circle (  0.89);

\draw[color=drawColor,line cap=round,line join=round,fill opacity=0.00,] (153.45, 74.62) circle (  0.89);

\draw[color=drawColor,line cap=round,line join=round,fill opacity=0.00,] (160.16,115.04) circle (  0.89);

\draw[color=drawColor,line cap=round,line join=round,fill opacity=0.00,] (166.88, 67.84) circle (  0.89);

\draw[color=drawColor,line cap=round,line join=round,fill opacity=0.00,] (173.59, 62.14) circle (  0.89);

\draw[color=drawColor,line cap=round,line join=round,fill opacity=0.00,] (180.30, 81.77) circle (  0.89);

\draw[color=drawColor,line cap=round,line join=round,fill opacity=0.00,] (187.02, 91.65) circle (  0.89);

\draw[color=drawColor,line cap=round,line join=round,fill opacity=0.00,] (193.73, 99.95) circle (  0.89);

\draw[color=drawColor,line cap=round,line join=round,fill opacity=0.00,] (200.44, 82.34) circle (  0.89);

\draw[color=drawColor,line cap=round,line join=round,fill opacity=0.00,] (207.16, 60.96) circle (  0.89);

\draw[color=drawColor,line cap=round,line join=round,fill opacity=0.00,] (213.87,105.13) circle (  0.89);

\draw[color=drawColor,line cap=round,line join=round,fill opacity=0.00,] (220.58, 76.97) circle (  0.89);

\draw[color=drawColor,line cap=round,line join=round,fill opacity=0.00,] (227.30, 97.83) circle (  0.89);

\draw[color=drawColor,line cap=round,line join=round,fill opacity=0.00,] (234.01, 64.62) circle (  0.89);

\draw[color=drawColor,line cap=round,line join=round,fill opacity=0.00,] (240.72, 36.26) circle (  0.89);

\draw[color=drawColor,line cap=round,line join=round,fill opacity=0.00,] (247.44, 98.88) circle (  0.89);

\draw[color=drawColor,line cap=round,line join=round,fill opacity=0.00,] (254.15,114.91) circle (  0.89);

\draw[color=drawColor,line cap=round,line join=round,fill opacity=0.00,] (260.87,111.15) circle (  0.89);

\draw[color=drawColor,line cap=round,line join=round,fill opacity=0.00,] (267.58, 50.39) circle (  0.89);

\draw[color=drawColor,line cap=round,line join=round,fill opacity=0.00,] (274.29, 93.87) circle (  0.89);

\draw[color=drawColor,line cap=round,line join=round,fill opacity=0.00,] (281.01,130.52) circle (  0.89);

\draw[color=drawColor,line cap=round,line join=round,fill opacity=0.00,] (287.72,158.53) circle (  0.89);

\draw[color=drawColor,line cap=round,line join=round,fill opacity=0.00,] (294.43,130.32) circle (  0.89);

\draw[color=drawColor,line cap=round,line join=round,fill opacity=0.00,] (301.15,116.72) circle (  0.89);

\draw[color=drawColor,line cap=round,line join=round,fill opacity=0.00,] (307.86, 98.49) circle (  0.89);

\draw[color=drawColor,line cap=round,line join=round,fill opacity=0.00,] (314.57, 63.16) circle (  0.89);

\draw[color=drawColor,line cap=round,line join=round,fill opacity=0.00,] (321.29, 63.77) circle (  0.89);

\draw[color=drawColor,line cap=round,line join=round,fill opacity=0.00,] (328.00, 55.91) circle (  0.89);

\draw[color=drawColor,line cap=round,line join=round,fill opacity=0.00,] (334.71, 80.02) circle (  0.89);

\draw[color=drawColor,line cap=round,line join=round,fill opacity=0.00,] (341.43, 75.25) circle (  0.89);

\draw[color=drawColor,line cap=round,line join=round,fill opacity=0.00,] (348.14,127.49) circle (  0.89);

\draw[color=drawColor,line cap=round,line join=round,fill opacity=0.00,] (354.85, 71.68) circle (  0.89);

\draw[color=drawColor,line cap=round,line join=round,fill opacity=0.00,] (361.57,139.70) circle (  0.89);

\draw[color=drawColor,line cap=round,line join=round,fill opacity=0.00,] (368.28, 87.58) circle (  0.89);

\draw[color=drawColor,line cap=round,line join=round,fill opacity=0.00,] (374.99,118.96) circle (  0.89);

\draw[color=drawColor,line cap=round,line join=round,fill opacity=0.00,] (381.71, 85.24) circle (  0.89);

\draw[color=drawColor,line cap=round,line join=round,fill opacity=0.00,] (388.42, 76.08) circle (  0.89);

\draw[color=drawColor,line cap=round,line join=round,fill opacity=0.00,] (395.13, 67.71) circle (  0.89);

\draw[color=drawColor,line cap=round,line join=round,fill opacity=0.00,] (401.85, 65.01) circle (  0.89);

\draw[color=drawColor,line cap=round,line join=round,fill opacity=0.00,] (408.56, 40.34) circle (  0.89);

\draw[color=drawColor,line cap=round,line join=round,fill opacity=0.00,] (415.28, 62.35) circle (  0.89);

\draw[color=drawColor,line cap=round,line join=round,fill opacity=0.00,] (421.99, 60.92) circle (  0.89);

\draw[color=drawColor,line cap=round,line join=round,fill opacity=0.00,] (428.70, 93.59) circle (  0.89);

\draw[color=drawColor,line cap=round,line join=round,fill opacity=0.00,] (435.42, 72.97) circle (  0.89);

\draw[color=drawColor,line cap=round,line join=round,fill opacity=0.00,] (442.13, 68.38) circle (  0.89);
\end{scope}
\begin{scope}
\path[clip] (  0.00,  0.00) rectangle (505.89,650.43);
\definecolor[named]{drawColor}{rgb}{0.00,0.00,0.00}

\node[color=drawColor,anchor=base,inner sep=0pt, outer sep=0pt, scale=  1.00] at (260.87,200.18) {(c) RPSS = 0.25 MC = 0.46%
};
\end{scope}
\end{tikzpicture}

   \caption{Sample ensemble forecasts for (a) Apr1, (b) Feb1 and (c) Nov1 seasonal flow volumes at Gunnison River Near Grand Junction (Site 6). Boxplots extend to the 5th and 95th percentiles of each ensemble. MC stands for the median correlation, the correlation of the median of the ensembles withe the historical record.}
   \label{fig:box-seas}
\end{figure*}

\begin{figure*}[htbp] %  figure placement: here, top, bottom, or page
   \centering
   %\includegraphics[width=.9\textwidth]{boxplots.pdf}\\
   % Created by tikzDevice version 0.6.1 on 2011-07-20 15:57:49
% !TEX encoding = UTF-8 Unicode
\begin{tikzpicture}[x=1pt,y=1pt]
\definecolor[named]{drawColor}{rgb}{0.00,0.00,0.00}
\definecolor[named]{fillColor}{rgb}{1.00,1.00,1.00}
\fill[color=fillColor,] (0,0) rectangle (469.75,614.29);
\begin{scope}
\path[clip] ( 32.47,426.16) rectangle (453.12,589.74);
\definecolor[named]{drawColor}{rgb}{0.00,0.00,0.00}

\draw[color=drawColor,line width= 1.2pt,line join=round,fill opacity=0.00,] ( 48.67,509.17) -- ( 53.62,509.17);

\draw[color=drawColor,dash pattern=on 4pt off 4pt ,line cap=round,line join=round,fill opacity=0.00,] ( 51.14,488.18) -- ( 51.14,503.88);

\draw[color=drawColor,dash pattern=on 4pt off 4pt ,line cap=round,line join=round,fill opacity=0.00,] ( 51.14,548.60) -- ( 51.14,514.51);

\draw[color=drawColor,line cap=round,line join=round,fill opacity=0.00,] ( 49.91,488.18) -- ( 52.38,488.18);

\draw[color=drawColor,line cap=round,line join=round,fill opacity=0.00,] ( 49.91,548.60) -- ( 52.38,548.60);

\draw[color=drawColor,line cap=round,line join=round,fill opacity=0.00,] ( 48.67,503.88) --
	( 53.62,503.88) --
	( 53.62,514.51) --
	( 48.67,514.51) --
	( 48.67,503.88);

\draw[color=drawColor,line width= 1.2pt,line join=round,fill opacity=0.00,] ( 54.85,480.53) -- ( 59.80,480.53);

\draw[color=drawColor,dash pattern=on 4pt off 4pt ,line cap=round,line join=round,fill opacity=0.00,] ( 57.33,467.90) -- ( 57.33,472.17);

\draw[color=drawColor,dash pattern=on 4pt off 4pt ,line cap=round,line join=round,fill opacity=0.00,] ( 57.33,501.17) -- ( 57.33,486.51);

\draw[color=drawColor,line cap=round,line join=round,fill opacity=0.00,] ( 56.09,467.90) -- ( 58.56,467.90);

\draw[color=drawColor,line cap=round,line join=round,fill opacity=0.00,] ( 56.09,501.17) -- ( 58.56,501.17);

\draw[color=drawColor,line cap=round,line join=round,fill opacity=0.00,] ( 54.85,472.17) --
	( 59.80,472.17) --
	( 59.80,486.51) --
	( 54.85,486.51) --
	( 54.85,472.17);

\draw[color=drawColor,line width= 1.2pt,line join=round,fill opacity=0.00,] ( 61.03,474.45) -- ( 65.98,474.45);

\draw[color=drawColor,dash pattern=on 4pt off 4pt ,line cap=round,line join=round,fill opacity=0.00,] ( 63.51,463.64) -- ( 63.51,469.70);

\draw[color=drawColor,dash pattern=on 4pt off 4pt ,line cap=round,line join=round,fill opacity=0.00,] ( 63.51,495.65) -- ( 63.51,480.88);

\draw[color=drawColor,line cap=round,line join=round,fill opacity=0.00,] ( 62.27,463.64) -- ( 64.74,463.64);

\draw[color=drawColor,line cap=round,line join=round,fill opacity=0.00,] ( 62.27,495.65) -- ( 64.74,495.65);

\draw[color=drawColor,line cap=round,line join=round,fill opacity=0.00,] ( 61.03,469.70) --
	( 65.98,469.70) --
	( 65.98,480.88) --
	( 61.03,480.88) --
	( 61.03,469.70);

\draw[color=drawColor,line width= 1.2pt,line join=round,fill opacity=0.00,] ( 67.22,543.57) -- ( 72.16,543.57);

\draw[color=drawColor,dash pattern=on 4pt off 4pt ,line cap=round,line join=round,fill opacity=0.00,] ( 69.69,518.96) -- ( 69.69,534.45);

\draw[color=drawColor,dash pattern=on 4pt off 4pt ,line cap=round,line join=round,fill opacity=0.00,] ( 69.69,583.68) -- ( 69.69,553.55);

\draw[color=drawColor,line cap=round,line join=round,fill opacity=0.00,] ( 68.45,518.96) -- ( 70.93,518.96);

\draw[color=drawColor,line cap=round,line join=round,fill opacity=0.00,] ( 68.45,583.68) -- ( 70.93,583.68);

\draw[color=drawColor,line cap=round,line join=round,fill opacity=0.00,] ( 67.22,534.45) --
	( 72.16,534.45) --
	( 72.16,553.55) --
	( 67.22,553.55) --
	( 67.22,534.45);

\draw[color=drawColor,line width= 1.2pt,line join=round,fill opacity=0.00,] ( 73.40,465.97) -- ( 78.35,465.97);

\draw[color=drawColor,dash pattern=on 4pt off 4pt ,line cap=round,line join=round,fill opacity=0.00,] ( 75.87,456.19) -- ( 75.87,463.63);

\draw[color=drawColor,dash pattern=on 4pt off 4pt ,line cap=round,line join=round,fill opacity=0.00,] ( 75.87,488.99) -- ( 75.87,474.84);

\draw[color=drawColor,line cap=round,line join=round,fill opacity=0.00,] ( 74.64,456.19) -- ( 77.11,456.19);

\draw[color=drawColor,line cap=round,line join=round,fill opacity=0.00,] ( 74.64,488.99) -- ( 77.11,488.99);

\draw[color=drawColor,line cap=round,line join=round,fill opacity=0.00,] ( 73.40,463.63) --
	( 78.35,463.63) --
	( 78.35,474.84) --
	( 73.40,474.84) --
	( 73.40,463.63);

\draw[color=drawColor,line width= 1.2pt,line join=round,fill opacity=0.00,] ( 79.58,463.52) -- ( 84.53,463.52);

\draw[color=drawColor,dash pattern=on 4pt off 4pt ,line cap=round,line join=round,fill opacity=0.00,] ( 82.05,454.70) -- ( 82.05,459.06);

\draw[color=drawColor,dash pattern=on 4pt off 4pt ,line cap=round,line join=round,fill opacity=0.00,] ( 82.05,487.46) -- ( 82.05,469.13);

\draw[color=drawColor,line cap=round,line join=round,fill opacity=0.00,] ( 80.82,454.70) -- ( 83.29,454.70);

\draw[color=drawColor,line cap=round,line join=round,fill opacity=0.00,] ( 80.82,487.46) -- ( 83.29,487.46);

\draw[color=drawColor,line cap=round,line join=round,fill opacity=0.00,] ( 79.58,459.06) --
	( 84.53,459.06) --
	( 84.53,469.13) --
	( 79.58,469.13) --
	( 79.58,459.06);

\draw[color=drawColor,line width= 1.2pt,line join=round,fill opacity=0.00,] ( 85.76,462.49) -- ( 90.71,462.49);

\draw[color=drawColor,dash pattern=on 4pt off 4pt ,line cap=round,line join=round,fill opacity=0.00,] ( 88.24,454.69) -- ( 88.24,458.02);

\draw[color=drawColor,dash pattern=on 4pt off 4pt ,line cap=round,line join=round,fill opacity=0.00,] ( 88.24,473.47) -- ( 88.24,466.87);

\draw[color=drawColor,line cap=round,line join=round,fill opacity=0.00,] ( 87.00,454.69) -- ( 89.47,454.69);

\draw[color=drawColor,line cap=round,line join=round,fill opacity=0.00,] ( 87.00,473.47) -- ( 89.47,473.47);

\draw[color=drawColor,line cap=round,line join=round,fill opacity=0.00,] ( 85.76,458.02) --
	( 90.71,458.02) --
	( 90.71,466.87) --
	( 85.76,466.87) --
	( 85.76,458.02);

\draw[color=drawColor,line width= 1.2pt,line join=round,fill opacity=0.00,] ( 91.95,499.65) -- ( 96.89,499.65);

\draw[color=drawColor,dash pattern=on 4pt off 4pt ,line cap=round,line join=round,fill opacity=0.00,] ( 94.42,470.64) -- ( 94.42,486.04);

\draw[color=drawColor,dash pattern=on 4pt off 4pt ,line cap=round,line join=round,fill opacity=0.00,] ( 94.42,516.83) -- ( 94.42,508.96);

\draw[color=drawColor,line cap=round,line join=round,fill opacity=0.00,] ( 93.18,470.64) -- ( 95.66,470.64);

\draw[color=drawColor,line cap=round,line join=round,fill opacity=0.00,] ( 93.18,516.83) -- ( 95.66,516.83);

\draw[color=drawColor,line cap=round,line join=round,fill opacity=0.00,] ( 91.95,486.04) --
	( 96.89,486.04) --
	( 96.89,508.96) --
	( 91.95,508.96) --
	( 91.95,486.04);

\draw[color=drawColor,line width= 1.2pt,line join=round,fill opacity=0.00,] ( 98.13,503.45) -- (103.08,503.45);

\draw[color=drawColor,dash pattern=on 4pt off 4pt ,line cap=round,line join=round,fill opacity=0.00,] (100.60,485.61) -- (100.60,500.64);

\draw[color=drawColor,dash pattern=on 4pt off 4pt ,line cap=round,line join=round,fill opacity=0.00,] (100.60,511.74) -- (100.60,506.92);

\draw[color=drawColor,line cap=round,line join=round,fill opacity=0.00,] ( 99.37,485.61) -- (101.84,485.61);

\draw[color=drawColor,line cap=round,line join=round,fill opacity=0.00,] ( 99.37,511.74) -- (101.84,511.74);

\draw[color=drawColor,line cap=round,line join=round,fill opacity=0.00,] ( 98.13,500.64) --
	(103.08,500.64) --
	(103.08,506.92) --
	( 98.13,506.92) --
	( 98.13,500.64);

\draw[color=drawColor,line width= 1.2pt,line join=round,fill opacity=0.00,] (104.31,498.87) -- (109.26,498.87);

\draw[color=drawColor,dash pattern=on 4pt off 4pt ,line cap=round,line join=round,fill opacity=0.00,] (106.78,470.53) -- (106.78,485.23);

\draw[color=drawColor,dash pattern=on 4pt off 4pt ,line cap=round,line join=round,fill opacity=0.00,] (106.78,515.43) -- (106.78,507.32);

\draw[color=drawColor,line cap=round,line join=round,fill opacity=0.00,] (105.55,470.53) -- (108.02,470.53);

\draw[color=drawColor,line cap=round,line join=round,fill opacity=0.00,] (105.55,515.43) -- (108.02,515.43);

\draw[color=drawColor,line cap=round,line join=round,fill opacity=0.00,] (104.31,485.23) --
	(109.26,485.23) --
	(109.26,507.32) --
	(104.31,507.32) --
	(104.31,485.23);

\draw[color=drawColor,line width= 1.2pt,line join=round,fill opacity=0.00,] (110.49,462.83) -- (115.44,462.83);

\draw[color=drawColor,dash pattern=on 4pt off 4pt ,line cap=round,line join=round,fill opacity=0.00,] (112.97,454.51) -- (112.97,457.44);

\draw[color=drawColor,dash pattern=on 4pt off 4pt ,line cap=round,line join=round,fill opacity=0.00,] (112.97,486.30) -- (112.97,468.00);

\draw[color=drawColor,line cap=round,line join=round,fill opacity=0.00,] (111.73,454.51) -- (114.20,454.51);

\draw[color=drawColor,line cap=round,line join=round,fill opacity=0.00,] (111.73,486.30) -- (114.20,486.30);

\draw[color=drawColor,line cap=round,line join=round,fill opacity=0.00,] (110.49,457.44) --
	(115.44,457.44) --
	(115.44,468.00) --
	(110.49,468.00) --
	(110.49,457.44);

\draw[color=drawColor,line width= 1.2pt,line join=round,fill opacity=0.00,] (116.68,467.99) -- (121.62,467.99);

\draw[color=drawColor,dash pattern=on 4pt off 4pt ,line cap=round,line join=round,fill opacity=0.00,] (119.15,455.28) -- (119.15,462.54);

\draw[color=drawColor,dash pattern=on 4pt off 4pt ,line cap=round,line join=round,fill opacity=0.00,] (119.15,489.13) -- (119.15,473.30);

\draw[color=drawColor,line cap=round,line join=round,fill opacity=0.00,] (117.91,455.28) -- (120.39,455.28);

\draw[color=drawColor,line cap=round,line join=round,fill opacity=0.00,] (117.91,489.13) -- (120.39,489.13);

\draw[color=drawColor,line cap=round,line join=round,fill opacity=0.00,] (116.68,462.54) --
	(121.62,462.54) --
	(121.62,473.30) --
	(116.68,473.30) --
	(116.68,462.54);

\draw[color=drawColor,line width= 1.2pt,line join=round,fill opacity=0.00,] (122.86,453.97) -- (127.80,453.97);

\draw[color=drawColor,dash pattern=on 4pt off 4pt ,line cap=round,line join=round,fill opacity=0.00,] (125.33,444.56) -- (125.33,447.93);

\draw[color=drawColor,dash pattern=on 4pt off 4pt ,line cap=round,line join=round,fill opacity=0.00,] (125.33,466.03) -- (125.33,461.07);

\draw[color=drawColor,line cap=round,line join=round,fill opacity=0.00,] (124.10,444.56) -- (126.57,444.56);

\draw[color=drawColor,line cap=round,line join=round,fill opacity=0.00,] (124.10,466.03) -- (126.57,466.03);

\draw[color=drawColor,line cap=round,line join=round,fill opacity=0.00,] (122.86,447.93) --
	(127.80,447.93) --
	(127.80,461.07) --
	(122.86,461.07) --
	(122.86,447.93);

\draw[color=drawColor,line width= 1.2pt,line join=round,fill opacity=0.00,] (129.04,512.00) -- (133.99,512.00);

\draw[color=drawColor,dash pattern=on 4pt off 4pt ,line cap=round,line join=round,fill opacity=0.00,] (131.51,489.77) -- (131.51,507.81);

\draw[color=drawColor,dash pattern=on 4pt off 4pt ,line cap=round,line join=round,fill opacity=0.00,] (131.51,546.36) -- (131.51,516.20);

\draw[color=drawColor,line cap=round,line join=round,fill opacity=0.00,] (130.28,489.77) -- (132.75,489.77);

\draw[color=drawColor,line cap=round,line join=round,fill opacity=0.00,] (130.28,546.36) -- (132.75,546.36);

\draw[color=drawColor,line cap=round,line join=round,fill opacity=0.00,] (129.04,507.81) --
	(133.99,507.81) --
	(133.99,516.20) --
	(129.04,516.20) --
	(129.04,507.81);

\draw[color=drawColor,line width= 1.2pt,line join=round,fill opacity=0.00,] (135.22,461.13) -- (140.17,461.13);

\draw[color=drawColor,dash pattern=on 4pt off 4pt ,line cap=round,line join=round,fill opacity=0.00,] (137.70,448.44) -- (137.70,455.55);

\draw[color=drawColor,dash pattern=on 4pt off 4pt ,line cap=round,line join=round,fill opacity=0.00,] (137.70,470.44) -- (137.70,465.59);

\draw[color=drawColor,line cap=round,line join=round,fill opacity=0.00,] (136.46,448.44) -- (138.93,448.44);

\draw[color=drawColor,line cap=round,line join=round,fill opacity=0.00,] (136.46,470.44) -- (138.93,470.44);

\draw[color=drawColor,line cap=round,line join=round,fill opacity=0.00,] (135.22,455.55) --
	(140.17,455.55) --
	(140.17,465.59) --
	(135.22,465.59) --
	(135.22,455.55);

\draw[color=drawColor,line width= 1.2pt,line join=round,fill opacity=0.00,] (141.41,461.02) -- (146.35,461.02);

\draw[color=drawColor,dash pattern=on 4pt off 4pt ,line cap=round,line join=round,fill opacity=0.00,] (143.88,447.36) -- (143.88,454.90);

\draw[color=drawColor,dash pattern=on 4pt off 4pt ,line cap=round,line join=round,fill opacity=0.00,] (143.88,469.81) -- (143.88,465.00);

\draw[color=drawColor,line cap=round,line join=round,fill opacity=0.00,] (142.64,447.36) -- (145.12,447.36);

\draw[color=drawColor,line cap=round,line join=round,fill opacity=0.00,] (142.64,469.81) -- (145.12,469.81);

\draw[color=drawColor,line cap=round,line join=round,fill opacity=0.00,] (141.41,454.90) --
	(146.35,454.90) --
	(146.35,465.00) --
	(141.41,465.00) --
	(141.41,454.90);

\draw[color=drawColor,line width= 1.2pt,line join=round,fill opacity=0.00,] (147.59,511.80) -- (152.53,511.80);

\draw[color=drawColor,dash pattern=on 4pt off 4pt ,line cap=round,line join=round,fill opacity=0.00,] (150.06,488.89) -- (150.06,507.85);

\draw[color=drawColor,dash pattern=on 4pt off 4pt ,line cap=round,line join=round,fill opacity=0.00,] (150.06,552.35) -- (150.06,516.70);

\draw[color=drawColor,line cap=round,line join=round,fill opacity=0.00,] (148.82,488.89) -- (151.30,488.89);

\draw[color=drawColor,line cap=round,line join=round,fill opacity=0.00,] (148.82,552.35) -- (151.30,552.35);

\draw[color=drawColor,line cap=round,line join=round,fill opacity=0.00,] (147.59,507.85) --
	(152.53,507.85) --
	(152.53,516.70) --
	(147.59,516.70) --
	(147.59,507.85);

\draw[color=drawColor,line width= 1.2pt,line join=round,fill opacity=0.00,] (153.77,472.91) -- (158.72,472.91);

\draw[color=drawColor,dash pattern=on 4pt off 4pt ,line cap=round,line join=round,fill opacity=0.00,] (156.24,461.70) -- (156.24,469.36);

\draw[color=drawColor,dash pattern=on 4pt off 4pt ,line cap=round,line join=round,fill opacity=0.00,] (156.24,498.15) -- (156.24,477.10);

\draw[color=drawColor,line cap=round,line join=round,fill opacity=0.00,] (155.01,461.70) -- (157.48,461.70);

\draw[color=drawColor,line cap=round,line join=round,fill opacity=0.00,] (155.01,498.15) -- (157.48,498.15);

\draw[color=drawColor,line cap=round,line join=round,fill opacity=0.00,] (153.77,469.36) --
	(158.72,469.36) --
	(158.72,477.10) --
	(153.77,477.10) --
	(153.77,469.36);

\draw[color=drawColor,line width= 1.2pt,line join=round,fill opacity=0.00,] (159.95,477.30) -- (164.90,477.30);

\draw[color=drawColor,dash pattern=on 4pt off 4pt ,line cap=round,line join=round,fill opacity=0.00,] (162.43,466.48) -- (162.43,471.78);

\draw[color=drawColor,dash pattern=on 4pt off 4pt ,line cap=round,line join=round,fill opacity=0.00,] (162.43,500.17) -- (162.43,482.06);

\draw[color=drawColor,line cap=round,line join=round,fill opacity=0.00,] (161.19,466.48) -- (163.66,466.48);

\draw[color=drawColor,line cap=round,line join=round,fill opacity=0.00,] (161.19,500.17) -- (163.66,500.17);

\draw[color=drawColor,line cap=round,line join=round,fill opacity=0.00,] (159.95,471.78) --
	(164.90,471.78) --
	(164.90,482.06) --
	(159.95,482.06) --
	(159.95,471.78);

\draw[color=drawColor,line width= 1.2pt,line join=round,fill opacity=0.00,] (166.14,477.12) -- (171.08,477.12);

\draw[color=drawColor,dash pattern=on 4pt off 4pt ,line cap=round,line join=round,fill opacity=0.00,] (168.61,466.72) -- (168.61,470.75);

\draw[color=drawColor,dash pattern=on 4pt off 4pt ,line cap=round,line join=round,fill opacity=0.00,] (168.61,499.70) -- (168.61,484.60);

\draw[color=drawColor,line cap=round,line join=round,fill opacity=0.00,] (167.37,466.72) -- (169.85,466.72);

\draw[color=drawColor,line cap=round,line join=round,fill opacity=0.00,] (167.37,499.70) -- (169.85,499.70);

\draw[color=drawColor,line cap=round,line join=round,fill opacity=0.00,] (166.14,470.75) --
	(171.08,470.75) --
	(171.08,484.60) --
	(166.14,484.60) --
	(166.14,470.75);

\draw[color=drawColor,line width= 1.2pt,line join=round,fill opacity=0.00,] (172.32,511.98) -- (177.26,511.98);

\draw[color=drawColor,dash pattern=on 4pt off 4pt ,line cap=round,line join=round,fill opacity=0.00,] (174.79,492.43) -- (174.79,508.50);

\draw[color=drawColor,dash pattern=on 4pt off 4pt ,line cap=round,line join=round,fill opacity=0.00,] (174.79,549.11) -- (174.79,517.34);

\draw[color=drawColor,line cap=round,line join=round,fill opacity=0.00,] (173.55,492.43) -- (176.03,492.43);

\draw[color=drawColor,line cap=round,line join=round,fill opacity=0.00,] (173.55,549.11) -- (176.03,549.11);

\draw[color=drawColor,line cap=round,line join=round,fill opacity=0.00,] (172.32,508.50) --
	(177.26,508.50) --
	(177.26,517.34) --
	(172.32,517.34) --
	(172.32,508.50);

\draw[color=drawColor,line width= 1.2pt,line join=round,fill opacity=0.00,] (178.50,480.63) -- (183.45,480.63);

\draw[color=drawColor,dash pattern=on 4pt off 4pt ,line cap=round,line join=round,fill opacity=0.00,] (180.97,468.44) -- (180.97,472.78);

\draw[color=drawColor,dash pattern=on 4pt off 4pt ,line cap=round,line join=round,fill opacity=0.00,] (180.97,493.20) -- (180.97,485.20);

\draw[color=drawColor,line cap=round,line join=round,fill opacity=0.00,] (179.74,468.44) -- (182.21,468.44);

\draw[color=drawColor,line cap=round,line join=round,fill opacity=0.00,] (179.74,493.20) -- (182.21,493.20);

\draw[color=drawColor,line cap=round,line join=round,fill opacity=0.00,] (178.50,472.78) --
	(183.45,472.78) --
	(183.45,485.20) --
	(178.50,485.20) --
	(178.50,472.78);

\draw[color=drawColor,line width= 1.2pt,line join=round,fill opacity=0.00,] (184.68,488.25) -- (189.63,488.25);

\draw[color=drawColor,dash pattern=on 4pt off 4pt ,line cap=round,line join=round,fill opacity=0.00,] (187.16,469.67) -- (187.16,480.16);

\draw[color=drawColor,dash pattern=on 4pt off 4pt ,line cap=round,line join=round,fill opacity=0.00,] (187.16,508.76) -- (187.16,501.08);

\draw[color=drawColor,line cap=round,line join=round,fill opacity=0.00,] (185.92,469.67) -- (188.39,469.67);

\draw[color=drawColor,line cap=round,line join=round,fill opacity=0.00,] (185.92,508.76) -- (188.39,508.76);

\draw[color=drawColor,line cap=round,line join=round,fill opacity=0.00,] (184.68,480.16) --
	(189.63,480.16) --
	(189.63,501.08) --
	(184.68,501.08) --
	(184.68,480.16);

\draw[color=drawColor,line width= 1.2pt,line join=round,fill opacity=0.00,] (190.87,486.64) -- (195.81,486.64);

\draw[color=drawColor,dash pattern=on 4pt off 4pt ,line cap=round,line join=round,fill opacity=0.00,] (193.34,469.95) -- (193.34,479.94);

\draw[color=drawColor,dash pattern=on 4pt off 4pt ,line cap=round,line join=round,fill opacity=0.00,] (193.34,504.47) -- (193.34,494.74);

\draw[color=drawColor,line cap=round,line join=round,fill opacity=0.00,] (192.10,469.95) -- (194.57,469.95);

\draw[color=drawColor,line cap=round,line join=round,fill opacity=0.00,] (192.10,504.47) -- (194.57,504.47);

\draw[color=drawColor,line cap=round,line join=round,fill opacity=0.00,] (190.87,479.94) --
	(195.81,479.94) --
	(195.81,494.74) --
	(190.87,494.74) --
	(190.87,479.94);

\draw[color=drawColor,line width= 1.2pt,line join=round,fill opacity=0.00,] (197.05,493.75) -- (201.99,493.75);

\draw[color=drawColor,dash pattern=on 4pt off 4pt ,line cap=round,line join=round,fill opacity=0.00,] (199.52,470.68) -- (199.52,483.16);

\draw[color=drawColor,dash pattern=on 4pt off 4pt ,line cap=round,line join=round,fill opacity=0.00,] (199.52,509.02) -- (199.52,501.19);

\draw[color=drawColor,line cap=round,line join=round,fill opacity=0.00,] (198.28,470.68) -- (200.76,470.68);

\draw[color=drawColor,line cap=round,line join=round,fill opacity=0.00,] (198.28,509.02) -- (200.76,509.02);

\draw[color=drawColor,line cap=round,line join=round,fill opacity=0.00,] (197.05,483.16) --
	(201.99,483.16) --
	(201.99,501.19) --
	(197.05,501.19) --
	(197.05,483.16);

\draw[color=drawColor,line width= 1.2pt,line join=round,fill opacity=0.00,] (203.23,474.70) -- (208.18,474.70);

\draw[color=drawColor,dash pattern=on 4pt off 4pt ,line cap=round,line join=round,fill opacity=0.00,] (205.70,464.76) -- (205.70,469.29);

\draw[color=drawColor,dash pattern=on 4pt off 4pt ,line cap=round,line join=round,fill opacity=0.00,] (205.70,498.29) -- (205.70,480.26);

\draw[color=drawColor,line cap=round,line join=round,fill opacity=0.00,] (204.47,464.76) -- (206.94,464.76);

\draw[color=drawColor,line cap=round,line join=round,fill opacity=0.00,] (204.47,498.29) -- (206.94,498.29);

\draw[color=drawColor,line cap=round,line join=round,fill opacity=0.00,] (203.23,469.29) --
	(208.18,469.29) --
	(208.18,480.26) --
	(203.23,480.26) --
	(203.23,469.29);

\draw[color=drawColor,line width= 1.2pt,line join=round,fill opacity=0.00,] (209.41,496.42) -- (214.36,496.42);

\draw[color=drawColor,dash pattern=on 4pt off 4pt ,line cap=round,line join=round,fill opacity=0.00,] (211.89,472.66) -- (211.89,483.62);

\draw[color=drawColor,dash pattern=on 4pt off 4pt ,line cap=round,line join=round,fill opacity=0.00,] (211.89,508.90) -- (211.89,501.42);

\draw[color=drawColor,line cap=round,line join=round,fill opacity=0.00,] (210.65,472.66) -- (213.12,472.66);

\draw[color=drawColor,line cap=round,line join=round,fill opacity=0.00,] (210.65,508.90) -- (213.12,508.90);

\draw[color=drawColor,line cap=round,line join=round,fill opacity=0.00,] (209.41,483.62) --
	(214.36,483.62) --
	(214.36,501.42) --
	(209.41,501.42) --
	(209.41,483.62);

\draw[color=drawColor,line width= 1.2pt,line join=round,fill opacity=0.00,] (215.59,474.66) -- (220.54,474.66);

\draw[color=drawColor,dash pattern=on 4pt off 4pt ,line cap=round,line join=round,fill opacity=0.00,] (218.07,464.08) -- (218.07,470.33);

\draw[color=drawColor,dash pattern=on 4pt off 4pt ,line cap=round,line join=round,fill opacity=0.00,] (218.07,499.17) -- (218.07,479.35);

\draw[color=drawColor,line cap=round,line join=round,fill opacity=0.00,] (216.83,464.08) -- (219.30,464.08);

\draw[color=drawColor,line cap=round,line join=round,fill opacity=0.00,] (216.83,499.17) -- (219.30,499.17);

\draw[color=drawColor,line cap=round,line join=round,fill opacity=0.00,] (215.59,470.33) --
	(220.54,470.33) --
	(220.54,479.35) --
	(215.59,479.35) --
	(215.59,470.33);

\draw[color=drawColor,line width= 1.2pt,line join=round,fill opacity=0.00,] (221.78,439.32) -- (226.72,439.32);

\draw[color=drawColor,dash pattern=on 4pt off 4pt ,line cap=round,line join=round,fill opacity=0.00,] (224.25,432.22) -- (224.25,436.52);

\draw[color=drawColor,dash pattern=on 4pt off 4pt ,line cap=round,line join=round,fill opacity=0.00,] (224.25,448.80) -- (224.25,442.31);

\draw[color=drawColor,line cap=round,line join=round,fill opacity=0.00,] (223.01,432.22) -- (225.49,432.22);

\draw[color=drawColor,line cap=round,line join=round,fill opacity=0.00,] (223.01,448.80) -- (225.49,448.80);

\draw[color=drawColor,line cap=round,line join=round,fill opacity=0.00,] (221.78,436.52) --
	(226.72,436.52) --
	(226.72,442.31) --
	(221.78,442.31) --
	(221.78,436.52);

\draw[color=drawColor,line width= 1.2pt,line join=round,fill opacity=0.00,] (227.96,506.63) -- (232.91,506.63);

\draw[color=drawColor,dash pattern=on 4pt off 4pt ,line cap=round,line join=round,fill opacity=0.00,] (230.43,487.83) -- (230.43,502.98);

\draw[color=drawColor,dash pattern=on 4pt off 4pt ,line cap=round,line join=round,fill opacity=0.00,] (230.43,516.84) -- (230.43,512.11);

\draw[color=drawColor,line cap=round,line join=round,fill opacity=0.00,] (229.20,487.83) -- (231.67,487.83);

\draw[color=drawColor,line cap=round,line join=round,fill opacity=0.00,] (229.20,516.84) -- (231.67,516.84);

\draw[color=drawColor,line cap=round,line join=round,fill opacity=0.00,] (227.96,502.98) --
	(232.91,502.98) --
	(232.91,512.11) --
	(227.96,512.11) --
	(227.96,502.98);

\draw[color=drawColor,line width= 1.2pt,line join=round,fill opacity=0.00,] (234.14,505.12) -- (239.09,505.12);

\draw[color=drawColor,dash pattern=on 4pt off 4pt ,line cap=round,line join=round,fill opacity=0.00,] (236.62,486.96) -- (236.62,501.99);

\draw[color=drawColor,dash pattern=on 4pt off 4pt ,line cap=round,line join=round,fill opacity=0.00,] (236.62,515.50) -- (236.62,510.34);

\draw[color=drawColor,line cap=round,line join=round,fill opacity=0.00,] (235.38,486.96) -- (237.85,486.96);

\draw[color=drawColor,line cap=round,line join=round,fill opacity=0.00,] (235.38,515.50) -- (237.85,515.50);

\draw[color=drawColor,line cap=round,line join=round,fill opacity=0.00,] (234.14,501.99) --
	(239.09,501.99) --
	(239.09,510.34) --
	(234.14,510.34) --
	(234.14,501.99);

\draw[color=drawColor,line width= 1.2pt,line join=round,fill opacity=0.00,] (240.32,511.95) -- (245.27,511.95);

\draw[color=drawColor,dash pattern=on 4pt off 4pt ,line cap=round,line join=round,fill opacity=0.00,] (242.80,491.99) -- (242.80,508.33);

\draw[color=drawColor,dash pattern=on 4pt off 4pt ,line cap=round,line join=round,fill opacity=0.00,] (242.80,544.44) -- (242.80,516.56);

\draw[color=drawColor,line cap=round,line join=round,fill opacity=0.00,] (241.56,491.99) -- (244.03,491.99);

\draw[color=drawColor,line cap=round,line join=round,fill opacity=0.00,] (241.56,544.44) -- (244.03,544.44);

\draw[color=drawColor,line cap=round,line join=round,fill opacity=0.00,] (240.32,508.33) --
	(245.27,508.33) --
	(245.27,516.56) --
	(240.32,516.56) --
	(240.32,508.33);

\draw[color=drawColor,line width= 1.2pt,line join=round,fill opacity=0.00,] (246.51,445.28) -- (251.45,445.28);

\draw[color=drawColor,dash pattern=on 4pt off 4pt ,line cap=round,line join=round,fill opacity=0.00,] (248.98,440.02) -- (248.98,442.41);

\draw[color=drawColor,dash pattern=on 4pt off 4pt ,line cap=round,line join=round,fill opacity=0.00,] (248.98,457.14) -- (248.98,451.07);

\draw[color=drawColor,line cap=round,line join=round,fill opacity=0.00,] (247.74,440.02) -- (250.22,440.02);

\draw[color=drawColor,line cap=round,line join=round,fill opacity=0.00,] (247.74,457.14) -- (250.22,457.14);

\draw[color=drawColor,line cap=round,line join=round,fill opacity=0.00,] (246.51,442.41) --
	(251.45,442.41) --
	(251.45,451.07) --
	(246.51,451.07) --
	(246.51,442.41);

\draw[color=drawColor,line width= 1.2pt,line join=round,fill opacity=0.00,] (252.69,508.23) -- (257.64,508.23);

\draw[color=drawColor,dash pattern=on 4pt off 4pt ,line cap=round,line join=round,fill opacity=0.00,] (255.16,487.73) -- (255.16,503.19);

\draw[color=drawColor,dash pattern=on 4pt off 4pt ,line cap=round,line join=round,fill opacity=0.00,] (255.16,517.46) -- (255.16,513.90);

\draw[color=drawColor,line cap=round,line join=round,fill opacity=0.00,] (253.93,487.73) -- (256.40,487.73);

\draw[color=drawColor,line cap=round,line join=round,fill opacity=0.00,] (253.93,517.46) -- (256.40,517.46);

\draw[color=drawColor,line cap=round,line join=round,fill opacity=0.00,] (252.69,503.19) --
	(257.64,503.19) --
	(257.64,513.90) --
	(252.69,513.90) --
	(252.69,503.19);

\draw[color=drawColor,line width= 1.2pt,line join=round,fill opacity=0.00,] (258.87,508.73) -- (263.82,508.73);

\draw[color=drawColor,dash pattern=on 4pt off 4pt ,line cap=round,line join=round,fill opacity=0.00,] (261.34,482.96) -- (261.34,499.69);

\draw[color=drawColor,dash pattern=on 4pt off 4pt ,line cap=round,line join=round,fill opacity=0.00,] (261.34,552.54) -- (261.34,515.81);

\draw[color=drawColor,line cap=round,line join=round,fill opacity=0.00,] (260.11,482.96) -- (262.58,482.96);

\draw[color=drawColor,line cap=round,line join=round,fill opacity=0.00,] (260.11,552.54) -- (262.58,552.54);

\draw[color=drawColor,line cap=round,line join=round,fill opacity=0.00,] (258.87,499.69) --
	(263.82,499.69) --
	(263.82,515.81) --
	(258.87,515.81) --
	(258.87,499.69);

\draw[color=drawColor,line width= 1.2pt,line join=round,fill opacity=0.00,] (265.05,514.80) -- (270.00,514.80);

\draw[color=drawColor,dash pattern=on 4pt off 4pt ,line cap=round,line join=round,fill opacity=0.00,] (267.53,489.33) -- (267.53,508.48);

\draw[color=drawColor,dash pattern=on 4pt off 4pt ,line cap=round,line join=round,fill opacity=0.00,] (267.53,558.27) -- (267.53,520.85);

\draw[color=drawColor,line cap=round,line join=round,fill opacity=0.00,] (266.29,489.33) -- (268.76,489.33);

\draw[color=drawColor,line cap=round,line join=round,fill opacity=0.00,] (266.29,558.27) -- (268.76,558.27);

\draw[color=drawColor,line cap=round,line join=round,fill opacity=0.00,] (265.05,508.48) --
	(270.00,508.48) --
	(270.00,520.85) --
	(265.05,520.85) --
	(265.05,508.48);

\draw[color=drawColor,line width= 1.2pt,line join=round,fill opacity=0.00,] (271.24,482.82) -- (276.18,482.82);

\draw[color=drawColor,dash pattern=on 4pt off 4pt ,line cap=round,line join=round,fill opacity=0.00,] (273.71,469.54) -- (273.71,471.76);

\draw[color=drawColor,dash pattern=on 4pt off 4pt ,line cap=round,line join=round,fill opacity=0.00,] (273.71,497.01) -- (273.71,491.28);

\draw[color=drawColor,line cap=round,line join=round,fill opacity=0.00,] (272.47,469.54) -- (274.95,469.54);

\draw[color=drawColor,line cap=round,line join=round,fill opacity=0.00,] (272.47,497.01) -- (274.95,497.01);

\draw[color=drawColor,line cap=round,line join=round,fill opacity=0.00,] (271.24,471.76) --
	(276.18,471.76) --
	(276.18,491.28) --
	(271.24,491.28) --
	(271.24,471.76);

\draw[color=drawColor,line width= 1.2pt,line join=round,fill opacity=0.00,] (277.42,511.75) -- (282.36,511.75);

\draw[color=drawColor,dash pattern=on 4pt off 4pt ,line cap=round,line join=round,fill opacity=0.00,] (279.89,488.67) -- (279.89,506.68);

\draw[color=drawColor,dash pattern=on 4pt off 4pt ,line cap=round,line join=round,fill opacity=0.00,] (279.89,553.16) -- (279.89,517.39);

\draw[color=drawColor,line cap=round,line join=round,fill opacity=0.00,] (278.66,488.67) -- (281.13,488.67);

\draw[color=drawColor,line cap=round,line join=round,fill opacity=0.00,] (278.66,553.16) -- (281.13,553.16);

\draw[color=drawColor,line cap=round,line join=round,fill opacity=0.00,] (277.42,506.68) --
	(282.36,506.68) --
	(282.36,517.39) --
	(277.42,517.39) --
	(277.42,506.68);

\draw[color=drawColor,line width= 1.2pt,line join=round,fill opacity=0.00,] (283.60,468.37) -- (288.55,468.37);

\draw[color=drawColor,dash pattern=on 4pt off 4pt ,line cap=round,line join=round,fill opacity=0.00,] (286.07,457.91) -- (286.07,464.41);

\draw[color=drawColor,dash pattern=on 4pt off 4pt ,line cap=round,line join=round,fill opacity=0.00,] (286.07,488.98) -- (286.07,475.64);

\draw[color=drawColor,line cap=round,line join=round,fill opacity=0.00,] (284.84,457.91) -- (287.31,457.91);

\draw[color=drawColor,line cap=round,line join=round,fill opacity=0.00,] (284.84,488.98) -- (287.31,488.98);

\draw[color=drawColor,line cap=round,line join=round,fill opacity=0.00,] (283.60,464.41) --
	(288.55,464.41) --
	(288.55,475.64) --
	(283.60,475.64) --
	(283.60,464.41);

\draw[color=drawColor,line width= 1.2pt,line join=round,fill opacity=0.00,] (289.78,473.69) -- (294.73,473.69);

\draw[color=drawColor,dash pattern=on 4pt off 4pt ,line cap=round,line join=round,fill opacity=0.00,] (292.26,459.45) -- (292.26,467.53);

\draw[color=drawColor,dash pattern=on 4pt off 4pt ,line cap=round,line join=round,fill opacity=0.00,] (292.26,497.23) -- (292.26,481.25);

\draw[color=drawColor,line cap=round,line join=round,fill opacity=0.00,] (291.02,459.45) -- (293.49,459.45);

\draw[color=drawColor,line cap=round,line join=round,fill opacity=0.00,] (291.02,497.23) -- (293.49,497.23);

\draw[color=drawColor,line cap=round,line join=round,fill opacity=0.00,] (289.78,467.53) --
	(294.73,467.53) --
	(294.73,481.25) --
	(289.78,481.25) --
	(289.78,467.53);

\draw[color=drawColor,line width= 1.2pt,line join=round,fill opacity=0.00,] (295.97,466.69) -- (300.91,466.69);

\draw[color=drawColor,dash pattern=on 4pt off 4pt ,line cap=round,line join=round,fill opacity=0.00,] (298.44,455.25) -- (298.44,460.79);

\draw[color=drawColor,dash pattern=on 4pt off 4pt ,line cap=round,line join=round,fill opacity=0.00,] (298.44,487.99) -- (298.44,472.93);

\draw[color=drawColor,line cap=round,line join=round,fill opacity=0.00,] (297.20,455.25) -- (299.68,455.25);

\draw[color=drawColor,line cap=round,line join=round,fill opacity=0.00,] (297.20,487.99) -- (299.68,487.99);

\draw[color=drawColor,line cap=round,line join=round,fill opacity=0.00,] (295.97,460.79) --
	(300.91,460.79) --
	(300.91,472.93) --
	(295.97,472.93) --
	(295.97,460.79);

\draw[color=drawColor,line width= 1.2pt,line join=round,fill opacity=0.00,] (302.15,453.59) -- (307.09,453.59);

\draw[color=drawColor,dash pattern=on 4pt off 4pt ,line cap=round,line join=round,fill opacity=0.00,] (304.62,444.43) -- (304.62,447.84);

\draw[color=drawColor,dash pattern=on 4pt off 4pt ,line cap=round,line join=round,fill opacity=0.00,] (304.62,465.63) -- (304.62,459.96);

\draw[color=drawColor,line cap=round,line join=round,fill opacity=0.00,] (303.39,444.43) -- (305.86,444.43);

\draw[color=drawColor,line cap=round,line join=round,fill opacity=0.00,] (303.39,465.63) -- (305.86,465.63);

\draw[color=drawColor,line cap=round,line join=round,fill opacity=0.00,] (302.15,447.84) --
	(307.09,447.84) --
	(307.09,459.96) --
	(302.15,459.96) --
	(302.15,447.84);

\draw[color=drawColor,line width= 1.2pt,line join=round,fill opacity=0.00,] (308.33,464.75) -- (313.28,464.75);

\draw[color=drawColor,dash pattern=on 4pt off 4pt ,line cap=round,line join=round,fill opacity=0.00,] (310.80,454.87) -- (310.80,459.06);

\draw[color=drawColor,dash pattern=on 4pt off 4pt ,line cap=round,line join=round,fill opacity=0.00,] (310.80,488.17) -- (310.80,471.10);

\draw[color=drawColor,line cap=round,line join=round,fill opacity=0.00,] (309.57,454.87) -- (312.04,454.87);

\draw[color=drawColor,line cap=round,line join=round,fill opacity=0.00,] (309.57,488.17) -- (312.04,488.17);

\draw[color=drawColor,line cap=round,line join=round,fill opacity=0.00,] (308.33,459.06) --
	(313.28,459.06) --
	(313.28,471.10) --
	(308.33,471.10) --
	(308.33,459.06);

\draw[color=drawColor,line width= 1.2pt,line join=round,fill opacity=0.00,] (314.51,461.36) -- (319.46,461.36);

\draw[color=drawColor,dash pattern=on 4pt off 4pt ,line cap=round,line join=round,fill opacity=0.00,] (316.99,448.45) -- (316.99,456.08);

\draw[color=drawColor,dash pattern=on 4pt off 4pt ,line cap=round,line join=round,fill opacity=0.00,] (316.99,470.35) -- (316.99,465.51);

\draw[color=drawColor,line cap=round,line join=round,fill opacity=0.00,] (315.75,448.45) -- (318.22,448.45);

\draw[color=drawColor,line cap=round,line join=round,fill opacity=0.00,] (315.75,470.35) -- (318.22,470.35);

\draw[color=drawColor,line cap=round,line join=round,fill opacity=0.00,] (314.51,456.08) --
	(319.46,456.08) --
	(319.46,465.51) --
	(314.51,465.51) --
	(314.51,456.08);

\draw[color=drawColor,line width= 1.2pt,line join=round,fill opacity=0.00,] (320.70,511.86) -- (325.64,511.86);

\draw[color=drawColor,dash pattern=on 4pt off 4pt ,line cap=round,line join=round,fill opacity=0.00,] (323.17,489.74) -- (323.17,507.48);

\draw[color=drawColor,dash pattern=on 4pt off 4pt ,line cap=round,line join=round,fill opacity=0.00,] (323.17,548.19) -- (323.17,516.63);

\draw[color=drawColor,line cap=round,line join=round,fill opacity=0.00,] (321.93,489.74) -- (324.41,489.74);

\draw[color=drawColor,line cap=round,line join=round,fill opacity=0.00,] (321.93,548.19) -- (324.41,548.19);

\draw[color=drawColor,line cap=round,line join=round,fill opacity=0.00,] (320.70,507.48) --
	(325.64,507.48) --
	(325.64,516.63) --
	(320.70,516.63) --
	(320.70,507.48);

\draw[color=drawColor,line width= 1.2pt,line join=round,fill opacity=0.00,] (326.88,465.81) -- (331.82,465.81);

\draw[color=drawColor,dash pattern=on 4pt off 4pt ,line cap=round,line join=round,fill opacity=0.00,] (329.35,455.24) -- (329.35,462.59);

\draw[color=drawColor,dash pattern=on 4pt off 4pt ,line cap=round,line join=round,fill opacity=0.00,] (329.35,488.47) -- (329.35,473.51);

\draw[color=drawColor,line cap=round,line join=round,fill opacity=0.00,] (328.11,455.24) -- (330.59,455.24);

\draw[color=drawColor,line cap=round,line join=round,fill opacity=0.00,] (328.11,488.47) -- (330.59,488.47);

\draw[color=drawColor,line cap=round,line join=round,fill opacity=0.00,] (326.88,462.59) --
	(331.82,462.59) --
	(331.82,473.51) --
	(326.88,473.51) --
	(326.88,462.59);

\draw[color=drawColor,line width= 1.2pt,line join=round,fill opacity=0.00,] (333.06,485.12) -- (338.01,485.12);

\draw[color=drawColor,dash pattern=on 4pt off 4pt ,line cap=round,line join=round,fill opacity=0.00,] (335.53,469.57) -- (335.53,473.87);

\draw[color=drawColor,dash pattern=on 4pt off 4pt ,line cap=round,line join=round,fill opacity=0.00,] (335.53,503.32) -- (335.53,493.55);

\draw[color=drawColor,line cap=round,line join=round,fill opacity=0.00,] (334.30,469.57) -- (336.77,469.57);

\draw[color=drawColor,line cap=round,line join=round,fill opacity=0.00,] (334.30,503.32) -- (336.77,503.32);

\draw[color=drawColor,line cap=round,line join=round,fill opacity=0.00,] (333.06,473.87) --
	(338.01,473.87) --
	(338.01,493.55) --
	(333.06,493.55) --
	(333.06,473.87);

\draw[color=drawColor,line width= 1.2pt,line join=round,fill opacity=0.00,] (339.24,478.96) -- (344.19,478.96);

\draw[color=drawColor,dash pattern=on 4pt off 4pt ,line cap=round,line join=round,fill opacity=0.00,] (341.72,466.84) -- (341.72,471.90);

\draw[color=drawColor,dash pattern=on 4pt off 4pt ,line cap=round,line join=round,fill opacity=0.00,] (341.72,500.14) -- (341.72,488.06);

\draw[color=drawColor,line cap=round,line join=round,fill opacity=0.00,] (340.48,466.84) -- (342.95,466.84);

\draw[color=drawColor,line cap=round,line join=round,fill opacity=0.00,] (340.48,500.14) -- (342.95,500.14);

\draw[color=drawColor,line cap=round,line join=round,fill opacity=0.00,] (339.24,471.90) --
	(344.19,471.90) --
	(344.19,488.06) --
	(339.24,488.06) --
	(339.24,471.90);

\draw[color=drawColor,line width= 1.2pt,line join=round,fill opacity=0.00,] (345.43,517.90) -- (350.37,517.90);

\draw[color=drawColor,dash pattern=on 4pt off 4pt ,line cap=round,line join=round,fill opacity=0.00,] (347.90,506.44) -- (347.90,512.63);

\draw[color=drawColor,dash pattern=on 4pt off 4pt ,line cap=round,line join=round,fill opacity=0.00,] (347.90,561.39) -- (347.90,551.30);

\draw[color=drawColor,line cap=round,line join=round,fill opacity=0.00,] (346.66,506.44) -- (349.13,506.44);

\draw[color=drawColor,line cap=round,line join=round,fill opacity=0.00,] (346.66,561.39) -- (349.13,561.39);

\draw[color=drawColor,line cap=round,line join=round,fill opacity=0.00,] (345.43,512.63) --
	(350.37,512.63) --
	(350.37,551.30) --
	(345.43,551.30) --
	(345.43,512.63);

\draw[color=drawColor,line width= 1.2pt,line join=round,fill opacity=0.00,] (351.61,482.08) -- (356.55,482.08);

\draw[color=drawColor,dash pattern=on 4pt off 4pt ,line cap=round,line join=round,fill opacity=0.00,] (354.08,461.12) -- (354.08,470.39);

\draw[color=drawColor,dash pattern=on 4pt off 4pt ,line cap=round,line join=round,fill opacity=0.00,] (354.08,508.43) -- (354.08,498.32);

\draw[color=drawColor,line cap=round,line join=round,fill opacity=0.00,] (352.84,461.12) -- (355.32,461.12);

\draw[color=drawColor,line cap=round,line join=round,fill opacity=0.00,] (352.84,508.43) -- (355.32,508.43);

\draw[color=drawColor,line cap=round,line join=round,fill opacity=0.00,] (351.61,470.39) --
	(356.55,470.39) --
	(356.55,498.32) --
	(351.61,498.32) --
	(351.61,470.39);

\draw[color=drawColor,line width= 1.2pt,line join=round,fill opacity=0.00,] (357.79,468.75) -- (362.74,468.75);

\draw[color=drawColor,dash pattern=on 4pt off 4pt ,line cap=round,line join=round,fill opacity=0.00,] (360.26,458.33) -- (360.26,464.51);

\draw[color=drawColor,dash pattern=on 4pt off 4pt ,line cap=round,line join=round,fill opacity=0.00,] (360.26,488.92) -- (360.26,476.05);

\draw[color=drawColor,line cap=round,line join=round,fill opacity=0.00,] (359.03,458.33) -- (361.50,458.33);

\draw[color=drawColor,line cap=round,line join=round,fill opacity=0.00,] (359.03,488.92) -- (361.50,488.92);

\draw[color=drawColor,line cap=round,line join=round,fill opacity=0.00,] (357.79,464.51) --
	(362.74,464.51) --
	(362.74,476.05) --
	(357.79,476.05) --
	(357.79,464.51);

\draw[color=drawColor,line width= 1.2pt,line join=round,fill opacity=0.00,] (363.97,454.59) -- (368.92,454.59);

\draw[color=drawColor,dash pattern=on 4pt off 4pt ,line cap=round,line join=round,fill opacity=0.00,] (366.45,442.65) -- (366.45,448.05);

\draw[color=drawColor,dash pattern=on 4pt off 4pt ,line cap=round,line join=round,fill opacity=0.00,] (366.45,469.41) -- (366.45,461.52);

\draw[color=drawColor,line cap=round,line join=round,fill opacity=0.00,] (365.21,442.65) -- (367.68,442.65);

\draw[color=drawColor,line cap=round,line join=round,fill opacity=0.00,] (365.21,469.41) -- (367.68,469.41);

\draw[color=drawColor,line cap=round,line join=round,fill opacity=0.00,] (363.97,448.05) --
	(368.92,448.05) --
	(368.92,461.52) --
	(363.97,461.52) --
	(363.97,448.05);

\draw[color=drawColor,line width= 1.2pt,line join=round,fill opacity=0.00,] (370.16,467.17) -- (375.10,467.17);

\draw[color=drawColor,dash pattern=on 4pt off 4pt ,line cap=round,line join=round,fill opacity=0.00,] (372.63,456.18) -- (372.63,463.50);

\draw[color=drawColor,dash pattern=on 4pt off 4pt ,line cap=round,line join=round,fill opacity=0.00,] (372.63,488.94) -- (372.63,475.83);

\draw[color=drawColor,line cap=round,line join=round,fill opacity=0.00,] (371.39,456.18) -- (373.86,456.18);

\draw[color=drawColor,line cap=round,line join=round,fill opacity=0.00,] (371.39,488.94) -- (373.86,488.94);

\draw[color=drawColor,line cap=round,line join=round,fill opacity=0.00,] (370.16,463.50) --
	(375.10,463.50) --
	(375.10,475.83) --
	(370.16,475.83) --
	(370.16,463.50);

\draw[color=drawColor,line width= 1.2pt,line join=round,fill opacity=0.00,] (376.34,450.13) -- (381.28,450.13);

\draw[color=drawColor,dash pattern=on 4pt off 4pt ,line cap=round,line join=round,fill opacity=0.00,] (378.81,442.18) -- (378.81,444.95);

\draw[color=drawColor,dash pattern=on 4pt off 4pt ,line cap=round,line join=round,fill opacity=0.00,] (378.81,463.21) -- (378.81,453.94);

\draw[color=drawColor,line cap=round,line join=round,fill opacity=0.00,] (377.57,442.18) -- (380.05,442.18);

\draw[color=drawColor,line cap=round,line join=round,fill opacity=0.00,] (377.57,463.21) -- (380.05,463.21);

\draw[color=drawColor,line cap=round,line join=round,fill opacity=0.00,] (376.34,444.95) --
	(381.28,444.95) --
	(381.28,453.94) --
	(376.34,453.94) --
	(376.34,444.95);

\draw[color=drawColor,line width= 1.2pt,line join=round,fill opacity=0.00,] (382.52,462.84) -- (387.47,462.84);

\draw[color=drawColor,dash pattern=on 4pt off 4pt ,line cap=round,line join=round,fill opacity=0.00,] (384.99,454.67) -- (384.99,457.90);

\draw[color=drawColor,dash pattern=on 4pt off 4pt ,line cap=round,line join=round,fill opacity=0.00,] (384.99,486.91) -- (384.99,468.14);

\draw[color=drawColor,line cap=round,line join=round,fill opacity=0.00,] (383.76,454.67) -- (386.23,454.67);

\draw[color=drawColor,line cap=round,line join=round,fill opacity=0.00,] (383.76,486.91) -- (386.23,486.91);

\draw[color=drawColor,line cap=round,line join=round,fill opacity=0.00,] (382.52,457.90) --
	(387.47,457.90) --
	(387.47,468.14) --
	(382.52,468.14) --
	(382.52,457.90);

\draw[color=drawColor,line width= 1.2pt,line join=round,fill opacity=0.00,] (388.70,464.82) -- (393.65,464.82);

\draw[color=drawColor,dash pattern=on 4pt off 4pt ,line cap=round,line join=round,fill opacity=0.00,] (391.18,454.81) -- (391.18,460.00);

\draw[color=drawColor,dash pattern=on 4pt off 4pt ,line cap=round,line join=round,fill opacity=0.00,] (391.18,487.76) -- (391.18,470.99);

\draw[color=drawColor,line cap=round,line join=round,fill opacity=0.00,] (389.94,454.81) -- (392.41,454.81);

\draw[color=drawColor,line cap=round,line join=round,fill opacity=0.00,] (389.94,487.76) -- (392.41,487.76);

\draw[color=drawColor,line cap=round,line join=round,fill opacity=0.00,] (388.70,460.00) --
	(393.65,460.00) --
	(393.65,470.99) --
	(388.70,470.99) --
	(388.70,460.00);

\draw[color=drawColor,line width= 1.2pt,line join=round,fill opacity=0.00,] (394.88,501.89) -- (399.83,501.89);

\draw[color=drawColor,dash pattern=on 4pt off 4pt ,line cap=round,line join=round,fill opacity=0.00,] (397.36,482.05) -- (397.36,495.36);

\draw[color=drawColor,dash pattern=on 4pt off 4pt ,line cap=round,line join=round,fill opacity=0.00,] (397.36,510.56) -- (397.36,506.34);

\draw[color=drawColor,line cap=round,line join=round,fill opacity=0.00,] (396.12,482.05) -- (398.59,482.05);

\draw[color=drawColor,line cap=round,line join=round,fill opacity=0.00,] (396.12,510.56) -- (398.59,510.56);

\draw[color=drawColor,line cap=round,line join=round,fill opacity=0.00,] (394.88,495.36) --
	(399.83,495.36) --
	(399.83,506.34) --
	(394.88,506.34) --
	(394.88,495.36);

\draw[color=drawColor,line width= 1.2pt,line join=round,fill opacity=0.00,] (401.07,467.84) -- (406.01,467.84);

\draw[color=drawColor,dash pattern=on 4pt off 4pt ,line cap=round,line join=round,fill opacity=0.00,] (403.54,456.03) -- (403.54,461.80);

\draw[color=drawColor,dash pattern=on 4pt off 4pt ,line cap=round,line join=round,fill opacity=0.00,] (403.54,488.78) -- (403.54,472.95);

\draw[color=drawColor,line cap=round,line join=round,fill opacity=0.00,] (402.30,456.03) -- (404.78,456.03);

\draw[color=drawColor,line cap=round,line join=round,fill opacity=0.00,] (402.30,488.78) -- (404.78,488.78);

\draw[color=drawColor,line cap=round,line join=round,fill opacity=0.00,] (401.07,461.80) --
	(406.01,461.80) --
	(406.01,472.95) --
	(401.07,472.95) --
	(401.07,461.80);

\draw[color=drawColor,line width= 1.2pt,line join=round,fill opacity=0.00,] (407.25,461.91) -- (412.20,461.91);

\draw[color=drawColor,dash pattern=on 4pt off 4pt ,line cap=round,line join=round,fill opacity=0.00,] (409.72,454.05) -- (409.72,457.22);

\draw[color=drawColor,dash pattern=on 4pt off 4pt ,line cap=round,line join=round,fill opacity=0.00,] (409.72,471.26) -- (409.72,466.74);

\draw[color=drawColor,line cap=round,line join=round,fill opacity=0.00,] (408.49,454.05) -- (410.96,454.05);

\draw[color=drawColor,line cap=round,line join=round,fill opacity=0.00,] (408.49,471.26) -- (410.96,471.26);

\draw[color=drawColor,line cap=round,line join=round,fill opacity=0.00,] (407.25,457.22) --
	(412.20,457.22) --
	(412.20,466.74) --
	(407.25,466.74) --
	(407.25,457.22);

\draw[color=drawColor,line width= 1.2pt,line join=round,fill opacity=0.00,] (413.43,484.99) -- (418.38,484.99);

\draw[color=drawColor,dash pattern=on 4pt off 4pt ,line cap=round,line join=round,fill opacity=0.00,] (415.90,466.15) -- (415.90,472.59);

\draw[color=drawColor,dash pattern=on 4pt off 4pt ,line cap=round,line join=round,fill opacity=0.00,] (415.90,509.50) -- (415.90,502.17);

\draw[color=drawColor,line cap=round,line join=round,fill opacity=0.00,] (414.67,466.15) -- (417.14,466.15);

\draw[color=drawColor,line cap=round,line join=round,fill opacity=0.00,] (414.67,509.50) -- (417.14,509.50);

\draw[color=drawColor,line cap=round,line join=round,fill opacity=0.00,] (413.43,472.59) --
	(418.38,472.59) --
	(418.38,502.17) --
	(413.43,502.17) --
	(413.43,472.59);

\draw[color=drawColor,line width= 1.2pt,line join=round,fill opacity=0.00,] (419.61,472.50) -- (424.56,472.50);

\draw[color=drawColor,dash pattern=on 4pt off 4pt ,line cap=round,line join=round,fill opacity=0.00,] (422.09,458.97) -- (422.09,467.64);

\draw[color=drawColor,dash pattern=on 4pt off 4pt ,line cap=round,line join=round,fill opacity=0.00,] (422.09,495.65) -- (422.09,479.81);

\draw[color=drawColor,line cap=round,line join=round,fill opacity=0.00,] (420.85,458.97) -- (423.32,458.97);

\draw[color=drawColor,line cap=round,line join=round,fill opacity=0.00,] (420.85,495.65) -- (423.32,495.65);

\draw[color=drawColor,line cap=round,line join=round,fill opacity=0.00,] (419.61,467.64) --
	(424.56,467.64) --
	(424.56,479.81) --
	(419.61,479.81) --
	(419.61,467.64);

\draw[color=drawColor,line width= 1.2pt,line join=round,fill opacity=0.00,] (425.80,464.24) -- (430.74,464.24);

\draw[color=drawColor,dash pattern=on 4pt off 4pt ,line cap=round,line join=round,fill opacity=0.00,] (428.27,453.71) -- (428.27,458.68);

\draw[color=drawColor,dash pattern=on 4pt off 4pt ,line cap=round,line join=round,fill opacity=0.00,] (428.27,487.83) -- (428.27,469.44);

\draw[color=drawColor,line cap=round,line join=round,fill opacity=0.00,] (427.03,453.71) -- (429.51,453.71);

\draw[color=drawColor,line cap=round,line join=round,fill opacity=0.00,] (427.03,487.83) -- (429.51,487.83);

\draw[color=drawColor,line cap=round,line join=round,fill opacity=0.00,] (425.80,458.68) --
	(430.74,458.68) --
	(430.74,469.44) --
	(425.80,469.44) --
	(425.80,458.68);

\draw[color=drawColor,line width= 1.2pt,line join=round,fill opacity=0.00,] (431.98,488.38) -- (436.93,488.38);

\draw[color=drawColor,dash pattern=on 4pt off 4pt ,line cap=round,line join=round,fill opacity=0.00,] (434.45,462.72) -- (434.45,471.83);

\draw[color=drawColor,dash pattern=on 4pt off 4pt ,line cap=round,line join=round,fill opacity=0.00,] (434.45,517.41) -- (434.45,509.86);

\draw[color=drawColor,line cap=round,line join=round,fill opacity=0.00,] (433.22,462.72) -- (435.69,462.72);

\draw[color=drawColor,line cap=round,line join=round,fill opacity=0.00,] (433.22,517.41) -- (435.69,517.41);

\draw[color=drawColor,line cap=round,line join=round,fill opacity=0.00,] (431.98,471.83) --
	(436.93,471.83) --
	(436.93,509.86) --
	(431.98,509.86) --
	(431.98,471.83);
\end{scope}
\begin{scope}
\path[clip] (  0.00,  0.00) rectangle (469.75,614.29);
\definecolor[named]{drawColor}{rgb}{0.00,0.00,0.00}

\draw[color=drawColor,line cap=round,line join=round,fill opacity=0.00,] ( 51.14,426.16) -- (434.45,426.16);

\draw[color=drawColor,line cap=round,line join=round,fill opacity=0.00,] ( 51.14,426.16) -- ( 51.14,422.20);

\draw[color=drawColor,line cap=round,line join=round,fill opacity=0.00,] ( 57.33,426.16) -- ( 57.33,422.20);

\draw[color=drawColor,line cap=round,line join=round,fill opacity=0.00,] ( 63.51,426.16) -- ( 63.51,422.20);

\draw[color=drawColor,line cap=round,line join=round,fill opacity=0.00,] ( 69.69,426.16) -- ( 69.69,422.20);

\draw[color=drawColor,line cap=round,line join=round,fill opacity=0.00,] ( 75.87,426.16) -- ( 75.87,422.20);

\draw[color=drawColor,line cap=round,line join=round,fill opacity=0.00,] ( 82.05,426.16) -- ( 82.05,422.20);

\draw[color=drawColor,line cap=round,line join=round,fill opacity=0.00,] ( 88.24,426.16) -- ( 88.24,422.20);

\draw[color=drawColor,line cap=round,line join=round,fill opacity=0.00,] ( 94.42,426.16) -- ( 94.42,422.20);

\draw[color=drawColor,line cap=round,line join=round,fill opacity=0.00,] (100.60,426.16) -- (100.60,422.20);

\draw[color=drawColor,line cap=round,line join=round,fill opacity=0.00,] (106.78,426.16) -- (106.78,422.20);

\draw[color=drawColor,line cap=round,line join=round,fill opacity=0.00,] (112.97,426.16) -- (112.97,422.20);

\draw[color=drawColor,line cap=round,line join=round,fill opacity=0.00,] (119.15,426.16) -- (119.15,422.20);

\draw[color=drawColor,line cap=round,line join=round,fill opacity=0.00,] (125.33,426.16) -- (125.33,422.20);

\draw[color=drawColor,line cap=round,line join=round,fill opacity=0.00,] (131.51,426.16) -- (131.51,422.20);

\draw[color=drawColor,line cap=round,line join=round,fill opacity=0.00,] (137.70,426.16) -- (137.70,422.20);

\draw[color=drawColor,line cap=round,line join=round,fill opacity=0.00,] (143.88,426.16) -- (143.88,422.20);

\draw[color=drawColor,line cap=round,line join=round,fill opacity=0.00,] (150.06,426.16) -- (150.06,422.20);

\draw[color=drawColor,line cap=round,line join=round,fill opacity=0.00,] (156.24,426.16) -- (156.24,422.20);

\draw[color=drawColor,line cap=round,line join=round,fill opacity=0.00,] (162.43,426.16) -- (162.43,422.20);

\draw[color=drawColor,line cap=round,line join=round,fill opacity=0.00,] (168.61,426.16) -- (168.61,422.20);

\draw[color=drawColor,line cap=round,line join=round,fill opacity=0.00,] (174.79,426.16) -- (174.79,422.20);

\draw[color=drawColor,line cap=round,line join=round,fill opacity=0.00,] (180.97,426.16) -- (180.97,422.20);

\draw[color=drawColor,line cap=round,line join=round,fill opacity=0.00,] (187.16,426.16) -- (187.16,422.20);

\draw[color=drawColor,line cap=round,line join=round,fill opacity=0.00,] (193.34,426.16) -- (193.34,422.20);

\draw[color=drawColor,line cap=round,line join=round,fill opacity=0.00,] (199.52,426.16) -- (199.52,422.20);

\draw[color=drawColor,line cap=round,line join=round,fill opacity=0.00,] (205.70,426.16) -- (205.70,422.20);

\draw[color=drawColor,line cap=round,line join=round,fill opacity=0.00,] (211.89,426.16) -- (211.89,422.20);

\draw[color=drawColor,line cap=round,line join=round,fill opacity=0.00,] (218.07,426.16) -- (218.07,422.20);

\draw[color=drawColor,line cap=round,line join=round,fill opacity=0.00,] (224.25,426.16) -- (224.25,422.20);

\draw[color=drawColor,line cap=round,line join=round,fill opacity=0.00,] (230.43,426.16) -- (230.43,422.20);

\draw[color=drawColor,line cap=round,line join=round,fill opacity=0.00,] (236.62,426.16) -- (236.62,422.20);

\draw[color=drawColor,line cap=round,line join=round,fill opacity=0.00,] (242.80,426.16) -- (242.80,422.20);

\draw[color=drawColor,line cap=round,line join=round,fill opacity=0.00,] (248.98,426.16) -- (248.98,422.20);

\draw[color=drawColor,line cap=round,line join=round,fill opacity=0.00,] (255.16,426.16) -- (255.16,422.20);

\draw[color=drawColor,line cap=round,line join=round,fill opacity=0.00,] (261.34,426.16) -- (261.34,422.20);

\draw[color=drawColor,line cap=round,line join=round,fill opacity=0.00,] (267.53,426.16) -- (267.53,422.20);

\draw[color=drawColor,line cap=round,line join=round,fill opacity=0.00,] (273.71,426.16) -- (273.71,422.20);

\draw[color=drawColor,line cap=round,line join=round,fill opacity=0.00,] (279.89,426.16) -- (279.89,422.20);

\draw[color=drawColor,line cap=round,line join=round,fill opacity=0.00,] (286.07,426.16) -- (286.07,422.20);

\draw[color=drawColor,line cap=round,line join=round,fill opacity=0.00,] (292.26,426.16) -- (292.26,422.20);

\draw[color=drawColor,line cap=round,line join=round,fill opacity=0.00,] (298.44,426.16) -- (298.44,422.20);

\draw[color=drawColor,line cap=round,line join=round,fill opacity=0.00,] (304.62,426.16) -- (304.62,422.20);

\draw[color=drawColor,line cap=round,line join=round,fill opacity=0.00,] (310.80,426.16) -- (310.80,422.20);

\draw[color=drawColor,line cap=round,line join=round,fill opacity=0.00,] (316.99,426.16) -- (316.99,422.20);

\draw[color=drawColor,line cap=round,line join=round,fill opacity=0.00,] (323.17,426.16) -- (323.17,422.20);

\draw[color=drawColor,line cap=round,line join=round,fill opacity=0.00,] (329.35,426.16) -- (329.35,422.20);

\draw[color=drawColor,line cap=round,line join=round,fill opacity=0.00,] (335.53,426.16) -- (335.53,422.20);

\draw[color=drawColor,line cap=round,line join=round,fill opacity=0.00,] (341.72,426.16) -- (341.72,422.20);

\draw[color=drawColor,line cap=round,line join=round,fill opacity=0.00,] (347.90,426.16) -- (347.90,422.20);

\draw[color=drawColor,line cap=round,line join=round,fill opacity=0.00,] (354.08,426.16) -- (354.08,422.20);

\draw[color=drawColor,line cap=round,line join=round,fill opacity=0.00,] (360.26,426.16) -- (360.26,422.20);

\draw[color=drawColor,line cap=round,line join=round,fill opacity=0.00,] (366.45,426.16) -- (366.45,422.20);

\draw[color=drawColor,line cap=round,line join=round,fill opacity=0.00,] (372.63,426.16) -- (372.63,422.20);

\draw[color=drawColor,line cap=round,line join=round,fill opacity=0.00,] (378.81,426.16) -- (378.81,422.20);

\draw[color=drawColor,line cap=round,line join=round,fill opacity=0.00,] (384.99,426.16) -- (384.99,422.20);

\draw[color=drawColor,line cap=round,line join=round,fill opacity=0.00,] (391.18,426.16) -- (391.18,422.20);

\draw[color=drawColor,line cap=round,line join=round,fill opacity=0.00,] (397.36,426.16) -- (397.36,422.20);

\draw[color=drawColor,line cap=round,line join=round,fill opacity=0.00,] (403.54,426.16) -- (403.54,422.20);

\draw[color=drawColor,line cap=round,line join=round,fill opacity=0.00,] (409.72,426.16) -- (409.72,422.20);

\draw[color=drawColor,line cap=round,line join=round,fill opacity=0.00,] (415.90,426.16) -- (415.90,422.20);

\draw[color=drawColor,line cap=round,line join=round,fill opacity=0.00,] (422.09,426.16) -- (422.09,422.20);

\draw[color=drawColor,line cap=round,line join=round,fill opacity=0.00,] (428.27,426.16) -- (428.27,422.20);

\draw[color=drawColor,line cap=round,line join=round,fill opacity=0.00,] (434.45,426.16) -- (434.45,422.20);

\node[color=drawColor,anchor=base,inner sep=0pt, outer sep=0pt, scale=  0.66] at ( 51.14,410.32) {1949%
};

\node[color=drawColor,anchor=base,inner sep=0pt, outer sep=0pt, scale=  0.66] at ( 75.87,410.32) {1953%
};

\node[color=drawColor,anchor=base,inner sep=0pt, outer sep=0pt, scale=  0.66] at (100.60,410.32) {1957%
};

\node[color=drawColor,anchor=base,inner sep=0pt, outer sep=0pt, scale=  0.66] at (125.33,410.32) {1961%
};

\node[color=drawColor,anchor=base,inner sep=0pt, outer sep=0pt, scale=  0.66] at (150.06,410.32) {1965%
};

\node[color=drawColor,anchor=base,inner sep=0pt, outer sep=0pt, scale=  0.66] at (174.79,410.32) {1969%
};

\node[color=drawColor,anchor=base,inner sep=0pt, outer sep=0pt, scale=  0.66] at (199.52,410.32) {1973%
};

\node[color=drawColor,anchor=base,inner sep=0pt, outer sep=0pt, scale=  0.66] at (224.25,410.32) {1977%
};

\node[color=drawColor,anchor=base,inner sep=0pt, outer sep=0pt, scale=  0.66] at (248.98,410.32) {1981%
};

\node[color=drawColor,anchor=base,inner sep=0pt, outer sep=0pt, scale=  0.66] at (273.71,410.32) {1985%
};

\node[color=drawColor,anchor=base,inner sep=0pt, outer sep=0pt, scale=  0.66] at (298.44,410.32) {1989%
};

\node[color=drawColor,anchor=base,inner sep=0pt, outer sep=0pt, scale=  0.66] at (323.17,410.32) {1993%
};

\node[color=drawColor,anchor=base,inner sep=0pt, outer sep=0pt, scale=  0.66] at (347.90,410.32) {1997%
};

\node[color=drawColor,anchor=base,inner sep=0pt, outer sep=0pt, scale=  0.66] at (372.63,410.32) {2001%
};

\node[color=drawColor,anchor=base,inner sep=0pt, outer sep=0pt, scale=  0.66] at (397.36,410.32) {2005%
};

\node[color=drawColor,anchor=base,inner sep=0pt, outer sep=0pt, scale=  0.66] at (422.09,410.32) {2009%
};

\draw[color=drawColor,line cap=round,line join=round,fill opacity=0.00,] ( 32.47,428.21) -- ( 32.47,577.21);

\draw[color=drawColor,line cap=round,line join=round,fill opacity=0.00,] ( 32.47,428.21) -- ( 28.51,428.21);

\draw[color=drawColor,line cap=round,line join=round,fill opacity=0.00,] ( 32.47,477.88) -- ( 28.51,477.88);

\draw[color=drawColor,line cap=round,line join=round,fill opacity=0.00,] ( 32.47,527.54) -- ( 28.51,527.54);

\draw[color=drawColor,line cap=round,line join=round,fill opacity=0.00,] ( 32.47,577.21) -- ( 28.51,577.21);

\node[rotate= 90.00,color=drawColor,anchor=base,inner sep=0pt, outer sep=0pt, scale=  0.66] at ( 24.55,428.21) {0%
};

\node[rotate= 90.00,color=drawColor,anchor=base,inner sep=0pt, outer sep=0pt, scale=  0.66] at ( 24.55,477.88) {500%
};

\node[rotate= 90.00,color=drawColor,anchor=base,inner sep=0pt, outer sep=0pt, scale=  0.66] at ( 24.55,527.54) {1000%
};

\node[rotate= 90.00,color=drawColor,anchor=base,inner sep=0pt, outer sep=0pt, scale=  0.66] at ( 24.55,577.21) {1500%
};
\end{scope}
\begin{scope}
\path[clip] (  0.00,409.53) rectangle (469.75,614.29);
\definecolor[named]{drawColor}{rgb}{0.00,0.00,0.00}

\node[rotate= 90.00,color=drawColor,anchor=base,inner sep=0pt, outer sep=0pt, scale=  0.66] at (  8.71,507.95) {FLow [KAF]%
};
\end{scope}
\begin{scope}
\path[clip] (  0.00,  0.00) rectangle (469.75,614.29);
\definecolor[named]{drawColor}{rgb}{0.00,0.00,0.00}

\draw[color=drawColor,line cap=round,line join=round,fill opacity=0.00,] ( 32.47,426.16) --
	(453.12,426.16) --
	(453.12,589.74) --
	( 32.47,589.74) --
	( 32.47,426.16);
\end{scope}
\begin{scope}
\path[clip] ( 32.47,426.16) rectangle (453.12,589.74);
\definecolor[named]{drawColor}{rgb}{1.00,0.00,0.00}

\draw[color=drawColor,line cap=round,line join=round,fill opacity=0.00,] ( 51.95,500.75) -- ( 56.52,478.90);

\draw[color=drawColor,line cap=round,line join=round,fill opacity=0.00,] ( 64.03,479.70) -- ( 69.17,518.52);

\draw[color=drawColor,line cap=round,line join=round,fill opacity=0.00,] ( 70.40,518.55) -- ( 75.16,492.46);

\draw[color=drawColor,line cap=round,line join=round,fill opacity=0.00,] ( 76.43,484.65) -- ( 81.49,449.30);

\draw[color=drawColor,line cap=round,line join=round,fill opacity=0.00,] ( 83.31,449.14) -- ( 86.98,460.15);

\draw[color=drawColor,line cap=round,line join=round,fill opacity=0.00,] ( 91.03,466.71) -- ( 91.62,467.30);

\draw[color=drawColor,line cap=round,line join=round,fill opacity=0.00,] ( 94.69,474.06) -- (100.34,557.89);

\draw[color=drawColor,line cap=round,line join=round,fill opacity=0.00,] (100.99,557.90) -- (106.40,502.99);

\draw[color=drawColor,line cap=round,line join=round,fill opacity=0.00,] (107.61,495.17) -- (112.15,473.78);

\draw[color=drawColor,line cap=round,line join=round,fill opacity=0.00,] (115.67,472.80) -- (116.45,473.63);

\draw[color=drawColor,line cap=round,line join=round,fill opacity=0.00,] (121.09,473.07) -- (123.39,469.00);

\draw[color=drawColor,line cap=round,line join=round,fill opacity=0.00,] (126.35,469.38) -- (130.49,484.90);

\draw[color=drawColor,line cap=round,line join=round,fill opacity=0.00,] (132.10,484.81) -- (137.11,451.29);

\draw[color=drawColor,line cap=round,line join=round,fill opacity=0.00,] (138.67,451.22) -- (142.90,467.86);

\draw[color=drawColor,line cap=round,line join=round,fill opacity=0.00,] (144.49,475.61) -- (149.45,507.07);

\draw[color=drawColor,line cap=round,line join=round,fill opacity=0.00,] (150.51,507.05) -- (155.79,461.30);

\draw[color=drawColor,line cap=round,line join=round,fill opacity=0.00,] (158.35,460.72) -- (160.32,463.83);

\draw[color=drawColor,line cap=round,line join=round,fill opacity=0.00,] (163.18,471.07) -- (167.85,495.18);

\draw[color=drawColor,line cap=round,line join=round,fill opacity=0.00,] (169.42,495.20) -- (173.98,473.51);

\draw[color=drawColor,line cap=round,line join=round,fill opacity=0.00,] (175.90,473.44) -- (179.86,486.95);

\draw[color=drawColor,line cap=round,line join=round,fill opacity=0.00,] (184.01,488.21) -- (184.12,488.11);

\draw[color=drawColor,line cap=round,line join=round,fill opacity=0.00,] (188.46,481.83) -- (192.03,471.66);

\draw[color=drawColor,line cap=round,line join=round,fill opacity=0.00,] (193.93,471.83) -- (198.93,504.91);

\draw[color=drawColor,line cap=round,line join=round,fill opacity=0.00,] (200.11,504.90) -- (205.12,471.40);

\draw[color=drawColor,line cap=round,line join=round,fill opacity=0.00,] (206.38,471.39) -- (211.21,499.07);

\draw[color=drawColor,line cap=round,line join=round,fill opacity=0.00,] (212.52,499.07) -- (217.44,468.52);

\draw[color=drawColor,line cap=round,line join=round,fill opacity=0.00,] (219.02,460.77) -- (223.29,443.61);

\draw[color=drawColor,line cap=round,line join=round,fill opacity=0.00,] (224.61,443.71) -- (230.07,503.91);

\draw[color=drawColor,line cap=round,line join=round,fill opacity=0.00,] (233.42,510.46) -- (233.63,510.65);

\draw[color=drawColor,line cap=round,line join=round,fill opacity=0.00,] (243.20,510.48) -- (248.57,458.28);

\draw[color=drawColor,line cap=round,line join=round,fill opacity=0.00,] (249.59,458.25) -- (254.56,490.36);

\draw[color=drawColor,line cap=round,line join=round,fill opacity=0.00,] (255.61,498.20) -- (260.90,544.49);

\draw[color=drawColor,line cap=round,line join=round,fill opacity=0.00,] (264.03,545.52) -- (264.85,544.63);

\draw[color=drawColor,line cap=round,line join=round,fill opacity=0.00,] (268.28,537.82) -- (272.96,513.58);

\draw[color=drawColor,line cap=round,line join=round,fill opacity=0.00,] (280.76,504.90) -- (285.20,485.19);

\draw[color=drawColor,line cap=round,line join=round,fill opacity=0.00,] (287.35,477.57) -- (290.99,466.84);

\draw[color=drawColor,line cap=round,line join=round,fill opacity=0.00,] (294.90,460.14) -- (295.79,459.15);

\draw[color=drawColor,line cap=round,line join=round,fill opacity=0.00,] (300.80,459.38) -- (302.26,461.35);

\draw[color=drawColor,line cap=round,line join=round,fill opacity=0.00,] (306.33,468.10) -- (309.10,473.90);

\draw[color=drawColor,line cap=round,line join=round,fill opacity=0.00,] (312.71,474.00) -- (315.08,469.70);

\draw[color=drawColor,line cap=round,line join=round,fill opacity=0.00,] (317.49,470.16) -- (322.67,510.58);

\draw[color=drawColor,line cap=round,line join=round,fill opacity=0.00,] (323.70,510.59) -- (328.82,472.45);

\draw[color=drawColor,line cap=round,line join=round,fill opacity=0.00,] (329.64,472.47) -- (335.24,549.10);

\draw[color=drawColor,line cap=round,line join=round,fill opacity=0.00,] (335.84,549.10) -- (341.41,477.53);

\draw[color=drawColor,line cap=round,line join=round,fill opacity=0.00,] (342.29,477.50) -- (347.33,512.14);

\draw[color=drawColor,line cap=round,line join=round,fill opacity=0.00,] (348.42,512.14) -- (353.56,473.64);

\draw[color=drawColor,line cap=round,line join=round,fill opacity=0.00,] (355.99,473.19) -- (358.36,477.51);

\draw[color=drawColor,line cap=round,line join=round,fill opacity=0.00,] (361.17,477.13) -- (365.54,458.43);

\draw[color=drawColor,line cap=round,line join=round,fill opacity=0.00,] (373.94,453.50) -- (377.50,443.35);

\draw[color=drawColor,line cap=round,line join=round,fill opacity=0.00,] (379.88,443.43) -- (383.92,457.81);

\draw[color=drawColor,line cap=round,line join=round,fill opacity=0.00,] (387.46,458.52) -- (388.71,456.95);

\draw[color=drawColor,line cap=round,line join=round,fill opacity=0.00,] (392.01,457.72) -- (396.52,478.49);

\draw[color=drawColor,line cap=round,line join=round,fill opacity=0.00,] (398.35,478.52) -- (402.55,462.34);

\draw[color=drawColor,line cap=round,line join=round,fill opacity=0.00,] ( 51.14,504.62) circle (  0.89);

\draw[color=drawColor,line cap=round,line join=round,fill opacity=0.00,] ( 57.33,475.02) circle (  0.89);

\draw[color=drawColor,line cap=round,line join=round,fill opacity=0.00,] ( 63.51,475.78) circle (  0.89);

\draw[color=drawColor,line cap=round,line join=round,fill opacity=0.00,] ( 69.69,522.45) circle (  0.89);

\draw[color=drawColor,line cap=round,line join=round,fill opacity=0.00,] ( 75.87,488.57) circle (  0.89);

\draw[color=drawColor,line cap=round,line join=round,fill opacity=0.00,] ( 82.05,445.38) circle (  0.89);

\draw[color=drawColor,line cap=round,line join=round,fill opacity=0.00,] ( 88.24,463.91) circle (  0.89);

\draw[color=drawColor,line cap=round,line join=round,fill opacity=0.00,] ( 94.42,470.11) circle (  0.89);

\draw[color=drawColor,line cap=round,line join=round,fill opacity=0.00,] (100.60,561.84) circle (  0.89);

\draw[color=drawColor,line cap=round,line join=round,fill opacity=0.00,] (106.78,499.05) circle (  0.89);

\draw[color=drawColor,line cap=round,line join=round,fill opacity=0.00,] (112.97,469.91) circle (  0.89);

\draw[color=drawColor,line cap=round,line join=round,fill opacity=0.00,] (119.15,476.52) circle (  0.89);

\draw[color=drawColor,line cap=round,line join=round,fill opacity=0.00,] (125.33,465.55) circle (  0.89);

\draw[color=drawColor,line cap=round,line join=round,fill opacity=0.00,] (131.51,488.73) circle (  0.89);

\draw[color=drawColor,line cap=round,line join=round,fill opacity=0.00,] (137.70,447.38) circle (  0.89);

\draw[color=drawColor,line cap=round,line join=round,fill opacity=0.00,] (143.88,471.70) circle (  0.89);

\draw[color=drawColor,line cap=round,line join=round,fill opacity=0.00,] (150.06,510.99) circle (  0.89);

\draw[color=drawColor,line cap=round,line join=round,fill opacity=0.00,] (156.24,457.37) circle (  0.89);

\draw[color=drawColor,line cap=round,line join=round,fill opacity=0.00,] (162.43,467.19) circle (  0.89);

\draw[color=drawColor,line cap=round,line join=round,fill opacity=0.00,] (168.61,499.07) circle (  0.89);

\draw[color=drawColor,line cap=round,line join=round,fill opacity=0.00,] (174.79,469.64) circle (  0.89);

\draw[color=drawColor,line cap=round,line join=round,fill opacity=0.00,] (180.97,490.75) circle (  0.89);

\draw[color=drawColor,line cap=round,line join=round,fill opacity=0.00,] (187.16,485.57) circle (  0.89);

\draw[color=drawColor,line cap=round,line join=round,fill opacity=0.00,] (193.34,467.92) circle (  0.89);

\draw[color=drawColor,line cap=round,line join=round,fill opacity=0.00,] (199.52,508.82) circle (  0.89);

\draw[color=drawColor,line cap=round,line join=round,fill opacity=0.00,] (205.70,467.49) circle (  0.89);

\draw[color=drawColor,line cap=round,line join=round,fill opacity=0.00,] (211.89,502.98) circle (  0.89);

\draw[color=drawColor,line cap=round,line join=round,fill opacity=0.00,] (218.07,464.61) circle (  0.89);

\draw[color=drawColor,line cap=round,line join=round,fill opacity=0.00,] (224.25,439.77) circle (  0.89);

\draw[color=drawColor,line cap=round,line join=round,fill opacity=0.00,] (230.43,507.86) circle (  0.89);

\draw[color=drawColor,line cap=round,line join=round,fill opacity=0.00,] (236.62,513.26) circle (  0.89);

\draw[color=drawColor,line cap=round,line join=round,fill opacity=0.00,] (242.80,514.42) circle (  0.89);

\draw[color=drawColor,line cap=round,line join=round,fill opacity=0.00,] (248.98,454.34) circle (  0.89);

\draw[color=drawColor,line cap=round,line join=round,fill opacity=0.00,] (255.16,494.27) circle (  0.89);

\draw[color=drawColor,line cap=round,line join=round,fill opacity=0.00,] (261.34,548.43) circle (  0.89);

\draw[color=drawColor,line cap=round,line join=round,fill opacity=0.00,] (267.53,541.71) circle (  0.89);

\draw[color=drawColor,line cap=round,line join=round,fill opacity=0.00,] (273.71,509.69) circle (  0.89);

\draw[color=drawColor,line cap=round,line join=round,fill opacity=0.00,] (279.89,508.76) circle (  0.89);

\draw[color=drawColor,line cap=round,line join=round,fill opacity=0.00,] (286.07,481.32) circle (  0.89);

\draw[color=drawColor,line cap=round,line join=round,fill opacity=0.00,] (292.26,463.09) circle (  0.89);

\draw[color=drawColor,line cap=round,line join=round,fill opacity=0.00,] (298.44,456.20) circle (  0.89);

\draw[color=drawColor,line cap=round,line join=round,fill opacity=0.00,] (304.62,464.53) circle (  0.89);

\draw[color=drawColor,line cap=round,line join=round,fill opacity=0.00,] (310.80,477.47) circle (  0.89);

\draw[color=drawColor,line cap=round,line join=round,fill opacity=0.00,] (316.99,466.24) circle (  0.89);

\draw[color=drawColor,line cap=round,line join=round,fill opacity=0.00,] (323.17,514.51) circle (  0.89);

\draw[color=drawColor,line cap=round,line join=round,fill opacity=0.00,] (329.35,468.53) circle (  0.89);

\draw[color=drawColor,line cap=round,line join=round,fill opacity=0.00,] (335.53,553.05) circle (  0.89);

\draw[color=drawColor,line cap=round,line join=round,fill opacity=0.00,] (341.72,473.59) circle (  0.89);

\draw[color=drawColor,line cap=round,line join=round,fill opacity=0.00,] (347.90,516.06) circle (  0.89);

\draw[color=drawColor,line cap=round,line join=round,fill opacity=0.00,] (354.08,469.71) circle (  0.89);

\draw[color=drawColor,line cap=round,line join=round,fill opacity=0.00,] (360.26,480.99) circle (  0.89);

\draw[color=drawColor,line cap=round,line join=round,fill opacity=0.00,] (366.45,454.58) circle (  0.89);

\draw[color=drawColor,line cap=round,line join=round,fill opacity=0.00,] (372.63,457.23) circle (  0.89);

\draw[color=drawColor,line cap=round,line join=round,fill opacity=0.00,] (378.81,439.61) circle (  0.89);

\draw[color=drawColor,line cap=round,line join=round,fill opacity=0.00,] (384.99,461.62) circle (  0.89);

\draw[color=drawColor,line cap=round,line join=round,fill opacity=0.00,] (391.18,453.85) circle (  0.89);

\draw[color=drawColor,line cap=round,line join=round,fill opacity=0.00,] (397.36,482.36) circle (  0.89);

\draw[color=drawColor,line cap=round,line join=round,fill opacity=0.00,] (403.54,458.51) circle (  0.89);

\draw[color=drawColor,line cap=round,line join=round,fill opacity=0.00,] (409.72,461.38) circle (  0.89);
\end{scope}
\begin{scope}
\path[clip] (  0.00,  0.00) rectangle (469.75,614.29);
\definecolor[named]{drawColor}{rgb}{0.00,0.00,0.00}

\node[color=drawColor,anchor=base,inner sep=0pt, outer sep=0pt, scale=  1.00] at (242.80,597.66) {(a) RPSS = 0.76 MC = 0.75%
};
\end{scope}
\begin{scope}
\path[clip] ( 32.47,221.40) rectangle (453.12,384.98);
\end{scope}
\begin{scope}
\path[clip] ( 32.47,221.40) rectangle (453.12,384.98);
\definecolor[named]{drawColor}{rgb}{0.00,0.00,0.00}

\draw[color=drawColor,line width= 1.2pt,line join=round,fill opacity=0.00,] ( 48.67,304.12) -- ( 53.62,304.12);

\draw[color=drawColor,dash pattern=on 4pt off 4pt ,line cap=round,line join=round,fill opacity=0.00,] ( 51.14,279.97) -- ( 51.14,294.82);

\draw[color=drawColor,dash pattern=on 4pt off 4pt ,line cap=round,line join=round,fill opacity=0.00,] ( 51.14,317.98) -- ( 51.14,310.16);

\draw[color=drawColor,line cap=round,line join=round,fill opacity=0.00,] ( 49.91,279.97) -- ( 52.38,279.97);

\draw[color=drawColor,line cap=round,line join=round,fill opacity=0.00,] ( 49.91,317.98) -- ( 52.38,317.98);

\draw[color=drawColor,line cap=round,line join=round,fill opacity=0.00,] ( 48.67,294.82) --
	( 53.62,294.82) --
	( 53.62,310.16) --
	( 48.67,310.16) --
	( 48.67,294.82);

\draw[color=drawColor,line width= 1.2pt,line join=round,fill opacity=0.00,] ( 54.85,273.11) -- ( 59.80,273.11);

\draw[color=drawColor,dash pattern=on 4pt off 4pt ,line cap=round,line join=round,fill opacity=0.00,] ( 57.33,257.88) -- ( 57.33,264.80);

\draw[color=drawColor,dash pattern=on 4pt off 4pt ,line cap=round,line join=round,fill opacity=0.00,] ( 57.33,304.74) -- ( 57.33,291.34);

\draw[color=drawColor,line cap=round,line join=round,fill opacity=0.00,] ( 56.09,257.88) -- ( 58.56,257.88);

\draw[color=drawColor,line cap=round,line join=round,fill opacity=0.00,] ( 56.09,304.74) -- ( 58.56,304.74);

\draw[color=drawColor,line cap=round,line join=round,fill opacity=0.00,] ( 54.85,264.80) --
	( 59.80,264.80) --
	( 59.80,291.34) --
	( 54.85,291.34) --
	( 54.85,264.80);

\draw[color=drawColor,line width= 1.2pt,line join=round,fill opacity=0.00,] ( 61.03,265.39) -- ( 65.98,265.39);

\draw[color=drawColor,dash pattern=on 4pt off 4pt ,line cap=round,line join=round,fill opacity=0.00,] ( 63.51,250.75) -- ( 63.51,259.46);

\draw[color=drawColor,dash pattern=on 4pt off 4pt ,line cap=round,line join=round,fill opacity=0.00,] ( 63.51,294.97) -- ( 63.51,272.22);

\draw[color=drawColor,line cap=round,line join=round,fill opacity=0.00,] ( 62.27,250.75) -- ( 64.74,250.75);

\draw[color=drawColor,line cap=round,line join=round,fill opacity=0.00,] ( 62.27,294.97) -- ( 64.74,294.97);

\draw[color=drawColor,line cap=round,line join=round,fill opacity=0.00,] ( 61.03,259.46) --
	( 65.98,259.46) --
	( 65.98,272.22) --
	( 61.03,272.22) --
	( 61.03,259.46);

\draw[color=drawColor,line width= 1.2pt,line join=round,fill opacity=0.00,] ( 67.22,324.94) -- ( 72.16,324.94);

\draw[color=drawColor,dash pattern=on 4pt off 4pt ,line cap=round,line join=round,fill opacity=0.00,] ( 69.69,305.05) -- ( 69.69,314.47);

\draw[color=drawColor,dash pattern=on 4pt off 4pt ,line cap=round,line join=round,fill opacity=0.00,] ( 69.69,375.53) -- ( 69.69,349.06);

\draw[color=drawColor,line cap=round,line join=round,fill opacity=0.00,] ( 68.45,305.05) -- ( 70.93,305.05);

\draw[color=drawColor,line cap=round,line join=round,fill opacity=0.00,] ( 68.45,375.53) -- ( 70.93,375.53);

\draw[color=drawColor,line cap=round,line join=round,fill opacity=0.00,] ( 67.22,314.47) --
	( 72.16,314.47) --
	( 72.16,349.06) --
	( 67.22,349.06) --
	( 67.22,314.47);

\draw[color=drawColor,line width= 1.2pt,line join=round,fill opacity=0.00,] ( 73.40,264.50) -- ( 78.35,264.50);

\draw[color=drawColor,dash pattern=on 4pt off 4pt ,line cap=round,line join=round,fill opacity=0.00,] ( 75.87,252.11) -- ( 75.87,260.75);

\draw[color=drawColor,dash pattern=on 4pt off 4pt ,line cap=round,line join=round,fill opacity=0.00,] ( 75.87,288.38) -- ( 75.87,272.39);

\draw[color=drawColor,line cap=round,line join=round,fill opacity=0.00,] ( 74.64,252.11) -- ( 77.11,252.11);

\draw[color=drawColor,line cap=round,line join=round,fill opacity=0.00,] ( 74.64,288.38) -- ( 77.11,288.38);

\draw[color=drawColor,line cap=round,line join=round,fill opacity=0.00,] ( 73.40,260.75) --
	( 78.35,260.75) --
	( 78.35,272.39) --
	( 73.40,272.39) --
	( 73.40,260.75);

\draw[color=drawColor,line width= 1.2pt,line join=round,fill opacity=0.00,] ( 79.58,259.06) -- ( 84.53,259.06);

\draw[color=drawColor,dash pattern=on 4pt off 4pt ,line cap=round,line join=round,fill opacity=0.00,] ( 82.05,247.27) -- ( 82.05,251.79);

\draw[color=drawColor,dash pattern=on 4pt off 4pt ,line cap=round,line join=round,fill opacity=0.00,] ( 82.05,285.18) -- ( 82.05,264.62);

\draw[color=drawColor,line cap=round,line join=round,fill opacity=0.00,] ( 80.82,247.27) -- ( 83.29,247.27);

\draw[color=drawColor,line cap=round,line join=round,fill opacity=0.00,] ( 80.82,285.18) -- ( 83.29,285.18);

\draw[color=drawColor,line cap=round,line join=round,fill opacity=0.00,] ( 79.58,251.79) --
	( 84.53,251.79) --
	( 84.53,264.62) --
	( 79.58,264.62) --
	( 79.58,251.79);

\draw[color=drawColor,line width= 1.2pt,line join=round,fill opacity=0.00,] ( 85.76,257.55) -- ( 90.71,257.55);

\draw[color=drawColor,dash pattern=on 4pt off 4pt ,line cap=round,line join=round,fill opacity=0.00,] ( 88.24,246.01) -- ( 88.24,250.40);

\draw[color=drawColor,dash pattern=on 4pt off 4pt ,line cap=round,line join=round,fill opacity=0.00,] ( 88.24,272.40) -- ( 88.24,261.45);

\draw[color=drawColor,line cap=round,line join=round,fill opacity=0.00,] ( 87.00,246.01) -- ( 89.47,246.01);

\draw[color=drawColor,line cap=round,line join=round,fill opacity=0.00,] ( 87.00,272.40) -- ( 89.47,272.40);

\draw[color=drawColor,line cap=round,line join=round,fill opacity=0.00,] ( 85.76,250.40) --
	( 90.71,250.40) --
	( 90.71,261.45) --
	( 85.76,261.45) --
	( 85.76,250.40);

\draw[color=drawColor,line width= 1.2pt,line join=round,fill opacity=0.00,] ( 91.95,294.72) -- ( 96.89,294.72);

\draw[color=drawColor,dash pattern=on 4pt off 4pt ,line cap=round,line join=round,fill opacity=0.00,] ( 94.42,262.40) -- ( 94.42,272.55);

\draw[color=drawColor,dash pattern=on 4pt off 4pt ,line cap=round,line join=round,fill opacity=0.00,] ( 94.42,320.33) -- ( 94.42,310.29);

\draw[color=drawColor,line cap=round,line join=round,fill opacity=0.00,] ( 93.18,262.40) -- ( 95.66,262.40);

\draw[color=drawColor,line cap=round,line join=round,fill opacity=0.00,] ( 93.18,320.33) -- ( 95.66,320.33);

\draw[color=drawColor,line cap=round,line join=round,fill opacity=0.00,] ( 91.95,272.55) --
	( 96.89,272.55) --
	( 96.89,310.29) --
	( 91.95,310.29) --
	( 91.95,272.55);

\draw[color=drawColor,line width= 1.2pt,line join=round,fill opacity=0.00,] ( 98.13,293.73) -- (103.08,293.73);

\draw[color=drawColor,dash pattern=on 4pt off 4pt ,line cap=round,line join=round,fill opacity=0.00,] (100.60,263.54) -- (100.60,272.33);

\draw[color=drawColor,dash pattern=on 4pt off 4pt ,line cap=round,line join=round,fill opacity=0.00,] (100.60,310.98) -- (100.60,302.89);

\draw[color=drawColor,line cap=round,line join=round,fill opacity=0.00,] ( 99.37,263.54) -- (101.84,263.54);

\draw[color=drawColor,line cap=round,line join=round,fill opacity=0.00,] ( 99.37,310.98) -- (101.84,310.98);

\draw[color=drawColor,line cap=round,line join=round,fill opacity=0.00,] ( 98.13,272.33) --
	(103.08,272.33) --
	(103.08,302.89) --
	( 98.13,302.89) --
	( 98.13,272.33);

\draw[color=drawColor,line width= 1.2pt,line join=round,fill opacity=0.00,] (104.31,304.93) -- (109.26,304.93);

\draw[color=drawColor,dash pattern=on 4pt off 4pt ,line cap=round,line join=round,fill opacity=0.00,] (106.78,266.14) -- (106.78,286.57);

\draw[color=drawColor,dash pattern=on 4pt off 4pt ,line cap=round,line join=round,fill opacity=0.00,] (106.78,320.26) -- (106.78,313.72);

\draw[color=drawColor,line cap=round,line join=round,fill opacity=0.00,] (105.55,266.14) -- (108.02,266.14);

\draw[color=drawColor,line cap=round,line join=round,fill opacity=0.00,] (105.55,320.26) -- (108.02,320.26);

\draw[color=drawColor,line cap=round,line join=round,fill opacity=0.00,] (104.31,286.57) --
	(109.26,286.57) --
	(109.26,313.72) --
	(104.31,313.72) --
	(104.31,286.57);

\draw[color=drawColor,line width= 1.2pt,line join=round,fill opacity=0.00,] (110.49,255.80) -- (115.44,255.80);

\draw[color=drawColor,dash pattern=on 4pt off 4pt ,line cap=round,line join=round,fill opacity=0.00,] (112.97,245.51) -- (112.97,250.15);

\draw[color=drawColor,dash pattern=on 4pt off 4pt ,line cap=round,line join=round,fill opacity=0.00,] (112.97,266.99) -- (112.97,260.96);

\draw[color=drawColor,line cap=round,line join=round,fill opacity=0.00,] (111.73,245.51) -- (114.20,245.51);

\draw[color=drawColor,line cap=round,line join=round,fill opacity=0.00,] (111.73,266.99) -- (114.20,266.99);

\draw[color=drawColor,line cap=round,line join=round,fill opacity=0.00,] (110.49,250.15) --
	(115.44,250.15) --
	(115.44,260.96) --
	(110.49,260.96) --
	(110.49,250.15);

\draw[color=drawColor,line width= 1.2pt,line join=round,fill opacity=0.00,] (116.68,261.42) -- (121.62,261.42);

\draw[color=drawColor,dash pattern=on 4pt off 4pt ,line cap=round,line join=round,fill opacity=0.00,] (119.15,248.57) -- (119.15,255.33);

\draw[color=drawColor,dash pattern=on 4pt off 4pt ,line cap=round,line join=round,fill opacity=0.00,] (119.15,286.18) -- (119.15,267.02);

\draw[color=drawColor,line cap=round,line join=round,fill opacity=0.00,] (117.91,248.57) -- (120.39,248.57);

\draw[color=drawColor,line cap=round,line join=round,fill opacity=0.00,] (117.91,286.18) -- (120.39,286.18);

\draw[color=drawColor,line cap=round,line join=round,fill opacity=0.00,] (116.68,255.33) --
	(121.62,255.33) --
	(121.62,267.02) --
	(116.68,267.02) --
	(116.68,255.33);

\draw[color=drawColor,line width= 1.2pt,line join=round,fill opacity=0.00,] (122.86,256.40) -- (127.80,256.40);

\draw[color=drawColor,dash pattern=on 4pt off 4pt ,line cap=round,line join=round,fill opacity=0.00,] (125.33,245.93) -- (125.33,250.42);

\draw[color=drawColor,dash pattern=on 4pt off 4pt ,line cap=round,line join=round,fill opacity=0.00,] (125.33,270.46) -- (125.33,261.59);

\draw[color=drawColor,line cap=round,line join=round,fill opacity=0.00,] (124.10,245.93) -- (126.57,245.93);

\draw[color=drawColor,line cap=round,line join=round,fill opacity=0.00,] (124.10,270.46) -- (126.57,270.46);

\draw[color=drawColor,line cap=round,line join=round,fill opacity=0.00,] (122.86,250.42) --
	(127.80,250.42) --
	(127.80,261.59) --
	(122.86,261.59) --
	(122.86,250.42);

\draw[color=drawColor,line width= 1.2pt,line join=round,fill opacity=0.00,] (129.04,302.91) -- (133.99,302.91);

\draw[color=drawColor,dash pattern=on 4pt off 4pt ,line cap=round,line join=round,fill opacity=0.00,] (131.51,266.07) -- (131.51,284.77);

\draw[color=drawColor,dash pattern=on 4pt off 4pt ,line cap=round,line join=round,fill opacity=0.00,] (131.51,318.33) -- (131.51,309.83);

\draw[color=drawColor,line cap=round,line join=round,fill opacity=0.00,] (130.28,266.07) -- (132.75,266.07);

\draw[color=drawColor,line cap=round,line join=round,fill opacity=0.00,] (130.28,318.33) -- (132.75,318.33);

\draw[color=drawColor,line cap=round,line join=round,fill opacity=0.00,] (129.04,284.77) --
	(133.99,284.77) --
	(133.99,309.83) --
	(129.04,309.83) --
	(129.04,284.77);

\draw[color=drawColor,line width= 1.2pt,line join=round,fill opacity=0.00,] (135.22,254.56) -- (140.17,254.56);

\draw[color=drawColor,dash pattern=on 4pt off 4pt ,line cap=round,line join=round,fill opacity=0.00,] (137.70,238.98) -- (137.70,247.93);

\draw[color=drawColor,dash pattern=on 4pt off 4pt ,line cap=round,line join=round,fill opacity=0.00,] (137.70,265.60) -- (137.70,259.78);

\draw[color=drawColor,line cap=round,line join=round,fill opacity=0.00,] (136.46,238.98) -- (138.93,238.98);

\draw[color=drawColor,line cap=round,line join=round,fill opacity=0.00,] (136.46,265.60) -- (138.93,265.60);

\draw[color=drawColor,line cap=round,line join=round,fill opacity=0.00,] (135.22,247.93) --
	(140.17,247.93) --
	(140.17,259.78) --
	(135.22,259.78) --
	(135.22,247.93);

\draw[color=drawColor,line width= 1.2pt,line join=round,fill opacity=0.00,] (141.41,254.72) -- (146.35,254.72);

\draw[color=drawColor,dash pattern=on 4pt off 4pt ,line cap=round,line join=round,fill opacity=0.00,] (143.88,239.42) -- (143.88,247.53);

\draw[color=drawColor,dash pattern=on 4pt off 4pt ,line cap=round,line join=round,fill opacity=0.00,] (143.88,266.01) -- (143.88,259.60);

\draw[color=drawColor,line cap=round,line join=round,fill opacity=0.00,] (142.64,239.42) -- (145.12,239.42);

\draw[color=drawColor,line cap=round,line join=round,fill opacity=0.00,] (142.64,266.01) -- (145.12,266.01);

\draw[color=drawColor,line cap=round,line join=round,fill opacity=0.00,] (141.41,247.53) --
	(146.35,247.53) --
	(146.35,259.60) --
	(141.41,259.60) --
	(141.41,247.53);

\draw[color=drawColor,line width= 1.2pt,line join=round,fill opacity=0.00,] (147.59,311.85) -- (152.53,311.85);

\draw[color=drawColor,dash pattern=on 4pt off 4pt ,line cap=round,line join=round,fill opacity=0.00,] (150.06,278.72) -- (150.06,299.58);

\draw[color=drawColor,dash pattern=on 4pt off 4pt ,line cap=round,line join=round,fill opacity=0.00,] (150.06,365.09) -- (150.06,320.52);

\draw[color=drawColor,line cap=round,line join=round,fill opacity=0.00,] (148.82,278.72) -- (151.30,278.72);

\draw[color=drawColor,line cap=round,line join=round,fill opacity=0.00,] (148.82,365.09) -- (151.30,365.09);

\draw[color=drawColor,line cap=round,line join=round,fill opacity=0.00,] (147.59,299.58) --
	(152.53,299.58) --
	(152.53,320.52) --
	(147.59,320.52) --
	(147.59,299.58);

\draw[color=drawColor,line width= 1.2pt,line join=round,fill opacity=0.00,] (153.77,297.96) -- (158.72,297.96);

\draw[color=drawColor,dash pattern=on 4pt off 4pt ,line cap=round,line join=round,fill opacity=0.00,] (156.24,267.47) -- (156.24,282.66);

\draw[color=drawColor,dash pattern=on 4pt off 4pt ,line cap=round,line join=round,fill opacity=0.00,] (156.24,328.22) -- (156.24,310.68);

\draw[color=drawColor,line cap=round,line join=round,fill opacity=0.00,] (155.01,267.47) -- (157.48,267.47);

\draw[color=drawColor,line cap=round,line join=round,fill opacity=0.00,] (155.01,328.22) -- (157.48,328.22);

\draw[color=drawColor,line cap=round,line join=round,fill opacity=0.00,] (153.77,282.66) --
	(158.72,282.66) --
	(158.72,310.68) --
	(153.77,310.68) --
	(153.77,282.66);

\draw[color=drawColor,line width= 1.2pt,line join=round,fill opacity=0.00,] (159.95,279.77) -- (164.90,279.77);

\draw[color=drawColor,dash pattern=on 4pt off 4pt ,line cap=round,line join=round,fill opacity=0.00,] (162.43,264.14) -- (162.43,269.58);

\draw[color=drawColor,dash pattern=on 4pt off 4pt ,line cap=round,line join=round,fill opacity=0.00,] (162.43,303.21) -- (162.43,289.60);

\draw[color=drawColor,line cap=round,line join=round,fill opacity=0.00,] (161.19,264.14) -- (163.66,264.14);

\draw[color=drawColor,line cap=round,line join=round,fill opacity=0.00,] (161.19,303.21) -- (163.66,303.21);

\draw[color=drawColor,line cap=round,line join=round,fill opacity=0.00,] (159.95,269.58) --
	(164.90,269.58) --
	(164.90,289.60) --
	(159.95,289.60) --
	(159.95,269.58);

\draw[color=drawColor,line width= 1.2pt,line join=round,fill opacity=0.00,] (166.14,273.92) -- (171.08,273.92);

\draw[color=drawColor,dash pattern=on 4pt off 4pt ,line cap=round,line join=round,fill opacity=0.00,] (168.61,261.57) -- (168.61,266.77);

\draw[color=drawColor,dash pattern=on 4pt off 4pt ,line cap=round,line join=round,fill opacity=0.00,] (168.61,299.78) -- (168.61,282.73);

\draw[color=drawColor,line cap=round,line join=round,fill opacity=0.00,] (167.37,261.57) -- (169.85,261.57);

\draw[color=drawColor,line cap=round,line join=round,fill opacity=0.00,] (167.37,299.78) -- (169.85,299.78);

\draw[color=drawColor,line cap=round,line join=round,fill opacity=0.00,] (166.14,266.77) --
	(171.08,266.77) --
	(171.08,282.73) --
	(166.14,282.73) --
	(166.14,266.77);

\draw[color=drawColor,line width= 1.2pt,line join=round,fill opacity=0.00,] (172.32,305.50) -- (177.26,305.50);

\draw[color=drawColor,dash pattern=on 4pt off 4pt ,line cap=round,line join=round,fill opacity=0.00,] (174.79,279.96) -- (174.79,296.35);

\draw[color=drawColor,dash pattern=on 4pt off 4pt ,line cap=round,line join=round,fill opacity=0.00,] (174.79,319.61) -- (174.79,311.49);

\draw[color=drawColor,line cap=round,line join=round,fill opacity=0.00,] (173.55,279.96) -- (176.03,279.96);

\draw[color=drawColor,line cap=round,line join=round,fill opacity=0.00,] (173.55,319.61) -- (176.03,319.61);

\draw[color=drawColor,line cap=round,line join=round,fill opacity=0.00,] (172.32,296.35) --
	(177.26,296.35) --
	(177.26,311.49) --
	(172.32,311.49) --
	(172.32,296.35);

\draw[color=drawColor,line width= 1.2pt,line join=round,fill opacity=0.00,] (178.50,286.94) -- (183.45,286.94);

\draw[color=drawColor,dash pattern=on 4pt off 4pt ,line cap=round,line join=round,fill opacity=0.00,] (180.97,265.26) -- (180.97,277.81);

\draw[color=drawColor,dash pattern=on 4pt off 4pt ,line cap=round,line join=round,fill opacity=0.00,] (180.97,310.20) -- (180.97,300.17);

\draw[color=drawColor,line cap=round,line join=round,fill opacity=0.00,] (179.74,265.26) -- (182.21,265.26);

\draw[color=drawColor,line cap=round,line join=round,fill opacity=0.00,] (179.74,310.20) -- (182.21,310.20);

\draw[color=drawColor,line cap=round,line join=round,fill opacity=0.00,] (178.50,277.81) --
	(183.45,277.81) --
	(183.45,300.17) --
	(178.50,300.17) --
	(178.50,277.81);

\draw[color=drawColor,line width= 1.2pt,line join=round,fill opacity=0.00,] (184.68,293.16) -- (189.63,293.16);

\draw[color=drawColor,dash pattern=on 4pt off 4pt ,line cap=round,line join=round,fill opacity=0.00,] (187.16,265.95) -- (187.16,280.96);

\draw[color=drawColor,dash pattern=on 4pt off 4pt ,line cap=round,line join=round,fill opacity=0.00,] (187.16,312.41) -- (187.16,303.89);

\draw[color=drawColor,line cap=round,line join=round,fill opacity=0.00,] (185.92,265.95) -- (188.39,265.95);

\draw[color=drawColor,line cap=round,line join=round,fill opacity=0.00,] (185.92,312.41) -- (188.39,312.41);

\draw[color=drawColor,line cap=round,line join=round,fill opacity=0.00,] (184.68,280.96) --
	(189.63,280.96) --
	(189.63,303.89) --
	(184.68,303.89) --
	(184.68,280.96);

\draw[color=drawColor,line width= 1.2pt,line join=round,fill opacity=0.00,] (190.87,309.77) -- (195.81,309.77);

\draw[color=drawColor,dash pattern=on 4pt off 4pt ,line cap=round,line join=round,fill opacity=0.00,] (193.34,265.16) -- (193.34,286.79);

\draw[color=drawColor,dash pattern=on 4pt off 4pt ,line cap=round,line join=round,fill opacity=0.00,] (193.34,323.69) -- (193.34,317.08);

\draw[color=drawColor,line cap=round,line join=round,fill opacity=0.00,] (192.10,265.16) -- (194.57,265.16);

\draw[color=drawColor,line cap=round,line join=round,fill opacity=0.00,] (192.10,323.69) -- (194.57,323.69);

\draw[color=drawColor,line cap=round,line join=round,fill opacity=0.00,] (190.87,286.79) --
	(195.81,286.79) --
	(195.81,317.08) --
	(190.87,317.08) --
	(190.87,286.79);

\draw[color=drawColor,line width= 1.2pt,line join=round,fill opacity=0.00,] (197.05,306.30) -- (201.99,306.30);

\draw[color=drawColor,dash pattern=on 4pt off 4pt ,line cap=round,line join=round,fill opacity=0.00,] (199.52,280.23) -- (199.52,296.50);

\draw[color=drawColor,dash pattern=on 4pt off 4pt ,line cap=round,line join=round,fill opacity=0.00,] (199.52,319.77) -- (199.52,312.42);

\draw[color=drawColor,line cap=round,line join=round,fill opacity=0.00,] (198.28,280.23) -- (200.76,280.23);

\draw[color=drawColor,line cap=round,line join=round,fill opacity=0.00,] (198.28,319.77) -- (200.76,319.77);

\draw[color=drawColor,line cap=round,line join=round,fill opacity=0.00,] (197.05,296.50) --
	(201.99,296.50) --
	(201.99,312.42) --
	(197.05,312.42) --
	(197.05,296.50);

\draw[color=drawColor,line width= 1.2pt,line join=round,fill opacity=0.00,] (203.23,277.12) -- (208.18,277.12);

\draw[color=drawColor,dash pattern=on 4pt off 4pt ,line cap=round,line join=round,fill opacity=0.00,] (205.70,262.56) -- (205.70,267.33);

\draw[color=drawColor,dash pattern=on 4pt off 4pt ,line cap=round,line join=round,fill opacity=0.00,] (205.70,300.06) -- (205.70,287.22);

\draw[color=drawColor,line cap=round,line join=round,fill opacity=0.00,] (204.47,262.56) -- (206.94,262.56);

\draw[color=drawColor,line cap=round,line join=round,fill opacity=0.00,] (204.47,300.06) -- (206.94,300.06);

\draw[color=drawColor,line cap=round,line join=round,fill opacity=0.00,] (203.23,267.33) --
	(208.18,267.33) --
	(208.18,287.22) --
	(203.23,287.22) --
	(203.23,267.33);

\draw[color=drawColor,line width= 1.2pt,line join=round,fill opacity=0.00,] (209.41,269.76) -- (214.36,269.76);

\draw[color=drawColor,dash pattern=on 4pt off 4pt ,line cap=round,line join=round,fill opacity=0.00,] (211.89,250.65) -- (211.89,262.73);

\draw[color=drawColor,dash pattern=on 4pt off 4pt ,line cap=round,line join=round,fill opacity=0.00,] (211.89,289.83) -- (211.89,278.67);

\draw[color=drawColor,line cap=round,line join=round,fill opacity=0.00,] (210.65,250.65) -- (213.12,250.65);

\draw[color=drawColor,line cap=round,line join=round,fill opacity=0.00,] (210.65,289.83) -- (213.12,289.83);

\draw[color=drawColor,line cap=round,line join=round,fill opacity=0.00,] (209.41,262.73) --
	(214.36,262.73) --
	(214.36,278.67) --
	(209.41,278.67) --
	(209.41,262.73);

\draw[color=drawColor,line width= 1.2pt,line join=round,fill opacity=0.00,] (215.59,264.72) -- (220.54,264.72);

\draw[color=drawColor,dash pattern=on 4pt off 4pt ,line cap=round,line join=round,fill opacity=0.00,] (218.07,252.21) -- (218.07,262.05);

\draw[color=drawColor,dash pattern=on 4pt off 4pt ,line cap=round,line join=round,fill opacity=0.00,] (218.07,289.57) -- (218.07,272.46);

\draw[color=drawColor,line cap=round,line join=round,fill opacity=0.00,] (216.83,252.21) -- (219.30,252.21);

\draw[color=drawColor,line cap=round,line join=round,fill opacity=0.00,] (216.83,289.57) -- (219.30,289.57);

\draw[color=drawColor,line cap=round,line join=round,fill opacity=0.00,] (215.59,262.05) --
	(220.54,262.05) --
	(220.54,272.46) --
	(215.59,272.46) --
	(215.59,262.05);

\draw[color=drawColor,line width= 1.2pt,line join=round,fill opacity=0.00,] (221.78,238.83) -- (226.72,238.83);

\draw[color=drawColor,dash pattern=on 4pt off 4pt ,line cap=round,line join=round,fill opacity=0.00,] (224.25,227.46) -- (224.25,232.47);

\draw[color=drawColor,dash pattern=on 4pt off 4pt ,line cap=round,line join=round,fill opacity=0.00,] (224.25,260.56) -- (224.25,248.67);

\draw[color=drawColor,line cap=round,line join=round,fill opacity=0.00,] (223.01,227.46) -- (225.49,227.46);

\draw[color=drawColor,line cap=round,line join=round,fill opacity=0.00,] (223.01,260.56) -- (225.49,260.56);

\draw[color=drawColor,line cap=round,line join=round,fill opacity=0.00,] (221.78,232.47) --
	(226.72,232.47) --
	(226.72,248.67) --
	(221.78,248.67) --
	(221.78,232.47);

\draw[color=drawColor,line width= 1.2pt,line join=round,fill opacity=0.00,] (227.96,281.45) -- (232.91,281.45);

\draw[color=drawColor,dash pattern=on 4pt off 4pt ,line cap=round,line join=round,fill opacity=0.00,] (230.43,259.25) -- (230.43,268.29);

\draw[color=drawColor,dash pattern=on 4pt off 4pt ,line cap=round,line join=round,fill opacity=0.00,] (230.43,308.71) -- (230.43,295.27);

\draw[color=drawColor,line cap=round,line join=round,fill opacity=0.00,] (229.20,259.25) -- (231.67,259.25);

\draw[color=drawColor,line cap=round,line join=round,fill opacity=0.00,] (229.20,308.71) -- (231.67,308.71);

\draw[color=drawColor,line cap=round,line join=round,fill opacity=0.00,] (227.96,268.29) --
	(232.91,268.29) --
	(232.91,295.27) --
	(227.96,295.27) --
	(227.96,268.29);

\draw[color=drawColor,line width= 1.2pt,line join=round,fill opacity=0.00,] (234.14,294.47) -- (239.09,294.47);

\draw[color=drawColor,dash pattern=on 4pt off 4pt ,line cap=round,line join=round,fill opacity=0.00,] (236.62,267.70) -- (236.62,281.15);

\draw[color=drawColor,dash pattern=on 4pt off 4pt ,line cap=round,line join=round,fill opacity=0.00,] (236.62,310.57) -- (236.62,301.58);

\draw[color=drawColor,line cap=round,line join=round,fill opacity=0.00,] (235.38,267.70) -- (237.85,267.70);

\draw[color=drawColor,line cap=round,line join=round,fill opacity=0.00,] (235.38,310.57) -- (237.85,310.57);

\draw[color=drawColor,line cap=round,line join=round,fill opacity=0.00,] (234.14,281.15) --
	(239.09,281.15) --
	(239.09,301.58) --
	(234.14,301.58) --
	(234.14,281.15);

\draw[color=drawColor,line width= 1.2pt,line join=round,fill opacity=0.00,] (240.32,281.25) -- (245.27,281.25);

\draw[color=drawColor,dash pattern=on 4pt off 4pt ,line cap=round,line join=round,fill opacity=0.00,] (242.80,263.99) -- (242.80,269.84);

\draw[color=drawColor,dash pattern=on 4pt off 4pt ,line cap=round,line join=round,fill opacity=0.00,] (242.80,310.33) -- (242.80,293.45);

\draw[color=drawColor,line cap=round,line join=round,fill opacity=0.00,] (241.56,263.99) -- (244.03,263.99);

\draw[color=drawColor,line cap=round,line join=round,fill opacity=0.00,] (241.56,310.33) -- (244.03,310.33);

\draw[color=drawColor,line cap=round,line join=round,fill opacity=0.00,] (240.32,269.84) --
	(245.27,269.84) --
	(245.27,293.45) --
	(240.32,293.45) --
	(240.32,269.84);

\draw[color=drawColor,line width= 1.2pt,line join=round,fill opacity=0.00,] (246.51,251.87) -- (251.45,251.87);

\draw[color=drawColor,dash pattern=on 4pt off 4pt ,line cap=round,line join=round,fill opacity=0.00,] (248.98,236.45) -- (248.98,244.71);

\draw[color=drawColor,dash pattern=on 4pt off 4pt ,line cap=round,line join=round,fill opacity=0.00,] (248.98,273.57) -- (248.98,259.98);

\draw[color=drawColor,line cap=round,line join=round,fill opacity=0.00,] (247.74,236.45) -- (250.22,236.45);

\draw[color=drawColor,line cap=round,line join=round,fill opacity=0.00,] (247.74,273.57) -- (250.22,273.57);

\draw[color=drawColor,line cap=round,line join=round,fill opacity=0.00,] (246.51,244.71) --
	(251.45,244.71) --
	(251.45,259.98) --
	(246.51,259.98) --
	(246.51,244.71);

\draw[color=drawColor,line width= 1.2pt,line join=round,fill opacity=0.00,] (252.69,297.48) -- (257.64,297.48);

\draw[color=drawColor,dash pattern=on 4pt off 4pt ,line cap=round,line join=round,fill opacity=0.00,] (255.16,268.37) -- (255.16,283.68);

\draw[color=drawColor,dash pattern=on 4pt off 4pt ,line cap=round,line join=round,fill opacity=0.00,] (255.16,315.09) -- (255.16,306.31);

\draw[color=drawColor,line cap=round,line join=round,fill opacity=0.00,] (253.93,268.37) -- (256.40,268.37);

\draw[color=drawColor,line cap=round,line join=round,fill opacity=0.00,] (253.93,315.09) -- (256.40,315.09);

\draw[color=drawColor,line cap=round,line join=round,fill opacity=0.00,] (252.69,283.68) --
	(257.64,283.68) --
	(257.64,306.31) --
	(252.69,306.31) --
	(252.69,283.68);

\draw[color=drawColor,line width= 1.2pt,line join=round,fill opacity=0.00,] (258.87,306.27) -- (263.82,306.27);

\draw[color=drawColor,dash pattern=on 4pt off 4pt ,line cap=round,line join=round,fill opacity=0.00,] (261.34,264.64) -- (261.34,281.91);

\draw[color=drawColor,dash pattern=on 4pt off 4pt ,line cap=round,line join=round,fill opacity=0.00,] (261.34,326.90) -- (261.34,313.82);

\draw[color=drawColor,line cap=round,line join=round,fill opacity=0.00,] (260.11,264.64) -- (262.58,264.64);

\draw[color=drawColor,line cap=round,line join=round,fill opacity=0.00,] (260.11,326.90) -- (262.58,326.90);

\draw[color=drawColor,line cap=round,line join=round,fill opacity=0.00,] (258.87,281.91) --
	(263.82,281.91) --
	(263.82,313.82) --
	(258.87,313.82) --
	(258.87,281.91);

\draw[color=drawColor,line width= 1.2pt,line join=round,fill opacity=0.00,] (265.05,326.10) -- (270.00,326.10);

\draw[color=drawColor,dash pattern=on 4pt off 4pt ,line cap=round,line join=round,fill opacity=0.00,] (267.53,305.44) -- (267.53,314.54);

\draw[color=drawColor,dash pattern=on 4pt off 4pt ,line cap=round,line join=round,fill opacity=0.00,] (267.53,378.92) -- (267.53,353.33);

\draw[color=drawColor,line cap=round,line join=round,fill opacity=0.00,] (266.29,305.44) -- (268.76,305.44);

\draw[color=drawColor,line cap=round,line join=round,fill opacity=0.00,] (266.29,378.92) -- (268.76,378.92);

\draw[color=drawColor,line cap=round,line join=round,fill opacity=0.00,] (265.05,314.54) --
	(270.00,314.54) --
	(270.00,353.33) --
	(265.05,353.33) --
	(265.05,314.54);

\draw[color=drawColor,line width= 1.2pt,line join=round,fill opacity=0.00,] (271.24,299.05) -- (276.18,299.05);

\draw[color=drawColor,dash pattern=on 4pt off 4pt ,line cap=round,line join=round,fill opacity=0.00,] (273.71,268.16) -- (273.71,283.69);

\draw[color=drawColor,dash pattern=on 4pt off 4pt ,line cap=round,line join=round,fill opacity=0.00,] (273.71,358.15) -- (273.71,311.11);

\draw[color=drawColor,line cap=round,line join=round,fill opacity=0.00,] (272.47,268.16) -- (274.95,268.16);

\draw[color=drawColor,line cap=round,line join=round,fill opacity=0.00,] (272.47,358.15) -- (274.95,358.15);

\draw[color=drawColor,line cap=round,line join=round,fill opacity=0.00,] (271.24,283.69) --
	(276.18,283.69) --
	(276.18,311.11) --
	(271.24,311.11) --
	(271.24,283.69);

\draw[color=drawColor,line width= 1.2pt,line join=round,fill opacity=0.00,] (277.42,297.64) -- (282.36,297.64);

\draw[color=drawColor,dash pattern=on 4pt off 4pt ,line cap=round,line join=round,fill opacity=0.00,] (279.89,267.14) -- (279.89,284.18);

\draw[color=drawColor,dash pattern=on 4pt off 4pt ,line cap=round,line join=round,fill opacity=0.00,] (279.89,319.14) -- (279.89,310.27);

\draw[color=drawColor,line cap=round,line join=round,fill opacity=0.00,] (278.66,267.14) -- (281.13,267.14);

\draw[color=drawColor,line cap=round,line join=round,fill opacity=0.00,] (278.66,319.14) -- (281.13,319.14);

\draw[color=drawColor,line cap=round,line join=round,fill opacity=0.00,] (277.42,284.18) --
	(282.36,284.18) --
	(282.36,310.27) --
	(277.42,310.27) --
	(277.42,284.18);

\draw[color=drawColor,line width= 1.2pt,line join=round,fill opacity=0.00,] (283.60,269.59) -- (288.55,269.59);

\draw[color=drawColor,dash pattern=on 4pt off 4pt ,line cap=round,line join=round,fill opacity=0.00,] (286.07,253.88) -- (286.07,263.73);

\draw[color=drawColor,dash pattern=on 4pt off 4pt ,line cap=round,line join=round,fill opacity=0.00,] (286.07,295.80) -- (286.07,276.80);

\draw[color=drawColor,line cap=round,line join=round,fill opacity=0.00,] (284.84,253.88) -- (287.31,253.88);

\draw[color=drawColor,line cap=round,line join=round,fill opacity=0.00,] (284.84,295.80) -- (287.31,295.80);

\draw[color=drawColor,line cap=round,line join=round,fill opacity=0.00,] (283.60,263.73) --
	(288.55,263.73) --
	(288.55,276.80) --
	(283.60,276.80) --
	(283.60,263.73);

\draw[color=drawColor,line width= 1.2pt,line join=round,fill opacity=0.00,] (289.78,274.23) -- (294.73,274.23);

\draw[color=drawColor,dash pattern=on 4pt off 4pt ,line cap=round,line join=round,fill opacity=0.00,] (292.26,259.77) -- (292.26,265.10);

\draw[color=drawColor,dash pattern=on 4pt off 4pt ,line cap=round,line join=round,fill opacity=0.00,] (292.26,301.96) -- (292.26,287.14);

\draw[color=drawColor,line cap=round,line join=round,fill opacity=0.00,] (291.02,259.77) -- (293.49,259.77);

\draw[color=drawColor,line cap=round,line join=round,fill opacity=0.00,] (291.02,301.96) -- (293.49,301.96);

\draw[color=drawColor,line cap=round,line join=round,fill opacity=0.00,] (289.78,265.10) --
	(294.73,265.10) --
	(294.73,287.14) --
	(289.78,287.14) --
	(289.78,265.10);

\draw[color=drawColor,line width= 1.2pt,line join=round,fill opacity=0.00,] (295.97,264.15) -- (300.91,264.15);

\draw[color=drawColor,dash pattern=on 4pt off 4pt ,line cap=round,line join=round,fill opacity=0.00,] (298.44,248.85) -- (298.44,257.18);

\draw[color=drawColor,dash pattern=on 4pt off 4pt ,line cap=round,line join=round,fill opacity=0.00,] (298.44,295.94) -- (298.44,272.24);

\draw[color=drawColor,line cap=round,line join=round,fill opacity=0.00,] (297.20,248.85) -- (299.68,248.85);

\draw[color=drawColor,line cap=round,line join=round,fill opacity=0.00,] (297.20,295.94) -- (299.68,295.94);

\draw[color=drawColor,line cap=round,line join=round,fill opacity=0.00,] (295.97,257.18) --
	(300.91,257.18) --
	(300.91,272.24) --
	(295.97,272.24) --
	(295.97,257.18);

\draw[color=drawColor,line width= 1.2pt,line join=round,fill opacity=0.00,] (302.15,243.84) -- (307.09,243.84);

\draw[color=drawColor,dash pattern=on 4pt off 4pt ,line cap=round,line join=round,fill opacity=0.00,] (304.62,232.94) -- (304.62,237.32);

\draw[color=drawColor,dash pattern=on 4pt off 4pt ,line cap=round,line join=round,fill opacity=0.00,] (304.62,259.91) -- (304.62,250.20);

\draw[color=drawColor,line cap=round,line join=round,fill opacity=0.00,] (303.39,232.94) -- (305.86,232.94);

\draw[color=drawColor,line cap=round,line join=round,fill opacity=0.00,] (303.39,259.91) -- (305.86,259.91);

\draw[color=drawColor,line cap=round,line join=round,fill opacity=0.00,] (302.15,237.32) --
	(307.09,237.32) --
	(307.09,250.20) --
	(302.15,250.20) --
	(302.15,237.32);

\draw[color=drawColor,line width= 1.2pt,line join=round,fill opacity=0.00,] (308.33,257.18) -- (313.28,257.18);

\draw[color=drawColor,dash pattern=on 4pt off 4pt ,line cap=round,line join=round,fill opacity=0.00,] (310.80,239.17) -- (310.80,249.00);

\draw[color=drawColor,dash pattern=on 4pt off 4pt ,line cap=round,line join=round,fill opacity=0.00,] (310.80,274.88) -- (310.80,262.05);

\draw[color=drawColor,line cap=round,line join=round,fill opacity=0.00,] (309.57,239.17) -- (312.04,239.17);

\draw[color=drawColor,line cap=round,line join=round,fill opacity=0.00,] (309.57,274.88) -- (312.04,274.88);

\draw[color=drawColor,line cap=round,line join=round,fill opacity=0.00,] (308.33,249.00) --
	(313.28,249.00) --
	(313.28,262.05) --
	(308.33,262.05) --
	(308.33,249.00);

\draw[color=drawColor,line width= 1.2pt,line join=round,fill opacity=0.00,] (314.51,261.67) -- (319.46,261.67);

\draw[color=drawColor,dash pattern=on 4pt off 4pt ,line cap=round,line join=round,fill opacity=0.00,] (316.99,248.51) -- (316.99,255.01);

\draw[color=drawColor,dash pattern=on 4pt off 4pt ,line cap=round,line join=round,fill opacity=0.00,] (316.99,286.41) -- (316.99,267.58);

\draw[color=drawColor,line cap=round,line join=round,fill opacity=0.00,] (315.75,248.51) -- (318.22,248.51);

\draw[color=drawColor,line cap=round,line join=round,fill opacity=0.00,] (315.75,286.41) -- (318.22,286.41);

\draw[color=drawColor,line cap=round,line join=round,fill opacity=0.00,] (314.51,255.01) --
	(319.46,255.01) --
	(319.46,267.58) --
	(314.51,267.58) --
	(314.51,255.01);

\draw[color=drawColor,line width= 1.2pt,line join=round,fill opacity=0.00,] (320.70,282.64) -- (325.64,282.64);

\draw[color=drawColor,dash pattern=on 4pt off 4pt ,line cap=round,line join=round,fill opacity=0.00,] (323.17,264.78) -- (323.17,271.17);

\draw[color=drawColor,dash pattern=on 4pt off 4pt ,line cap=round,line join=round,fill opacity=0.00,] (323.17,308.29) -- (323.17,294.09);

\draw[color=drawColor,line cap=round,line join=round,fill opacity=0.00,] (321.93,264.78) -- (324.41,264.78);

\draw[color=drawColor,line cap=round,line join=round,fill opacity=0.00,] (321.93,308.29) -- (324.41,308.29);

\draw[color=drawColor,line cap=round,line join=round,fill opacity=0.00,] (320.70,271.17) --
	(325.64,271.17) --
	(325.64,294.09) --
	(320.70,294.09) --
	(320.70,271.17);

\draw[color=drawColor,line width= 1.2pt,line join=round,fill opacity=0.00,] (326.88,262.82) -- (331.82,262.82);

\draw[color=drawColor,dash pattern=on 4pt off 4pt ,line cap=round,line join=round,fill opacity=0.00,] (329.35,248.84) -- (329.35,256.11);

\draw[color=drawColor,dash pattern=on 4pt off 4pt ,line cap=round,line join=round,fill opacity=0.00,] (329.35,287.87) -- (329.35,270.77);

\draw[color=drawColor,line cap=round,line join=round,fill opacity=0.00,] (328.11,248.84) -- (330.59,248.84);

\draw[color=drawColor,line cap=round,line join=round,fill opacity=0.00,] (328.11,287.87) -- (330.59,287.87);

\draw[color=drawColor,line cap=round,line join=round,fill opacity=0.00,] (326.88,256.11) --
	(331.82,256.11) --
	(331.82,270.77) --
	(326.88,270.77) --
	(326.88,256.11);

\draw[color=drawColor,line width= 1.2pt,line join=round,fill opacity=0.00,] (333.06,274.83) -- (338.01,274.83);

\draw[color=drawColor,dash pattern=on 4pt off 4pt ,line cap=round,line join=round,fill opacity=0.00,] (335.53,261.62) -- (335.53,268.26);

\draw[color=drawColor,dash pattern=on 4pt off 4pt ,line cap=round,line join=round,fill opacity=0.00,] (335.53,301.66) -- (335.53,285.86);

\draw[color=drawColor,line cap=round,line join=round,fill opacity=0.00,] (334.30,261.62) -- (336.77,261.62);

\draw[color=drawColor,line cap=round,line join=round,fill opacity=0.00,] (334.30,301.66) -- (336.77,301.66);

\draw[color=drawColor,line cap=round,line join=round,fill opacity=0.00,] (333.06,268.26) --
	(338.01,268.26) --
	(338.01,285.86) --
	(333.06,285.86) --
	(333.06,268.26);

\draw[color=drawColor,line width= 1.2pt,line join=round,fill opacity=0.00,] (339.24,268.94) -- (344.19,268.94);

\draw[color=drawColor,dash pattern=on 4pt off 4pt ,line cap=round,line join=round,fill opacity=0.00,] (341.72,254.17) -- (341.72,263.19);

\draw[color=drawColor,dash pattern=on 4pt off 4pt ,line cap=round,line join=round,fill opacity=0.00,] (341.72,298.08) -- (341.72,274.78);

\draw[color=drawColor,line cap=round,line join=round,fill opacity=0.00,] (340.48,254.17) -- (342.95,254.17);

\draw[color=drawColor,line cap=round,line join=round,fill opacity=0.00,] (340.48,298.08) -- (342.95,298.08);

\draw[color=drawColor,line cap=round,line join=round,fill opacity=0.00,] (339.24,263.19) --
	(344.19,263.19) --
	(344.19,274.78) --
	(339.24,274.78) --
	(339.24,263.19);

\draw[color=drawColor,line width= 1.2pt,line join=round,fill opacity=0.00,] (345.43,319.94) -- (350.37,319.94);

\draw[color=drawColor,dash pattern=on 4pt off 4pt ,line cap=round,line join=round,fill opacity=0.00,] (347.90,289.69) -- (347.90,310.63);

\draw[color=drawColor,dash pattern=on 4pt off 4pt ,line cap=round,line join=round,fill opacity=0.00,] (347.90,371.57) -- (347.90,334.33);

\draw[color=drawColor,line cap=round,line join=round,fill opacity=0.00,] (346.66,289.69) -- (349.13,289.69);

\draw[color=drawColor,line cap=round,line join=round,fill opacity=0.00,] (346.66,371.57) -- (349.13,371.57);

\draw[color=drawColor,line cap=round,line join=round,fill opacity=0.00,] (345.43,310.63) --
	(350.37,310.63) --
	(350.37,334.33) --
	(345.43,334.33) --
	(345.43,310.63);

\draw[color=drawColor,line width= 1.2pt,line join=round,fill opacity=0.00,] (351.61,279.67) -- (356.55,279.67);

\draw[color=drawColor,dash pattern=on 4pt off 4pt ,line cap=round,line join=round,fill opacity=0.00,] (354.08,253.63) -- (354.08,265.69);

\draw[color=drawColor,dash pattern=on 4pt off 4pt ,line cap=round,line join=round,fill opacity=0.00,] (354.08,316.36) -- (354.08,302.83);

\draw[color=drawColor,line cap=round,line join=round,fill opacity=0.00,] (352.84,253.63) -- (355.32,253.63);

\draw[color=drawColor,line cap=round,line join=round,fill opacity=0.00,] (352.84,316.36) -- (355.32,316.36);

\draw[color=drawColor,line cap=round,line join=round,fill opacity=0.00,] (351.61,265.69) --
	(356.55,265.69) --
	(356.55,302.83) --
	(351.61,302.83) --
	(351.61,265.69);

\draw[color=drawColor,line width= 1.2pt,line join=round,fill opacity=0.00,] (357.79,269.69) -- (362.74,269.69);

\draw[color=drawColor,dash pattern=on 4pt off 4pt ,line cap=round,line join=round,fill opacity=0.00,] (360.26,253.26) -- (360.26,264.23);

\draw[color=drawColor,dash pattern=on 4pt off 4pt ,line cap=round,line join=round,fill opacity=0.00,] (360.26,297.93) -- (360.26,279.42);

\draw[color=drawColor,line cap=round,line join=round,fill opacity=0.00,] (359.03,253.26) -- (361.50,253.26);

\draw[color=drawColor,line cap=round,line join=round,fill opacity=0.00,] (359.03,297.93) -- (361.50,297.93);

\draw[color=drawColor,line cap=round,line join=round,fill opacity=0.00,] (357.79,264.23) --
	(362.74,264.23) --
	(362.74,279.42) --
	(357.79,279.42) --
	(357.79,264.23);

\draw[color=drawColor,line width= 1.2pt,line join=round,fill opacity=0.00,] (363.97,247.14) -- (368.92,247.14);

\draw[color=drawColor,dash pattern=on 4pt off 4pt ,line cap=round,line join=round,fill opacity=0.00,] (366.45,229.56) -- (366.45,235.75);

\draw[color=drawColor,dash pattern=on 4pt off 4pt ,line cap=round,line join=round,fill opacity=0.00,] (366.45,272.37) -- (366.45,258.39);

\draw[color=drawColor,line cap=round,line join=round,fill opacity=0.00,] (365.21,229.56) -- (367.68,229.56);

\draw[color=drawColor,line cap=round,line join=round,fill opacity=0.00,] (365.21,272.37) -- (367.68,272.37);

\draw[color=drawColor,line cap=round,line join=round,fill opacity=0.00,] (363.97,235.75) --
	(368.92,235.75) --
	(368.92,258.39) --
	(363.97,258.39) --
	(363.97,235.75);

\draw[color=drawColor,line width= 1.2pt,line join=round,fill opacity=0.00,] (370.16,262.55) -- (375.10,262.55);

\draw[color=drawColor,dash pattern=on 4pt off 4pt ,line cap=round,line join=round,fill opacity=0.00,] (372.63,249.43) -- (372.63,257.72);

\draw[color=drawColor,dash pattern=on 4pt off 4pt ,line cap=round,line join=round,fill opacity=0.00,] (372.63,287.46) -- (372.63,270.49);

\draw[color=drawColor,line cap=round,line join=round,fill opacity=0.00,] (371.39,249.43) -- (373.86,249.43);

\draw[color=drawColor,line cap=round,line join=round,fill opacity=0.00,] (371.39,287.46) -- (373.86,287.46);

\draw[color=drawColor,line cap=round,line join=round,fill opacity=0.00,] (370.16,257.72) --
	(375.10,257.72) --
	(375.10,270.49) --
	(370.16,270.49) --
	(370.16,257.72);

\draw[color=drawColor,line width= 1.2pt,line join=round,fill opacity=0.00,] (376.34,255.10) -- (381.28,255.10);

\draw[color=drawColor,dash pattern=on 4pt off 4pt ,line cap=round,line join=round,fill opacity=0.00,] (378.81,239.67) -- (378.81,247.87);

\draw[color=drawColor,dash pattern=on 4pt off 4pt ,line cap=round,line join=round,fill opacity=0.00,] (378.81,266.75) -- (378.81,260.35);

\draw[color=drawColor,line cap=round,line join=round,fill opacity=0.00,] (377.57,239.67) -- (380.05,239.67);

\draw[color=drawColor,line cap=round,line join=round,fill opacity=0.00,] (377.57,266.75) -- (380.05,266.75);

\draw[color=drawColor,line cap=round,line join=round,fill opacity=0.00,] (376.34,247.87) --
	(381.28,247.87) --
	(381.28,260.35) --
	(376.34,260.35) --
	(376.34,247.87);

\draw[color=drawColor,line width= 1.2pt,line join=round,fill opacity=0.00,] (382.52,255.02) -- (387.47,255.02);

\draw[color=drawColor,dash pattern=on 4pt off 4pt ,line cap=round,line join=round,fill opacity=0.00,] (384.99,233.80) -- (384.99,247.25);

\draw[color=drawColor,dash pattern=on 4pt off 4pt ,line cap=round,line join=round,fill opacity=0.00,] (384.99,285.06) -- (384.99,261.31);

\draw[color=drawColor,line cap=round,line join=round,fill opacity=0.00,] (383.76,233.80) -- (386.23,233.80);

\draw[color=drawColor,line cap=round,line join=round,fill opacity=0.00,] (383.76,285.06) -- (386.23,285.06);

\draw[color=drawColor,line cap=round,line join=round,fill opacity=0.00,] (382.52,247.25) --
	(387.47,247.25) --
	(387.47,261.31) --
	(382.52,261.31) --
	(382.52,247.25);

\draw[color=drawColor,line width= 1.2pt,line join=round,fill opacity=0.00,] (388.70,265.86) -- (393.65,265.86);

\draw[color=drawColor,dash pattern=on 4pt off 4pt ,line cap=round,line join=round,fill opacity=0.00,] (391.18,249.78) -- (391.18,258.42);

\draw[color=drawColor,dash pattern=on 4pt off 4pt ,line cap=round,line join=round,fill opacity=0.00,] (391.18,294.25) -- (391.18,273.92);

\draw[color=drawColor,line cap=round,line join=round,fill opacity=0.00,] (389.94,249.78) -- (392.41,249.78);

\draw[color=drawColor,line cap=round,line join=round,fill opacity=0.00,] (389.94,294.25) -- (392.41,294.25);

\draw[color=drawColor,line cap=round,line join=round,fill opacity=0.00,] (388.70,258.42) --
	(393.65,258.42) --
	(393.65,273.92) --
	(388.70,273.92) --
	(388.70,258.42);

\draw[color=drawColor,line width= 1.2pt,line join=round,fill opacity=0.00,] (394.88,299.62) -- (399.83,299.62);

\draw[color=drawColor,dash pattern=on 4pt off 4pt ,line cap=round,line join=round,fill opacity=0.00,] (397.36,268.60) -- (397.36,285.65);

\draw[color=drawColor,dash pattern=on 4pt off 4pt ,line cap=round,line join=round,fill opacity=0.00,] (397.36,324.46) -- (397.36,310.49);

\draw[color=drawColor,line cap=round,line join=round,fill opacity=0.00,] (396.12,268.60) -- (398.59,268.60);

\draw[color=drawColor,line cap=round,line join=round,fill opacity=0.00,] (396.12,324.46) -- (398.59,324.46);

\draw[color=drawColor,line cap=round,line join=round,fill opacity=0.00,] (394.88,285.65) --
	(399.83,285.65) --
	(399.83,310.49) --
	(394.88,310.49) --
	(394.88,285.65);

\draw[color=drawColor,line width= 1.2pt,line join=round,fill opacity=0.00,] (401.07,268.84) -- (406.01,268.84);

\draw[color=drawColor,dash pattern=on 4pt off 4pt ,line cap=round,line join=round,fill opacity=0.00,] (403.54,252.59) -- (403.54,261.93);

\draw[color=drawColor,dash pattern=on 4pt off 4pt ,line cap=round,line join=round,fill opacity=0.00,] (403.54,299.44) -- (403.54,276.26);

\draw[color=drawColor,line cap=round,line join=round,fill opacity=0.00,] (402.30,252.59) -- (404.78,252.59);

\draw[color=drawColor,line cap=round,line join=round,fill opacity=0.00,] (402.30,299.44) -- (404.78,299.44);

\draw[color=drawColor,line cap=round,line join=round,fill opacity=0.00,] (401.07,261.93) --
	(406.01,261.93) --
	(406.01,276.26) --
	(401.07,276.26) --
	(401.07,261.93);

\draw[color=drawColor,line width= 1.2pt,line join=round,fill opacity=0.00,] (407.25,264.66) -- (412.20,264.66);

\draw[color=drawColor,dash pattern=on 4pt off 4pt ,line cap=round,line join=round,fill opacity=0.00,] (409.72,251.96) -- (409.72,259.87);

\draw[color=drawColor,dash pattern=on 4pt off 4pt ,line cap=round,line join=round,fill opacity=0.00,] (409.72,287.78) -- (409.72,272.39);

\draw[color=drawColor,line cap=round,line join=round,fill opacity=0.00,] (408.49,251.96) -- (410.96,251.96);

\draw[color=drawColor,line cap=round,line join=round,fill opacity=0.00,] (408.49,287.78) -- (410.96,287.78);

\draw[color=drawColor,line cap=round,line join=round,fill opacity=0.00,] (407.25,259.87) --
	(412.20,259.87) --
	(412.20,272.39) --
	(407.25,272.39) --
	(407.25,259.87);

\draw[color=drawColor,line width= 1.2pt,line join=round,fill opacity=0.00,] (413.43,272.53) -- (418.38,272.53);

\draw[color=drawColor,dash pattern=on 4pt off 4pt ,line cap=round,line join=round,fill opacity=0.00,] (415.90,252.17) -- (415.90,264.16);

\draw[color=drawColor,dash pattern=on 4pt off 4pt ,line cap=round,line join=round,fill opacity=0.00,] (415.90,297.96) -- (415.90,284.19);

\draw[color=drawColor,line cap=round,line join=round,fill opacity=0.00,] (414.67,252.17) -- (417.14,252.17);

\draw[color=drawColor,line cap=round,line join=round,fill opacity=0.00,] (414.67,297.96) -- (417.14,297.96);

\draw[color=drawColor,line cap=round,line join=round,fill opacity=0.00,] (413.43,264.16) --
	(418.38,264.16) --
	(418.38,284.19) --
	(413.43,284.19) --
	(413.43,264.16);

\draw[color=drawColor,line width= 1.2pt,line join=round,fill opacity=0.00,] (419.61,266.24) -- (424.56,266.24);

\draw[color=drawColor,dash pattern=on 4pt off 4pt ,line cap=round,line join=round,fill opacity=0.00,] (422.09,252.29) -- (422.09,261.89);

\draw[color=drawColor,dash pattern=on 4pt off 4pt ,line cap=round,line join=round,fill opacity=0.00,] (422.09,290.86) -- (422.09,275.70);

\draw[color=drawColor,line cap=round,line join=round,fill opacity=0.00,] (420.85,252.29) -- (423.32,252.29);

\draw[color=drawColor,line cap=round,line join=round,fill opacity=0.00,] (420.85,290.86) -- (423.32,290.86);

\draw[color=drawColor,line cap=round,line join=round,fill opacity=0.00,] (419.61,261.89) --
	(424.56,261.89) --
	(424.56,275.70) --
	(419.61,275.70) --
	(419.61,261.89);

\draw[color=drawColor,line width= 1.2pt,line join=round,fill opacity=0.00,] (425.80,263.10) -- (430.74,263.10);

\draw[color=drawColor,dash pattern=on 4pt off 4pt ,line cap=round,line join=round,fill opacity=0.00,] (428.27,244.96) -- (428.27,254.89);

\draw[color=drawColor,dash pattern=on 4pt off 4pt ,line cap=round,line join=round,fill opacity=0.00,] (428.27,287.94) -- (428.27,267.94);

\draw[color=drawColor,line cap=round,line join=round,fill opacity=0.00,] (427.03,244.96) -- (429.51,244.96);

\draw[color=drawColor,line cap=round,line join=round,fill opacity=0.00,] (427.03,287.94) -- (429.51,287.94);

\draw[color=drawColor,line cap=round,line join=round,fill opacity=0.00,] (425.80,254.89) --
	(430.74,254.89) --
	(430.74,267.94) --
	(425.80,267.94) --
	(425.80,254.89);

\draw[color=drawColor,line width= 1.2pt,line join=round,fill opacity=0.00,] (431.98,280.47) -- (436.93,280.47);

\draw[color=drawColor,dash pattern=on 4pt off 4pt ,line cap=round,line join=round,fill opacity=0.00,] (434.45,250.37) -- (434.45,265.17);

\draw[color=drawColor,dash pattern=on 4pt off 4pt ,line cap=round,line join=round,fill opacity=0.00,] (434.45,313.47) -- (434.45,300.92);

\draw[color=drawColor,line cap=round,line join=round,fill opacity=0.00,] (433.22,250.37) -- (435.69,250.37);

\draw[color=drawColor,line cap=round,line join=round,fill opacity=0.00,] (433.22,313.47) -- (435.69,313.47);

\draw[color=drawColor,line cap=round,line join=round,fill opacity=0.00,] (431.98,265.17) --
	(436.93,265.17) --
	(436.93,300.92) --
	(431.98,300.92) --
	(431.98,265.17);
\end{scope}
\begin{scope}
\path[clip] (  0.00,  0.00) rectangle (469.75,614.29);
\definecolor[named]{drawColor}{rgb}{0.00,0.00,0.00}

\draw[color=drawColor,line cap=round,line join=round,fill opacity=0.00,] ( 51.14,221.40) -- (434.45,221.40);

\draw[color=drawColor,line cap=round,line join=round,fill opacity=0.00,] ( 51.14,221.40) -- ( 51.14,217.44);

\draw[color=drawColor,line cap=round,line join=round,fill opacity=0.00,] ( 57.33,221.40) -- ( 57.33,217.44);

\draw[color=drawColor,line cap=round,line join=round,fill opacity=0.00,] ( 63.51,221.40) -- ( 63.51,217.44);

\draw[color=drawColor,line cap=round,line join=round,fill opacity=0.00,] ( 69.69,221.40) -- ( 69.69,217.44);

\draw[color=drawColor,line cap=round,line join=round,fill opacity=0.00,] ( 75.87,221.40) -- ( 75.87,217.44);

\draw[color=drawColor,line cap=round,line join=round,fill opacity=0.00,] ( 82.05,221.40) -- ( 82.05,217.44);

\draw[color=drawColor,line cap=round,line join=round,fill opacity=0.00,] ( 88.24,221.40) -- ( 88.24,217.44);

\draw[color=drawColor,line cap=round,line join=round,fill opacity=0.00,] ( 94.42,221.40) -- ( 94.42,217.44);

\draw[color=drawColor,line cap=round,line join=round,fill opacity=0.00,] (100.60,221.40) -- (100.60,217.44);

\draw[color=drawColor,line cap=round,line join=round,fill opacity=0.00,] (106.78,221.40) -- (106.78,217.44);

\draw[color=drawColor,line cap=round,line join=round,fill opacity=0.00,] (112.97,221.40) -- (112.97,217.44);

\draw[color=drawColor,line cap=round,line join=round,fill opacity=0.00,] (119.15,221.40) -- (119.15,217.44);

\draw[color=drawColor,line cap=round,line join=round,fill opacity=0.00,] (125.33,221.40) -- (125.33,217.44);

\draw[color=drawColor,line cap=round,line join=round,fill opacity=0.00,] (131.51,221.40) -- (131.51,217.44);

\draw[color=drawColor,line cap=round,line join=round,fill opacity=0.00,] (137.70,221.40) -- (137.70,217.44);

\draw[color=drawColor,line cap=round,line join=round,fill opacity=0.00,] (143.88,221.40) -- (143.88,217.44);

\draw[color=drawColor,line cap=round,line join=round,fill opacity=0.00,] (150.06,221.40) -- (150.06,217.44);

\draw[color=drawColor,line cap=round,line join=round,fill opacity=0.00,] (156.24,221.40) -- (156.24,217.44);

\draw[color=drawColor,line cap=round,line join=round,fill opacity=0.00,] (162.43,221.40) -- (162.43,217.44);

\draw[color=drawColor,line cap=round,line join=round,fill opacity=0.00,] (168.61,221.40) -- (168.61,217.44);

\draw[color=drawColor,line cap=round,line join=round,fill opacity=0.00,] (174.79,221.40) -- (174.79,217.44);

\draw[color=drawColor,line cap=round,line join=round,fill opacity=0.00,] (180.97,221.40) -- (180.97,217.44);

\draw[color=drawColor,line cap=round,line join=round,fill opacity=0.00,] (187.16,221.40) -- (187.16,217.44);

\draw[color=drawColor,line cap=round,line join=round,fill opacity=0.00,] (193.34,221.40) -- (193.34,217.44);

\draw[color=drawColor,line cap=round,line join=round,fill opacity=0.00,] (199.52,221.40) -- (199.52,217.44);

\draw[color=drawColor,line cap=round,line join=round,fill opacity=0.00,] (205.70,221.40) -- (205.70,217.44);

\draw[color=drawColor,line cap=round,line join=round,fill opacity=0.00,] (211.89,221.40) -- (211.89,217.44);

\draw[color=drawColor,line cap=round,line join=round,fill opacity=0.00,] (218.07,221.40) -- (218.07,217.44);

\draw[color=drawColor,line cap=round,line join=round,fill opacity=0.00,] (224.25,221.40) -- (224.25,217.44);

\draw[color=drawColor,line cap=round,line join=round,fill opacity=0.00,] (230.43,221.40) -- (230.43,217.44);

\draw[color=drawColor,line cap=round,line join=round,fill opacity=0.00,] (236.62,221.40) -- (236.62,217.44);

\draw[color=drawColor,line cap=round,line join=round,fill opacity=0.00,] (242.80,221.40) -- (242.80,217.44);

\draw[color=drawColor,line cap=round,line join=round,fill opacity=0.00,] (248.98,221.40) -- (248.98,217.44);

\draw[color=drawColor,line cap=round,line join=round,fill opacity=0.00,] (255.16,221.40) -- (255.16,217.44);

\draw[color=drawColor,line cap=round,line join=round,fill opacity=0.00,] (261.34,221.40) -- (261.34,217.44);

\draw[color=drawColor,line cap=round,line join=round,fill opacity=0.00,] (267.53,221.40) -- (267.53,217.44);

\draw[color=drawColor,line cap=round,line join=round,fill opacity=0.00,] (273.71,221.40) -- (273.71,217.44);

\draw[color=drawColor,line cap=round,line join=round,fill opacity=0.00,] (279.89,221.40) -- (279.89,217.44);

\draw[color=drawColor,line cap=round,line join=round,fill opacity=0.00,] (286.07,221.40) -- (286.07,217.44);

\draw[color=drawColor,line cap=round,line join=round,fill opacity=0.00,] (292.26,221.40) -- (292.26,217.44);

\draw[color=drawColor,line cap=round,line join=round,fill opacity=0.00,] (298.44,221.40) -- (298.44,217.44);

\draw[color=drawColor,line cap=round,line join=round,fill opacity=0.00,] (304.62,221.40) -- (304.62,217.44);

\draw[color=drawColor,line cap=round,line join=round,fill opacity=0.00,] (310.80,221.40) -- (310.80,217.44);

\draw[color=drawColor,line cap=round,line join=round,fill opacity=0.00,] (316.99,221.40) -- (316.99,217.44);

\draw[color=drawColor,line cap=round,line join=round,fill opacity=0.00,] (323.17,221.40) -- (323.17,217.44);

\draw[color=drawColor,line cap=round,line join=round,fill opacity=0.00,] (329.35,221.40) -- (329.35,217.44);

\draw[color=drawColor,line cap=round,line join=round,fill opacity=0.00,] (335.53,221.40) -- (335.53,217.44);

\draw[color=drawColor,line cap=round,line join=round,fill opacity=0.00,] (341.72,221.40) -- (341.72,217.44);

\draw[color=drawColor,line cap=round,line join=round,fill opacity=0.00,] (347.90,221.40) -- (347.90,217.44);

\draw[color=drawColor,line cap=round,line join=round,fill opacity=0.00,] (354.08,221.40) -- (354.08,217.44);

\draw[color=drawColor,line cap=round,line join=round,fill opacity=0.00,] (360.26,221.40) -- (360.26,217.44);

\draw[color=drawColor,line cap=round,line join=round,fill opacity=0.00,] (366.45,221.40) -- (366.45,217.44);

\draw[color=drawColor,line cap=round,line join=round,fill opacity=0.00,] (372.63,221.40) -- (372.63,217.44);

\draw[color=drawColor,line cap=round,line join=round,fill opacity=0.00,] (378.81,221.40) -- (378.81,217.44);

\draw[color=drawColor,line cap=round,line join=round,fill opacity=0.00,] (384.99,221.40) -- (384.99,217.44);

\draw[color=drawColor,line cap=round,line join=round,fill opacity=0.00,] (391.18,221.40) -- (391.18,217.44);

\draw[color=drawColor,line cap=round,line join=round,fill opacity=0.00,] (397.36,221.40) -- (397.36,217.44);

\draw[color=drawColor,line cap=round,line join=round,fill opacity=0.00,] (403.54,221.40) -- (403.54,217.44);

\draw[color=drawColor,line cap=round,line join=round,fill opacity=0.00,] (409.72,221.40) -- (409.72,217.44);

\draw[color=drawColor,line cap=round,line join=round,fill opacity=0.00,] (415.90,221.40) -- (415.90,217.44);

\draw[color=drawColor,line cap=round,line join=round,fill opacity=0.00,] (422.09,221.40) -- (422.09,217.44);

\draw[color=drawColor,line cap=round,line join=round,fill opacity=0.00,] (428.27,221.40) -- (428.27,217.44);

\draw[color=drawColor,line cap=round,line join=round,fill opacity=0.00,] (434.45,221.40) -- (434.45,217.44);

\node[color=drawColor,anchor=base,inner sep=0pt, outer sep=0pt, scale=  0.66] at ( 51.14,205.56) {1949%
};

\node[color=drawColor,anchor=base,inner sep=0pt, outer sep=0pt, scale=  0.66] at ( 75.87,205.56) {1953%
};

\node[color=drawColor,anchor=base,inner sep=0pt, outer sep=0pt, scale=  0.66] at (100.60,205.56) {1957%
};

\node[color=drawColor,anchor=base,inner sep=0pt, outer sep=0pt, scale=  0.66] at (125.33,205.56) {1961%
};

\node[color=drawColor,anchor=base,inner sep=0pt, outer sep=0pt, scale=  0.66] at (150.06,205.56) {1965%
};

\node[color=drawColor,anchor=base,inner sep=0pt, outer sep=0pt, scale=  0.66] at (174.79,205.56) {1969%
};

\node[color=drawColor,anchor=base,inner sep=0pt, outer sep=0pt, scale=  0.66] at (199.52,205.56) {1973%
};

\node[color=drawColor,anchor=base,inner sep=0pt, outer sep=0pt, scale=  0.66] at (224.25,205.56) {1977%
};

\node[color=drawColor,anchor=base,inner sep=0pt, outer sep=0pt, scale=  0.66] at (248.98,205.56) {1981%
};

\node[color=drawColor,anchor=base,inner sep=0pt, outer sep=0pt, scale=  0.66] at (273.71,205.56) {1985%
};

\node[color=drawColor,anchor=base,inner sep=0pt, outer sep=0pt, scale=  0.66] at (298.44,205.56) {1989%
};

\node[color=drawColor,anchor=base,inner sep=0pt, outer sep=0pt, scale=  0.66] at (323.17,205.56) {1993%
};

\node[color=drawColor,anchor=base,inner sep=0pt, outer sep=0pt, scale=  0.66] at (347.90,205.56) {1997%
};

\node[color=drawColor,anchor=base,inner sep=0pt, outer sep=0pt, scale=  0.66] at (372.63,205.56) {2001%
};

\node[color=drawColor,anchor=base,inner sep=0pt, outer sep=0pt, scale=  0.66] at (397.36,205.56) {2005%
};

\node[color=drawColor,anchor=base,inner sep=0pt, outer sep=0pt, scale=  0.66] at (422.09,205.56) {2009%
};

\draw[color=drawColor,line cap=round,line join=round,fill opacity=0.00,] ( 32.47,240.01) -- ( 32.47,378.12);

\draw[color=drawColor,line cap=round,line join=round,fill opacity=0.00,] ( 32.47,240.01) -- ( 28.51,240.01);

\draw[color=drawColor,line cap=round,line join=round,fill opacity=0.00,] ( 32.47,263.03) -- ( 28.51,263.03);

\draw[color=drawColor,line cap=round,line join=round,fill opacity=0.00,] ( 32.47,286.05) -- ( 28.51,286.05);

\draw[color=drawColor,line cap=round,line join=round,fill opacity=0.00,] ( 32.47,309.07) -- ( 28.51,309.07);

\draw[color=drawColor,line cap=round,line join=round,fill opacity=0.00,] ( 32.47,332.09) -- ( 28.51,332.09);

\draw[color=drawColor,line cap=round,line join=round,fill opacity=0.00,] ( 32.47,355.10) -- ( 28.51,355.10);

\draw[color=drawColor,line cap=round,line join=round,fill opacity=0.00,] ( 32.47,378.12) -- ( 28.51,378.12);

\node[rotate= 90.00,color=drawColor,anchor=base,inner sep=0pt, outer sep=0pt, scale=  0.66] at ( 24.55,240.01) {200%
};

\node[rotate= 90.00,color=drawColor,anchor=base,inner sep=0pt, outer sep=0pt, scale=  0.66] at ( 24.55,263.03) {400%
};

\node[rotate= 90.00,color=drawColor,anchor=base,inner sep=0pt, outer sep=0pt, scale=  0.66] at ( 24.55,286.05) {600%
};

\node[rotate= 90.00,color=drawColor,anchor=base,inner sep=0pt, outer sep=0pt, scale=  0.66] at ( 24.55,309.07) {800%
};

\node[rotate= 90.00,color=drawColor,anchor=base,inner sep=0pt, outer sep=0pt, scale=  0.66] at ( 24.55,332.09) {1000%
};

\node[rotate= 90.00,color=drawColor,anchor=base,inner sep=0pt, outer sep=0pt, scale=  0.66] at ( 24.55,355.10) {1200%
};

\node[rotate= 90.00,color=drawColor,anchor=base,inner sep=0pt, outer sep=0pt, scale=  0.66] at ( 24.55,378.12) {1400%
};
\end{scope}
\begin{scope}
\path[clip] (  0.00,204.77) rectangle (469.75,409.53);
\definecolor[named]{drawColor}{rgb}{0.00,0.00,0.00}

\node[rotate= 90.00,color=drawColor,anchor=base,inner sep=0pt, outer sep=0pt, scale=  0.66] at (  8.71,303.19) {FLow [KAF]%
};
\end{scope}
\begin{scope}
\path[clip] (  0.00,  0.00) rectangle (469.75,614.29);
\definecolor[named]{drawColor}{rgb}{0.00,0.00,0.00}

\draw[color=drawColor,line cap=round,line join=round,fill opacity=0.00,] ( 32.47,221.40) --
	(453.12,221.40) --
	(453.12,384.98) --
	( 32.47,384.98) --
	( 32.47,221.40);
\end{scope}
\begin{scope}
\path[clip] ( 32.47,221.40) rectangle (453.12,384.98);
\definecolor[named]{drawColor}{rgb}{1.00,0.00,0.00}

\draw[color=drawColor,line cap=round,line join=round,fill opacity=0.00,] ( 51.85,301.63) -- ( 56.62,275.13);

\draw[color=drawColor,line cap=round,line join=round,fill opacity=0.00,] ( 63.96,276.04) -- ( 69.24,322.24);

\draw[color=drawColor,line cap=round,line join=round,fill opacity=0.00,] ( 70.31,322.27) -- ( 75.26,290.83);

\draw[color=drawColor,line cap=round,line join=round,fill opacity=0.00,] ( 76.36,282.99) -- ( 81.57,240.82);

\draw[color=drawColor,line cap=round,line join=round,fill opacity=0.00,] ( 83.15,240.69) -- ( 87.14,254.54);

\draw[color=drawColor,line cap=round,line join=round,fill opacity=0.00,] ( 90.82,261.35) -- ( 91.84,262.53);

\draw[color=drawColor,line cap=round,line join=round,fill opacity=0.00,] ( 94.65,269.49) -- (100.37,367.87);

\draw[color=drawColor,line cap=round,line join=round,fill opacity=0.00,] (100.94,367.88) -- (106.45,303.01);

\draw[color=drawColor,line cap=round,line join=round,fill opacity=0.00,] (107.50,295.17) -- (112.25,269.20);

\draw[color=drawColor,line cap=round,line join=round,fill opacity=0.00,] (115.45,268.38) -- (116.66,269.88);

\draw[color=drawColor,line cap=round,line join=round,fill opacity=0.00,] (120.88,269.41) -- (123.60,263.82);

\draw[color=drawColor,line cap=round,line join=round,fill opacity=0.00,] (126.22,264.12) -- (130.63,283.25);

\draw[color=drawColor,line cap=round,line join=round,fill opacity=0.00,] (132.02,283.18) -- (137.19,243.13);

\draw[color=drawColor,line cap=round,line join=round,fill opacity=0.00,] (138.55,243.07) -- (143.03,263.51);

\draw[color=drawColor,line cap=round,line join=round,fill opacity=0.00,] (144.41,271.30) -- (149.53,308.98);

\draw[color=drawColor,line cap=round,line join=round,fill opacity=0.00,] (150.45,308.96) -- (155.85,254.71);

\draw[color=drawColor,line cap=round,line join=round,fill opacity=0.00,] (158.13,254.25) -- (160.54,258.67);

\draw[color=drawColor,line cap=round,line join=round,fill opacity=0.00,] (163.08,266.05) -- (167.96,295.19);

\draw[color=drawColor,line cap=round,line join=round,fill opacity=0.00,] (169.31,295.20) -- (174.08,268.88);

\draw[color=drawColor,line cap=round,line join=round,fill opacity=0.00,] (175.76,268.83) -- (180.00,285.61);

\draw[color=drawColor,line cap=round,line join=round,fill opacity=0.00,] (183.82,286.69) -- (184.31,286.21);

\draw[color=drawColor,line cap=round,line join=round,fill opacity=0.00,] (188.30,279.66) -- (192.19,266.79);

\draw[color=drawColor,line cap=round,line join=round,fill opacity=0.00,] (193.85,266.92) -- (199.01,306.46);

\draw[color=drawColor,line cap=round,line join=round,fill opacity=0.00,] (200.03,306.46) -- (205.20,266.42);

\draw[color=drawColor,line cap=round,line join=round,fill opacity=0.00,] (206.29,266.41) -- (211.30,299.70);

\draw[color=drawColor,line cap=round,line join=round,fill opacity=0.00,] (212.43,299.70) -- (217.52,263.09);

\draw[color=drawColor,line cap=round,line join=round,fill opacity=0.00,] (218.90,255.30) -- (223.42,234.25);

\draw[color=drawColor,line cap=round,line join=round,fill opacity=0.00,] (224.56,234.33) -- (230.12,305.33);

\draw[color=drawColor,line cap=round,line join=round,fill opacity=0.00,] (233.22,312.09) -- (233.83,312.72);

\draw[color=drawColor,line cap=round,line join=round,fill opacity=0.00,] (243.15,312.93) -- (248.63,251.21);

\draw[color=drawColor,line cap=round,line join=round,fill opacity=0.00,] (249.50,251.19) -- (254.64,289.60);

\draw[color=drawColor,line cap=round,line join=round,fill opacity=0.00,] (255.55,297.47) -- (260.96,352.34);

\draw[color=drawColor,line cap=round,line join=round,fill opacity=0.00,] (263.81,353.18) -- (265.06,351.60);

\draw[color=drawColor,line cap=round,line join=round,fill opacity=0.00,] (268.18,344.60) -- (273.06,315.31);

\draw[color=drawColor,line cap=round,line join=round,fill opacity=0.00,] (280.65,306.43) -- (285.32,282.42);

\draw[color=drawColor,line cap=round,line join=round,fill opacity=0.00,] (287.19,274.73) -- (291.14,261.20);

\draw[color=drawColor,line cap=round,line join=round,fill opacity=0.00,] (294.68,254.27) -- (296.01,252.55);

\draw[color=drawColor,line cap=round,line join=round,fill opacity=0.00,] (300.58,252.76) -- (302.49,255.74);

\draw[color=drawColor,line cap=round,line join=round,fill opacity=0.00,] (306.13,262.73) -- (309.29,270.41);

\draw[color=drawColor,line cap=round,line join=round,fill opacity=0.00,] (312.50,270.49) -- (315.29,264.62);

\draw[color=drawColor,line cap=round,line join=round,fill opacity=0.00,] (317.42,264.98) -- (322.73,313.05);

\draw[color=drawColor,line cap=round,line join=round,fill opacity=0.00,] (323.63,313.05) -- (328.89,267.63);

\draw[color=drawColor,line cap=round,line join=round,fill opacity=0.00,] (329.60,267.65) -- (335.28,357.68);

\draw[color=drawColor,line cap=round,line join=round,fill opacity=0.00,] (335.80,357.69) -- (341.45,273.51);

\draw[color=drawColor,line cap=round,line join=round,fill opacity=0.00,] (342.21,273.49) -- (347.40,314.85);

\draw[color=drawColor,line cap=round,line join=round,fill opacity=0.00,] (348.35,314.85) -- (353.63,269.01);

\draw[color=drawColor,line cap=round,line join=round,fill opacity=0.00,] (355.78,268.66) -- (358.57,274.56);

\draw[color=drawColor,line cap=round,line join=round,fill opacity=0.00,] (361.05,274.26) -- (365.66,251.42);

\draw[color=drawColor,line cap=round,line join=round,fill opacity=0.00,] (373.78,246.83) -- (377.66,233.99);

\draw[color=drawColor,line cap=round,line join=round,fill opacity=0.00,] (379.74,234.05) -- (384.06,251.85);

\draw[color=drawColor,line cap=round,line join=round,fill opacity=0.00,] (387.23,252.44) -- (388.93,249.96);

\draw[color=drawColor,line cap=round,line join=round,fill opacity=0.00,] (391.90,250.59) -- (396.63,275.84);

\draw[color=drawColor,line cap=round,line join=round,fill opacity=0.00,] (398.22,275.86) -- (402.68,255.96);

\draw[color=drawColor,line cap=round,line join=round,fill opacity=0.00,] ( 51.14,305.53) circle (  0.89);

\draw[color=drawColor,line cap=round,line join=round,fill opacity=0.00,] ( 57.33,271.23) circle (  0.89);

\draw[color=drawColor,line cap=round,line join=round,fill opacity=0.00,] ( 63.51,272.11) circle (  0.89);

\draw[color=drawColor,line cap=round,line join=round,fill opacity=0.00,] ( 69.69,326.18) circle (  0.89);

\draw[color=drawColor,line cap=round,line join=round,fill opacity=0.00,] ( 75.87,286.92) circle (  0.89);

\draw[color=drawColor,line cap=round,line join=round,fill opacity=0.00,] ( 82.05,236.89) circle (  0.89);

\draw[color=drawColor,line cap=round,line join=round,fill opacity=0.00,] ( 88.24,258.35) circle (  0.89);

\draw[color=drawColor,line cap=round,line join=round,fill opacity=0.00,] ( 94.42,265.53) circle (  0.89);

\draw[color=drawColor,line cap=round,line join=round,fill opacity=0.00,] (100.60,371.83) circle (  0.89);

\draw[color=drawColor,line cap=round,line join=round,fill opacity=0.00,] (106.78,299.07) circle (  0.89);

\draw[color=drawColor,line cap=round,line join=round,fill opacity=0.00,] (112.97,265.30) circle (  0.89);

\draw[color=drawColor,line cap=round,line join=round,fill opacity=0.00,] (119.15,272.97) circle (  0.89);

\draw[color=drawColor,line cap=round,line join=round,fill opacity=0.00,] (125.33,260.26) circle (  0.89);

\draw[color=drawColor,line cap=round,line join=round,fill opacity=0.00,] (131.51,287.11) circle (  0.89);

\draw[color=drawColor,line cap=round,line join=round,fill opacity=0.00,] (137.70,239.20) circle (  0.89);

\draw[color=drawColor,line cap=round,line join=round,fill opacity=0.00,] (143.88,267.38) circle (  0.89);

\draw[color=drawColor,line cap=round,line join=round,fill opacity=0.00,] (150.06,312.90) circle (  0.89);

\draw[color=drawColor,line cap=round,line join=round,fill opacity=0.00,] (156.24,250.77) circle (  0.89);

\draw[color=drawColor,line cap=round,line join=round,fill opacity=0.00,] (162.43,262.15) circle (  0.89);

\draw[color=drawColor,line cap=round,line join=round,fill opacity=0.00,] (168.61,299.09) circle (  0.89);

\draw[color=drawColor,line cap=round,line join=round,fill opacity=0.00,] (174.79,264.99) circle (  0.89);

\draw[color=drawColor,line cap=round,line join=round,fill opacity=0.00,] (180.97,289.45) circle (  0.89);

\draw[color=drawColor,line cap=round,line join=round,fill opacity=0.00,] (187.16,283.45) circle (  0.89);

\draw[color=drawColor,line cap=round,line join=round,fill opacity=0.00,] (193.34,263.00) circle (  0.89);

\draw[color=drawColor,line cap=round,line join=round,fill opacity=0.00,] (199.52,310.39) circle (  0.89);

\draw[color=drawColor,line cap=round,line join=round,fill opacity=0.00,] (205.70,262.50) circle (  0.89);

\draw[color=drawColor,line cap=round,line join=round,fill opacity=0.00,] (211.89,303.62) circle (  0.89);

\draw[color=drawColor,line cap=round,line join=round,fill opacity=0.00,] (218.07,259.17) circle (  0.89);

\draw[color=drawColor,line cap=round,line join=round,fill opacity=0.00,] (224.25,230.38) circle (  0.89);

\draw[color=drawColor,line cap=round,line join=round,fill opacity=0.00,] (230.43,309.27) circle (  0.89);

\draw[color=drawColor,line cap=round,line join=round,fill opacity=0.00,] (236.62,315.53) circle (  0.89);

\draw[color=drawColor,line cap=round,line join=round,fill opacity=0.00,] (242.80,316.87) circle (  0.89);

\draw[color=drawColor,line cap=round,line join=round,fill opacity=0.00,] (248.98,247.26) circle (  0.89);

\draw[color=drawColor,line cap=round,line join=round,fill opacity=0.00,] (255.16,293.53) circle (  0.89);

\draw[color=drawColor,line cap=round,line join=round,fill opacity=0.00,] (261.34,356.29) circle (  0.89);

\draw[color=drawColor,line cap=round,line join=round,fill opacity=0.00,] (267.53,348.50) circle (  0.89);

\draw[color=drawColor,line cap=round,line join=round,fill opacity=0.00,] (273.71,311.40) circle (  0.89);

\draw[color=drawColor,line cap=round,line join=round,fill opacity=0.00,] (279.89,310.32) circle (  0.89);

\draw[color=drawColor,line cap=round,line join=round,fill opacity=0.00,] (286.07,278.53) circle (  0.89);

\draw[color=drawColor,line cap=round,line join=round,fill opacity=0.00,] (292.26,257.40) circle (  0.89);

\draw[color=drawColor,line cap=round,line join=round,fill opacity=0.00,] (298.44,249.42) circle (  0.89);

\draw[color=drawColor,line cap=round,line join=round,fill opacity=0.00,] (304.62,259.07) circle (  0.89);

\draw[color=drawColor,line cap=round,line join=round,fill opacity=0.00,] (310.80,274.07) circle (  0.89);

\draw[color=drawColor,line cap=round,line join=round,fill opacity=0.00,] (316.99,261.05) circle (  0.89);

\draw[color=drawColor,line cap=round,line join=round,fill opacity=0.00,] (323.17,316.99) circle (  0.89);

\draw[color=drawColor,line cap=round,line join=round,fill opacity=0.00,] (329.35,263.70) circle (  0.89);

\draw[color=drawColor,line cap=round,line join=round,fill opacity=0.00,] (335.53,361.64) circle (  0.89);

\draw[color=drawColor,line cap=round,line join=round,fill opacity=0.00,] (341.72,269.56) circle (  0.89);

\draw[color=drawColor,line cap=round,line join=round,fill opacity=0.00,] (347.90,318.78) circle (  0.89);

\draw[color=drawColor,line cap=round,line join=round,fill opacity=0.00,] (354.08,265.08) circle (  0.89);

\draw[color=drawColor,line cap=round,line join=round,fill opacity=0.00,] (360.26,278.14) circle (  0.89);

\draw[color=drawColor,line cap=round,line join=round,fill opacity=0.00,] (366.45,247.54) circle (  0.89);

\draw[color=drawColor,line cap=round,line join=round,fill opacity=0.00,] (372.63,250.62) circle (  0.89);

\draw[color=drawColor,line cap=round,line join=round,fill opacity=0.00,] (378.81,230.20) circle (  0.89);

\draw[color=drawColor,line cap=round,line join=round,fill opacity=0.00,] (384.99,255.70) circle (  0.89);

\draw[color=drawColor,line cap=round,line join=round,fill opacity=0.00,] (391.18,246.70) circle (  0.89);

\draw[color=drawColor,line cap=round,line join=round,fill opacity=0.00,] (397.36,279.73) circle (  0.89);

\draw[color=drawColor,line cap=round,line join=round,fill opacity=0.00,] (403.54,252.09) circle (  0.89);

\draw[color=drawColor,line cap=round,line join=round,fill opacity=0.00,] (409.72,255.43) circle (  0.89);
\end{scope}
\begin{scope}
\path[clip] (  0.00,  0.00) rectangle (469.75,614.29);
\definecolor[named]{drawColor}{rgb}{0.00,0.00,0.00}

\node[color=drawColor,anchor=base,inner sep=0pt, outer sep=0pt, scale=  1.00] at (242.80,392.90) {(b) RPSS = 0.56 MC = 0.66%
};
\end{scope}
\begin{scope}
\path[clip] ( 32.47, 16.63) rectangle (453.12,180.21);
\end{scope}
\begin{scope}
\path[clip] ( 32.47, 16.63) rectangle (453.12,180.21);
\definecolor[named]{drawColor}{rgb}{0.00,0.00,0.00}

\draw[color=drawColor,line width= 1.2pt,line join=round,fill opacity=0.00,] ( 48.67, 78.73) -- ( 53.62, 78.73);

\draw[color=drawColor,dash pattern=on 4pt off 4pt ,line cap=round,line join=round,fill opacity=0.00,] ( 51.14, 57.00) -- ( 51.14, 71.26);

\draw[color=drawColor,dash pattern=on 4pt off 4pt ,line cap=round,line join=round,fill opacity=0.00,] ( 51.14,118.15) -- ( 51.14, 89.89);

\draw[color=drawColor,line cap=round,line join=round,fill opacity=0.00,] ( 49.91, 57.00) -- ( 52.38, 57.00);

\draw[color=drawColor,line cap=round,line join=round,fill opacity=0.00,] ( 49.91,118.15) -- ( 52.38,118.15);

\draw[color=drawColor,line cap=round,line join=round,fill opacity=0.00,] ( 48.67, 71.26) --
	( 53.62, 71.26) --
	( 53.62, 89.89) --
	( 48.67, 89.89) --
	( 48.67, 71.26);

\draw[color=drawColor,line width= 1.2pt,line join=round,fill opacity=0.00,] ( 54.85, 87.36) -- ( 59.80, 87.36);

\draw[color=drawColor,dash pattern=on 4pt off 4pt ,line cap=round,line join=round,fill opacity=0.00,] ( 57.33, 69.15) -- ( 57.33, 75.20);

\draw[color=drawColor,dash pattern=on 4pt off 4pt ,line cap=round,line join=round,fill opacity=0.00,] ( 57.33,123.21) -- ( 57.33,100.04);

\draw[color=drawColor,line cap=round,line join=round,fill opacity=0.00,] ( 56.09, 69.15) -- ( 58.56, 69.15);

\draw[color=drawColor,line cap=round,line join=round,fill opacity=0.00,] ( 56.09,123.21) -- ( 58.56,123.21);

\draw[color=drawColor,line cap=round,line join=round,fill opacity=0.00,] ( 54.85, 75.20) --
	( 59.80, 75.20) --
	( 59.80,100.04) --
	( 54.85,100.04) --
	( 54.85, 75.20);

\draw[color=drawColor,line width= 1.2pt,line join=round,fill opacity=0.00,] ( 61.03, 75.85) -- ( 65.98, 75.85);

\draw[color=drawColor,dash pattern=on 4pt off 4pt ,line cap=round,line join=round,fill opacity=0.00,] ( 63.51, 52.74) -- ( 63.51, 66.57);

\draw[color=drawColor,dash pattern=on 4pt off 4pt ,line cap=round,line join=round,fill opacity=0.00,] ( 63.51,117.68) -- ( 63.51, 90.93);

\draw[color=drawColor,line cap=round,line join=round,fill opacity=0.00,] ( 62.27, 52.74) -- ( 64.74, 52.74);

\draw[color=drawColor,line cap=round,line join=round,fill opacity=0.00,] ( 62.27,117.68) -- ( 64.74,117.68);

\draw[color=drawColor,line cap=round,line join=round,fill opacity=0.00,] ( 61.03, 66.57) --
	( 65.98, 66.57) --
	( 65.98, 90.93) --
	( 61.03, 90.93) --
	( 61.03, 66.57);

\draw[color=drawColor,line width= 1.2pt,line join=round,fill opacity=0.00,] ( 67.22, 88.29) -- ( 72.16, 88.29);

\draw[color=drawColor,dash pattern=on 4pt off 4pt ,line cap=round,line join=round,fill opacity=0.00,] ( 69.69, 70.02) -- ( 69.69, 76.32);

\draw[color=drawColor,dash pattern=on 4pt off 4pt ,line cap=round,line join=round,fill opacity=0.00,] ( 69.69,123.02) -- ( 69.69,100.18);

\draw[color=drawColor,line cap=round,line join=round,fill opacity=0.00,] ( 68.45, 70.02) -- ( 70.93, 70.02);

\draw[color=drawColor,line cap=round,line join=round,fill opacity=0.00,] ( 68.45,123.02) -- ( 70.93,123.02);

\draw[color=drawColor,line cap=round,line join=round,fill opacity=0.00,] ( 67.22, 76.32) --
	( 72.16, 76.32) --
	( 72.16,100.18) --
	( 67.22,100.18) --
	( 67.22, 76.32);

\draw[color=drawColor,line width= 1.2pt,line join=round,fill opacity=0.00,] ( 73.40, 69.98) -- ( 78.35, 69.98);

\draw[color=drawColor,dash pattern=on 4pt off 4pt ,line cap=round,line join=round,fill opacity=0.00,] ( 75.87, 45.46) -- ( 75.87, 58.72);

\draw[color=drawColor,dash pattern=on 4pt off 4pt ,line cap=round,line join=round,fill opacity=0.00,] ( 75.87,105.83) -- ( 75.87, 79.44);

\draw[color=drawColor,line cap=round,line join=round,fill opacity=0.00,] ( 74.64, 45.46) -- ( 77.11, 45.46);

\draw[color=drawColor,line cap=round,line join=round,fill opacity=0.00,] ( 74.64,105.83) -- ( 77.11,105.83);

\draw[color=drawColor,line cap=round,line join=round,fill opacity=0.00,] ( 73.40, 58.72) --
	( 78.35, 58.72) --
	( 78.35, 79.44) --
	( 73.40, 79.44) --
	( 73.40, 58.72);

\draw[color=drawColor,line width= 1.2pt,line join=round,fill opacity=0.00,] ( 79.58, 75.32) -- ( 84.53, 75.32);

\draw[color=drawColor,dash pattern=on 4pt off 4pt ,line cap=round,line join=round,fill opacity=0.00,] ( 82.05, 56.29) -- ( 82.05, 69.17);

\draw[color=drawColor,dash pattern=on 4pt off 4pt ,line cap=round,line join=round,fill opacity=0.00,] ( 82.05,115.83) -- ( 82.05, 85.50);

\draw[color=drawColor,line cap=round,line join=round,fill opacity=0.00,] ( 80.82, 56.29) -- ( 83.29, 56.29);

\draw[color=drawColor,line cap=round,line join=round,fill opacity=0.00,] ( 80.82,115.83) -- ( 83.29,115.83);

\draw[color=drawColor,line cap=round,line join=round,fill opacity=0.00,] ( 79.58, 69.17) --
	( 84.53, 69.17) --
	( 84.53, 85.50) --
	( 79.58, 85.50) --
	( 79.58, 69.17);

\draw[color=drawColor,line width= 1.2pt,line join=round,fill opacity=0.00,] ( 85.76, 74.57) -- ( 90.71, 74.57);

\draw[color=drawColor,dash pattern=on 4pt off 4pt ,line cap=round,line join=round,fill opacity=0.00,] ( 88.24, 54.73) -- ( 88.24, 66.78);

\draw[color=drawColor,dash pattern=on 4pt off 4pt ,line cap=round,line join=round,fill opacity=0.00,] ( 88.24,113.10) -- ( 88.24, 84.71);

\draw[color=drawColor,line cap=round,line join=round,fill opacity=0.00,] ( 87.00, 54.73) -- ( 89.47, 54.73);

\draw[color=drawColor,line cap=round,line join=round,fill opacity=0.00,] ( 87.00,113.10) -- ( 89.47,113.10);

\draw[color=drawColor,line cap=round,line join=round,fill opacity=0.00,] ( 85.76, 66.78) --
	( 90.71, 66.78) --
	( 90.71, 84.71) --
	( 85.76, 84.71) --
	( 85.76, 66.78);

\draw[color=drawColor,line width= 1.2pt,line join=round,fill opacity=0.00,] ( 91.95, 73.55) -- ( 96.89, 73.55);

\draw[color=drawColor,dash pattern=on 4pt off 4pt ,line cap=round,line join=round,fill opacity=0.00,] ( 94.42, 53.05) -- ( 94.42, 65.21);

\draw[color=drawColor,dash pattern=on 4pt off 4pt ,line cap=round,line join=round,fill opacity=0.00,] ( 94.42,113.07) -- ( 94.42, 84.33);

\draw[color=drawColor,line cap=round,line join=round,fill opacity=0.00,] ( 93.18, 53.05) -- ( 95.66, 53.05);

\draw[color=drawColor,line cap=round,line join=round,fill opacity=0.00,] ( 93.18,113.07) -- ( 95.66,113.07);

\draw[color=drawColor,line cap=round,line join=round,fill opacity=0.00,] ( 91.95, 65.21) --
	( 96.89, 65.21) --
	( 96.89, 84.33) --
	( 91.95, 84.33) --
	( 91.95, 65.21);

\draw[color=drawColor,line width= 1.2pt,line join=round,fill opacity=0.00,] ( 98.13,100.92) -- (103.08,100.92);

\draw[color=drawColor,dash pattern=on 4pt off 4pt ,line cap=round,line join=round,fill opacity=0.00,] (100.60, 62.83) -- (100.60, 79.89);

\draw[color=drawColor,dash pattern=on 4pt off 4pt ,line cap=round,line join=round,fill opacity=0.00,] (100.60,140.59) -- (100.60,125.67);

\draw[color=drawColor,line cap=round,line join=round,fill opacity=0.00,] ( 99.37, 62.83) -- (101.84, 62.83);

\draw[color=drawColor,line cap=round,line join=round,fill opacity=0.00,] ( 99.37,140.59) -- (101.84,140.59);

\draw[color=drawColor,line cap=round,line join=round,fill opacity=0.00,] ( 98.13, 79.89) --
	(103.08, 79.89) --
	(103.08,125.67) --
	( 98.13,125.67) --
	( 98.13, 79.89);

\draw[color=drawColor,line width= 1.2pt,line join=round,fill opacity=0.00,] (104.31,102.49) -- (109.26,102.49);

\draw[color=drawColor,dash pattern=on 4pt off 4pt ,line cap=round,line join=round,fill opacity=0.00,] (106.78, 73.34) -- (106.78, 85.02);

\draw[color=drawColor,dash pattern=on 4pt off 4pt ,line cap=round,line join=round,fill opacity=0.00,] (106.78,147.28) -- (106.78,127.18);

\draw[color=drawColor,line cap=round,line join=round,fill opacity=0.00,] (105.55, 73.34) -- (108.02, 73.34);

\draw[color=drawColor,line cap=round,line join=round,fill opacity=0.00,] (105.55,147.28) -- (108.02,147.28);

\draw[color=drawColor,line cap=round,line join=round,fill opacity=0.00,] (104.31, 85.02) --
	(109.26, 85.02) --
	(109.26,127.18) --
	(104.31,127.18) --
	(104.31, 85.02);

\draw[color=drawColor,line width= 1.2pt,line join=round,fill opacity=0.00,] (110.49, 81.65) -- (115.44, 81.65);

\draw[color=drawColor,dash pattern=on 4pt off 4pt ,line cap=round,line join=round,fill opacity=0.00,] (112.97, 59.30) -- (112.97, 72.29);

\draw[color=drawColor,dash pattern=on 4pt off 4pt ,line cap=round,line join=round,fill opacity=0.00,] (112.97,119.47) -- (112.97, 92.16);

\draw[color=drawColor,line cap=round,line join=round,fill opacity=0.00,] (111.73, 59.30) -- (114.20, 59.30);

\draw[color=drawColor,line cap=round,line join=round,fill opacity=0.00,] (111.73,119.47) -- (114.20,119.47);

\draw[color=drawColor,line cap=round,line join=round,fill opacity=0.00,] (110.49, 72.29) --
	(115.44, 72.29) --
	(115.44, 92.16) --
	(110.49, 92.16) --
	(110.49, 72.29);

\draw[color=drawColor,line width= 1.2pt,line join=round,fill opacity=0.00,] (116.68, 83.25) -- (121.62, 83.25);

\draw[color=drawColor,dash pattern=on 4pt off 4pt ,line cap=round,line join=round,fill opacity=0.00,] (119.15, 57.38) -- (119.15, 73.05);

\draw[color=drawColor,dash pattern=on 4pt off 4pt ,line cap=round,line join=round,fill opacity=0.00,] (119.15,130.02) -- (119.15,102.14);

\draw[color=drawColor,line cap=round,line join=round,fill opacity=0.00,] (117.91, 57.38) -- (120.39, 57.38);

\draw[color=drawColor,line cap=round,line join=round,fill opacity=0.00,] (117.91,130.02) -- (120.39,130.02);

\draw[color=drawColor,line cap=round,line join=round,fill opacity=0.00,] (116.68, 73.05) --
	(121.62, 73.05) --
	(121.62,102.14) --
	(116.68,102.14) --
	(116.68, 73.05);

\draw[color=drawColor,line width= 1.2pt,line join=round,fill opacity=0.00,] (122.86, 84.21) -- (127.80, 84.21);

\draw[color=drawColor,dash pattern=on 4pt off 4pt ,line cap=round,line join=round,fill opacity=0.00,] (125.33, 65.69) -- (125.33, 74.01);

\draw[color=drawColor,dash pattern=on 4pt off 4pt ,line cap=round,line join=round,fill opacity=0.00,] (125.33,122.66) -- (125.33, 93.42);

\draw[color=drawColor,line cap=round,line join=round,fill opacity=0.00,] (124.10, 65.69) -- (126.57, 65.69);

\draw[color=drawColor,line cap=round,line join=round,fill opacity=0.00,] (124.10,122.66) -- (126.57,122.66);

\draw[color=drawColor,line cap=round,line join=round,fill opacity=0.00,] (122.86, 74.01) --
	(127.80, 74.01) --
	(127.80, 93.42) --
	(122.86, 93.42) --
	(122.86, 74.01);

\draw[color=drawColor,line width= 1.2pt,line join=round,fill opacity=0.00,] (129.04, 93.68) -- (133.99, 93.68);

\draw[color=drawColor,dash pattern=on 4pt off 4pt ,line cap=round,line join=round,fill opacity=0.00,] (131.51, 72.10) -- (131.51, 79.65);

\draw[color=drawColor,dash pattern=on 4pt off 4pt ,line cap=round,line join=round,fill opacity=0.00,] (131.51,127.86) -- (131.51,110.26);

\draw[color=drawColor,line cap=round,line join=round,fill opacity=0.00,] (130.28, 72.10) -- (132.75, 72.10);

\draw[color=drawColor,line cap=round,line join=round,fill opacity=0.00,] (130.28,127.86) -- (132.75,127.86);

\draw[color=drawColor,line cap=round,line join=round,fill opacity=0.00,] (129.04, 79.65) --
	(133.99, 79.65) --
	(133.99,110.26) --
	(129.04,110.26) --
	(129.04, 79.65);

\draw[color=drawColor,line width= 1.2pt,line join=round,fill opacity=0.00,] (135.22, 79.11) -- (140.17, 79.11);

\draw[color=drawColor,dash pattern=on 4pt off 4pt ,line cap=round,line join=round,fill opacity=0.00,] (137.70, 57.54) -- (137.70, 71.21);

\draw[color=drawColor,dash pattern=on 4pt off 4pt ,line cap=round,line join=round,fill opacity=0.00,] (137.70,118.68) -- (137.70, 89.78);

\draw[color=drawColor,line cap=round,line join=round,fill opacity=0.00,] (136.46, 57.54) -- (138.93, 57.54);

\draw[color=drawColor,line cap=round,line join=round,fill opacity=0.00,] (136.46,118.68) -- (138.93,118.68);

\draw[color=drawColor,line cap=round,line join=round,fill opacity=0.00,] (135.22, 71.21) --
	(140.17, 71.21) --
	(140.17, 89.78) --
	(135.22, 89.78) --
	(135.22, 71.21);

\draw[color=drawColor,line width= 1.2pt,line join=round,fill opacity=0.00,] (141.41, 86.10) -- (146.35, 86.10);

\draw[color=drawColor,dash pattern=on 4pt off 4pt ,line cap=round,line join=round,fill opacity=0.00,] (143.88, 63.54) -- (143.88, 74.92);

\draw[color=drawColor,dash pattern=on 4pt off 4pt ,line cap=round,line join=round,fill opacity=0.00,] (143.88,132.27) -- (143.88,107.86);

\draw[color=drawColor,line cap=round,line join=round,fill opacity=0.00,] (142.64, 63.54) -- (145.12, 63.54);

\draw[color=drawColor,line cap=round,line join=round,fill opacity=0.00,] (142.64,132.27) -- (145.12,132.27);

\draw[color=drawColor,line cap=round,line join=round,fill opacity=0.00,] (141.41, 74.92) --
	(146.35, 74.92) --
	(146.35,107.86) --
	(141.41,107.86) --
	(141.41, 74.92);

\draw[color=drawColor,line width= 1.2pt,line join=round,fill opacity=0.00,] (147.59, 74.08) -- (152.53, 74.08);

\draw[color=drawColor,dash pattern=on 4pt off 4pt ,line cap=round,line join=round,fill opacity=0.00,] (150.06, 53.74) -- (150.06, 65.60);

\draw[color=drawColor,dash pattern=on 4pt off 4pt ,line cap=round,line join=round,fill opacity=0.00,] (150.06,107.64) -- (150.06, 84.78);

\draw[color=drawColor,line cap=round,line join=round,fill opacity=0.00,] (148.82, 53.74) -- (151.30, 53.74);

\draw[color=drawColor,line cap=round,line join=round,fill opacity=0.00,] (148.82,107.64) -- (151.30,107.64);

\draw[color=drawColor,line cap=round,line join=round,fill opacity=0.00,] (147.59, 65.60) --
	(152.53, 65.60) --
	(152.53, 84.78) --
	(147.59, 84.78) --
	(147.59, 65.60);

\draw[color=drawColor,line width= 1.2pt,line join=round,fill opacity=0.00,] (153.77, 71.93) -- (158.72, 71.93);

\draw[color=drawColor,dash pattern=on 4pt off 4pt ,line cap=round,line join=round,fill opacity=0.00,] (156.24, 41.38) -- (156.24, 60.32);

\draw[color=drawColor,dash pattern=on 4pt off 4pt ,line cap=round,line join=round,fill opacity=0.00,] (156.24,145.40) -- (156.24, 97.43);

\draw[color=drawColor,line cap=round,line join=round,fill opacity=0.00,] (155.01, 41.38) -- (157.48, 41.38);

\draw[color=drawColor,line cap=round,line join=round,fill opacity=0.00,] (155.01,145.40) -- (157.48,145.40);

\draw[color=drawColor,line cap=round,line join=round,fill opacity=0.00,] (153.77, 60.32) --
	(158.72, 60.32) --
	(158.72, 97.43) --
	(153.77, 97.43) --
	(153.77, 60.32);

\draw[color=drawColor,line width= 1.2pt,line join=round,fill opacity=0.00,] (159.95, 83.33) -- (164.90, 83.33);

\draw[color=drawColor,dash pattern=on 4pt off 4pt ,line cap=round,line join=round,fill opacity=0.00,] (162.43, 60.29) -- (162.43, 72.72);

\draw[color=drawColor,dash pattern=on 4pt off 4pt ,line cap=round,line join=round,fill opacity=0.00,] (162.43,141.04) -- (162.43,102.02);

\draw[color=drawColor,line cap=round,line join=round,fill opacity=0.00,] (161.19, 60.29) -- (163.66, 60.29);

\draw[color=drawColor,line cap=round,line join=round,fill opacity=0.00,] (161.19,141.04) -- (163.66,141.04);

\draw[color=drawColor,line cap=round,line join=round,fill opacity=0.00,] (159.95, 72.72) --
	(164.90, 72.72) --
	(164.90,102.02) --
	(159.95,102.02) --
	(159.95, 72.72);

\draw[color=drawColor,line width= 1.2pt,line join=round,fill opacity=0.00,] (166.14, 74.94) -- (171.08, 74.94);

\draw[color=drawColor,dash pattern=on 4pt off 4pt ,line cap=round,line join=round,fill opacity=0.00,] (168.61, 53.70) -- (168.61, 66.73);

\draw[color=drawColor,dash pattern=on 4pt off 4pt ,line cap=round,line join=round,fill opacity=0.00,] (168.61,116.76) -- (168.61, 85.62);

\draw[color=drawColor,line cap=round,line join=round,fill opacity=0.00,] (167.37, 53.70) -- (169.85, 53.70);

\draw[color=drawColor,line cap=round,line join=round,fill opacity=0.00,] (167.37,116.76) -- (169.85,116.76);

\draw[color=drawColor,line cap=round,line join=round,fill opacity=0.00,] (166.14, 66.73) --
	(171.08, 66.73) --
	(171.08, 85.62) --
	(166.14, 85.62) --
	(166.14, 66.73);

\draw[color=drawColor,line width= 1.2pt,line join=round,fill opacity=0.00,] (172.32, 92.56) -- (177.26, 92.56);

\draw[color=drawColor,dash pattern=on 4pt off 4pt ,line cap=round,line join=round,fill opacity=0.00,] (174.79, 71.49) -- (174.79, 78.48);

\draw[color=drawColor,dash pattern=on 4pt off 4pt ,line cap=round,line join=round,fill opacity=0.00,] (174.79,128.77) -- (174.79,104.77);

\draw[color=drawColor,line cap=round,line join=round,fill opacity=0.00,] (173.55, 71.49) -- (176.03, 71.49);

\draw[color=drawColor,line cap=round,line join=round,fill opacity=0.00,] (173.55,128.77) -- (176.03,128.77);

\draw[color=drawColor,line cap=round,line join=round,fill opacity=0.00,] (172.32, 78.48) --
	(177.26, 78.48) --
	(177.26,104.77) --
	(172.32,104.77) --
	(172.32, 78.48);

\draw[color=drawColor,line width= 1.2pt,line join=round,fill opacity=0.00,] (178.50, 91.96) -- (183.45, 91.96);

\draw[color=drawColor,dash pattern=on 4pt off 4pt ,line cap=round,line join=round,fill opacity=0.00,] (180.97, 71.17) -- (180.97, 78.27);

\draw[color=drawColor,dash pattern=on 4pt off 4pt ,line cap=round,line join=round,fill opacity=0.00,] (180.97,129.98) -- (180.97,106.62);

\draw[color=drawColor,line cap=round,line join=round,fill opacity=0.00,] (179.74, 71.17) -- (182.21, 71.17);

\draw[color=drawColor,line cap=round,line join=round,fill opacity=0.00,] (179.74,129.98) -- (182.21,129.98);

\draw[color=drawColor,line cap=round,line join=round,fill opacity=0.00,] (178.50, 78.27) --
	(183.45, 78.27) --
	(183.45,106.62) --
	(178.50,106.62) --
	(178.50, 78.27);

\draw[color=drawColor,line width= 1.2pt,line join=round,fill opacity=0.00,] (184.68, 94.03) -- (189.63, 94.03);

\draw[color=drawColor,dash pattern=on 4pt off 4pt ,line cap=round,line join=round,fill opacity=0.00,] (187.16, 71.66) -- (187.16, 78.76);

\draw[color=drawColor,dash pattern=on 4pt off 4pt ,line cap=round,line join=round,fill opacity=0.00,] (187.16,133.83) -- (187.16,114.09);

\draw[color=drawColor,line cap=round,line join=round,fill opacity=0.00,] (185.92, 71.66) -- (188.39, 71.66);

\draw[color=drawColor,line cap=round,line join=round,fill opacity=0.00,] (185.92,133.83) -- (188.39,133.83);

\draw[color=drawColor,line cap=round,line join=round,fill opacity=0.00,] (184.68, 78.76) --
	(189.63, 78.76) --
	(189.63,114.09) --
	(184.68,114.09) --
	(184.68, 78.76);

\draw[color=drawColor,line width= 1.2pt,line join=round,fill opacity=0.00,] (190.87,101.78) -- (195.81,101.78);

\draw[color=drawColor,dash pattern=on 4pt off 4pt ,line cap=round,line join=round,fill opacity=0.00,] (193.34, 73.40) -- (193.34, 88.65);

\draw[color=drawColor,dash pattern=on 4pt off 4pt ,line cap=round,line join=round,fill opacity=0.00,] (193.34,143.39) -- (193.34,127.81);

\draw[color=drawColor,line cap=round,line join=round,fill opacity=0.00,] (192.10, 73.40) -- (194.57, 73.40);

\draw[color=drawColor,line cap=round,line join=round,fill opacity=0.00,] (192.10,143.39) -- (194.57,143.39);

\draw[color=drawColor,line cap=round,line join=round,fill opacity=0.00,] (190.87, 88.65) --
	(195.81, 88.65) --
	(195.81,127.81) --
	(190.87,127.81) --
	(190.87, 88.65);

\draw[color=drawColor,line width= 1.2pt,line join=round,fill opacity=0.00,] (197.05, 95.72) -- (201.99, 95.72);

\draw[color=drawColor,dash pattern=on 4pt off 4pt ,line cap=round,line join=round,fill opacity=0.00,] (199.52, 73.43) -- (199.52, 82.29);

\draw[color=drawColor,dash pattern=on 4pt off 4pt ,line cap=round,line join=round,fill opacity=0.00,] (199.52,136.85) -- (199.52,115.81);

\draw[color=drawColor,line cap=round,line join=round,fill opacity=0.00,] (198.28, 73.43) -- (200.76, 73.43);

\draw[color=drawColor,line cap=round,line join=round,fill opacity=0.00,] (198.28,136.85) -- (200.76,136.85);

\draw[color=drawColor,line cap=round,line join=round,fill opacity=0.00,] (197.05, 82.29) --
	(201.99, 82.29) --
	(201.99,115.81) --
	(197.05,115.81) --
	(197.05, 82.29);

\draw[color=drawColor,line width= 1.2pt,line join=round,fill opacity=0.00,] (203.23, 94.00) -- (208.18, 94.00);

\draw[color=drawColor,dash pattern=on 4pt off 4pt ,line cap=round,line join=round,fill opacity=0.00,] (205.70, 68.02) -- (205.70, 77.04);

\draw[color=drawColor,dash pattern=on 4pt off 4pt ,line cap=round,line join=round,fill opacity=0.00,] (205.70,137.41) -- (205.70,120.77);

\draw[color=drawColor,line cap=round,line join=round,fill opacity=0.00,] (204.47, 68.02) -- (206.94, 68.02);

\draw[color=drawColor,line cap=round,line join=round,fill opacity=0.00,] (204.47,137.41) -- (206.94,137.41);

\draw[color=drawColor,line cap=round,line join=round,fill opacity=0.00,] (203.23, 77.04) --
	(208.18, 77.04) --
	(208.18,120.77) --
	(203.23,120.77) --
	(203.23, 77.04);

\draw[color=drawColor,line width= 1.2pt,line join=round,fill opacity=0.00,] (209.41, 82.57) -- (214.36, 82.57);

\draw[color=drawColor,dash pattern=on 4pt off 4pt ,line cap=round,line join=round,fill opacity=0.00,] (211.89, 58.14) -- (211.89, 72.05);

\draw[color=drawColor,dash pattern=on 4pt off 4pt ,line cap=round,line join=round,fill opacity=0.00,] (211.89,119.85) -- (211.89, 91.94);

\draw[color=drawColor,line cap=round,line join=round,fill opacity=0.00,] (210.65, 58.14) -- (213.12, 58.14);

\draw[color=drawColor,line cap=round,line join=round,fill opacity=0.00,] (210.65,119.85) -- (213.12,119.85);

\draw[color=drawColor,line cap=round,line join=round,fill opacity=0.00,] (209.41, 72.05) --
	(214.36, 72.05) --
	(214.36, 91.94) --
	(209.41, 91.94) --
	(209.41, 72.05);

\draw[color=drawColor,line width= 1.2pt,line join=round,fill opacity=0.00,] (215.59,101.47) -- (220.54,101.47);

\draw[color=drawColor,dash pattern=on 4pt off 4pt ,line cap=round,line join=round,fill opacity=0.00,] (218.07, 71.27) -- (218.07, 84.77);

\draw[color=drawColor,dash pattern=on 4pt off 4pt ,line cap=round,line join=round,fill opacity=0.00,] (218.07,150.73) -- (218.07,130.68);

\draw[color=drawColor,line cap=round,line join=round,fill opacity=0.00,] (216.83, 71.27) -- (219.30, 71.27);

\draw[color=drawColor,line cap=round,line join=round,fill opacity=0.00,] (216.83,150.73) -- (219.30,150.73);

\draw[color=drawColor,line cap=round,line join=round,fill opacity=0.00,] (215.59, 84.77) --
	(220.54, 84.77) --
	(220.54,130.68) --
	(215.59,130.68) --
	(215.59, 84.77);

\draw[color=drawColor,line width= 1.2pt,line join=round,fill opacity=0.00,] (221.78, 77.73) -- (226.72, 77.73);

\draw[color=drawColor,dash pattern=on 4pt off 4pt ,line cap=round,line join=round,fill opacity=0.00,] (224.25, 55.18) -- (224.25, 69.39);

\draw[color=drawColor,dash pattern=on 4pt off 4pt ,line cap=round,line join=round,fill opacity=0.00,] (224.25,121.64) -- (224.25, 90.85);

\draw[color=drawColor,line cap=round,line join=round,fill opacity=0.00,] (223.01, 55.18) -- (225.49, 55.18);

\draw[color=drawColor,line cap=round,line join=round,fill opacity=0.00,] (223.01,121.64) -- (225.49,121.64);

\draw[color=drawColor,line cap=round,line join=round,fill opacity=0.00,] (221.78, 69.39) --
	(226.72, 69.39) --
	(226.72, 90.85) --
	(221.78, 90.85) --
	(221.78, 69.39);

\draw[color=drawColor,line width= 1.2pt,line join=round,fill opacity=0.00,] (227.96, 80.71) -- (232.91, 80.71);

\draw[color=drawColor,dash pattern=on 4pt off 4pt ,line cap=round,line join=round,fill opacity=0.00,] (230.43, 56.65) -- (230.43, 71.50);

\draw[color=drawColor,dash pattern=on 4pt off 4pt ,line cap=round,line join=round,fill opacity=0.00,] (230.43,121.06) -- (230.43, 89.31);

\draw[color=drawColor,line cap=round,line join=round,fill opacity=0.00,] (229.20, 56.65) -- (231.67, 56.65);

\draw[color=drawColor,line cap=round,line join=round,fill opacity=0.00,] (229.20,121.06) -- (231.67,121.06);

\draw[color=drawColor,line cap=round,line join=round,fill opacity=0.00,] (227.96, 71.50) --
	(232.91, 71.50) --
	(232.91, 89.31) --
	(227.96, 89.31) --
	(227.96, 71.50);

\draw[color=drawColor,line width= 1.2pt,line join=round,fill opacity=0.00,] (234.14, 71.09) -- (239.09, 71.09);

\draw[color=drawColor,dash pattern=on 4pt off 4pt ,line cap=round,line join=round,fill opacity=0.00,] (236.62, 51.38) -- (236.62, 63.59);

\draw[color=drawColor,dash pattern=on 4pt off 4pt ,line cap=round,line join=round,fill opacity=0.00,] (236.62,107.07) -- (236.62, 83.60);

\draw[color=drawColor,line cap=round,line join=round,fill opacity=0.00,] (235.38, 51.38) -- (237.85, 51.38);

\draw[color=drawColor,line cap=round,line join=round,fill opacity=0.00,] (235.38,107.07) -- (237.85,107.07);

\draw[color=drawColor,line cap=round,line join=round,fill opacity=0.00,] (234.14, 63.59) --
	(239.09, 63.59) --
	(239.09, 83.60) --
	(234.14, 83.60) --
	(234.14, 63.59);

\draw[color=drawColor,line width= 1.2pt,line join=round,fill opacity=0.00,] (240.32, 78.25) -- (245.27, 78.25);

\draw[color=drawColor,dash pattern=on 4pt off 4pt ,line cap=round,line join=round,fill opacity=0.00,] (242.80, 55.84) -- (242.80, 70.02);

\draw[color=drawColor,dash pattern=on 4pt off 4pt ,line cap=round,line join=round,fill opacity=0.00,] (242.80,119.29) -- (242.80, 88.68);

\draw[color=drawColor,line cap=round,line join=round,fill opacity=0.00,] (241.56, 55.84) -- (244.03, 55.84);

\draw[color=drawColor,line cap=round,line join=round,fill opacity=0.00,] (241.56,119.29) -- (244.03,119.29);

\draw[color=drawColor,line cap=round,line join=round,fill opacity=0.00,] (240.32, 70.02) --
	(245.27, 70.02) --
	(245.27, 88.68) --
	(240.32, 88.68) --
	(240.32, 70.02);

\draw[color=drawColor,line width= 1.2pt,line join=round,fill opacity=0.00,] (246.51, 74.14) -- (251.45, 74.14);

\draw[color=drawColor,dash pattern=on 4pt off 4pt ,line cap=round,line join=round,fill opacity=0.00,] (248.98, 48.56) -- (248.98, 63.09);

\draw[color=drawColor,dash pattern=on 4pt off 4pt ,line cap=round,line join=round,fill opacity=0.00,] (248.98,121.77) -- (248.98, 88.44);

\draw[color=drawColor,line cap=round,line join=round,fill opacity=0.00,] (247.74, 48.56) -- (250.22, 48.56);

\draw[color=drawColor,line cap=round,line join=round,fill opacity=0.00,] (247.74,121.77) -- (250.22,121.77);

\draw[color=drawColor,line cap=round,line join=round,fill opacity=0.00,] (246.51, 63.09) --
	(251.45, 63.09) --
	(251.45, 88.44) --
	(246.51, 88.44) --
	(246.51, 63.09);

\draw[color=drawColor,line width= 1.2pt,line join=round,fill opacity=0.00,] (252.69,115.10) -- (257.64,115.10);

\draw[color=drawColor,dash pattern=on 4pt off 4pt ,line cap=round,line join=round,fill opacity=0.00,] (255.16, 74.67) -- (255.16, 95.72);

\draw[color=drawColor,dash pattern=on 4pt off 4pt ,line cap=round,line join=round,fill opacity=0.00,] (255.16,148.59) -- (255.16,134.05);

\draw[color=drawColor,line cap=round,line join=round,fill opacity=0.00,] (253.93, 74.67) -- (256.40, 74.67);

\draw[color=drawColor,line cap=round,line join=round,fill opacity=0.00,] (253.93,148.59) -- (256.40,148.59);

\draw[color=drawColor,line cap=round,line join=round,fill opacity=0.00,] (252.69, 95.72) --
	(257.64, 95.72) --
	(257.64,134.05) --
	(252.69,134.05) --
	(252.69, 95.72);

\draw[color=drawColor,line width= 1.2pt,line join=round,fill opacity=0.00,] (258.87,124.29) -- (263.82,124.29);

\draw[color=drawColor,dash pattern=on 4pt off 4pt ,line cap=round,line join=round,fill opacity=0.00,] (261.34, 74.58) -- (261.34, 97.85);

\draw[color=drawColor,dash pattern=on 4pt off 4pt ,line cap=round,line join=round,fill opacity=0.00,] (261.34,174.15) -- (261.34,139.44);

\draw[color=drawColor,line cap=round,line join=round,fill opacity=0.00,] (260.11, 74.58) -- (262.58, 74.58);

\draw[color=drawColor,line cap=round,line join=round,fill opacity=0.00,] (260.11,174.15) -- (262.58,174.15);

\draw[color=drawColor,line cap=round,line join=round,fill opacity=0.00,] (258.87, 97.85) --
	(263.82, 97.85) --
	(263.82,139.44) --
	(258.87,139.44) --
	(258.87, 97.85);

\draw[color=drawColor,line width= 1.2pt,line join=round,fill opacity=0.00,] (265.05, 95.36) -- (270.00, 95.36);

\draw[color=drawColor,dash pattern=on 4pt off 4pt ,line cap=round,line join=round,fill opacity=0.00,] (267.53, 70.22) -- (267.53, 80.84);

\draw[color=drawColor,dash pattern=on 4pt off 4pt ,line cap=round,line join=round,fill opacity=0.00,] (267.53,143.72) -- (267.53,124.85);

\draw[color=drawColor,line cap=round,line join=round,fill opacity=0.00,] (266.29, 70.22) -- (268.76, 70.22);

\draw[color=drawColor,line cap=round,line join=round,fill opacity=0.00,] (266.29,143.72) -- (268.76,143.72);

\draw[color=drawColor,line cap=round,line join=round,fill opacity=0.00,] (265.05, 80.84) --
	(270.00, 80.84) --
	(270.00,124.85) --
	(265.05,124.85) --
	(265.05, 80.84);

\draw[color=drawColor,line width= 1.2pt,line join=round,fill opacity=0.00,] (271.24,109.13) -- (276.18,109.13);

\draw[color=drawColor,dash pattern=on 4pt off 4pt ,line cap=round,line join=round,fill opacity=0.00,] (273.71, 74.08) -- (273.71, 92.07);

\draw[color=drawColor,dash pattern=on 4pt off 4pt ,line cap=round,line join=round,fill opacity=0.00,] (273.71,152.14) -- (273.71,129.89);

\draw[color=drawColor,line cap=round,line join=round,fill opacity=0.00,] (272.47, 74.08) -- (274.95, 74.08);

\draw[color=drawColor,line cap=round,line join=round,fill opacity=0.00,] (272.47,152.14) -- (274.95,152.14);

\draw[color=drawColor,line cap=round,line join=round,fill opacity=0.00,] (271.24, 92.07) --
	(276.18, 92.07) --
	(276.18,129.89) --
	(271.24,129.89) --
	(271.24, 92.07);

\draw[color=drawColor,line width= 1.2pt,line join=round,fill opacity=0.00,] (277.42,106.97) -- (282.36,106.97);

\draw[color=drawColor,dash pattern=on 4pt off 4pt ,line cap=round,line join=round,fill opacity=0.00,] (279.89, 74.17) -- (279.89, 91.54);

\draw[color=drawColor,dash pattern=on 4pt off 4pt ,line cap=round,line join=round,fill opacity=0.00,] (279.89,147.68) -- (279.89,128.63);

\draw[color=drawColor,line cap=round,line join=round,fill opacity=0.00,] (278.66, 74.17) -- (281.13, 74.17);

\draw[color=drawColor,line cap=round,line join=round,fill opacity=0.00,] (278.66,147.68) -- (281.13,147.68);

\draw[color=drawColor,line cap=round,line join=round,fill opacity=0.00,] (277.42, 91.54) --
	(282.36, 91.54) --
	(282.36,128.63) --
	(277.42,128.63) --
	(277.42, 91.54);

\draw[color=drawColor,line width= 1.2pt,line join=round,fill opacity=0.00,] (283.60, 89.85) -- (288.55, 89.85);

\draw[color=drawColor,dash pattern=on 4pt off 4pt ,line cap=round,line join=round,fill opacity=0.00,] (286.07, 68.45) -- (286.07, 75.47);

\draw[color=drawColor,dash pattern=on 4pt off 4pt ,line cap=round,line join=round,fill opacity=0.00,] (286.07,128.72) -- (286.07,104.50);

\draw[color=drawColor,line cap=round,line join=round,fill opacity=0.00,] (284.84, 68.45) -- (287.31, 68.45);

\draw[color=drawColor,line cap=round,line join=round,fill opacity=0.00,] (284.84,128.72) -- (287.31,128.72);

\draw[color=drawColor,line cap=round,line join=round,fill opacity=0.00,] (283.60, 75.47) --
	(288.55, 75.47) --
	(288.55,104.50) --
	(283.60,104.50) --
	(283.60, 75.47);

\draw[color=drawColor,line width= 1.2pt,line join=round,fill opacity=0.00,] (289.78, 84.96) -- (294.73, 84.96);

\draw[color=drawColor,dash pattern=on 4pt off 4pt ,line cap=round,line join=round,fill opacity=0.00,] (292.26, 57.32) -- (292.26, 72.88);

\draw[color=drawColor,dash pattern=on 4pt off 4pt ,line cap=round,line join=round,fill opacity=0.00,] (292.26,145.38) -- (292.26,107.06);

\draw[color=drawColor,line cap=round,line join=round,fill opacity=0.00,] (291.02, 57.32) -- (293.49, 57.32);

\draw[color=drawColor,line cap=round,line join=round,fill opacity=0.00,] (291.02,145.38) -- (293.49,145.38);

\draw[color=drawColor,line cap=round,line join=round,fill opacity=0.00,] (289.78, 72.88) --
	(294.73, 72.88) --
	(294.73,107.06) --
	(289.78,107.06) --
	(289.78, 72.88);

\draw[color=drawColor,line width= 1.2pt,line join=round,fill opacity=0.00,] (295.97, 67.84) -- (300.91, 67.84);

\draw[color=drawColor,dash pattern=on 4pt off 4pt ,line cap=round,line join=round,fill opacity=0.00,] (298.44, 36.21) -- (298.44, 56.15);

\draw[color=drawColor,dash pattern=on 4pt off 4pt ,line cap=round,line join=round,fill opacity=0.00,] (298.44,102.57) -- (298.44, 77.91);

\draw[color=drawColor,line cap=round,line join=round,fill opacity=0.00,] (297.20, 36.21) -- (299.68, 36.21);

\draw[color=drawColor,line cap=round,line join=round,fill opacity=0.00,] (297.20,102.57) -- (299.68,102.57);

\draw[color=drawColor,line cap=round,line join=round,fill opacity=0.00,] (295.97, 56.15) --
	(300.91, 56.15) --
	(300.91, 77.91) --
	(295.97, 77.91) --
	(295.97, 56.15);

\draw[color=drawColor,line width= 1.2pt,line join=round,fill opacity=0.00,] (302.15, 89.76) -- (307.09, 89.76);

\draw[color=drawColor,dash pattern=on 4pt off 4pt ,line cap=round,line join=round,fill opacity=0.00,] (304.62, 58.83) -- (304.62, 74.26);

\draw[color=drawColor,dash pattern=on 4pt off 4pt ,line cap=round,line join=round,fill opacity=0.00,] (304.62,137.16) -- (304.62,113.49);

\draw[color=drawColor,line cap=round,line join=round,fill opacity=0.00,] (303.39, 58.83) -- (305.86, 58.83);

\draw[color=drawColor,line cap=round,line join=round,fill opacity=0.00,] (303.39,137.16) -- (305.86,137.16);

\draw[color=drawColor,line cap=round,line join=round,fill opacity=0.00,] (302.15, 74.26) --
	(307.09, 74.26) --
	(307.09,113.49) --
	(302.15,113.49) --
	(302.15, 74.26);

\draw[color=drawColor,line width= 1.2pt,line join=round,fill opacity=0.00,] (308.33, 80.47) -- (313.28, 80.47);

\draw[color=drawColor,dash pattern=on 4pt off 4pt ,line cap=round,line join=round,fill opacity=0.00,] (310.80, 55.72) -- (310.80, 71.06);

\draw[color=drawColor,dash pattern=on 4pt off 4pt ,line cap=round,line join=round,fill opacity=0.00,] (310.80,116.85) -- (310.80, 93.93);

\draw[color=drawColor,line cap=round,line join=round,fill opacity=0.00,] (309.57, 55.72) -- (312.04, 55.72);

\draw[color=drawColor,line cap=round,line join=round,fill opacity=0.00,] (309.57,116.85) -- (312.04,116.85);

\draw[color=drawColor,line cap=round,line join=round,fill opacity=0.00,] (308.33, 71.06) --
	(313.28, 71.06) --
	(313.28, 93.93) --
	(308.33, 93.93) --
	(308.33, 71.06);

\draw[color=drawColor,line width= 1.2pt,line join=round,fill opacity=0.00,] (314.51, 82.40) -- (319.46, 82.40);

\draw[color=drawColor,dash pattern=on 4pt off 4pt ,line cap=round,line join=round,fill opacity=0.00,] (316.99, 57.72) -- (316.99, 72.49);

\draw[color=drawColor,dash pattern=on 4pt off 4pt ,line cap=round,line join=round,fill opacity=0.00,] (316.99,121.30) -- (316.99, 92.86);

\draw[color=drawColor,line cap=round,line join=round,fill opacity=0.00,] (315.75, 57.72) -- (318.22, 57.72);

\draw[color=drawColor,line cap=round,line join=round,fill opacity=0.00,] (315.75,121.30) -- (318.22,121.30);

\draw[color=drawColor,line cap=round,line join=round,fill opacity=0.00,] (314.51, 72.49) --
	(319.46, 72.49) --
	(319.46, 92.86) --
	(314.51, 92.86) --
	(314.51, 72.49);

\draw[color=drawColor,line width= 1.2pt,line join=round,fill opacity=0.00,] (320.70, 84.43) -- (325.64, 84.43);

\draw[color=drawColor,dash pattern=on 4pt off 4pt ,line cap=round,line join=round,fill opacity=0.00,] (323.17, 68.05) -- (323.17, 74.96);

\draw[color=drawColor,dash pattern=on 4pt off 4pt ,line cap=round,line join=round,fill opacity=0.00,] (323.17,123.23) -- (323.17, 95.54);

\draw[color=drawColor,line cap=round,line join=round,fill opacity=0.00,] (321.93, 68.05) -- (324.41, 68.05);

\draw[color=drawColor,line cap=round,line join=round,fill opacity=0.00,] (321.93,123.23) -- (324.41,123.23);

\draw[color=drawColor,line cap=round,line join=round,fill opacity=0.00,] (320.70, 74.96) --
	(325.64, 74.96) --
	(325.64, 95.54) --
	(320.70, 95.54) --
	(320.70, 74.96);

\draw[color=drawColor,line width= 1.2pt,line join=round,fill opacity=0.00,] (326.88, 85.62) -- (331.82, 85.62);

\draw[color=drawColor,dash pattern=on 4pt off 4pt ,line cap=round,line join=round,fill opacity=0.00,] (329.35, 67.55) -- (329.35, 75.37);

\draw[color=drawColor,dash pattern=on 4pt off 4pt ,line cap=round,line join=round,fill opacity=0.00,] (329.35,133.03) -- (329.35,103.79);

\draw[color=drawColor,line cap=round,line join=round,fill opacity=0.00,] (328.11, 67.55) -- (330.59, 67.55);

\draw[color=drawColor,line cap=round,line join=round,fill opacity=0.00,] (328.11,133.03) -- (330.59,133.03);

\draw[color=drawColor,line cap=round,line join=round,fill opacity=0.00,] (326.88, 75.37) --
	(331.82, 75.37) --
	(331.82,103.79) --
	(326.88,103.79) --
	(326.88, 75.37);

\draw[color=drawColor,line width= 1.2pt,line join=round,fill opacity=0.00,] (333.06, 90.04) -- (338.01, 90.04);

\draw[color=drawColor,dash pattern=on 4pt off 4pt ,line cap=round,line join=round,fill opacity=0.00,] (335.53, 70.81) -- (335.53, 77.91);

\draw[color=drawColor,dash pattern=on 4pt off 4pt ,line cap=round,line join=round,fill opacity=0.00,] (335.53,123.18) -- (335.53,101.00);

\draw[color=drawColor,line cap=round,line join=round,fill opacity=0.00,] (334.30, 70.81) -- (336.77, 70.81);

\draw[color=drawColor,line cap=round,line join=round,fill opacity=0.00,] (334.30,123.18) -- (336.77,123.18);

\draw[color=drawColor,line cap=round,line join=round,fill opacity=0.00,] (333.06, 77.91) --
	(338.01, 77.91) --
	(338.01,101.00) --
	(333.06,101.00) --
	(333.06, 77.91);

\draw[color=drawColor,line width= 1.2pt,line join=round,fill opacity=0.00,] (339.24, 78.37) -- (344.19, 78.37);

\draw[color=drawColor,dash pattern=on 4pt off 4pt ,line cap=round,line join=round,fill opacity=0.00,] (341.72, 54.23) -- (341.72, 69.65);

\draw[color=drawColor,dash pattern=on 4pt off 4pt ,line cap=round,line join=round,fill opacity=0.00,] (341.72,119.95) -- (341.72, 88.90);

\draw[color=drawColor,line cap=round,line join=round,fill opacity=0.00,] (340.48, 54.23) -- (342.95, 54.23);

\draw[color=drawColor,line cap=round,line join=round,fill opacity=0.00,] (340.48,119.95) -- (342.95,119.95);

\draw[color=drawColor,line cap=round,line join=round,fill opacity=0.00,] (339.24, 69.65) --
	(344.19, 69.65) --
	(344.19, 88.90) --
	(339.24, 88.90) --
	(339.24, 69.65);

\draw[color=drawColor,line width= 1.2pt,line join=round,fill opacity=0.00,] (345.43, 92.87) -- (350.37, 92.87);

\draw[color=drawColor,dash pattern=on 4pt off 4pt ,line cap=round,line join=round,fill opacity=0.00,] (347.90, 71.51) -- (347.90, 78.13);

\draw[color=drawColor,dash pattern=on 4pt off 4pt ,line cap=round,line join=round,fill opacity=0.00,] (347.90,132.13) -- (347.90,106.72);

\draw[color=drawColor,line cap=round,line join=round,fill opacity=0.00,] (346.66, 71.51) -- (349.13, 71.51);

\draw[color=drawColor,line cap=round,line join=round,fill opacity=0.00,] (346.66,132.13) -- (349.13,132.13);

\draw[color=drawColor,line cap=round,line join=round,fill opacity=0.00,] (345.43, 78.13) --
	(350.37, 78.13) --
	(350.37,106.72) --
	(345.43,106.72) --
	(345.43, 78.13);

\draw[color=drawColor,line width= 1.2pt,line join=round,fill opacity=0.00,] (351.61,117.33) -- (356.55,117.33);

\draw[color=drawColor,dash pattern=on 4pt off 4pt ,line cap=round,line join=round,fill opacity=0.00,] (354.08, 74.51) -- (354.08, 94.83);

\draw[color=drawColor,dash pattern=on 4pt off 4pt ,line cap=round,line join=round,fill opacity=0.00,] (354.08,159.58) -- (354.08,137.90);

\draw[color=drawColor,line cap=round,line join=round,fill opacity=0.00,] (352.84, 74.51) -- (355.32, 74.51);

\draw[color=drawColor,line cap=round,line join=round,fill opacity=0.00,] (352.84,159.58) -- (355.32,159.58);

\draw[color=drawColor,line cap=round,line join=round,fill opacity=0.00,] (351.61, 94.83) --
	(356.55, 94.83) --
	(356.55,137.90) --
	(351.61,137.90) --
	(351.61, 94.83);

\draw[color=drawColor,line width= 1.2pt,line join=round,fill opacity=0.00,] (357.79,109.72) -- (362.74,109.72);

\draw[color=drawColor,dash pattern=on 4pt off 4pt ,line cap=round,line join=round,fill opacity=0.00,] (360.26, 74.22) -- (360.26, 92.00);

\draw[color=drawColor,dash pattern=on 4pt off 4pt ,line cap=round,line join=round,fill opacity=0.00,] (360.26,147.35) -- (360.26,129.32);

\draw[color=drawColor,line cap=round,line join=round,fill opacity=0.00,] (359.03, 74.22) -- (361.50, 74.22);

\draw[color=drawColor,line cap=round,line join=round,fill opacity=0.00,] (359.03,147.35) -- (361.50,147.35);

\draw[color=drawColor,line cap=round,line join=round,fill opacity=0.00,] (357.79, 92.00) --
	(362.74, 92.00) --
	(362.74,129.32) --
	(357.79,129.32) --
	(357.79, 92.00);

\draw[color=drawColor,line width= 1.2pt,line join=round,fill opacity=0.00,] (363.97, 72.26) -- (368.92, 72.26);

\draw[color=drawColor,dash pattern=on 4pt off 4pt ,line cap=round,line join=round,fill opacity=0.00,] (366.45, 45.50) -- (366.45, 62.45);

\draw[color=drawColor,dash pattern=on 4pt off 4pt ,line cap=round,line join=round,fill opacity=0.00,] (366.45,109.22) -- (366.45, 84.07);

\draw[color=drawColor,line cap=round,line join=round,fill opacity=0.00,] (365.21, 45.50) -- (367.68, 45.50);

\draw[color=drawColor,line cap=round,line join=round,fill opacity=0.00,] (365.21,109.22) -- (367.68,109.22);

\draw[color=drawColor,line cap=round,line join=round,fill opacity=0.00,] (363.97, 62.45) --
	(368.92, 62.45) --
	(368.92, 84.07) --
	(363.97, 84.07) --
	(363.97, 62.45);

\draw[color=drawColor,line width= 1.2pt,line join=round,fill opacity=0.00,] (370.16, 85.53) -- (375.10, 85.53);

\draw[color=drawColor,dash pattern=on 4pt off 4pt ,line cap=round,line join=round,fill opacity=0.00,] (372.63, 64.22) -- (372.63, 74.45);

\draw[color=drawColor,dash pattern=on 4pt off 4pt ,line cap=round,line join=round,fill opacity=0.00,] (372.63,131.64) -- (372.63,102.21);

\draw[color=drawColor,line cap=round,line join=round,fill opacity=0.00,] (371.39, 64.22) -- (373.86, 64.22);

\draw[color=drawColor,line cap=round,line join=round,fill opacity=0.00,] (371.39,131.64) -- (373.86,131.64);

\draw[color=drawColor,line cap=round,line join=round,fill opacity=0.00,] (370.16, 74.45) --
	(375.10, 74.45) --
	(375.10,102.21) --
	(370.16,102.21) --
	(370.16, 74.45);

\draw[color=drawColor,line width= 1.2pt,line join=round,fill opacity=0.00,] (376.34, 82.89) -- (381.28, 82.89);

\draw[color=drawColor,dash pattern=on 4pt off 4pt ,line cap=round,line join=round,fill opacity=0.00,] (378.81, 60.47) -- (378.81, 73.13);

\draw[color=drawColor,dash pattern=on 4pt off 4pt ,line cap=round,line join=round,fill opacity=0.00,] (378.81,120.71) -- (378.81, 92.01);

\draw[color=drawColor,line cap=round,line join=round,fill opacity=0.00,] (377.57, 60.47) -- (380.05, 60.47);

\draw[color=drawColor,line cap=round,line join=round,fill opacity=0.00,] (377.57,120.71) -- (380.05,120.71);

\draw[color=drawColor,line cap=round,line join=round,fill opacity=0.00,] (376.34, 73.13) --
	(381.28, 73.13) --
	(381.28, 92.01) --
	(376.34, 92.01) --
	(376.34, 73.13);

\draw[color=drawColor,line width= 1.2pt,line join=round,fill opacity=0.00,] (382.52, 80.49) -- (387.47, 80.49);

\draw[color=drawColor,dash pattern=on 4pt off 4pt ,line cap=round,line join=round,fill opacity=0.00,] (384.99, 57.58) -- (384.99, 71.42);

\draw[color=drawColor,dash pattern=on 4pt off 4pt ,line cap=round,line join=round,fill opacity=0.00,] (384.99,121.16) -- (384.99, 90.26);

\draw[color=drawColor,line cap=round,line join=round,fill opacity=0.00,] (383.76, 57.58) -- (386.23, 57.58);

\draw[color=drawColor,line cap=round,line join=round,fill opacity=0.00,] (383.76,121.16) -- (386.23,121.16);

\draw[color=drawColor,line cap=round,line join=round,fill opacity=0.00,] (382.52, 71.42) --
	(387.47, 71.42) --
	(387.47, 90.26) --
	(382.52, 90.26) --
	(382.52, 71.42);

\draw[color=drawColor,line width= 1.2pt,line join=round,fill opacity=0.00,] (388.70, 71.21) -- (393.65, 71.21);

\draw[color=drawColor,dash pattern=on 4pt off 4pt ,line cap=round,line join=round,fill opacity=0.00,] (391.18, 45.70) -- (391.18, 60.20);

\draw[color=drawColor,dash pattern=on 4pt off 4pt ,line cap=round,line join=round,fill opacity=0.00,] (391.18,114.04) -- (391.18, 81.88);

\draw[color=drawColor,line cap=round,line join=round,fill opacity=0.00,] (389.94, 45.70) -- (392.41, 45.70);

\draw[color=drawColor,line cap=round,line join=round,fill opacity=0.00,] (389.94,114.04) -- (392.41,114.04);

\draw[color=drawColor,line cap=round,line join=round,fill opacity=0.00,] (388.70, 60.20) --
	(393.65, 60.20) --
	(393.65, 81.88) --
	(388.70, 81.88) --
	(388.70, 60.20);

\draw[color=drawColor,line width= 1.2pt,line join=round,fill opacity=0.00,] (394.88,102.53) -- (399.83,102.53);

\draw[color=drawColor,dash pattern=on 4pt off 4pt ,line cap=round,line join=round,fill opacity=0.00,] (397.36, 72.58) -- (397.36, 85.88);

\draw[color=drawColor,dash pattern=on 4pt off 4pt ,line cap=round,line join=round,fill opacity=0.00,] (397.36,138.58) -- (397.36,125.22);

\draw[color=drawColor,line cap=round,line join=round,fill opacity=0.00,] (396.12, 72.58) -- (398.59, 72.58);

\draw[color=drawColor,line cap=round,line join=round,fill opacity=0.00,] (396.12,138.58) -- (398.59,138.58);

\draw[color=drawColor,line cap=round,line join=round,fill opacity=0.00,] (394.88, 85.88) --
	(399.83, 85.88) --
	(399.83,125.22) --
	(394.88,125.22) --
	(394.88, 85.88);

\draw[color=drawColor,line width= 1.2pt,line join=round,fill opacity=0.00,] (401.07, 83.62) -- (406.01, 83.62);

\draw[color=drawColor,dash pattern=on 4pt off 4pt ,line cap=round,line join=round,fill opacity=0.00,] (403.54, 53.68) -- (403.54, 72.03);

\draw[color=drawColor,dash pattern=on 4pt off 4pt ,line cap=round,line join=round,fill opacity=0.00,] (403.54,126.72) -- (403.54,100.93);

\draw[color=drawColor,line cap=round,line join=round,fill opacity=0.00,] (402.30, 53.68) -- (404.78, 53.68);

\draw[color=drawColor,line cap=round,line join=round,fill opacity=0.00,] (402.30,126.72) -- (404.78,126.72);

\draw[color=drawColor,line cap=round,line join=round,fill opacity=0.00,] (401.07, 72.03) --
	(406.01, 72.03) --
	(406.01,100.93) --
	(401.07,100.93) --
	(401.07, 72.03);

\draw[color=drawColor,line width= 1.2pt,line join=round,fill opacity=0.00,] (407.25, 88.13) -- (412.20, 88.13);

\draw[color=drawColor,dash pattern=on 4pt off 4pt ,line cap=round,line join=round,fill opacity=0.00,] (409.72, 70.55) -- (409.72, 76.25);

\draw[color=drawColor,dash pattern=on 4pt off 4pt ,line cap=round,line join=round,fill opacity=0.00,] (409.72,123.68) -- (409.72, 99.03);

\draw[color=drawColor,line cap=round,line join=round,fill opacity=0.00,] (408.49, 70.55) -- (410.96, 70.55);

\draw[color=drawColor,line cap=round,line join=round,fill opacity=0.00,] (408.49,123.68) -- (410.96,123.68);

\draw[color=drawColor,line cap=round,line join=round,fill opacity=0.00,] (407.25, 76.25) --
	(412.20, 76.25) --
	(412.20, 99.03) --
	(407.25, 99.03) --
	(407.25, 76.25);

\draw[color=drawColor,line width= 1.2pt,line join=round,fill opacity=0.00,] (413.43, 76.60) -- (418.38, 76.60);

\draw[color=drawColor,dash pattern=on 4pt off 4pt ,line cap=round,line join=round,fill opacity=0.00,] (415.90, 39.10) -- (415.90, 66.13);

\draw[color=drawColor,dash pattern=on 4pt off 4pt ,line cap=round,line join=round,fill opacity=0.00,] (415.90,126.96) -- (415.90, 93.56);

\draw[color=drawColor,line cap=round,line join=round,fill opacity=0.00,] (414.67, 39.10) -- (417.14, 39.10);

\draw[color=drawColor,line cap=round,line join=round,fill opacity=0.00,] (414.67,126.96) -- (417.14,126.96);

\draw[color=drawColor,line cap=round,line join=round,fill opacity=0.00,] (413.43, 66.13) --
	(418.38, 66.13) --
	(418.38, 93.56) --
	(413.43, 93.56) --
	(413.43, 66.13);

\draw[color=drawColor,line width= 1.2pt,line join=round,fill opacity=0.00,] (419.61, 73.96) -- (424.56, 73.96);

\draw[color=drawColor,dash pattern=on 4pt off 4pt ,line cap=round,line join=round,fill opacity=0.00,] (422.09, 49.22) -- (422.09, 64.54);

\draw[color=drawColor,dash pattern=on 4pt off 4pt ,line cap=round,line join=round,fill opacity=0.00,] (422.09,116.16) -- (422.09, 85.88);

\draw[color=drawColor,line cap=round,line join=round,fill opacity=0.00,] (420.85, 49.22) -- (423.32, 49.22);

\draw[color=drawColor,line cap=round,line join=round,fill opacity=0.00,] (420.85,116.16) -- (423.32,116.16);

\draw[color=drawColor,line cap=round,line join=round,fill opacity=0.00,] (419.61, 64.54) --
	(424.56, 64.54) --
	(424.56, 85.88) --
	(419.61, 85.88) --
	(419.61, 64.54);

\draw[color=drawColor,line width= 1.2pt,line join=round,fill opacity=0.00,] (425.80, 78.12) -- (430.74, 78.12);

\draw[color=drawColor,dash pattern=on 4pt off 4pt ,line cap=round,line join=round,fill opacity=0.00,] (428.27, 50.03) -- (428.27, 66.79);

\draw[color=drawColor,dash pattern=on 4pt off 4pt ,line cap=round,line join=round,fill opacity=0.00,] (428.27,128.32) -- (428.27, 95.27);

\draw[color=drawColor,line cap=round,line join=round,fill opacity=0.00,] (427.03, 50.03) -- (429.51, 50.03);

\draw[color=drawColor,line cap=round,line join=round,fill opacity=0.00,] (427.03,128.32) -- (429.51,128.32);

\draw[color=drawColor,line cap=round,line join=round,fill opacity=0.00,] (425.80, 66.79) --
	(430.74, 66.79) --
	(430.74, 95.27) --
	(425.80, 95.27) --
	(425.80, 66.79);

\draw[color=drawColor,line width= 1.2pt,line join=round,fill opacity=0.00,] (431.98, 70.67) -- (436.93, 70.67);

\draw[color=drawColor,dash pattern=on 4pt off 4pt ,line cap=round,line join=round,fill opacity=0.00,] (434.45, 22.69) -- (434.45, 49.28);

\draw[color=drawColor,dash pattern=on 4pt off 4pt ,line cap=round,line join=round,fill opacity=0.00,] (434.45,117.99) -- (434.45, 85.45);

\draw[color=drawColor,line cap=round,line join=round,fill opacity=0.00,] (433.22, 22.69) -- (435.69, 22.69);

\draw[color=drawColor,line cap=round,line join=round,fill opacity=0.00,] (433.22,117.99) -- (435.69,117.99);

\draw[color=drawColor,line cap=round,line join=round,fill opacity=0.00,] (431.98, 49.28) --
	(436.93, 49.28) --
	(436.93, 85.45) --
	(431.98, 85.45) --
	(431.98, 49.28);
\end{scope}
\begin{scope}
\path[clip] (  0.00,  0.00) rectangle (469.75,614.29);
\definecolor[named]{drawColor}{rgb}{0.00,0.00,0.00}

\draw[color=drawColor,line cap=round,line join=round,fill opacity=0.00,] ( 51.14, 16.63) -- (434.45, 16.63);

\draw[color=drawColor,line cap=round,line join=round,fill opacity=0.00,] ( 51.14, 16.63) -- ( 51.14, 12.67);

\draw[color=drawColor,line cap=round,line join=round,fill opacity=0.00,] ( 57.33, 16.63) -- ( 57.33, 12.67);

\draw[color=drawColor,line cap=round,line join=round,fill opacity=0.00,] ( 63.51, 16.63) -- ( 63.51, 12.67);

\draw[color=drawColor,line cap=round,line join=round,fill opacity=0.00,] ( 69.69, 16.63) -- ( 69.69, 12.67);

\draw[color=drawColor,line cap=round,line join=round,fill opacity=0.00,] ( 75.87, 16.63) -- ( 75.87, 12.67);

\draw[color=drawColor,line cap=round,line join=round,fill opacity=0.00,] ( 82.05, 16.63) -- ( 82.05, 12.67);

\draw[color=drawColor,line cap=round,line join=round,fill opacity=0.00,] ( 88.24, 16.63) -- ( 88.24, 12.67);

\draw[color=drawColor,line cap=round,line join=round,fill opacity=0.00,] ( 94.42, 16.63) -- ( 94.42, 12.67);

\draw[color=drawColor,line cap=round,line join=round,fill opacity=0.00,] (100.60, 16.63) -- (100.60, 12.67);

\draw[color=drawColor,line cap=round,line join=round,fill opacity=0.00,] (106.78, 16.63) -- (106.78, 12.67);

\draw[color=drawColor,line cap=round,line join=round,fill opacity=0.00,] (112.97, 16.63) -- (112.97, 12.67);

\draw[color=drawColor,line cap=round,line join=round,fill opacity=0.00,] (119.15, 16.63) -- (119.15, 12.67);

\draw[color=drawColor,line cap=round,line join=round,fill opacity=0.00,] (125.33, 16.63) -- (125.33, 12.67);

\draw[color=drawColor,line cap=round,line join=round,fill opacity=0.00,] (131.51, 16.63) -- (131.51, 12.67);

\draw[color=drawColor,line cap=round,line join=round,fill opacity=0.00,] (137.70, 16.63) -- (137.70, 12.67);

\draw[color=drawColor,line cap=round,line join=round,fill opacity=0.00,] (143.88, 16.63) -- (143.88, 12.67);

\draw[color=drawColor,line cap=round,line join=round,fill opacity=0.00,] (150.06, 16.63) -- (150.06, 12.67);

\draw[color=drawColor,line cap=round,line join=round,fill opacity=0.00,] (156.24, 16.63) -- (156.24, 12.67);

\draw[color=drawColor,line cap=round,line join=round,fill opacity=0.00,] (162.43, 16.63) -- (162.43, 12.67);

\draw[color=drawColor,line cap=round,line join=round,fill opacity=0.00,] (168.61, 16.63) -- (168.61, 12.67);

\draw[color=drawColor,line cap=round,line join=round,fill opacity=0.00,] (174.79, 16.63) -- (174.79, 12.67);

\draw[color=drawColor,line cap=round,line join=round,fill opacity=0.00,] (180.97, 16.63) -- (180.97, 12.67);

\draw[color=drawColor,line cap=round,line join=round,fill opacity=0.00,] (187.16, 16.63) -- (187.16, 12.67);

\draw[color=drawColor,line cap=round,line join=round,fill opacity=0.00,] (193.34, 16.63) -- (193.34, 12.67);

\draw[color=drawColor,line cap=round,line join=round,fill opacity=0.00,] (199.52, 16.63) -- (199.52, 12.67);

\draw[color=drawColor,line cap=round,line join=round,fill opacity=0.00,] (205.70, 16.63) -- (205.70, 12.67);

\draw[color=drawColor,line cap=round,line join=round,fill opacity=0.00,] (211.89, 16.63) -- (211.89, 12.67);

\draw[color=drawColor,line cap=round,line join=round,fill opacity=0.00,] (218.07, 16.63) -- (218.07, 12.67);

\draw[color=drawColor,line cap=round,line join=round,fill opacity=0.00,] (224.25, 16.63) -- (224.25, 12.67);

\draw[color=drawColor,line cap=round,line join=round,fill opacity=0.00,] (230.43, 16.63) -- (230.43, 12.67);

\draw[color=drawColor,line cap=round,line join=round,fill opacity=0.00,] (236.62, 16.63) -- (236.62, 12.67);

\draw[color=drawColor,line cap=round,line join=round,fill opacity=0.00,] (242.80, 16.63) -- (242.80, 12.67);

\draw[color=drawColor,line cap=round,line join=round,fill opacity=0.00,] (248.98, 16.63) -- (248.98, 12.67);

\draw[color=drawColor,line cap=round,line join=round,fill opacity=0.00,] (255.16, 16.63) -- (255.16, 12.67);

\draw[color=drawColor,line cap=round,line join=round,fill opacity=0.00,] (261.34, 16.63) -- (261.34, 12.67);

\draw[color=drawColor,line cap=round,line join=round,fill opacity=0.00,] (267.53, 16.63) -- (267.53, 12.67);

\draw[color=drawColor,line cap=round,line join=round,fill opacity=0.00,] (273.71, 16.63) -- (273.71, 12.67);

\draw[color=drawColor,line cap=round,line join=round,fill opacity=0.00,] (279.89, 16.63) -- (279.89, 12.67);

\draw[color=drawColor,line cap=round,line join=round,fill opacity=0.00,] (286.07, 16.63) -- (286.07, 12.67);

\draw[color=drawColor,line cap=round,line join=round,fill opacity=0.00,] (292.26, 16.63) -- (292.26, 12.67);

\draw[color=drawColor,line cap=round,line join=round,fill opacity=0.00,] (298.44, 16.63) -- (298.44, 12.67);

\draw[color=drawColor,line cap=round,line join=round,fill opacity=0.00,] (304.62, 16.63) -- (304.62, 12.67);

\draw[color=drawColor,line cap=round,line join=round,fill opacity=0.00,] (310.80, 16.63) -- (310.80, 12.67);

\draw[color=drawColor,line cap=round,line join=round,fill opacity=0.00,] (316.99, 16.63) -- (316.99, 12.67);

\draw[color=drawColor,line cap=round,line join=round,fill opacity=0.00,] (323.17, 16.63) -- (323.17, 12.67);

\draw[color=drawColor,line cap=round,line join=round,fill opacity=0.00,] (329.35, 16.63) -- (329.35, 12.67);

\draw[color=drawColor,line cap=round,line join=round,fill opacity=0.00,] (335.53, 16.63) -- (335.53, 12.67);

\draw[color=drawColor,line cap=round,line join=round,fill opacity=0.00,] (341.72, 16.63) -- (341.72, 12.67);

\draw[color=drawColor,line cap=round,line join=round,fill opacity=0.00,] (347.90, 16.63) -- (347.90, 12.67);

\draw[color=drawColor,line cap=round,line join=round,fill opacity=0.00,] (354.08, 16.63) -- (354.08, 12.67);

\draw[color=drawColor,line cap=round,line join=round,fill opacity=0.00,] (360.26, 16.63) -- (360.26, 12.67);

\draw[color=drawColor,line cap=round,line join=round,fill opacity=0.00,] (366.45, 16.63) -- (366.45, 12.67);

\draw[color=drawColor,line cap=round,line join=round,fill opacity=0.00,] (372.63, 16.63) -- (372.63, 12.67);

\draw[color=drawColor,line cap=round,line join=round,fill opacity=0.00,] (378.81, 16.63) -- (378.81, 12.67);

\draw[color=drawColor,line cap=round,line join=round,fill opacity=0.00,] (384.99, 16.63) -- (384.99, 12.67);

\draw[color=drawColor,line cap=round,line join=round,fill opacity=0.00,] (391.18, 16.63) -- (391.18, 12.67);

\draw[color=drawColor,line cap=round,line join=round,fill opacity=0.00,] (397.36, 16.63) -- (397.36, 12.67);

\draw[color=drawColor,line cap=round,line join=round,fill opacity=0.00,] (403.54, 16.63) -- (403.54, 12.67);

\draw[color=drawColor,line cap=round,line join=round,fill opacity=0.00,] (409.72, 16.63) -- (409.72, 12.67);

\draw[color=drawColor,line cap=round,line join=round,fill opacity=0.00,] (415.90, 16.63) -- (415.90, 12.67);

\draw[color=drawColor,line cap=round,line join=round,fill opacity=0.00,] (422.09, 16.63) -- (422.09, 12.67);

\draw[color=drawColor,line cap=round,line join=round,fill opacity=0.00,] (428.27, 16.63) -- (428.27, 12.67);

\draw[color=drawColor,line cap=round,line join=round,fill opacity=0.00,] (434.45, 16.63) -- (434.45, 12.67);

\node[color=drawColor,anchor=base,inner sep=0pt, outer sep=0pt, scale=  0.66] at ( 51.14,  0.79) {1949%
};

\node[color=drawColor,anchor=base,inner sep=0pt, outer sep=0pt, scale=  0.66] at ( 75.87,  0.79) {1953%
};

\node[color=drawColor,anchor=base,inner sep=0pt, outer sep=0pt, scale=  0.66] at (100.60,  0.79) {1957%
};

\node[color=drawColor,anchor=base,inner sep=0pt, outer sep=0pt, scale=  0.66] at (125.33,  0.79) {1961%
};

\node[color=drawColor,anchor=base,inner sep=0pt, outer sep=0pt, scale=  0.66] at (150.06,  0.79) {1965%
};

\node[color=drawColor,anchor=base,inner sep=0pt, outer sep=0pt, scale=  0.66] at (174.79,  0.79) {1969%
};

\node[color=drawColor,anchor=base,inner sep=0pt, outer sep=0pt, scale=  0.66] at (199.52,  0.79) {1973%
};

\node[color=drawColor,anchor=base,inner sep=0pt, outer sep=0pt, scale=  0.66] at (224.25,  0.79) {1977%
};

\node[color=drawColor,anchor=base,inner sep=0pt, outer sep=0pt, scale=  0.66] at (248.98,  0.79) {1981%
};

\node[color=drawColor,anchor=base,inner sep=0pt, outer sep=0pt, scale=  0.66] at (273.71,  0.79) {1985%
};

\node[color=drawColor,anchor=base,inner sep=0pt, outer sep=0pt, scale=  0.66] at (298.44,  0.79) {1989%
};

\node[color=drawColor,anchor=base,inner sep=0pt, outer sep=0pt, scale=  0.66] at (323.17,  0.79) {1993%
};

\node[color=drawColor,anchor=base,inner sep=0pt, outer sep=0pt, scale=  0.66] at (347.90,  0.79) {1997%
};

\node[color=drawColor,anchor=base,inner sep=0pt, outer sep=0pt, scale=  0.66] at (372.63,  0.79) {2001%
};

\node[color=drawColor,anchor=base,inner sep=0pt, outer sep=0pt, scale=  0.66] at (397.36,  0.79) {2005%
};

\node[color=drawColor,anchor=base,inner sep=0pt, outer sep=0pt, scale=  0.66] at (422.09,  0.79) {2009%
};

\draw[color=drawColor,line cap=round,line join=round,fill opacity=0.00,] ( 32.47, 39.14) -- ( 32.47,168.63);

\draw[color=drawColor,line cap=round,line join=round,fill opacity=0.00,] ( 32.47, 39.14) -- ( 28.51, 39.14);

\draw[color=drawColor,line cap=round,line join=round,fill opacity=0.00,] ( 32.47, 71.51) -- ( 28.51, 71.51);

\draw[color=drawColor,line cap=round,line join=round,fill opacity=0.00,] ( 32.47,103.89) -- ( 28.51,103.89);

\draw[color=drawColor,line cap=round,line join=round,fill opacity=0.00,] ( 32.47,136.26) -- ( 28.51,136.26);

\draw[color=drawColor,line cap=round,line join=round,fill opacity=0.00,] ( 32.47,168.63) -- ( 28.51,168.63);

\node[rotate= 90.00,color=drawColor,anchor=base,inner sep=0pt, outer sep=0pt, scale=  0.66] at ( 24.55, 39.14) {200%
};

\node[rotate= 90.00,color=drawColor,anchor=base,inner sep=0pt, outer sep=0pt, scale=  0.66] at ( 24.55, 71.51) {400%
};

\node[rotate= 90.00,color=drawColor,anchor=base,inner sep=0pt, outer sep=0pt, scale=  0.66] at ( 24.55,103.89) {600%
};

\node[rotate= 90.00,color=drawColor,anchor=base,inner sep=0pt, outer sep=0pt, scale=  0.66] at ( 24.55,136.26) {800%
};

\node[rotate= 90.00,color=drawColor,anchor=base,inner sep=0pt, outer sep=0pt, scale=  0.66] at ( 24.55,168.63) {1000%
};
\end{scope}
\begin{scope}
\path[clip] (  0.00,  0.00) rectangle (469.75,204.77);
\definecolor[named]{drawColor}{rgb}{0.00,0.00,0.00}

\node[rotate= 90.00,color=drawColor,anchor=base,inner sep=0pt, outer sep=0pt, scale=  0.66] at (  8.71, 98.42) {FLow [KAF]%
};
\end{scope}
\begin{scope}
\path[clip] (  0.00,  0.00) rectangle (469.75,614.29);
\definecolor[named]{drawColor}{rgb}{0.00,0.00,0.00}

\draw[color=drawColor,line cap=round,line join=round,fill opacity=0.00,] ( 32.47, 16.63) --
	(453.12, 16.63) --
	(453.12,180.21) --
	( 32.47,180.21) --
	( 32.47, 16.63);
\end{scope}
\begin{scope}
\path[clip] ( 32.47, 16.63) rectangle (453.12,180.21);
\definecolor[named]{drawColor}{rgb}{1.00,0.00,0.00}

\draw[color=drawColor,line cap=round,line join=round,fill opacity=0.00,] ( 51.65,127.35) -- ( 56.82, 86.97);

\draw[color=drawColor,line cap=round,line join=round,fill opacity=0.00,] ( 63.83, 88.22) -- ( 69.37,156.38);

\draw[color=drawColor,line cap=round,line join=round,fill opacity=0.00,] ( 70.13,156.39) -- ( 75.43,109.05);

\draw[color=drawColor,line cap=round,line join=round,fill opacity=0.00,] ( 76.22,101.17) -- ( 81.71, 38.69);

\draw[color=drawColor,line cap=round,line join=round,fill opacity=0.00,] ( 82.85, 38.63) -- ( 87.44, 61.05);

\draw[color=drawColor,line cap=round,line join=round,fill opacity=0.00,] ( 90.30, 68.31) -- ( 92.35, 71.65);

\draw[color=drawColor,line cap=round,line join=round,fill opacity=0.00,] ( 94.58, 78.99) -- (100.44,220.57);

\draw[color=drawColor,line cap=round,line join=round,fill opacity=0.00,] (100.84,220.57) -- (106.55,126.15);

\draw[color=drawColor,line cap=round,line join=round,fill opacity=0.00,] (107.30,118.27) -- (112.46, 78.63);

\draw[color=drawColor,line cap=round,line join=round,fill opacity=0.00,] (114.94, 78.14) -- (117.18, 82.05);

\draw[color=drawColor,line cap=round,line join=round,fill opacity=0.00,] (120.44, 81.75) -- (124.04, 71.36);

\draw[color=drawColor,line cap=round,line join=round,fill opacity=0.00,] (125.97, 71.52) -- (130.87,101.47);

\draw[color=drawColor,line cap=round,line join=round,fill opacity=0.00,] (131.88,101.44) -- (137.33, 41.94);

\draw[color=drawColor,line cap=round,line join=round,fill opacity=0.00,] (138.31, 41.91) -- (143.27, 73.72);

\draw[color=drawColor,line cap=round,line join=round,fill opacity=0.00,] (144.26, 81.57) -- (149.68,137.71);

\draw[color=drawColor,line cap=round,line join=round,fill opacity=0.00,] (150.34,137.70) -- (155.96, 58.22);

\draw[color=drawColor,line cap=round,line join=round,fill opacity=0.00,] (157.67, 57.97) -- (161.00, 66.58);

\draw[color=drawColor,line cap=round,line join=round,fill opacity=0.00,] (162.89, 74.21) -- (168.14,118.30);

\draw[color=drawColor,line cap=round,line join=round,fill opacity=0.00,] (169.11,118.31) -- (174.28, 78.19);

\draw[color=drawColor,line cap=round,line join=round,fill opacity=0.00,] (175.49, 78.17) -- (180.27,104.77);

\draw[color=drawColor,line cap=round,line join=round,fill opacity=0.00,] (183.31,105.48) -- (184.81,103.43);

\draw[color=drawColor,line cap=round,line join=round,fill opacity=0.00,] (187.99, 96.36) -- (192.51, 75.34);

\draw[color=drawColor,line cap=round,line join=round,fill opacity=0.00,] (193.70, 75.41) -- (199.15,134.18);

\draw[color=drawColor,line cap=round,line join=round,fill opacity=0.00,] (199.88,134.18) -- (205.34, 74.71);

\draw[color=drawColor,line cap=round,line join=round,fill opacity=0.00,] (206.12, 74.70) -- (211.46,124.66);

\draw[color=drawColor,line cap=round,line join=round,fill opacity=0.00,] (212.28,124.66) -- (217.68, 70.02);

\draw[color=drawColor,line cap=round,line join=round,fill opacity=0.00,] (218.67, 62.17) -- (223.65, 29.51);

\draw[color=drawColor,line cap=round,line join=round,fill opacity=0.00,] (224.47, 29.55) -- (230.21,132.60);

\draw[color=drawColor,line cap=round,line join=round,fill opacity=0.00,] (232.71,139.79) -- (234.34,142.11);

\draw[color=drawColor,line cap=round,line join=round,fill opacity=0.00,] (243.05,143.29) -- (248.73, 53.29);

\draw[color=drawColor,line cap=round,line join=round,fill opacity=0.00,] (249.35, 53.28) -- (254.79,110.47);

\draw[color=drawColor,line cap=round,line join=round,fill opacity=0.00,] (255.44,118.36) -- (261.07,198.71);

\draw[color=drawColor,line cap=round,line join=round,fill opacity=0.00,] (263.29,199.22) -- (265.58,195.17);

\draw[color=drawColor,line cap=round,line join=round,fill opacity=0.00,] (267.99,187.79) -- (273.24,143.48);

\draw[color=drawColor,line cap=round,line join=round,fill opacity=0.00,] (280.43,134.10) -- (285.53, 97.23);

\draw[color=drawColor,line cap=round,line join=round,fill opacity=0.00,] (286.88, 89.43) -- (291.45, 67.47);

\draw[color=drawColor,line cap=round,line join=round,fill opacity=0.00,] (294.17, 60.13) -- (296.53, 55.84);

\draw[color=drawColor,line cap=round,line join=round,fill opacity=0.00,] (300.08, 55.98) -- (302.98, 62.34);

\draw[color=drawColor,line cap=round,line join=round,fill opacity=0.00,] (305.74, 69.75) -- (309.69, 83.24);

\draw[color=drawColor,line cap=round,line join=round,fill opacity=0.00,] (312.07, 83.29) -- (315.72, 72.48);

\draw[color=drawColor,line cap=round,line join=round,fill opacity=0.00,] (317.30, 72.67) -- (322.86,143.45);

\draw[color=drawColor,line cap=round,line join=round,fill opacity=0.00,] (323.49,143.45) -- (329.03, 76.40);

\draw[color=drawColor,line cap=round,line join=round,fill opacity=0.00,] (329.53, 76.41) -- (335.36,206.23);

\draw[color=drawColor,line cap=round,line join=round,fill opacity=0.00,] (335.72,206.24) -- (341.53, 84.66);

\draw[color=drawColor,line cap=round,line join=round,fill opacity=0.00,] (342.07, 84.65) -- (347.55,145.98);

\draw[color=drawColor,line cap=round,line join=round,fill opacity=0.00,] (348.22,145.97) -- (353.76, 78.34);

\draw[color=drawColor,line cap=round,line join=round,fill opacity=0.00,] (355.34, 78.15) -- (359.00, 89.01);

\draw[color=drawColor,line cap=round,line join=round,fill opacity=0.00,] (360.83, 88.84) -- (365.88, 53.65);

\draw[color=drawColor,line cap=round,line join=round,fill opacity=0.00,] (373.46, 50.18) -- (377.98, 29.22);

\draw[color=drawColor,line cap=round,line join=round,fill opacity=0.00,] (379.48, 29.25) -- (384.32, 57.30);

\draw[color=drawColor,line cap=round,line join=round,fill opacity=0.00,] (386.73, 57.65) -- (389.44, 52.10);

\draw[color=drawColor,line cap=round,line join=round,fill opacity=0.00,] (391.70, 52.47) -- (396.84, 91.07);

\draw[color=drawColor,line cap=round,line join=round,fill opacity=0.00,] (397.98, 91.09) -- (402.92, 60.04);

\draw[color=drawColor,line cap=round,line join=round,fill opacity=0.00,] ( 51.14,131.28) circle (  0.89);

\draw[color=drawColor,line cap=round,line join=round,fill opacity=0.00,] ( 57.33, 83.05) circle (  0.89);

\draw[color=drawColor,line cap=round,line join=round,fill opacity=0.00,] ( 63.51, 84.28) circle (  0.89);

\draw[color=drawColor,line cap=round,line join=round,fill opacity=0.00,] ( 69.69,160.32) circle (  0.89);

\draw[color=drawColor,line cap=round,line join=round,fill opacity=0.00,] ( 75.87,105.12) circle (  0.89);

\draw[color=drawColor,line cap=round,line join=round,fill opacity=0.00,] ( 82.05, 34.75) circle (  0.89);

\draw[color=drawColor,line cap=round,line join=round,fill opacity=0.00,] ( 88.24, 64.93) circle (  0.89);

\draw[color=drawColor,line cap=round,line join=round,fill opacity=0.00,] ( 94.42, 75.03) circle (  0.89);

\draw[color=drawColor,line cap=round,line join=round,fill opacity=0.00,] (100.60,224.52) circle (  0.89);

\draw[color=drawColor,line cap=round,line join=round,fill opacity=0.00,] (106.78,122.19) circle (  0.89);

\draw[color=drawColor,line cap=round,line join=round,fill opacity=0.00,] (112.97, 74.71) circle (  0.89);

\draw[color=drawColor,line cap=round,line join=round,fill opacity=0.00,] (119.15, 85.49) circle (  0.89);

\draw[color=drawColor,line cap=round,line join=round,fill opacity=0.00,] (125.33, 67.61) circle (  0.89);

\draw[color=drawColor,line cap=round,line join=round,fill opacity=0.00,] (131.51,105.38) circle (  0.89);

\draw[color=drawColor,line cap=round,line join=round,fill opacity=0.00,] (137.70, 38.00) circle (  0.89);

\draw[color=drawColor,line cap=round,line join=round,fill opacity=0.00,] (143.88, 77.63) circle (  0.89);

\draw[color=drawColor,line cap=round,line join=round,fill opacity=0.00,] (150.06,141.65) circle (  0.89);

\draw[color=drawColor,line cap=round,line join=round,fill opacity=0.00,] (156.24, 54.27) circle (  0.89);

\draw[color=drawColor,line cap=round,line join=round,fill opacity=0.00,] (162.43, 70.27) circle (  0.89);

\draw[color=drawColor,line cap=round,line join=round,fill opacity=0.00,] (168.61,122.23) circle (  0.89);

\draw[color=drawColor,line cap=round,line join=round,fill opacity=0.00,] (174.79, 74.27) circle (  0.89);

\draw[color=drawColor,line cap=round,line join=round,fill opacity=0.00,] (180.97,108.67) circle (  0.89);

\draw[color=drawColor,line cap=round,line join=round,fill opacity=0.00,] (187.16,100.24) circle (  0.89);

\draw[color=drawColor,line cap=round,line join=round,fill opacity=0.00,] (193.34, 71.47) circle (  0.89);

\draw[color=drawColor,line cap=round,line join=round,fill opacity=0.00,] (199.52,138.12) circle (  0.89);

\draw[color=drawColor,line cap=round,line join=round,fill opacity=0.00,] (205.70, 70.76) circle (  0.89);

\draw[color=drawColor,line cap=round,line join=round,fill opacity=0.00,] (211.89,128.60) circle (  0.89);

\draw[color=drawColor,line cap=round,line join=round,fill opacity=0.00,] (218.07, 66.08) circle (  0.89);

\draw[color=drawColor,line cap=round,line join=round,fill opacity=0.00,] (224.25, 25.60) circle (  0.89);

\draw[color=drawColor,line cap=round,line join=round,fill opacity=0.00,] (230.43,136.55) circle (  0.89);

\draw[color=drawColor,line cap=round,line join=round,fill opacity=0.00,] (236.62,145.35) circle (  0.89);

\draw[color=drawColor,line cap=round,line join=round,fill opacity=0.00,] (242.80,147.24) circle (  0.89);

\draw[color=drawColor,line cap=round,line join=round,fill opacity=0.00,] (248.98, 49.34) circle (  0.89);

\draw[color=drawColor,line cap=round,line join=round,fill opacity=0.00,] (255.16,114.41) circle (  0.89);

\draw[color=drawColor,line cap=round,line join=round,fill opacity=0.00,] (261.34,202.66) circle (  0.89);

\draw[color=drawColor,line cap=round,line join=round,fill opacity=0.00,] (267.53,191.72) circle (  0.89);

\draw[color=drawColor,line cap=round,line join=round,fill opacity=0.00,] (273.71,139.55) circle (  0.89);

\draw[color=drawColor,line cap=round,line join=round,fill opacity=0.00,] (279.89,138.02) circle (  0.89);

\draw[color=drawColor,line cap=round,line join=round,fill opacity=0.00,] (286.07, 93.31) circle (  0.89);

\draw[color=drawColor,line cap=round,line join=round,fill opacity=0.00,] (292.26, 63.59) circle (  0.89);

\draw[color=drawColor,line cap=round,line join=round,fill opacity=0.00,] (298.44, 52.38) circle (  0.89);

\draw[color=drawColor,line cap=round,line join=round,fill opacity=0.00,] (304.62, 65.95) circle (  0.89);

\draw[color=drawColor,line cap=round,line join=round,fill opacity=0.00,] (310.80, 87.04) circle (  0.89);

\draw[color=drawColor,line cap=round,line join=round,fill opacity=0.00,] (316.99, 68.73) circle (  0.89);

\draw[color=drawColor,line cap=round,line join=round,fill opacity=0.00,] (323.17,147.39) circle (  0.89);

\draw[color=drawColor,line cap=round,line join=round,fill opacity=0.00,] (329.35, 72.46) circle (  0.89);

\draw[color=drawColor,line cap=round,line join=round,fill opacity=0.00,] (335.53,210.19) circle (  0.89);

\draw[color=drawColor,line cap=round,line join=round,fill opacity=0.00,] (341.72, 80.70) circle (  0.89);

\draw[color=drawColor,line cap=round,line join=round,fill opacity=0.00,] (347.90,149.92) circle (  0.89);

\draw[color=drawColor,line cap=round,line join=round,fill opacity=0.00,] (354.08, 74.40) circle (  0.89);

\draw[color=drawColor,line cap=round,line join=round,fill opacity=0.00,] (360.26, 92.76) circle (  0.89);

\draw[color=drawColor,line cap=round,line join=round,fill opacity=0.00,] (366.45, 49.73) circle (  0.89);

\draw[color=drawColor,line cap=round,line join=round,fill opacity=0.00,] (372.63, 54.05) circle (  0.89);

\draw[color=drawColor,line cap=round,line join=round,fill opacity=0.00,] (378.81, 25.35) circle (  0.89);

\draw[color=drawColor,line cap=round,line join=round,fill opacity=0.00,] (384.99, 61.21) circle (  0.89);

\draw[color=drawColor,line cap=round,line join=round,fill opacity=0.00,] (391.18, 48.54) circle (  0.89);

\draw[color=drawColor,line cap=round,line join=round,fill opacity=0.00,] (397.36, 95.00) circle (  0.89);

\draw[color=drawColor,line cap=round,line join=round,fill opacity=0.00,] (403.54, 56.13) circle (  0.89);

\draw[color=drawColor,line cap=round,line join=round,fill opacity=0.00,] (409.72, 60.82) circle (  0.89);
\end{scope}
\begin{scope}
\path[clip] (  0.00,  0.00) rectangle (469.75,614.29);
\definecolor[named]{drawColor}{rgb}{0.00,0.00,0.00}

\node[color=drawColor,anchor=base,inner sep=0pt, outer sep=0pt, scale=  1.00] at (242.80,188.13) {(c) RPSS = 0.11 MC = 0.39%
};
\end{scope}
\end{tikzpicture}

   \caption{Same as Figure 3 but for June flows.}
   \label{fig:box}
\end{figure*}

\begin{figure*}[htbp] %  figure placement: here, top, bottom, or page
   \centering
   %\includegraphics[width=.9\textwidth]{boxplots-retro.pdf}\\
   % Created by tikzDevice version 0.6.1 on 2011-07-07 10:49:03
% !TEX encoding = UTF-8 Unicode
\begin{tikzpicture}[x=1pt,y=1pt]
\definecolor[named]{drawColor}{rgb}{0.00,0.00,0.00}
\definecolor[named]{fillColor}{rgb}{1.00,1.00,1.00}
\fill[color=fillColor,] (0,0) rectangle (505.89,650.43);
\begin{scope}
\path[clip] ( 32.47,450.25) rectangle (489.26,625.88);
\definecolor[named]{drawColor}{rgb}{0.00,0.00,0.00}

\draw[color=drawColor,line width= 1.2pt,line join=round,fill opacity=0.00,] ( 52.21,560.04) -- ( 74.77,560.04);

\draw[color=drawColor,dash pattern=on 4pt off 4pt ,line cap=round,line join=round,fill opacity=0.00,] ( 63.49,525.66) -- ( 63.49,548.70);

\draw[color=drawColor,dash pattern=on 4pt off 4pt ,line cap=round,line join=round,fill opacity=0.00,] ( 63.49,593.43) -- ( 63.49,578.52);

\draw[color=drawColor,line cap=round,line join=round,fill opacity=0.00,] ( 57.85,525.66) -- ( 69.13,525.66);

\draw[color=drawColor,line cap=round,line join=round,fill opacity=0.00,] ( 57.85,593.43) -- ( 69.13,593.43);

\draw[color=drawColor,line cap=round,line join=round,fill opacity=0.00,] ( 52.21,548.70) --
	( 74.77,548.70) --
	( 74.77,578.52) --
	( 52.21,578.52) --
	( 52.21,548.70);

\draw[color=drawColor,line width= 1.2pt,line join=round,fill opacity=0.00,] ( 80.41,509.52) -- (102.96,509.52);

\draw[color=drawColor,dash pattern=on 4pt off 4pt ,line cap=round,line join=round,fill opacity=0.00,] ( 91.69,497.06) -- ( 91.69,505.63);

\draw[color=drawColor,dash pattern=on 4pt off 4pt ,line cap=round,line join=round,fill opacity=0.00,] ( 91.69,521.99) -- ( 91.69,512.20);

\draw[color=drawColor,line cap=round,line join=round,fill opacity=0.00,] ( 86.05,497.06) -- ( 97.32,497.06);

\draw[color=drawColor,line cap=round,line join=round,fill opacity=0.00,] ( 86.05,521.99) -- ( 97.32,521.99);

\draw[color=drawColor,line cap=round,line join=round,fill opacity=0.00,] ( 80.41,505.63) --
	(102.96,505.63) --
	(102.96,512.20) --
	( 80.41,512.20) --
	( 80.41,505.63);

\draw[color=drawColor,line width= 1.2pt,line join=round,fill opacity=0.00,] (108.60,532.30) -- (131.16,532.30);

\draw[color=drawColor,dash pattern=on 4pt off 4pt ,line cap=round,line join=round,fill opacity=0.00,] (119.88,515.98) -- (119.88,528.50);

\draw[color=drawColor,dash pattern=on 4pt off 4pt ,line cap=round,line join=round,fill opacity=0.00,] (119.88,549.20) -- (119.88,536.88);

\draw[color=drawColor,line cap=round,line join=round,fill opacity=0.00,] (114.24,515.98) -- (125.52,515.98);

\draw[color=drawColor,line cap=round,line join=round,fill opacity=0.00,] (114.24,549.20) -- (125.52,549.20);

\draw[color=drawColor,line cap=round,line join=round,fill opacity=0.00,] (108.60,528.50) --
	(131.16,528.50) --
	(131.16,536.88) --
	(108.60,536.88) --
	(108.60,528.50);

\draw[color=drawColor,line width= 1.2pt,line join=round,fill opacity=0.00,] (136.80,525.35) -- (159.36,525.35);

\draw[color=drawColor,dash pattern=on 4pt off 4pt ,line cap=round,line join=round,fill opacity=0.00,] (148.08,505.24) -- (148.08,518.44);

\draw[color=drawColor,dash pattern=on 4pt off 4pt ,line cap=round,line join=round,fill opacity=0.00,] (148.08,555.08) -- (148.08,533.15);

\draw[color=drawColor,line cap=round,line join=round,fill opacity=0.00,] (142.44,505.24) -- (153.72,505.24);

\draw[color=drawColor,line cap=round,line join=round,fill opacity=0.00,] (142.44,555.08) -- (153.72,555.08);

\draw[color=drawColor,line cap=round,line join=round,fill opacity=0.00,] (136.80,518.44) --
	(159.36,518.44) --
	(159.36,533.15) --
	(136.80,533.15) --
	(136.80,518.44);

\draw[color=drawColor,line width= 1.2pt,line join=round,fill opacity=0.00,] (165.00,567.97) -- (187.55,567.97);

\draw[color=drawColor,dash pattern=on 4pt off 4pt ,line cap=round,line join=round,fill opacity=0.00,] (176.28,533.79) -- (176.28,559.81);

\draw[color=drawColor,dash pattern=on 4pt off 4pt ,line cap=round,line join=round,fill opacity=0.00,] (176.28,608.96) -- (176.28,587.54);

\draw[color=drawColor,line cap=round,line join=round,fill opacity=0.00,] (170.64,533.79) -- (181.91,533.79);

\draw[color=drawColor,line cap=round,line join=round,fill opacity=0.00,] (170.64,608.96) -- (181.91,608.96);

\draw[color=drawColor,line cap=round,line join=round,fill opacity=0.00,] (165.00,559.81) --
	(187.55,559.81) --
	(187.55,587.54) --
	(165.00,587.54) --
	(165.00,559.81);

\draw[color=drawColor,line width= 1.2pt,line join=round,fill opacity=0.00,] (193.19,528.31) -- (215.75,528.31);

\draw[color=drawColor,dash pattern=on 4pt off 4pt ,line cap=round,line join=round,fill opacity=0.00,] (204.47,504.96) -- (204.47,517.42);

\draw[color=drawColor,dash pattern=on 4pt off 4pt ,line cap=round,line join=round,fill opacity=0.00,] (204.47,570.33) -- (204.47,538.62);

\draw[color=drawColor,line cap=round,line join=round,fill opacity=0.00,] (198.83,504.96) -- (210.11,504.96);

\draw[color=drawColor,line cap=round,line join=round,fill opacity=0.00,] (198.83,570.33) -- (210.11,570.33);

\draw[color=drawColor,line cap=round,line join=round,fill opacity=0.00,] (193.19,517.42) --
	(215.75,517.42) --
	(215.75,538.62) --
	(193.19,538.62) --
	(193.19,517.42);

\draw[color=drawColor,line width= 1.2pt,line join=round,fill opacity=0.00,] (221.39,513.04) -- (243.95,513.04);

\draw[color=drawColor,dash pattern=on 4pt off 4pt ,line cap=round,line join=round,fill opacity=0.00,] (232.67,500.65) -- (232.67,510.03);

\draw[color=drawColor,dash pattern=on 4pt off 4pt ,line cap=round,line join=round,fill opacity=0.00,] (232.67,526.24) -- (232.67,516.53);

\draw[color=drawColor,line cap=round,line join=round,fill opacity=0.00,] (227.03,500.65) -- (238.31,500.65);

\draw[color=drawColor,line cap=round,line join=round,fill opacity=0.00,] (227.03,526.24) -- (238.31,526.24);

\draw[color=drawColor,line cap=round,line join=round,fill opacity=0.00,] (221.39,510.03) --
	(243.95,510.03) --
	(243.95,516.53) --
	(221.39,516.53) --
	(221.39,510.03);

\draw[color=drawColor,line width= 1.2pt,line join=round,fill opacity=0.00,] (249.59,497.71) -- (272.14,497.71);

\draw[color=drawColor,dash pattern=on 4pt off 4pt ,line cap=round,line join=round,fill opacity=0.00,] (260.87,474.28) -- (260.87,490.45);

\draw[color=drawColor,dash pattern=on 4pt off 4pt ,line cap=round,line join=round,fill opacity=0.00,] (260.87,518.54) -- (260.87,504.48);

\draw[color=drawColor,line cap=round,line join=round,fill opacity=0.00,] (255.23,474.28) -- (266.50,474.28);

\draw[color=drawColor,line cap=round,line join=round,fill opacity=0.00,] (255.23,518.54) -- (266.50,518.54);

\draw[color=drawColor,line cap=round,line join=round,fill opacity=0.00,] (249.59,490.45) --
	(272.14,490.45) --
	(272.14,504.48) --
	(249.59,504.48) --
	(249.59,490.45);

\draw[color=drawColor,line width= 1.2pt,line join=round,fill opacity=0.00,] (277.78,510.31) -- (300.34,510.31);

\draw[color=drawColor,dash pattern=on 4pt off 4pt ,line cap=round,line join=round,fill opacity=0.00,] (289.06,498.82) -- (289.06,507.41);

\draw[color=drawColor,dash pattern=on 4pt off 4pt ,line cap=round,line join=round,fill opacity=0.00,] (289.06,521.51) -- (289.06,513.23);

\draw[color=drawColor,line cap=round,line join=round,fill opacity=0.00,] (283.42,498.82) -- (294.70,498.82);

\draw[color=drawColor,line cap=round,line join=round,fill opacity=0.00,] (283.42,521.51) -- (294.70,521.51);

\draw[color=drawColor,line cap=round,line join=round,fill opacity=0.00,] (277.78,507.41) --
	(300.34,507.41) --
	(300.34,513.23) --
	(277.78,513.23) --
	(277.78,507.41);

\draw[color=drawColor,line width= 1.2pt,line join=round,fill opacity=0.00,] (305.98,490.03) -- (328.54,490.03);

\draw[color=drawColor,dash pattern=on 4pt off 4pt ,line cap=round,line join=round,fill opacity=0.00,] (317.26,475.18) -- (317.26,485.04);

\draw[color=drawColor,dash pattern=on 4pt off 4pt ,line cap=round,line join=round,fill opacity=0.00,] (317.26,508.13) -- (317.26,494.46);

\draw[color=drawColor,line cap=round,line join=round,fill opacity=0.00,] (311.62,475.18) -- (322.90,475.18);

\draw[color=drawColor,line cap=round,line join=round,fill opacity=0.00,] (311.62,508.13) -- (322.90,508.13);

\draw[color=drawColor,line cap=round,line join=round,fill opacity=0.00,] (305.98,485.04) --
	(328.54,485.04) --
	(328.54,494.46) --
	(305.98,494.46) --
	(305.98,485.04);

\draw[color=drawColor,line width= 1.2pt,line join=round,fill opacity=0.00,] (334.18,507.52) -- (356.73,507.52);

\draw[color=drawColor,dash pattern=on 4pt off 4pt ,line cap=round,line join=round,fill opacity=0.00,] (345.45,494.95) -- (345.45,503.48);

\draw[color=drawColor,dash pattern=on 4pt off 4pt ,line cap=round,line join=round,fill opacity=0.00,] (345.45,520.19) -- (345.45,510.18);

\draw[color=drawColor,line cap=round,line join=round,fill opacity=0.00,] (339.82,494.95) -- (351.09,494.95);

\draw[color=drawColor,line cap=round,line join=round,fill opacity=0.00,] (339.82,520.19) -- (351.09,520.19);

\draw[color=drawColor,line cap=round,line join=round,fill opacity=0.00,] (334.18,503.48) --
	(356.73,503.48) --
	(356.73,510.18) --
	(334.18,510.18) --
	(334.18,503.48);

\draw[color=drawColor,line width= 1.2pt,line join=round,fill opacity=0.00,] (362.37,508.12) -- (384.93,508.12);

\draw[color=drawColor,dash pattern=on 4pt off 4pt ,line cap=round,line join=round,fill opacity=0.00,] (373.65,489.96) -- (373.65,502.84);

\draw[color=drawColor,dash pattern=on 4pt off 4pt ,line cap=round,line join=round,fill opacity=0.00,] (373.65,524.79) -- (373.65,511.80);

\draw[color=drawColor,line cap=round,line join=round,fill opacity=0.00,] (368.01,489.96) -- (379.29,489.96);

\draw[color=drawColor,line cap=round,line join=round,fill opacity=0.00,] (368.01,524.79) -- (379.29,524.79);

\draw[color=drawColor,line cap=round,line join=round,fill opacity=0.00,] (362.37,502.84) --
	(384.93,502.84) --
	(384.93,511.80) --
	(362.37,511.80) --
	(362.37,502.84);

\draw[color=drawColor,line width= 1.2pt,line join=round,fill opacity=0.00,] (390.57,540.38) -- (413.13,540.38);

\draw[color=drawColor,dash pattern=on 4pt off 4pt ,line cap=round,line join=round,fill opacity=0.00,] (401.85,521.28) -- (401.85,536.42);

\draw[color=drawColor,dash pattern=on 4pt off 4pt ,line cap=round,line join=round,fill opacity=0.00,] (401.85,566.00) -- (401.85,548.27);

\draw[color=drawColor,line cap=round,line join=round,fill opacity=0.00,] (396.21,521.28) -- (407.49,521.28);

\draw[color=drawColor,line cap=round,line join=round,fill opacity=0.00,] (396.21,566.00) -- (407.49,566.00);

\draw[color=drawColor,line cap=round,line join=round,fill opacity=0.00,] (390.57,536.42) --
	(413.13,536.42) --
	(413.13,548.27) --
	(390.57,548.27) --
	(390.57,536.42);

\draw[color=drawColor,line width= 1.2pt,line join=round,fill opacity=0.00,] (418.77,511.32) -- (441.32,511.32);

\draw[color=drawColor,dash pattern=on 4pt off 4pt ,line cap=round,line join=round,fill opacity=0.00,] (430.04,493.74) -- (430.04,507.08);

\draw[color=drawColor,dash pattern=on 4pt off 4pt ,line cap=round,line join=round,fill opacity=0.00,] (430.04,529.98) -- (430.04,516.33);

\draw[color=drawColor,line cap=round,line join=round,fill opacity=0.00,] (424.41,493.74) -- (435.68,493.74);

\draw[color=drawColor,line cap=round,line join=round,fill opacity=0.00,] (424.41,529.98) -- (435.68,529.98);

\draw[color=drawColor,line cap=round,line join=round,fill opacity=0.00,] (418.77,507.08) --
	(441.32,507.08) --
	(441.32,516.33) --
	(418.77,516.33) --
	(418.77,507.08);

\draw[color=drawColor,line width= 1.2pt,line join=round,fill opacity=0.00,] (446.96,504.59) -- (469.52,504.59);

\draw[color=drawColor,dash pattern=on 4pt off 4pt ,line cap=round,line join=round,fill opacity=0.00,] (458.24,491.55) -- (458.24,501.51);

\draw[color=drawColor,dash pattern=on 4pt off 4pt ,line cap=round,line join=round,fill opacity=0.00,] (458.24,518.38) -- (458.24,508.33);

\draw[color=drawColor,line cap=round,line join=round,fill opacity=0.00,] (452.60,491.55) -- (463.88,491.55);

\draw[color=drawColor,line cap=round,line join=round,fill opacity=0.00,] (452.60,518.38) -- (463.88,518.38);

\draw[color=drawColor,line cap=round,line join=round,fill opacity=0.00,] (446.96,501.51) --
	(469.52,501.51) --
	(469.52,508.33) --
	(446.96,508.33) --
	(446.96,501.51);
\end{scope}
\begin{scope}
\path[clip] (  0.00,  0.00) rectangle (505.89,650.43);
\definecolor[named]{drawColor}{rgb}{0.00,0.00,0.00}

\draw[color=drawColor,line cap=round,line join=round,fill opacity=0.00,] ( 63.49,450.25) -- (458.24,450.25);

\draw[color=drawColor,line cap=round,line join=round,fill opacity=0.00,] ( 63.49,450.25) -- ( 63.49,446.29);

\draw[color=drawColor,line cap=round,line join=round,fill opacity=0.00,] ( 91.69,450.25) -- ( 91.69,446.29);

\draw[color=drawColor,line cap=round,line join=round,fill opacity=0.00,] (119.88,450.25) -- (119.88,446.29);

\draw[color=drawColor,line cap=round,line join=round,fill opacity=0.00,] (148.08,450.25) -- (148.08,446.29);

\draw[color=drawColor,line cap=round,line join=round,fill opacity=0.00,] (176.28,450.25) -- (176.28,446.29);

\draw[color=drawColor,line cap=round,line join=round,fill opacity=0.00,] (204.47,450.25) -- (204.47,446.29);

\draw[color=drawColor,line cap=round,line join=round,fill opacity=0.00,] (232.67,450.25) -- (232.67,446.29);

\draw[color=drawColor,line cap=round,line join=round,fill opacity=0.00,] (260.87,450.25) -- (260.87,446.29);

\draw[color=drawColor,line cap=round,line join=round,fill opacity=0.00,] (289.06,450.25) -- (289.06,446.29);

\draw[color=drawColor,line cap=round,line join=round,fill opacity=0.00,] (317.26,450.25) -- (317.26,446.29);

\draw[color=drawColor,line cap=round,line join=round,fill opacity=0.00,] (345.45,450.25) -- (345.45,446.29);

\draw[color=drawColor,line cap=round,line join=round,fill opacity=0.00,] (373.65,450.25) -- (373.65,446.29);

\draw[color=drawColor,line cap=round,line join=round,fill opacity=0.00,] (401.85,450.25) -- (401.85,446.29);

\draw[color=drawColor,line cap=round,line join=round,fill opacity=0.00,] (430.04,450.25) -- (430.04,446.29);

\draw[color=drawColor,line cap=round,line join=round,fill opacity=0.00,] (458.24,450.25) -- (458.24,446.29);

\node[color=drawColor,anchor=base,inner sep=0pt, outer sep=0pt, scale=  0.66] at ( 63.49,434.41) {1993%
};

\node[color=drawColor,anchor=base,inner sep=0pt, outer sep=0pt, scale=  0.66] at ( 91.69,434.41) {1994%
};

\node[color=drawColor,anchor=base,inner sep=0pt, outer sep=0pt, scale=  0.66] at (119.88,434.41) {1995%
};

\node[color=drawColor,anchor=base,inner sep=0pt, outer sep=0pt, scale=  0.66] at (148.08,434.41) {1996%
};

\node[color=drawColor,anchor=base,inner sep=0pt, outer sep=0pt, scale=  0.66] at (176.28,434.41) {1997%
};

\node[color=drawColor,anchor=base,inner sep=0pt, outer sep=0pt, scale=  0.66] at (204.47,434.41) {1998%
};

\node[color=drawColor,anchor=base,inner sep=0pt, outer sep=0pt, scale=  0.66] at (232.67,434.41) {1999%
};

\node[color=drawColor,anchor=base,inner sep=0pt, outer sep=0pt, scale=  0.66] at (260.87,434.41) {2000%
};

\node[color=drawColor,anchor=base,inner sep=0pt, outer sep=0pt, scale=  0.66] at (289.06,434.41) {2001%
};

\node[color=drawColor,anchor=base,inner sep=0pt, outer sep=0pt, scale=  0.66] at (317.26,434.41) {2002%
};

\node[color=drawColor,anchor=base,inner sep=0pt, outer sep=0pt, scale=  0.66] at (345.45,434.41) {2003%
};

\node[color=drawColor,anchor=base,inner sep=0pt, outer sep=0pt, scale=  0.66] at (373.65,434.41) {2004%
};

\node[color=drawColor,anchor=base,inner sep=0pt, outer sep=0pt, scale=  0.66] at (401.85,434.41) {2005%
};

\node[color=drawColor,anchor=base,inner sep=0pt, outer sep=0pt, scale=  0.66] at (430.04,434.41) {2006%
};

\node[color=drawColor,anchor=base,inner sep=0pt, outer sep=0pt, scale=  0.66] at (458.24,434.41) {2007%
};

\draw[color=drawColor,line cap=round,line join=round,fill opacity=0.00,] ( 32.47,456.76) -- ( 32.47,619.37);

\draw[color=drawColor,line cap=round,line join=round,fill opacity=0.00,] ( 32.47,456.76) -- ( 28.51,456.76);

\draw[color=drawColor,line cap=round,line join=round,fill opacity=0.00,] ( 32.47,479.99) -- ( 28.51,479.99);

\draw[color=drawColor,line cap=round,line join=round,fill opacity=0.00,] ( 32.47,503.22) -- ( 28.51,503.22);

\draw[color=drawColor,line cap=round,line join=round,fill opacity=0.00,] ( 32.47,526.45) -- ( 28.51,526.45);

\draw[color=drawColor,line cap=round,line join=round,fill opacity=0.00,] ( 32.47,549.68) -- ( 28.51,549.68);

\draw[color=drawColor,line cap=round,line join=round,fill opacity=0.00,] ( 32.47,572.91) -- ( 28.51,572.91);

\draw[color=drawColor,line cap=round,line join=round,fill opacity=0.00,] ( 32.47,596.14) -- ( 28.51,596.14);

\draw[color=drawColor,line cap=round,line join=round,fill opacity=0.00,] ( 32.47,619.37) -- ( 28.51,619.37);

\node[rotate= 90.00,color=drawColor,anchor=base,inner sep=0pt, outer sep=0pt, scale=  0.66] at ( 24.55,456.76) {0%
};

\node[rotate= 90.00,color=drawColor,anchor=base,inner sep=0pt, outer sep=0pt, scale=  0.66] at ( 24.55,479.99) {500%
};

\node[rotate= 90.00,color=drawColor,anchor=base,inner sep=0pt, outer sep=0pt, scale=  0.66] at ( 24.55,503.22) {1000%
};

\node[rotate= 90.00,color=drawColor,anchor=base,inner sep=0pt, outer sep=0pt, scale=  0.66] at ( 24.55,526.45) {1500%
};

\node[rotate= 90.00,color=drawColor,anchor=base,inner sep=0pt, outer sep=0pt, scale=  0.66] at ( 24.55,549.68) {2000%
};

\node[rotate= 90.00,color=drawColor,anchor=base,inner sep=0pt, outer sep=0pt, scale=  0.66] at ( 24.55,572.91) {2500%
};

\node[rotate= 90.00,color=drawColor,anchor=base,inner sep=0pt, outer sep=0pt, scale=  0.66] at ( 24.55,596.14) {3000%
};

\node[rotate= 90.00,color=drawColor,anchor=base,inner sep=0pt, outer sep=0pt, scale=  0.66] at ( 24.55,619.37) {3500%
};

\draw[color=drawColor,line cap=round,line join=round,fill opacity=0.00,] ( 32.47,450.25) --
	(489.26,450.25) --
	(489.26,625.88) --
	( 32.47,625.88) --
	( 32.47,450.25);
\end{scope}
\begin{scope}
\path[clip] ( 32.47,450.25) rectangle (489.26,625.88);
\definecolor[named]{drawColor}{rgb}{1.00,0.00,0.00}

\draw[color=drawColor,line cap=round,line join=round,fill opacity=0.00,] ( 65.09,572.39) -- ( 90.08,515.86);

\draw[color=drawColor,line cap=round,line join=round,fill opacity=0.00,] ( 93.04,515.96) -- (118.53,586.25);

\draw[color=drawColor,line cap=round,line join=round,fill opacity=0.00,] (121.58,586.39) -- (146.38,533.99);

\draw[color=drawColor,line cap=round,line join=round,fill opacity=0.00,] (150.53,533.52) -- (173.83,563.15);

\draw[color=drawColor,line cap=round,line join=round,fill opacity=0.00,] (178.61,563.07) -- (202.13,530.93);

\draw[color=drawColor,line cap=round,line join=round,fill opacity=0.00,] (208.18,526.35) -- (228.96,518.64);

\draw[color=drawColor,line cap=round,line join=round,fill opacity=0.00,] (236.42,515.99) -- (257.11,508.97);

\draw[color=drawColor,line cap=round,line join=round,fill opacity=0.00,] (264.80,507.27) -- (285.13,505.04);

\draw[color=drawColor,line cap=round,line join=round,fill opacity=0.00,] (291.86,501.81) -- (314.46,479.22);

\draw[color=drawColor,line cap=round,line join=round,fill opacity=0.00,] (320.21,479.06) -- (342.50,498.94);

\draw[color=drawColor,line cap=round,line join=round,fill opacity=0.00,] (349.41,501.34) -- (369.70,500.17);

\draw[color=drawColor,line cap=round,line join=round,fill opacity=0.00,] (376.04,503.10) -- (399.46,534.12);

\draw[color=drawColor,line cap=round,line join=round,fill opacity=0.00,] (404.89,534.74) -- (427.01,516.25);

\draw[color=drawColor,line cap=round,line join=round,fill opacity=0.00,] (433.94,512.98) -- (454.35,509.19);

\draw[color=drawColor,line cap=round,line join=round,fill opacity=0.00,] ( 63.49,576.01) circle (  0.89);

\draw[color=drawColor,line cap=round,line join=round,fill opacity=0.00,] ( 91.69,512.24) circle (  0.89);

\draw[color=drawColor,line cap=round,line join=round,fill opacity=0.00,] (119.88,589.97) circle (  0.89);

\draw[color=drawColor,line cap=round,line join=round,fill opacity=0.00,] (148.08,530.41) circle (  0.89);

\draw[color=drawColor,line cap=round,line join=round,fill opacity=0.00,] (176.28,566.27) circle (  0.89);

\draw[color=drawColor,line cap=round,line join=round,fill opacity=0.00,] (204.47,527.73) circle (  0.89);

\draw[color=drawColor,line cap=round,line join=round,fill opacity=0.00,] (232.67,517.26) circle (  0.89);

\draw[color=drawColor,line cap=round,line join=round,fill opacity=0.00,] (260.87,507.70) circle (  0.89);

\draw[color=drawColor,line cap=round,line join=round,fill opacity=0.00,] (289.06,504.61) circle (  0.89);

\draw[color=drawColor,line cap=round,line join=round,fill opacity=0.00,] (317.26,476.42) circle (  0.89);

\draw[color=drawColor,line cap=round,line join=round,fill opacity=0.00,] (345.45,501.57) circle (  0.89);

\draw[color=drawColor,line cap=round,line join=round,fill opacity=0.00,] (373.65,499.94) circle (  0.89);

\draw[color=drawColor,line cap=round,line join=round,fill opacity=0.00,] (401.85,537.28) circle (  0.89);

\draw[color=drawColor,line cap=round,line join=round,fill opacity=0.00,] (430.04,513.71) circle (  0.89);

\draw[color=drawColor,line cap=round,line join=round,fill opacity=0.00,] (458.24,508.47) circle (  0.89);
\definecolor[named]{drawColor}{rgb}{0.00,0.00,1.00}

\draw[color=drawColor,line cap=round,line join=round,fill opacity=0.00,] ( 63.49,526.45) --
	( 91.69,565.94) --
	(119.88,500.90) --
	(148.08,521.80) --
	(176.28,540.39) --
	(204.47,503.22) --
	(232.67,505.54) --
	(260.87,484.63) --
	(289.06,503.22) --
	(317.26,513.90) --
	(345.45,493.93) --
	(373.65,517.16) --
	(401.85,545.03) --
	(430.04,527.84) --
	(458.24,539.92);

\draw[color=drawColor,dash pattern=on 4pt off 4pt ,line cap=round,line join=round,fill opacity=0.00,] ( 63.49,505.54) --
	( 91.69,545.03) --
	(119.88,473.48) --
	(148.08,494.39) --
	(176.28,521.34) --
	(204.47,483.94) --
	(232.67,486.26) --
	(260.87,470.70) --
	(289.06,483.94) --
	(317.26,494.39) --
	(345.45,472.79) --
	(373.65,496.71) --
	(401.85,523.66) --
	(430.04,507.40) --
	(458.24,518.55);

\draw[color=drawColor,dash pattern=on 4pt off 4pt ,line cap=round,line join=round,fill opacity=0.00,] ( 63.49,547.36) --
	( 91.69,586.85) --
	(119.88,528.31) --
	(148.08,549.22) --
	(176.28,559.44) --
	(204.47,522.73) --
	(232.67,524.59) --
	(260.87,503.22) --
	(289.06,522.27) --
	(317.26,528.77) --
	(345.45,515.07) --
	(373.65,537.60) --
	(401.85,566.41) --
	(430.04,548.29) --
	(458.24,561.30);
\end{scope}
\begin{scope}
\path[clip] (  0.00,  0.00) rectangle (505.89,650.43);
\definecolor[named]{drawColor}{rgb}{0.00,0.00,0.00}

\node[color=drawColor,anchor=base,inner sep=0pt, outer sep=0pt, scale=  1.00] at (260.87,633.80) {(a) RPSS = 0.90 MC = 0.86%
};
\end{scope}
\begin{scope}
\path[clip] ( 32.47,233.44) rectangle (489.26,409.07);
\end{scope}
\begin{scope}
\path[clip] ( 32.47,233.44) rectangle (489.26,409.07);
\definecolor[named]{drawColor}{rgb}{0.00,0.00,0.00}

\draw[color=drawColor,line width= 1.2pt,line join=round,fill opacity=0.00,] ( 52.21,315.37) -- ( 74.77,315.37);

\draw[color=drawColor,dash pattern=on 4pt off 4pt ,line cap=round,line join=round,fill opacity=0.00,] ( 63.49,296.84) -- ( 63.49,311.13);

\draw[color=drawColor,dash pattern=on 4pt off 4pt ,line cap=round,line join=round,fill opacity=0.00,] ( 63.49,336.23) -- ( 63.49,321.18);

\draw[color=drawColor,line cap=round,line join=round,fill opacity=0.00,] ( 57.85,296.84) -- ( 69.13,296.84);

\draw[color=drawColor,line cap=round,line join=round,fill opacity=0.00,] ( 57.85,336.23) -- ( 69.13,336.23);

\draw[color=drawColor,line cap=round,line join=round,fill opacity=0.00,] ( 52.21,311.13) --
	( 74.77,311.13) --
	( 74.77,321.18) --
	( 52.21,321.18) --
	( 52.21,311.13);

\draw[color=drawColor,line width= 1.2pt,line join=round,fill opacity=0.00,] ( 80.41,293.77) -- (102.96,293.77);

\draw[color=drawColor,dash pattern=on 4pt off 4pt ,line cap=round,line join=round,fill opacity=0.00,] ( 91.69,277.98) -- ( 91.69,289.90);

\draw[color=drawColor,dash pattern=on 4pt off 4pt ,line cap=round,line join=round,fill opacity=0.00,] ( 91.69,312.90) -- ( 91.69,299.32);

\draw[color=drawColor,line cap=round,line join=round,fill opacity=0.00,] ( 86.05,277.98) -- ( 97.32,277.98);

\draw[color=drawColor,line cap=round,line join=round,fill opacity=0.00,] ( 86.05,312.90) -- ( 97.32,312.90);

\draw[color=drawColor,line cap=round,line join=round,fill opacity=0.00,] ( 80.41,289.90) --
	(102.96,289.90) --
	(102.96,299.32) --
	( 80.41,299.32) --
	( 80.41,289.90);

\draw[color=drawColor,line width= 1.2pt,line join=round,fill opacity=0.00,] (108.60,307.24) -- (131.16,307.24);

\draw[color=drawColor,dash pattern=on 4pt off 4pt ,line cap=round,line join=round,fill opacity=0.00,] (119.88,290.89) -- (119.88,302.28);

\draw[color=drawColor,dash pattern=on 4pt off 4pt ,line cap=round,line join=round,fill opacity=0.00,] (119.88,333.85) -- (119.88,315.00);

\draw[color=drawColor,line cap=round,line join=round,fill opacity=0.00,] (114.24,290.89) -- (125.52,290.89);

\draw[color=drawColor,line cap=round,line join=round,fill opacity=0.00,] (114.24,333.85) -- (125.52,333.85);

\draw[color=drawColor,line cap=round,line join=round,fill opacity=0.00,] (108.60,302.28) --
	(131.16,302.28) --
	(131.16,315.00) --
	(108.60,315.00) --
	(108.60,302.28);

\draw[color=drawColor,line width= 1.2pt,line join=round,fill opacity=0.00,] (136.80,300.98) -- (159.36,300.98);

\draw[color=drawColor,dash pattern=on 4pt off 4pt ,line cap=round,line join=round,fill opacity=0.00,] (148.08,285.89) -- (148.08,296.45);

\draw[color=drawColor,dash pattern=on 4pt off 4pt ,line cap=round,line join=round,fill opacity=0.00,] (148.08,321.42) -- (148.08,306.68);

\draw[color=drawColor,line cap=round,line join=round,fill opacity=0.00,] (142.44,285.89) -- (153.72,285.89);

\draw[color=drawColor,line cap=round,line join=round,fill opacity=0.00,] (142.44,321.42) -- (153.72,321.42);

\draw[color=drawColor,line cap=round,line join=round,fill opacity=0.00,] (136.80,296.45) --
	(159.36,296.45) --
	(159.36,306.68) --
	(136.80,306.68) --
	(136.80,296.45);

\draw[color=drawColor,line width= 1.2pt,line join=round,fill opacity=0.00,] (165.00,351.69) -- (187.55,351.69);

\draw[color=drawColor,dash pattern=on 4pt off 4pt ,line cap=round,line join=round,fill opacity=0.00,] (176.28,309.18) -- (176.28,338.91);

\draw[color=drawColor,dash pattern=on 4pt off 4pt ,line cap=round,line join=round,fill opacity=0.00,] (176.28,394.94) -- (176.28,363.40);

\draw[color=drawColor,line cap=round,line join=round,fill opacity=0.00,] (170.64,309.18) -- (181.91,309.18);

\draw[color=drawColor,line cap=round,line join=round,fill opacity=0.00,] (170.64,394.94) -- (181.91,394.94);

\draw[color=drawColor,line cap=round,line join=round,fill opacity=0.00,] (165.00,338.91) --
	(187.55,338.91) --
	(187.55,363.40) --
	(165.00,363.40) --
	(165.00,338.91);

\draw[color=drawColor,line width= 1.2pt,line join=round,fill opacity=0.00,] (193.19,312.43) -- (215.75,312.43);

\draw[color=drawColor,dash pattern=on 4pt off 4pt ,line cap=round,line join=round,fill opacity=0.00,] (204.47,278.92) -- (204.47,297.15);

\draw[color=drawColor,dash pattern=on 4pt off 4pt ,line cap=round,line join=round,fill opacity=0.00,] (204.47,370.72) -- (204.47,329.56);

\draw[color=drawColor,line cap=round,line join=round,fill opacity=0.00,] (198.83,278.92) -- (210.11,278.92);

\draw[color=drawColor,line cap=round,line join=round,fill opacity=0.00,] (198.83,370.72) -- (210.11,370.72);

\draw[color=drawColor,line cap=round,line join=round,fill opacity=0.00,] (193.19,297.15) --
	(215.75,297.15) --
	(215.75,329.56) --
	(193.19,329.56) --
	(193.19,297.15);

\draw[color=drawColor,line width= 1.2pt,line join=round,fill opacity=0.00,] (221.39,302.56) -- (243.95,302.56);

\draw[color=drawColor,dash pattern=on 4pt off 4pt ,line cap=round,line join=round,fill opacity=0.00,] (232.67,286.18) -- (232.67,298.61);

\draw[color=drawColor,dash pattern=on 4pt off 4pt ,line cap=round,line join=round,fill opacity=0.00,] (232.67,332.28) -- (232.67,312.61);

\draw[color=drawColor,line cap=round,line join=round,fill opacity=0.00,] (227.03,286.18) -- (238.31,286.18);

\draw[color=drawColor,line cap=round,line join=round,fill opacity=0.00,] (227.03,332.28) -- (238.31,332.28);

\draw[color=drawColor,line cap=round,line join=round,fill opacity=0.00,] (221.39,298.61) --
	(243.95,298.61) --
	(243.95,312.61) --
	(221.39,312.61) --
	(221.39,298.61);

\draw[color=drawColor,line width= 1.2pt,line join=round,fill opacity=0.00,] (249.59,283.30) -- (272.14,283.30);

\draw[color=drawColor,dash pattern=on 4pt off 4pt ,line cap=round,line join=round,fill opacity=0.00,] (260.87,249.35) -- (260.87,271.86);

\draw[color=drawColor,dash pattern=on 4pt off 4pt ,line cap=round,line join=round,fill opacity=0.00,] (260.87,310.12) -- (260.87,291.51);

\draw[color=drawColor,line cap=round,line join=round,fill opacity=0.00,] (255.23,249.35) -- (266.50,249.35);

\draw[color=drawColor,line cap=round,line join=round,fill opacity=0.00,] (255.23,310.12) -- (266.50,310.12);

\draw[color=drawColor,line cap=round,line join=round,fill opacity=0.00,] (249.59,271.86) --
	(272.14,271.86) --
	(272.14,291.51) --
	(249.59,291.51) --
	(249.59,271.86);

\draw[color=drawColor,line width= 1.2pt,line join=round,fill opacity=0.00,] (277.78,295.22) -- (300.34,295.22);

\draw[color=drawColor,dash pattern=on 4pt off 4pt ,line cap=round,line join=round,fill opacity=0.00,] (289.06,281.96) -- (289.06,292.47);

\draw[color=drawColor,dash pattern=on 4pt off 4pt ,line cap=round,line join=round,fill opacity=0.00,] (289.06,310.23) -- (289.06,299.60);

\draw[color=drawColor,line cap=round,line join=round,fill opacity=0.00,] (283.42,281.96) -- (294.70,281.96);

\draw[color=drawColor,line cap=round,line join=round,fill opacity=0.00,] (283.42,310.23) -- (294.70,310.23);

\draw[color=drawColor,line cap=round,line join=round,fill opacity=0.00,] (277.78,292.47) --
	(300.34,292.47) --
	(300.34,299.60) --
	(277.78,299.60) --
	(277.78,292.47);

\draw[color=drawColor,line width= 1.2pt,line join=round,fill opacity=0.00,] (305.98,286.07) -- (328.54,286.07);

\draw[color=drawColor,dash pattern=on 4pt off 4pt ,line cap=round,line join=round,fill opacity=0.00,] (317.26,269.74) -- (317.26,282.09);

\draw[color=drawColor,dash pattern=on 4pt off 4pt ,line cap=round,line join=round,fill opacity=0.00,] (317.26,303.93) -- (317.26,291.02);

\draw[color=drawColor,line cap=round,line join=round,fill opacity=0.00,] (311.62,269.74) -- (322.90,269.74);

\draw[color=drawColor,line cap=round,line join=round,fill opacity=0.00,] (311.62,303.93) -- (322.90,303.93);

\draw[color=drawColor,line cap=round,line join=round,fill opacity=0.00,] (305.98,282.09) --
	(328.54,282.09) --
	(328.54,291.02) --
	(305.98,291.02) --
	(305.98,282.09);

\draw[color=drawColor,line width= 1.2pt,line join=round,fill opacity=0.00,] (334.18,287.61) -- (356.73,287.61);

\draw[color=drawColor,dash pattern=on 4pt off 4pt ,line cap=round,line join=round,fill opacity=0.00,] (345.45,257.84) -- (345.45,278.92);

\draw[color=drawColor,dash pattern=on 4pt off 4pt ,line cap=round,line join=round,fill opacity=0.00,] (345.45,309.11) -- (345.45,293.03);

\draw[color=drawColor,line cap=round,line join=round,fill opacity=0.00,] (339.82,257.84) -- (351.09,257.84);

\draw[color=drawColor,line cap=round,line join=round,fill opacity=0.00,] (339.82,309.11) -- (351.09,309.11);

\draw[color=drawColor,line cap=round,line join=round,fill opacity=0.00,] (334.18,278.92) --
	(356.73,278.92) --
	(356.73,293.03) --
	(334.18,293.03) --
	(334.18,278.92);

\draw[color=drawColor,line width= 1.2pt,line join=round,fill opacity=0.00,] (362.37,295.82) -- (384.93,295.82);

\draw[color=drawColor,dash pattern=on 4pt off 4pt ,line cap=round,line join=round,fill opacity=0.00,] (373.65,280.10) -- (373.65,292.02);

\draw[color=drawColor,dash pattern=on 4pt off 4pt ,line cap=round,line join=round,fill opacity=0.00,] (373.65,318.34) -- (373.65,302.57);

\draw[color=drawColor,line cap=round,line join=round,fill opacity=0.00,] (368.01,280.10) -- (379.29,280.10);

\draw[color=drawColor,line cap=round,line join=round,fill opacity=0.00,] (368.01,318.34) -- (379.29,318.34);

\draw[color=drawColor,line cap=round,line join=round,fill opacity=0.00,] (362.37,292.02) --
	(384.93,292.02) --
	(384.93,302.57) --
	(362.37,302.57) --
	(362.37,292.02);

\draw[color=drawColor,line width= 1.2pt,line join=round,fill opacity=0.00,] (390.57,324.16) -- (413.13,324.16);

\draw[color=drawColor,dash pattern=on 4pt off 4pt ,line cap=round,line join=round,fill opacity=0.00,] (401.85,301.42) -- (401.85,319.28);

\draw[color=drawColor,dash pattern=on 4pt off 4pt ,line cap=round,line join=round,fill opacity=0.00,] (401.85,379.96) -- (401.85,345.43);

\draw[color=drawColor,line cap=round,line join=round,fill opacity=0.00,] (396.21,301.42) -- (407.49,301.42);

\draw[color=drawColor,line cap=round,line join=round,fill opacity=0.00,] (396.21,379.96) -- (407.49,379.96);

\draw[color=drawColor,line cap=round,line join=round,fill opacity=0.00,] (390.57,319.28) --
	(413.13,319.28) --
	(413.13,345.43) --
	(390.57,345.43) --
	(390.57,319.28);

\draw[color=drawColor,line width= 1.2pt,line join=round,fill opacity=0.00,] (418.77,298.82) -- (441.32,298.82);

\draw[color=drawColor,dash pattern=on 4pt off 4pt ,line cap=round,line join=round,fill opacity=0.00,] (430.04,277.52) -- (430.04,294.46);

\draw[color=drawColor,dash pattern=on 4pt off 4pt ,line cap=round,line join=round,fill opacity=0.00,] (430.04,323.27) -- (430.04,306.03);

\draw[color=drawColor,line cap=round,line join=round,fill opacity=0.00,] (424.41,277.52) -- (435.68,277.52);

\draw[color=drawColor,line cap=round,line join=round,fill opacity=0.00,] (424.41,323.27) -- (435.68,323.27);

\draw[color=drawColor,line cap=round,line join=round,fill opacity=0.00,] (418.77,294.46) --
	(441.32,294.46) --
	(441.32,306.03) --
	(418.77,306.03) --
	(418.77,294.46);

\draw[color=drawColor,line width= 1.2pt,line join=round,fill opacity=0.00,] (446.96,296.38) -- (469.52,296.38);

\draw[color=drawColor,dash pattern=on 4pt off 4pt ,line cap=round,line join=round,fill opacity=0.00,] (458.24,283.07) -- (458.24,293.11);

\draw[color=drawColor,dash pattern=on 4pt off 4pt ,line cap=round,line join=round,fill opacity=0.00,] (458.24,310.12) -- (458.24,299.94);

\draw[color=drawColor,line cap=round,line join=round,fill opacity=0.00,] (452.60,283.07) -- (463.88,283.07);

\draw[color=drawColor,line cap=round,line join=round,fill opacity=0.00,] (452.60,310.12) -- (463.88,310.12);

\draw[color=drawColor,line cap=round,line join=round,fill opacity=0.00,] (446.96,293.11) --
	(469.52,293.11) --
	(469.52,299.94) --
	(446.96,299.94) --
	(446.96,293.11);
\end{scope}
\begin{scope}
\path[clip] (  0.00,  0.00) rectangle (505.89,650.43);
\definecolor[named]{drawColor}{rgb}{0.00,0.00,0.00}

\draw[color=drawColor,line cap=round,line join=round,fill opacity=0.00,] ( 63.49,233.44) -- (458.24,233.44);

\draw[color=drawColor,line cap=round,line join=round,fill opacity=0.00,] ( 63.49,233.44) -- ( 63.49,229.48);

\draw[color=drawColor,line cap=round,line join=round,fill opacity=0.00,] ( 91.69,233.44) -- ( 91.69,229.48);

\draw[color=drawColor,line cap=round,line join=round,fill opacity=0.00,] (119.88,233.44) -- (119.88,229.48);

\draw[color=drawColor,line cap=round,line join=round,fill opacity=0.00,] (148.08,233.44) -- (148.08,229.48);

\draw[color=drawColor,line cap=round,line join=round,fill opacity=0.00,] (176.28,233.44) -- (176.28,229.48);

\draw[color=drawColor,line cap=round,line join=round,fill opacity=0.00,] (204.47,233.44) -- (204.47,229.48);

\draw[color=drawColor,line cap=round,line join=round,fill opacity=0.00,] (232.67,233.44) -- (232.67,229.48);

\draw[color=drawColor,line cap=round,line join=round,fill opacity=0.00,] (260.87,233.44) -- (260.87,229.48);

\draw[color=drawColor,line cap=round,line join=round,fill opacity=0.00,] (289.06,233.44) -- (289.06,229.48);

\draw[color=drawColor,line cap=round,line join=round,fill opacity=0.00,] (317.26,233.44) -- (317.26,229.48);

\draw[color=drawColor,line cap=round,line join=round,fill opacity=0.00,] (345.45,233.44) -- (345.45,229.48);

\draw[color=drawColor,line cap=round,line join=round,fill opacity=0.00,] (373.65,233.44) -- (373.65,229.48);

\draw[color=drawColor,line cap=round,line join=round,fill opacity=0.00,] (401.85,233.44) -- (401.85,229.48);

\draw[color=drawColor,line cap=round,line join=round,fill opacity=0.00,] (430.04,233.44) -- (430.04,229.48);

\draw[color=drawColor,line cap=round,line join=round,fill opacity=0.00,] (458.24,233.44) -- (458.24,229.48);

\node[color=drawColor,anchor=base,inner sep=0pt, outer sep=0pt, scale=  0.66] at ( 63.49,217.60) {1993%
};

\node[color=drawColor,anchor=base,inner sep=0pt, outer sep=0pt, scale=  0.66] at ( 91.69,217.60) {1994%
};

\node[color=drawColor,anchor=base,inner sep=0pt, outer sep=0pt, scale=  0.66] at (119.88,217.60) {1995%
};

\node[color=drawColor,anchor=base,inner sep=0pt, outer sep=0pt, scale=  0.66] at (148.08,217.60) {1996%
};

\node[color=drawColor,anchor=base,inner sep=0pt, outer sep=0pt, scale=  0.66] at (176.28,217.60) {1997%
};

\node[color=drawColor,anchor=base,inner sep=0pt, outer sep=0pt, scale=  0.66] at (204.47,217.60) {1998%
};

\node[color=drawColor,anchor=base,inner sep=0pt, outer sep=0pt, scale=  0.66] at (232.67,217.60) {1999%
};

\node[color=drawColor,anchor=base,inner sep=0pt, outer sep=0pt, scale=  0.66] at (260.87,217.60) {2000%
};

\node[color=drawColor,anchor=base,inner sep=0pt, outer sep=0pt, scale=  0.66] at (289.06,217.60) {2001%
};

\node[color=drawColor,anchor=base,inner sep=0pt, outer sep=0pt, scale=  0.66] at (317.26,217.60) {2002%
};

\node[color=drawColor,anchor=base,inner sep=0pt, outer sep=0pt, scale=  0.66] at (345.45,217.60) {2003%
};

\node[color=drawColor,anchor=base,inner sep=0pt, outer sep=0pt, scale=  0.66] at (373.65,217.60) {2004%
};

\node[color=drawColor,anchor=base,inner sep=0pt, outer sep=0pt, scale=  0.66] at (401.85,217.60) {2005%
};

\node[color=drawColor,anchor=base,inner sep=0pt, outer sep=0pt, scale=  0.66] at (430.04,217.60) {2006%
};

\node[color=drawColor,anchor=base,inner sep=0pt, outer sep=0pt, scale=  0.66] at (458.24,217.60) {2007%
};

\draw[color=drawColor,line cap=round,line join=round,fill opacity=0.00,] ( 32.47,239.95) -- ( 32.47,402.56);

\draw[color=drawColor,line cap=round,line join=round,fill opacity=0.00,] ( 32.47,239.95) -- ( 28.51,239.95);

\draw[color=drawColor,line cap=round,line join=round,fill opacity=0.00,] ( 32.47,263.18) -- ( 28.51,263.18);

\draw[color=drawColor,line cap=round,line join=round,fill opacity=0.00,] ( 32.47,286.41) -- ( 28.51,286.41);

\draw[color=drawColor,line cap=round,line join=round,fill opacity=0.00,] ( 32.47,309.64) -- ( 28.51,309.64);

\draw[color=drawColor,line cap=round,line join=round,fill opacity=0.00,] ( 32.47,332.87) -- ( 28.51,332.87);

\draw[color=drawColor,line cap=round,line join=round,fill opacity=0.00,] ( 32.47,356.10) -- ( 28.51,356.10);

\draw[color=drawColor,line cap=round,line join=round,fill opacity=0.00,] ( 32.47,379.33) -- ( 28.51,379.33);

\draw[color=drawColor,line cap=round,line join=round,fill opacity=0.00,] ( 32.47,402.56) -- ( 28.51,402.56);

\node[rotate= 90.00,color=drawColor,anchor=base,inner sep=0pt, outer sep=0pt, scale=  0.66] at ( 24.55,239.95) {0%
};

\node[rotate= 90.00,color=drawColor,anchor=base,inner sep=0pt, outer sep=0pt, scale=  0.66] at ( 24.55,263.18) {500%
};

\node[rotate= 90.00,color=drawColor,anchor=base,inner sep=0pt, outer sep=0pt, scale=  0.66] at ( 24.55,286.41) {1000%
};

\node[rotate= 90.00,color=drawColor,anchor=base,inner sep=0pt, outer sep=0pt, scale=  0.66] at ( 24.55,309.64) {1500%
};

\node[rotate= 90.00,color=drawColor,anchor=base,inner sep=0pt, outer sep=0pt, scale=  0.66] at ( 24.55,332.87) {2000%
};

\node[rotate= 90.00,color=drawColor,anchor=base,inner sep=0pt, outer sep=0pt, scale=  0.66] at ( 24.55,356.10) {2500%
};

\node[rotate= 90.00,color=drawColor,anchor=base,inner sep=0pt, outer sep=0pt, scale=  0.66] at ( 24.55,379.33) {3000%
};

\node[rotate= 90.00,color=drawColor,anchor=base,inner sep=0pt, outer sep=0pt, scale=  0.66] at ( 24.55,402.56) {3500%
};

\draw[color=drawColor,line cap=round,line join=round,fill opacity=0.00,] ( 32.47,233.44) --
	(489.26,233.44) --
	(489.26,409.07) --
	( 32.47,409.07) --
	( 32.47,233.44);
\end{scope}
\begin{scope}
\path[clip] ( 32.47,233.44) rectangle (489.26,409.07);
\definecolor[named]{drawColor}{rgb}{1.00,0.00,0.00}

\draw[color=drawColor,line cap=round,line join=round,fill opacity=0.00,] ( 65.09,355.58) -- ( 90.08,299.05);

\draw[color=drawColor,line cap=round,line join=round,fill opacity=0.00,] ( 93.04,299.15) -- (118.53,369.44);

\draw[color=drawColor,line cap=round,line join=round,fill opacity=0.00,] (121.58,369.58) -- (146.38,317.18);

\draw[color=drawColor,line cap=round,line join=round,fill opacity=0.00,] (150.53,316.71) -- (173.83,346.34);

\draw[color=drawColor,line cap=round,line join=round,fill opacity=0.00,] (178.61,346.26) -- (202.13,314.12);

\draw[color=drawColor,line cap=round,line join=round,fill opacity=0.00,] (208.18,309.54) -- (228.96,301.83);

\draw[color=drawColor,line cap=round,line join=round,fill opacity=0.00,] (236.42,299.18) -- (257.11,292.16);

\draw[color=drawColor,line cap=round,line join=round,fill opacity=0.00,] (264.80,290.46) -- (285.13,288.23);

\draw[color=drawColor,line cap=round,line join=round,fill opacity=0.00,] (291.86,285.00) -- (314.46,262.41);

\draw[color=drawColor,line cap=round,line join=round,fill opacity=0.00,] (320.21,262.25) -- (342.50,282.13);

\draw[color=drawColor,line cap=round,line join=round,fill opacity=0.00,] (349.41,284.53) -- (369.70,283.36);

\draw[color=drawColor,line cap=round,line join=round,fill opacity=0.00,] (376.04,286.29) -- (399.46,317.31);

\draw[color=drawColor,line cap=round,line join=round,fill opacity=0.00,] (404.89,317.93) -- (427.01,299.44);

\draw[color=drawColor,line cap=round,line join=round,fill opacity=0.00,] (433.94,296.17) -- (454.35,292.38);

\draw[color=drawColor,line cap=round,line join=round,fill opacity=0.00,] ( 63.49,359.20) circle (  0.89);

\draw[color=drawColor,line cap=round,line join=round,fill opacity=0.00,] ( 91.69,295.43) circle (  0.89);

\draw[color=drawColor,line cap=round,line join=round,fill opacity=0.00,] (119.88,373.16) circle (  0.89);

\draw[color=drawColor,line cap=round,line join=round,fill opacity=0.00,] (148.08,313.60) circle (  0.89);

\draw[color=drawColor,line cap=round,line join=round,fill opacity=0.00,] (176.28,349.46) circle (  0.89);

\draw[color=drawColor,line cap=round,line join=round,fill opacity=0.00,] (204.47,310.92) circle (  0.89);

\draw[color=drawColor,line cap=round,line join=round,fill opacity=0.00,] (232.67,300.45) circle (  0.89);

\draw[color=drawColor,line cap=round,line join=round,fill opacity=0.00,] (260.87,290.89) circle (  0.89);

\draw[color=drawColor,line cap=round,line join=round,fill opacity=0.00,] (289.06,287.80) circle (  0.89);

\draw[color=drawColor,line cap=round,line join=round,fill opacity=0.00,] (317.26,259.61) circle (  0.89);

\draw[color=drawColor,line cap=round,line join=round,fill opacity=0.00,] (345.45,284.76) circle (  0.89);

\draw[color=drawColor,line cap=round,line join=round,fill opacity=0.00,] (373.65,283.13) circle (  0.89);

\draw[color=drawColor,line cap=round,line join=round,fill opacity=0.00,] (401.85,320.47) circle (  0.89);

\draw[color=drawColor,line cap=round,line join=round,fill opacity=0.00,] (430.04,296.90) circle (  0.89);

\draw[color=drawColor,line cap=round,line join=round,fill opacity=0.00,] (458.24,291.66) circle (  0.89);
\definecolor[named]{drawColor}{rgb}{0.00,0.00,1.00}

\draw[color=drawColor,line cap=round,line join=round,fill opacity=0.00,] ( 63.49,318.00) --
	( 91.69,344.49) --
	(119.88,300.35) --
	(148.08,311.96) --
	(176.28,325.90) --
	(204.47,304.99) --
	(232.67,284.09) --
	(260.87,278.28) --
	(289.06,286.41) --
	(317.26,275.95) --
	(345.45,292.91) --
	(373.65,300.35) --
	(401.85,351.46) --
	(430.04,314.29) --
	(458.24,303.60);

\draw[color=drawColor,dash pattern=on 4pt off 4pt ,line cap=round,line join=round,fill opacity=0.00,] ( 63.49,290.59) --
	( 91.69,317.07) --
	(119.88,262.25) --
	(148.08,273.86) --
	(176.28,297.56) --
	(204.47,276.65) --
	(232.67,255.74) --
	(260.87,252.96) --
	(289.06,258.07) --
	(317.26,247.71) --
	(345.45,263.18) --
	(373.65,270.15) --
	(401.85,321.72) --
	(430.04,284.55) --
	(458.24,273.86);

\draw[color=drawColor,dash pattern=on 4pt off 4pt ,line cap=round,line join=round,fill opacity=0.00,] ( 63.49,345.42) --
	( 91.69,371.90) --
	(119.88,338.45) --
	(148.08,350.06) --
	(176.28,354.24) --
	(204.47,333.34) --
	(232.67,312.43) --
	(260.87,314.29) --
	(289.06,314.75) --
	(317.26,304.06) --
	(345.45,322.65) --
	(373.65,330.55) --
	(401.85,381.19) --
	(430.04,344.02) --
	(458.24,333.34);
\end{scope}
\begin{scope}
\path[clip] (  0.00,  0.00) rectangle (505.89,650.43);
\definecolor[named]{drawColor}{rgb}{0.00,0.00,0.00}

\node[color=drawColor,anchor=base,inner sep=0pt, outer sep=0pt, scale=  1.00] at (260.87,416.99) {(b) RPSS = 0.72 MC = 0.70%
};
\end{scope}
\begin{scope}
\path[clip] ( 32.47, 16.63) rectangle (489.26,192.26);
\end{scope}
\begin{scope}
\path[clip] ( 32.47, 16.63) rectangle (489.26,192.26);
\definecolor[named]{drawColor}{rgb}{0.00,0.00,0.00}

\draw[color=drawColor,line width= 1.2pt,line join=round,fill opacity=0.00,] ( 52.21, 90.01) -- ( 74.77, 90.01);

\draw[color=drawColor,dash pattern=on 4pt off 4pt ,line cap=round,line join=round,fill opacity=0.00,] ( 63.49, 66.54) -- ( 63.49, 83.07);

\draw[color=drawColor,dash pattern=on 4pt off 4pt ,line cap=round,line join=round,fill opacity=0.00,] ( 63.49,115.89) -- ( 63.49, 96.47);

\draw[color=drawColor,line cap=round,line join=round,fill opacity=0.00,] ( 57.85, 66.54) -- ( 69.13, 66.54);

\draw[color=drawColor,line cap=round,line join=round,fill opacity=0.00,] ( 57.85,115.89) -- ( 69.13,115.89);

\draw[color=drawColor,line cap=round,line join=round,fill opacity=0.00,] ( 52.21, 83.07) --
	( 74.77, 83.07) --
	( 74.77, 96.47) --
	( 52.21, 96.47) --
	( 52.21, 83.07);

\draw[color=drawColor,line width= 1.2pt,line join=round,fill opacity=0.00,] ( 80.41, 90.18) -- (102.96, 90.18);

\draw[color=drawColor,dash pattern=on 4pt off 4pt ,line cap=round,line join=round,fill opacity=0.00,] ( 91.69, 68.85) -- ( 91.69, 83.26);

\draw[color=drawColor,dash pattern=on 4pt off 4pt ,line cap=round,line join=round,fill opacity=0.00,] ( 91.69,123.06) -- ( 91.69, 99.45);

\draw[color=drawColor,line cap=round,line join=round,fill opacity=0.00,] ( 86.05, 68.85) -- ( 97.32, 68.85);

\draw[color=drawColor,line cap=round,line join=round,fill opacity=0.00,] ( 86.05,123.06) -- ( 97.32,123.06);

\draw[color=drawColor,line cap=round,line join=round,fill opacity=0.00,] ( 80.41, 83.26) --
	(102.96, 83.26) --
	(102.96, 99.45) --
	( 80.41, 99.45) --
	( 80.41, 83.26);

\draw[color=drawColor,line width= 1.2pt,line join=round,fill opacity=0.00,] (108.60, 95.30) -- (131.16, 95.30);

\draw[color=drawColor,dash pattern=on 4pt off 4pt ,line cap=round,line join=round,fill opacity=0.00,] (119.88, 72.88) -- (119.88, 88.08);

\draw[color=drawColor,dash pattern=on 4pt off 4pt ,line cap=round,line join=round,fill opacity=0.00,] (119.88,122.68) -- (119.88,101.93);

\draw[color=drawColor,line cap=round,line join=round,fill opacity=0.00,] (114.24, 72.88) -- (125.52, 72.88);

\draw[color=drawColor,line cap=round,line join=round,fill opacity=0.00,] (114.24,122.68) -- (125.52,122.68);

\draw[color=drawColor,line cap=round,line join=round,fill opacity=0.00,] (108.60, 88.08) --
	(131.16, 88.08) --
	(131.16,101.93) --
	(108.60,101.93) --
	(108.60, 88.08);

\draw[color=drawColor,line width= 1.2pt,line join=round,fill opacity=0.00,] (136.80, 84.06) -- (159.36, 84.06);

\draw[color=drawColor,dash pattern=on 4pt off 4pt ,line cap=round,line join=round,fill opacity=0.00,] (148.08, 61.51) -- (148.08, 78.55);

\draw[color=drawColor,dash pattern=on 4pt off 4pt ,line cap=round,line join=round,fill opacity=0.00,] (148.08,109.20) -- (148.08, 90.81);

\draw[color=drawColor,line cap=round,line join=round,fill opacity=0.00,] (142.44, 61.51) -- (153.72, 61.51);

\draw[color=drawColor,line cap=round,line join=round,fill opacity=0.00,] (142.44,109.20) -- (153.72,109.20);

\draw[color=drawColor,line cap=round,line join=round,fill opacity=0.00,] (136.80, 78.55) --
	(159.36, 78.55) --
	(159.36, 90.81) --
	(136.80, 90.81) --
	(136.80, 78.55);

\draw[color=drawColor,line width= 1.2pt,line join=round,fill opacity=0.00,] (165.00, 97.90) -- (187.55, 97.90);

\draw[color=drawColor,dash pattern=on 4pt off 4pt ,line cap=round,line join=round,fill opacity=0.00,] (176.28, 76.96) -- (176.28, 92.52);

\draw[color=drawColor,dash pattern=on 4pt off 4pt ,line cap=round,line join=round,fill opacity=0.00,] (176.28,121.29) -- (176.28,104.09);

\draw[color=drawColor,line cap=round,line join=round,fill opacity=0.00,] (170.64, 76.96) -- (181.91, 76.96);

\draw[color=drawColor,line cap=round,line join=round,fill opacity=0.00,] (170.64,121.29) -- (181.91,121.29);

\draw[color=drawColor,line cap=round,line join=round,fill opacity=0.00,] (165.00, 92.52) --
	(187.55, 92.52) --
	(187.55,104.09) --
	(165.00,104.09) --
	(165.00, 92.52);

\draw[color=drawColor,line width= 1.2pt,line join=round,fill opacity=0.00,] (193.19,109.93) -- (215.75,109.93);

\draw[color=drawColor,dash pattern=on 4pt off 4pt ,line cap=round,line join=round,fill opacity=0.00,] (204.47, 78.89) -- (204.47,101.29);

\draw[color=drawColor,dash pattern=on 4pt off 4pt ,line cap=round,line join=round,fill opacity=0.00,] (204.47,171.87) -- (204.47,133.09);

\draw[color=drawColor,line cap=round,line join=round,fill opacity=0.00,] (198.83, 78.89) -- (210.11, 78.89);

\draw[color=drawColor,line cap=round,line join=round,fill opacity=0.00,] (198.83,171.87) -- (210.11,171.87);

\draw[color=drawColor,line cap=round,line join=round,fill opacity=0.00,] (193.19,101.29) --
	(215.75,101.29) --
	(215.75,133.09) --
	(193.19,133.09) --
	(193.19,101.29);

\draw[color=drawColor,line width= 1.2pt,line join=round,fill opacity=0.00,] (221.39,105.51) -- (243.95,105.51);

\draw[color=drawColor,dash pattern=on 4pt off 4pt ,line cap=round,line join=round,fill opacity=0.00,] (232.67, 74.26) -- (232.67, 98.69);

\draw[color=drawColor,dash pattern=on 4pt off 4pt ,line cap=round,line join=round,fill opacity=0.00,] (232.67,151.71) -- (232.67,119.92);

\draw[color=drawColor,line cap=round,line join=round,fill opacity=0.00,] (227.03, 74.26) -- (238.31, 74.26);

\draw[color=drawColor,line cap=round,line join=round,fill opacity=0.00,] (227.03,151.71) -- (238.31,151.71);

\draw[color=drawColor,line cap=round,line join=round,fill opacity=0.00,] (221.39, 98.69) --
	(243.95, 98.69) --
	(243.95,119.92) --
	(221.39,119.92) --
	(221.39, 98.69);

\draw[color=drawColor,line width= 1.2pt,line join=round,fill opacity=0.00,] (249.59, 80.64) -- (272.14, 80.64);

\draw[color=drawColor,dash pattern=on 4pt off 4pt ,line cap=round,line join=round,fill opacity=0.00,] (260.87, 60.72) -- (260.87, 75.57);

\draw[color=drawColor,dash pattern=on 4pt off 4pt ,line cap=round,line join=round,fill opacity=0.00,] (260.87,100.33) -- (260.87, 85.51);

\draw[color=drawColor,line cap=round,line join=round,fill opacity=0.00,] (255.23, 60.72) -- (266.50, 60.72);

\draw[color=drawColor,line cap=round,line join=round,fill opacity=0.00,] (255.23,100.33) -- (266.50,100.33);

\draw[color=drawColor,line cap=round,line join=round,fill opacity=0.00,] (249.59, 75.57) --
	(272.14, 75.57) --
	(272.14, 85.51) --
	(249.59, 85.51) --
	(249.59, 75.57);

\draw[color=drawColor,line width= 1.2pt,line join=round,fill opacity=0.00,] (277.78, 90.19) -- (300.34, 90.19);

\draw[color=drawColor,dash pattern=on 4pt off 4pt ,line cap=round,line join=round,fill opacity=0.00,] (289.06, 68.10) -- (289.06, 82.34);

\draw[color=drawColor,dash pattern=on 4pt off 4pt ,line cap=round,line join=round,fill opacity=0.00,] (289.06,123.58) -- (289.06, 99.24);

\draw[color=drawColor,line cap=round,line join=round,fill opacity=0.00,] (283.42, 68.10) -- (294.70, 68.10);

\draw[color=drawColor,line cap=round,line join=round,fill opacity=0.00,] (283.42,123.58) -- (294.70,123.58);

\draw[color=drawColor,line cap=round,line join=round,fill opacity=0.00,] (277.78, 82.34) --
	(300.34, 82.34) --
	(300.34, 99.24) --
	(277.78, 99.24) --
	(277.78, 82.34);

\draw[color=drawColor,line width= 1.2pt,line join=round,fill opacity=0.00,] (305.98, 87.35) -- (328.54, 87.35);

\draw[color=drawColor,dash pattern=on 4pt off 4pt ,line cap=round,line join=round,fill opacity=0.00,] (317.26, 66.20) -- (317.26, 81.86);

\draw[color=drawColor,dash pattern=on 4pt off 4pt ,line cap=round,line join=round,fill opacity=0.00,] (317.26,112.99) -- (317.26, 94.44);

\draw[color=drawColor,line cap=round,line join=round,fill opacity=0.00,] (311.62, 66.20) -- (322.90, 66.20);

\draw[color=drawColor,line cap=round,line join=round,fill opacity=0.00,] (311.62,112.99) -- (322.90,112.99);

\draw[color=drawColor,line cap=round,line join=round,fill opacity=0.00,] (305.98, 81.86) --
	(328.54, 81.86) --
	(328.54, 94.44) --
	(305.98, 94.44) --
	(305.98, 81.86);

\draw[color=drawColor,line width= 1.2pt,line join=round,fill opacity=0.00,] (334.18, 84.79) -- (356.73, 84.79);

\draw[color=drawColor,dash pattern=on 4pt off 4pt ,line cap=round,line join=round,fill opacity=0.00,] (345.45, 62.87) -- (345.45, 79.99);

\draw[color=drawColor,dash pattern=on 4pt off 4pt ,line cap=round,line join=round,fill opacity=0.00,] (345.45,109.22) -- (345.45, 91.71);

\draw[color=drawColor,line cap=round,line join=round,fill opacity=0.00,] (339.82, 62.87) -- (351.09, 62.87);

\draw[color=drawColor,line cap=round,line join=round,fill opacity=0.00,] (339.82,109.22) -- (351.09,109.22);

\draw[color=drawColor,line cap=round,line join=round,fill opacity=0.00,] (334.18, 79.99) --
	(356.73, 79.99) --
	(356.73, 91.71) --
	(334.18, 91.71) --
	(334.18, 79.99);

\draw[color=drawColor,line width= 1.2pt,line join=round,fill opacity=0.00,] (362.37, 77.63) -- (384.93, 77.63);

\draw[color=drawColor,dash pattern=on 4pt off 4pt ,line cap=round,line join=round,fill opacity=0.00,] (373.65, 52.75) -- (373.65, 71.29);

\draw[color=drawColor,dash pattern=on 4pt off 4pt ,line cap=round,line join=round,fill opacity=0.00,] (373.65,102.40) -- (373.65, 83.79);

\draw[color=drawColor,line cap=round,line join=round,fill opacity=0.00,] (368.01, 52.75) -- (379.29, 52.75);

\draw[color=drawColor,line cap=round,line join=round,fill opacity=0.00,] (368.01,102.40) -- (379.29,102.40);

\draw[color=drawColor,line cap=round,line join=round,fill opacity=0.00,] (362.37, 71.29) --
	(384.93, 71.29) --
	(384.93, 83.79) --
	(362.37, 83.79) --
	(362.37, 71.29);

\draw[color=drawColor,line width= 1.2pt,line join=round,fill opacity=0.00,] (390.57,102.04) -- (413.13,102.04);

\draw[color=drawColor,dash pattern=on 4pt off 4pt ,line cap=round,line join=round,fill opacity=0.00,] (401.85, 74.94) -- (401.85, 93.51);

\draw[color=drawColor,dash pattern=on 4pt off 4pt ,line cap=round,line join=round,fill opacity=0.00,] (401.85,132.43) -- (401.85,109.13);

\draw[color=drawColor,line cap=round,line join=round,fill opacity=0.00,] (396.21, 74.94) -- (407.49, 74.94);

\draw[color=drawColor,line cap=round,line join=round,fill opacity=0.00,] (396.21,132.43) -- (407.49,132.43);

\draw[color=drawColor,line cap=round,line join=round,fill opacity=0.00,] (390.57, 93.51) --
	(413.13, 93.51) --
	(413.13,109.13) --
	(390.57,109.13) --
	(390.57, 93.51);

\draw[color=drawColor,line width= 1.2pt,line join=round,fill opacity=0.00,] (418.77, 86.56) -- (441.32, 86.56);

\draw[color=drawColor,dash pattern=on 4pt off 4pt ,line cap=round,line join=round,fill opacity=0.00,] (430.04, 50.80) -- (430.04, 78.04);

\draw[color=drawColor,dash pattern=on 4pt off 4pt ,line cap=round,line join=round,fill opacity=0.00,] (430.04,128.00) -- (430.04, 98.11);

\draw[color=drawColor,line cap=round,line join=round,fill opacity=0.00,] (424.41, 50.80) -- (435.68, 50.80);

\draw[color=drawColor,line cap=round,line join=round,fill opacity=0.00,] (424.41,128.00) -- (435.68,128.00);

\draw[color=drawColor,line cap=round,line join=round,fill opacity=0.00,] (418.77, 78.04) --
	(441.32, 78.04) --
	(441.32, 98.11) --
	(418.77, 98.11) --
	(418.77, 78.04);

\draw[color=drawColor,line width= 1.2pt,line join=round,fill opacity=0.00,] (446.96, 92.03) -- (469.52, 92.03);

\draw[color=drawColor,dash pattern=on 4pt off 4pt ,line cap=round,line join=round,fill opacity=0.00,] (458.24, 71.95) -- (458.24, 84.68);

\draw[color=drawColor,dash pattern=on 4pt off 4pt ,line cap=round,line join=round,fill opacity=0.00,] (458.24,115.95) -- (458.24, 97.28);

\draw[color=drawColor,line cap=round,line join=round,fill opacity=0.00,] (452.60, 71.95) -- (463.88, 71.95);

\draw[color=drawColor,line cap=round,line join=round,fill opacity=0.00,] (452.60,115.95) -- (463.88,115.95);

\draw[color=drawColor,line cap=round,line join=round,fill opacity=0.00,] (446.96, 84.68) --
	(469.52, 84.68) --
	(469.52, 97.28) --
	(446.96, 97.28) --
	(446.96, 84.68);
\end{scope}
\begin{scope}
\path[clip] (  0.00,  0.00) rectangle (505.89,650.43);
\definecolor[named]{drawColor}{rgb}{0.00,0.00,0.00}

\draw[color=drawColor,line cap=round,line join=round,fill opacity=0.00,] ( 63.49, 16.63) -- (458.24, 16.63);

\draw[color=drawColor,line cap=round,line join=round,fill opacity=0.00,] ( 63.49, 16.63) -- ( 63.49, 12.67);

\draw[color=drawColor,line cap=round,line join=round,fill opacity=0.00,] ( 91.69, 16.63) -- ( 91.69, 12.67);

\draw[color=drawColor,line cap=round,line join=round,fill opacity=0.00,] (119.88, 16.63) -- (119.88, 12.67);

\draw[color=drawColor,line cap=round,line join=round,fill opacity=0.00,] (148.08, 16.63) -- (148.08, 12.67);

\draw[color=drawColor,line cap=round,line join=round,fill opacity=0.00,] (176.28, 16.63) -- (176.28, 12.67);

\draw[color=drawColor,line cap=round,line join=round,fill opacity=0.00,] (204.47, 16.63) -- (204.47, 12.67);

\draw[color=drawColor,line cap=round,line join=round,fill opacity=0.00,] (232.67, 16.63) -- (232.67, 12.67);

\draw[color=drawColor,line cap=round,line join=round,fill opacity=0.00,] (260.87, 16.63) -- (260.87, 12.67);

\draw[color=drawColor,line cap=round,line join=round,fill opacity=0.00,] (289.06, 16.63) -- (289.06, 12.67);

\draw[color=drawColor,line cap=round,line join=round,fill opacity=0.00,] (317.26, 16.63) -- (317.26, 12.67);

\draw[color=drawColor,line cap=round,line join=round,fill opacity=0.00,] (345.45, 16.63) -- (345.45, 12.67);

\draw[color=drawColor,line cap=round,line join=round,fill opacity=0.00,] (373.65, 16.63) -- (373.65, 12.67);

\draw[color=drawColor,line cap=round,line join=round,fill opacity=0.00,] (401.85, 16.63) -- (401.85, 12.67);

\draw[color=drawColor,line cap=round,line join=round,fill opacity=0.00,] (430.04, 16.63) -- (430.04, 12.67);

\draw[color=drawColor,line cap=round,line join=round,fill opacity=0.00,] (458.24, 16.63) -- (458.24, 12.67);

\node[color=drawColor,anchor=base,inner sep=0pt, outer sep=0pt, scale=  0.66] at ( 63.49,  0.79) {1993%
};

\node[color=drawColor,anchor=base,inner sep=0pt, outer sep=0pt, scale=  0.66] at ( 91.69,  0.79) {1994%
};

\node[color=drawColor,anchor=base,inner sep=0pt, outer sep=0pt, scale=  0.66] at (119.88,  0.79) {1995%
};

\node[color=drawColor,anchor=base,inner sep=0pt, outer sep=0pt, scale=  0.66] at (148.08,  0.79) {1996%
};

\node[color=drawColor,anchor=base,inner sep=0pt, outer sep=0pt, scale=  0.66] at (176.28,  0.79) {1997%
};

\node[color=drawColor,anchor=base,inner sep=0pt, outer sep=0pt, scale=  0.66] at (204.47,  0.79) {1998%
};

\node[color=drawColor,anchor=base,inner sep=0pt, outer sep=0pt, scale=  0.66] at (232.67,  0.79) {1999%
};

\node[color=drawColor,anchor=base,inner sep=0pt, outer sep=0pt, scale=  0.66] at (260.87,  0.79) {2000%
};

\node[color=drawColor,anchor=base,inner sep=0pt, outer sep=0pt, scale=  0.66] at (289.06,  0.79) {2001%
};

\node[color=drawColor,anchor=base,inner sep=0pt, outer sep=0pt, scale=  0.66] at (317.26,  0.79) {2002%
};

\node[color=drawColor,anchor=base,inner sep=0pt, outer sep=0pt, scale=  0.66] at (345.45,  0.79) {2003%
};

\node[color=drawColor,anchor=base,inner sep=0pt, outer sep=0pt, scale=  0.66] at (373.65,  0.79) {2004%
};

\node[color=drawColor,anchor=base,inner sep=0pt, outer sep=0pt, scale=  0.66] at (401.85,  0.79) {2005%
};

\node[color=drawColor,anchor=base,inner sep=0pt, outer sep=0pt, scale=  0.66] at (430.04,  0.79) {2006%
};

\node[color=drawColor,anchor=base,inner sep=0pt, outer sep=0pt, scale=  0.66] at (458.24,  0.79) {2007%
};

\draw[color=drawColor,line cap=round,line join=round,fill opacity=0.00,] ( 32.47, 23.14) -- ( 32.47,185.75);

\draw[color=drawColor,line cap=round,line join=round,fill opacity=0.00,] ( 32.47, 23.14) -- ( 28.51, 23.14);

\draw[color=drawColor,line cap=round,line join=round,fill opacity=0.00,] ( 32.47, 46.37) -- ( 28.51, 46.37);

\draw[color=drawColor,line cap=round,line join=round,fill opacity=0.00,] ( 32.47, 69.60) -- ( 28.51, 69.60);

\draw[color=drawColor,line cap=round,line join=round,fill opacity=0.00,] ( 32.47, 92.83) -- ( 28.51, 92.83);

\draw[color=drawColor,line cap=round,line join=round,fill opacity=0.00,] ( 32.47,116.06) -- ( 28.51,116.06);

\draw[color=drawColor,line cap=round,line join=round,fill opacity=0.00,] ( 32.47,139.29) -- ( 28.51,139.29);

\draw[color=drawColor,line cap=round,line join=round,fill opacity=0.00,] ( 32.47,162.52) -- ( 28.51,162.52);

\draw[color=drawColor,line cap=round,line join=round,fill opacity=0.00,] ( 32.47,185.75) -- ( 28.51,185.75);

\node[rotate= 90.00,color=drawColor,anchor=base,inner sep=0pt, outer sep=0pt, scale=  0.66] at ( 24.55, 23.14) {0%
};

\node[rotate= 90.00,color=drawColor,anchor=base,inner sep=0pt, outer sep=0pt, scale=  0.66] at ( 24.55, 46.37) {500%
};

\node[rotate= 90.00,color=drawColor,anchor=base,inner sep=0pt, outer sep=0pt, scale=  0.66] at ( 24.55, 69.60) {1000%
};

\node[rotate= 90.00,color=drawColor,anchor=base,inner sep=0pt, outer sep=0pt, scale=  0.66] at ( 24.55, 92.83) {1500%
};

\node[rotate= 90.00,color=drawColor,anchor=base,inner sep=0pt, outer sep=0pt, scale=  0.66] at ( 24.55,116.06) {2000%
};

\node[rotate= 90.00,color=drawColor,anchor=base,inner sep=0pt, outer sep=0pt, scale=  0.66] at ( 24.55,139.29) {2500%
};

\node[rotate= 90.00,color=drawColor,anchor=base,inner sep=0pt, outer sep=0pt, scale=  0.66] at ( 24.55,162.52) {3000%
};

\node[rotate= 90.00,color=drawColor,anchor=base,inner sep=0pt, outer sep=0pt, scale=  0.66] at ( 24.55,185.75) {3500%
};

\draw[color=drawColor,line cap=round,line join=round,fill opacity=0.00,] ( 32.47, 16.63) --
	(489.26, 16.63) --
	(489.26,192.26) --
	( 32.47,192.26) --
	( 32.47, 16.63);
\end{scope}
\begin{scope}
\path[clip] ( 32.47, 16.63) rectangle (489.26,192.26);
\definecolor[named]{drawColor}{rgb}{1.00,0.00,0.00}

\draw[color=drawColor,line cap=round,line join=round,fill opacity=0.00,] ( 65.09,138.77) -- ( 90.08, 82.24);

\draw[color=drawColor,line cap=round,line join=round,fill opacity=0.00,] ( 93.04, 82.34) -- (118.53,152.63);

\draw[color=drawColor,line cap=round,line join=round,fill opacity=0.00,] (121.58,152.77) -- (146.38,100.37);

\draw[color=drawColor,line cap=round,line join=round,fill opacity=0.00,] (150.53, 99.90) -- (173.83,129.53);

\draw[color=drawColor,line cap=round,line join=round,fill opacity=0.00,] (178.61,129.45) -- (202.13, 97.31);

\draw[color=drawColor,line cap=round,line join=round,fill opacity=0.00,] (208.18, 92.73) -- (228.96, 85.02);

\draw[color=drawColor,line cap=round,line join=round,fill opacity=0.00,] (236.42, 82.37) -- (257.11, 75.35);

\draw[color=drawColor,line cap=round,line join=round,fill opacity=0.00,] (264.80, 73.65) -- (285.13, 71.42);

\draw[color=drawColor,line cap=round,line join=round,fill opacity=0.00,] (291.86, 68.19) -- (314.46, 45.60);

\draw[color=drawColor,line cap=round,line join=round,fill opacity=0.00,] (320.21, 45.44) -- (342.50, 65.32);

\draw[color=drawColor,line cap=round,line join=round,fill opacity=0.00,] (349.41, 67.72) -- (369.70, 66.55);

\draw[color=drawColor,line cap=round,line join=round,fill opacity=0.00,] (376.04, 69.48) -- (399.46,100.50);

\draw[color=drawColor,line cap=round,line join=round,fill opacity=0.00,] (404.89,101.12) -- (427.01, 82.63);

\draw[color=drawColor,line cap=round,line join=round,fill opacity=0.00,] (433.94, 79.36) -- (454.35, 75.57);

\draw[color=drawColor,line cap=round,line join=round,fill opacity=0.00,] ( 63.49,142.39) circle (  0.89);

\draw[color=drawColor,line cap=round,line join=round,fill opacity=0.00,] ( 91.69, 78.62) circle (  0.89);

\draw[color=drawColor,line cap=round,line join=round,fill opacity=0.00,] (119.88,156.35) circle (  0.89);

\draw[color=drawColor,line cap=round,line join=round,fill opacity=0.00,] (148.08, 96.79) circle (  0.89);

\draw[color=drawColor,line cap=round,line join=round,fill opacity=0.00,] (176.28,132.65) circle (  0.89);

\draw[color=drawColor,line cap=round,line join=round,fill opacity=0.00,] (204.47, 94.11) circle (  0.89);

\draw[color=drawColor,line cap=round,line join=round,fill opacity=0.00,] (232.67, 83.64) circle (  0.89);

\draw[color=drawColor,line cap=round,line join=round,fill opacity=0.00,] (260.87, 74.08) circle (  0.89);

\draw[color=drawColor,line cap=round,line join=round,fill opacity=0.00,] (289.06, 70.99) circle (  0.89);

\draw[color=drawColor,line cap=round,line join=round,fill opacity=0.00,] (317.26, 42.80) circle (  0.89);

\draw[color=drawColor,line cap=round,line join=round,fill opacity=0.00,] (345.45, 67.95) circle (  0.89);

\draw[color=drawColor,line cap=round,line join=round,fill opacity=0.00,] (373.65, 66.32) circle (  0.89);

\draw[color=drawColor,line cap=round,line join=round,fill opacity=0.00,] (401.85,103.66) circle (  0.89);

\draw[color=drawColor,line cap=round,line join=round,fill opacity=0.00,] (430.04, 80.09) circle (  0.89);

\draw[color=drawColor,line cap=round,line join=round,fill opacity=0.00,] (458.24, 74.85) circle (  0.89);
\end{scope}
\begin{scope}
\path[clip] (  0.00,  0.00) rectangle (505.89,650.43);
\definecolor[named]{drawColor}{rgb}{0.00,0.00,0.00}

\node[color=drawColor,anchor=base,inner sep=0pt, outer sep=0pt, scale=  1.00] at (260.87,200.18) {(c) RPSS = 0.40 MC = 0.37%
};
\end{scope}
\end{tikzpicture}
 
   \caption{Same as Figure \ref{fig:box-seas} but for retroactive forecasts overlaid with the CBRFC coordinated forecast where available. The blue solid lines are the 10th and 90th percentile and the dashed blue line is the 50th percentile of the coordinated forecast.}
   \label{fig:box-retro}
\end{figure*}
 
%% latex table generated in R 2.12.0 by xtable 1.5-6 package
% Fri Jan  7 18:01:58 2011
\begin{sidewaystable}[ht]
\begin{center}
\caption{RPSS after disaggregation and retroactive cross validation for each lead time.}\label{tab:retro}
\begin{tabular}{rrrrr|rrrr|rrrr|rrrr}
  \toprule
  &\multicolumn{4}{c}{April 1} & \multicolumn{4}{c}{Feb 1} & \multicolumn{4}{c}{Jan 1} & \multicolumn{4}{c}{Nov 1}\\
  \midrule
Node & April & May & June & July & April & May & June & July & April & May & June & July & April & May & June & July \\ 
  \midrule
  1 & -0.04 (0.19) & 0.30 (0.36) & 0.56 (0.81) & 0.45 (0.67) & -0.18 (0.04) & -0.20 (0.03) & 0.02 (0.60) & 0.05 (0.52) & -0.25 (-0.02) & -0.11 (0.12) & 0.15 (0.18) & 0.34 (0.40) \\ 
    2 & 0.23 (0.24) & 0.71 (0.71) & 0.81 (0.83) & 0.86 (0.65) & -0.17 (0.06) & 0.34 (0.42) & 0.61 (0.68) & 0.76 (0.55) & -0.27 (-0.02) & 0.30 (0.21) & 0.57 (0.41) & 0.64 (0.37) \\ 
    3 & -0.13 (0.11) & 0.40 (0.42) & 0.60 (0.70) & 0.60 (0.46) & -0.29 (-0.02) & -0.21 (0.10) & 0.41 (0.67) & 0.29 (0.45) & -0.04 (0.44) & 0.04 (0.19) & 0.40 (0.33) & 0.16 (0.28) \\ 
    4 & 0.11 (-0.01) & 0.38 (0.78) & 0.69 (0.74) & 0.83 (0.56) & -0.22 (-0.02) & -0.11 (0.42) & 0.34 (0.71) & 0.68 (0.52) & 0.14 (0.09) & -0.11 (-0.15) & 0.22 (0.39) & 0.55 (0.41) \\ 
    5 & -0.27 (0.45) & 0.55 (0.83) & 0.82 (0.71) & 0.71 (0.55) & -0.24 (0.21) & 0.52 (0.56) & 0.51 (0.65) & 0.56 (0.48) & -0.25 (-0.34) & 0.34 (0.29) & 0.53 (0.36) & 0.44 (0.43) \\ 
    6 & 0.20 (0.63) & 0.81 (0.91) & 0.74 (0.69) & 0.74 (0.50) & 0.16 (0.41) & 0.71 (0.64) & 0.52 (0.57) & 0.60 (0.37) & 0.13 (0.29) & 0.57 (0.35) & 0.57 (0.23) & 0.52 (0.32) \\ 
    7 & 0.32 (0.84) & 0.46 (0.84) & 0.56 (0.73) & 0.70 (0.63) & 0.18 (0.60) & 0.42 (0.58) & 0.40 (0.75) & 0.62 (0.70) & 0.09 (0.35) & 0.25 (0.38) & 0.27 (0.51) & 0.51 (0.62) \\ 
    8 & 0.27 (0.66) & 0.40 (0.54) & 0.11 (0.41) & 0.39 (0.40) & 0.27 (-0.16) & 0.15 (-0.23) & 0.12 (0.26) & 0.11 (0.37) & 0.21 (-0.19) & 0.21 (-0.05) & 0.04 (0.14) & -0.42 (-0.10) \\ 
    9 & 0.29 (0.53) & 0.49 (0.71) & 0.53 (0.71) & 0.57 (0.75) & -0.14 (0.06) & -0.02 (0.40) & 0.23 (0.54) & 0.22 (0.46) & -0.06 (0.34) & 0.07 (0.40) & 0.27 (0.35) & -0.04 (0.22) \\ 
    10 & 0.23 (0.37) & 0.18 (0.02) & 0.63 (0.54) & 0.60 (0.61) & 0.06 (-0.41) & 0.02 (0.06) & 0.25 (0.51) & 0.22 (0.41) & 0.24 (0.23) & -0.12 (0.25) & 0.01 (0.57) & 0.24 (0.34) \\ 
    11 & 0.62 (0.76) & 0.73 (0.72) & 0.55 (0.68) & 0.46 (0.30) & 0.07 (0.45) & 0.55 (0.70) & 0.23 (0.63) & 0.20 (0.36) & -0.47 (0.41) & 0.22 (0.67) & 0.20 (0.64) & 0.16 (0.50) \\ 
    12 & 0.05 (0.29) & 0.27 (0.73) & 0.76 (0.83) & 0.70 (0.70) & -0.53 (-0.14) & 0.16 (0.26) & 0.45 (0.65) & 0.40 (0.57) & -0.57 (-0.34) & -0.31 (0.23) & 0.17 (0.37) & 0.10 (0.41) \\ 
    13 & -0.05 (0.36) & 0.76 (0.81) & 0.50 (0.79) & 0.76 (0.61) & -0.48 (0.01) & 0.36 (0.54) & 0.38 (0.71) & 0.55 (0.64) & -0.68 (-0.35) & 0.44 (0.49) & 0.66 (0.50) & 0.68 (0.50) \\ 
    14 & 0.18 (0.60) & 0.43 (0.78) & 0.75 (0.79) & 0.66 (0.59) & -0.25 (0.16) & 0.13 (0.59) & 0.53 (0.79) & 0.06 (0.50) & -0.35 (0.40) & 0.23 (0.55) & 0.36 (0.85) & -0.22 (0.47) \\ 
    15 & 0.26 (0.25) & 0.59 (0.65) & 0.74 (0.84) & 0.69 (0.64) & -0.08 (0.17) & 0.09 (0.28) & 0.44 (0.67) & 0.26 (0.52) & -0.09 (0.12) & -0.37 (0.35) & 0.59 (0.50) & -0.06 (0.45) \\ 
    16 & 0.32 (0.41) & 0.20 (0.30) & 0.12 (0.49) & 0.75 (0.72) & 0.24 (-0.13) & 0.19 (-0.22) & -0.24 (-0.24) & 0.48 (0.59) & 0.19 (0.07) & 0.21 (-0.15) & -0.29 (0.51) & 0.31 (0.55) \\ 
    17 & -0.03 (0.65) & 0.34 (0.56) & 0.71 (0.88) & 0.78 (0.40) & -0.17 (0.32) & -0.04 (0.27) & 0.60 (0.72) & 0.58 (0.42) & 0.07 (-0.13) & -0.30 (0.10) & 0.67 (0.60) & 0.54 (0.43) \\ 
    18 & 0.23 (0.82) & 0.15 (0.67) & 0.47 (0.60) & 0.65 (0.56) & 0.15 (0.23) & 0.28 (0.69) & 0.46 (0.85) & 0.47 (0.66) & 0.01 (0.20) & 0.31 (0.45) & 0.31 (0.59) & 0.36 (0.68) \\ 
    19 & 0.15 (0.79) & 0.40 (0.84) & 0.56 (0.74) & 0.88 (0.55) & 0.13 (0.24) & 0.12 (0.37) & 0.50 (0.83) & 0.62 (0.63) & 0.07 (0.07) & -0.19 (0.15) & 0.33 (0.59) & 0.49 (0.67) \\ 
    20 & 0.55 (0.74) & 0.64 (0.84) & 0.18 (0.57) & 0.69 (0.70) & 0.32 (0.32) & 0.47 (0.69) & -0.13 (0.14) & 0.42 (0.68) & 0.35 (0.27) & -0.02 (0.44) & -0.12 (0.18) & 0.62 (0.48) \\
   \bottomrule
\end{tabular}
\end{center}
\end{sidewaystable}

 
\section{Conclusions}

We have shown that the framework of \cite{Bracken:2010cw} can be extended successfully to twenty sites in the UCRB. This method is parsimonious and requires very little data input compared to existing physically based forecast models.  We made slight modifications to the method including incorporating the disaggregation method of \cite{Nowak:2010ha}. The skills are positive at almost all the locations even after spatial and temporal disaggregation and at long lead times. Skills increase with decrease in lead time, while, June has the highest predictability because that is the peak flow month in the UCRB on average. We also see that larger tributaries tend to have higher skill as they are modulated better by large scale climate forcings. While individual sites do show negative skills, taken as a whole, we see a vast improvement over climatology. Even greater improvement in skill could likely be gained by combining this method with existing forecasts such as the CBRFC coordinated forecast, perhaps using Bayesian model combination techniques \citep{Rajagopalan2002, Raftery:2005un, Duan:2007ts}.  The basin wide skillful forecasts at long lead times are an important contribution for efficient water resources management and planning. We plan to test the utility of these forecasts in a basin-wide water management model. 

\newpage

\bibliographystyle{elsarticle-num-names}
\bibliography{references}



\end{document}  
