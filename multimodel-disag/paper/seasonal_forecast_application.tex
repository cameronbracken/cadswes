
\documentclass[11pt,twoside]{article}

%%------------page layout
\usepackage[margin=1in]{geometry} %changes margins
\usepackage[parfill]{parskip} % begin paragraphs with an empty line not indent
\usepackage{multicol} %content in many columns
\usepackage[toc,page]{appendix} %easier way to set up appendix
\usepackage{setspace}
\usepackage{lineno}
\usepackage{sectsty}

\sectionfont{\large}
\subsectionfont{\normalsize}
\subsubsectionfont{\bf}


%%------------font choices
\usepackage[scaled=.9]{couriers} %fixed with font scaled
\usepackage{mathpazo} %default font palatino and math font fourier
\usepackage[utf8]{inputenc}

%%------------graphics
\usepackage{graphicx,epstopdf} %functionality for including graphics, 
\usepackage{subfigure} %a b c figures with different captions
\usepackage{wallpaper} %background image

%%------------mathematics %extra math symbols
\usepackage{amsmath,amssymb,amsthm} 
\usepackage{siunitx} %good one

%%------------bibliography
\usepackage{natbib}   
%\setcitestyle{square,aysep={},yysep={;}}
\bibliographystyle{agufull04}%abbrvnat,plainnat,unsrtnat

%%------------tables
\usepackage{booktabs} %nice tables i.e. pleasing

%%------------misc
\usepackage[pdftex,bookmarks,colorlinks,breaklinks]{hyperref}  %
\hypersetup{linkcolor=black,citecolor=black,filecolor=black,urlcolor=black}


%%-----------page header declaration
\newcommand{\Ohead}{}           %header for Odd pages
\newcommand{\Ehead}{}  %header for Even pages
\usepackage{fancyhdr} 
\pagestyle{fancy}
\fancyhead{}
\fancyfoot{} 
\renewcommand{\headrulewidth}{0pt}
\renewcommand{\footrulewidth}{0pt}
\fancyhead[CE]{}%{\small \Ehead}
\fancyhead[CO]{}%{\small \Ohead}  
\fancyhead[L]{\thepage}   %page numbers

\begin{document}

\textbf{\Large Multisite seasonal ensemble forecast application to the Upper Colorado River Basin}

Cameron Bracken, Balaji Rajagopalan


 \doublespacing 
 \linenumbers
{\flushleft\small
%We demonstrate the feasbility of extending recent forecast and dissagreagation techniques to all natural flow nodes in the Upper Colorado River Basin (20 sites).

The generation of forecasts at a large number of sites is crucial for operations of the Upper Colorado River Basin.  We show that the framework of \cite{Bracken:2010p2682} can be extended to generate monthly peak season (April-July) forecasts at twenty sites in the Upper Colorado River Basin at long lead times.     We also incorporate the flexible disaggregation method of \cite{Nowak:2010p2738} into the framework to perform spatial and temporal disaggregation.  With few exceptions, skills are positive for all sites, months and lead times.  The November 1 forecasts show an average 16\% increase in skill over climatology for all the sites considered. 
}

\section{Introduction}
The nonparametric multisite ensemble forecast framework developed by \cite{Bracken:2010p2682} was able to show skill at long lead times in forcasts generated for four key sites in the Upper Colorado River Basin (Colorado River near Cisco, Utah, Green River at Green River, Utah, San Juan River near Bluff, Utah, and Colorado River at Lees Ferry, Arizona).  While promising, these results leave open the question of this skill translating to a larger number of sites on the same river network. The generation of forecasts at a large number of sites is crucial for operations of the Upper Colorado River Basin.  For example, the Bureau of Reclmation's primary water supply forecast model, the ``24-Month Study'' requires input at 12 locations in the Upper Basin.  



%%%%%%%%%%%%%%%%%%%%%%%%%%%%%%%%%%%%%%%%%%%%%%%%%%%%%%%%%%%%
%%%%%%%%%%%%%%%%%%%%%%%%%%%%%%%%%%%%%%%%%%%%%%%%%%%%%%%%%%%%
\section{Study Area}
We 

%%%%%%%%%%%%%%%%%%%%%%%%%%%%%%%%%%%%%%%%%%%%%%%%%%%%%%%%%%%%
%%%%%%%%%%%%%%%%%%%%%%%%%%%%%%%%%%%%%%%%%%%%%%%%%%%%%%%%%%%%
\section{Data}

%%%%%%%%%%%%%%%%%%%%%%%%%%%%%%%%%%%%%%%%%%%%%%%%%%%%%%%%%%%%
%%%%%%%%%%%%%%%%%%%%%%%%%%%%%%%%%%%%%%%%%%%%%%%%%%%%%%%%%%%%
\section{Methodology}

We use the the same framework as \cite{Bracken:2010p2682} with some notable changes


\bibliography{../../references}


\end{document}  
